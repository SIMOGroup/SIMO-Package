\documentclass[12pt, a4paper, twoside]{memoir}
%\let\STARTCODE\relax
%\let\STOPCODE\relax
%\STARTCODE
\usepackage{color,calc,graphicx, soul}
\definecolor{nicered}{rgb}{.647,.129,.149}
\makeatletter
\newlength\dlf@normtxtw
\setlength\dlf@normtxtw{\textwidth}
\def\myhelvetfont{\def\sfdefault{mdput}}
\newsavebox{\feline@chapter}
\newcommand\feline@chapter@marker[1][4cm]{%
  \sbox\feline@chapter{%
    \resizebox{!}{#1}{\fboxsep=1pt%
      \colorbox{nicered}{\color{white}\bfseries\rmfamily\thechapter}%
    }}%
  \rotatebox{90}{%
    \resizebox{%
      \heightof{\usebox{\feline@chapter}}+\depthof{\usebox{\feline@chapter}}}%
    {!}{\scshape\so\@chapapp}}\quad%
  \raisebox{\depthof{\usebox{\feline@chapter}}}{\usebox{\feline@chapter}}%
}
\newcommand\feline@chm[1][4cm]{%
  \sbox\feline@chapter{\feline@chapter@marker[#1]}%
  \makebox[0pt][l]{% aka \rlap
    \makebox[1cm][r]{\usebox\feline@chapter}%
  }}
\makechapterstyle{daleif1}
    {
        %\renewcommand\chapnamefont{\normalfont\Large\scshape\raggedleft\so}
        \renewcommand\chaptitlefont{\normalfont\huge\bfseries\scshape\color{nicered}}
        \renewcommand\chapternamenum{}
        \renewcommand\printchaptername{}
        \renewcommand\printchapternum{\null\hfill\feline@chm[2.5cm]\par}
        \renewcommand\afterchapternum{\par\vskip\midchapskip}
        \renewcommand\printchaptertitle[1]{\chaptitlefont\raggedleft ##1\par}
    }
\makeatother
\chapterstyle{daleif1}
%\STOPCODE
\setlength\afterchapskip {\onelineskip }
\setlength\beforechapskip {\onelineskip }
\usepackage{lipsum}

\setlength{\parskip}{0.5\baselineskip}
\usepackage{savesym}
\usepackage[headsepline,footsepline]{scrlayer-scrpage}
%\usepackage{structuralanalysis}
\savesymbol{notation}
\usepackage[top=2cm,bottom=2cm,left=2cm,right=2cm,headsep=10pt]{geometry} % Page margins
\usepackage{etex}
% choose options for [] as required from the list
% in the Reference Guide
\usepackage{amsmath}
\setcounter{MaxMatrixCols}{20}
\usepackage{amsthm} % For including math equations, theorems, symbols, etc
\usepackage{amsfonts} % need for \mathbb
\usepackage{commath}
\usepackage[nomessages]{fp} %An extensive collection of arithmetic operations for fixed point real numbers of high precision.
\usepackage{mathrsfs}
\usepackage{mathtools}

%\usepackage[nomath, variablett]{lmodern}
\renewcommand{\rmdefault}{ptm}		% set Times as the default text font
\usepackage[subscriptcorrection]{mtpro2}
\raggedbottom %makes all pages the height of the text on that page. No extra vertical space is added.

\savesymbol{openbox}
\savesymbol{newfloat}
\usepackage{type1cm}
%\usepackage[Symbol]{upgreek}
\usepackage{makeidx}         % allows index generation
\usepackage{graphicx}        % standard LaTeX graphics tool
                             % when including figure files
\usepackage{multicol}        % used for the two-column index
\usepackage[bottom]{footmisc}% places footnotes at page bottom
\usepackage[T1]{fontenc}
\usepackage{textcomp}

\usepackage{float}%placing figures and table precisely
\usepackage[center, labelfont={it}, textfont={it}]{caption}%center: provides captions where each line is centered.
\usepackage{subcaption}

\usepackage{comment}
\pdfoptionpdfminorversion 7 %Use pdf version 1.6
%%settings for Tikz figures
%--------------------------------------------------------
\usepackage{tikz}
\usepackage{tikz-3dplot}
\usepackage{pgfplots} % to use TikZ figures
%\usepackage{tikz-qtree}
\pgfplotsset{compat=newest}
%\pgfplotsset{plot coordinates/math parser=false}
%\usepackage{tikz,pgffor}
%\usepackage{tikz-3dplot}
\usetikzlibrary{calc}
\usetikzlibrary{spy}
%\newlength\figureheight
%\newlength\figurewidth
\usetikzlibrary{external}
\tikzexternalize[prefix=pdffigures/] % activate externalization!
\usepgfplotslibrary{patchplots}
\usetikzlibrary{quotes,angles}
\usetikzlibrary{decorations,decorations.markings,decorations.text,positioning, shadows, arrows, backgrounds, shapes, patterns, 3d, fit, bending}

\usepackage{siunitx}
\usepgflibrary{arrows.meta} % LATEX and plain TEX and pure pgf
\def\centerarc[#1](#2)(#3:#4:#5){ \draw[#1] ($(#2)+({#5*cos(#3)},{#5*sin(#3)})$) arc (#3:#4:#5); } %[draw options] (center) (initial angle:final angle:radius)

% Define the layers to draw the diagram
\pgfdeclarelayer{background}
\pgfdeclarelayer{foreground}
\pgfsetlayers{background,main,foreground}

% Define the layers to draw the diagram
\pgfdeclarelayer{background}
\pgfdeclarelayer{foreground}
\pgfsetlayers{background,main,foreground}

% Define block styles
\tikzstyle{materia}=[draw, fill=blue!20, text width=8.0em, text centered,
  minimum height=1.5em,drop shadow]
\tikzstyle{etape} = [materia, text width=20em, minimum width=10em,
  minimum height=3em, rounded corners, drop shadow]
\tikzstyle{texto} = [above, text width=6em, text centered]
\tikzstyle{linepart} = [draw, thick, color=black!50, dashed]
\tikzstyle{line} = [draw, thick, color=black!50, -latex']
\tikzstyle{ur}=[draw, text centered, minimum height=0.01em]

% Define distances for bordering
\newcommand{\blockdist}{1.3}
\newcommand{\edgedist}{1.5}

\newcommand{\etape}[2]{node (p#1) [etape]
  {#2}}

% Draw background
\newcommand{\background}[5]{%
  \begin{pgfonlayer}{background}
    % Left-top corner of the background rectangle
    \path (#1.west |- #2.north)+(-0.5,0.25) node (a1) {};
    % Right-bottom corner of the background rectanle
    \path (#3.east |- #4.south)+(+0.5,-0.25) node (a2) {};
    % Draw the background
    \path[fill=yellow!20,rounded corners, draw=black!50, dashed]
      (a1) rectangle (a2);
      \path (#3.east |- #2.north)+(0,0.25)--(#1.west |- #2.north) node[midway] (#5-n) {};
      \path (#3.east |- #2.south)+(0,-0.35)--(#1.west |- #2.south) node[midway] (#5-s) {};
      \path (#3.east |- #2.north)+(0.7,0)--(#3.east |- #4.south) node[midway] (#5-w) {};
  \end{pgfonlayer}}

\newcommand{\transreceptor}[3]{%
  \path [linepart] (#1.east) -- node [above]
    {#2} (#3);}
%--------------------------------------------------------

\usepackage{scalefnt} %used to scale fonts
\usepackage{IEEEtrantools}
\usepackage{array}%to center cell contents of a LaTeX table whose columns have fixed widths

%% setting for code listing
%----------------------------------------------------------------------------------
\usepackage[framed]{matlab-prettifier}
%\usepackage{fontspec}
\usepackage{lstautogobble}
\usepackage{inconsolata}
%\setmonofont{Consolas}
%\newfontfamily\Consolas{Consolas}
%\lstset{basicstyle=\Consolas}
%\usepackage{listings}
\lstset{
  style = Matlab-bw,
  basicstyle=\ttfamily\footnotesize,
  escapechar = `,
  mlcommentstyle = \color[RGB]{034,139,034},
  mlstringstyle = \color[RGB]{160,032,240},
  mlsyscomstyle = \color[RGB]{178,140,0},
  mlsectiontitlestyle = \color[gray]{0}\bfseries\itshape,
  mlsharedvarstyle = \color[gray]{0},
  %mlshowsectionrules = true,
  showspaces = false,
  columns=fullflexible,
  autogobble=true,
}

\newcounter{mainlisting}[chapter]
\newcounter{sublisting}[chapter]

%------------------------------------------------------------------------------------
% end of setting for code listing
\usepackage{hyperref}
\hypersetup{
    %bookmarks=true,         % show bookmarks bar?
    unicode=true,          % non-Latin characters in Acrobat’s bookmarks
    pdftoolbar=true,        % show Acrobat’s toolbar?
    pdfmenubar=true,        % show Acrobat’s menu?
    pdffitwindow=false,     % window fit to page when opened
    pdfstartview={FitH},    % fits the width of the page to the window
    pdftitle={SIMO Package User Guide},    % title
    pdfauthor={Various Authors},     % author
    pdfsubject={Subject},   % subject of the document
    pdfcreator={Creator},   % creator of the document
    pdfproducer={Producer}, % producer of the document
    pdfkeywords={keyword1} {key2} {key3}, % list of keywords
    pdfnewwindow=true,      % links in new window
    colorlinks=true,       % false: boxed links; true: colored links
    %linkcolor=black,          % color of internal links (change box color with linkbordercolor)
%    citecolor=black,        % color of links to bibliography
%    filecolor=black,      % color of file links
%    urlcolor=black,           % color of external links
    pageanchor=false
}
\usepackage{bookmark}
\usepackage{multirow}
\usepackage{array}
\newcolumntype{L}[1]{>{\raggedright\let\newline\\\arraybackslash\hspace{0pt}}m{#1}}
\newcolumntype{C}[1]{>{\centering\let\newline\\\arraybackslash\hspace{0pt}}m{#1}}
\newcolumntype{R}[1]{>{\raggedleft\let\newline\\\arraybackslash\hspace{0pt}}m{#1}}
\usepackage{longtable}
\usepackage{threeparttable}
%\usepackage{psfrag}
%\usepackage{cases}
%Separate bibliographies per chapter
%--------------------------------------------------
\usepackage[square,numbers,sectionbib]{natbib}
\usepackage{chapterbib}
\usepackage[toc, nonumberlist]{glossaries}
\makeglossaries

\begin{document}
\clearpage
%% temporary titles
% command to provide stretchy vertical space in proportion
\newcommand\nbvspace[1][3]{\vspace*{\stretch{#1}}}
% allow some slack to avoid under/overfull boxes
\newcommand\nbstretchyspace{\spaceskip0.5em plus 0.25em minus 0.25em}
% To improve spacing on titlepages
\newcommand{\nbtitlestretch}{\spaceskip0.6em}
\thispagestyle{empty}
\begin{center}
\bfseries
\large Authors\\
\nbvspace[5]
{\Huge
SIMO Package\\
User Manual}

\nbvspace[15]

\end{center}
\frontmatter%%%%%%%%%%%%%%%%%%%%%%%%%%%%%%%%%%%%%%%%%%%%%%%%%%%%%%
\clearpage
\thispagestyle{plain}
\cleardoublepage

%\pdfbookmark{\contentsname}{toc}
\tableofcontents
\cleardoublepage

\mainmatter%%%%%%%%%%%%%%%%%%%%%%%%%%%%%%%%%%%%%%%%%%%%%%%%%%%%%%%
%Term definitions
\newglossaryentry{meshbdry}{name=MeshBdry, description={Mesh of the Boundary}}
\newglossaryentry{nurbsbdry}{name=NURBSBdry, description={NURBS of the Boundary}}
\newglossaryentry{idx}{name=Idx, description={Index}}
\newglossaryentry{idcs}{name=Idcs, description={Indices}}
\newglossaryentry{vals}{name=Vals, description={Values}}
\newglossaryentry{bdryidcs}{name=BdryIdcs, description={Indices of boundary control points}}
\newglossaryentry{bdryvals}{name=BdryVals, description={Values of boundary control points}}
\newglossaryentry{drchlt}{name=Drchlt, description={Dirichlet}}
\newglossaryentry{neuman}{name=Neuman, description={Neumann}}
\newglossaryentry{kntvect}{name=KntVect, description={Knot Vector}}
\newglossaryentry{ctrlpts}{name=CtrlPts, description={Control Points}}
\newglossaryentry{curv}{name=Curv, description={Curve}}
\newglossaryentry{surf}{name=Surf, description={Surface}}
\newglossaryentry{volu}{name=Volu, description={Volume}}
\newglossaryentry{nsd}{name=NSD, description={Number of dimensional space}}
\newglossaryentry{uqkntvect}{name=uqKntVect, description={Unique knot values}}
\newglossaryentry{ctrlpts4d}{name=CtrlPts4D, description={Control point coordinates in homogeneous space}}
\newglossaryentry{ctrlpts3d}{name=CtrlPts3D, description={Control point coordinates in Cartesian space}}
\newglossaryentry{nctrlpts}{name=NCtrlPts, description={Number of control points}}
\newglossaryentry{nnp}{name=NNP, description={Number of total control points for a patch}}
\newglossaryentry{nel}{name=NEl, description={Number of elements}}
\newglossaryentry{nen}{name=NEN, description={Number of local basis functions}}
\newglossaryentry{dof}{name=Dof, description={Number of degree of freedom per control point}}
\newglossaryentry{el}{name=El, description={Element connectivity matrix}}
\newglossaryentry{neldir}{name=NElDir, description={Number of element in each direction}}
\newglossaryentry{ndof}{name=NDof, description={Number of degree of freedoms for a patch}}
\newglossaryentry{compdofs}{name=CompDofs, description={Degree of freedoms in components}}

%\clearpage
%\phantomsection
%Print the glossary
\glsaddall\printglossaries
%\addtocontents{toc}{test}

\let\clearforchapter\par % cheating, but saves some space
\chapter{Fundamentals of the IsoGeometric Analysis} 

\section{Univariate B-splines}

\subsection{Knot vector and B-spline functions, refinement, spline derivatives}

\subsection{B-spline curves}

\section{Multivariate tensor product B-splines}
A B-spline patch (curve, surface, volume) is essentially defined by these corresponding formulas
\begin{equation}
    \mathbf{C}(\xi) = \sum_{i = 1}^{n} N_{i,p}(\xi) \mathbf{P}_i.
\end{equation}

\begin{equation}
    \mathbf{S}(\xi,\eta)=\sum_{i=1}^n \sum_{j=1}^m N_{i,p}(\xi) M_{j,q}(\eta) \mathbf{P}_{i,j},
\end{equation}

\begin{equation}
    \mathbf{S}(\xi,\eta, \zeta)=\sum_{i=1}^n \sum_{j=1}^m \sum_{k=1}^l N_{i,p}(\xi) M_{j,q}(\eta) L_{k,l}(\zeta) \mathbf{P}_{i,j,k},
\end{equation}
where $N_{i,p}(\xi), M_{j,q}(\eta), L_{k,l}(\zeta)$ are univariate basis functions, $\mathbf{P}$ is the 4-vector of control point coordinates.
\subsection{Knot vectors, B-spline functions}

\subsection{NURBS geometries, single patch and multi--patch domains}

\section{IsoGeometric Analysis}

\subsection{Basic Idea and Fundamentals}

\subsection{Isogeometric Analysis in Detail}

\subsubsection{B-splines and NURBS as Basis Functions}

\subsection{Mesh Structure in Isogeometric Analysis}

\subsection{Geometry Mapping}

\subsection{Refinement strategies}
\let\clearforchapter\par % cheating, but saves some space
\chapter{Mathematical Models and Variational Formulation} 
Nam dui ligula, fringilla a, euismod sodales, sollicitudin vel, wisi. Morbi auctor lorem non justo. Nam lacus libero, pretium at, lobortis vitae, ultricies et, tellus. Donec aliquet, tortor sed accumsan bibendum, erat ligula aliquet magna, vitae ornare odio metus a mi.

\par\fancybreak{$***$}\par

\section{Mathematical Models}
\subsection{Heat conduction}

\subsection{Continuum Mechanics}

\subsection{Truss}

\subsection{Beam}

\subsection{Plate}

\subsection{Shell}

\subsection{Gradient Elasticity}

\subsection{Limit Analysis}

\section{Variational Formulation}


\graphicspath{{./figures/chapter3/}}
\let\clearforchapter\par % cheating, but saves some space
\chapter{Implementation}
This chapter is mainly devoted to the implementation of SIMO package within MATLAB.

\par\fancybreak{$***$}\par

%\section{MATLAB Data Types}
%\subsection{Multidimensional arrays}
%All variables of all data types in MATLAB are multidimensional arrays. A vector is a one-dimensional array and a matrix is a %two-dimensional array. An array having more than two dimensions is called a multidimensional array in MATLAB. Multidimensional arrays in %MATLAB are an extension of the normal two-dimensional matrix. In this section,
%\begin{figure}[H]
%    \centering
%    \begin{subfigure}[b]{0.3\textwidth}
%        \includegraphics[width=\textwidth]{HorizontalSlices.pdf}
%        \caption{Horizontal}
%        \label{fig:Frontal}
%    \end{subfigure}
%    \quad
%    \begin{subfigure}[b]{0.3\textwidth}
%        \includegraphics[width=\textwidth]{LateralSlices.pdf}
%        \caption{Lateral}
%        \label{fig:Frontal}
%    \end{subfigure}
%    \quad
%    \begin{subfigure}[b]{0.3\textwidth}
%        \includegraphics[width=\textwidth]{FrontalSlices.pdf}
%        \caption{Frontal}
%        \label{fig:Frontal}
%    \end{subfigure}
%    \caption{Slices of a three--dimensional array}
%    \label{fig:3darray}
%\end{figure}
\section{Data Representation}
\subsection{Knot vector(s) and Control Point's Coordinates}
The main information in isogeometric analysis is given by knot vector(s) and control points. In SIMO Package, these data structures are stored as following
\begin{itemize}
    \item Knot vectors are stored in a cell array. So,
        \begin{itemize}
            \item For curve: a B-spline (NURBS) curve is parametrized by one knot vector
            \begin{lstlisting}
                KntVect = {[ 0, 0, 0, 0.25, 0.25, 0.5, 0.5, 0.75, 0.75, 1, 1, 1]}
            \end{lstlisting}
            \item For surface: a B-spline (NURBS) surface is parametrized by two knot vectors
            \begin{lstlisting}
                KntVect = {[ 0, 0, 0, 0.25, 0.25, 0.5, 0.5, 0.75, 0.75, 1, 1, 1],...
                 [0, 0, 1, 1]}
            \end{lstlisting}
            \item For solid: a B-spline (NURBS) surface is parametrized by three knot vectors
            \begin{lstlisting}
                KntVect = {[ 0, 0, 0, 0.25, 0.25, 0.5, 0.5, 0.75, 0.75, 1, 1, 1],...
                 [0, 0, 1, 1],...
                 [0, 0, 0, 0.5, 0.5, 1, 1, 1]}
            \end{lstlisting}
        \end{itemize}
    \item Control points are stored in multi--dimensional array
        \begin{itemize}
            \item For curve, coordinates (including weights) of control points are populated into a two--dimensional array, where each column contains coordinates of a control point (which together construct a control polyline). In SIMO Package, the size of the first dimension of this array should be always specified equal to 4 ($x \cdot w, y \cdot w, z \cdot w, w$). This means the control point's data is added in form of homogeneous coordinates. The below code snippet is taken as an illustration for clarifying this point. Fig.~\ref{fig:Ch3NURBSCircTriBasedMethod} is a visualization of the way we store control point's coordinates.
                \begin{lstlisting}
                    radius = 1;

                    % Control point's coordinates in homogeneous space

                    CtrlPts = zeros(4, 7);
                    CtrlPts(1 : 2, 1, 1) = [cos(pi / 6); sin(pi / 6)];
                    CtrlPts(1 : 2, 2, 1) = [0; 1 / cos(pi / 3)];
                    CtrlPts(1 : 2, 3, 1) = [-cos(pi / 6); sin(pi / 6)];
                    CtrlPts(1 : 2, 4, 1) = [-tan(pi / 3); -1];
                    CtrlPts(1 : 2, 5, 1) = [0; -1];
                    CtrlPts(1 : 2, 6, 1) = [tan(pi / 3); -1];
                    CtrlPts(1 : 2, 7, 1) = [cos(pi / 6); sin(pi / 6)];

                    CtrlPts(1 : 3, :, :) = CtrlPts(1 : 3, :, :) * radius;
                    % assign the weights
                    CtrlPts(4, :, :) = 1;
                    W = 1 / 2;
                    CtrlPts(:, 2 : 2 : end, :) = CtrlPts(:, 2 : 2 : end, :) * W;
                \end{lstlisting}

                \begin{figure}[H]
                    \centering
                    \tikzsetnextfilename{Ch3NURBSCircTriBasedMethod}
                    \normalsize
                    % This file was created by matlab2tikz.
%
\begin{tikzpicture}
\begin{axis}[%
width=15cm,
scale only axis,
xmin=-4,
xmax=4,
ymin=-1.5,
ymax=4,
hide axis,
unit vector ratio=1 1 1,%
]
\addplot [color=black,solid,line width=1.0pt,forget plot]
  table[row sep=crcr]{%
0.866025403784439	0.5\\
0.838168389850029	0.545411542100284\\
0.806997382104876	0.590555014605648\\
0.772401035807743	0.635135135135135\\
0.73429697648732	0.678828365878725\\
0.692637809770694	0.721285563751317\\
0.647417049431085	0.762135922330097\\
0.598674672737423	0.800992282249173\\
0.546501970217177	0.837457817772779\\
0.491045332042723	0.871134020618557\\
0.432508612600145	0.90162980209546\\
0.371153744479045	0.928571428571428\\
0.307299336826736	0.951612903225806\\
0.241317090801213	0.970446320868516\\
0.173625992253259	0.984811664641555\\
0.104684389468449	0.994505494505494\\
0.0349802182679148	0.999388004895961\\
-0.0349802182679149	0.99938800489596\\
-0.104684389468449	0.994505494505494\\
-0.173625992253259	0.984811664641555\\
-0.241317090801213	0.970446320868516\\
-0.307299336826736	0.951612903225806\\
-0.371153744479045	0.928571428571428\\
-0.432508612600145	0.90162980209546\\
-0.491045332042723	0.871134020618556\\
-0.546501970217177	0.837457817772778\\
-0.598674672737424	0.800992282249173\\
-0.647417049431085	0.762135922330097\\
-0.692637809770694	0.721285563751317\\
-0.73429697648732	0.678828365878725\\
-0.772401035807743	0.635135135135135\\
-0.806997382104876	0.590555014605647\\
-0.838168389850029	0.545411542100284\\
-0.866025403784439	0.5\\
-0.891424445901106	0.453169347209082\\
-0.914934336033219	0.403602726387536\\
-0.936243679766961	0.351351351351351\\
-0.955031097904114	0.29650565262076\\
-0.970970526476968	0.239199157007376\\
-0.98373759459009	0.179611650485437\\
-0.993017001031771	0.117971334068357\\
-0.998510729897694	0.0545556805399323\\
-0.999946857977908	-0.0103092783505157\\
-0.997088619723876	-0.0762514551804427\\
-0.989743318610787	-0.142857142857143\\
-0.977770617175979	-0.209677419354839\\
-0.961089712281886	-0.276236429433052\\
-0.939684915649457	-0.342041312272175\\
-0.913609217179188	-0.406593406593407\\
-0.882985509611306	-0.469400244798042\\
-0.848005291343391	-0.52998776009792\\
-0.808924827710739	-0.587912087912088\\
-0.766058923396198	-0.642770352369381\\
-0.719772621480673	-0.694209891435465\\
-0.670471280349243	-0.741935483870968\\
-0.618589574131741	-0.785714285714286\\
-0.564580007123732	-0.825378346915018\\
-0.508901525935185	-0.860824742268041\\
-0.452008759680517	-0.892013498312711\\
-0.394342328294347	-0.91896361631753\\
-0.336320545159005	-0.941747572815534\\
-0.278332716706274	-0.960484720758693\\
-0.220734121416794	-0.975334018499486\\
-0.163842643959218	-0.986486486486487\\
-0.107936953928342	-0.994157740993184\\
-0.0532560560510769	-0.998580889309366\\
0	-1\\
0.0532560560510774	-0.998580889309366\\
0.107936953928343	-0.994157740993184\\
0.163842643959218	-0.986486486486486\\
0.220734121416794	-0.975334018499486\\
0.278332716706274	-0.960484720758693\\
0.336320545159006	-0.941747572815534\\
0.394342328294347	-0.91896361631753\\
0.452008759680517	-0.892013498312711\\
0.508901525935185	-0.860824742268041\\
0.564580007123732	-0.825378346915018\\
0.618589574131742	-0.785714285714286\\
0.670471280349243	-0.741935483870968\\
0.719772621480673	-0.694209891435464\\
0.766058923396198	-0.64277035236938\\
0.80892482771074	-0.587912087912087\\
0.848005291343392	-0.529987760097919\\
0.882985509611306	-0.469400244798042\\
0.913609217179188	-0.406593406593406\\
0.939684915649458	-0.342041312272175\\
0.961089712281886	-0.276236429433051\\
0.977770617175979	-0.209677419354839\\
0.989743318610787	-0.142857142857143\\
0.997088619723876	-0.0762514551804422\\
0.999946857977908	-0.0103092783505156\\
0.998510729897693	0.0545556805399327\\
0.993017001031771	0.117971334068357\\
0.98373759459009	0.179611650485437\\
0.970970526476968	0.239199157007376\\
0.955031097904114	0.29650565262076\\
0.936243679766961	0.351351351351351\\
0.914934336033219	0.403602726387536\\
0.891424445901106	0.453169347209082\\
0.866025403784439	0.5\\
};
%\addplot [color=red,mark size=3.3pt,only marks,mark=*,mark options={solid},forget plot]
%  table[row sep=crcr]{%
%0.866025403784439	0.5\\
%0	2\\
%-0.866025403784439	0.5\\
%-1.73205080756888	-1\\
%0	-1\\
%1.73205080756888	-1\\
%0.866025403784439	0.5\\
%};
\addplot [color=black,dashed,forget plot]
  table[row sep=crcr]{%
0.866025403784439	0.5\\
0	2\\
-0.866025403784439	0.5\\
-1.73205080756888	-1\\
0	-1\\
1.73205080756888	-1\\
0.866025403784439	0.5\\
};
%\draw [black] (axis cs:0.866025403784439, 0.5) circle (2.5pt) node [right, color=black] {$\mathbf{P}_{1,7}(0.8660, 0.5)$};
%\draw [black] (axis cs:0, 2) circle (2.5pt) node [above, color=black] {$\mathbf{P}_{2}(0, 2)$};
%\draw [black] (axis cs:-0.866025403784439, 0.5) circle (2.5pt) node [left, color=black] {$\mathbf{P}_{3}(-0.8660, 0.5)$};
%\draw [black] (axis cs:-1.73205080756888, -1) circle (2.5pt) node [below, color=black] {$\mathbf{P}_{4}(-1.7321, -1)$};
%\draw [black] (axis cs:0, -1) circle (2.5pt) node [below, color=black] {$\mathbf{P}_{5}(0, -1)$};
%\draw [black] (axis cs:1.73205080756888, -1) circle (2.5pt) node [below, color=black] {$\mathbf{P}_{6}(1.7321, -1)$};

\shade [ball color = red] (axis cs:0.866025403784439, 0.5) circle (3.3pt) node [right] {$\mathbf{P}_{1,7}(0.8660, 0.5)$};
\shade [ball color = red] (axis cs:0, 2) circle (3.3pt) node [above] {$\mathbf{P}_{2}(0, 2)$};
\shade [ball color = red] (axis cs:-0.866025403784439, 0.5) circle (3.3pt) node [left] {$\mathbf{P}_{3}(-0.8660, 0.5)$};
\shade [ball color = red] (axis cs:-1.73205080756888, -1) circle (3.3pt) node [below] {$\mathbf{P}_{4}(-1.7321, -1)$};
\shade [ball color = red] (axis cs:0, -1) circle (3.3pt) node [below] {$\mathbf{P}_{5}(0, -1)$};
\shade [ball color = red] (axis cs:1.73205080756888, -1) circle (3.3pt) node [below] {$\mathbf{P}_{6}(1.7321, -1)$};

\draw[red,line width=1.0pt, -stealth, decoration={
    text along path, text={$\xi$} {--} direction, raise=1ex, text align={right}, text color={red}
    }, postaction={decorate}, dashed] (0.866025403784439, 0.5) arc [start angle=30, end angle=30+180, radius=1];

%\addplot [color=red, line width=1.0pt, mark=none,
%decoration={
%    text along path, text={$\xi$} {--} direction, raise=1ex, text align={right}, text color={red}
%    }, postaction={decorate}, dashed, ->, >=stealth]
%  table[row sep=crcr]{%
%0.866025403784439	0.5\\
%0.838168389850029	0.545411542100284\\
%0.806997382104876	0.590555014605648\\
%0.772401035807743	0.635135135135135\\
%0.73429697648732	0.678828365878725\\
%0.692637809770694	0.721285563751317\\
%0.647417049431085	0.762135922330097\\
%0.598674672737423	0.800992282249173\\
%0.546501970217177	0.837457817772779\\
%0.491045332042723	0.871134020618557\\
%0.432508612600145	0.90162980209546\\
%0.371153744479045	0.928571428571428\\
%0.307299336826736	0.951612903225806\\
%0.241317090801213	0.970446320868516\\
%0.173625992253259	0.984811664641555\\
%0.104684389468449	0.994505494505494\\
%0.0349802182679148	0.999388004895961\\
%-0.0349802182679149	0.99938800489596\\
%-0.104684389468449	0.994505494505494\\
%-0.173625992253259	0.984811664641555\\
%-0.241317090801213	0.970446320868516\\
%-0.307299336826736	0.951612903225806\\
%-0.371153744479045	0.928571428571428\\
%-0.432508612600145	0.90162980209546\\
%-0.491045332042723	0.871134020618556\\
%-0.546501970217177	0.837457817772778\\
%-0.598674672737424	0.800992282249173\\
%-0.647417049431085	0.762135922330097\\
%-0.692637809770694	0.721285563751317\\
%-0.73429697648732	0.678828365878725\\
%-0.772401035807743	0.635135135135135\\
%-0.806997382104876	0.590555014605647\\
%-0.838168389850029	0.545411542100284\\
%-0.866025403784439	0.5\\
%-0.891424445901106	0.453169347209082\\
%-0.914934336033219	0.403602726387536\\
%-0.936243679766961	0.351351351351351\\
%-0.955031097904114	0.29650565262076\\
%-0.970970526476968	0.239199157007376\\
%-0.98373759459009	0.179611650485437\\
%-0.993017001031771	0.117971334068357\\
%-0.998510729897694	0.0545556805399323\\
%-0.999946857977908	-0.0103092783505157\\
%-0.997088619723876	-0.0762514551804427\\
%-0.989743318610787	-0.142857142857143\\
%-0.977770617175979	-0.209677419354839\\
%-0.961089712281886	-0.276236429433052\\
%-0.939684915649457	-0.342041312272175\\
%-0.913609217179188	-0.406593406593407\\
%-0.882985509611306	-0.469400244798042\\
%-0.848005291343391	-0.52998776009792\\
%-0.808924827710739	-0.587912087912088\\
%%-0.766058923396198	-0.642770352369381\\
%%-0.719772621480673	-0.694209891435465\\
%%-0.670471280349243	-0.741935483870968\\
%%-0.618589574131741	-0.785714285714286\\
%%-0.564580007123732	-0.825378346915018\\
%%-0.508901525935185	-0.860824742268041\\
%%-0.452008759680517	-0.892013498312711\\
%%-0.394342328294347	-0.91896361631753\\
%%-0.336320545159005	-0.941747572815534\\
%%-0.278332716706274	-0.960484720758693\\
%%-0.220734121416794	-0.975334018499486\\
%%-0.163842643959218	-0.986486486486487\\
%%-0.107936953928342	-0.994157740993184\\
%%-0.0532560560510769	-0.998580889309366\\
%%0	-1\\
%%0.0532560560510774	-0.998580889309366\\
%%0.107936953928343	-0.994157740993184\\
%%0.163842643959218	-0.986486486486486\\
%%0.220734121416794	-0.975334018499486\\
%%0.278332716706274	-0.960484720758693\\
%%0.336320545159006	-0.941747572815534\\
%%0.394342328294347	-0.91896361631753\\
%%0.452008759680517	-0.892013498312711\\
%%0.508901525935185	-0.860824742268041\\
%%0.564580007123732	-0.825378346915018\\
%%0.618589574131742	-0.785714285714286\\
%%0.670471280349243	-0.741935483870968\\
%%0.719772621480673	-0.694209891435464\\
%%0.766058923396198	-0.64277035236938\\
%%0.80892482771074	-0.587912087912087\\
%%0.848005291343392	-0.529987760097919\\
%%0.882985509611306	-0.469400244798042\\
%%0.913609217179188	-0.406593406593406\\
%%0.939684915649458	-0.342041312272175\\
%%0.961089712281886	-0.276236429433051\\
%%0.977770617175979	-0.209677419354839\\
%%0.989743318610787	-0.142857142857143\\
%%0.997088619723876	-0.0762514551804422\\
%%0.999946857977908	-0.0103092783505156\\
%%0.998510729897693	0.0545556805399327\\
%%0.993017001031771	0.117971334068357\\
%%0.98373759459009	0.179611650485437\\
%%0.970970526476968	0.239199157007376\\
%%0.955031097904114	0.29650565262076\\
%%0.936243679766961	0.351351351351351\\
%%0.914934336033219	0.403602726387536\\
%%0.891424445901106	0.453169347209082\\
%%0.866025403784439	0.5\\
%};
\end{axis}

%%\draw[-stealth] (O) -- (0, 2);
%\draw[red,line width=1.5pt,-stealth] (O) arc [start angle=30, end angle=30+90, radius=2];
\begin{scope}[shift = {(0, -4cm)}, scale=1, thin]
\draw[fill=gray!10, fill opacity=0.70] (0, 0) rectangle (14, 2.5);
%\foreach \x in {2, 4, ..., 12}
%{
%    \draw (\x, 0) -- (\x, 4);
%}
%\foreach \y in {1, 2, ..., 3}
%{
%    \draw (0, \y) -- (14, \y);
%}
{\fontencoding{T1}\selectfont \ttfamily
    \node [rectangle, anchor=west, fill=red!50, fill opacity=0.70, align=center, text width=1cm, left] at (0, 2.0) {x w};
    \node [rectangle, anchor=west, fill=red!50, fill opacity=0.70, align=center, text width=1cm, left] at (0, 1.5) {y w};
    \node [rectangle, anchor=west, fill=red!50, fill opacity=0.70, align=center, text width=1cm, left] at (0, 1.0) {z w};
    \node [rectangle, anchor=west, fill=red!50, fill opacity=0.70, align=center, text width=1cm, left] at (0, 0.5) {w};

    \node [rectangle, anchor=west, fill=red!50, fill opacity=0.70, align=center, above] at (1, 2.5) {$\mathbf{P}_1$};
    \node [rectangle, anchor=west, fill=red!50, fill opacity=0.70, align=center, above] at (3, 2.5) {$\mathbf{P}_2$};
    \node [rectangle, anchor=west, fill=red!50, fill opacity=0.70, align=center, above] at (5, 2.5) {$\mathbf{P}_3$};
    \node [rectangle, anchor=west, fill=red!50, fill opacity=0.70, align=center, above] at (7, 2.5) {$\mathbf{P}_4$};
    \node [rectangle, anchor=west, fill=red!50, fill opacity=0.70, align=center, above] at (9, 2.5) {$\mathbf{P}_5$};
    \node [rectangle, anchor=west, fill=red!50, fill opacity=0.70, align=center, above] at (11, 2.5) {$\mathbf{P}_6$};
    \node [rectangle, anchor=west, fill=red!50, fill opacity=0.70, align=center, above] at (13, 2.5) {$\mathbf{P}_7$};

    \node at (1, 2.0) {0.8660};
    \node at (1, 1.5) {0.5000};
    \node at (1, 1.0) {0.0000};
    \node at (1, 0.5) {1.0000};
    %----------------------------------
    \node at (3, 2.0) {0.0000};
    \node at (3, 1.5) {1.0000};
    \node at (3, 1.0) {0.0000};
    \node at (3, 0.5) {0.5000};
    %----------------------------------
    \node at (5, 2.0) {-0.8660};
    \node at (5, 1.5) {0.5000};
    \node at (5, 1.0) {0.0000};
    \node at (5, 0.5) {1.0000};
    %----------------------------------
    \node at (7, 2.0) {-0.8660};
    \node at (7, 1.5) {-0.5000};
    \node at (7, 1.0) {0.0000};
    \node at (7, 0.5) {0.5000};
    %----------------------------------
    \node at (9, 2.0) {0.0000};
    \node at (9, 1.5) {-1.0000};
    \node at (9, 1.0) {0.0000};
    \node at (9, 0.5) {1.0000};
    %----------------------------------
    \node at (11, 2.0) {0.8660};
    \node at (11, 1.5) {-0.5000};
    \node at (11, 1.0) {0.0000};
    \node at (11, 0.5) {0.5000};
    %----------------------------------
    \node at (13, 2.0) {0.8660};
    \node at (13, 1.5) {0.5000};
    \node at (13, 1.0) {0.0000};
    \node at (13, 0.5) {1.0000};
    %----------------------------------
    }
\end{scope}
\end{tikzpicture}% 
                    \caption{Demonstration of data storage for constructing a circle.}
                    \label{fig:Ch3NURBSCircTriBasedMethod}
                 \end{figure}
                 The arrow labelled $\xi$--direction in the Fig.~\ref{fig:Ch3NURBSCircTriBasedMethod} indicates the direction that parameter values $\xi$ (knot values) grow from the lower to the higher ones (usually from $0$ to $1$ in practices), which correspond to the way from one end of the curve to the other (orient from the first to the second control point). This information is used to generate the curve and identify the direction on it (in order to determine the orientation of the curve -- sign of jacobian of mapping, boundary's positions).
            \item For surface: Control point's coordinates are stored in three--dimensional array. Each frontal slice of the array represent a collection of control points which form the corresponding control polyline in the first direction and similarly for horizontal slices. Polylines in the first direction together with those in the second direction are linked to create the so-called control polygon of the considering surface. The images of these slices of the three--dimensional array in MATLAB can be depicted as in the Fig.~\ref{fig:3darray}. A segment of code along with Fig.~\ref{fig:Ch3SurfAQuarterOfACylinder} are also given below to demonstrate how to declare this type of data structure in MATLAB and the corresponding visualization of this work.
                \begin{figure}[H]
                    \centering
                    \begin{subfigure}[b]{0.3\textwidth}
                        \tikzsetnextfilename{Ch3FrontalSlices}
                        \normalsize
                        \begin{tikzpicture}
    [x={(-0.5cm,-0.5cm)}, y={(1cm,0cm)}, z={(0cm,1cm)},
    scale=2, thin, double, every node/.append style={transform shape}, on grid]
    \newcommand\drawface{\draw[fill=gray!10, fill opacity=0.70] (-1,-1) rectangle (1,1)}
    % face #1
    \begin{scope}[canvas is yz plane at x=-1]
        \drawface;
        %\node[black]  {$\mathbf{T}(:, :, k)$};
        \node[black]  {$\mathbf{T}(:, :, 5)$};
    \end{scope}

    % face #2
    \begin{scope}[canvas is yz plane at x=-0.5]
        \drawface;
        %\node[blue]  {Blue};
        \node[black]  {$\mathbf{T}(:, :, 4)$};
    \end{scope}

    % face #2
    \begin{scope}[canvas is yz plane at x=0]
        \drawface;
        %\node[orange]  {Orange};
        \node[black]  {$\mathbf{T}(:, :, 3)$};
    \end{scope}

    % face #2
    \begin{scope}[canvas is yz plane at x=0.5]
        \drawface;
        %\node[green]  {Green};
        \node[black]  {$\mathbf{T}(:, :, 2)$};
    \end{scope}

    \begin{scope}[canvas is yz plane at x=1]
    \drawface;
    \node[black]  {$\mathbf{T}(:, :, 1)$};
    %\node[red]  {Red};
    %\foreach \x in {-1,0.5,...,1}{
%    \draw (\x,0) -- (\x,1);
%    \draw[line width=1pt] (1*0.5,0) -- (1*0.5,1);
%    \draw[line width=1pt,red] (1*0.25,0) -- (1*0.25,1);
%    \draw[line width=1pt,red] (1*0.75,0) -- (1*0.75,1);
    %}
    \end{scope}
\end{tikzpicture} 
                        \caption{Frontal $\mathbf{T}(:, :, k)$}
                        \label{Ch3FrontalSlices}
                    \end{subfigure}
                    \qquad \qquad
                    \begin{subfigure}[b]{0.3\textwidth}
                        \tikzsetnextfilename{Ch3LateralSlices}
                        \begin{tikzpicture}
    [x={(-0.5cm,-0.5cm)}, y={(1cm,0cm)}, z={(0cm,1cm)},
    scale=2,fill opacity=0.80, thin, double, every node/.append style={transform shape}, on grid]
    \newcommand\drawface{\draw[fill=gray!10] (-1,-1) rectangle (1,1)}
    \begin{scope}[canvas is zx plane at y=-1]
       \drawface;
       %\node[green,rotate=-90] {Green};
        \node[black,rotate=-90]  {$\mathbf{T}(:, 1, :)$};
    \end{scope}
    \begin{scope}[canvas is zx plane at y=-0.5]
       \drawface;
       %\node[green,rotate=-90] {Green};
       \node[black,rotate=-90]  {$\mathbf{T}(:, 2, :)$};
    \end{scope}
    \begin{scope}[canvas is zx plane at y=0]
       \drawface;
       %\node[green,rotate=-90] {Green};
       \node[black,rotate=-90]  {$\mathbf{T}(:, 3, :)$};
    \end{scope}
    \begin{scope}[canvas is zx plane at y=0.5]
       \drawface;
       %\node[green,rotate=-90] {Green};
       \node[black,rotate=-90]  {$\mathbf{T}(:, 4, :)$};
    \end{scope}
    \begin{scope}[canvas is zx plane at y=1]
       \drawface;
       %\node[green,rotate=-90] {Green};
       \node[black,rotate=-90]  {$\mathbf{T}(:, 5, :)$};
    \end{scope}
\end{tikzpicture} 
                        \caption{Lateral $\mathbf{T}(:, j, :)$}
                        \label{fig:Ch3LateralSlices}
                    \end{subfigure}
                    \caption{Slices of a three--dimensional array}
                    \label{fig:3darray}
            \end{figure}
            \begin{lstlisting}
                radius = 1;
                L = 1.5;
                % control points for generate a quarter of a cylinder
                CtrlPts = zeros(4, 3, 2);

                CtrlPts(1 : 3, 1, 1) = [0; 0; 0];
                CtrlPts(1 : 3, 2, 1) = [0; 0; radius];
                CtrlPts(1 : 3, 3, 1) = [0; radius; radius];

                CtrlPts(1 : 3, 1, 2) = [L; 0; 0];
                CtrlPts(1 : 3, 2, 2) = [L; 0; radius];
                CtrlPts(1 : 3, 3, 2) = [L; radius; radius];

                % assign the weights
                CtrlPts(4, :, :) = 1;
                W = 1 / sqrt(2);

                CtrlPts(:, 2, :) = CtrlPts(:, 2, :) * W;
            \end{lstlisting}
            \begin{figure}[H]
                \centering
                \tikzsetnextfilename{Ch3SurfAQuarterOfACylinder}
                \normalsize
                % This file was created by matlab2tikz.
%
\definecolor{mycolor1}{rgb}{0.38000,0.54800,0.24000}%
%
\begin{tikzpicture}

\begin{axis}[%
width=10cm,
scale only axis,
%plot box ratio=1.5 1 1,
point meta min=0,
point meta max=1,
xmin=0,
xmax=1.5,
tick align=outside,
xlabel={x},
ymin=0,
ymax=1,
ylabel={y},
zmin=0,
zmax=1,
zlabel={z},
view={-37.5}{30},
%hide axis,
unit vector ratio=1 1 1,%
]

\addplot3[%
surf,
shader=interp,colormap={mymap}{[1pt] rgb(0pt)=(0.692,0.936,0.936); rgb(2pt)=(0.692,0.936,0.936)},mesh/rows=101]
table[row sep=crcr, point meta=\thisrow{c}] {%
%
x	y	z	c\\
0	0	0	0.710195569525497\\
0	0.000100583311362513	0.0141829653360114	0.71240460418299\\
0	0.000404645907256436	0.0284451766772965	0.714630690536905\\
0	0.000915607803433	0.0427829086109896	0.716916312064106\\
0	0.00163681893109844	0.057192295687301	0.719289019360234\\
0	0.00257155309398959	0.0716693330697585	0.721780367061016\\
0	0.00372300185733414	0.086209877460988	0.724425655492835\\
0	0.00509426838162598	0.100809648312589	0.727263413850304\\
0	0.00668836121491816	0.115464229327074	0.73033458997498\\
0	0.00850818805808555	0.13016907025918	0.733681422514709\\
0	0.0105565495182326	0.144919489023162	0.737345986077549\\
0	0.0128361328661109	0.159710674111862	0.741368418705642\\
0	0.0153495058140643	0.174537687332543	0.745784862958202\\
0	0.0180991103316243	0.189395466863524	0.750625176092311\\
0	0.0210872565164405	0.204278830634725	0.755910489868999\\
0	0.0243161165387281	0.219182480034174	0.761650724664453\\
0	0.0277877186778607	0.234101003941464	0.767842183867616\\
0	0.0315039414701067	0.24902888308801	0.77446537092041\\
0	0.0354665079868145	0.263960494742775	0.781483180780606\\
0	0.0396769802625738	0.278890117720924	0.788839618268074\\
0	0.0441367538930258	0.293811937711588	0.796459186325544\\
0	0.0488470528220538	0.308720052919643	0.804247066923454\\
0	0.0538089243380495	0.323608480015096	0.812090186161949\\
0	0.0590232342988274	0.338471160382329	0.819859213932694\\
0	0.064490662604533	0.3533019666601	0.827411499063639\\
0	0.0702116989375697	0.368094709561873	0.834594885813505\\
0	0.0761866387881432	0.382843144964686	0.841252300287178\\
0	0.0824155797834956	0.397540981253432	0.847226939729082\\
0	0.0888984183382709	0.412181886906127	0.852367847905043\\
0	0.0956348466427212	0.42675949830445	0.856535620027829\\
0	0.102624350004627	0.441267427752584	0.859607954619617\\
0	0.109866204559871	0.455699271686219	0.861484760266623\\
0	0.117359475365564	0.470048619052377	0.862092534251539\\
0	0.125103014888515	0.484309059839721	0.861387758068368\\
0	0.133095461900593	0.498474193737904	0.859359100890819\\
0	0.14133524079124	0.512537638903683	0.856028283765737\\
0	0.149820561306021	0.526493040810599	0.851449530870183\\
0	0.158549418718625	0.540334081158348	0.845707614739505\\
0	0.167519594442214	0.554054486817265	0.83891458434315\\
0	0.176728657084455	0.567648038782867	0.831205342370836\\
0	0.186173963948925	0.581108581114919	0.822732305417528\\
0	0.195852662983903	0.594430029835237	0.813659432944309\\
0	0.205761695177907	0.607606381758231	0.804155944104868\\
0	0.215897797399558	0.620631723228143	0.794390053406969\\
0	0.226257505677663	0.633500238737009	0.784523046062799\\
0	0.236837158915675	0.646206219397571	0.774703982861589\\
0	0.247632903032949	0.658744071245709	0.765065275211015\\
0	0.258640695523528	0.671108323347402	0.755719307772965\\
0	0.269856310421543	0.683293635685828	0.746756214005914\\
0	0.28127534366066	0.695294806804925	0.738242834604296\\
0	0.292893218813452	0.707106781186547	0.730222816007836\\
0	0.304705193195075	0.71872465633934	0.722717741087767\\
0	0.316706364314172	0.730143689578457	0.715729131143265\\
0	0.328891676652598	0.741359304476472	0.709241120561858\\
0	0.341255928754291	0.752367096967051	0.703223584571265\\
0	0.353793780602429	0.763162841084325	0.697635496597417\\
0	0.366499761262991	0.773742494322337	0.692428303594618\\
0	0.379368276771857	0.784102202600442	0.687549132887069\\
0	0.392393618241769	0.794238304822092	0.682943679235329\\
0	0.405569970164763	0.804147337016097	0.678558662193046\\
0	0.418891418885081	0.813826036051075	0.674343787412872\\
0	0.432351961217133	0.823271342915545	0.670253187715551\\
0	0.445945513182735	0.832480405557787	0.666246357336158\\
0	0.459665918841652	0.841450581281375	0.662288623474185\\
0	0.473506959189401	0.850179438693979	0.658351221673191\\
0	0.487462361096317	0.85866475920876	0.654411055137886\\
0	0.501525806262096	0.866904538099407	0.650450223206467\\
0	0.515690940160279	0.874896985111485	0.64645540187496\\
0	0.529951380947623	0.882640524634437	0.642417151055155\\
0	0.544300728313782	0.890133795440129	0.638329210947586\\
0	0.558732572247415	0.897375649995373	0.634187835387498\\
0	0.57324050169555	0.904365153357279	0.629991194998378\\
0	0.587818113093873	0.911101581661729	0.625738868900923\\
0	0.602459018746568	0.917584420216504	0.62143143163314\\
0	0.617156855035314	0.923813361211857	0.617070132480554\\
0	0.631905290438127	0.92978830106243	0.612656657833838\\
0	0.6466980333399	0.935509337395467	0.608192963377087\\
0	0.661528839617671	0.940976765701173	0.603681161495105\\
0	0.676391519984904	0.946191075661951	0.599123449738607\\
0	0.691279947080357	0.951152947177946	0.59452206790159\\
0	0.706188062288412	0.955863246106974	0.589879273662389\\
0	0.721109882279076	0.960323019737426	0.585197329319933\\
0	0.736039505257225	0.964533492013186	0.580478494543978\\
0	0.75097111691199	0.968496058529893	0.575725022018313\\
0	0.765898996058536	0.972212281322139	0.570939154289557\\
0	0.780817519965826	0.975683883461272	0.566123121053175\\
0	0.795721169365275	0.97891274348356	0.561279136605561\\
0	0.810604533136476	0.981900889668376	0.556409397399198\\
0	0.825462312667457	0.984650494185936	0.551516079695736\\
0	0.840289325888138	0.987163867133889	0.546601337319641\\
0	0.855080510976839	0.989443450481768	0.541667299515139\\
0	0.86983092974082	0.991491811941914	0.536716068908598\\
0	0.884535770672926	0.993311638785082	0.53174971957814\\
0	0.899190351687411	0.994905731618374	0.526770295231797\\
0	0.913790122539012	0.996276998142666	0.521779807495107\\
0	0.928330666930242	0.997428446906011	0.516780234308684\\
0	0.942807704312699	0.998363181068902	0.511773518435872\\
0	0.95721709138901	0.999084392196567	0.506761566080168\\
0	0.971554823322703	0.999595354092743	0.501746245611874\\
0	0.985817034663989	0.999899416688637	0.496729386403004\\
0	1	1	0.491669087346906\\
0.015	0	0	0.710195569525497\\
0.015	0.000100583311362513	0.0141829653360114	0.71240460418299\\
0.015	0.000404645907256436	0.0284451766772965	0.714630690536905\\
0.015	0.000915607803433	0.0427829086109896	0.716916312064106\\
0.015	0.00163681893109844	0.057192295687301	0.719289019360234\\
0.015	0.00257155309398959	0.0716693330697585	0.721780367061016\\
0.015	0.00372300185733414	0.086209877460988	0.724425655492835\\
0.015	0.00509426838162598	0.100809648312589	0.727263413850304\\
0.015	0.00668836121491816	0.115464229327074	0.73033458997498\\
0.015	0.00850818805808555	0.13016907025918	0.733681422514709\\
0.015	0.0105565495182326	0.144919489023162	0.737345986077549\\
0.015	0.0128361328661109	0.159710674111862	0.741368418705642\\
0.015	0.0153495058140643	0.174537687332543	0.745784862958201\\
0.015	0.0180991103316243	0.189395466863524	0.750625176092311\\
0.015	0.0210872565164405	0.204278830634725	0.755910489868998\\
0.015	0.0243161165387281	0.219182480034174	0.761650724664453\\
0.015	0.0277877186778607	0.234101003941464	0.767842183867616\\
0.015	0.0315039414701067	0.24902888308801	0.77446537092041\\
0.015	0.0354665079868145	0.263960494742775	0.781483180780606\\
0.015	0.0396769802625738	0.278890117720924	0.788839618268073\\
0.015	0.0441367538930258	0.293811937711588	0.796459186325544\\
0.015	0.0488470528220538	0.308720052919643	0.804247066923454\\
0.015	0.0538089243380495	0.323608480015096	0.81209018616195\\
0.015	0.0590232342988274	0.338471160382329	0.819859213932694\\
0.015	0.064490662604533	0.3533019666601	0.827411499063639\\
0.015	0.0702116989375697	0.368094709561873	0.834594885813505\\
0.015	0.0761866387881432	0.382843144964686	0.84125230028718\\
0.015	0.0824155797834956	0.397540981253432	0.84722693972908\\
0.015	0.0888984183382709	0.412181886906127	0.85236784790504\\
0.015	0.0956348466427212	0.42675949830445	0.85653562002783\\
0.015	0.102624350004627	0.441267427752584	0.859607954619617\\
0.015	0.109866204559871	0.455699271686219	0.86148476026662\\
0.015	0.117359475365564	0.470048619052377	0.862092534251541\\
0.015	0.125103014888515	0.484309059839721	0.861387758068369\\
0.015	0.133095461900593	0.498474193737904	0.859359100890822\\
0.015	0.14133524079124	0.512537638903683	0.856028283765737\\
0.015	0.149820561306021	0.526493040810599	0.851449530870183\\
0.015	0.158549418718625	0.540334081158348	0.845707614739505\\
0.015	0.167519594442214	0.554054486817265	0.83891458434315\\
0.015	0.176728657084455	0.567648038782867	0.831205342370836\\
0.015	0.186173963948925	0.581108581114919	0.822732305417528\\
0.015	0.195852662983903	0.594430029835237	0.813659432944309\\
0.015	0.205761695177907	0.607606381758231	0.804155944104868\\
0.015	0.215897797399558	0.620631723228143	0.79439005340697\\
0.015	0.226257505677663	0.633500238737009	0.784523046062802\\
0.015	0.236837158915675	0.646206219397571	0.774703982861589\\
0.015	0.247632903032949	0.658744071245709	0.765065275211017\\
0.015	0.258640695523528	0.671108323347402	0.755719307772964\\
0.015	0.269856310421543	0.683293635685828	0.746756214005914\\
0.015	0.28127534366066	0.695294806804925	0.738242834604294\\
0.015	0.292893218813452	0.707106781186547	0.730222816007835\\
0.015	0.304705193195075	0.71872465633934	0.722717741087767\\
0.015	0.316706364314172	0.730143689578457	0.715729131143265\\
0.015	0.328891676652598	0.741359304476472	0.709241120561857\\
0.015	0.341255928754291	0.752367096967051	0.703223584571265\\
0.015	0.353793780602429	0.763162841084325	0.697635496597417\\
0.015	0.366499761262991	0.773742494322337	0.692428303594618\\
0.015	0.379368276771857	0.784102202600442	0.687549132887069\\
0.015	0.392393618241769	0.794238304822092	0.682943679235329\\
0.015	0.405569970164763	0.804147337016097	0.678558662193046\\
0.015	0.418891418885081	0.813826036051075	0.674343787412873\\
0.015	0.432351961217132	0.823271342915545	0.670253187715552\\
0.015	0.445945513182735	0.832480405557787	0.666246357336157\\
0.015	0.459665918841652	0.841450581281375	0.662288623474185\\
0.015	0.473506959189401	0.850179438693979	0.658351221673191\\
0.015	0.487462361096317	0.85866475920876	0.654411055137886\\
0.015	0.501525806262096	0.866904538099407	0.650450223206467\\
0.015	0.515690940160279	0.874896985111485	0.646455401874959\\
0.015	0.529951380947623	0.882640524634437	0.642417151055155\\
0.015	0.544300728313782	0.890133795440129	0.638329210947585\\
0.015	0.558732572247415	0.897375649995373	0.634187835387497\\
0.015	0.57324050169555	0.904365153357279	0.629991194998376\\
0.015	0.587818113093873	0.911101581661729	0.625738868900925\\
0.015	0.602459018746568	0.917584420216504	0.62143143163314\\
0.015	0.617156855035314	0.923813361211857	0.617070132480554\\
0.015	0.631905290438127	0.92978830106243	0.612656657833839\\
0.015	0.6466980333399	0.935509337395467	0.608192963377087\\
0.015	0.661528839617671	0.940976765701173	0.603681161495105\\
0.015	0.676391519984904	0.946191075661951	0.599123449738607\\
0.015	0.691279947080357	0.951152947177946	0.59452206790159\\
0.015	0.706188062288412	0.955863246106974	0.58987927366239\\
0.015	0.721109882279076	0.960323019737426	0.585197329319936\\
0.015	0.736039505257225	0.964533492013186	0.580478494543976\\
0.015	0.75097111691199	0.968496058529893	0.575725022018313\\
0.015	0.765898996058536	0.972212281322139	0.570939154289558\\
0.015	0.780817519965826	0.975683883461272	0.566123121053174\\
0.015	0.795721169365275	0.97891274348356	0.561279136605558\\
0.015	0.810604533136476	0.981900889668376	0.556409397399199\\
0.015	0.825462312667457	0.984650494185936	0.551516079695736\\
0.015	0.840289325888138	0.987163867133889	0.546601337319641\\
0.015	0.855080510976839	0.989443450481767	0.541667299515139\\
0.015	0.86983092974082	0.991491811941914	0.536716068908598\\
0.015	0.884535770672926	0.993311638785082	0.531749719578141\\
0.015	0.899190351687411	0.994905731618374	0.526770295231794\\
0.015	0.913790122539012	0.996276998142666	0.521779807495104\\
0.015	0.928330666930242	0.997428446906011	0.516780234308685\\
0.015	0.942807704312699	0.998363181068902	0.511773518435872\\
0.015	0.95721709138901	0.999084392196567	0.506761566080168\\
0.015	0.971554823322703	0.999595354092743	0.501746245611874\\
0.015	0.985817034663989	0.999899416688637	0.496729386403004\\
0.015	1	1	0.491669087346906\\
0.03	0	0	0.710195569525497\\
0.03	0.000100583311362513	0.0141829653360114	0.71240460418299\\
0.03	0.000404645907256436	0.0284451766772965	0.714630690536905\\
0.03	0.000915607803433	0.0427829086109896	0.716916312064106\\
0.03	0.00163681893109844	0.057192295687301	0.719289019360234\\
0.03	0.00257155309398959	0.0716693330697585	0.721780367061016\\
0.03	0.00372300185733414	0.086209877460988	0.724425655492835\\
0.03	0.00509426838162598	0.100809648312589	0.727263413850304\\
0.03	0.00668836121491816	0.115464229327074	0.73033458997498\\
0.03	0.00850818805808555	0.13016907025918	0.733681422514709\\
0.03	0.0105565495182326	0.144919489023162	0.737345986077549\\
0.03	0.0128361328661109	0.159710674111862	0.741368418705642\\
0.03	0.0153495058140643	0.174537687332543	0.745784862958202\\
0.03	0.0180991103316243	0.189395466863524	0.750625176092311\\
0.03	0.0210872565164405	0.204278830634725	0.755910489868999\\
0.03	0.0243161165387281	0.219182480034174	0.761650724664453\\
0.03	0.0277877186778607	0.234101003941464	0.767842183867616\\
0.03	0.0315039414701067	0.24902888308801	0.77446537092041\\
0.03	0.0354665079868145	0.263960494742775	0.781483180780606\\
0.03	0.0396769802625738	0.278890117720924	0.788839618268073\\
0.03	0.0441367538930258	0.293811937711588	0.796459186325544\\
0.03	0.0488470528220538	0.308720052919643	0.804247066923454\\
0.03	0.0538089243380495	0.323608480015096	0.81209018616195\\
0.03	0.0590232342988274	0.338471160382329	0.819859213932694\\
0.03	0.064490662604533	0.3533019666601	0.82741149906364\\
0.03	0.0702116989375697	0.368094709561873	0.834594885813504\\
0.03	0.0761866387881433	0.382843144964686	0.841252300287178\\
0.03	0.0824155797834956	0.397540981253432	0.847226939729081\\
0.03	0.0888984183382709	0.412181886906127	0.852367847905042\\
0.03	0.0956348466427212	0.42675949830445	0.856535620027829\\
0.03	0.102624350004627	0.441267427752584	0.859607954619619\\
0.03	0.109866204559871	0.455699271686218	0.861484760266622\\
0.03	0.117359475365564	0.470048619052377	0.862092534251541\\
0.03	0.125103014888515	0.484309059839721	0.861387758068367\\
0.03	0.133095461900593	0.498474193737904	0.85935910089082\\
0.03	0.14133524079124	0.512537638903683	0.856028283765737\\
0.03	0.149820561306021	0.526493040810599	0.851449530870183\\
0.03	0.158549418718625	0.540334081158348	0.845707614739502\\
0.03	0.167519594442214	0.554054486817265	0.838914584343152\\
0.03	0.176728657084455	0.567648038782867	0.831205342370836\\
0.03	0.186173963948925	0.581108581114919	0.822732305417528\\
0.03	0.195852662983903	0.594430029835237	0.813659432944309\\
0.03	0.205761695177907	0.607606381758231	0.804155944104867\\
0.03	0.215897797399558	0.620631723228143	0.79439005340697\\
0.03	0.226257505677663	0.633500238737009	0.7845230460628\\
0.03	0.236837158915675	0.646206219397571	0.774703982861589\\
0.03	0.247632903032949	0.658744071245709	0.765065275211015\\
0.03	0.258640695523528	0.671108323347402	0.755719307772963\\
0.03	0.269856310421543	0.683293635685828	0.746756214005915\\
0.03	0.28127534366066	0.695294806804925	0.738242834604294\\
0.03	0.292893218813452	0.707106781186547	0.730222816007835\\
0.03	0.304705193195076	0.71872465633934	0.722717741087766\\
0.03	0.316706364314172	0.730143689578457	0.715729131143265\\
0.03	0.328891676652598	0.741359304476472	0.709241120561857\\
0.03	0.341255928754291	0.752367096967051	0.703223584571265\\
0.03	0.353793780602429	0.763162841084325	0.697635496597418\\
0.03	0.366499761262991	0.773742494322337	0.692428303594618\\
0.03	0.379368276771857	0.784102202600442	0.687549132887069\\
0.03	0.392393618241769	0.794238304822092	0.682943679235329\\
0.03	0.405569970164763	0.804147337016097	0.678558662193046\\
0.03	0.418891418885081	0.813826036051075	0.674343787412871\\
0.03	0.432351961217133	0.823271342915545	0.670253187715552\\
0.03	0.445945513182735	0.832480405557787	0.666246357336158\\
0.03	0.459665918841652	0.841450581281375	0.662288623474185\\
0.03	0.473506959189401	0.850179438693979	0.658351221673191\\
0.03	0.487462361096317	0.85866475920876	0.654411055137886\\
0.03	0.501525806262096	0.866904538099407	0.650450223206468\\
0.03	0.51569094016028	0.874896985111485	0.646455401874959\\
0.03	0.529951380947623	0.882640524634437	0.642417151055155\\
0.03	0.544300728313782	0.890133795440129	0.638329210947585\\
0.03	0.558732572247415	0.897375649995373	0.634187835387497\\
0.03	0.57324050169555	0.904365153357279	0.629991194998375\\
0.03	0.587818113093873	0.911101581661729	0.625738868900923\\
0.03	0.602459018746568	0.917584420216504	0.621431431633141\\
0.03	0.617156855035314	0.923813361211857	0.617070132480552\\
0.03	0.631905290438127	0.92978830106243	0.612656657833838\\
0.03	0.6466980333399	0.935509337395467	0.608192963377088\\
0.03	0.661528839617671	0.940976765701173	0.603681161495105\\
0.03	0.676391519984904	0.946191075661951	0.599123449738607\\
0.03	0.691279947080357	0.951152947177946	0.594522067901591\\
0.03	0.706188062288412	0.955863246106974	0.589879273662389\\
0.03	0.721109882279076	0.960323019737426	0.585197329319935\\
0.03	0.736039505257225	0.964533492013186	0.580478494543978\\
0.03	0.75097111691199	0.968496058529893	0.575725022018312\\
0.03	0.765898996058536	0.972212281322139	0.570939154289558\\
0.03	0.780817519965826	0.975683883461272	0.566123121053176\\
0.03	0.795721169365275	0.97891274348356	0.561279136605558\\
0.03	0.810604533136476	0.981900889668376	0.556409397399198\\
0.03	0.825462312667457	0.984650494185936	0.551516079695736\\
0.03	0.840289325888138	0.987163867133889	0.546601337319641\\
0.03	0.855080510976839	0.989443450481768	0.541667299515139\\
0.03	0.86983092974082	0.991491811941914	0.536716068908596\\
0.03	0.884535770672926	0.993311638785082	0.531749719578142\\
0.03	0.899190351687411	0.994905731618374	0.526770295231798\\
0.03	0.913790122539012	0.996276998142666	0.521779807495102\\
0.03	0.928330666930242	0.997428446906011	0.516780234308684\\
0.03	0.942807704312699	0.998363181068902	0.511773518435873\\
0.03	0.95721709138901	0.999084392196567	0.506761566080168\\
0.03	0.971554823322703	0.999595354092743	0.501746245611874\\
0.03	0.985817034663989	0.999899416688637	0.496729386403004\\
0.03	1	1	0.491669087346906\\
0.045	0	0	0.710195569525497\\
0.045	0.000100583311362513	0.0141829653360114	0.71240460418299\\
0.045	0.000404645907256436	0.0284451766772965	0.714630690536905\\
0.045	0.000915607803433	0.0427829086109896	0.716916312064106\\
0.045	0.00163681893109844	0.057192295687301	0.719289019360234\\
0.045	0.00257155309398959	0.0716693330697585	0.721780367061016\\
0.045	0.00372300185733414	0.086209877460988	0.724425655492835\\
0.045	0.00509426838162598	0.100809648312589	0.727263413850304\\
0.045	0.00668836121491816	0.115464229327074	0.73033458997498\\
0.045	0.00850818805808555	0.13016907025918	0.733681422514709\\
0.045	0.0105565495182326	0.144919489023162	0.737345986077549\\
0.045	0.0128361328661109	0.159710674111862	0.741368418705642\\
0.045	0.0153495058140643	0.174537687332543	0.745784862958202\\
0.045	0.0180991103316243	0.189395466863524	0.750625176092311\\
0.045	0.0210872565164405	0.204278830634725	0.755910489868999\\
0.045	0.0243161165387281	0.219182480034174	0.761650724664453\\
0.045	0.0277877186778607	0.234101003941464	0.767842183867616\\
0.045	0.0315039414701067	0.24902888308801	0.77446537092041\\
0.045	0.0354665079868145	0.263960494742775	0.781483180780606\\
0.045	0.0396769802625738	0.278890117720924	0.788839618268073\\
0.045	0.0441367538930258	0.293811937711588	0.796459186325545\\
0.045	0.0488470528220538	0.308720052919643	0.804247066923454\\
0.045	0.0538089243380495	0.323608480015096	0.812090186161949\\
0.045	0.0590232342988274	0.338471160382329	0.819859213932694\\
0.045	0.064490662604533	0.3533019666601	0.827411499063639\\
0.045	0.0702116989375697	0.368094709561873	0.834594885813505\\
0.045	0.0761866387881432	0.382843144964686	0.841252300287178\\
0.045	0.0824155797834956	0.397540981253432	0.847226939729081\\
0.045	0.0888984183382709	0.412181886906127	0.852367847905042\\
0.045	0.0956348466427212	0.42675949830445	0.856535620027828\\
0.045	0.102624350004627	0.441267427752584	0.859607954619617\\
0.045	0.109866204559871	0.455699271686219	0.861484760266623\\
0.045	0.117359475365564	0.470048619052377	0.862092534251538\\
0.045	0.125103014888515	0.484309059839721	0.861387758068366\\
0.045	0.133095461900593	0.498474193737904	0.859359100890819\\
0.045	0.14133524079124	0.512537638903683	0.856028283765737\\
0.045	0.149820561306021	0.526493040810599	0.851449530870182\\
0.045	0.158549418718625	0.540334081158348	0.845707614739505\\
0.045	0.167519594442214	0.554054486817265	0.83891458434315\\
0.045	0.176728657084455	0.567648038782867	0.831205342370836\\
0.045	0.186173963948925	0.581108581114919	0.82273230541753\\
0.045	0.195852662983903	0.594430029835237	0.813659432944309\\
0.045	0.205761695177907	0.607606381758231	0.804155944104868\\
0.045	0.215897797399558	0.620631723228143	0.79439005340697\\
0.045	0.226257505677663	0.633500238737009	0.784523046062799\\
0.045	0.236837158915675	0.646206219397571	0.774703982861591\\
0.045	0.247632903032949	0.658744071245709	0.765065275211015\\
0.045	0.258640695523528	0.671108323347402	0.755719307772965\\
0.045	0.269856310421543	0.683293635685828	0.746756214005914\\
0.045	0.28127534366066	0.695294806804925	0.738242834604295\\
0.045	0.292893218813453	0.707106781186548	0.730222816007832\\
0.045	0.304705193195075	0.71872465633934	0.722717741087767\\
0.045	0.316706364314172	0.730143689578457	0.715729131143268\\
0.045	0.328891676652598	0.741359304476472	0.709241120561857\\
0.045	0.341255928754291	0.752367096967051	0.703223584571265\\
0.045	0.353793780602429	0.763162841084325	0.697635496597417\\
0.045	0.366499761262991	0.773742494322337	0.692428303594618\\
0.045	0.379368276771857	0.784102202600442	0.687549132887069\\
0.045	0.392393618241769	0.794238304822092	0.682943679235329\\
0.045	0.405569970164763	0.804147337016097	0.678558662193046\\
0.045	0.418891418885081	0.813826036051075	0.674343787412872\\
0.045	0.432351961217132	0.823271342915545	0.670253187715552\\
0.045	0.445945513182735	0.832480405557787	0.666246357336157\\
0.045	0.459665918841652	0.841450581281375	0.662288623474185\\
0.045	0.473506959189401	0.850179438693979	0.658351221673192\\
0.045	0.487462361096317	0.85866475920876	0.654411055137886\\
0.045	0.501525806262096	0.866904538099407	0.650450223206467\\
0.045	0.515690940160279	0.874896985111485	0.64645540187496\\
0.045	0.529951380947623	0.882640524634437	0.642417151055155\\
0.045	0.544300728313782	0.890133795440129	0.638329210947586\\
0.045	0.558732572247415	0.897375649995373	0.634187835387498\\
0.045	0.57324050169555	0.904365153357279	0.629991194998378\\
0.045	0.587818113093873	0.911101581661729	0.625738868900924\\
0.045	0.602459018746568	0.917584420216504	0.621431431633137\\
0.045	0.617156855035314	0.923813361211857	0.617070132480554\\
0.045	0.631905290438127	0.92978830106243	0.612656657833839\\
0.045	0.6466980333399	0.935509337395467	0.608192963377087\\
0.045	0.661528839617671	0.940976765701173	0.603681161495104\\
0.045	0.676391519984904	0.946191075661951	0.599123449738608\\
0.045	0.691279947080357	0.951152947177946	0.594522067901589\\
0.045	0.706188062288412	0.955863246106974	0.58987927366239\\
0.045	0.721109882279076	0.960323019737426	0.585197329319937\\
0.045	0.736039505257225	0.964533492013186	0.580478494543976\\
0.045	0.75097111691199	0.968496058529893	0.575725022018313\\
0.045	0.765898996058536	0.972212281322139	0.570939154289559\\
0.045	0.780817519965826	0.975683883461272	0.566123121053173\\
0.045	0.795721169365275	0.97891274348356	0.561279136605558\\
0.045	0.810604533136476	0.981900889668376	0.556409397399199\\
0.045	0.825462312667457	0.984650494185936	0.551516079695734\\
0.045	0.840289325888138	0.987163867133889	0.546601337319641\\
0.045	0.855080510976839	0.989443450481767	0.54166729951514\\
0.045	0.86983092974082	0.991491811941914	0.536716068908599\\
0.045	0.884535770672926	0.993311638785082	0.53174971957814\\
0.045	0.899190351687411	0.994905731618374	0.526770295231795\\
0.045	0.913790122539012	0.996276998142666	0.521779807495104\\
0.045	0.928330666930242	0.997428446906011	0.516780234308682\\
0.045	0.942807704312699	0.998363181068902	0.511773518435872\\
0.045	0.95721709138901	0.999084392196567	0.50676156608017\\
0.045	0.971554823322703	0.999595354092743	0.501746245611874\\
0.045	0.985817034663989	0.999899416688637	0.496729386403004\\
0.045	1	1	0.491669087346906\\
0.06	0	0	0.710195569525497\\
0.06	0.000100583311362513	0.0141829653360114	0.71240460418299\\
0.06	0.000404645907256436	0.0284451766772965	0.714630690536905\\
0.06	0.000915607803433	0.0427829086109896	0.716916312064106\\
0.06	0.00163681893109844	0.057192295687301	0.719289019360234\\
0.06	0.00257155309398959	0.0716693330697585	0.721780367061016\\
0.06	0.00372300185733414	0.086209877460988	0.724425655492835\\
0.06	0.00509426838162598	0.100809648312589	0.727263413850304\\
0.06	0.00668836121491816	0.115464229327074	0.73033458997498\\
0.06	0.00850818805808555	0.13016907025918	0.733681422514709\\
0.06	0.0105565495182326	0.144919489023162	0.737345986077549\\
0.06	0.0128361328661109	0.159710674111862	0.741368418705642\\
0.06	0.0153495058140643	0.174537687332543	0.745784862958202\\
0.06	0.0180991103316243	0.189395466863524	0.750625176092311\\
0.06	0.0210872565164405	0.204278830634725	0.755910489868998\\
0.06	0.0243161165387281	0.219182480034174	0.761650724664453\\
0.06	0.0277877186778607	0.234101003941464	0.767842183867616\\
0.06	0.0315039414701067	0.24902888308801	0.77446537092041\\
0.06	0.0354665079868145	0.263960494742775	0.781483180780606\\
0.06	0.0396769802625738	0.278890117720924	0.788839618268073\\
0.06	0.0441367538930257	0.293811937711588	0.796459186325544\\
0.06	0.0488470528220538	0.308720052919643	0.804247066923454\\
0.06	0.0538089243380495	0.323608480015096	0.81209018616195\\
0.06	0.0590232342988274	0.338471160382329	0.819859213932694\\
0.06	0.064490662604533	0.3533019666601	0.827411499063639\\
0.06	0.0702116989375697	0.368094709561873	0.834594885813505\\
0.06	0.0761866387881432	0.382843144964686	0.841252300287181\\
0.06	0.0824155797834956	0.397540981253432	0.84722693972908\\
0.06	0.0888984183382709	0.412181886906127	0.85236784790504\\
0.06	0.0956348466427213	0.42675949830445	0.856535620027829\\
0.06	0.102624350004627	0.441267427752584	0.859607954619614\\
0.06	0.109866204559871	0.455699271686219	0.86148476026662\\
0.06	0.117359475365564	0.470048619052377	0.862092534251541\\
0.06	0.125103014888515	0.484309059839721	0.861387758068371\\
0.06	0.133095461900593	0.498474193737904	0.85935910089082\\
0.06	0.14133524079124	0.512537638903683	0.856028283765738\\
0.06	0.149820561306021	0.526493040810599	0.851449530870183\\
0.06	0.158549418718625	0.540334081158348	0.845707614739509\\
0.06	0.167519594442214	0.554054486817265	0.838914584343148\\
0.06	0.176728657084455	0.567648038782867	0.831205342370837\\
0.06	0.186173963948925	0.581108581114919	0.822732305417529\\
0.06	0.195852662983903	0.594430029835237	0.813659432944308\\
0.06	0.205761695177907	0.607606381758231	0.804155944104868\\
0.06	0.215897797399558	0.620631723228143	0.794390053406969\\
0.06	0.226257505677663	0.633500238737009	0.784523046062803\\
0.06	0.236837158915675	0.646206219397571	0.774703982861588\\
0.06	0.247632903032949	0.658744071245709	0.765065275211014\\
0.06	0.258640695523528	0.671108323347402	0.755719307772965\\
0.06	0.269856310421543	0.683293635685828	0.746756214005914\\
0.06	0.28127534366066	0.695294806804925	0.738242834604293\\
0.06	0.292893218813453	0.707106781186547	0.730222816007835\\
0.06	0.304705193195075	0.71872465633934	0.722717741087768\\
0.06	0.316706364314172	0.730143689578457	0.715729131143264\\
0.06	0.328891676652598	0.741359304476472	0.709241120561857\\
0.06	0.341255928754291	0.752367096967051	0.703223584571266\\
0.06	0.353793780602429	0.763162841084325	0.697635496597418\\
0.06	0.366499761262991	0.773742494322337	0.692428303594618\\
0.06	0.379368276771857	0.784102202600442	0.687549132887069\\
0.06	0.392393618241769	0.794238304822092	0.682943679235329\\
0.06	0.405569970164763	0.804147337016097	0.678558662193047\\
0.06	0.418891418885081	0.813826036051075	0.674343787412872\\
0.06	0.432351961217133	0.823271342915545	0.67025318771555\\
0.06	0.445945513182735	0.832480405557786	0.666246357336159\\
0.06	0.459665918841652	0.841450581281375	0.662288623474185\\
0.06	0.473506959189401	0.850179438693979	0.658351221673191\\
0.06	0.487462361096317	0.85866475920876	0.654411055137886\\
0.06	0.501525806262096	0.866904538099407	0.650450223206467\\
0.06	0.515690940160279	0.874896985111485	0.646455401874959\\
0.06	0.529951380947623	0.882640524634437	0.642417151055155\\
0.06	0.544300728313782	0.890133795440129	0.638329210947584\\
0.06	0.558732572247415	0.897375649995372	0.634187835387498\\
0.06	0.57324050169555	0.904365153357279	0.629991194998376\\
0.06	0.587818113093873	0.911101581661729	0.625738868900925\\
0.06	0.602459018746568	0.917584420216504	0.621431431633139\\
0.06	0.617156855035314	0.923813361211857	0.617070132480554\\
0.06	0.631905290438127	0.92978830106243	0.61265665783384\\
0.06	0.6466980333399	0.935509337395467	0.608192963377088\\
0.06	0.661528839617671	0.940976765701173	0.603681161495105\\
0.06	0.676391519984904	0.946191075661951	0.599123449738608\\
0.06	0.691279947080357	0.951152947177946	0.59452206790159\\
0.06	0.706188062288412	0.955863246106974	0.589879273662387\\
0.06	0.721109882279076	0.960323019737426	0.585197329319936\\
0.06	0.736039505257226	0.964533492013186	0.580478494543978\\
0.06	0.75097111691199	0.968496058529893	0.575725022018314\\
0.06	0.765898996058536	0.972212281322139	0.570939154289559\\
0.06	0.780817519965826	0.975683883461272	0.566123121053176\\
0.06	0.795721169365275	0.97891274348356	0.561279136605556\\
0.06	0.810604533136476	0.981900889668376	0.5564093973992\\
0.06	0.825462312667457	0.984650494185936	0.551516079695735\\
0.06	0.840289325888138	0.987163867133889	0.54660133731964\\
0.06	0.855080510976839	0.989443450481768	0.541667299515141\\
0.06	0.86983092974082	0.991491811941914	0.536716068908598\\
0.06	0.884535770672926	0.993311638785082	0.531749719578142\\
0.06	0.899190351687411	0.994905731618374	0.526770295231796\\
0.06	0.913790122539012	0.996276998142666	0.521779807495105\\
0.06	0.928330666930242	0.997428446906011	0.516780234308685\\
0.06	0.942807704312699	0.998363181068902	0.511773518435871\\
0.06	0.95721709138901	0.999084392196567	0.506761566080166\\
0.06	0.971554823322704	0.999595354092743	0.501746245611874\\
0.06	0.985817034663989	0.999899416688637	0.496729386403004\\
0.06	1	1	0.491669087346906\\
0.075	0	0	0.710195569525497\\
0.075	0.000100583311362513	0.0141829653360114	0.71240460418299\\
0.075	0.000404645907256436	0.0284451766772965	0.714630690536905\\
0.075	0.000915607803433	0.0427829086109896	0.716916312064106\\
0.075	0.00163681893109844	0.057192295687301	0.719289019360234\\
0.075	0.00257155309398959	0.0716693330697584	0.721780367061016\\
0.075	0.00372300185733414	0.086209877460988	0.724425655492835\\
0.075	0.00509426838162598	0.100809648312589	0.727263413850304\\
0.075	0.00668836121491816	0.115464229327074	0.73033458997498\\
0.075	0.00850818805808555	0.13016907025918	0.733681422514709\\
0.075	0.0105565495182326	0.144919489023162	0.737345986077549\\
0.075	0.0128361328661109	0.159710674111862	0.741368418705642\\
0.075	0.0153495058140643	0.174537687332543	0.745784862958201\\
0.075	0.0180991103316243	0.189395466863524	0.750625176092311\\
0.075	0.0210872565164405	0.204278830634725	0.755910489868999\\
0.075	0.0243161165387281	0.219182480034174	0.761650724664453\\
0.075	0.0277877186778607	0.234101003941464	0.767842183867616\\
0.075	0.0315039414701067	0.24902888308801	0.77446537092041\\
0.075	0.0354665079868145	0.263960494742775	0.781483180780606\\
0.075	0.0396769802625738	0.278890117720924	0.788839618268073\\
0.075	0.0441367538930258	0.293811937711588	0.796459186325544\\
0.075	0.0488470528220538	0.308720052919643	0.804247066923454\\
0.075	0.0538089243380495	0.323608480015097	0.81209018616195\\
0.075	0.0590232342988274	0.338471160382329	0.819859213932692\\
0.075	0.064490662604533	0.3533019666601	0.827411499063639\\
0.075	0.0702116989375697	0.368094709561873	0.834594885813504\\
0.075	0.0761866387881432	0.382843144964686	0.84125230028718\\
0.075	0.0824155797834956	0.397540981253432	0.847226939729081\\
0.075	0.0888984183382709	0.412181886906127	0.85236784790504\\
0.075	0.0956348466427213	0.42675949830445	0.856535620027831\\
0.075	0.102624350004627	0.441267427752584	0.859607954619618\\
0.075	0.109866204559871	0.455699271686218	0.86148476026662\\
0.075	0.117359475365564	0.470048619052377	0.862092534251541\\
0.075	0.125103014888515	0.484309059839721	0.861387758068372\\
0.075	0.133095461900593	0.498474193737904	0.859359100890822\\
0.075	0.14133524079124	0.512537638903683	0.856028283765737\\
0.075	0.149820561306021	0.526493040810599	0.851449530870182\\
0.075	0.158549418718625	0.540334081158348	0.845707614739505\\
0.075	0.167519594442214	0.554054486817265	0.838914584343151\\
0.075	0.176728657084455	0.567648038782867	0.831205342370835\\
0.075	0.186173963948925	0.581108581114919	0.822732305417527\\
0.075	0.195852662983903	0.594430029835237	0.813659432944309\\
0.075	0.205761695177907	0.607606381758231	0.804155944104866\\
0.075	0.215897797399558	0.620631723228143	0.794390053406973\\
0.075	0.226257505677663	0.633500238737009	0.784523046062798\\
0.075	0.236837158915675	0.646206219397571	0.774703982861588\\
0.075	0.247632903032949	0.658744071245709	0.765065275211018\\
0.075	0.258640695523528	0.671108323347402	0.755719307772964\\
0.075	0.269856310421543	0.683293635685828	0.746756214005911\\
0.075	0.28127534366066	0.695294806804925	0.738242834604294\\
0.075	0.292893218813452	0.707106781186547	0.730222816007835\\
0.075	0.304705193195076	0.71872465633934	0.722717741087767\\
0.075	0.316706364314172	0.730143689578457	0.715729131143266\\
0.075	0.328891676652598	0.741359304476472	0.709241120561857\\
0.075	0.341255928754291	0.752367096967051	0.703223584571265\\
0.075	0.353793780602429	0.763162841084325	0.697635496597417\\
0.075	0.366499761262991	0.773742494322337	0.692428303594618\\
0.075	0.379368276771857	0.784102202600442	0.687549132887069\\
0.075	0.392393618241769	0.794238304822092	0.682943679235328\\
0.075	0.405569970164763	0.804147337016097	0.678558662193045\\
0.075	0.418891418885081	0.813826036051075	0.674343787412872\\
0.075	0.432351961217132	0.823271342915545	0.670253187715552\\
0.075	0.445945513182735	0.832480405557787	0.666246357336158\\
0.075	0.459665918841652	0.841450581281375	0.662288623474185\\
0.075	0.473506959189401	0.850179438693979	0.658351221673192\\
0.075	0.487462361096317	0.85866475920876	0.654411055137886\\
0.075	0.501525806262096	0.866904538099407	0.650450223206467\\
0.075	0.515690940160279	0.874896985111485	0.64645540187496\\
0.075	0.529951380947623	0.882640524634437	0.642417151055155\\
0.075	0.544300728313782	0.890133795440129	0.638329210947584\\
0.075	0.558732572247415	0.897375649995373	0.634187835387498\\
0.075	0.57324050169555	0.904365153357279	0.629991194998376\\
0.075	0.587818113093873	0.911101581661729	0.625738868900924\\
0.075	0.602459018746568	0.917584420216504	0.621431431633139\\
0.075	0.617156855035314	0.923813361211857	0.617070132480553\\
0.075	0.631905290438127	0.92978830106243	0.612656657833839\\
0.075	0.6466980333399	0.935509337395467	0.608192963377087\\
0.075	0.661528839617671	0.940976765701173	0.603681161495106\\
0.075	0.676391519984904	0.946191075661951	0.599123449738607\\
0.075	0.691279947080357	0.951152947177946	0.594522067901591\\
0.075	0.706188062288412	0.955863246106974	0.58987927366239\\
0.075	0.721109882279076	0.960323019737426	0.585197329319936\\
0.075	0.736039505257225	0.964533492013186	0.580478494543977\\
0.075	0.75097111691199	0.968496058529893	0.575725022018313\\
0.075	0.765898996058536	0.972212281322139	0.570939154289558\\
0.075	0.780817519965825	0.975683883461272	0.566123121053176\\
0.075	0.795721169365275	0.97891274348356	0.561279136605556\\
0.075	0.810604533136476	0.981900889668376	0.556409397399197\\
0.075	0.825462312667457	0.984650494185936	0.551516079695736\\
0.075	0.840289325888138	0.987163867133889	0.546601337319642\\
0.075	0.855080510976838	0.989443450481767	0.54166729951514\\
0.075	0.86983092974082	0.991491811941914	0.536716068908596\\
0.075	0.884535770672926	0.993311638785082	0.531749719578142\\
0.075	0.899190351687411	0.994905731618374	0.526770295231796\\
0.075	0.913790122539012	0.996276998142666	0.521779807495104\\
0.075	0.928330666930242	0.997428446906011	0.516780234308687\\
0.075	0.942807704312699	0.998363181068902	0.511773518435872\\
0.075	0.95721709138901	0.999084392196567	0.506761566080166\\
0.075	0.971554823322703	0.999595354092743	0.501746245611874\\
0.075	0.985817034663989	0.999899416688637	0.496729386403004\\
0.075	1	1	0.491669087346906\\
0.09	0	0	0.710195569525497\\
0.09	0.000100583311362513	0.0141829653360114	0.71240460418299\\
0.09	0.000404645907256436	0.0284451766772965	0.714630690536905\\
0.09	0.000915607803433	0.0427829086109896	0.716916312064106\\
0.09	0.00163681893109844	0.057192295687301	0.719289019360234\\
0.09	0.00257155309398959	0.0716693330697584	0.721780367061016\\
0.09	0.00372300185733414	0.086209877460988	0.724425655492835\\
0.09	0.00509426838162598	0.100809648312589	0.727263413850304\\
0.09	0.00668836121491816	0.115464229327074	0.73033458997498\\
0.09	0.00850818805808555	0.13016907025918	0.733681422514709\\
0.09	0.0105565495182326	0.144919489023162	0.737345986077549\\
0.09	0.0128361328661109	0.159710674111862	0.741368418705642\\
0.09	0.0153495058140643	0.174537687332543	0.745784862958202\\
0.09	0.0180991103316243	0.189395466863524	0.750625176092311\\
0.09	0.0210872565164405	0.204278830634725	0.755910489868998\\
0.09	0.0243161165387281	0.219182480034174	0.761650724664453\\
0.09	0.0277877186778607	0.234101003941464	0.767842183867616\\
0.09	0.0315039414701067	0.24902888308801	0.77446537092041\\
0.09	0.0354665079868145	0.263960494742775	0.781483180780607\\
0.09	0.0396769802625738	0.278890117720924	0.788839618268073\\
0.09	0.0441367538930258	0.293811937711588	0.796459186325544\\
0.09	0.0488470528220538	0.308720052919643	0.804247066923454\\
0.09	0.0538089243380495	0.323608480015096	0.812090186161949\\
0.09	0.0590232342988274	0.338471160382329	0.819859213932694\\
0.09	0.064490662604533	0.3533019666601	0.827411499063639\\
0.09	0.0702116989375697	0.368094709561873	0.834594885813504\\
0.09	0.0761866387881433	0.382843144964686	0.841252300287177\\
0.09	0.0824155797834956	0.397540981253432	0.847226939729081\\
0.09	0.0888984183382709	0.412181886906127	0.852367847905042\\
0.09	0.0956348466427212	0.42675949830445	0.856535620027829\\
0.09	0.102624350004627	0.441267427752584	0.859607954619619\\
0.09	0.109866204559871	0.455699271686218	0.861484760266624\\
0.09	0.117359475365564	0.470048619052377	0.862092534251539\\
0.09	0.125103014888515	0.484309059839721	0.861387758068369\\
0.09	0.133095461900593	0.498474193737904	0.859359100890822\\
0.09	0.14133524079124	0.512537638903683	0.856028283765735\\
0.09	0.149820561306021	0.526493040810599	0.851449530870183\\
0.09	0.158549418718625	0.540334081158348	0.845707614739501\\
0.09	0.167519594442214	0.554054486817265	0.83891458434315\\
0.09	0.176728657084455	0.567648038782867	0.831205342370837\\
0.09	0.186173963948925	0.581108581114919	0.822732305417529\\
0.09	0.195852662983903	0.594430029835237	0.813659432944308\\
0.09	0.205761695177907	0.607606381758231	0.804155944104869\\
0.09	0.215897797399558	0.620631723228144	0.79439005340697\\
0.09	0.226257505677663	0.633500238737009	0.784523046062799\\
0.09	0.236837158915675	0.646206219397571	0.774703982861593\\
0.09	0.247632903032949	0.658744071245709	0.765065275211016\\
0.09	0.258640695523528	0.671108323347402	0.75571930777296\\
0.09	0.269856310421543	0.683293635685828	0.746756214005913\\
0.09	0.28127534366066	0.695294806804925	0.738242834604297\\
0.09	0.292893218813452	0.707106781186547	0.730222816007834\\
0.09	0.304705193195076	0.71872465633934	0.722717741087769\\
0.09	0.316706364314172	0.730143689578457	0.715729131143266\\
0.09	0.328891676652598	0.741359304476472	0.709241120561856\\
0.09	0.341255928754291	0.752367096967051	0.703223584571265\\
0.09	0.353793780602429	0.763162841084325	0.697635496597418\\
0.09	0.366499761262991	0.773742494322337	0.692428303594619\\
0.09	0.379368276771857	0.784102202600443	0.68754913288707\\
0.09	0.392393618241769	0.794238304822092	0.682943679235328\\
0.09	0.405569970164763	0.804147337016097	0.678558662193046\\
0.09	0.418891418885081	0.813826036051075	0.674343787412872\\
0.09	0.432351961217133	0.823271342915545	0.670253187715551\\
0.09	0.445945513182735	0.832480405557787	0.666246357336159\\
0.09	0.459665918841652	0.841450581281375	0.662288623474184\\
0.09	0.473506959189401	0.850179438693979	0.65835122167319\\
0.09	0.487462361096317	0.85866475920876	0.654411055137887\\
0.09	0.501525806262096	0.866904538099407	0.650450223206468\\
0.09	0.51569094016028	0.874896985111485	0.646455401874959\\
0.09	0.529951380947623	0.882640524634437	0.642417151055156\\
0.09	0.544300728313782	0.890133795440129	0.638329210947585\\
0.09	0.558732572247415	0.897375649995373	0.634187835387497\\
0.09	0.57324050169555	0.904365153357279	0.629991194998375\\
0.09	0.587818113093873	0.911101581661729	0.625738868900924\\
0.09	0.602459018746568	0.917584420216505	0.621431431633141\\
0.09	0.617156855035314	0.923813361211857	0.617070132480551\\
0.09	0.631905290438127	0.92978830106243	0.612656657833839\\
0.09	0.6466980333399	0.935509337395467	0.608192963377089\\
0.09	0.661528839617671	0.940976765701173	0.603681161495105\\
0.09	0.676391519984904	0.946191075661951	0.599123449738607\\
0.09	0.691279947080357	0.951152947177946	0.594522067901591\\
0.09	0.706188062288412	0.955863246106974	0.589879273662389\\
0.09	0.721109882279076	0.960323019737426	0.585197329319934\\
0.09	0.736039505257226	0.964533492013186	0.580478494543978\\
0.09	0.75097111691199	0.968496058529893	0.575725022018314\\
0.09	0.765898996058536	0.972212281322139	0.570939154289556\\
0.09	0.780817519965826	0.975683883461272	0.566123121053176\\
0.09	0.795721169365275	0.97891274348356	0.561279136605559\\
0.09	0.810604533136476	0.981900889668376	0.556409397399199\\
0.09	0.825462312667457	0.984650494185936	0.551516079695737\\
0.09	0.840289325888138	0.987163867133889	0.546601337319641\\
0.09	0.855080510976838	0.989443450481768	0.541667299515139\\
0.09	0.86983092974082	0.991491811941914	0.536716068908594\\
0.09	0.884535770672926	0.993311638785082	0.531749719578141\\
0.09	0.899190351687411	0.994905731618374	0.526770295231797\\
0.09	0.913790122539012	0.996276998142666	0.521779807495104\\
0.09	0.928330666930242	0.997428446906011	0.516780234308683\\
0.09	0.942807704312699	0.998363181068902	0.511773518435872\\
0.09	0.95721709138901	0.999084392196567	0.506761566080168\\
0.09	0.971554823322704	0.999595354092743	0.501746245611875\\
0.09	0.985817034663989	0.999899416688637	0.496729386403003\\
0.09	1	1	0.491669087346906\\
0.105	0	0	0.710195569525497\\
0.105	0.000100583311362513	0.0141829653360114	0.71240460418299\\
0.105	0.000404645907256436	0.0284451766772965	0.714630690536905\\
0.105	0.000915607803433	0.0427829086109896	0.716916312064106\\
0.105	0.00163681893109844	0.057192295687301	0.719289019360234\\
0.105	0.00257155309398959	0.0716693330697584	0.721780367061016\\
0.105	0.00372300185733414	0.086209877460988	0.724425655492835\\
0.105	0.00509426838162598	0.100809648312589	0.727263413850304\\
0.105	0.00668836121491816	0.115464229327074	0.73033458997498\\
0.105	0.00850818805808555	0.13016907025918	0.733681422514709\\
0.105	0.0105565495182326	0.144919489023162	0.737345986077549\\
0.105	0.0128361328661109	0.159710674111862	0.741368418705642\\
0.105	0.0153495058140643	0.174537687332543	0.745784862958201\\
0.105	0.0180991103316243	0.189395466863524	0.750625176092311\\
0.105	0.0210872565164405	0.204278830634725	0.755910489868999\\
0.105	0.0243161165387281	0.219182480034174	0.761650724664453\\
0.105	0.0277877186778607	0.234101003941464	0.767842183867616\\
0.105	0.0315039414701067	0.24902888308801	0.77446537092041\\
0.105	0.0354665079868145	0.263960494742775	0.781483180780606\\
0.105	0.0396769802625738	0.278890117720924	0.788839618268073\\
0.105	0.0441367538930258	0.293811937711588	0.796459186325544\\
0.105	0.0488470528220538	0.308720052919643	0.804247066923455\\
0.105	0.0538089243380495	0.323608480015096	0.812090186161949\\
0.105	0.0590232342988274	0.338471160382329	0.819859213932694\\
0.105	0.064490662604533	0.3533019666601	0.82741149906364\\
0.105	0.0702116989375697	0.368094709561873	0.834594885813505\\
0.105	0.0761866387881433	0.382843144964686	0.841252300287179\\
0.105	0.0824155797834956	0.397540981253432	0.847226939729081\\
0.105	0.0888984183382709	0.412181886906127	0.852367847905043\\
0.105	0.0956348466427213	0.42675949830445	0.856535620027829\\
0.105	0.102624350004627	0.441267427752584	0.859607954619617\\
0.105	0.109866204559871	0.455699271686219	0.861484760266623\\
0.105	0.117359475365564	0.470048619052377	0.862092534251541\\
0.105	0.125103014888515	0.484309059839721	0.861387758068368\\
0.105	0.133095461900593	0.498474193737904	0.859359100890819\\
0.105	0.14133524079124	0.512537638903683	0.856028283765738\\
0.105	0.149820561306021	0.526493040810599	0.851449530870182\\
0.105	0.158549418718625	0.540334081158348	0.845707614739502\\
0.105	0.167519594442214	0.554054486817265	0.838914584343152\\
0.105	0.176728657084455	0.567648038782867	0.83120534237084\\
0.105	0.186173963948925	0.581108581114919	0.822732305417529\\
0.105	0.195852662983903	0.594430029835237	0.813659432944307\\
0.105	0.205761695177907	0.607606381758231	0.80415594410487\\
0.105	0.215897797399558	0.620631723228144	0.794390053406967\\
0.105	0.226257505677663	0.633500238737009	0.784523046062803\\
0.105	0.236837158915675	0.646206219397571	0.774703982861589\\
0.105	0.247632903032949	0.658744071245709	0.765065275211012\\
0.105	0.258640695523528	0.671108323347402	0.755719307772961\\
0.105	0.269856310421543	0.683293635685828	0.746756214005916\\
0.105	0.28127534366066	0.695294806804925	0.738242834604294\\
0.105	0.292893218813452	0.707106781186547	0.730222816007835\\
0.105	0.304705193195075	0.71872465633934	0.722717741087769\\
0.105	0.316706364314172	0.730143689578457	0.715729131143265\\
0.105	0.328891676652598	0.741359304476472	0.709241120561856\\
0.105	0.341255928754291	0.752367096967051	0.703223584571266\\
0.105	0.353793780602429	0.763162841084325	0.697635496597418\\
0.105	0.366499761262991	0.773742494322337	0.692428303594618\\
0.105	0.379368276771857	0.784102202600443	0.68754913288707\\
0.105	0.392393618241769	0.794238304822092	0.682943679235328\\
0.105	0.405569970164763	0.804147337016097	0.678558662193046\\
0.105	0.418891418885081	0.813826036051075	0.674343787412873\\
0.105	0.432351961217133	0.823271342915545	0.670253187715551\\
0.105	0.445945513182735	0.832480405557787	0.666246357336158\\
0.105	0.459665918841652	0.841450581281375	0.662288623474186\\
0.105	0.473506959189401	0.850179438693979	0.658351221673192\\
0.105	0.487462361096317	0.85866475920876	0.654411055137885\\
0.105	0.501525806262096	0.866904538099407	0.650450223206467\\
0.105	0.515690940160279	0.874896985111485	0.64645540187496\\
0.105	0.529951380947623	0.882640524634437	0.642417151055156\\
0.105	0.544300728313782	0.890133795440129	0.638329210947586\\
0.105	0.558732572247415	0.897375649995373	0.634187835387496\\
0.105	0.57324050169555	0.904365153357279	0.629991194998378\\
0.105	0.587818113093873	0.911101581661729	0.625738868900923\\
0.105	0.602459018746568	0.917584420216504	0.621431431633139\\
0.105	0.617156855035314	0.923813361211857	0.617070132480552\\
0.105	0.631905290438127	0.92978830106243	0.612656657833837\\
0.105	0.6466980333399	0.935509337395467	0.608192963377089\\
0.105	0.661528839617671	0.940976765701173	0.603681161495106\\
0.105	0.676391519984904	0.946191075661951	0.599123449738607\\
0.105	0.691279947080357	0.951152947177946	0.59452206790159\\
0.105	0.706188062288412	0.955863246106974	0.58987927366239\\
0.105	0.721109882279076	0.960323019737426	0.585197329319935\\
0.105	0.736039505257226	0.964533492013186	0.580478494543976\\
0.105	0.75097111691199	0.968496058529893	0.575725022018315\\
0.105	0.765898996058536	0.972212281322139	0.570939154289557\\
0.105	0.780817519965826	0.975683883461272	0.566123121053173\\
0.105	0.795721169365275	0.97891274348356	0.561279136605556\\
0.105	0.810604533136476	0.981900889668376	0.556409397399199\\
0.105	0.825462312667457	0.984650494185936	0.551516079695737\\
0.105	0.840289325888138	0.987163867133889	0.546601337319641\\
0.105	0.855080510976838	0.989443450481767	0.541667299515141\\
0.105	0.86983092974082	0.991491811941914	0.536716068908597\\
0.105	0.884535770672926	0.993311638785082	0.531749719578141\\
0.105	0.899190351687411	0.994905731618374	0.526770295231796\\
0.105	0.913790122539012	0.996276998142666	0.521779807495102\\
0.105	0.928330666930242	0.997428446906011	0.516780234308686\\
0.105	0.942807704312699	0.998363181068902	0.511773518435872\\
0.105	0.95721709138901	0.999084392196567	0.506761566080164\\
0.105	0.971554823322703	0.999595354092743	0.501746245611876\\
0.105	0.985817034663989	0.999899416688638	0.496729386403005\\
0.105	1	1	0.491669087346895\\
0.12	0	0	0.710195569525497\\
0.12	0.000100583311362513	0.0141829653360114	0.71240460418299\\
0.12	0.000404645907256436	0.0284451766772965	0.714630690536905\\
0.12	0.000915607803433	0.0427829086109896	0.716916312064106\\
0.12	0.00163681893109844	0.057192295687301	0.719289019360234\\
0.12	0.00257155309398959	0.0716693330697585	0.721780367061016\\
0.12	0.00372300185733414	0.086209877460988	0.724425655492835\\
0.12	0.00509426838162598	0.100809648312589	0.727263413850304\\
0.12	0.00668836121491816	0.115464229327074	0.73033458997498\\
0.12	0.00850818805808555	0.13016907025918	0.733681422514709\\
0.12	0.0105565495182326	0.144919489023162	0.737345986077549\\
0.12	0.0128361328661109	0.159710674111862	0.741368418705642\\
0.12	0.0153495058140643	0.174537687332543	0.745784862958202\\
0.12	0.0180991103316243	0.189395466863524	0.750625176092311\\
0.12	0.0210872565164405	0.204278830634725	0.755910489868999\\
0.12	0.0243161165387281	0.219182480034174	0.761650724664453\\
0.12	0.0277877186778607	0.234101003941464	0.767842183867616\\
0.12	0.0315039414701067	0.24902888308801	0.77446537092041\\
0.12	0.0354665079868145	0.263960494742775	0.781483180780606\\
0.12	0.0396769802625738	0.278890117720924	0.788839618268073\\
0.12	0.0441367538930258	0.293811937711588	0.796459186325544\\
0.12	0.0488470528220537	0.308720052919643	0.804247066923454\\
0.12	0.0538089243380495	0.323608480015096	0.812090186161951\\
0.12	0.0590232342988274	0.338471160382329	0.819859213932693\\
0.12	0.064490662604533	0.3533019666601	0.827411499063639\\
0.12	0.0702116989375697	0.368094709561873	0.834594885813504\\
0.12	0.0761866387881432	0.382843144964686	0.841252300287182\\
0.12	0.0824155797834956	0.397540981253432	0.847226939729079\\
0.12	0.0888984183382709	0.412181886906127	0.85236784790504\\
0.12	0.0956348466427212	0.42675949830445	0.856535620027829\\
0.12	0.102624350004627	0.441267427752584	0.859607954619617\\
0.12	0.109866204559871	0.455699271686219	0.861484760266618\\
0.12	0.117359475365564	0.470048619052377	0.862092534251541\\
0.12	0.125103014888515	0.484309059839721	0.861387758068368\\
0.12	0.133095461900593	0.498474193737904	0.85935910089082\\
0.12	0.14133524079124	0.512537638903683	0.85602828376574\\
0.12	0.149820561306021	0.526493040810599	0.851449530870183\\
0.12	0.158549418718625	0.540334081158348	0.845707614739509\\
0.12	0.167519594442213	0.554054486817265	0.838914584343155\\
0.12	0.176728657084455	0.567648038782868	0.831205342370836\\
0.12	0.186173963948925	0.581108581114919	0.822732305417526\\
0.12	0.195852662983903	0.594430029835236	0.813659432944308\\
0.12	0.205761695177907	0.607606381758231	0.804155944104869\\
0.12	0.215897797399558	0.620631723228143	0.794390053406968\\
0.12	0.226257505677663	0.633500238737009	0.784523046062801\\
0.12	0.236837158915675	0.646206219397571	0.774703982861586\\
0.12	0.247632903032949	0.658744071245709	0.765065275211014\\
0.12	0.258640695523528	0.671108323347402	0.755719307772964\\
0.12	0.269856310421543	0.683293635685828	0.746756214005915\\
0.12	0.28127534366066	0.695294806804925	0.738242834604294\\
0.12	0.292893218813452	0.707106781186547	0.730222816007837\\
0.12	0.304705193195075	0.71872465633934	0.722717741087767\\
0.12	0.316706364314172	0.730143689578457	0.715729131143265\\
0.12	0.328891676652598	0.741359304476472	0.709241120561856\\
0.12	0.341255928754291	0.752367096967051	0.703223584571265\\
0.12	0.353793780602429	0.763162841084325	0.697635496597418\\
0.12	0.366499761262991	0.773742494322337	0.692428303594619\\
0.12	0.379368276771857	0.784102202600443	0.68754913288707\\
0.12	0.392393618241769	0.794238304822092	0.682943679235329\\
0.12	0.405569970164763	0.804147337016097	0.678558662193046\\
0.12	0.418891418885081	0.813826036051075	0.674343787412871\\
0.12	0.432351961217133	0.823271342915545	0.670253187715553\\
0.12	0.445945513182735	0.832480405557787	0.666246357336158\\
0.12	0.459665918841652	0.841450581281375	0.662288623474183\\
0.12	0.473506959189401	0.850179438693979	0.658351221673194\\
0.12	0.487462361096317	0.85866475920876	0.654411055137887\\
0.12	0.501525806262096	0.866904538099407	0.650450223206466\\
0.12	0.515690940160279	0.874896985111485	0.646455401874959\\
0.12	0.529951380947623	0.882640524634437	0.642417151055155\\
0.12	0.544300728313782	0.890133795440129	0.638329210947586\\
0.12	0.558732572247415	0.897375649995373	0.634187835387498\\
0.12	0.57324050169555	0.904365153357279	0.629991194998378\\
0.12	0.587818113093873	0.911101581661729	0.625738868900926\\
0.12	0.602459018746568	0.917584420216505	0.621431431633138\\
0.12	0.617156855035314	0.923813361211857	0.617070132480553\\
0.12	0.631905290438127	0.92978830106243	0.612656657833837\\
0.12	0.6466980333399	0.935509337395467	0.608192963377087\\
0.12	0.661528839617671	0.940976765701173	0.603681161495105\\
0.12	0.676391519984904	0.946191075661951	0.599123449738608\\
0.12	0.691279947080357	0.951152947177946	0.594522067901591\\
0.12	0.706188062288412	0.955863246106974	0.589879273662388\\
0.12	0.721109882279076	0.960323019737426	0.585197329319935\\
0.12	0.736039505257225	0.964533492013185	0.580478494543975\\
0.12	0.75097111691199	0.968496058529893	0.575725022018316\\
0.12	0.765898996058536	0.972212281322139	0.57093915428956\\
0.12	0.780817519965825	0.975683883461272	0.566123121053175\\
0.12	0.795721169365275	0.97891274348356	0.561279136605557\\
0.12	0.810604533136476	0.981900889668376	0.556409397399196\\
0.12	0.825462312667457	0.984650494185936	0.551516079695738\\
0.12	0.840289325888138	0.987163867133889	0.546601337319642\\
0.12	0.855080510976839	0.989443450481768	0.541667299515139\\
0.12	0.86983092974082	0.991491811941914	0.536716068908598\\
0.12	0.884535770672926	0.993311638785082	0.531749719578141\\
0.12	0.899190351687411	0.994905731618374	0.526770295231796\\
0.12	0.913790122539012	0.996276998142666	0.521779807495103\\
0.12	0.928330666930242	0.99742844690601	0.516780234308682\\
0.12	0.942807704312699	0.998363181068902	0.511773518435875\\
0.12	0.95721709138901	0.999084392196567	0.50676156608017\\
0.12	0.971554823322704	0.999595354092744	0.501746245611874\\
0.12	0.985817034663989	0.999899416688637	0.496729386403002\\
0.12	1	1	0.491669087346908\\
0.135	0	0	0.710195569525497\\
0.135	0.000100583311362513	0.0141829653360114	0.71240460418299\\
0.135	0.000404645907256436	0.0284451766772965	0.714630690536905\\
0.135	0.000915607803433	0.0427829086109896	0.716916312064106\\
0.135	0.00163681893109844	0.057192295687301	0.719289019360234\\
0.135	0.00257155309398959	0.0716693330697585	0.721780367061016\\
0.135	0.00372300185733414	0.086209877460988	0.724425655492835\\
0.135	0.00509426838162599	0.100809648312589	0.727263413850304\\
0.135	0.00668836121491816	0.115464229327074	0.73033458997498\\
0.135	0.00850818805808554	0.13016907025918	0.733681422514709\\
0.135	0.0105565495182326	0.144919489023162	0.737345986077549\\
0.135	0.0128361328661109	0.159710674111862	0.741368418705642\\
0.135	0.0153495058140643	0.174537687332543	0.745784862958201\\
0.135	0.0180991103316243	0.189395466863524	0.750625176092311\\
0.135	0.0210872565164405	0.204278830634725	0.755910489868998\\
0.135	0.0243161165387281	0.219182480034174	0.761650724664453\\
0.135	0.0277877186778607	0.234101003941464	0.767842183867616\\
0.135	0.0315039414701067	0.24902888308801	0.77446537092041\\
0.135	0.0354665079868145	0.263960494742775	0.781483180780606\\
0.135	0.0396769802625738	0.278890117720924	0.788839618268073\\
0.135	0.0441367538930258	0.293811937711588	0.796459186325544\\
0.135	0.0488470528220537	0.308720052919643	0.804247066923454\\
0.135	0.0538089243380495	0.323608480015097	0.812090186161949\\
0.135	0.0590232342988274	0.338471160382329	0.819859213932694\\
0.135	0.064490662604533	0.3533019666601	0.82741149906364\\
0.135	0.0702116989375697	0.368094709561873	0.834594885813504\\
0.135	0.0761866387881432	0.382843144964686	0.841252300287179\\
0.135	0.0824155797834956	0.397540981253432	0.847226939729081\\
0.135	0.0888984183382709	0.412181886906127	0.852367847905039\\
0.135	0.0956348466427212	0.42675949830445	0.856535620027829\\
0.135	0.102624350004627	0.441267427752584	0.859607954619617\\
0.135	0.109866204559871	0.455699271686218	0.861484760266621\\
0.135	0.117359475365564	0.470048619052377	0.86209253425154\\
0.135	0.125103014888515	0.484309059839721	0.861387758068367\\
0.135	0.133095461900593	0.498474193737904	0.859359100890821\\
0.135	0.14133524079124	0.512537638903683	0.856028283765737\\
0.135	0.149820561306021	0.526493040810599	0.851449530870184\\
0.135	0.158549418718625	0.540334081158348	0.845707614739509\\
0.135	0.167519594442214	0.554054486817265	0.838914584343153\\
0.135	0.176728657084455	0.567648038782867	0.831205342370836\\
0.135	0.186173963948925	0.581108581114919	0.822732305417526\\
0.135	0.195852662983903	0.594430029835237	0.813659432944311\\
0.135	0.205761695177907	0.607606381758231	0.804155944104868\\
0.135	0.215897797399558	0.620631723228143	0.794390053406969\\
0.135	0.226257505677663	0.633500238737009	0.7845230460628\\
0.135	0.236837158915675	0.646206219397571	0.774703982861588\\
0.135	0.247632903032949	0.658744071245709	0.765065275211015\\
0.135	0.258640695523528	0.671108323347402	0.755719307772964\\
0.135	0.269856310421543	0.683293635685828	0.746756214005915\\
0.135	0.28127534366066	0.695294806804925	0.738242834604294\\
0.135	0.292893218813452	0.707106781186547	0.730222816007836\\
0.135	0.304705193195075	0.71872465633934	0.722717741087767\\
0.135	0.316706364314172	0.730143689578457	0.715729131143267\\
0.135	0.328891676652598	0.741359304476473	0.709241120561857\\
0.135	0.341255928754291	0.752367096967051	0.703223584571264\\
0.135	0.353793780602429	0.763162841084325	0.697635496597417\\
0.135	0.366499761262991	0.773742494322337	0.692428303594618\\
0.135	0.379368276771857	0.784102202600442	0.687549132887069\\
0.135	0.392393618241769	0.794238304822092	0.682943679235329\\
0.135	0.405569970164763	0.804147337016097	0.678558662193046\\
0.135	0.418891418885081	0.813826036051075	0.674343787412872\\
0.135	0.432351961217133	0.823271342915545	0.670253187715552\\
0.135	0.445945513182735	0.832480405557787	0.666246357336158\\
0.135	0.459665918841652	0.841450581281375	0.662288623474184\\
0.135	0.473506959189401	0.850179438693979	0.658351221673191\\
0.135	0.487462361096317	0.85866475920876	0.654411055137886\\
0.135	0.501525806262096	0.866904538099407	0.650450223206467\\
0.135	0.515690940160279	0.874896985111485	0.646455401874961\\
0.135	0.529951380947623	0.882640524634437	0.642417151055155\\
0.135	0.544300728313782	0.890133795440129	0.638329210947584\\
0.135	0.558732572247415	0.897375649995373	0.634187835387498\\
0.135	0.57324050169555	0.904365153357279	0.629991194998376\\
0.135	0.587818113093873	0.911101581661729	0.625738868900925\\
0.135	0.602459018746568	0.917584420216504	0.62143143163314\\
0.135	0.617156855035314	0.923813361211857	0.617070132480554\\
0.135	0.631905290438127	0.92978830106243	0.612656657833838\\
0.135	0.6466980333399	0.935509337395467	0.608192963377087\\
0.135	0.661528839617671	0.940976765701173	0.603681161495105\\
0.135	0.676391519984904	0.946191075661951	0.599123449738607\\
0.135	0.691279947080357	0.951152947177946	0.594522067901591\\
0.135	0.706188062288412	0.955863246106974	0.589879273662389\\
0.135	0.721109882279076	0.960323019737426	0.585197329319934\\
0.135	0.736039505257225	0.964533492013186	0.580478494543977\\
0.135	0.75097111691199	0.968496058529893	0.575725022018311\\
0.135	0.765898996058536	0.972212281322139	0.570939154289558\\
0.135	0.780817519965826	0.975683883461272	0.566123121053176\\
0.135	0.795721169365275	0.97891274348356	0.561279136605559\\
0.135	0.810604533136476	0.981900889668376	0.556409397399197\\
0.135	0.825462312667457	0.984650494185936	0.551516079695736\\
0.135	0.840289325888138	0.987163867133889	0.546601337319641\\
0.135	0.855080510976838	0.989443450481767	0.54166729951514\\
0.135	0.86983092974082	0.991491811941914	0.536716068908597\\
0.135	0.884535770672926	0.993311638785082	0.531749719578143\\
0.135	0.899190351687411	0.994905731618374	0.526770295231796\\
0.135	0.913790122539012	0.996276998142666	0.521779807495104\\
0.135	0.928330666930242	0.997428446906011	0.516780234308682\\
0.135	0.942807704312699	0.998363181068902	0.511773518435872\\
0.135	0.95721709138901	0.999084392196567	0.50676156608017\\
0.135	0.971554823322703	0.999595354092743	0.501746245611875\\
0.135	0.985817034663989	0.999899416688637	0.496729386403003\\
0.135	1	1	0.491669087346906\\
0.15	0	0	0.710195569525497\\
0.15	0.000100583311362513	0.0141829653360114	0.71240460418299\\
0.15	0.000404645907256436	0.0284451766772965	0.714630690536905\\
0.15	0.000915607803433	0.0427829086109896	0.716916312064106\\
0.15	0.00163681893109844	0.057192295687301	0.719289019360234\\
0.15	0.00257155309398959	0.0716693330697584	0.721780367061016\\
0.15	0.00372300185733414	0.0862098774609879	0.724425655492835\\
0.15	0.00509426838162598	0.100809648312589	0.727263413850304\\
0.15	0.00668836121491816	0.115464229327074	0.73033458997498\\
0.15	0.00850818805808555	0.13016907025918	0.733681422514709\\
0.15	0.0105565495182326	0.144919489023162	0.737345986077549\\
0.15	0.0128361328661109	0.159710674111862	0.741368418705642\\
0.15	0.0153495058140643	0.174537687332543	0.745784862958202\\
0.15	0.0180991103316243	0.189395466863524	0.750625176092311\\
0.15	0.0210872565164405	0.204278830634725	0.755910489868998\\
0.15	0.0243161165387281	0.219182480034174	0.761650724664453\\
0.15	0.0277877186778607	0.234101003941464	0.767842183867616\\
0.15	0.0315039414701067	0.24902888308801	0.77446537092041\\
0.15	0.0354665079868145	0.263960494742775	0.781483180780606\\
0.15	0.0396769802625738	0.278890117720924	0.788839618268073\\
0.15	0.0441367538930258	0.293811937711588	0.796459186325544\\
0.15	0.0488470528220538	0.308720052919643	0.804247066923455\\
0.15	0.0538089243380495	0.323608480015096	0.81209018616195\\
0.15	0.0590232342988274	0.338471160382329	0.819859213932693\\
0.15	0.064490662604533	0.3533019666601	0.827411499063639\\
0.15	0.0702116989375697	0.368094709561873	0.834594885813505\\
0.15	0.0761866387881432	0.382843144964686	0.841252300287178\\
0.15	0.0824155797834956	0.397540981253432	0.847226939729081\\
0.15	0.0888984183382709	0.412181886906127	0.852367847905042\\
0.15	0.0956348466427212	0.42675949830445	0.85653562002783\\
0.15	0.102624350004627	0.441267427752585	0.859607954619618\\
0.15	0.109866204559871	0.455699271686219	0.861484760266625\\
0.15	0.117359475365564	0.470048619052377	0.862092534251542\\
0.15	0.125103014888515	0.484309059839721	0.861387758068369\\
0.15	0.133095461900593	0.498474193737904	0.859359100890819\\
0.15	0.14133524079124	0.512537638903683	0.856028283765737\\
0.15	0.149820561306021	0.526493040810599	0.851449530870181\\
0.15	0.158549418718625	0.540334081158348	0.845707614739506\\
0.15	0.167519594442214	0.554054486817265	0.838914584343148\\
0.15	0.176728657084455	0.567648038782868	0.831205342370835\\
0.15	0.186173963948925	0.581108581114919	0.822732305417528\\
0.15	0.195852662983903	0.594430029835237	0.813659432944309\\
0.15	0.205761695177907	0.607606381758231	0.804155944104868\\
0.15	0.215897797399558	0.620631723228143	0.794390053406969\\
0.15	0.226257505677663	0.633500238737009	0.7845230460628\\
0.15	0.236837158915675	0.646206219397571	0.774703982861588\\
0.15	0.247632903032949	0.658744071245709	0.765065275211016\\
0.15	0.258640695523528	0.671108323347402	0.755719307772966\\
0.15	0.269856310421543	0.683293635685828	0.746756214005913\\
0.15	0.28127534366066	0.695294806804924	0.738242834604294\\
0.15	0.292893218813452	0.707106781186547	0.730222816007835\\
0.15	0.304705193195075	0.71872465633934	0.722717741087767\\
0.15	0.316706364314172	0.730143689578457	0.715729131143267\\
0.15	0.328891676652598	0.741359304476472	0.709241120561857\\
0.15	0.341255928754291	0.752367096967051	0.703223584571265\\
0.15	0.353793780602429	0.763162841084325	0.697635496597419\\
0.15	0.366499761262991	0.773742494322338	0.692428303594618\\
0.15	0.379368276771857	0.784102202600442	0.687549132887069\\
0.15	0.392393618241769	0.794238304822092	0.682943679235329\\
0.15	0.405569970164763	0.804147337016097	0.678558662193044\\
0.15	0.418891418885081	0.813826036051075	0.674343787412871\\
0.15	0.432351961217133	0.823271342915545	0.670253187715552\\
0.15	0.445945513182735	0.832480405557787	0.666246357336159\\
0.15	0.459665918841652	0.841450581281375	0.662288623474186\\
0.15	0.473506959189401	0.850179438693979	0.658351221673191\\
0.15	0.487462361096317	0.85866475920876	0.654411055137885\\
0.15	0.501525806262096	0.866904538099407	0.650450223206466\\
0.15	0.515690940160279	0.874896985111485	0.646455401874959\\
0.15	0.529951380947623	0.882640524634437	0.642417151055155\\
0.15	0.544300728313782	0.890133795440129	0.638329210947585\\
0.15	0.558732572247415	0.897375649995373	0.634187835387498\\
0.15	0.57324050169555	0.904365153357279	0.629991194998375\\
0.15	0.587818113093873	0.911101581661729	0.625738868900924\\
0.15	0.602459018746568	0.917584420216504	0.62143143163314\\
0.15	0.617156855035314	0.923813361211857	0.617070132480554\\
0.15	0.631905290438127	0.92978830106243	0.612656657833839\\
0.15	0.6466980333399	0.935509337395467	0.608192963377087\\
0.15	0.661528839617671	0.940976765701173	0.603681161495106\\
0.15	0.676391519984904	0.946191075661951	0.599123449738608\\
0.15	0.691279947080357	0.951152947177946	0.594522067901589\\
0.15	0.706188062288412	0.955863246106974	0.589879273662388\\
0.15	0.721109882279076	0.960323019737426	0.585197329319936\\
0.15	0.736039505257225	0.964533492013186	0.580478494543978\\
0.15	0.75097111691199	0.968496058529893	0.575725022018313\\
0.15	0.765898996058536	0.972212281322139	0.570939154289557\\
0.15	0.780817519965826	0.975683883461272	0.566123121053174\\
0.15	0.795721169365275	0.97891274348356	0.561279136605558\\
0.15	0.810604533136476	0.981900889668376	0.556409397399198\\
0.15	0.825462312667457	0.984650494185936	0.551516079695738\\
0.15	0.840289325888138	0.987163867133889	0.546601337319641\\
0.15	0.855080510976839	0.989443450481768	0.541667299515137\\
0.15	0.86983092974082	0.991491811941914	0.536716068908597\\
0.15	0.884535770672926	0.993311638785082	0.53174971957814\\
0.15	0.899190351687411	0.994905731618374	0.526770295231795\\
0.15	0.913790122539012	0.996276998142666	0.521779807495103\\
0.15	0.928330666930242	0.99742844690601	0.516780234308685\\
0.15	0.942807704312699	0.998363181068902	0.511773518435871\\
0.15	0.95721709138901	0.999084392196567	0.506761566080166\\
0.15	0.971554823322703	0.999595354092743	0.501746245611878\\
0.15	0.985817034663989	0.999899416688638	0.496729386403005\\
0.15	1	1	0.491669087346895\\
0.165	0	0	0.710195569525497\\
0.165	0.000100583311362513	0.0141829653360114	0.71240460418299\\
0.165	0.000404645907256436	0.0284451766772965	0.714630690536905\\
0.165	0.000915607803433	0.0427829086109896	0.716916312064106\\
0.165	0.00163681893109844	0.057192295687301	0.719289019360234\\
0.165	0.00257155309398959	0.0716693330697585	0.721780367061016\\
0.165	0.00372300185733414	0.086209877460988	0.724425655492835\\
0.165	0.00509426838162598	0.100809648312589	0.727263413850304\\
0.165	0.00668836121491816	0.115464229327074	0.73033458997498\\
0.165	0.00850818805808555	0.13016907025918	0.733681422514709\\
0.165	0.0105565495182326	0.144919489023162	0.737345986077549\\
0.165	0.0128361328661109	0.159710674111862	0.741368418705642\\
0.165	0.0153495058140643	0.174537687332543	0.745784862958202\\
0.165	0.0180991103316243	0.189395466863524	0.750625176092311\\
0.165	0.0210872565164405	0.204278830634725	0.755910489868998\\
0.165	0.0243161165387281	0.219182480034174	0.761650724664453\\
0.165	0.0277877186778607	0.234101003941464	0.767842183867616\\
0.165	0.0315039414701067	0.24902888308801	0.77446537092041\\
0.165	0.0354665079868145	0.263960494742775	0.781483180780606\\
0.165	0.0396769802625738	0.278890117720924	0.788839618268073\\
0.165	0.0441367538930258	0.293811937711588	0.796459186325544\\
0.165	0.0488470528220538	0.308720052919643	0.804247066923454\\
0.165	0.0538089243380495	0.323608480015096	0.812090186161949\\
0.165	0.0590232342988274	0.338471160382329	0.819859213932694\\
0.165	0.064490662604533	0.3533019666601	0.827411499063638\\
0.165	0.0702116989375697	0.368094709561873	0.834594885813505\\
0.165	0.0761866387881432	0.382843144964686	0.841252300287181\\
0.165	0.0824155797834956	0.397540981253432	0.84722693972908\\
0.165	0.0888984183382709	0.412181886906127	0.852367847905043\\
0.165	0.0956348466427212	0.42675949830445	0.856535620027829\\
0.165	0.102624350004627	0.441267427752584	0.859607954619617\\
0.165	0.109866204559871	0.455699271686219	0.861484760266622\\
0.165	0.117359475365564	0.470048619052377	0.86209253425154\\
0.165	0.125103014888515	0.484309059839721	0.861387758068368\\
0.165	0.133095461900593	0.498474193737904	0.859359100890821\\
0.165	0.14133524079124	0.512537638903683	0.856028283765731\\
0.165	0.149820561306021	0.526493040810599	0.85144953087018\\
0.165	0.158549418718625	0.540334081158348	0.845707614739505\\
0.165	0.167519594442214	0.554054486817265	0.838914584343147\\
0.165	0.176728657084455	0.567648038782867	0.831205342370834\\
0.165	0.186173963948925	0.581108581114919	0.822732305417529\\
0.165	0.195852662983903	0.594430029835237	0.813659432944309\\
0.165	0.205761695177907	0.607606381758231	0.804155944104868\\
0.165	0.215897797399558	0.620631723228143	0.79439005340697\\
0.165	0.226257505677663	0.633500238737009	0.7845230460628\\
0.165	0.236837158915675	0.646206219397571	0.774703982861589\\
0.165	0.247632903032949	0.658744071245709	0.765065275211017\\
0.165	0.258640695523528	0.671108323347402	0.755719307772964\\
0.165	0.269856310421543	0.683293635685828	0.746756214005914\\
0.165	0.28127534366066	0.695294806804925	0.738242834604295\\
0.165	0.292893218813452	0.707106781186547	0.730222816007836\\
0.165	0.304705193195075	0.71872465633934	0.722717741087767\\
0.165	0.316706364314172	0.730143689578457	0.715729131143265\\
0.165	0.328891676652598	0.741359304476472	0.709241120561859\\
0.165	0.341255928754291	0.752367096967051	0.703223584571265\\
0.165	0.353793780602429	0.763162841084325	0.697635496597416\\
0.165	0.366499761262991	0.773742494322337	0.692428303594618\\
0.165	0.379368276771857	0.784102202600442	0.68754913288707\\
0.165	0.392393618241769	0.794238304822092	0.682943679235329\\
0.165	0.405569970164763	0.804147337016097	0.678558662193046\\
0.165	0.418891418885081	0.813826036051075	0.674343787412872\\
0.165	0.432351961217133	0.823271342915545	0.670253187715551\\
0.165	0.445945513182735	0.832480405557787	0.666246357336158\\
0.165	0.459665918841652	0.841450581281375	0.662288623474185\\
0.165	0.473506959189401	0.850179438693979	0.658351221673193\\
0.165	0.487462361096317	0.85866475920876	0.654411055137887\\
0.165	0.501525806262096	0.866904538099407	0.650450223206466\\
0.165	0.515690940160279	0.874896985111485	0.646455401874959\\
0.165	0.529951380947623	0.882640524634437	0.642417151055155\\
0.165	0.544300728313782	0.890133795440129	0.638329210947586\\
0.165	0.558732572247415	0.897375649995373	0.634187835387499\\
0.165	0.57324050169555	0.904365153357279	0.629991194998375\\
0.165	0.587818113093873	0.911101581661729	0.625738868900923\\
0.165	0.602459018746568	0.917584420216504	0.62143143163314\\
0.165	0.617156855035314	0.923813361211857	0.617070132480553\\
0.165	0.631905290438127	0.92978830106243	0.612656657833838\\
0.165	0.6466980333399	0.935509337395467	0.608192963377086\\
0.165	0.661528839617671	0.940976765701173	0.603681161495105\\
0.165	0.676391519984904	0.946191075661951	0.599123449738609\\
0.165	0.691279947080357	0.951152947177946	0.594522067901591\\
0.165	0.706188062288412	0.955863246106974	0.589879273662387\\
0.165	0.721109882279076	0.960323019737426	0.585197329319936\\
0.165	0.736039505257225	0.964533492013186	0.580478494543978\\
0.165	0.75097111691199	0.968496058529893	0.575725022018313\\
0.165	0.765898996058536	0.972212281322139	0.570939154289557\\
0.165	0.780817519965826	0.975683883461272	0.566123121053175\\
0.165	0.795721169365275	0.97891274348356	0.561279136605559\\
0.165	0.810604533136476	0.981900889668376	0.556409397399198\\
0.165	0.825462312667457	0.984650494185936	0.551516079695734\\
0.165	0.840289325888138	0.987163867133889	0.546601337319641\\
0.165	0.855080510976839	0.989443450481768	0.541667299515139\\
0.165	0.86983092974082	0.991491811941914	0.536716068908599\\
0.165	0.884535770672926	0.993311638785082	0.531749719578141\\
0.165	0.899190351687411	0.994905731618374	0.526770295231794\\
0.165	0.913790122539012	0.996276998142666	0.521779807495104\\
0.165	0.928330666930242	0.997428446906011	0.516780234308685\\
0.165	0.942807704312699	0.998363181068902	0.511773518435872\\
0.165	0.95721709138901	0.999084392196567	0.506761566080166\\
0.165	0.971554823322703	0.999595354092743	0.501746245611873\\
0.165	0.985817034663989	0.999899416688637	0.496729386403004\\
0.165	1	1	0.491669087346906\\
0.18	0	0	0.710195569525497\\
0.18	0.000100583311362513	0.0141829653360114	0.71240460418299\\
0.18	0.000404645907256436	0.0284451766772965	0.714630690536905\\
0.18	0.000915607803433	0.0427829086109896	0.716916312064106\\
0.18	0.00163681893109844	0.057192295687301	0.719289019360234\\
0.18	0.00257155309398959	0.0716693330697585	0.721780367061016\\
0.18	0.00372300185733414	0.086209877460988	0.724425655492835\\
0.18	0.00509426838162598	0.100809648312589	0.727263413850304\\
0.18	0.00668836121491816	0.115464229327074	0.73033458997498\\
0.18	0.00850818805808555	0.13016907025918	0.733681422514709\\
0.18	0.0105565495182326	0.144919489023162	0.737345986077549\\
0.18	0.0128361328661109	0.159710674111862	0.741368418705642\\
0.18	0.0153495058140643	0.174537687332543	0.745784862958202\\
0.18	0.0180991103316243	0.189395466863524	0.750625176092311\\
0.18	0.0210872565164405	0.204278830634725	0.755910489868998\\
0.18	0.0243161165387281	0.219182480034174	0.761650724664453\\
0.18	0.0277877186778607	0.234101003941464	0.767842183867616\\
0.18	0.0315039414701067	0.24902888308801	0.77446537092041\\
0.18	0.0354665079868145	0.263960494742775	0.781483180780606\\
0.18	0.0396769802625738	0.278890117720924	0.788839618268073\\
0.18	0.0441367538930258	0.293811937711588	0.796459186325544\\
0.18	0.0488470528220538	0.308720052919643	0.804247066923455\\
0.18	0.0538089243380495	0.323608480015096	0.81209018616195\\
0.18	0.0590232342988274	0.338471160382329	0.819859213932694\\
0.18	0.064490662604533	0.3533019666601	0.82741149906364\\
0.18	0.0702116989375697	0.368094709561873	0.834594885813505\\
0.18	0.0761866387881432	0.382843144964686	0.841252300287179\\
0.18	0.0824155797834956	0.397540981253432	0.847226939729081\\
0.18	0.0888984183382709	0.412181886906127	0.852367847905041\\
0.18	0.0956348466427213	0.42675949830445	0.85653562002783\\
0.18	0.102624350004627	0.441267427752584	0.859607954619617\\
0.18	0.109866204559871	0.455699271686219	0.861484760266621\\
0.18	0.117359475365564	0.470048619052377	0.862092534251539\\
0.18	0.125103014888515	0.484309059839721	0.861387758068369\\
0.18	0.133095461900593	0.498474193737904	0.85935910089082\\
0.18	0.14133524079124	0.512537638903683	0.856028283765733\\
0.18	0.149820561306021	0.526493040810599	0.851449530870183\\
0.18	0.158549418718625	0.540334081158348	0.845707614739502\\
0.18	0.167519594442214	0.554054486817265	0.83891458434315\\
0.18	0.176728657084455	0.567648038782867	0.831205342370836\\
0.18	0.186173963948925	0.581108581114919	0.822732305417527\\
0.18	0.195852662983903	0.594430029835237	0.813659432944309\\
0.18	0.205761695177907	0.607606381758231	0.804155944104867\\
0.18	0.215897797399558	0.620631723228143	0.79439005340697\\
0.18	0.226257505677663	0.633500238737009	0.7845230460628\\
0.18	0.236837158915675	0.646206219397571	0.774703982861589\\
0.18	0.247632903032949	0.658744071245709	0.765065275211016\\
0.18	0.258640695523528	0.671108323347402	0.755719307772964\\
0.18	0.269856310421543	0.683293635685828	0.746756214005914\\
0.18	0.28127534366066	0.695294806804925	0.738242834604294\\
0.18	0.292893218813452	0.707106781186547	0.730222816007833\\
0.18	0.304705193195075	0.71872465633934	0.722717741087767\\
0.18	0.316706364314172	0.730143689578457	0.715729131143268\\
0.18	0.328891676652598	0.741359304476472	0.709241120561857\\
0.18	0.341255928754291	0.752367096967051	0.703223584571265\\
0.18	0.353793780602429	0.763162841084325	0.697635496597417\\
0.18	0.366499761262991	0.773742494322337	0.692428303594618\\
0.18	0.379368276771857	0.784102202600442	0.687549132887069\\
0.18	0.392393618241769	0.794238304822092	0.682943679235329\\
0.18	0.405569970164763	0.804147337016097	0.678558662193046\\
0.18	0.418891418885081	0.813826036051075	0.674343787412872\\
0.18	0.432351961217133	0.823271342915545	0.670253187715552\\
0.18	0.445945513182735	0.832480405557787	0.666246357336158\\
0.18	0.459665918841652	0.841450581281375	0.662288623474184\\
0.18	0.473506959189401	0.850179438693979	0.658351221673191\\
0.18	0.487462361096317	0.85866475920876	0.654411055137887\\
0.18	0.501525806262096	0.866904538099407	0.650450223206467\\
0.18	0.515690940160279	0.874896985111485	0.646455401874959\\
0.18	0.529951380947623	0.882640524634437	0.642417151055155\\
0.18	0.544300728313782	0.890133795440129	0.638329210947585\\
0.18	0.558732572247415	0.897375649995373	0.634187835387499\\
0.18	0.57324050169555	0.904365153357279	0.629991194998376\\
0.18	0.587818113093873	0.911101581661729	0.625738868900924\\
0.18	0.602459018746568	0.917584420216504	0.62143143163314\\
0.18	0.617156855035314	0.923813361211857	0.617070132480554\\
0.18	0.631905290438127	0.92978830106243	0.612656657833838\\
0.18	0.6466980333399	0.935509337395467	0.608192963377087\\
0.18	0.661528839617671	0.940976765701173	0.603681161495105\\
0.18	0.676391519984904	0.946191075661951	0.599123449738607\\
0.18	0.691279947080357	0.951152947177946	0.594522067901591\\
0.18	0.706188062288412	0.955863246106974	0.589879273662389\\
0.18	0.721109882279076	0.960323019737426	0.585197329319936\\
0.18	0.736039505257225	0.964533492013186	0.580478494543978\\
0.18	0.75097111691199	0.968496058529893	0.575725022018316\\
0.18	0.765898996058536	0.972212281322139	0.570939154289558\\
0.18	0.780817519965826	0.975683883461272	0.566123121053174\\
0.18	0.795721169365275	0.97891274348356	0.561279136605558\\
0.18	0.810604533136476	0.981900889668376	0.556409397399198\\
0.18	0.825462312667457	0.984650494185936	0.551516079695736\\
0.18	0.840289325888138	0.987163867133889	0.546601337319641\\
0.18	0.855080510976839	0.989443450481768	0.541667299515139\\
0.18	0.86983092974082	0.991491811941914	0.536716068908598\\
0.18	0.884535770672926	0.993311638785082	0.531749719578142\\
0.18	0.899190351687411	0.994905731618374	0.526770295231796\\
0.18	0.913790122539012	0.996276998142666	0.521779807495104\\
0.18	0.928330666930242	0.997428446906011	0.516780234308685\\
0.18	0.942807704312699	0.998363181068902	0.511773518435873\\
0.18	0.95721709138901	0.999084392196567	0.506761566080168\\
0.18	0.971554823322704	0.999595354092744	0.501746245611874\\
0.18	0.985817034663989	0.999899416688637	0.496729386403002\\
0.18	1	1	0.491669087346908\\
0.195	0	0	0.710195569525497\\
0.195	0.000100583311362513	0.0141829653360114	0.71240460418299\\
0.195	0.000404645907256436	0.0284451766772965	0.714630690536905\\
0.195	0.000915607803433	0.0427829086109896	0.716916312064106\\
0.195	0.00163681893109844	0.057192295687301	0.719289019360234\\
0.195	0.00257155309398959	0.0716693330697585	0.721780367061016\\
0.195	0.00372300185733414	0.086209877460988	0.724425655492835\\
0.195	0.00509426838162598	0.100809648312589	0.727263413850304\\
0.195	0.00668836121491816	0.115464229327074	0.73033458997498\\
0.195	0.00850818805808555	0.13016907025918	0.733681422514709\\
0.195	0.0105565495182326	0.144919489023162	0.737345986077549\\
0.195	0.0128361328661109	0.159710674111862	0.741368418705642\\
0.195	0.0153495058140643	0.174537687332543	0.745784862958202\\
0.195	0.0180991103316243	0.189395466863524	0.750625176092311\\
0.195	0.0210872565164405	0.204278830634725	0.755910489868998\\
0.195	0.0243161165387281	0.219182480034174	0.761650724664453\\
0.195	0.0277877186778607	0.234101003941464	0.767842183867616\\
0.195	0.0315039414701067	0.24902888308801	0.77446537092041\\
0.195	0.0354665079868145	0.263960494742775	0.781483180780606\\
0.195	0.0396769802625738	0.278890117720924	0.788839618268073\\
0.195	0.0441367538930258	0.293811937711588	0.796459186325544\\
0.195	0.0488470528220538	0.308720052919643	0.804247066923454\\
0.195	0.0538089243380495	0.323608480015096	0.81209018616195\\
0.195	0.0590232342988274	0.338471160382329	0.819859213932694\\
0.195	0.064490662604533	0.3533019666601	0.827411499063639\\
0.195	0.0702116989375697	0.368094709561873	0.834594885813506\\
0.195	0.0761866387881432	0.382843144964686	0.841252300287181\\
0.195	0.0824155797834956	0.397540981253432	0.847226939729081\\
0.195	0.0888984183382709	0.412181886906127	0.85236784790504\\
0.195	0.0956348466427212	0.42675949830445	0.85653562002783\\
0.195	0.102624350004627	0.441267427752584	0.859607954619617\\
0.195	0.109866204559871	0.455699271686219	0.861484760266622\\
0.195	0.117359475365564	0.470048619052377	0.86209253425154\\
0.195	0.125103014888515	0.484309059839721	0.861387758068372\\
0.195	0.133095461900593	0.498474193737904	0.85935910089082\\
0.195	0.14133524079124	0.512537638903683	0.856028283765738\\
0.195	0.149820561306021	0.526493040810599	0.851449530870182\\
0.195	0.158549418718625	0.540334081158348	0.845707614739505\\
0.195	0.167519594442214	0.554054486817265	0.838914584343151\\
0.195	0.176728657084455	0.567648038782867	0.831205342370836\\
0.195	0.186173963948925	0.581108581114919	0.82273230541753\\
0.195	0.195852662983903	0.594430029835237	0.813659432944309\\
0.195	0.205761695177907	0.607606381758231	0.804155944104868\\
0.195	0.215897797399558	0.620631723228143	0.794390053406969\\
0.195	0.226257505677663	0.633500238737009	0.7845230460628\\
0.195	0.236837158915675	0.646206219397571	0.774703982861589\\
0.195	0.247632903032949	0.658744071245709	0.765065275211016\\
0.195	0.258640695523528	0.671108323347402	0.755719307772964\\
0.195	0.269856310421543	0.683293635685828	0.746756214005914\\
0.195	0.28127534366066	0.695294806804925	0.738242834604294\\
0.195	0.292893218813452	0.707106781186547	0.730222816007836\\
0.195	0.304705193195075	0.71872465633934	0.722717741087768\\
0.195	0.316706364314172	0.730143689578457	0.715729131143265\\
0.195	0.328891676652598	0.741359304476472	0.709241120561857\\
0.195	0.341255928754291	0.752367096967051	0.703223584571265\\
0.195	0.353793780602429	0.763162841084325	0.697635496597417\\
0.195	0.366499761262991	0.773742494322337	0.692428303594618\\
0.195	0.379368276771857	0.784102202600442	0.687549132887069\\
0.195	0.392393618241769	0.794238304822092	0.682943679235329\\
0.195	0.405569970164763	0.804147337016097	0.678558662193046\\
0.195	0.418891418885081	0.813826036051075	0.674343787412872\\
0.195	0.432351961217133	0.823271342915545	0.670253187715552\\
0.195	0.445945513182735	0.832480405557787	0.666246357336157\\
0.195	0.459665918841652	0.841450581281375	0.662288623474185\\
0.195	0.473506959189401	0.850179438693979	0.658351221673192\\
0.195	0.487462361096317	0.85866475920876	0.654411055137887\\
0.195	0.501525806262096	0.866904538099407	0.650450223206467\\
0.195	0.515690940160279	0.874896985111485	0.646455401874959\\
0.195	0.529951380947623	0.882640524634437	0.642417151055156\\
0.195	0.544300728313782	0.890133795440129	0.638329210947585\\
0.195	0.558732572247415	0.897375649995373	0.634187835387497\\
0.195	0.57324050169555	0.904365153357279	0.629991194998376\\
0.195	0.587818113093873	0.911101581661729	0.625738868900924\\
0.195	0.602459018746568	0.917584420216504	0.62143143163314\\
0.195	0.617156855035314	0.923813361211857	0.617070132480554\\
0.195	0.631905290438127	0.92978830106243	0.612656657833838\\
0.195	0.6466980333399	0.935509337395467	0.608192963377087\\
0.195	0.661528839617671	0.940976765701173	0.603681161495105\\
0.195	0.676391519984904	0.946191075661951	0.599123449738607\\
0.195	0.691279947080357	0.951152947177946	0.59452206790159\\
0.195	0.706188062288412	0.955863246106974	0.589879273662389\\
0.195	0.721109882279076	0.960323019737426	0.585197329319936\\
0.195	0.736039505257225	0.964533492013186	0.580478494543975\\
0.195	0.75097111691199	0.968496058529893	0.575725022018313\\
0.195	0.765898996058536	0.972212281322139	0.570939154289561\\
0.195	0.780817519965826	0.975683883461272	0.566123121053175\\
0.195	0.795721169365275	0.97891274348356	0.561279136605558\\
0.195	0.810604533136476	0.981900889668376	0.556409397399198\\
0.195	0.825462312667457	0.984650494185936	0.551516079695735\\
0.195	0.840289325888138	0.987163867133889	0.546601337319641\\
0.195	0.855080510976839	0.989443450481768	0.541667299515139\\
0.195	0.86983092974082	0.991491811941914	0.536716068908597\\
0.195	0.884535770672926	0.993311638785082	0.531749719578141\\
0.195	0.899190351687411	0.994905731618374	0.526770295231796\\
0.195	0.913790122539012	0.996276998142666	0.521779807495104\\
0.195	0.928330666930242	0.997428446906011	0.516780234308684\\
0.195	0.942807704312699	0.998363181068902	0.511773518435873\\
0.195	0.95721709138901	0.999084392196567	0.506761566080168\\
0.195	0.971554823322703	0.999595354092743	0.501746245611875\\
0.195	0.985817034663989	0.999899416688637	0.496729386403004\\
0.195	1	1	0.491669087346906\\
0.21	0	0	0.710195569525497\\
0.21	0.000100583311362513	0.0141829653360114	0.71240460418299\\
0.21	0.000404645907256436	0.0284451766772965	0.714630690536905\\
0.21	0.000915607803433	0.0427829086109896	0.716916312064106\\
0.21	0.00163681893109844	0.057192295687301	0.719289019360234\\
0.21	0.00257155309398959	0.0716693330697584	0.721780367061016\\
0.21	0.00372300185733414	0.086209877460988	0.724425655492835\\
0.21	0.00509426838162598	0.100809648312589	0.727263413850304\\
0.21	0.00668836121491816	0.115464229327074	0.73033458997498\\
0.21	0.00850818805808555	0.13016907025918	0.733681422514709\\
0.21	0.0105565495182326	0.144919489023162	0.737345986077549\\
0.21	0.0128361328661109	0.159710674111862	0.741368418705642\\
0.21	0.0153495058140643	0.174537687332543	0.745784862958202\\
0.21	0.0180991103316243	0.189395466863524	0.750625176092311\\
0.21	0.0210872565164405	0.204278830634725	0.755910489868998\\
0.21	0.0243161165387281	0.219182480034174	0.761650724664453\\
0.21	0.0277877186778607	0.234101003941464	0.767842183867616\\
0.21	0.0315039414701067	0.24902888308801	0.77446537092041\\
0.21	0.0354665079868145	0.263960494742775	0.781483180780606\\
0.21	0.0396769802625738	0.278890117720924	0.788839618268074\\
0.21	0.0441367538930258	0.293811937711588	0.796459186325544\\
0.21	0.0488470528220538	0.308720052919643	0.804247066923454\\
0.21	0.0538089243380495	0.323608480015096	0.81209018616195\\
0.21	0.0590232342988274	0.338471160382329	0.819859213932694\\
0.21	0.064490662604533	0.3533019666601	0.827411499063638\\
0.21	0.0702116989375697	0.368094709561873	0.834594885813504\\
0.21	0.0761866387881432	0.382843144964686	0.841252300287179\\
0.21	0.0824155797834956	0.397540981253432	0.84722693972908\\
0.21	0.0888984183382709	0.412181886906127	0.852367847905042\\
0.21	0.0956348466427212	0.42675949830445	0.856535620027829\\
0.21	0.102624350004627	0.441267427752584	0.859607954619617\\
0.21	0.109866204559871	0.455699271686219	0.861484760266619\\
0.21	0.117359475365564	0.470048619052377	0.86209253425154\\
0.21	0.125103014888515	0.484309059839721	0.861387758068369\\
0.21	0.133095461900593	0.498474193737904	0.859359100890822\\
0.21	0.14133524079124	0.512537638903683	0.856028283765737\\
0.21	0.149820561306021	0.526493040810599	0.851449530870183\\
0.21	0.158549418718625	0.540334081158348	0.845707614739508\\
0.21	0.167519594442213	0.554054486817265	0.83891458434315\\
0.21	0.176728657084455	0.567648038782867	0.831205342370839\\
0.21	0.186173963948925	0.581108581114919	0.822732305417529\\
0.21	0.195852662983903	0.594430029835237	0.813659432944308\\
0.21	0.205761695177907	0.607606381758231	0.804155944104868\\
0.21	0.215897797399558	0.620631723228143	0.794390053406972\\
0.21	0.226257505677663	0.633500238737009	0.7845230460628\\
0.21	0.236837158915675	0.646206219397571	0.774703982861589\\
0.21	0.247632903032949	0.658744071245709	0.765065275211016\\
0.21	0.258640695523528	0.671108323347402	0.755719307772965\\
0.21	0.269856310421543	0.683293635685828	0.746756214005914\\
0.21	0.28127534366066	0.695294806804925	0.738242834604294\\
0.21	0.292893218813452	0.707106781186547	0.730222816007836\\
0.21	0.304705193195075	0.71872465633934	0.722717741087767\\
0.21	0.316706364314172	0.730143689578457	0.715729131143265\\
0.21	0.328891676652598	0.741359304476472	0.709241120561857\\
0.21	0.341255928754291	0.752367096967051	0.703223584571265\\
0.21	0.353793780602429	0.763162841084325	0.697635496597417\\
0.21	0.366499761262991	0.773742494322337	0.692428303594619\\
0.21	0.379368276771857	0.784102202600442	0.68754913288707\\
0.21	0.392393618241769	0.794238304822092	0.682943679235329\\
0.21	0.405569970164763	0.804147337016097	0.678558662193045\\
0.21	0.418891418885081	0.813826036051075	0.674343787412872\\
0.21	0.432351961217133	0.823271342915545	0.670253187715552\\
0.21	0.445945513182735	0.832480405557787	0.666246357336158\\
0.21	0.459665918841652	0.841450581281375	0.662288623474184\\
0.21	0.473506959189401	0.850179438693979	0.658351221673191\\
0.21	0.487462361096317	0.85866475920876	0.654411055137887\\
0.21	0.501525806262096	0.866904538099407	0.650450223206468\\
0.21	0.515690940160279	0.874896985111485	0.646455401874959\\
0.21	0.529951380947623	0.882640524634437	0.642417151055155\\
0.21	0.544300728313782	0.890133795440129	0.638329210947586\\
0.21	0.558732572247415	0.897375649995373	0.634187835387498\\
0.21	0.57324050169555	0.904365153357279	0.629991194998375\\
0.21	0.587818113093873	0.911101581661729	0.625738868900923\\
0.21	0.602459018746568	0.917584420216504	0.62143143163314\\
0.21	0.617156855035314	0.923813361211857	0.617070132480553\\
0.21	0.631905290438127	0.92978830106243	0.612656657833838\\
0.21	0.6466980333399	0.935509337395467	0.608192963377087\\
0.21	0.661528839617671	0.940976765701173	0.603681161495105\\
0.21	0.676391519984904	0.946191075661951	0.599123449738607\\
0.21	0.691279947080357	0.951152947177946	0.59452206790159\\
0.21	0.706188062288412	0.955863246106974	0.589879273662389\\
0.21	0.721109882279076	0.960323019737426	0.585197329319936\\
0.21	0.736039505257225	0.964533492013186	0.580478494543978\\
0.21	0.75097111691199	0.968496058529893	0.575725022018313\\
0.21	0.765898996058536	0.972212281322139	0.570939154289558\\
0.21	0.780817519965826	0.975683883461272	0.566123121053175\\
0.21	0.795721169365275	0.97891274348356	0.561279136605559\\
0.21	0.810604533136476	0.981900889668376	0.556409397399199\\
0.21	0.825462312667457	0.984650494185936	0.551516079695735\\
0.21	0.840289325888138	0.987163867133889	0.546601337319639\\
0.21	0.855080510976839	0.989443450481768	0.541667299515141\\
0.21	0.86983092974082	0.991491811941915	0.536716068908597\\
0.21	0.884535770672926	0.993311638785082	0.53174971957814\\
0.21	0.899190351687411	0.994905731618374	0.526770295231796\\
0.21	0.913790122539012	0.996276998142666	0.521779807495104\\
0.21	0.928330666930242	0.997428446906011	0.516780234308684\\
0.21	0.942807704312699	0.998363181068902	0.511773518435872\\
0.21	0.95721709138901	0.999084392196567	0.506761566080168\\
0.21	0.971554823322703	0.999595354092743	0.501746245611874\\
0.21	0.985817034663989	0.999899416688637	0.496729386403004\\
0.21	1	1	0.491669087346906\\
0.225	0	0	0.710195569525497\\
0.225	0.000100583311362513	0.0141829653360114	0.71240460418299\\
0.225	0.000404645907256436	0.0284451766772965	0.714630690536905\\
0.225	0.000915607803433	0.0427829086109896	0.716916312064106\\
0.225	0.00163681893109844	0.057192295687301	0.719289019360234\\
0.225	0.00257155309398959	0.0716693330697585	0.721780367061016\\
0.225	0.00372300185733414	0.0862098774609879	0.724425655492835\\
0.225	0.00509426838162598	0.100809648312589	0.727263413850304\\
0.225	0.00668836121491816	0.115464229327074	0.73033458997498\\
0.225	0.00850818805808555	0.13016907025918	0.733681422514709\\
0.225	0.0105565495182326	0.144919489023162	0.737345986077549\\
0.225	0.0128361328661109	0.159710674111862	0.741368418705642\\
0.225	0.0153495058140643	0.174537687332543	0.745784862958201\\
0.225	0.0180991103316243	0.189395466863524	0.750625176092311\\
0.225	0.0210872565164405	0.204278830634725	0.755910489868999\\
0.225	0.0243161165387281	0.219182480034174	0.761650724664453\\
0.225	0.0277877186778607	0.234101003941464	0.767842183867616\\
0.225	0.0315039414701067	0.24902888308801	0.77446537092041\\
0.225	0.0354665079868145	0.263960494742775	0.781483180780606\\
0.225	0.0396769802625738	0.278890117720924	0.788839618268073\\
0.225	0.0441367538930257	0.293811937711588	0.796459186325544\\
0.225	0.0488470528220538	0.308720052919643	0.804247066923454\\
0.225	0.0538089243380495	0.323608480015097	0.81209018616195\\
0.225	0.0590232342988274	0.338471160382329	0.819859213932694\\
0.225	0.064490662604533	0.3533019666601	0.82741149906364\\
0.225	0.0702116989375697	0.368094709561873	0.834594885813502\\
0.225	0.0761866387881432	0.382843144964686	0.841252300287179\\
0.225	0.0824155797834956	0.397540981253432	0.847226939729081\\
0.225	0.0888984183382709	0.412181886906127	0.85236784790504\\
0.225	0.0956348466427213	0.42675949830445	0.85653562002783\\
0.225	0.102624350004627	0.441267427752584	0.859607954619618\\
0.225	0.109866204559871	0.455699271686218	0.861484760266622\\
0.225	0.117359475365564	0.470048619052377	0.862092534251541\\
0.225	0.125103014888515	0.484309059839721	0.861387758068367\\
0.225	0.133095461900593	0.498474193737904	0.859359100890822\\
0.225	0.14133524079124	0.512537638903683	0.856028283765737\\
0.225	0.149820561306021	0.526493040810599	0.851449530870183\\
0.225	0.158549418718625	0.540334081158348	0.845707614739502\\
0.225	0.167519594442214	0.554054486817265	0.838914584343151\\
0.225	0.176728657084455	0.567648038782868	0.831205342370839\\
0.225	0.186173963948925	0.581108581114919	0.822732305417527\\
0.225	0.195852662983903	0.594430029835237	0.813659432944309\\
0.225	0.205761695177907	0.607606381758231	0.804155944104869\\
0.225	0.215897797399558	0.620631723228144	0.794390053406971\\
0.225	0.226257505677663	0.633500238737009	0.784523046062799\\
0.225	0.236837158915675	0.646206219397571	0.774703982861591\\
0.225	0.247632903032949	0.658744071245709	0.765065275211016\\
0.225	0.258640695523528	0.671108323347402	0.755719307772964\\
0.225	0.269856310421543	0.683293635685828	0.746756214005913\\
0.225	0.28127534366066	0.695294806804925	0.738242834604294\\
0.225	0.292893218813452	0.707106781186547	0.730222816007836\\
0.225	0.304705193195075	0.71872465633934	0.722717741087767\\
0.225	0.316706364314172	0.730143689578457	0.715729131143265\\
0.225	0.328891676652598	0.741359304476472	0.709241120561856\\
0.225	0.341255928754291	0.752367096967051	0.703223584571265\\
0.225	0.353793780602429	0.763162841084325	0.697635496597417\\
0.225	0.366499761262991	0.773742494322337	0.69242830359462\\
0.225	0.379368276771857	0.784102202600443	0.68754913288707\\
0.225	0.392393618241769	0.794238304822092	0.682943679235329\\
0.225	0.405569970164763	0.804147337016097	0.678558662193046\\
0.225	0.418891418885081	0.813826036051075	0.674343787412872\\
0.225	0.432351961217133	0.823271342915545	0.670253187715552\\
0.225	0.445945513182735	0.832480405557787	0.666246357336158\\
0.225	0.459665918841652	0.841450581281375	0.662288623474185\\
0.225	0.473506959189401	0.850179438693979	0.65835122167319\\
0.225	0.487462361096317	0.85866475920876	0.654411055137885\\
0.225	0.501525806262096	0.866904538099407	0.650450223206468\\
0.225	0.515690940160279	0.874896985111485	0.64645540187496\\
0.225	0.529951380947623	0.882640524634437	0.642417151055155\\
0.225	0.544300728313782	0.890133795440129	0.638329210947585\\
0.225	0.558732572247415	0.897375649995373	0.634187835387498\\
0.225	0.57324050169555	0.904365153357279	0.629991194998377\\
0.225	0.587818113093873	0.911101581661729	0.625738868900924\\
0.225	0.602459018746568	0.917584420216504	0.621431431633138\\
0.225	0.617156855035314	0.923813361211857	0.617070132480554\\
0.225	0.631905290438127	0.92978830106243	0.612656657833837\\
0.225	0.6466980333399	0.935509337395467	0.608192963377087\\
0.225	0.661528839617671	0.940976765701173	0.603681161495105\\
0.225	0.676391519984904	0.946191075661951	0.599123449738608\\
0.225	0.691279947080357	0.951152947177946	0.59452206790159\\
0.225	0.706188062288412	0.955863246106974	0.589879273662389\\
0.225	0.721109882279076	0.960323019737426	0.585197329319937\\
0.225	0.736039505257225	0.964533492013186	0.580478494543978\\
0.225	0.75097111691199	0.968496058529893	0.575725022018313\\
0.225	0.765898996058536	0.972212281322139	0.570939154289558\\
0.225	0.780817519965826	0.975683883461272	0.566123121053175\\
0.225	0.795721169365275	0.97891274348356	0.561279136605558\\
0.225	0.810604533136476	0.981900889668376	0.5564093973992\\
0.225	0.825462312667457	0.984650494185936	0.551516079695736\\
0.225	0.840289325888138	0.987163867133889	0.546601337319639\\
0.225	0.855080510976839	0.989443450481767	0.541667299515139\\
0.225	0.86983092974082	0.991491811941914	0.536716068908596\\
0.225	0.884535770672926	0.993311638785082	0.531749719578142\\
0.225	0.899190351687411	0.994905731618374	0.526770295231797\\
0.225	0.913790122539012	0.996276998142666	0.521779807495102\\
0.225	0.928330666930242	0.997428446906011	0.516780234308685\\
0.225	0.942807704312699	0.998363181068902	0.511773518435872\\
0.225	0.95721709138901	0.999084392196567	0.506761566080166\\
0.225	0.971554823322703	0.999595354092743	0.501746245611875\\
0.225	0.985817034663989	0.999899416688637	0.496729386403004\\
0.225	1	1	0.491669087346906\\
0.24	0	0	0.710195569525497\\
0.24	0.000100583311362513	0.0141829653360114	0.71240460418299\\
0.24	0.000404645907256436	0.0284451766772965	0.714630690536905\\
0.24	0.000915607803433	0.0427829086109896	0.716916312064106\\
0.24	0.00163681893109844	0.057192295687301	0.719289019360234\\
0.24	0.00257155309398959	0.0716693330697585	0.721780367061016\\
0.24	0.00372300185733414	0.0862098774609879	0.724425655492835\\
0.24	0.00509426838162598	0.100809648312589	0.727263413850304\\
0.24	0.00668836121491816	0.115464229327074	0.73033458997498\\
0.24	0.00850818805808555	0.13016907025918	0.733681422514709\\
0.24	0.0105565495182326	0.144919489023162	0.737345986077549\\
0.24	0.0128361328661109	0.159710674111862	0.741368418705642\\
0.24	0.0153495058140643	0.174537687332543	0.745784862958202\\
0.24	0.0180991103316243	0.189395466863524	0.750625176092311\\
0.24	0.0210872565164405	0.204278830634725	0.755910489868998\\
0.24	0.0243161165387281	0.219182480034174	0.761650724664453\\
0.24	0.0277877186778607	0.234101003941464	0.767842183867616\\
0.24	0.0315039414701067	0.24902888308801	0.77446537092041\\
0.24	0.0354665079868145	0.263960494742775	0.781483180780606\\
0.24	0.0396769802625738	0.278890117720924	0.788839618268073\\
0.24	0.0441367538930258	0.293811937711588	0.796459186325544\\
0.24	0.0488470528220538	0.308720052919643	0.804247066923454\\
0.24	0.0538089243380495	0.323608480015096	0.812090186161949\\
0.24	0.0590232342988274	0.338471160382329	0.819859213932694\\
0.24	0.064490662604533	0.3533019666601	0.827411499063641\\
0.24	0.0702116989375697	0.368094709561873	0.834594885813504\\
0.24	0.0761866387881432	0.382843144964686	0.84125230028718\\
0.24	0.0824155797834956	0.397540981253432	0.847226939729083\\
0.24	0.0888984183382709	0.412181886906127	0.852367847905041\\
0.24	0.0956348466427212	0.42675949830445	0.85653562002783\\
0.24	0.102624350004627	0.441267427752584	0.859607954619617\\
0.24	0.109866204559871	0.455699271686219	0.861484760266624\\
0.24	0.117359475365564	0.470048619052377	0.862092534251541\\
0.24	0.125103014888515	0.484309059839721	0.861387758068369\\
0.24	0.133095461900593	0.498474193737904	0.859359100890819\\
0.24	0.14133524079124	0.512537638903683	0.856028283765738\\
0.24	0.149820561306021	0.526493040810599	0.851449530870182\\
0.24	0.158549418718625	0.540334081158348	0.845707614739506\\
0.24	0.167519594442214	0.554054486817265	0.838914584343152\\
0.24	0.176728657084455	0.567648038782868	0.831205342370836\\
0.24	0.186173963948925	0.581108581114919	0.822732305417527\\
0.24	0.195852662983903	0.594430029835237	0.813659432944309\\
0.24	0.205761695177907	0.607606381758231	0.804155944104869\\
0.24	0.215897797399558	0.620631723228144	0.79439005340697\\
0.24	0.226257505677663	0.633500238737009	0.784523046062801\\
0.24	0.236837158915675	0.646206219397571	0.774703982861589\\
0.24	0.247632903032949	0.658744071245709	0.765065275211015\\
0.24	0.258640695523528	0.671108323347402	0.755719307772964\\
0.24	0.269856310421543	0.683293635685828	0.746756214005914\\
0.24	0.28127534366066	0.695294806804925	0.738242834604294\\
0.24	0.292893218813452	0.707106781186547	0.730222816007836\\
0.24	0.304705193195075	0.71872465633934	0.722717741087767\\
0.24	0.316706364314172	0.730143689578457	0.715729131143265\\
0.24	0.328891676652598	0.741359304476472	0.709241120561856\\
0.24	0.341255928754291	0.752367096967051	0.703223584571266\\
0.24	0.353793780602429	0.763162841084325	0.697635496597419\\
0.24	0.366499761262991	0.773742494322337	0.692428303594618\\
0.24	0.379368276771857	0.784102202600443	0.687549132887069\\
0.24	0.392393618241769	0.794238304822092	0.682943679235328\\
0.24	0.405569970164763	0.804147337016097	0.678558662193046\\
0.24	0.418891418885081	0.813826036051075	0.674343787412873\\
0.24	0.432351961217132	0.823271342915545	0.670253187715552\\
0.24	0.445945513182735	0.832480405557787	0.666246357336156\\
0.24	0.459665918841652	0.841450581281375	0.662288623474185\\
0.24	0.473506959189401	0.850179438693979	0.658351221673191\\
0.24	0.487462361096317	0.85866475920876	0.654411055137886\\
0.24	0.501525806262096	0.866904538099407	0.650450223206468\\
0.24	0.515690940160279	0.874896985111485	0.646455401874959\\
0.24	0.529951380947623	0.882640524634437	0.642417151055155\\
0.24	0.544300728313782	0.890133795440129	0.638329210947585\\
0.24	0.558732572247415	0.897375649995373	0.634187835387499\\
0.24	0.57324050169555	0.904365153357279	0.629991194998376\\
0.24	0.587818113093873	0.911101581661729	0.625738868900924\\
0.24	0.602459018746568	0.917584420216505	0.62143143163314\\
0.24	0.617156855035314	0.923813361211857	0.617070132480554\\
0.24	0.631905290438127	0.92978830106243	0.612656657833839\\
0.24	0.6466980333399	0.935509337395467	0.608192963377087\\
0.24	0.661528839617671	0.940976765701173	0.603681161495104\\
0.24	0.676391519984904	0.94619107566195	0.599123449738607\\
0.24	0.691279947080357	0.951152947177946	0.59452206790159\\
0.24	0.706188062288412	0.955863246106974	0.589879273662389\\
0.24	0.721109882279076	0.960323019737426	0.585197329319937\\
0.24	0.736039505257226	0.964533492013186	0.580478494543978\\
0.24	0.75097111691199	0.968496058529893	0.575725022018312\\
0.24	0.765898996058536	0.972212281322139	0.570939154289557\\
0.24	0.780817519965826	0.975683883461272	0.566123121053177\\
0.24	0.795721169365275	0.97891274348356	0.561279136605558\\
0.24	0.810604533136476	0.981900889668376	0.556409397399198\\
0.24	0.825462312667457	0.984650494185936	0.551516079695738\\
0.24	0.840289325888138	0.987163867133889	0.546601337319642\\
0.24	0.855080510976839	0.989443450481768	0.541667299515139\\
0.24	0.86983092974082	0.991491811941915	0.536716068908596\\
0.24	0.884535770672926	0.993311638785082	0.53174971957814\\
0.24	0.899190351687411	0.994905731618374	0.526770295231798\\
0.24	0.913790122539012	0.996276998142666	0.521779807495103\\
0.24	0.928330666930242	0.997428446906011	0.516780234308682\\
0.24	0.942807704312699	0.998363181068902	0.511773518435873\\
0.24	0.95721709138901	0.999084392196567	0.506761566080166\\
0.24	0.971554823322703	0.999595354092743	0.501746245611874\\
0.24	0.985817034663989	0.999899416688637	0.496729386403005\\
0.24	1	1	0.491669087346904\\
0.255	0	0	0.710195569525497\\
0.255	0.000100583311362513	0.0141829653360114	0.71240460418299\\
0.255	0.000404645907256436	0.0284451766772965	0.714630690536905\\
0.255	0.000915607803433	0.0427829086109896	0.716916312064106\\
0.255	0.00163681893109844	0.057192295687301	0.719289019360234\\
0.255	0.00257155309398959	0.0716693330697585	0.721780367061016\\
0.255	0.00372300185733414	0.086209877460988	0.724425655492835\\
0.255	0.00509426838162598	0.100809648312589	0.727263413850304\\
0.255	0.00668836121491816	0.115464229327074	0.73033458997498\\
0.255	0.00850818805808555	0.13016907025918	0.733681422514709\\
0.255	0.0105565495182326	0.144919489023162	0.737345986077549\\
0.255	0.0128361328661109	0.159710674111862	0.741368418705642\\
0.255	0.0153495058140643	0.174537687332543	0.745784862958202\\
0.255	0.0180991103316243	0.189395466863524	0.750625176092311\\
0.255	0.0210872565164405	0.204278830634725	0.755910489868998\\
0.255	0.0243161165387281	0.219182480034174	0.761650724664453\\
0.255	0.0277877186778607	0.234101003941464	0.767842183867616\\
0.255	0.0315039414701067	0.24902888308801	0.77446537092041\\
0.255	0.0354665079868145	0.263960494742775	0.781483180780606\\
0.255	0.0396769802625738	0.278890117720924	0.788839618268074\\
0.255	0.0441367538930258	0.293811937711588	0.796459186325544\\
0.255	0.0488470528220537	0.308720052919643	0.804247066923454\\
0.255	0.0538089243380495	0.323608480015096	0.81209018616195\\
0.255	0.0590232342988274	0.338471160382329	0.819859213932692\\
0.255	0.064490662604533	0.3533019666601	0.82741149906364\\
0.255	0.0702116989375697	0.368094709561873	0.834594885813507\\
0.255	0.0761866387881432	0.382843144964686	0.841252300287178\\
0.255	0.0824155797834956	0.397540981253432	0.84722693972908\\
0.255	0.0888984183382709	0.412181886906127	0.852367847905043\\
0.255	0.0956348466427212	0.42675949830445	0.85653562002783\\
0.255	0.102624350004627	0.441267427752584	0.859607954619618\\
0.255	0.109866204559871	0.455699271686219	0.861484760266621\\
0.255	0.117359475365564	0.470048619052377	0.862092534251541\\
0.255	0.125103014888515	0.484309059839721	0.861387758068367\\
0.255	0.133095461900593	0.498474193737904	0.859359100890819\\
0.255	0.14133524079124	0.512537638903683	0.856028283765735\\
0.255	0.149820561306021	0.526493040810599	0.851449530870183\\
0.255	0.158549418718625	0.540334081158348	0.845707614739508\\
0.255	0.167519594442214	0.554054486817265	0.83891458434315\\
0.255	0.176728657084455	0.567648038782868	0.831205342370835\\
0.255	0.186173963948925	0.581108581114919	0.822732305417528\\
0.255	0.195852662983903	0.594430029835237	0.813659432944309\\
0.255	0.205761695177907	0.607606381758231	0.804155944104867\\
0.255	0.215897797399558	0.620631723228144	0.794390053406967\\
0.255	0.226257505677663	0.633500238737009	0.784523046062799\\
0.255	0.236837158915675	0.646206219397571	0.774703982861587\\
0.255	0.247632903032949	0.658744071245709	0.765065275211015\\
0.255	0.258640695523528	0.671108323347402	0.755719307772963\\
0.255	0.269856310421543	0.683293635685828	0.746756214005913\\
0.255	0.28127534366066	0.695294806804925	0.738242834604294\\
0.255	0.292893218813452	0.707106781186547	0.730222816007835\\
0.255	0.304705193195076	0.71872465633934	0.722717741087767\\
0.255	0.316706364314172	0.730143689578457	0.715729131143265\\
0.255	0.328891676652598	0.741359304476472	0.709241120561857\\
0.255	0.341255928754291	0.752367096967051	0.703223584571266\\
0.255	0.353793780602429	0.763162841084325	0.697635496597417\\
0.255	0.366499761262991	0.773742494322337	0.692428303594619\\
0.255	0.379368276771857	0.784102202600443	0.687549132887069\\
0.255	0.392393618241769	0.794238304822092	0.682943679235328\\
0.255	0.405569970164763	0.804147337016097	0.678558662193046\\
0.255	0.418891418885081	0.813826036051075	0.674343787412872\\
0.255	0.432351961217133	0.823271342915545	0.670253187715552\\
0.255	0.445945513182735	0.832480405557787	0.666246357336158\\
0.255	0.459665918841652	0.841450581281375	0.662288623474184\\
0.255	0.473506959189401	0.850179438693979	0.658351221673192\\
0.255	0.487462361096317	0.85866475920876	0.654411055137885\\
0.255	0.501525806262096	0.866904538099407	0.650450223206467\\
0.255	0.515690940160279	0.874896985111485	0.64645540187496\\
0.255	0.529951380947623	0.882640524634437	0.642417151055155\\
0.255	0.544300728313782	0.890133795440129	0.638329210947586\\
0.255	0.558732572247415	0.897375649995373	0.634187835387498\\
0.255	0.57324050169555	0.904365153357279	0.629991194998376\\
0.255	0.587818113093873	0.911101581661729	0.625738868900924\\
0.255	0.602459018746568	0.917584420216504	0.621431431633139\\
0.255	0.617156855035314	0.923813361211857	0.617070132480553\\
0.255	0.631905290438127	0.92978830106243	0.61265665783384\\
0.255	0.6466980333399	0.935509337395467	0.608192963377089\\
0.255	0.661528839617671	0.940976765701173	0.603681161495106\\
0.255	0.676391519984904	0.946191075661951	0.599123449738606\\
0.255	0.691279947080357	0.951152947177946	0.594522067901591\\
0.255	0.706188062288412	0.955863246106974	0.589879273662388\\
0.255	0.721109882279076	0.960323019737426	0.585197329319936\\
0.255	0.736039505257225	0.964533492013186	0.580478494543977\\
0.255	0.75097111691199	0.968496058529893	0.575725022018314\\
0.255	0.765898996058536	0.972212281322139	0.570939154289557\\
0.255	0.780817519965826	0.975683883461272	0.566123121053175\\
0.255	0.795721169365275	0.97891274348356	0.561279136605558\\
0.255	0.810604533136476	0.981900889668376	0.556409397399197\\
0.255	0.825462312667457	0.984650494185936	0.551516079695736\\
0.255	0.840289325888138	0.987163867133889	0.546601337319641\\
0.255	0.855080510976839	0.989443450481767	0.541667299515139\\
0.255	0.86983092974082	0.991491811941914	0.536716068908598\\
0.255	0.884535770672926	0.993311638785082	0.531749719578141\\
0.255	0.899190351687411	0.994905731618374	0.526770295231795\\
0.255	0.913790122539012	0.996276998142666	0.521779807495105\\
0.255	0.928330666930242	0.997428446906011	0.516780234308684\\
0.255	0.942807704312699	0.998363181068902	0.511773518435871\\
0.255	0.95721709138901	0.999084392196567	0.506761566080171\\
0.255	0.971554823322703	0.999595354092744	0.501746245611874\\
0.255	0.985817034663989	0.999899416688637	0.496729386403002\\
0.255	1	1	0.491669087346908\\
0.27	0	0	0.710195569525497\\
0.27	0.000100583311362513	0.0141829653360114	0.71240460418299\\
0.27	0.000404645907256436	0.0284451766772965	0.714630690536905\\
0.27	0.000915607803433	0.0427829086109896	0.716916312064106\\
0.27	0.00163681893109844	0.057192295687301	0.719289019360234\\
0.27	0.00257155309398959	0.0716693330697585	0.721780367061016\\
0.27	0.00372300185733414	0.0862098774609879	0.724425655492835\\
0.27	0.00509426838162598	0.100809648312589	0.727263413850304\\
0.27	0.00668836121491816	0.115464229327074	0.73033458997498\\
0.27	0.00850818805808555	0.13016907025918	0.733681422514709\\
0.27	0.0105565495182326	0.144919489023162	0.737345986077549\\
0.27	0.0128361328661109	0.159710674111862	0.741368418705642\\
0.27	0.0153495058140643	0.174537687332543	0.745784862958201\\
0.27	0.0180991103316243	0.189395466863524	0.750625176092311\\
0.27	0.0210872565164405	0.204278830634725	0.755910489868998\\
0.27	0.0243161165387281	0.219182480034174	0.761650724664453\\
0.27	0.0277877186778607	0.234101003941464	0.767842183867616\\
0.27	0.0315039414701067	0.24902888308801	0.77446537092041\\
0.27	0.0354665079868145	0.263960494742775	0.781483180780606\\
0.27	0.0396769802625737	0.278890117720924	0.788839618268073\\
0.27	0.0441367538930258	0.293811937711588	0.796459186325544\\
0.27	0.0488470528220538	0.308720052919643	0.804247066923454\\
0.27	0.0538089243380495	0.323608480015096	0.812090186161949\\
0.27	0.0590232342988274	0.338471160382329	0.819859213932692\\
0.27	0.064490662604533	0.3533019666601	0.827411499063638\\
0.27	0.0702116989375697	0.368094709561873	0.834594885813505\\
0.27	0.0761866387881432	0.382843144964686	0.84125230028718\\
0.27	0.0824155797834956	0.397540981253432	0.847226939729081\\
0.27	0.0888984183382709	0.412181886906127	0.852367847905041\\
0.27	0.0956348466427212	0.42675949830445	0.856535620027829\\
0.27	0.102624350004627	0.441267427752584	0.859607954619617\\
0.27	0.109866204559871	0.455699271686219	0.861484760266619\\
0.27	0.117359475365564	0.470048619052377	0.862092534251542\\
0.27	0.125103014888515	0.484309059839721	0.86138775806837\\
0.27	0.133095461900593	0.498474193737904	0.859359100890822\\
0.27	0.14133524079124	0.512537638903683	0.856028283765735\\
0.27	0.149820561306021	0.526493040810599	0.851449530870183\\
0.27	0.158549418718625	0.540334081158348	0.845707614739505\\
0.27	0.167519594442213	0.554054486817265	0.83891458434315\\
0.27	0.176728657084455	0.567648038782867	0.831205342370838\\
0.27	0.186173963948925	0.581108581114919	0.822732305417529\\
0.27	0.195852662983903	0.594430029835237	0.813659432944307\\
0.27	0.205761695177907	0.607606381758231	0.804155944104866\\
0.27	0.215897797399558	0.620631723228143	0.794390053406969\\
0.27	0.226257505677663	0.633500238737009	0.7845230460628\\
0.27	0.236837158915675	0.646206219397571	0.774703982861588\\
0.27	0.247632903032949	0.658744071245709	0.765065275211015\\
0.27	0.258640695523528	0.671108323347402	0.755719307772963\\
0.27	0.269856310421543	0.683293635685828	0.746756214005912\\
0.27	0.28127534366066	0.695294806804924	0.738242834604296\\
0.27	0.292893218813452	0.707106781186547	0.730222816007835\\
0.27	0.304705193195075	0.71872465633934	0.722717741087768\\
0.27	0.316706364314172	0.730143689578457	0.715729131143267\\
0.27	0.328891676652598	0.741359304476472	0.709241120561858\\
0.27	0.341255928754291	0.752367096967051	0.703223584571265\\
0.27	0.353793780602429	0.763162841084325	0.697635496597418\\
0.27	0.366499761262991	0.773742494322337	0.69242830359462\\
0.27	0.379368276771857	0.784102202600443	0.687549132887069\\
0.27	0.392393618241769	0.794238304822092	0.682943679235328\\
0.27	0.405569970164763	0.804147337016097	0.678558662193045\\
0.27	0.418891418885081	0.813826036051075	0.674343787412872\\
0.27	0.432351961217133	0.823271342915545	0.670253187715552\\
0.27	0.445945513182735	0.832480405557787	0.666246357336159\\
0.27	0.459665918841652	0.841450581281375	0.662288623474184\\
0.27	0.473506959189401	0.850179438693979	0.65835122167319\\
0.27	0.487462361096317	0.85866475920876	0.654411055137888\\
0.27	0.501525806262096	0.866904538099407	0.650450223206467\\
0.27	0.515690940160279	0.874896985111485	0.646455401874959\\
0.27	0.529951380947623	0.882640524634437	0.642417151055155\\
0.27	0.544300728313782	0.890133795440129	0.638329210947586\\
0.27	0.558732572247415	0.897375649995373	0.634187835387498\\
0.27	0.57324050169555	0.904365153357279	0.629991194998374\\
0.27	0.587818113093873	0.911101581661729	0.625738868900924\\
0.27	0.602459018746568	0.917584420216504	0.62143143163314\\
0.27	0.617156855035314	0.923813361211857	0.617070132480553\\
0.27	0.631905290438127	0.92978830106243	0.612656657833839\\
0.27	0.6466980333399	0.935509337395467	0.608192963377086\\
0.27	0.661528839617671	0.940976765701173	0.603681161495105\\
0.27	0.676391519984904	0.946191075661951	0.599123449738608\\
0.27	0.691279947080357	0.951152947177946	0.594522067901591\\
0.27	0.706188062288412	0.955863246106974	0.589879273662389\\
0.27	0.721109882279076	0.960323019737426	0.585197329319934\\
0.27	0.736039505257225	0.964533492013186	0.580478494543978\\
0.27	0.75097111691199	0.968496058529893	0.575725022018315\\
0.27	0.765898996058536	0.972212281322139	0.570939154289557\\
0.27	0.780817519965826	0.975683883461272	0.566123121053174\\
0.27	0.795721169365275	0.97891274348356	0.561279136605559\\
0.27	0.810604533136476	0.981900889668376	0.556409397399196\\
0.27	0.825462312667457	0.984650494185936	0.551516079695735\\
0.27	0.840289325888138	0.987163867133889	0.546601337319644\\
0.27	0.855080510976839	0.989443450481768	0.541667299515139\\
0.27	0.86983092974082	0.991491811941914	0.536716068908596\\
0.27	0.884535770672926	0.993311638785082	0.531749719578141\\
0.27	0.899190351687411	0.994905731618374	0.526770295231795\\
0.27	0.913790122539012	0.996276998142666	0.521779807495104\\
0.27	0.928330666930242	0.997428446906011	0.516780234308687\\
0.27	0.942807704312699	0.998363181068902	0.511773518435871\\
0.27	0.95721709138901	0.999084392196567	0.506761566080164\\
0.27	0.971554823322703	0.999595354092743	0.501746245611875\\
0.27	0.985817034663989	0.999899416688637	0.496729386403005\\
0.27	1	1	0.491669087346904\\
0.285	0	0	0.710195569525497\\
0.285	0.000100583311362513	0.0141829653360114	0.71240460418299\\
0.285	0.000404645907256436	0.0284451766772965	0.714630690536905\\
0.285	0.000915607803433	0.0427829086109896	0.716916312064106\\
0.285	0.00163681893109844	0.057192295687301	0.719289019360234\\
0.285	0.00257155309398959	0.0716693330697584	0.721780367061016\\
0.285	0.00372300185733414	0.086209877460988	0.724425655492835\\
0.285	0.00509426838162599	0.100809648312589	0.727263413850304\\
0.285	0.00668836121491816	0.115464229327074	0.73033458997498\\
0.285	0.00850818805808555	0.13016907025918	0.733681422514709\\
0.285	0.0105565495182326	0.144919489023162	0.737345986077549\\
0.285	0.0128361328661109	0.159710674111862	0.741368418705642\\
0.285	0.0153495058140643	0.174537687332543	0.745784862958201\\
0.285	0.0180991103316243	0.189395466863524	0.750625176092311\\
0.285	0.0210872565164405	0.204278830634725	0.755910489868999\\
0.285	0.0243161165387281	0.219182480034174	0.761650724664453\\
0.285	0.0277877186778607	0.234101003941464	0.767842183867616\\
0.285	0.0315039414701067	0.24902888308801	0.77446537092041\\
0.285	0.0354665079868145	0.263960494742775	0.781483180780606\\
0.285	0.0396769802625738	0.278890117720924	0.788839618268073\\
0.285	0.0441367538930258	0.293811937711588	0.796459186325543\\
0.285	0.0488470528220537	0.308720052919643	0.804247066923455\\
0.285	0.0538089243380495	0.323608480015096	0.812090186161949\\
0.285	0.0590232342988274	0.338471160382329	0.819859213932694\\
0.285	0.064490662604533	0.3533019666601	0.827411499063638\\
0.285	0.0702116989375697	0.368094709561873	0.834594885813504\\
0.285	0.0761866387881432	0.382843144964686	0.841252300287181\\
0.285	0.0824155797834956	0.397540981253432	0.84722693972908\\
0.285	0.0888984183382709	0.412181886906127	0.852367847905041\\
0.285	0.0956348466427213	0.42675949830445	0.856535620027827\\
0.285	0.102624350004627	0.441267427752584	0.859607954619616\\
0.285	0.109866204559871	0.455699271686218	0.861484760266622\\
0.285	0.117359475365564	0.470048619052377	0.86209253425154\\
0.285	0.125103014888515	0.484309059839721	0.861387758068368\\
0.285	0.133095461900593	0.498474193737904	0.85935910089082\\
0.285	0.14133524079124	0.512537638903683	0.856028283765737\\
0.285	0.149820561306021	0.526493040810599	0.851449530870182\\
0.285	0.158549418718625	0.540334081158348	0.845707614739504\\
0.285	0.167519594442214	0.554054486817265	0.83891458434315\\
0.285	0.176728657084455	0.567648038782868	0.831205342370837\\
0.285	0.186173963948925	0.581108581114919	0.822732305417527\\
0.285	0.195852662983903	0.594430029835237	0.813659432944309\\
0.285	0.205761695177907	0.607606381758231	0.80415594410487\\
0.285	0.215897797399558	0.620631723228143	0.794390053406969\\
0.285	0.226257505677663	0.633500238737009	0.784523046062802\\
0.285	0.236837158915675	0.646206219397571	0.774703982861588\\
0.285	0.247632903032949	0.658744071245709	0.765065275211017\\
0.285	0.258640695523528	0.671108323347402	0.755719307772964\\
0.285	0.269856310421543	0.683293635685828	0.746756214005913\\
0.285	0.28127534366066	0.695294806804924	0.738242834604296\\
0.285	0.292893218813452	0.707106781186547	0.730222816007837\\
0.285	0.304705193195075	0.71872465633934	0.722717741087769\\
0.285	0.316706364314172	0.730143689578457	0.715729131143265\\
0.285	0.328891676652598	0.741359304476472	0.709241120561857\\
0.285	0.341255928754291	0.752367096967051	0.703223584571265\\
0.285	0.353793780602429	0.763162841084325	0.697635496597416\\
0.285	0.366499761262991	0.773742494322337	0.692428303594618\\
0.285	0.379368276771857	0.784102202600442	0.687549132887069\\
0.285	0.392393618241769	0.794238304822092	0.682943679235329\\
0.285	0.405569970164763	0.804147337016097	0.678558662193046\\
0.285	0.418891418885081	0.813826036051075	0.674343787412872\\
0.285	0.432351961217133	0.823271342915545	0.670253187715552\\
0.285	0.445945513182735	0.832480405557787	0.666246357336158\\
0.285	0.459665918841652	0.841450581281375	0.662288623474185\\
0.285	0.473506959189401	0.850179438693979	0.658351221673191\\
0.285	0.487462361096317	0.85866475920876	0.654411055137887\\
0.285	0.501525806262096	0.866904538099407	0.650450223206468\\
0.285	0.515690940160279	0.874896985111485	0.646455401874959\\
0.285	0.529951380947623	0.882640524634437	0.642417151055155\\
0.285	0.544300728313782	0.890133795440129	0.638329210947585\\
0.285	0.558732572247415	0.897375649995373	0.634187835387499\\
0.285	0.57324050169555	0.904365153357279	0.629991194998375\\
0.285	0.587818113093873	0.911101581661729	0.625738868900924\\
0.285	0.602459018746568	0.917584420216504	0.621431431633139\\
0.285	0.617156855035314	0.923813361211857	0.617070132480553\\
0.285	0.631905290438127	0.92978830106243	0.612656657833839\\
0.285	0.6466980333399	0.935509337395467	0.608192963377087\\
0.285	0.661528839617671	0.940976765701173	0.603681161495104\\
0.285	0.676391519984904	0.946191075661951	0.599123449738608\\
0.285	0.691279947080357	0.951152947177946	0.594522067901592\\
0.285	0.706188062288412	0.955863246106974	0.589879273662391\\
0.285	0.721109882279076	0.960323019737426	0.585197329319934\\
0.285	0.736039505257225	0.964533492013186	0.580478494543973\\
0.285	0.75097111691199	0.968496058529893	0.575725022018316\\
0.285	0.765898996058536	0.972212281322139	0.57093915428956\\
0.285	0.780817519965825	0.975683883461272	0.566123121053173\\
0.285	0.795721169365275	0.97891274348356	0.561279136605559\\
0.285	0.810604533136476	0.981900889668376	0.556409397399199\\
0.285	0.825462312667457	0.984650494185936	0.551516079695734\\
0.285	0.840289325888138	0.987163867133889	0.546601337319639\\
0.285	0.855080510976839	0.989443450481767	0.541667299515141\\
0.285	0.86983092974082	0.991491811941914	0.536716068908597\\
0.285	0.884535770672926	0.993311638785082	0.531749719578141\\
0.285	0.899190351687411	0.994905731618374	0.526770295231795\\
0.285	0.913790122539012	0.996276998142666	0.521779807495103\\
0.285	0.928330666930242	0.997428446906011	0.516780234308687\\
0.285	0.942807704312699	0.998363181068902	0.511773518435872\\
0.285	0.95721709138901	0.999084392196567	0.506761566080164\\
0.285	0.971554823322703	0.999595354092743	0.501746245611873\\
0.285	0.985817034663989	0.999899416688637	0.496729386403005\\
0.285	1	1	0.491669087346904\\
0.3	0	0	0.710195569525497\\
0.3	0.000100583311362513	0.0141829653360114	0.71240460418299\\
0.3	0.000404645907256436	0.0284451766772965	0.714630690536905\\
0.3	0.000915607803433	0.0427829086109896	0.716916312064106\\
0.3	0.00163681893109844	0.057192295687301	0.719289019360234\\
0.3	0.00257155309398959	0.0716693330697585	0.721780367061016\\
0.3	0.00372300185733414	0.086209877460988	0.724425655492835\\
0.3	0.00509426838162598	0.100809648312589	0.727263413850304\\
0.3	0.00668836121491816	0.115464229327074	0.73033458997498\\
0.3	0.00850818805808555	0.13016907025918	0.733681422514709\\
0.3	0.0105565495182326	0.144919489023162	0.737345986077549\\
0.3	0.0128361328661109	0.159710674111862	0.741368418705642\\
0.3	0.0153495058140643	0.174537687332543	0.745784862958202\\
0.3	0.0180991103316243	0.189395466863524	0.750625176092311\\
0.3	0.0210872565164405	0.204278830634725	0.755910489868998\\
0.3	0.0243161165387281	0.219182480034174	0.761650724664453\\
0.3	0.0277877186778607	0.234101003941464	0.767842183867616\\
0.3	0.0315039414701067	0.24902888308801	0.77446537092041\\
0.3	0.0354665079868145	0.263960494742775	0.781483180780606\\
0.3	0.0396769802625738	0.278890117720924	0.788839618268074\\
0.3	0.0441367538930258	0.293811937711588	0.796459186325544\\
0.3	0.0488470528220538	0.308720052919643	0.804247066923455\\
0.3	0.0538089243380495	0.323608480015096	0.81209018616195\\
0.3	0.0590232342988274	0.338471160382329	0.819859213932693\\
0.3	0.064490662604533	0.3533019666601	0.827411499063639\\
0.3	0.0702116989375697	0.368094709561873	0.834594885813504\\
0.3	0.0761866387881432	0.382843144964686	0.841252300287179\\
0.3	0.0824155797834956	0.397540981253432	0.847226939729081\\
0.3	0.0888984183382709	0.412181886906127	0.852367847905041\\
0.3	0.0956348466427212	0.42675949830445	0.85653562002783\\
0.3	0.102624350004627	0.441267427752584	0.859607954619619\\
0.3	0.109866204559871	0.455699271686219	0.861484760266624\\
0.3	0.117359475365564	0.470048619052377	0.862092534251539\\
0.3	0.125103014888515	0.484309059839721	0.861387758068367\\
0.3	0.133095461900593	0.498474193737904	0.859359100890821\\
0.3	0.14133524079124	0.512537638903683	0.856028283765737\\
0.3	0.149820561306021	0.526493040810599	0.851449530870183\\
0.3	0.158549418718625	0.540334081158348	0.845707614739504\\
0.3	0.167519594442214	0.554054486817265	0.83891458434315\\
0.3	0.176728657084455	0.567648038782867	0.831205342370834\\
0.3	0.186173963948925	0.581108581114919	0.822732305417527\\
0.3	0.195852662983903	0.594430029835237	0.813659432944307\\
0.3	0.205761695177907	0.607606381758231	0.804155944104867\\
0.3	0.215897797399558	0.620631723228143	0.794390053406971\\
0.3	0.226257505677663	0.633500238737009	0.7845230460628\\
0.3	0.236837158915675	0.646206219397571	0.774703982861587\\
0.3	0.247632903032949	0.658744071245709	0.765065275211017\\
0.3	0.258640695523528	0.671108323347402	0.755719307772963\\
0.3	0.269856310421543	0.683293635685828	0.746756214005915\\
0.3	0.28127534366066	0.695294806804925	0.738242834604297\\
0.3	0.292893218813452	0.707106781186547	0.730222816007836\\
0.3	0.304705193195075	0.71872465633934	0.722717741087768\\
0.3	0.316706364314172	0.730143689578457	0.715729131143265\\
0.3	0.328891676652598	0.741359304476472	0.709241120561857\\
0.3	0.341255928754291	0.752367096967051	0.703223584571265\\
0.3	0.353793780602429	0.763162841084325	0.697635496597417\\
0.3	0.366499761262991	0.773742494322337	0.692428303594618\\
0.3	0.379368276771857	0.784102202600442	0.687549132887069\\
0.3	0.392393618241769	0.794238304822092	0.68294367923533\\
0.3	0.405569970164763	0.804147337016097	0.678558662193046\\
0.3	0.418891418885081	0.813826036051075	0.674343787412871\\
0.3	0.432351961217133	0.823271342915545	0.670253187715551\\
0.3	0.445945513182735	0.832480405557787	0.666246357336157\\
0.3	0.459665918841652	0.841450581281375	0.662288623474184\\
0.3	0.473506959189401	0.850179438693979	0.658351221673191\\
0.3	0.487462361096317	0.85866475920876	0.654411055137886\\
0.3	0.501525806262096	0.866904538099407	0.650450223206469\\
0.3	0.51569094016028	0.874896985111485	0.646455401874959\\
0.3	0.529951380947623	0.882640524634437	0.642417151055154\\
0.3	0.544300728313782	0.890133795440129	0.638329210947585\\
0.3	0.558732572247415	0.897375649995373	0.634187835387499\\
0.3	0.57324050169555	0.904365153357279	0.629991194998376\\
0.3	0.587818113093873	0.911101581661729	0.625738868900924\\
0.3	0.602459018746568	0.917584420216505	0.621431431633141\\
0.3	0.617156855035314	0.923813361211857	0.617070132480551\\
0.3	0.631905290438127	0.92978830106243	0.612656657833838\\
0.3	0.6466980333399	0.935509337395467	0.608192963377089\\
0.3	0.661528839617671	0.940976765701173	0.603681161495104\\
0.3	0.676391519984904	0.94619107566195	0.599123449738607\\
0.3	0.691279947080357	0.951152947177946	0.594522067901593\\
0.3	0.706188062288412	0.955863246106974	0.589879273662389\\
0.3	0.721109882279076	0.960323019737426	0.585197329319934\\
0.3	0.736039505257225	0.964533492013185	0.580478494543975\\
0.3	0.75097111691199	0.968496058529893	0.575725022018313\\
0.3	0.765898996058536	0.972212281322139	0.570939154289562\\
0.3	0.780817519965826	0.975683883461272	0.566123121053176\\
0.3	0.795721169365275	0.97891274348356	0.561279136605558\\
0.3	0.810604533136476	0.981900889668376	0.556409397399196\\
0.3	0.825462312667457	0.984650494185936	0.551516079695735\\
0.3	0.840289325888138	0.987163867133889	0.546601337319642\\
0.3	0.855080510976839	0.989443450481768	0.541667299515139\\
0.3	0.86983092974082	0.991491811941914	0.536716068908597\\
0.3	0.884535770672926	0.993311638785082	0.53174971957814\\
0.3	0.899190351687411	0.994905731618374	0.526770295231796\\
0.3	0.913790122539012	0.996276998142666	0.521779807495104\\
0.3	0.928330666930242	0.997428446906011	0.516780234308682\\
0.3	0.942807704312699	0.998363181068902	0.511773518435872\\
0.3	0.95721709138901	0.999084392196567	0.50676156608017\\
0.3	0.971554823322704	0.999595354092744	0.501746245611874\\
0.3	0.985817034663989	0.999899416688637	0.496729386403001\\
0.3	1	1	0.491669087346908\\
0.315	0	0	0.710195569525497\\
0.315	0.000100583311362513	0.0141829653360114	0.71240460418299\\
0.315	0.000404645907256436	0.0284451766772965	0.714630690536905\\
0.315	0.000915607803433	0.0427829086109896	0.716916312064106\\
0.315	0.00163681893109844	0.057192295687301	0.719289019360234\\
0.315	0.00257155309398959	0.0716693330697584	0.721780367061016\\
0.315	0.00372300185733414	0.086209877460988	0.724425655492835\\
0.315	0.00509426838162598	0.100809648312589	0.727263413850304\\
0.315	0.00668836121491816	0.115464229327074	0.73033458997498\\
0.315	0.00850818805808555	0.13016907025918	0.733681422514709\\
0.315	0.0105565495182326	0.144919489023162	0.737345986077549\\
0.315	0.0128361328661109	0.159710674111862	0.741368418705642\\
0.315	0.0153495058140643	0.174537687332543	0.745784862958202\\
0.315	0.0180991103316243	0.189395466863524	0.750625176092311\\
0.315	0.0210872565164405	0.204278830634725	0.755910489868998\\
0.315	0.0243161165387281	0.219182480034174	0.761650724664453\\
0.315	0.0277877186778607	0.234101003941464	0.767842183867616\\
0.315	0.0315039414701067	0.24902888308801	0.77446537092041\\
0.315	0.0354665079868145	0.263960494742775	0.781483180780606\\
0.315	0.0396769802625738	0.278890117720924	0.788839618268073\\
0.315	0.0441367538930258	0.293811937711588	0.796459186325544\\
0.315	0.0488470528220538	0.308720052919643	0.804247066923454\\
0.315	0.0538089243380495	0.323608480015096	0.81209018616195\\
0.315	0.0590232342988274	0.338471160382329	0.819859213932693\\
0.315	0.064490662604533	0.3533019666601	0.827411499063639\\
0.315	0.0702116989375697	0.368094709561873	0.834594885813504\\
0.315	0.0761866387881432	0.382843144964686	0.841252300287177\\
0.315	0.0824155797834956	0.397540981253432	0.847226939729081\\
0.315	0.0888984183382709	0.412181886906127	0.852367847905042\\
0.315	0.0956348466427212	0.42675949830445	0.85653562002783\\
0.315	0.102624350004627	0.441267427752584	0.859607954619618\\
0.315	0.109866204559871	0.455699271686219	0.861484760266621\\
0.315	0.117359475365564	0.470048619052377	0.862092534251539\\
0.315	0.125103014888515	0.484309059839721	0.861387758068368\\
0.315	0.133095461900593	0.498474193737904	0.85935910089082\\
0.315	0.14133524079124	0.512537638903683	0.856028283765734\\
0.315	0.149820561306021	0.526493040810599	0.851449530870182\\
0.315	0.158549418718625	0.540334081158348	0.845707614739502\\
0.315	0.167519594442214	0.554054486817265	0.838914584343151\\
0.315	0.176728657084455	0.567648038782867	0.831205342370839\\
0.315	0.186173963948925	0.581108581114919	0.822732305417527\\
0.315	0.195852662983903	0.594430029835236	0.813659432944309\\
0.315	0.205761695177907	0.607606381758231	0.804155944104868\\
0.315	0.215897797399558	0.620631723228143	0.794390053406971\\
0.315	0.226257505677663	0.633500238737009	0.784523046062798\\
0.315	0.236837158915675	0.646206219397571	0.774703982861589\\
0.315	0.247632903032949	0.658744071245709	0.765065275211016\\
0.315	0.258640695523528	0.671108323347402	0.755719307772964\\
0.315	0.269856310421543	0.683293635685828	0.746756214005915\\
0.315	0.28127534366066	0.695294806804925	0.738242834604294\\
0.315	0.292893218813452	0.707106781186547	0.730222816007836\\
0.315	0.304705193195075	0.71872465633934	0.722717741087766\\
0.315	0.316706364314172	0.730143689578457	0.715729131143265\\
0.315	0.328891676652598	0.741359304476472	0.709241120561856\\
0.315	0.341255928754291	0.752367096967051	0.703223584571266\\
0.315	0.353793780602429	0.763162841084325	0.697635496597418\\
0.315	0.366499761262991	0.773742494322337	0.692428303594617\\
0.315	0.379368276771857	0.784102202600442	0.68754913288707\\
0.315	0.392393618241769	0.794238304822092	0.68294367923533\\
0.315	0.405569970164763	0.804147337016097	0.678558662193046\\
0.315	0.418891418885081	0.813826036051075	0.674343787412872\\
0.315	0.432351961217133	0.823271342915545	0.670253187715552\\
0.315	0.445945513182735	0.832480405557787	0.666246357336158\\
0.315	0.459665918841652	0.841450581281375	0.662288623474185\\
0.315	0.473506959189401	0.850179438693979	0.658351221673192\\
0.315	0.487462361096317	0.85866475920876	0.654411055137886\\
0.315	0.501525806262096	0.866904538099407	0.650450223206467\\
0.315	0.515690940160279	0.874896985111485	0.64645540187496\\
0.315	0.529951380947623	0.882640524634437	0.642417151055155\\
0.315	0.544300728313782	0.890133795440129	0.638329210947585\\
0.315	0.558732572247415	0.897375649995373	0.634187835387498\\
0.315	0.57324050169555	0.904365153357279	0.629991194998376\\
0.315	0.587818113093873	0.911101581661729	0.625738868900924\\
0.315	0.602459018746568	0.917584420216504	0.62143143163314\\
0.315	0.617156855035314	0.923813361211857	0.617070132480554\\
0.315	0.631905290438127	0.92978830106243	0.612656657833838\\
0.315	0.6466980333399	0.935509337395467	0.608192963377088\\
0.315	0.661528839617671	0.940976765701173	0.603681161495105\\
0.315	0.676391519984904	0.946191075661951	0.599123449738606\\
0.315	0.691279947080357	0.951152947177946	0.59452206790159\\
0.315	0.706188062288412	0.955863246106974	0.589879273662389\\
0.315	0.721109882279076	0.960323019737426	0.585197329319936\\
0.315	0.736039505257225	0.964533492013186	0.580478494543977\\
0.315	0.75097111691199	0.968496058529893	0.575725022018313\\
0.315	0.765898996058536	0.972212281322139	0.570939154289557\\
0.315	0.780817519965826	0.975683883461272	0.566123121053176\\
0.315	0.795721169365275	0.97891274348356	0.561279136605558\\
0.315	0.810604533136476	0.981900889668376	0.556409397399197\\
0.315	0.825462312667457	0.984650494185936	0.551516079695736\\
0.315	0.840289325888138	0.987163867133889	0.546601337319642\\
0.315	0.855080510976839	0.989443450481768	0.541667299515141\\
0.315	0.86983092974082	0.991491811941915	0.536716068908599\\
0.315	0.884535770672926	0.993311638785082	0.53174971957814\\
0.315	0.899190351687411	0.994905731618374	0.526770295231794\\
0.315	0.913790122539012	0.996276998142666	0.521779807495105\\
0.315	0.928330666930242	0.997428446906011	0.516780234308684\\
0.315	0.942807704312699	0.998363181068902	0.511773518435872\\
0.315	0.95721709138901	0.999084392196567	0.506761566080166\\
0.315	0.971554823322703	0.999595354092743	0.501746245611877\\
0.315	0.985817034663989	0.999899416688638	0.496729386403005\\
0.315	1	1	0.491669087346895\\
0.33	0	0	0.710195569525497\\
0.33	0.000100583311362513	0.0141829653360114	0.71240460418299\\
0.33	0.000404645907256436	0.0284451766772965	0.714630690536905\\
0.33	0.000915607803433	0.0427829086109896	0.716916312064106\\
0.33	0.00163681893109844	0.057192295687301	0.719289019360234\\
0.33	0.00257155309398959	0.0716693330697585	0.721780367061016\\
0.33	0.00372300185733414	0.086209877460988	0.724425655492835\\
0.33	0.00509426838162598	0.100809648312589	0.727263413850304\\
0.33	0.00668836121491816	0.115464229327074	0.73033458997498\\
0.33	0.00850818805808555	0.13016907025918	0.733681422514709\\
0.33	0.0105565495182326	0.144919489023162	0.737345986077549\\
0.33	0.0128361328661109	0.159710674111862	0.741368418705642\\
0.33	0.0153495058140643	0.174537687332543	0.745784862958202\\
0.33	0.0180991103316243	0.189395466863524	0.750625176092311\\
0.33	0.0210872565164405	0.204278830634725	0.755910489868998\\
0.33	0.0243161165387281	0.219182480034174	0.761650724664453\\
0.33	0.0277877186778607	0.234101003941464	0.767842183867616\\
0.33	0.0315039414701067	0.24902888308801	0.77446537092041\\
0.33	0.0354665079868145	0.263960494742775	0.781483180780606\\
0.33	0.0396769802625738	0.278890117720924	0.788839618268073\\
0.33	0.0441367538930258	0.293811937711588	0.796459186325544\\
0.33	0.0488470528220538	0.308720052919643	0.804247066923455\\
0.33	0.0538089243380495	0.323608480015096	0.81209018616195\\
0.33	0.0590232342988274	0.338471160382329	0.819859213932693\\
0.33	0.064490662604533	0.3533019666601	0.827411499063639\\
0.33	0.0702116989375697	0.368094709561873	0.834594885813505\\
0.33	0.0761866387881432	0.382843144964686	0.841252300287179\\
0.33	0.0824155797834956	0.397540981253432	0.847226939729081\\
0.33	0.0888984183382709	0.412181886906127	0.852367847905042\\
0.33	0.0956348466427213	0.42675949830445	0.856535620027828\\
0.33	0.102624350004627	0.441267427752584	0.859607954619617\\
0.33	0.109866204559871	0.455699271686219	0.861484760266618\\
0.33	0.117359475365564	0.470048619052377	0.86209253425154\\
0.33	0.125103014888515	0.484309059839721	0.86138775806837\\
0.33	0.133095461900593	0.498474193737904	0.859359100890818\\
0.33	0.14133524079124	0.512537638903683	0.856028283765736\\
0.33	0.149820561306021	0.526493040810599	0.851449530870182\\
0.33	0.158549418718625	0.540334081158348	0.845707614739505\\
0.33	0.167519594442214	0.554054486817265	0.838914584343152\\
0.33	0.176728657084455	0.567648038782868	0.831205342370836\\
0.33	0.186173963948925	0.581108581114919	0.822732305417528\\
0.33	0.195852662983903	0.594430029835237	0.81365943294431\\
0.33	0.205761695177907	0.607606381758231	0.804155944104868\\
0.33	0.215897797399558	0.620631723228143	0.79439005340697\\
0.33	0.226257505677663	0.633500238737009	0.7845230460628\\
0.33	0.236837158915675	0.646206219397571	0.774703982861589\\
0.33	0.247632903032949	0.658744071245709	0.765065275211016\\
0.33	0.258640695523528	0.671108323347402	0.755719307772964\\
0.33	0.269856310421543	0.683293635685828	0.746756214005913\\
0.33	0.28127534366066	0.695294806804924	0.738242834604294\\
0.33	0.292893218813452	0.707106781186547	0.730222816007835\\
0.33	0.304705193195075	0.71872465633934	0.722717741087767\\
0.33	0.316706364314172	0.730143689578457	0.715729131143267\\
0.33	0.328891676652598	0.741359304476472	0.709241120561857\\
0.33	0.341255928754291	0.752367096967051	0.703223584571265\\
0.33	0.353793780602429	0.763162841084325	0.697635496597418\\
0.33	0.366499761262991	0.773742494322337	0.692428303594618\\
0.33	0.379368276771857	0.784102202600442	0.687549132887069\\
0.33	0.392393618241769	0.794238304822092	0.682943679235329\\
0.33	0.405569970164763	0.804147337016097	0.678558662193047\\
0.33	0.418891418885081	0.813826036051075	0.674343787412872\\
0.33	0.432351961217133	0.823271342915545	0.670253187715551\\
0.33	0.445945513182735	0.832480405557787	0.666246357336157\\
0.33	0.459665918841652	0.841450581281375	0.662288623474184\\
0.33	0.473506959189401	0.850179438693979	0.658351221673192\\
0.33	0.487462361096317	0.85866475920876	0.654411055137887\\
0.33	0.501525806262096	0.866904538099407	0.650450223206468\\
0.33	0.515690940160279	0.874896985111485	0.646455401874959\\
0.33	0.529951380947623	0.882640524634437	0.642417151055156\\
0.33	0.544300728313782	0.890133795440129	0.638329210947585\\
0.33	0.558732572247415	0.897375649995373	0.634187835387498\\
0.33	0.57324050169555	0.904365153357279	0.629991194998376\\
0.33	0.587818113093873	0.911101581661729	0.625738868900924\\
0.33	0.602459018746568	0.917584420216504	0.62143143163314\\
0.33	0.617156855035314	0.923813361211857	0.617070132480554\\
0.33	0.631905290438127	0.92978830106243	0.612656657833838\\
0.33	0.6466980333399	0.935509337395467	0.608192963377087\\
0.33	0.661528839617671	0.940976765701173	0.603681161495105\\
0.33	0.676391519984904	0.946191075661951	0.599123449738607\\
0.33	0.691279947080357	0.951152947177946	0.59452206790159\\
0.33	0.706188062288412	0.955863246106974	0.589879273662389\\
0.33	0.721109882279076	0.960323019737426	0.585197329319936\\
0.33	0.736039505257225	0.964533492013186	0.580478494543975\\
0.33	0.75097111691199	0.968496058529893	0.575725022018315\\
0.33	0.765898996058536	0.972212281322139	0.57093915428956\\
0.33	0.780817519965826	0.975683883461272	0.566123121053175\\
0.33	0.795721169365275	0.97891274348356	0.561279136605558\\
0.33	0.810604533136476	0.981900889668376	0.556409397399196\\
0.33	0.825462312667457	0.984650494185936	0.551516079695735\\
0.33	0.840289325888138	0.987163867133889	0.546601337319642\\
0.33	0.855080510976839	0.989443450481768	0.541667299515139\\
0.33	0.86983092974082	0.991491811941914	0.536716068908595\\
0.33	0.884535770672926	0.993311638785082	0.531749719578141\\
0.33	0.899190351687411	0.994905731618374	0.526770295231798\\
0.33	0.913790122539012	0.996276998142666	0.521779807495105\\
0.33	0.928330666930242	0.997428446906011	0.516780234308684\\
0.33	0.942807704312699	0.998363181068902	0.511773518435873\\
0.33	0.95721709138901	0.999084392196567	0.506761566080168\\
0.33	0.971554823322703	0.999595354092743	0.501746245611874\\
0.33	0.985817034663989	0.999899416688637	0.496729386403004\\
0.33	1	1	0.491669087346906\\
0.345	0	0	0.710195569525497\\
0.345	0.000100583311362513	0.0141829653360114	0.71240460418299\\
0.345	0.000404645907256436	0.0284451766772965	0.714630690536905\\
0.345	0.000915607803433	0.0427829086109896	0.716916312064106\\
0.345	0.00163681893109844	0.057192295687301	0.719289019360234\\
0.345	0.00257155309398959	0.0716693330697584	0.721780367061016\\
0.345	0.00372300185733414	0.0862098774609879	0.724425655492835\\
0.345	0.00509426838162598	0.100809648312589	0.727263413850304\\
0.345	0.00668836121491816	0.115464229327074	0.73033458997498\\
0.345	0.00850818805808555	0.13016907025918	0.733681422514709\\
0.345	0.0105565495182326	0.144919489023162	0.737345986077549\\
0.345	0.0128361328661109	0.159710674111862	0.741368418705642\\
0.345	0.0153495058140643	0.174537687332543	0.745784862958202\\
0.345	0.0180991103316243	0.189395466863524	0.750625176092311\\
0.345	0.0210872565164405	0.204278830634725	0.755910489868998\\
0.345	0.0243161165387281	0.219182480034174	0.761650724664453\\
0.345	0.0277877186778607	0.234101003941464	0.767842183867616\\
0.345	0.0315039414701067	0.24902888308801	0.77446537092041\\
0.345	0.0354665079868145	0.263960494742775	0.781483180780606\\
0.345	0.0396769802625738	0.278890117720924	0.788839618268073\\
0.345	0.0441367538930258	0.293811937711588	0.796459186325544\\
0.345	0.0488470528220538	0.308720052919643	0.804247066923455\\
0.345	0.0538089243380495	0.323608480015096	0.81209018616195\\
0.345	0.0590232342988274	0.338471160382329	0.819859213932694\\
0.345	0.064490662604533	0.3533019666601	0.827411499063639\\
0.345	0.0702116989375697	0.368094709561873	0.834594885813504\\
0.345	0.0761866387881432	0.382843144964686	0.841252300287181\\
0.345	0.0824155797834956	0.397540981253432	0.847226939729081\\
0.345	0.0888984183382709	0.412181886906127	0.85236784790504\\
0.345	0.0956348466427212	0.42675949830445	0.85653562002783\\
0.345	0.102624350004627	0.441267427752584	0.859607954619617\\
0.345	0.109866204559871	0.455699271686218	0.861484760266622\\
0.345	0.117359475365564	0.470048619052377	0.86209253425154\\
0.345	0.125103014888515	0.484309059839721	0.86138775806837\\
0.345	0.133095461900593	0.498474193737904	0.85935910089082\\
0.345	0.14133524079124	0.512537638903683	0.856028283765738\\
0.345	0.149820561306021	0.526493040810599	0.851449530870183\\
0.345	0.158549418718625	0.540334081158348	0.845707614739508\\
0.345	0.167519594442213	0.554054486817265	0.83891458434315\\
0.345	0.176728657084455	0.567648038782867	0.831205342370834\\
0.345	0.186173963948925	0.581108581114919	0.822732305417529\\
0.345	0.195852662983903	0.594430029835237	0.813659432944308\\
0.345	0.205761695177907	0.607606381758231	0.804155944104868\\
0.345	0.215897797399558	0.620631723228143	0.79439005340697\\
0.345	0.226257505677663	0.633500238737009	0.7845230460628\\
0.345	0.236837158915675	0.646206219397571	0.774703982861588\\
0.345	0.247632903032949	0.658744071245709	0.765065275211016\\
0.345	0.258640695523528	0.671108323347402	0.755719307772965\\
0.345	0.269856310421543	0.683293635685828	0.746756214005914\\
0.345	0.28127534366066	0.695294806804925	0.738242834604295\\
0.345	0.292893218813452	0.707106781186547	0.730222816007836\\
0.345	0.304705193195075	0.71872465633934	0.722717741087767\\
0.345	0.316706364314172	0.730143689578457	0.715729131143265\\
0.345	0.328891676652598	0.741359304476472	0.709241120561857\\
0.345	0.341255928754291	0.752367096967051	0.703223584571265\\
0.345	0.353793780602429	0.763162841084325	0.697635496597417\\
0.345	0.366499761262991	0.773742494322337	0.692428303594618\\
0.345	0.379368276771857	0.784102202600442	0.687549132887069\\
0.345	0.392393618241769	0.794238304822092	0.682943679235329\\
0.345	0.405569970164763	0.804147337016097	0.678558662193046\\
0.345	0.418891418885081	0.813826036051075	0.674343787412872\\
0.345	0.432351961217133	0.823271342915545	0.670253187715552\\
0.345	0.445945513182735	0.832480405557787	0.666246357336159\\
0.345	0.459665918841652	0.841450581281375	0.662288623474185\\
0.345	0.473506959189401	0.850179438693979	0.658351221673189\\
0.345	0.487462361096317	0.85866475920876	0.654411055137886\\
0.345	0.501525806262096	0.866904538099407	0.650450223206468\\
0.345	0.515690940160279	0.874896985111485	0.646455401874959\\
0.345	0.529951380947623	0.882640524634437	0.642417151055155\\
0.345	0.544300728313782	0.890133795440129	0.638329210947586\\
0.345	0.558732572247415	0.897375649995373	0.634187835387498\\
0.345	0.57324050169555	0.904365153357279	0.629991194998376\\
0.345	0.587818113093873	0.911101581661729	0.625738868900924\\
0.345	0.602459018746568	0.917584420216504	0.62143143163314\\
0.345	0.617156855035314	0.923813361211857	0.617070132480554\\
0.345	0.631905290438127	0.92978830106243	0.612656657833838\\
0.345	0.6466980333399	0.935509337395467	0.608192963377087\\
0.345	0.661528839617671	0.940976765701173	0.603681161495105\\
0.345	0.676391519984904	0.946191075661951	0.599123449738608\\
0.345	0.691279947080357	0.951152947177946	0.59452206790159\\
0.345	0.706188062288412	0.955863246106974	0.589879273662389\\
0.345	0.721109882279076	0.960323019737426	0.585197329319936\\
0.345	0.736039505257225	0.964533492013186	0.580478494543978\\
0.345	0.75097111691199	0.968496058529893	0.575725022018314\\
0.345	0.765898996058536	0.972212281322139	0.570939154289558\\
0.345	0.780817519965826	0.975683883461272	0.566123121053173\\
0.345	0.795721169365275	0.97891274348356	0.561279136605559\\
0.345	0.810604533136476	0.981900889668376	0.556409397399198\\
0.345	0.825462312667457	0.984650494185936	0.551516079695736\\
0.345	0.840289325888138	0.987163867133889	0.546601337319641\\
0.345	0.855080510976839	0.989443450481768	0.541667299515141\\
0.345	0.86983092974082	0.991491811941915	0.536716068908598\\
0.345	0.884535770672926	0.993311638785082	0.531749719578138\\
0.345	0.899190351687411	0.994905731618374	0.526770295231795\\
0.345	0.913790122539012	0.996276998142666	0.521779807495104\\
0.345	0.928330666930242	0.997428446906011	0.516780234308683\\
0.345	0.942807704312699	0.998363181068902	0.511773518435872\\
0.345	0.95721709138901	0.999084392196567	0.506761566080168\\
0.345	0.971554823322703	0.999595354092743	0.501746245611874\\
0.345	0.985817034663989	0.999899416688637	0.496729386403005\\
0.345	1	1	0.491669087346904\\
0.36	0	0	0.710195569525497\\
0.36	0.000100583311362513	0.0141829653360114	0.71240460418299\\
0.36	0.000404645907256436	0.0284451766772965	0.714630690536905\\
0.36	0.000915607803433	0.0427829086109896	0.716916312064106\\
0.36	0.00163681893109844	0.057192295687301	0.719289019360234\\
0.36	0.00257155309398959	0.0716693330697585	0.721780367061016\\
0.36	0.00372300185733414	0.086209877460988	0.724425655492835\\
0.36	0.00509426838162598	0.100809648312589	0.727263413850304\\
0.36	0.00668836121491816	0.115464229327074	0.73033458997498\\
0.36	0.00850818805808555	0.13016907025918	0.733681422514709\\
0.36	0.0105565495182326	0.144919489023162	0.737345986077549\\
0.36	0.0128361328661109	0.159710674111862	0.741368418705642\\
0.36	0.0153495058140643	0.174537687332543	0.745784862958202\\
0.36	0.0180991103316243	0.189395466863524	0.750625176092311\\
0.36	0.0210872565164405	0.204278830634725	0.755910489868998\\
0.36	0.0243161165387281	0.219182480034174	0.761650724664453\\
0.36	0.0277877186778607	0.234101003941464	0.767842183867616\\
0.36	0.0315039414701067	0.24902888308801	0.77446537092041\\
0.36	0.0354665079868145	0.263960494742775	0.781483180780606\\
0.36	0.0396769802625738	0.278890117720924	0.788839618268073\\
0.36	0.0441367538930258	0.293811937711588	0.796459186325544\\
0.36	0.0488470528220538	0.308720052919643	0.804247066923454\\
0.36	0.0538089243380495	0.323608480015096	0.81209018616195\\
0.36	0.0590232342988274	0.338471160382329	0.819859213932694\\
0.36	0.064490662604533	0.3533019666601	0.827411499063639\\
0.36	0.0702116989375697	0.368094709561873	0.834594885813504\\
0.36	0.0761866387881432	0.382843144964686	0.841252300287179\\
0.36	0.0824155797834956	0.397540981253432	0.847226939729081\\
0.36	0.0888984183382709	0.412181886906127	0.85236784790504\\
0.36	0.0956348466427212	0.42675949830445	0.85653562002783\\
0.36	0.102624350004627	0.441267427752584	0.859607954619617\\
0.36	0.109866204559871	0.455699271686219	0.861484760266623\\
0.36	0.117359475365564	0.470048619052377	0.862092534251541\\
0.36	0.125103014888515	0.484309059839721	0.861387758068367\\
0.36	0.133095461900593	0.498474193737904	0.85935910089082\\
0.36	0.14133524079124	0.512537638903683	0.856028283765736\\
0.36	0.149820561306021	0.526493040810599	0.851449530870183\\
0.36	0.158549418718625	0.540334081158348	0.845707614739505\\
0.36	0.167519594442214	0.554054486817265	0.83891458434315\\
0.36	0.176728657084455	0.567648038782867	0.831205342370836\\
0.36	0.186173963948925	0.581108581114919	0.822732305417527\\
0.36	0.195852662983903	0.594430029835237	0.813659432944309\\
0.36	0.205761695177907	0.607606381758231	0.804155944104867\\
0.36	0.215897797399558	0.620631723228143	0.79439005340697\\
0.36	0.226257505677663	0.633500238737009	0.7845230460628\\
0.36	0.236837158915675	0.646206219397571	0.774703982861589\\
0.36	0.247632903032949	0.658744071245709	0.765065275211016\\
0.36	0.258640695523528	0.671108323347402	0.755719307772964\\
0.36	0.269856310421543	0.683293635685828	0.746756214005914\\
0.36	0.28127534366066	0.695294806804925	0.738242834604294\\
0.36	0.292893218813452	0.707106781186547	0.730222816007836\\
0.36	0.304705193195075	0.71872465633934	0.722717741087767\\
0.36	0.316706364314172	0.730143689578457	0.715729131143265\\
0.36	0.328891676652598	0.741359304476472	0.709241120561857\\
0.36	0.341255928754291	0.752367096967051	0.703223584571265\\
0.36	0.353793780602429	0.763162841084325	0.697635496597417\\
0.36	0.366499761262991	0.773742494322337	0.692428303594618\\
0.36	0.379368276771857	0.784102202600442	0.687549132887069\\
0.36	0.392393618241769	0.794238304822092	0.682943679235329\\
0.36	0.405569970164763	0.804147337016097	0.678558662193046\\
0.36	0.418891418885081	0.813826036051075	0.674343787412872\\
0.36	0.432351961217133	0.823271342915545	0.670253187715551\\
0.36	0.445945513182735	0.832480405557787	0.666246357336156\\
0.36	0.459665918841652	0.841450581281375	0.662288623474185\\
0.36	0.473506959189401	0.850179438693979	0.658351221673193\\
0.36	0.487462361096317	0.85866475920876	0.654411055137886\\
0.36	0.501525806262096	0.866904538099407	0.650450223206467\\
0.36	0.515690940160279	0.874896985111485	0.646455401874959\\
0.36	0.529951380947623	0.882640524634437	0.642417151055155\\
0.36	0.544300728313782	0.890133795440129	0.638329210947585\\
0.36	0.558732572247415	0.897375649995373	0.634187835387499\\
0.36	0.57324050169555	0.904365153357279	0.629991194998376\\
0.36	0.587818113093873	0.911101581661729	0.625738868900923\\
0.36	0.602459018746568	0.917584420216504	0.62143143163314\\
0.36	0.617156855035314	0.923813361211857	0.617070132480554\\
0.36	0.631905290438127	0.92978830106243	0.612656657833838\\
0.36	0.6466980333399	0.935509337395467	0.608192963377087\\
0.36	0.661528839617671	0.940976765701173	0.603681161495104\\
0.36	0.676391519984904	0.94619107566195	0.599123449738607\\
0.36	0.691279947080357	0.951152947177946	0.594522067901592\\
0.36	0.706188062288412	0.955863246106974	0.589879273662389\\
0.36	0.721109882279076	0.960323019737426	0.585197329319936\\
0.36	0.736039505257225	0.964533492013186	0.580478494543978\\
0.36	0.75097111691199	0.968496058529893	0.575725022018313\\
0.36	0.765898996058536	0.972212281322139	0.570939154289558\\
0.36	0.780817519965826	0.975683883461272	0.566123121053175\\
0.36	0.795721169365275	0.97891274348356	0.561279136605558\\
0.36	0.810604533136476	0.981900889668376	0.556409397399198\\
0.36	0.825462312667457	0.984650494185936	0.551516079695736\\
0.36	0.840289325888138	0.987163867133889	0.546601337319641\\
0.36	0.855080510976839	0.989443450481768	0.541667299515139\\
0.36	0.86983092974082	0.991491811941914	0.5367160689086\\
0.36	0.884535770672926	0.993311638785082	0.531749719578141\\
0.36	0.899190351687411	0.994905731618374	0.526770295231796\\
0.36	0.913790122539012	0.996276998142666	0.521779807495104\\
0.36	0.928330666930242	0.997428446906011	0.516780234308682\\
0.36	0.942807704312699	0.998363181068902	0.511773518435871\\
0.36	0.95721709138901	0.999084392196567	0.506761566080168\\
0.36	0.971554823322703	0.999595354092743	0.501746245611874\\
0.36	0.985817034663989	0.999899416688637	0.496729386403004\\
0.36	1	1	0.491669087346906\\
0.375	0	0	0.710195569525497\\
0.375	0.000100583311362513	0.0141829653360114	0.71240460418299\\
0.375	0.000404645907256436	0.0284451766772965	0.714630690536905\\
0.375	0.000915607803433	0.0427829086109896	0.716916312064106\\
0.375	0.00163681893109844	0.057192295687301	0.719289019360234\\
0.375	0.00257155309398959	0.0716693330697585	0.721780367061016\\
0.375	0.00372300185733414	0.086209877460988	0.724425655492835\\
0.375	0.00509426838162598	0.100809648312589	0.727263413850304\\
0.375	0.00668836121491816	0.115464229327074	0.73033458997498\\
0.375	0.00850818805808555	0.13016907025918	0.733681422514709\\
0.375	0.0105565495182326	0.144919489023162	0.737345986077549\\
0.375	0.0128361328661109	0.159710674111862	0.741368418705642\\
0.375	0.0153495058140643	0.174537687332543	0.745784862958202\\
0.375	0.0180991103316243	0.189395466863524	0.750625176092311\\
0.375	0.0210872565164405	0.204278830634725	0.755910489868998\\
0.375	0.0243161165387281	0.219182480034174	0.761650724664453\\
0.375	0.0277877186778607	0.234101003941464	0.767842183867616\\
0.375	0.0315039414701067	0.24902888308801	0.77446537092041\\
0.375	0.0354665079868145	0.263960494742775	0.781483180780606\\
0.375	0.0396769802625738	0.278890117720924	0.788839618268073\\
0.375	0.0441367538930258	0.293811937711588	0.796459186325544\\
0.375	0.0488470528220538	0.308720052919643	0.804247066923454\\
0.375	0.0538089243380495	0.323608480015096	0.81209018616195\\
0.375	0.0590232342988274	0.338471160382329	0.819859213932694\\
0.375	0.064490662604533	0.3533019666601	0.82741149906364\\
0.375	0.0702116989375697	0.368094709561873	0.834594885813504\\
0.375	0.0761866387881432	0.382843144964686	0.84125230028718\\
0.375	0.0824155797834956	0.397540981253432	0.847226939729081\\
0.375	0.0888984183382709	0.412181886906127	0.85236784790504\\
0.375	0.0956348466427212	0.42675949830445	0.856535620027828\\
0.375	0.102624350004627	0.441267427752584	0.859607954619617\\
0.375	0.109866204559871	0.455699271686219	0.861484760266622\\
0.375	0.117359475365564	0.470048619052377	0.862092534251541\\
0.375	0.125103014888515	0.484309059839721	0.861387758068369\\
0.375	0.133095461900593	0.498474193737904	0.859359100890822\\
0.375	0.14133524079124	0.512537638903683	0.856028283765737\\
0.375	0.149820561306021	0.526493040810599	0.851449530870183\\
0.375	0.158549418718625	0.540334081158348	0.845707614739506\\
0.375	0.167519594442214	0.554054486817265	0.83891458434315\\
0.375	0.176728657084455	0.567648038782867	0.831205342370836\\
0.375	0.186173963948925	0.581108581114919	0.82273230541753\\
0.375	0.195852662983903	0.594430029835237	0.813659432944309\\
0.375	0.205761695177907	0.607606381758231	0.804155944104867\\
0.375	0.215897797399558	0.620631723228143	0.794390053406969\\
0.375	0.226257505677663	0.633500238737009	0.7845230460628\\
0.375	0.236837158915675	0.646206219397571	0.774703982861589\\
0.375	0.247632903032949	0.658744071245709	0.765065275211016\\
0.375	0.258640695523528	0.671108323347402	0.755719307772964\\
0.375	0.269856310421543	0.683293635685828	0.746756214005914\\
0.375	0.28127534366066	0.695294806804925	0.738242834604294\\
0.375	0.292893218813452	0.707106781186547	0.730222816007836\\
0.375	0.304705193195075	0.71872465633934	0.722717741087767\\
0.375	0.316706364314172	0.730143689578457	0.715729131143265\\
0.375	0.328891676652598	0.741359304476472	0.709241120561857\\
0.375	0.341255928754291	0.752367096967051	0.703223584571265\\
0.375	0.353793780602429	0.763162841084325	0.697635496597417\\
0.375	0.366499761262991	0.773742494322337	0.692428303594618\\
0.375	0.379368276771857	0.784102202600442	0.687549132887069\\
0.375	0.392393618241769	0.794238304822092	0.682943679235329\\
0.375	0.405569970164763	0.804147337016097	0.678558662193046\\
0.375	0.418891418885081	0.813826036051075	0.674343787412872\\
0.375	0.432351961217133	0.823271342915545	0.670253187715552\\
0.375	0.445945513182735	0.832480405557787	0.666246357336158\\
0.375	0.459665918841652	0.841450581281375	0.662288623474184\\
0.375	0.473506959189401	0.850179438693979	0.658351221673191\\
0.375	0.487462361096317	0.85866475920876	0.654411055137887\\
0.375	0.501525806262096	0.866904538099407	0.650450223206467\\
0.375	0.515690940160279	0.874896985111485	0.646455401874959\\
0.375	0.529951380947623	0.882640524634437	0.642417151055155\\
0.375	0.544300728313782	0.890133795440129	0.638329210947585\\
0.375	0.558732572247415	0.897375649995373	0.634187835387498\\
0.375	0.57324050169555	0.904365153357279	0.629991194998376\\
0.375	0.587818113093873	0.911101581661729	0.625738868900924\\
0.375	0.602459018746568	0.917584420216504	0.62143143163314\\
0.375	0.617156855035314	0.923813361211857	0.617070132480554\\
0.375	0.631905290438127	0.92978830106243	0.612656657833838\\
0.375	0.6466980333399	0.935509337395467	0.608192963377087\\
0.375	0.661528839617671	0.940976765701173	0.603681161495105\\
0.375	0.676391519984904	0.946191075661951	0.599123449738606\\
0.375	0.691279947080357	0.951152947177946	0.59452206790159\\
0.375	0.706188062288412	0.955863246106974	0.589879273662389\\
0.375	0.721109882279076	0.960323019737426	0.585197329319936\\
0.375	0.736039505257225	0.964533492013186	0.580478494543978\\
0.375	0.75097111691199	0.968496058529893	0.575725022018313\\
0.375	0.765898996058536	0.972212281322139	0.570939154289558\\
0.375	0.780817519965826	0.975683883461272	0.566123121053175\\
0.375	0.795721169365275	0.97891274348356	0.561279136605558\\
0.375	0.810604533136476	0.981900889668376	0.556409397399198\\
0.375	0.825462312667457	0.984650494185936	0.551516079695735\\
0.375	0.840289325888138	0.987163867133889	0.546601337319641\\
0.375	0.855080510976839	0.989443450481768	0.541667299515139\\
0.375	0.86983092974082	0.991491811941914	0.536716068908598\\
0.375	0.884535770672926	0.993311638785082	0.531749719578141\\
0.375	0.899190351687411	0.994905731618374	0.526770295231796\\
0.375	0.913790122539012	0.996276998142666	0.521779807495105\\
0.375	0.928330666930242	0.997428446906011	0.516780234308684\\
0.375	0.942807704312699	0.998363181068902	0.511773518435873\\
0.375	0.95721709138901	0.999084392196567	0.506761566080168\\
0.375	0.971554823322703	0.999595354092743	0.501746245611875\\
0.375	0.985817034663989	0.999899416688637	0.496729386403004\\
0.375	1	1	0.491669087346906\\
0.39	0	0	0.710195569525497\\
0.39	0.000100583311362513	0.0141829653360114	0.71240460418299\\
0.39	0.000404645907256436	0.0284451766772965	0.714630690536905\\
0.39	0.000915607803433	0.0427829086109896	0.716916312064106\\
0.39	0.00163681893109844	0.057192295687301	0.719289019360234\\
0.39	0.00257155309398959	0.0716693330697585	0.721780367061016\\
0.39	0.00372300185733414	0.086209877460988	0.724425655492835\\
0.39	0.00509426838162598	0.100809648312589	0.727263413850304\\
0.39	0.00668836121491816	0.115464229327074	0.73033458997498\\
0.39	0.00850818805808555	0.13016907025918	0.733681422514709\\
0.39	0.0105565495182326	0.144919489023162	0.737345986077549\\
0.39	0.0128361328661109	0.159710674111862	0.741368418705642\\
0.39	0.0153495058140643	0.174537687332543	0.745784862958202\\
0.39	0.0180991103316243	0.189395466863524	0.750625176092311\\
0.39	0.0210872565164405	0.204278830634725	0.755910489868998\\
0.39	0.0243161165387281	0.219182480034174	0.761650724664453\\
0.39	0.0277877186778607	0.234101003941464	0.767842183867616\\
0.39	0.0315039414701067	0.24902888308801	0.77446537092041\\
0.39	0.0354665079868145	0.263960494742775	0.781483180780606\\
0.39	0.0396769802625738	0.278890117720924	0.788839618268073\\
0.39	0.0441367538930258	0.293811937711588	0.796459186325544\\
0.39	0.0488470528220538	0.308720052919643	0.804247066923454\\
0.39	0.0538089243380495	0.323608480015097	0.81209018616195\\
0.39	0.0590232342988274	0.338471160382329	0.819859213932693\\
0.39	0.064490662604533	0.3533019666601	0.827411499063639\\
0.39	0.0702116989375697	0.368094709561873	0.834594885813505\\
0.39	0.0761866387881433	0.382843144964686	0.841252300287178\\
0.39	0.0824155797834956	0.397540981253432	0.847226939729081\\
0.39	0.0888984183382709	0.412181886906127	0.852367847905041\\
0.39	0.0956348466427212	0.42675949830445	0.856535620027829\\
0.39	0.102624350004627	0.441267427752584	0.859607954619617\\
0.39	0.109866204559871	0.455699271686219	0.861484760266622\\
0.39	0.117359475365564	0.470048619052377	0.862092534251539\\
0.39	0.125103014888515	0.484309059839721	0.861387758068368\\
0.39	0.133095461900593	0.498474193737904	0.859359100890821\\
0.39	0.14133524079124	0.512537638903683	0.856028283765735\\
0.39	0.149820561306021	0.526493040810599	0.851449530870183\\
0.39	0.158549418718625	0.540334081158348	0.845707614739504\\
0.39	0.167519594442214	0.554054486817265	0.83891458434315\\
0.39	0.176728657084455	0.567648038782867	0.831205342370837\\
0.39	0.186173963948925	0.581108581114919	0.822732305417528\\
0.39	0.195852662983903	0.594430029835237	0.813659432944308\\
0.39	0.205761695177907	0.607606381758231	0.804155944104868\\
0.39	0.215897797399558	0.620631723228143	0.794390053406969\\
0.39	0.226257505677663	0.633500238737009	0.7845230460628\\
0.39	0.236837158915675	0.646206219397571	0.774703982861589\\
0.39	0.247632903032949	0.658744071245709	0.765065275211016\\
0.39	0.258640695523528	0.671108323347402	0.755719307772964\\
0.39	0.269856310421543	0.683293635685828	0.746756214005914\\
0.39	0.28127534366066	0.695294806804925	0.738242834604294\\
0.39	0.292893218813452	0.707106781186547	0.730222816007835\\
0.39	0.304705193195076	0.71872465633934	0.722717741087767\\
0.39	0.316706364314172	0.730143689578457	0.715729131143266\\
0.39	0.328891676652598	0.741359304476472	0.709241120561857\\
0.39	0.341255928754291	0.752367096967051	0.703223584571265\\
0.39	0.353793780602429	0.763162841084325	0.697635496597417\\
0.39	0.366499761262991	0.773742494322337	0.692428303594618\\
0.39	0.379368276771857	0.784102202600442	0.687549132887069\\
0.39	0.392393618241769	0.794238304822092	0.682943679235329\\
0.39	0.405569970164763	0.804147337016097	0.678558662193046\\
0.39	0.418891418885081	0.813826036051075	0.674343787412872\\
0.39	0.432351961217133	0.823271342915545	0.670253187715552\\
0.39	0.445945513182735	0.832480405557787	0.666246357336158\\
0.39	0.459665918841652	0.841450581281375	0.662288623474184\\
0.39	0.473506959189401	0.850179438693979	0.658351221673191\\
0.39	0.487462361096317	0.85866475920876	0.654411055137886\\
0.39	0.501525806262096	0.866904538099407	0.650450223206468\\
0.39	0.515690940160279	0.874896985111485	0.646455401874959\\
0.39	0.529951380947623	0.882640524634437	0.642417151055155\\
0.39	0.544300728313782	0.890133795440129	0.638329210947586\\
0.39	0.558732572247415	0.897375649995373	0.634187835387499\\
0.39	0.57324050169555	0.904365153357279	0.629991194998376\\
0.39	0.587818113093873	0.911101581661729	0.625738868900924\\
0.39	0.602459018746568	0.917584420216504	0.62143143163314\\
0.39	0.617156855035314	0.923813361211857	0.617070132480554\\
0.39	0.631905290438127	0.92978830106243	0.612656657833838\\
0.39	0.6466980333399	0.935509337395467	0.608192963377087\\
0.39	0.661528839617671	0.940976765701173	0.603681161495105\\
0.39	0.676391519984904	0.946191075661951	0.599123449738607\\
0.39	0.691279947080357	0.951152947177946	0.59452206790159\\
0.39	0.706188062288412	0.955863246106974	0.589879273662389\\
0.39	0.721109882279076	0.960323019737426	0.585197329319936\\
0.39	0.736039505257225	0.964533492013186	0.580478494543978\\
0.39	0.75097111691199	0.968496058529893	0.575725022018312\\
0.39	0.765898996058536	0.972212281322139	0.570939154289558\\
0.39	0.780817519965826	0.975683883461272	0.566123121053177\\
0.39	0.795721169365275	0.97891274348356	0.561279136605558\\
0.39	0.810604533136476	0.981900889668376	0.556409397399199\\
0.39	0.825462312667457	0.984650494185936	0.551516079695736\\
0.39	0.840289325888138	0.987163867133889	0.546601337319639\\
0.39	0.855080510976839	0.989443450481768	0.541667299515139\\
0.39	0.86983092974082	0.991491811941914	0.536716068908597\\
0.39	0.884535770672926	0.993311638785082	0.531749719578142\\
0.39	0.899190351687411	0.994905731618374	0.526770295231796\\
0.39	0.913790122539012	0.996276998142666	0.521779807495103\\
0.39	0.928330666930242	0.997428446906011	0.516780234308682\\
0.39	0.942807704312699	0.998363181068902	0.511773518435872\\
0.39	0.95721709138901	0.999084392196567	0.506761566080168\\
0.39	0.971554823322703	0.999595354092743	0.501746245611874\\
0.39	0.985817034663989	0.999899416688637	0.496729386403005\\
0.39	1	1	0.491669087346904\\
0.405	0	0	0.710195569525497\\
0.405	0.000100583311362513	0.0141829653360114	0.71240460418299\\
0.405	0.000404645907256436	0.0284451766772965	0.714630690536905\\
0.405	0.000915607803433	0.0427829086109896	0.716916312064106\\
0.405	0.00163681893109844	0.057192295687301	0.719289019360234\\
0.405	0.00257155309398959	0.0716693330697584	0.721780367061016\\
0.405	0.00372300185733414	0.086209877460988	0.724425655492835\\
0.405	0.00509426838162598	0.100809648312589	0.727263413850304\\
0.405	0.00668836121491816	0.115464229327074	0.73033458997498\\
0.405	0.00850818805808555	0.13016907025918	0.733681422514709\\
0.405	0.0105565495182326	0.144919489023162	0.737345986077549\\
0.405	0.0128361328661109	0.159710674111862	0.741368418705642\\
0.405	0.0153495058140643	0.174537687332543	0.745784862958202\\
0.405	0.0180991103316243	0.189395466863524	0.750625176092311\\
0.405	0.0210872565164405	0.204278830634725	0.755910489868998\\
0.405	0.0243161165387281	0.219182480034174	0.761650724664453\\
0.405	0.0277877186778607	0.234101003941464	0.767842183867616\\
0.405	0.0315039414701067	0.24902888308801	0.77446537092041\\
0.405	0.0354665079868145	0.263960494742775	0.781483180780606\\
0.405	0.0396769802625737	0.278890117720924	0.788839618268073\\
0.405	0.0441367538930258	0.293811937711588	0.796459186325544\\
0.405	0.0488470528220538	0.308720052919643	0.804247066923454\\
0.405	0.0538089243380495	0.323608480015096	0.812090186161949\\
0.405	0.0590232342988274	0.338471160382329	0.819859213932693\\
0.405	0.064490662604533	0.3533019666601	0.827411499063638\\
0.405	0.0702116989375697	0.368094709561873	0.834594885813505\\
0.405	0.0761866387881432	0.382843144964686	0.841252300287178\\
0.405	0.0824155797834956	0.397540981253432	0.847226939729081\\
0.405	0.0888984183382709	0.412181886906127	0.852367847905042\\
0.405	0.0956348466427212	0.42675949830445	0.85653562002783\\
0.405	0.102624350004627	0.441267427752584	0.859607954619617\\
0.405	0.109866204559871	0.455699271686219	0.861484760266618\\
0.405	0.117359475365564	0.470048619052377	0.862092534251537\\
0.405	0.125103014888515	0.484309059839721	0.861387758068368\\
0.405	0.133095461900593	0.498474193737904	0.859359100890821\\
0.405	0.14133524079124	0.512537638903683	0.856028283765735\\
0.405	0.149820561306021	0.526493040810599	0.851449530870183\\
0.405	0.158549418718625	0.540334081158348	0.845707614739505\\
0.405	0.167519594442214	0.554054486817265	0.83891458434315\\
0.405	0.176728657084455	0.567648038782867	0.831205342370836\\
0.405	0.186173963948925	0.581108581114919	0.822732305417527\\
0.405	0.195852662983903	0.594430029835237	0.813659432944309\\
0.405	0.205761695177907	0.607606381758231	0.804155944104868\\
0.405	0.215897797399558	0.620631723228143	0.79439005340697\\
0.405	0.226257505677663	0.633500238737009	0.7845230460628\\
0.405	0.236837158915675	0.646206219397571	0.774703982861589\\
0.405	0.247632903032949	0.658744071245709	0.765065275211016\\
0.405	0.258640695523528	0.671108323347402	0.755719307772965\\
0.405	0.269856310421543	0.683293635685828	0.746756214005914\\
0.405	0.28127534366066	0.695294806804925	0.738242834604294\\
0.405	0.292893218813452	0.707106781186547	0.730222816007836\\
0.405	0.304705193195075	0.71872465633934	0.722717741087767\\
0.405	0.316706364314172	0.730143689578457	0.715729131143265\\
0.405	0.328891676652598	0.741359304476472	0.709241120561857\\
0.405	0.341255928754291	0.752367096967051	0.703223584571265\\
0.405	0.353793780602429	0.763162841084325	0.697635496597417\\
0.405	0.366499761262991	0.773742494322337	0.692428303594618\\
0.405	0.379368276771857	0.784102202600442	0.687549132887069\\
0.405	0.392393618241769	0.794238304822092	0.682943679235329\\
0.405	0.405569970164763	0.804147337016097	0.678558662193046\\
0.405	0.418891418885081	0.813826036051075	0.674343787412872\\
0.405	0.432351961217133	0.823271342915545	0.670253187715552\\
0.405	0.445945513182735	0.832480405557787	0.666246357336159\\
0.405	0.459665918841652	0.841450581281375	0.662288623474185\\
0.405	0.473506959189401	0.850179438693979	0.658351221673191\\
0.405	0.487462361096317	0.85866475920876	0.654411055137886\\
0.405	0.501525806262096	0.866904538099407	0.650450223206467\\
0.405	0.515690940160279	0.874896985111485	0.646455401874959\\
0.405	0.529951380947623	0.882640524634437	0.642417151055155\\
0.405	0.544300728313782	0.890133795440129	0.638329210947585\\
0.405	0.558732572247415	0.897375649995373	0.634187835387499\\
0.405	0.57324050169555	0.904365153357279	0.629991194998376\\
0.405	0.587818113093873	0.911101581661729	0.625738868900924\\
0.405	0.602459018746568	0.917584420216504	0.62143143163314\\
0.405	0.617156855035314	0.923813361211857	0.617070132480554\\
0.405	0.631905290438127	0.92978830106243	0.612656657833838\\
0.405	0.6466980333399	0.935509337395467	0.608192963377087\\
0.405	0.661528839617671	0.940976765701173	0.603681161495105\\
0.405	0.676391519984904	0.946191075661951	0.599123449738608\\
0.405	0.691279947080357	0.951152947177946	0.594522067901589\\
0.405	0.706188062288412	0.955863246106974	0.589879273662387\\
0.405	0.721109882279076	0.960323019737426	0.585197329319937\\
0.405	0.736039505257225	0.964533492013186	0.580478494543978\\
0.405	0.75097111691199	0.968496058529893	0.575725022018313\\
0.405	0.765898996058536	0.972212281322139	0.570939154289557\\
0.405	0.780817519965826	0.975683883461272	0.566123121053175\\
0.405	0.795721169365275	0.97891274348356	0.561279136605558\\
0.405	0.810604533136476	0.981900889668376	0.556409397399198\\
0.405	0.825462312667457	0.984650494185936	0.551516079695736\\
0.405	0.840289325888138	0.987163867133889	0.546601337319641\\
0.405	0.855080510976839	0.989443450481768	0.54166729951514\\
0.405	0.86983092974082	0.991491811941915	0.536716068908598\\
0.405	0.884535770672926	0.993311638785082	0.53174971957814\\
0.405	0.899190351687411	0.994905731618374	0.526770295231797\\
0.405	0.913790122539012	0.996276998142666	0.521779807495105\\
0.405	0.928330666930242	0.997428446906011	0.516780234308684\\
0.405	0.942807704312699	0.998363181068902	0.511773518435871\\
0.405	0.95721709138901	0.999084392196567	0.506761566080168\\
0.405	0.971554823322703	0.999595354092743	0.501746245611873\\
0.405	0.985817034663989	0.999899416688637	0.496729386403004\\
0.405	1	1	0.491669087346906\\
0.42	0	0	0.710195569525497\\
0.42	0.000100583311362513	0.0141829653360114	0.71240460418299\\
0.42	0.000404645907256436	0.0284451766772965	0.714630690536905\\
0.42	0.000915607803433	0.0427829086109896	0.716916312064106\\
0.42	0.00163681893109844	0.057192295687301	0.719289019360234\\
0.42	0.00257155309398959	0.0716693330697585	0.721780367061016\\
0.42	0.00372300185733414	0.0862098774609879	0.724425655492835\\
0.42	0.00509426838162598	0.100809648312589	0.727263413850304\\
0.42	0.00668836121491816	0.115464229327074	0.73033458997498\\
0.42	0.00850818805808555	0.13016907025918	0.733681422514709\\
0.42	0.0105565495182326	0.144919489023162	0.737345986077549\\
0.42	0.0128361328661109	0.159710674111862	0.741368418705642\\
0.42	0.0153495058140643	0.174537687332543	0.745784862958202\\
0.42	0.0180991103316243	0.189395466863524	0.750625176092311\\
0.42	0.0210872565164405	0.204278830634725	0.755910489868998\\
0.42	0.0243161165387281	0.219182480034174	0.761650724664453\\
0.42	0.0277877186778607	0.234101003941464	0.767842183867616\\
0.42	0.0315039414701067	0.24902888308801	0.77446537092041\\
0.42	0.0354665079868145	0.263960494742775	0.781483180780607\\
0.42	0.0396769802625738	0.278890117720924	0.788839618268074\\
0.42	0.0441367538930258	0.293811937711588	0.796459186325544\\
0.42	0.0488470528220537	0.308720052919643	0.804247066923455\\
0.42	0.0538089243380495	0.323608480015096	0.81209018616195\\
0.42	0.0590232342988274	0.338471160382329	0.819859213932694\\
0.42	0.064490662604533	0.3533019666601	0.82741149906364\\
0.42	0.0702116989375697	0.368094709561873	0.834594885813504\\
0.42	0.0761866387881432	0.382843144964686	0.841252300287181\\
0.42	0.0824155797834956	0.397540981253432	0.847226939729081\\
0.42	0.0888984183382709	0.412181886906127	0.85236784790504\\
0.42	0.0956348466427212	0.42675949830445	0.856535620027829\\
0.42	0.102624350004627	0.441267427752584	0.859607954619617\\
0.42	0.109866204559871	0.455699271686218	0.861484760266622\\
0.42	0.117359475365564	0.470048619052377	0.862092534251541\\
0.42	0.125103014888515	0.484309059839721	0.861387758068369\\
0.42	0.133095461900593	0.498474193737904	0.859359100890819\\
0.42	0.14133524079124	0.512537638903683	0.856028283765736\\
0.42	0.149820561306021	0.526493040810599	0.851449530870183\\
0.42	0.158549418718625	0.540334081158348	0.845707614739505\\
0.42	0.167519594442214	0.554054486817265	0.838914584343153\\
0.42	0.176728657084455	0.567648038782867	0.831205342370838\\
0.42	0.186173963948925	0.581108581114919	0.822732305417529\\
0.42	0.195852662983903	0.594430029835237	0.813659432944309\\
0.42	0.205761695177907	0.607606381758231	0.804155944104869\\
0.42	0.215897797399558	0.620631723228143	0.794390053406972\\
0.42	0.226257505677663	0.633500238737009	0.7845230460628\\
0.42	0.236837158915675	0.646206219397571	0.774703982861589\\
0.42	0.247632903032949	0.658744071245709	0.765065275211016\\
0.42	0.258640695523528	0.671108323347402	0.755719307772964\\
0.42	0.269856310421543	0.683293635685828	0.746756214005913\\
0.42	0.28127534366066	0.695294806804925	0.738242834604294\\
0.42	0.292893218813452	0.707106781186547	0.730222816007836\\
0.42	0.304705193195075	0.71872465633934	0.722717741087767\\
0.42	0.316706364314172	0.730143689578457	0.715729131143265\\
0.42	0.328891676652598	0.741359304476472	0.709241120561857\\
0.42	0.341255928754291	0.752367096967051	0.703223584571265\\
0.42	0.353793780602429	0.763162841084325	0.697635496597417\\
0.42	0.366499761262991	0.773742494322337	0.692428303594618\\
0.42	0.379368276771857	0.784102202600442	0.687549132887069\\
0.42	0.392393618241769	0.794238304822092	0.682943679235329\\
0.42	0.405569970164763	0.804147337016097	0.678558662193046\\
0.42	0.418891418885081	0.813826036051075	0.674343787412872\\
0.42	0.432351961217133	0.823271342915545	0.670253187715552\\
0.42	0.445945513182735	0.832480405557787	0.666246357336158\\
0.42	0.459665918841652	0.841450581281375	0.662288623474185\\
0.42	0.473506959189401	0.850179438693979	0.658351221673191\\
0.42	0.487462361096317	0.85866475920876	0.654411055137886\\
0.42	0.501525806262096	0.866904538099407	0.650450223206468\\
0.42	0.515690940160279	0.874896985111485	0.646455401874959\\
0.42	0.529951380947623	0.882640524634437	0.642417151055155\\
0.42	0.544300728313782	0.890133795440129	0.638329210947585\\
0.42	0.558732572247415	0.897375649995373	0.634187835387498\\
0.42	0.57324050169555	0.904365153357279	0.629991194998375\\
0.42	0.587818113093873	0.911101581661729	0.625738868900925\\
0.42	0.602459018746568	0.917584420216504	0.62143143163314\\
0.42	0.617156855035314	0.923813361211857	0.617070132480554\\
0.42	0.631905290438127	0.92978830106243	0.612656657833838\\
0.42	0.6466980333399	0.935509337395467	0.608192963377088\\
0.42	0.661528839617671	0.940976765701173	0.603681161495105\\
0.42	0.676391519984904	0.946191075661951	0.599123449738607\\
0.42	0.691279947080357	0.951152947177946	0.594522067901592\\
0.42	0.706188062288412	0.955863246106974	0.589879273662388\\
0.42	0.721109882279076	0.960323019737426	0.585197329319934\\
0.42	0.736039505257225	0.964533492013186	0.580478494543978\\
0.42	0.75097111691199	0.968496058529893	0.575725022018313\\
0.42	0.765898996058536	0.972212281322139	0.570939154289558\\
0.42	0.780817519965826	0.975683883461272	0.566123121053177\\
0.42	0.795721169365275	0.97891274348356	0.561279136605557\\
0.42	0.810604533136476	0.981900889668376	0.556409397399198\\
0.42	0.825462312667457	0.984650494185936	0.551516079695738\\
0.42	0.840289325888138	0.987163867133889	0.546601337319641\\
0.42	0.855080510976839	0.989443450481768	0.541667299515139\\
0.42	0.86983092974082	0.991491811941914	0.536716068908595\\
0.42	0.884535770672926	0.993311638785082	0.531749719578141\\
0.42	0.899190351687411	0.994905731618374	0.526770295231796\\
0.42	0.913790122539012	0.996276998142666	0.521779807495104\\
0.42	0.928330666930242	0.997428446906011	0.516780234308685\\
0.42	0.942807704312699	0.998363181068902	0.511773518435872\\
0.42	0.95721709138901	0.999084392196567	0.50676156608017\\
0.42	0.971554823322704	0.999595354092744	0.501746245611874\\
0.42	0.985817034663989	0.999899416688637	0.496729386403002\\
0.42	1	1	0.491669087346908\\
0.435	0	0	0.710195569525497\\
0.435	0.000100583311362513	0.0141829653360114	0.71240460418299\\
0.435	0.000404645907256436	0.0284451766772965	0.714630690536905\\
0.435	0.000915607803433	0.0427829086109896	0.716916312064106\\
0.435	0.00163681893109844	0.057192295687301	0.719289019360234\\
0.435	0.00257155309398959	0.0716693330697585	0.721780367061016\\
0.435	0.00372300185733414	0.0862098774609879	0.724425655492835\\
0.435	0.00509426838162598	0.100809648312589	0.727263413850304\\
0.435	0.00668836121491816	0.115464229327074	0.73033458997498\\
0.435	0.00850818805808555	0.13016907025918	0.733681422514709\\
0.435	0.0105565495182326	0.144919489023162	0.737345986077549\\
0.435	0.0128361328661109	0.159710674111862	0.741368418705642\\
0.435	0.0153495058140643	0.174537687332543	0.745784862958201\\
0.435	0.0180991103316243	0.189395466863524	0.750625176092311\\
0.435	0.0210872565164405	0.204278830634725	0.755910489868999\\
0.435	0.0243161165387281	0.219182480034174	0.761650724664453\\
0.435	0.0277877186778607	0.234101003941464	0.767842183867616\\
0.435	0.0315039414701067	0.24902888308801	0.77446537092041\\
0.435	0.0354665079868145	0.263960494742775	0.781483180780606\\
0.435	0.0396769802625737	0.278890117720924	0.788839618268074\\
0.435	0.0441367538930258	0.293811937711588	0.796459186325545\\
0.435	0.0488470528220537	0.308720052919643	0.804247066923454\\
0.435	0.0538089243380495	0.323608480015097	0.81209018616195\\
0.435	0.0590232342988274	0.338471160382329	0.819859213932693\\
0.435	0.064490662604533	0.3533019666601	0.827411499063639\\
0.435	0.0702116989375697	0.368094709561873	0.834594885813505\\
0.435	0.0761866387881433	0.382843144964686	0.841252300287178\\
0.435	0.0824155797834956	0.397540981253432	0.847226939729082\\
0.435	0.0888984183382709	0.412181886906127	0.85236784790504\\
0.435	0.0956348466427213	0.42675949830445	0.856535620027829\\
0.435	0.102624350004627	0.441267427752585	0.859607954619617\\
0.435	0.109866204559871	0.455699271686219	0.861484760266622\\
0.435	0.117359475365564	0.470048619052377	0.862092534251543\\
0.435	0.125103014888515	0.484309059839721	0.861387758068368\\
0.435	0.133095461900593	0.498474193737904	0.859359100890819\\
0.435	0.14133524079124	0.512537638903683	0.856028283765736\\
0.435	0.149820561306021	0.526493040810599	0.851449530870183\\
0.435	0.158549418718625	0.540334081158348	0.845707614739504\\
0.435	0.167519594442213	0.554054486817265	0.838914584343153\\
0.435	0.176728657084455	0.567648038782868	0.831205342370835\\
0.435	0.186173963948925	0.581108581114919	0.822732305417527\\
0.435	0.195852662983903	0.594430029835237	0.813659432944309\\
0.435	0.205761695177907	0.607606381758231	0.804155944104866\\
0.435	0.215897797399558	0.620631723228144	0.79439005340697\\
0.435	0.226257505677663	0.633500238737009	0.784523046062797\\
0.435	0.236837158915675	0.646206219397571	0.774703982861591\\
0.435	0.247632903032949	0.658744071245709	0.765065275211016\\
0.435	0.258640695523528	0.671108323347402	0.755719307772964\\
0.435	0.269856310421543	0.683293635685828	0.746756214005913\\
0.435	0.28127534366066	0.695294806804925	0.738242834604294\\
0.435	0.292893218813452	0.707106781186547	0.730222816007835\\
0.435	0.304705193195076	0.71872465633934	0.722717741087767\\
0.435	0.316706364314172	0.730143689578457	0.715729131143266\\
0.435	0.328891676652598	0.741359304476472	0.709241120561857\\
0.435	0.341255928754291	0.752367096967051	0.703223584571265\\
0.435	0.353793780602429	0.763162841084325	0.697635496597418\\
0.435	0.366499761262991	0.773742494322337	0.692428303594619\\
0.435	0.379368276771857	0.784102202600443	0.687549132887069\\
0.435	0.392393618241769	0.794238304822092	0.682943679235328\\
0.435	0.405569970164763	0.804147337016097	0.678558662193046\\
0.435	0.418891418885081	0.813826036051075	0.674343787412873\\
0.435	0.432351961217132	0.823271342915545	0.670253187715552\\
0.435	0.445945513182735	0.832480405557787	0.666246357336156\\
0.435	0.459665918841652	0.841450581281375	0.662288623474185\\
0.435	0.473506959189401	0.850179438693979	0.658351221673193\\
0.435	0.487462361096317	0.85866475920876	0.654411055137885\\
0.435	0.501525806262096	0.866904538099407	0.650450223206467\\
0.435	0.51569094016028	0.874896985111485	0.64645540187496\\
0.435	0.529951380947623	0.882640524634437	0.642417151055154\\
0.435	0.544300728313782	0.890133795440129	0.638329210947585\\
0.435	0.558732572247415	0.897375649995373	0.634187835387499\\
0.435	0.57324050169555	0.904365153357279	0.629991194998374\\
0.435	0.587818113093873	0.911101581661729	0.625738868900923\\
0.435	0.602459018746568	0.917584420216504	0.62143143163314\\
0.435	0.617156855035314	0.923813361211857	0.617070132480554\\
0.435	0.631905290438127	0.92978830106243	0.612656657833839\\
0.435	0.6466980333399	0.935509337395467	0.608192963377087\\
0.435	0.661528839617671	0.940976765701173	0.603681161495106\\
0.435	0.676391519984904	0.946191075661951	0.599123449738608\\
0.435	0.691279947080357	0.951152947177946	0.594522067901591\\
0.435	0.706188062288412	0.955863246106974	0.589879273662389\\
0.435	0.721109882279076	0.960323019737426	0.585197329319935\\
0.435	0.736039505257225	0.964533492013186	0.580478494543978\\
0.435	0.75097111691199	0.968496058529893	0.575725022018313\\
0.435	0.765898996058536	0.972212281322139	0.570939154289558\\
0.435	0.780817519965826	0.975683883461272	0.566123121053175\\
0.435	0.795721169365275	0.97891274348356	0.561279136605558\\
0.435	0.810604533136476	0.981900889668376	0.556409397399199\\
0.435	0.825462312667457	0.984650494185936	0.551516079695738\\
0.435	0.840289325888138	0.987163867133889	0.546601337319641\\
0.435	0.855080510976839	0.989443450481767	0.541667299515139\\
0.435	0.86983092974082	0.991491811941914	0.536716068908597\\
0.435	0.884535770672926	0.993311638785082	0.531749719578143\\
0.435	0.899190351687411	0.994905731618374	0.526770295231797\\
0.435	0.913790122539012	0.996276998142666	0.521779807495101\\
0.435	0.928330666930242	0.997428446906011	0.516780234308684\\
0.435	0.942807704312699	0.998363181068902	0.51177351843587\\
0.435	0.95721709138901	0.999084392196567	0.506761566080167\\
0.435	0.971554823322703	0.999595354092743	0.501746245611876\\
0.435	0.985817034663989	0.999899416688637	0.496729386403004\\
0.435	1	1	0.491669087346906\\
0.45	0	0	0.710195569525497\\
0.45	0.000100583311362513	0.0141829653360114	0.71240460418299\\
0.45	0.000404645907256436	0.0284451766772965	0.714630690536905\\
0.45	0.000915607803433	0.0427829086109896	0.716916312064106\\
0.45	0.00163681893109844	0.057192295687301	0.719289019360234\\
0.45	0.00257155309398959	0.0716693330697584	0.721780367061016\\
0.45	0.00372300185733414	0.0862098774609879	0.724425655492835\\
0.45	0.00509426838162598	0.100809648312589	0.727263413850304\\
0.45	0.00668836121491816	0.115464229327074	0.73033458997498\\
0.45	0.00850818805808555	0.13016907025918	0.733681422514709\\
0.45	0.0105565495182326	0.144919489023162	0.737345986077549\\
0.45	0.0128361328661109	0.159710674111862	0.741368418705642\\
0.45	0.0153495058140643	0.174537687332543	0.745784862958202\\
0.45	0.0180991103316243	0.189395466863524	0.750625176092311\\
0.45	0.0210872565164405	0.204278830634725	0.755910489868998\\
0.45	0.0243161165387281	0.219182480034174	0.761650724664453\\
0.45	0.0277877186778607	0.234101003941464	0.767842183867616\\
0.45	0.0315039414701067	0.24902888308801	0.77446537092041\\
0.45	0.0354665079868145	0.263960494742775	0.781483180780606\\
0.45	0.0396769802625737	0.278890117720924	0.788839618268073\\
0.45	0.0441367538930258	0.293811937711588	0.796459186325545\\
0.45	0.0488470528220538	0.308720052919643	0.804247066923454\\
0.45	0.0538089243380495	0.323608480015097	0.812090186161949\\
0.45	0.0590232342988274	0.338471160382329	0.819859213932693\\
0.45	0.064490662604533	0.3533019666601	0.827411499063639\\
0.45	0.0702116989375697	0.368094709561873	0.834594885813505\\
0.45	0.0761866387881433	0.382843144964686	0.841252300287179\\
0.45	0.0824155797834956	0.397540981253432	0.847226939729081\\
0.45	0.0888984183382709	0.412181886906127	0.852367847905043\\
0.45	0.0956348466427212	0.42675949830445	0.85653562002783\\
0.45	0.102624350004627	0.441267427752585	0.859607954619618\\
0.45	0.109866204559871	0.455699271686218	0.861484760266619\\
0.45	0.117359475365564	0.470048619052377	0.862092534251541\\
0.45	0.125103014888515	0.484309059839721	0.861387758068369\\
0.45	0.133095461900593	0.498474193737904	0.859359100890822\\
0.45	0.14133524079124	0.512537638903683	0.856028283765738\\
0.45	0.149820561306021	0.526493040810599	0.851449530870182\\
0.45	0.158549418718625	0.540334081158348	0.845707614739506\\
0.45	0.167519594442213	0.554054486817265	0.83891458434315\\
0.45	0.176728657084455	0.567648038782867	0.831205342370835\\
0.45	0.186173963948925	0.581108581114919	0.822732305417528\\
0.45	0.195852662983903	0.594430029835237	0.813659432944307\\
0.45	0.205761695177907	0.607606381758231	0.804155944104868\\
0.45	0.215897797399558	0.620631723228144	0.79439005340697\\
0.45	0.226257505677663	0.633500238737009	0.784523046062801\\
0.45	0.236837158915675	0.646206219397571	0.774703982861592\\
0.45	0.247632903032949	0.658744071245709	0.765065275211015\\
0.45	0.258640695523528	0.671108323347402	0.755719307772964\\
0.45	0.269856310421543	0.683293635685828	0.746756214005913\\
0.45	0.28127534366066	0.695294806804925	0.738242834604295\\
0.45	0.292893218813452	0.707106781186547	0.730222816007835\\
0.45	0.304705193195076	0.71872465633934	0.722717741087768\\
0.45	0.316706364314172	0.730143689578457	0.715729131143265\\
0.45	0.328891676652598	0.741359304476472	0.709241120561855\\
0.45	0.341255928754291	0.752367096967051	0.703223584571265\\
0.45	0.353793780602429	0.763162841084325	0.697635496597419\\
0.45	0.366499761262991	0.773742494322337	0.692428303594619\\
0.45	0.379368276771857	0.784102202600443	0.687549132887069\\
0.45	0.392393618241769	0.794238304822092	0.682943679235329\\
0.45	0.405569970164763	0.804147337016097	0.678558662193046\\
0.45	0.418891418885081	0.813826036051075	0.674343787412872\\
0.45	0.432351961217133	0.823271342915545	0.670253187715552\\
0.45	0.445945513182735	0.832480405557787	0.666246357336157\\
0.45	0.459665918841652	0.841450581281375	0.662288623474183\\
0.45	0.473506959189401	0.850179438693979	0.658351221673193\\
0.45	0.487462361096317	0.85866475920876	0.654411055137887\\
0.45	0.501525806262096	0.866904538099407	0.650450223206467\\
0.45	0.515690940160279	0.874896985111485	0.64645540187496\\
0.45	0.529951380947623	0.882640524634437	0.642417151055155\\
0.45	0.544300728313782	0.890133795440129	0.638329210947586\\
0.45	0.558732572247415	0.897375649995373	0.634187835387498\\
0.45	0.57324050169555	0.904365153357279	0.629991194998376\\
0.45	0.587818113093873	0.911101581661729	0.625738868900923\\
0.45	0.602459018746568	0.917584420216504	0.621431431633139\\
0.45	0.617156855035314	0.923813361211857	0.617070132480553\\
0.45	0.631905290438127	0.92978830106243	0.612656657833839\\
0.45	0.6466980333399	0.935509337395467	0.608192963377087\\
0.45	0.661528839617671	0.940976765701173	0.603681161495104\\
0.45	0.676391519984904	0.94619107566195	0.599123449738608\\
0.45	0.691279947080357	0.951152947177946	0.594522067901592\\
0.45	0.706188062288412	0.955863246106974	0.58987927366239\\
0.45	0.721109882279076	0.960323019737426	0.585197329319936\\
0.45	0.736039505257225	0.964533492013186	0.580478494543978\\
0.45	0.75097111691199	0.968496058529893	0.575725022018312\\
0.45	0.765898996058536	0.972212281322139	0.570939154289557\\
0.45	0.780817519965826	0.975683883461272	0.566123121053176\\
0.45	0.795721169365275	0.97891274348356	0.561279136605558\\
0.45	0.810604533136476	0.981900889668376	0.556409397399197\\
0.45	0.825462312667457	0.984650494185936	0.551516079695735\\
0.45	0.840289325888138	0.987163867133889	0.54660133731964\\
0.45	0.855080510976838	0.989443450481768	0.541667299515141\\
0.45	0.86983092974082	0.991491811941915	0.536716068908596\\
0.45	0.884535770672926	0.993311638785082	0.531749719578142\\
0.45	0.899190351687411	0.994905731618374	0.5267702952318\\
0.45	0.913790122539012	0.996276998142666	0.521779807495101\\
0.45	0.928330666930241	0.99742844690601	0.516780234308682\\
0.45	0.942807704312699	0.998363181068902	0.511773518435873\\
0.45	0.95721709138901	0.999084392196567	0.50676156608017\\
0.45	0.971554823322704	0.999595354092744	0.501746245611876\\
0.45	0.985817034663989	0.999899416688637	0.496729386403002\\
0.45	1	1	0.491669087346908\\
0.465	0	0	0.710195569525497\\
0.465	0.000100583311362513	0.0141829653360114	0.71240460418299\\
0.465	0.000404645907256436	0.0284451766772965	0.714630690536905\\
0.465	0.000915607803433	0.0427829086109896	0.716916312064106\\
0.465	0.00163681893109844	0.057192295687301	0.719289019360234\\
0.465	0.00257155309398959	0.0716693330697585	0.721780367061016\\
0.465	0.00372300185733414	0.0862098774609879	0.724425655492835\\
0.465	0.00509426838162598	0.100809648312589	0.727263413850304\\
0.465	0.00668836121491816	0.115464229327074	0.73033458997498\\
0.465	0.00850818805808555	0.13016907025918	0.733681422514709\\
0.465	0.0105565495182326	0.144919489023162	0.737345986077549\\
0.465	0.0128361328661109	0.159710674111862	0.741368418705642\\
0.465	0.0153495058140643	0.174537687332543	0.745784862958201\\
0.465	0.0180991103316243	0.189395466863524	0.750625176092311\\
0.465	0.0210872565164405	0.204278830634725	0.755910489868999\\
0.465	0.0243161165387281	0.219182480034174	0.761650724664453\\
0.465	0.0277877186778607	0.234101003941464	0.767842183867616\\
0.465	0.0315039414701067	0.24902888308801	0.77446537092041\\
0.465	0.0354665079868145	0.263960494742775	0.781483180780606\\
0.465	0.0396769802625737	0.278890117720924	0.788839618268073\\
0.465	0.0441367538930258	0.293811937711588	0.796459186325544\\
0.465	0.0488470528220538	0.308720052919643	0.804247066923455\\
0.465	0.0538089243380495	0.323608480015096	0.812090186161949\\
0.465	0.0590232342988274	0.338471160382329	0.819859213932694\\
0.465	0.064490662604533	0.3533019666601	0.82741149906364\\
0.465	0.0702116989375697	0.368094709561873	0.834594885813502\\
0.465	0.0761866387881433	0.382843144964686	0.841252300287179\\
0.465	0.0824155797834956	0.397540981253432	0.847226939729081\\
0.465	0.0888984183382709	0.412181886906127	0.852367847905043\\
0.465	0.0956348466427212	0.42675949830445	0.85653562002783\\
0.465	0.102624350004627	0.441267427752585	0.859607954619618\\
0.465	0.109866204559871	0.455699271686218	0.86148476026662\\
0.465	0.117359475365564	0.470048619052377	0.86209253425154\\
0.465	0.125103014888515	0.484309059839721	0.86138775806837\\
0.465	0.133095461900593	0.498474193737904	0.85935910089082\\
0.465	0.14133524079124	0.512537638903683	0.856028283765734\\
0.465	0.149820561306021	0.526493040810599	0.851449530870183\\
0.465	0.158549418718625	0.540334081158348	0.845707614739507\\
0.465	0.167519594442213	0.554054486817265	0.838914584343148\\
0.465	0.176728657084455	0.567648038782868	0.831205342370835\\
0.465	0.186173963948925	0.581108581114919	0.822732305417528\\
0.465	0.195852662983903	0.594430029835237	0.813659432944309\\
0.465	0.205761695177907	0.607606381758231	0.804155944104872\\
0.465	0.215897797399558	0.620631723228144	0.79439005340697\\
0.465	0.226257505677663	0.633500238737009	0.784523046062802\\
0.465	0.236837158915675	0.646206219397571	0.774703982861587\\
0.465	0.247632903032949	0.658744071245709	0.765065275211016\\
0.465	0.258640695523528	0.671108323347402	0.755719307772964\\
0.465	0.269856310421543	0.683293635685828	0.746756214005914\\
0.465	0.28127534366066	0.695294806804925	0.738242834604294\\
0.465	0.292893218813452	0.707106781186547	0.730222816007833\\
0.465	0.304705193195076	0.71872465633934	0.722717741087767\\
0.465	0.316706364314172	0.730143689578457	0.715729131143266\\
0.465	0.328891676652598	0.741359304476472	0.709241120561857\\
0.465	0.341255928754291	0.752367096967051	0.703223584571265\\
0.465	0.353793780602429	0.763162841084325	0.697635496597417\\
0.465	0.366499761262991	0.773742494322337	0.692428303594619\\
0.465	0.379368276771857	0.784102202600443	0.68754913288707\\
0.465	0.392393618241769	0.794238304822092	0.682943679235329\\
0.465	0.405569970164763	0.804147337016097	0.678558662193046\\
0.465	0.418891418885081	0.813826036051075	0.674343787412872\\
0.465	0.432351961217132	0.823271342915545	0.670253187715552\\
0.465	0.445945513182735	0.832480405557787	0.666246357336158\\
0.465	0.459665918841652	0.841450581281375	0.662288623474184\\
0.465	0.473506959189401	0.850179438693979	0.658351221673191\\
0.465	0.487462361096317	0.85866475920876	0.654411055137885\\
0.465	0.501525806262096	0.866904538099407	0.650450223206467\\
0.465	0.515690940160279	0.874896985111485	0.64645540187496\\
0.465	0.529951380947623	0.882640524634437	0.642417151055155\\
0.465	0.544300728313782	0.890133795440129	0.638329210947585\\
0.465	0.558732572247415	0.897375649995373	0.6341878353875\\
0.465	0.57324050169555	0.904365153357279	0.629991194998376\\
0.465	0.587818113093873	0.911101581661729	0.625738868900922\\
0.465	0.602459018746568	0.917584420216505	0.62143143163314\\
0.465	0.617156855035314	0.923813361211857	0.617070132480553\\
0.465	0.631905290438127	0.92978830106243	0.612656657833839\\
0.465	0.6466980333399	0.935509337395467	0.608192963377086\\
0.465	0.661528839617671	0.940976765701173	0.603681161495104\\
0.465	0.676391519984904	0.94619107566195	0.599123449738608\\
0.465	0.691279947080357	0.951152947177946	0.59452206790159\\
0.465	0.706188062288412	0.955863246106974	0.589879273662389\\
0.465	0.721109882279076	0.960323019737426	0.585197329319937\\
0.465	0.736039505257225	0.964533492013186	0.580478494543978\\
0.465	0.75097111691199	0.968496058529893	0.575725022018313\\
0.465	0.765898996058536	0.972212281322139	0.570939154289559\\
0.465	0.780817519965826	0.975683883461272	0.566123121053174\\
0.465	0.795721169365275	0.97891274348356	0.561279136605558\\
0.465	0.810604533136476	0.981900889668376	0.5564093973992\\
0.465	0.825462312667457	0.984650494185936	0.551516079695737\\
0.465	0.840289325888138	0.987163867133889	0.546601337319639\\
0.465	0.855080510976838	0.989443450481768	0.541667299515138\\
0.465	0.86983092974082	0.991491811941914	0.536716068908596\\
0.465	0.884535770672926	0.993311638785082	0.531749719578141\\
0.465	0.899190351687411	0.994905731618374	0.5267702952318\\
0.465	0.913790122539012	0.996276998142666	0.521779807495104\\
0.465	0.928330666930242	0.997428446906011	0.516780234308682\\
0.465	0.942807704312699	0.998363181068902	0.51177351843587\\
0.465	0.95721709138901	0.999084392196567	0.506761566080169\\
0.465	0.971554823322704	0.999595354092744	0.501746245611876\\
0.465	0.985817034663989	0.999899416688637	0.496729386403002\\
0.465	1	1	0.491669087346908\\
0.48	0	0	0.710195569525497\\
0.48	0.000100583311362513	0.0141829653360114	0.71240460418299\\
0.48	0.000404645907256436	0.0284451766772965	0.714630690536905\\
0.48	0.000915607803433	0.0427829086109896	0.716916312064106\\
0.48	0.00163681893109844	0.057192295687301	0.719289019360234\\
0.48	0.00257155309398959	0.0716693330697585	0.721780367061016\\
0.48	0.00372300185733414	0.0862098774609879	0.724425655492835\\
0.48	0.00509426838162598	0.100809648312589	0.727263413850304\\
0.48	0.00668836121491816	0.115464229327074	0.73033458997498\\
0.48	0.00850818805808555	0.13016907025918	0.733681422514709\\
0.48	0.0105565495182326	0.144919489023162	0.737345986077549\\
0.48	0.0128361328661109	0.159710674111862	0.741368418705642\\
0.48	0.0153495058140643	0.174537687332543	0.745784862958201\\
0.48	0.0180991103316243	0.189395466863524	0.750625176092311\\
0.48	0.0210872565164405	0.204278830634725	0.755910489868999\\
0.48	0.0243161165387281	0.219182480034174	0.761650724664453\\
0.48	0.0277877186778607	0.234101003941464	0.767842183867616\\
0.48	0.0315039414701067	0.24902888308801	0.77446537092041\\
0.48	0.0354665079868145	0.263960494742775	0.781483180780606\\
0.48	0.0396769802625737	0.278890117720924	0.788839618268073\\
0.48	0.0441367538930258	0.293811937711588	0.796459186325545\\
0.48	0.0488470528220538	0.308720052919643	0.804247066923454\\
0.48	0.0538089243380495	0.323608480015097	0.812090186161949\\
0.48	0.0590232342988274	0.338471160382329	0.819859213932694\\
0.48	0.064490662604533	0.3533019666601	0.827411499063639\\
0.48	0.0702116989375697	0.368094709561873	0.834594885813503\\
0.48	0.0761866387881432	0.382843144964686	0.841252300287178\\
0.48	0.0824155797834956	0.397540981253432	0.847226939729081\\
0.48	0.0888984183382709	0.412181886906127	0.85236784790504\\
0.48	0.0956348466427212	0.42675949830445	0.85653562002783\\
0.48	0.102624350004627	0.441267427752584	0.859607954619619\\
0.48	0.109866204559871	0.455699271686218	0.861484760266623\\
0.48	0.117359475365564	0.470048619052377	0.862092534251539\\
0.48	0.125103014888515	0.484309059839721	0.86138775806837\\
0.48	0.133095461900593	0.498474193737904	0.85935910089082\\
0.48	0.14133524079124	0.512537638903683	0.856028283765735\\
0.48	0.149820561306021	0.526493040810599	0.851449530870183\\
0.48	0.158549418718625	0.540334081158348	0.845707614739506\\
0.48	0.167519594442213	0.554054486817265	0.838914584343146\\
0.48	0.176728657084455	0.567648038782867	0.831205342370833\\
0.48	0.186173963948925	0.581108581114919	0.822732305417529\\
0.48	0.195852662983903	0.594430029835237	0.81365943294431\\
0.48	0.205761695177907	0.607606381758231	0.804155944104869\\
0.48	0.215897797399558	0.620631723228144	0.794390053406968\\
0.48	0.226257505677663	0.633500238737009	0.784523046062799\\
0.48	0.236837158915675	0.646206219397571	0.77470398286159\\
0.48	0.247632903032949	0.658744071245709	0.765065275211017\\
0.48	0.258640695523528	0.671108323347402	0.755719307772962\\
0.48	0.269856310421543	0.683293635685828	0.746756214005913\\
0.48	0.28127534366066	0.695294806804925	0.738242834604294\\
0.48	0.292893218813452	0.707106781186547	0.730222816007836\\
0.48	0.304705193195075	0.71872465633934	0.722717741087767\\
0.48	0.316706364314172	0.730143689578457	0.715729131143265\\
0.48	0.328891676652598	0.741359304476472	0.709241120561858\\
0.48	0.341255928754291	0.752367096967051	0.703223584571265\\
0.48	0.353793780602429	0.763162841084325	0.697635496597417\\
0.48	0.366499761262991	0.773742494322337	0.692428303594619\\
0.48	0.379368276771857	0.784102202600443	0.68754913288707\\
0.48	0.392393618241769	0.794238304822092	0.682943679235329\\
0.48	0.405569970164763	0.804147337016097	0.678558662193045\\
0.48	0.418891418885081	0.813826036051075	0.674343787412872\\
0.48	0.432351961217133	0.823271342915545	0.670253187715552\\
0.48	0.445945513182735	0.832480405557787	0.666246357336159\\
0.48	0.459665918841652	0.841450581281375	0.662288623474185\\
0.48	0.473506959189401	0.850179438693979	0.658351221673191\\
0.48	0.487462361096317	0.85866475920876	0.654411055137886\\
0.48	0.501525806262096	0.866904538099407	0.650450223206467\\
0.48	0.51569094016028	0.874896985111485	0.646455401874959\\
0.48	0.529951380947623	0.882640524634437	0.642417151055154\\
0.48	0.544300728313782	0.890133795440129	0.638329210947584\\
0.48	0.558732572247415	0.897375649995372	0.634187835387498\\
0.48	0.57324050169555	0.904365153357279	0.629991194998379\\
0.48	0.587818113093873	0.911101581661729	0.625738868900924\\
0.48	0.602459018746568	0.917584420216504	0.621431431633138\\
0.48	0.617156855035314	0.923813361211857	0.617070132480552\\
0.48	0.631905290438127	0.92978830106243	0.612656657833839\\
0.48	0.6466980333399	0.935509337395467	0.608192963377089\\
0.48	0.661528839617671	0.940976765701173	0.603681161495105\\
0.48	0.676391519984904	0.946191075661951	0.599123449738607\\
0.48	0.691279947080357	0.951152947177946	0.59452206790159\\
0.48	0.706188062288412	0.955863246106974	0.589879273662389\\
0.48	0.721109882279076	0.960323019737426	0.585197329319936\\
0.48	0.736039505257225	0.964533492013186	0.580478494543978\\
0.48	0.75097111691199	0.968496058529893	0.575725022018313\\
0.48	0.765898996058536	0.972212281322139	0.570939154289557\\
0.48	0.780817519965826	0.975683883461272	0.566123121053177\\
0.48	0.795721169365275	0.97891274348356	0.56127913660556\\
0.48	0.810604533136476	0.981900889668376	0.556409397399195\\
0.48	0.825462312667457	0.984650494185936	0.551516079695736\\
0.48	0.840289325888138	0.987163867133889	0.546601337319643\\
0.48	0.855080510976839	0.989443450481768	0.541667299515141\\
0.48	0.86983092974082	0.991491811941915	0.536716068908596\\
0.48	0.884535770672926	0.993311638785082	0.531749719578138\\
0.48	0.899190351687411	0.994905731618374	0.526770295231797\\
0.48	0.913790122539012	0.996276998142666	0.521779807495102\\
0.48	0.928330666930241	0.99742844690601	0.516780234308684\\
0.48	0.942807704312699	0.998363181068902	0.511773518435874\\
0.48	0.95721709138901	0.999084392196567	0.506761566080169\\
0.48	0.971554823322704	0.999595354092744	0.501746245611874\\
0.48	0.985817034663989	0.999899416688637	0.496729386403001\\
0.48	1	1	0.491669087346908\\
0.495	0	0	0.710195569525497\\
0.495	0.000100583311362513	0.0141829653360114	0.71240460418299\\
0.495	0.000404645907256436	0.0284451766772965	0.714630690536905\\
0.495	0.000915607803433	0.0427829086109896	0.716916312064106\\
0.495	0.00163681893109844	0.057192295687301	0.719289019360234\\
0.495	0.00257155309398959	0.0716693330697585	0.721780367061016\\
0.495	0.00372300185733414	0.0862098774609879	0.724425655492835\\
0.495	0.00509426838162598	0.100809648312589	0.727263413850304\\
0.495	0.00668836121491816	0.115464229327074	0.73033458997498\\
0.495	0.00850818805808555	0.13016907025918	0.733681422514709\\
0.495	0.0105565495182326	0.144919489023162	0.737345986077549\\
0.495	0.0128361328661109	0.159710674111862	0.741368418705642\\
0.495	0.0153495058140643	0.174537687332543	0.745784862958202\\
0.495	0.0180991103316243	0.189395466863524	0.750625176092311\\
0.495	0.0210872565164405	0.204278830634725	0.755910489868999\\
0.495	0.0243161165387281	0.219182480034174	0.761650724664453\\
0.495	0.0277877186778607	0.234101003941464	0.767842183867616\\
0.495	0.0315039414701067	0.24902888308801	0.77446537092041\\
0.495	0.0354665079868145	0.263960494742775	0.781483180780606\\
0.495	0.0396769802625737	0.278890117720924	0.788839618268072\\
0.495	0.0441367538930258	0.293811937711588	0.796459186325544\\
0.495	0.0488470528220538	0.308720052919643	0.804247066923454\\
0.495	0.0538089243380495	0.323608480015097	0.81209018616195\\
0.495	0.0590232342988274	0.338471160382329	0.819859213932693\\
0.495	0.064490662604533	0.3533019666601	0.82741149906364\\
0.495	0.0702116989375697	0.368094709561873	0.834594885813506\\
0.495	0.0761866387881432	0.382843144964686	0.841252300287178\\
0.495	0.0824155797834956	0.397540981253432	0.847226939729081\\
0.495	0.0888984183382709	0.412181886906127	0.852367847905041\\
0.495	0.0956348466427212	0.42675949830445	0.856535620027828\\
0.495	0.102624350004627	0.441267427752584	0.859607954619618\\
0.495	0.109866204559871	0.455699271686219	0.861484760266622\\
0.495	0.117359475365564	0.470048619052377	0.862092534251541\\
0.495	0.125103014888515	0.484309059839721	0.861387758068369\\
0.495	0.133095461900593	0.498474193737904	0.859359100890821\\
0.495	0.14133524079124	0.512537638903683	0.856028283765737\\
0.495	0.149820561306021	0.526493040810599	0.851449530870183\\
0.495	0.158549418718625	0.540334081158348	0.845707614739505\\
0.495	0.167519594442213	0.554054486817265	0.838914584343153\\
0.495	0.176728657084455	0.567648038782867	0.831205342370836\\
0.495	0.186173963948925	0.581108581114919	0.822732305417529\\
0.495	0.195852662983903	0.594430029835237	0.813659432944309\\
0.495	0.205761695177907	0.607606381758231	0.804155944104868\\
0.495	0.215897797399558	0.620631723228144	0.79439005340697\\
0.495	0.226257505677663	0.633500238737009	0.784523046062798\\
0.495	0.236837158915675	0.646206219397571	0.774703982861591\\
0.495	0.247632903032949	0.658744071245709	0.765065275211013\\
0.495	0.258640695523528	0.671108323347402	0.755719307772964\\
0.495	0.269856310421543	0.683293635685828	0.746756214005914\\
0.495	0.28127534366066	0.695294806804925	0.738242834604295\\
0.495	0.292893218813453	0.707106781186548	0.730222816007835\\
0.495	0.304705193195076	0.71872465633934	0.722717741087766\\
0.495	0.316706364314172	0.730143689578457	0.715729131143266\\
0.495	0.328891676652598	0.741359304476472	0.709241120561857\\
0.495	0.341255928754291	0.752367096967051	0.703223584571265\\
0.495	0.353793780602429	0.763162841084325	0.697635496597417\\
0.495	0.366499761262991	0.773742494322337	0.692428303594619\\
0.495	0.379368276771857	0.784102202600443	0.68754913288707\\
0.495	0.392393618241769	0.794238304822092	0.682943679235328\\
0.495	0.405569970164763	0.804147337016097	0.678558662193046\\
0.495	0.418891418885081	0.813826036051075	0.674343787412872\\
0.495	0.432351961217133	0.823271342915545	0.670253187715552\\
0.495	0.445945513182735	0.832480405557787	0.666246357336157\\
0.495	0.459665918841652	0.841450581281375	0.662288623474184\\
0.495	0.473506959189401	0.850179438693979	0.658351221673193\\
0.495	0.487462361096317	0.85866475920876	0.654411055137886\\
0.495	0.501525806262096	0.866904538099407	0.650450223206467\\
0.495	0.515690940160279	0.874896985111485	0.646455401874961\\
0.495	0.529951380947623	0.882640524634437	0.642417151055155\\
0.495	0.544300728313782	0.890133795440129	0.638329210947586\\
0.495	0.558732572247415	0.897375649995373	0.634187835387498\\
0.495	0.57324050169555	0.904365153357279	0.629991194998374\\
0.495	0.587818113093873	0.911101581661729	0.625738868900924\\
0.495	0.602459018746568	0.917584420216504	0.621431431633141\\
0.495	0.617156855035314	0.923813361211857	0.617070132480554\\
0.495	0.631905290438127	0.92978830106243	0.612656657833837\\
0.495	0.6466980333399	0.935509337395467	0.608192963377087\\
0.495	0.661528839617671	0.940976765701173	0.603681161495104\\
0.495	0.676391519984904	0.94619107566195	0.599123449738608\\
0.495	0.691279947080357	0.951152947177946	0.59452206790159\\
0.495	0.706188062288412	0.955863246106974	0.589879273662388\\
0.495	0.721109882279076	0.960323019737426	0.585197329319937\\
0.495	0.736039505257225	0.964533492013186	0.580478494543978\\
0.495	0.75097111691199	0.968496058529893	0.575725022018312\\
0.495	0.765898996058536	0.972212281322139	0.570939154289558\\
0.495	0.780817519965826	0.975683883461272	0.566123121053175\\
0.495	0.795721169365275	0.97891274348356	0.561279136605558\\
0.495	0.810604533136476	0.981900889668376	0.556409397399199\\
0.495	0.825462312667457	0.984650494185936	0.551516079695735\\
0.495	0.840289325888138	0.987163867133889	0.546601337319641\\
0.495	0.855080510976839	0.989443450481767	0.541667299515139\\
0.495	0.86983092974082	0.991491811941914	0.536716068908599\\
0.495	0.884535770672926	0.993311638785082	0.531749719578143\\
0.495	0.899190351687411	0.994905731618374	0.526770295231795\\
0.495	0.913790122539012	0.996276998142666	0.5217798074951\\
0.495	0.928330666930242	0.99742844690601	0.516780234308682\\
0.495	0.942807704312699	0.998363181068902	0.511773518435873\\
0.495	0.95721709138901	0.999084392196567	0.50676156608017\\
0.495	0.971554823322704	0.999595354092744	0.501746245611877\\
0.495	0.985817034663989	0.999899416688638	0.496729386403002\\
0.495	1	1	0.491669087346898\\
0.51	0	0	0.710195569525497\\
0.51	0.000100583311362513	0.0141829653360114	0.71240460418299\\
0.51	0.000404645907256436	0.0284451766772965	0.714630690536905\\
0.51	0.000915607803433	0.0427829086109896	0.716916312064106\\
0.51	0.00163681893109844	0.057192295687301	0.719289019360234\\
0.51	0.00257155309398959	0.0716693330697584	0.721780367061016\\
0.51	0.00372300185733414	0.086209877460988	0.724425655492835\\
0.51	0.00509426838162598	0.100809648312589	0.727263413850304\\
0.51	0.00668836121491816	0.115464229327074	0.73033458997498\\
0.51	0.00850818805808555	0.13016907025918	0.733681422514709\\
0.51	0.0105565495182326	0.144919489023162	0.737345986077549\\
0.51	0.0128361328661109	0.159710674111862	0.741368418705642\\
0.51	0.0153495058140643	0.174537687332543	0.745784862958202\\
0.51	0.0180991103316243	0.189395466863524	0.750625176092311\\
0.51	0.0210872565164405	0.204278830634725	0.755910489868999\\
0.51	0.0243161165387281	0.219182480034174	0.761650724664453\\
0.51	0.0277877186778607	0.234101003941464	0.767842183867616\\
0.51	0.0315039414701067	0.24902888308801	0.77446537092041\\
0.51	0.0354665079868145	0.263960494742775	0.781483180780606\\
0.51	0.0396769802625737	0.278890117720924	0.788839618268073\\
0.51	0.0441367538930258	0.293811937711588	0.796459186325544\\
0.51	0.0488470528220538	0.308720052919643	0.804247066923455\\
0.51	0.0538089243380495	0.323608480015097	0.812090186161949\\
0.51	0.0590232342988274	0.338471160382329	0.819859213932693\\
0.51	0.064490662604533	0.3533019666601	0.827411499063639\\
0.51	0.0702116989375697	0.368094709561873	0.834594885813507\\
0.51	0.0761866387881433	0.382843144964686	0.841252300287179\\
0.51	0.0824155797834956	0.397540981253432	0.84722693972908\\
0.51	0.0888984183382709	0.412181886906127	0.852367847905042\\
0.51	0.0956348466427213	0.42675949830445	0.85653562002783\\
0.51	0.102624350004627	0.441267427752584	0.859607954619617\\
0.51	0.109866204559871	0.455699271686218	0.861484760266619\\
0.51	0.117359475365564	0.470048619052377	0.862092534251542\\
0.51	0.125103014888515	0.484309059839721	0.861387758068366\\
0.51	0.133095461900593	0.498474193737904	0.85935910089082\\
0.51	0.14133524079124	0.512537638903683	0.856028283765738\\
0.51	0.149820561306021	0.526493040810599	0.851449530870183\\
0.51	0.158549418718625	0.540334081158348	0.845707614739506\\
0.51	0.167519594442213	0.554054486817265	0.83891458434315\\
0.51	0.176728657084455	0.567648038782867	0.831205342370838\\
0.51	0.186173963948925	0.581108581114919	0.82273230541753\\
0.51	0.195852662983903	0.594430029835237	0.813659432944309\\
0.51	0.205761695177907	0.607606381758231	0.80415594410487\\
0.51	0.215897797399558	0.620631723228144	0.794390053406967\\
0.51	0.226257505677663	0.633500238737009	0.7845230460628\\
0.51	0.236837158915675	0.646206219397571	0.774703982861589\\
0.51	0.247632903032949	0.658744071245709	0.765065275211014\\
0.51	0.258640695523528	0.671108323347402	0.755719307772964\\
0.51	0.269856310421543	0.683293635685828	0.746756214005913\\
0.51	0.28127534366066	0.695294806804925	0.738242834604294\\
0.51	0.292893218813452	0.707106781186547	0.730222816007833\\
0.51	0.304705193195076	0.71872465633934	0.722717741087767\\
0.51	0.316706364314172	0.730143689578457	0.715729131143267\\
0.51	0.328891676652598	0.741359304476472	0.709241120561857\\
0.51	0.341255928754291	0.752367096967051	0.703223584571266\\
0.51	0.353793780602429	0.763162841084325	0.697635496597418\\
0.51	0.366499761262991	0.773742494322337	0.692428303594619\\
0.51	0.379368276771857	0.784102202600443	0.68754913288707\\
0.51	0.392393618241769	0.794238304822092	0.682943679235329\\
0.51	0.405569970164763	0.804147337016097	0.678558662193045\\
0.51	0.418891418885081	0.813826036051075	0.674343787412872\\
0.51	0.432351961217133	0.823271342915545	0.670253187715552\\
0.51	0.445945513182735	0.832480405557787	0.666246357336159\\
0.51	0.459665918841652	0.841450581281375	0.662288623474184\\
0.51	0.473506959189401	0.850179438693979	0.658351221673191\\
0.51	0.487462361096317	0.85866475920876	0.654411055137887\\
0.51	0.501525806262096	0.866904538099407	0.650450223206467\\
0.51	0.515690940160279	0.874896985111485	0.646455401874959\\
0.51	0.529951380947623	0.882640524634437	0.642417151055155\\
0.51	0.544300728313782	0.890133795440129	0.638329210947586\\
0.51	0.558732572247415	0.897375649995373	0.6341878353875\\
0.51	0.57324050169555	0.904365153357279	0.629991194998375\\
0.51	0.587818113093873	0.911101581661729	0.625738868900923\\
0.51	0.602459018746568	0.917584420216504	0.62143143163314\\
0.51	0.617156855035314	0.923813361211857	0.617070132480554\\
0.51	0.631905290438127	0.92978830106243	0.612656657833839\\
0.51	0.6466980333399	0.935509337395467	0.608192963377088\\
0.51	0.661528839617671	0.940976765701173	0.603681161495104\\
0.51	0.676391519984904	0.94619107566195	0.599123449738608\\
0.51	0.691279947080357	0.951152947177946	0.594522067901591\\
0.51	0.706188062288412	0.955863246106974	0.589879273662387\\
0.51	0.721109882279076	0.960323019737426	0.585197329319936\\
0.51	0.736039505257225	0.964533492013185	0.580478494543978\\
0.51	0.75097111691199	0.968496058529893	0.575725022018313\\
0.51	0.765898996058536	0.972212281322139	0.570939154289556\\
0.51	0.780817519965826	0.975683883461272	0.566123121053177\\
0.51	0.795721169365275	0.97891274348356	0.56127913660556\\
0.51	0.810604533136476	0.981900889668376	0.556409397399197\\
0.51	0.825462312667457	0.984650494185936	0.551516079695735\\
0.51	0.840289325888138	0.987163867133889	0.546601337319641\\
0.51	0.855080510976839	0.989443450481768	0.541667299515141\\
0.51	0.86983092974082	0.991491811941915	0.536716068908596\\
0.51	0.884535770672926	0.993311638785082	0.531749719578142\\
0.51	0.899190351687411	0.994905731618374	0.526770295231799\\
0.51	0.913790122539012	0.996276998142666	0.521779807495101\\
0.51	0.928330666930241	0.99742844690601	0.516780234308683\\
0.51	0.942807704312699	0.998363181068902	0.511773518435874\\
0.51	0.95721709138901	0.999084392196567	0.506761566080166\\
0.51	0.971554823322703	0.999595354092743	0.501746245611875\\
0.51	0.985817034663989	0.999899416688637	0.496729386403004\\
0.51	1	1	0.491669087346906\\
0.525	0	0	0.710195569525497\\
0.525	0.000100583311362513	0.0141829653360114	0.71240460418299\\
0.525	0.000404645907256436	0.0284451766772965	0.714630690536905\\
0.525	0.000915607803433	0.0427829086109896	0.716916312064106\\
0.525	0.00163681893109844	0.057192295687301	0.719289019360234\\
0.525	0.00257155309398959	0.0716693330697585	0.721780367061016\\
0.525	0.00372300185733414	0.086209877460988	0.724425655492835\\
0.525	0.00509426838162598	0.100809648312589	0.727263413850304\\
0.525	0.00668836121491816	0.115464229327074	0.73033458997498\\
0.525	0.00850818805808555	0.13016907025918	0.733681422514709\\
0.525	0.0105565495182326	0.144919489023162	0.737345986077549\\
0.525	0.0128361328661109	0.159710674111862	0.741368418705642\\
0.525	0.0153495058140643	0.174537687332543	0.745784862958202\\
0.525	0.0180991103316243	0.189395466863524	0.750625176092311\\
0.525	0.0210872565164405	0.204278830634725	0.755910489868998\\
0.525	0.0243161165387281	0.219182480034174	0.761650724664453\\
0.525	0.0277877186778607	0.234101003941464	0.767842183867616\\
0.525	0.0315039414701067	0.24902888308801	0.77446537092041\\
0.525	0.0354665079868145	0.263960494742775	0.781483180780606\\
0.525	0.0396769802625738	0.278890117720924	0.788839618268073\\
0.525	0.0441367538930258	0.293811937711588	0.796459186325544\\
0.525	0.0488470528220538	0.308720052919643	0.804247066923455\\
0.525	0.0538089243380495	0.323608480015096	0.81209018616195\\
0.525	0.0590232342988274	0.338471160382329	0.819859213932693\\
0.525	0.064490662604533	0.3533019666601	0.827411499063638\\
0.525	0.0702116989375697	0.368094709561873	0.834594885813505\\
0.525	0.0761866387881432	0.382843144964686	0.841252300287181\\
0.525	0.0824155797834956	0.397540981253432	0.84722693972908\\
0.525	0.0888984183382709	0.412181886906127	0.852367847905041\\
0.525	0.0956348466427212	0.42675949830445	0.856535620027829\\
0.525	0.102624350004627	0.441267427752584	0.859607954619617\\
0.525	0.109866204559871	0.455699271686218	0.861484760266621\\
0.525	0.117359475365564	0.470048619052377	0.86209253425154\\
0.525	0.125103014888515	0.484309059839721	0.861387758068369\\
0.525	0.133095461900593	0.498474193737904	0.859359100890821\\
0.525	0.14133524079124	0.512537638903683	0.856028283765734\\
0.525	0.149820561306021	0.526493040810599	0.851449530870182\\
0.525	0.158549418718625	0.540334081158348	0.845707614739502\\
0.525	0.167519594442214	0.554054486817265	0.838914584343148\\
0.525	0.176728657084455	0.567648038782868	0.83120534237084\\
0.525	0.186173963948925	0.581108581114919	0.822732305417527\\
0.525	0.195852662983903	0.594430029835237	0.813659432944309\\
0.525	0.205761695177907	0.607606381758231	0.804155944104868\\
0.525	0.215897797399558	0.620631723228144	0.794390053406969\\
0.525	0.226257505677663	0.633500238737009	0.784523046062802\\
0.525	0.236837158915675	0.646206219397571	0.774703982861587\\
0.525	0.247632903032949	0.658744071245709	0.765065275211016\\
0.525	0.258640695523528	0.671108323347402	0.755719307772964\\
0.525	0.269856310421543	0.683293635685828	0.746756214005914\\
0.525	0.28127534366066	0.695294806804925	0.738242834604294\\
0.525	0.292893218813452	0.707106781186547	0.730222816007835\\
0.525	0.304705193195076	0.71872465633934	0.722717741087768\\
0.525	0.316706364314172	0.730143689578457	0.715729131143266\\
0.525	0.328891676652598	0.741359304476472	0.709241120561857\\
0.525	0.341255928754291	0.752367096967051	0.703223584571265\\
0.525	0.353793780602429	0.763162841084325	0.697635496597417\\
0.525	0.366499761262991	0.773742494322337	0.692428303594619\\
0.525	0.379368276771857	0.784102202600443	0.687549132887069\\
0.525	0.392393618241769	0.794238304822092	0.682943679235328\\
0.525	0.405569970164763	0.804147337016097	0.678558662193046\\
0.525	0.418891418885081	0.813826036051075	0.674343787412872\\
0.525	0.432351961217133	0.823271342915545	0.670253187715552\\
0.525	0.445945513182735	0.832480405557787	0.666246357336158\\
0.525	0.459665918841652	0.841450581281375	0.662288623474185\\
0.525	0.473506959189401	0.850179438693979	0.658351221673191\\
0.525	0.487462361096317	0.85866475920876	0.654411055137886\\
0.525	0.501525806262096	0.866904538099407	0.650450223206468\\
0.525	0.515690940160279	0.874896985111485	0.646455401874959\\
0.525	0.529951380947623	0.882640524634437	0.642417151055155\\
0.525	0.544300728313782	0.890133795440129	0.638329210947586\\
0.525	0.558732572247415	0.897375649995373	0.6341878353875\\
0.525	0.57324050169555	0.904365153357279	0.629991194998376\\
0.525	0.587818113093873	0.911101581661729	0.625738868900923\\
0.525	0.602459018746568	0.917584420216504	0.62143143163314\\
0.525	0.617156855035314	0.923813361211857	0.617070132480554\\
0.525	0.631905290438127	0.92978830106243	0.612656657833838\\
0.525	0.6466980333399	0.935509337395467	0.608192963377088\\
0.525	0.661528839617671	0.940976765701173	0.603681161495104\\
0.525	0.676391519984904	0.94619107566195	0.599123449738606\\
0.525	0.691279947080357	0.951152947177946	0.594522067901592\\
0.525	0.706188062288412	0.955863246106974	0.589879273662388\\
0.525	0.721109882279076	0.960323019737426	0.585197329319936\\
0.525	0.736039505257225	0.964533492013186	0.580478494543977\\
0.525	0.75097111691199	0.968496058529893	0.575725022018313\\
0.525	0.765898996058536	0.972212281322139	0.570939154289557\\
0.525	0.780817519965826	0.975683883461272	0.566123121053176\\
0.525	0.795721169365275	0.97891274348356	0.561279136605559\\
0.525	0.810604533136476	0.981900889668376	0.556409397399199\\
0.525	0.825462312667457	0.984650494185936	0.551516079695734\\
0.525	0.840289325888138	0.987163867133889	0.546601337319639\\
0.525	0.855080510976839	0.989443450481768	0.541667299515139\\
0.525	0.86983092974082	0.991491811941914	0.536716068908598\\
0.525	0.884535770672926	0.993311638785082	0.531749719578143\\
0.525	0.899190351687411	0.994905731618374	0.526770295231797\\
0.525	0.913790122539012	0.996276998142666	0.521779807495104\\
0.525	0.928330666930241	0.997428446906011	0.516780234308683\\
0.525	0.942807704312699	0.998363181068902	0.511773518435872\\
0.525	0.95721709138901	0.999084392196567	0.506761566080168\\
0.525	0.971554823322703	0.999595354092743	0.501746245611874\\
0.525	0.985817034663989	0.999899416688637	0.496729386403004\\
0.525	1	1	0.491669087346906\\
0.54	0	0	0.710195569525497\\
0.54	0.000100583311362513	0.0141829653360114	0.71240460418299\\
0.54	0.000404645907256436	0.0284451766772965	0.714630690536905\\
0.54	0.000915607803433	0.0427829086109896	0.716916312064106\\
0.54	0.00163681893109844	0.057192295687301	0.719289019360234\\
0.54	0.00257155309398959	0.0716693330697585	0.721780367061016\\
0.54	0.00372300185733414	0.0862098774609879	0.724425655492835\\
0.54	0.00509426838162598	0.100809648312589	0.727263413850304\\
0.54	0.00668836121491816	0.115464229327074	0.73033458997498\\
0.54	0.00850818805808555	0.13016907025918	0.733681422514709\\
0.54	0.0105565495182326	0.144919489023162	0.737345986077549\\
0.54	0.0128361328661109	0.159710674111862	0.741368418705642\\
0.54	0.0153495058140643	0.174537687332543	0.745784862958202\\
0.54	0.0180991103316243	0.189395466863524	0.750625176092311\\
0.54	0.0210872565164405	0.204278830634725	0.755910489868999\\
0.54	0.0243161165387281	0.219182480034174	0.761650724664453\\
0.54	0.0277877186778607	0.234101003941464	0.767842183867616\\
0.54	0.0315039414701067	0.24902888308801	0.77446537092041\\
0.54	0.0354665079868145	0.263960494742775	0.781483180780606\\
0.54	0.0396769802625738	0.278890117720924	0.788839618268073\\
0.54	0.0441367538930258	0.293811937711588	0.796459186325544\\
0.54	0.0488470528220538	0.308720052919643	0.804247066923454\\
0.54	0.0538089243380495	0.323608480015096	0.81209018616195\\
0.54	0.0590232342988274	0.338471160382329	0.819859213932694\\
0.54	0.064490662604533	0.3533019666601	0.827411499063639\\
0.54	0.0702116989375697	0.368094709561873	0.834594885813504\\
0.54	0.0761866387881432	0.382843144964686	0.84125230028718\\
0.54	0.0824155797834956	0.397540981253432	0.847226939729081\\
0.54	0.0888984183382709	0.412181886906127	0.852367847905041\\
0.54	0.0956348466427213	0.42675949830445	0.856535620027829\\
0.54	0.102624350004627	0.441267427752585	0.859607954619618\\
0.54	0.109866204559871	0.455699271686218	0.861484760266624\\
0.54	0.117359475365564	0.470048619052377	0.86209253425154\\
0.54	0.125103014888515	0.484309059839721	0.861387758068371\\
0.54	0.133095461900593	0.498474193737904	0.85935910089082\\
0.54	0.14133524079124	0.512537638903683	0.856028283765735\\
0.54	0.149820561306021	0.526493040810599	0.851449530870182\\
0.54	0.158549418718625	0.540334081158348	0.845707614739505\\
0.54	0.167519594442214	0.554054486817265	0.838914584343152\\
0.54	0.176728657084455	0.567648038782868	0.831205342370836\\
0.54	0.186173963948925	0.581108581114919	0.822732305417527\\
0.54	0.195852662983903	0.594430029835237	0.813659432944309\\
0.54	0.205761695177907	0.607606381758231	0.804155944104869\\
0.54	0.215897797399558	0.620631723228144	0.794390053406969\\
0.54	0.226257505677663	0.633500238737009	0.784523046062799\\
0.54	0.236837158915675	0.646206219397571	0.774703982861589\\
0.54	0.247632903032949	0.658744071245709	0.765065275211016\\
0.54	0.258640695523528	0.671108323347402	0.755719307772964\\
0.54	0.269856310421543	0.683293635685828	0.746756214005913\\
0.54	0.28127534366066	0.695294806804925	0.738242834604294\\
0.54	0.292893218813452	0.707106781186547	0.730222816007836\\
0.54	0.304705193195075	0.71872465633934	0.722717741087767\\
0.54	0.316706364314172	0.730143689578457	0.715729131143265\\
0.54	0.328891676652598	0.741359304476472	0.709241120561857\\
0.54	0.341255928754291	0.752367096967051	0.703223584571265\\
0.54	0.353793780602429	0.763162841084325	0.697635496597417\\
0.54	0.366499761262991	0.773742494322337	0.692428303594619\\
0.54	0.379368276771857	0.784102202600443	0.687549132887069\\
0.54	0.392393618241769	0.794238304822092	0.682943679235329\\
0.54	0.405569970164763	0.804147337016097	0.678558662193046\\
0.54	0.418891418885081	0.813826036051075	0.674343787412872\\
0.54	0.432351961217133	0.823271342915545	0.670253187715552\\
0.54	0.445945513182735	0.832480405557787	0.666246357336157\\
0.54	0.459665918841652	0.841450581281375	0.662288623474184\\
0.54	0.473506959189401	0.850179438693979	0.658351221673192\\
0.54	0.487462361096317	0.85866475920876	0.654411055137887\\
0.54	0.501525806262096	0.866904538099407	0.650450223206467\\
0.54	0.515690940160279	0.874896985111485	0.646455401874959\\
0.54	0.529951380947623	0.882640524634437	0.642417151055155\\
0.54	0.544300728313782	0.890133795440129	0.638329210947585\\
0.54	0.558732572247415	0.897375649995373	0.634187835387498\\
0.54	0.57324050169555	0.904365153357279	0.629991194998377\\
0.54	0.587818113093873	0.911101581661729	0.625738868900924\\
0.54	0.602459018746568	0.917584420216504	0.62143143163314\\
0.54	0.617156855035314	0.923813361211857	0.617070132480554\\
0.54	0.631905290438127	0.92978830106243	0.612656657833838\\
0.54	0.6466980333399	0.935509337395467	0.608192963377087\\
0.54	0.661528839617671	0.940976765701173	0.603681161495105\\
0.54	0.676391519984904	0.946191075661951	0.599123449738606\\
0.54	0.691279947080357	0.951152947177946	0.59452206790159\\
0.54	0.706188062288412	0.955863246106974	0.589879273662389\\
0.54	0.721109882279076	0.960323019737426	0.585197329319936\\
0.54	0.736039505257225	0.964533492013186	0.580478494543975\\
0.54	0.75097111691199	0.968496058529893	0.575725022018315\\
0.54	0.765898996058536	0.972212281322139	0.57093915428956\\
0.54	0.780817519965826	0.975683883461272	0.566123121053175\\
0.54	0.795721169365275	0.97891274348356	0.561279136605558\\
0.54	0.810604533136476	0.981900889668376	0.556409397399197\\
0.54	0.825462312667457	0.984650494185936	0.551516079695736\\
0.54	0.840289325888138	0.987163867133889	0.546601337319642\\
0.54	0.855080510976839	0.989443450481768	0.541667299515139\\
0.54	0.86983092974082	0.991491811941914	0.536716068908595\\
0.54	0.884535770672926	0.993311638785082	0.53174971957814\\
0.54	0.899190351687411	0.994905731618374	0.526770295231797\\
0.54	0.913790122539012	0.996276998142666	0.521779807495104\\
0.54	0.928330666930242	0.997428446906011	0.516780234308687\\
0.54	0.942807704312699	0.998363181068902	0.511773518435871\\
0.54	0.95721709138901	0.999084392196567	0.506761566080168\\
0.54	0.971554823322704	0.999595354092744	0.501746245611877\\
0.54	0.985817034663989	0.999899416688637	0.496729386403002\\
0.54	1	1	0.491669087346908\\
0.555	0	0	0.710195569525497\\
0.555	0.000100583311362513	0.0141829653360114	0.71240460418299\\
0.555	0.000404645907256436	0.0284451766772965	0.714630690536905\\
0.555	0.000915607803433	0.0427829086109896	0.716916312064106\\
0.555	0.00163681893109844	0.057192295687301	0.719289019360234\\
0.555	0.00257155309398959	0.0716693330697585	0.721780367061016\\
0.555	0.00372300185733414	0.086209877460988	0.724425655492835\\
0.555	0.00509426838162598	0.100809648312589	0.727263413850304\\
0.555	0.00668836121491816	0.115464229327074	0.73033458997498\\
0.555	0.00850818805808555	0.13016907025918	0.733681422514709\\
0.555	0.0105565495182326	0.144919489023162	0.737345986077549\\
0.555	0.0128361328661109	0.159710674111862	0.741368418705642\\
0.555	0.0153495058140643	0.174537687332543	0.745784862958202\\
0.555	0.0180991103316243	0.189395466863524	0.750625176092311\\
0.555	0.0210872565164405	0.204278830634725	0.755910489868998\\
0.555	0.0243161165387281	0.219182480034174	0.761650724664453\\
0.555	0.0277877186778607	0.234101003941464	0.767842183867616\\
0.555	0.0315039414701067	0.24902888308801	0.77446537092041\\
0.555	0.0354665079868145	0.263960494742775	0.781483180780606\\
0.555	0.0396769802625738	0.278890117720924	0.788839618268073\\
0.555	0.0441367538930258	0.293811937711588	0.796459186325544\\
0.555	0.0488470528220538	0.308720052919643	0.804247066923454\\
0.555	0.0538089243380495	0.323608480015097	0.81209018616195\\
0.555	0.0590232342988274	0.338471160382329	0.819859213932693\\
0.555	0.064490662604533	0.3533019666601	0.827411499063639\\
0.555	0.0702116989375697	0.368094709561873	0.834594885813504\\
0.555	0.0761866387881433	0.382843144964686	0.84125230028718\\
0.555	0.0824155797834956	0.397540981253432	0.847226939729081\\
0.555	0.0888984183382709	0.412181886906127	0.852367847905041\\
0.555	0.0956348466427212	0.42675949830445	0.856535620027831\\
0.555	0.102624350004627	0.441267427752585	0.859607954619617\\
0.555	0.109866204559871	0.455699271686219	0.861484760266622\\
0.555	0.117359475365564	0.470048619052377	0.86209253425154\\
0.555	0.125103014888515	0.484309059839721	0.861387758068369\\
0.555	0.133095461900593	0.498474193737904	0.859359100890821\\
0.555	0.14133524079124	0.512537638903683	0.856028283765737\\
0.555	0.149820561306021	0.526493040810599	0.851449530870183\\
0.555	0.158549418718625	0.540334081158348	0.845707614739508\\
0.555	0.167519594442214	0.554054486817265	0.83891458434315\\
0.555	0.176728657084455	0.567648038782868	0.831205342370836\\
0.555	0.186173963948925	0.581108581114919	0.822732305417529\\
0.555	0.195852662983903	0.594430029835237	0.813659432944309\\
0.555	0.205761695177907	0.607606381758231	0.804155944104869\\
0.555	0.215897797399558	0.620631723228144	0.794390053406969\\
0.555	0.226257505677663	0.633500238737009	0.784523046062799\\
0.555	0.236837158915675	0.646206219397571	0.774703982861588\\
0.555	0.247632903032949	0.658744071245709	0.765065275211016\\
0.555	0.258640695523528	0.671108323347402	0.755719307772964\\
0.555	0.269856310421543	0.683293635685828	0.746756214005914\\
0.555	0.28127534366066	0.695294806804925	0.738242834604294\\
0.555	0.292893218813452	0.707106781186547	0.730222816007836\\
0.555	0.304705193195075	0.71872465633934	0.722717741087767\\
0.555	0.316706364314172	0.730143689578457	0.715729131143265\\
0.555	0.328891676652598	0.741359304476472	0.709241120561857\\
0.555	0.341255928754291	0.752367096967051	0.703223584571265\\
0.555	0.353793780602429	0.763162841084325	0.697635496597417\\
0.555	0.366499761262991	0.773742494322337	0.692428303594619\\
0.555	0.379368276771857	0.784102202600443	0.687549132887069\\
0.555	0.392393618241769	0.794238304822092	0.682943679235328\\
0.555	0.405569970164763	0.804147337016097	0.678558662193046\\
0.555	0.418891418885081	0.813826036051075	0.674343787412872\\
0.555	0.432351961217133	0.823271342915545	0.670253187715552\\
0.555	0.445945513182735	0.832480405557787	0.666246357336158\\
0.555	0.459665918841652	0.841450581281375	0.662288623474184\\
0.555	0.473506959189401	0.850179438693979	0.658351221673191\\
0.555	0.487462361096317	0.85866475920876	0.654411055137887\\
0.555	0.501525806262096	0.866904538099407	0.650450223206468\\
0.555	0.515690940160279	0.874896985111485	0.646455401874959\\
0.555	0.529951380947623	0.882640524634437	0.642417151055155\\
0.555	0.544300728313782	0.890133795440129	0.638329210947585\\
0.555	0.558732572247415	0.897375649995373	0.634187835387498\\
0.555	0.57324050169555	0.904365153357279	0.629991194998375\\
0.555	0.587818113093873	0.911101581661729	0.625738868900924\\
0.555	0.602459018746568	0.917584420216504	0.62143143163314\\
0.555	0.617156855035314	0.923813361211857	0.617070132480554\\
0.555	0.631905290438127	0.92978830106243	0.612656657833838\\
0.555	0.6466980333399	0.935509337395467	0.608192963377087\\
0.555	0.661528839617671	0.940976765701173	0.603681161495105\\
0.555	0.676391519984904	0.946191075661951	0.599123449738608\\
0.555	0.691279947080357	0.951152947177946	0.594522067901589\\
0.555	0.706188062288412	0.955863246106974	0.589879273662389\\
0.555	0.721109882279076	0.960323019737426	0.585197329319937\\
0.555	0.736039505257225	0.964533492013186	0.580478494543978\\
0.555	0.75097111691199	0.968496058529893	0.575725022018313\\
0.555	0.765898996058536	0.972212281322139	0.570939154289558\\
0.555	0.780817519965826	0.975683883461272	0.566123121053175\\
0.555	0.795721169365275	0.97891274348356	0.561279136605558\\
0.555	0.810604533136476	0.981900889668376	0.556409397399199\\
0.555	0.825462312667457	0.984650494185936	0.551516079695735\\
0.555	0.840289325888138	0.987163867133889	0.546601337319639\\
0.555	0.855080510976839	0.989443450481768	0.541667299515138\\
0.555	0.86983092974082	0.991491811941914	0.536716068908599\\
0.555	0.884535770672926	0.993311638785082	0.531749719578141\\
0.555	0.899190351687411	0.994905731618374	0.526770295231796\\
0.555	0.913790122539012	0.996276998142666	0.521779807495103\\
0.555	0.928330666930241	0.997428446906011	0.51678023430868\\
0.555	0.942807704312699	0.998363181068902	0.511773518435871\\
0.555	0.95721709138901	0.999084392196567	0.506761566080168\\
0.555	0.971554823322703	0.999595354092743	0.501746245611877\\
0.555	0.985817034663989	0.999899416688637	0.496729386403005\\
0.555	1	1	0.491669087346904\\
0.57	0	0	0.710195569525497\\
0.57	0.000100583311362513	0.0141829653360114	0.71240460418299\\
0.57	0.000404645907256436	0.0284451766772965	0.714630690536905\\
0.57	0.000915607803433	0.0427829086109896	0.716916312064106\\
0.57	0.00163681893109844	0.057192295687301	0.719289019360234\\
0.57	0.00257155309398959	0.0716693330697585	0.721780367061016\\
0.57	0.00372300185733414	0.086209877460988	0.724425655492835\\
0.57	0.00509426838162598	0.100809648312589	0.727263413850304\\
0.57	0.00668836121491816	0.115464229327074	0.73033458997498\\
0.57	0.00850818805808555	0.13016907025918	0.733681422514709\\
0.57	0.0105565495182326	0.144919489023162	0.737345986077549\\
0.57	0.0128361328661109	0.159710674111862	0.741368418705642\\
0.57	0.0153495058140643	0.174537687332543	0.745784862958201\\
0.57	0.0180991103316243	0.189395466863524	0.750625176092311\\
0.57	0.0210872565164405	0.204278830634725	0.755910489868998\\
0.57	0.0243161165387281	0.219182480034174	0.761650724664453\\
0.57	0.0277877186778607	0.234101003941464	0.767842183867616\\
0.57	0.0315039414701067	0.24902888308801	0.77446537092041\\
0.57	0.0354665079868145	0.263960494742775	0.781483180780606\\
0.57	0.0396769802625738	0.278890117720924	0.788839618268073\\
0.57	0.0441367538930258	0.293811937711588	0.796459186325544\\
0.57	0.0488470528220538	0.308720052919643	0.804247066923454\\
0.57	0.0538089243380495	0.323608480015096	0.812090186161949\\
0.57	0.0590232342988274	0.338471160382329	0.819859213932694\\
0.57	0.064490662604533	0.3533019666601	0.827411499063639\\
0.57	0.0702116989375697	0.368094709561873	0.834594885813504\\
0.57	0.0761866387881433	0.382843144964686	0.841252300287177\\
0.57	0.0824155797834956	0.397540981253432	0.847226939729081\\
0.57	0.0888984183382709	0.412181886906127	0.852367847905041\\
0.57	0.0956348466427212	0.42675949830445	0.85653562002783\\
0.57	0.102624350004627	0.441267427752585	0.859607954619618\\
0.57	0.109866204559871	0.455699271686218	0.86148476026662\\
0.57	0.117359475365564	0.470048619052377	0.862092534251539\\
0.57	0.125103014888515	0.484309059839721	0.861387758068365\\
0.57	0.133095461900593	0.498474193737904	0.85935910089082\\
0.57	0.14133524079124	0.512537638903683	0.856028283765736\\
0.57	0.149820561306021	0.526493040810599	0.851449530870183\\
0.57	0.158549418718625	0.540334081158348	0.845707614739502\\
0.57	0.167519594442214	0.554054486817265	0.838914584343153\\
0.57	0.176728657084455	0.567648038782868	0.831205342370837\\
0.57	0.186173963948925	0.581108581114919	0.822732305417528\\
0.57	0.195852662983903	0.594430029835237	0.813659432944308\\
0.57	0.205761695177907	0.607606381758231	0.804155944104869\\
0.57	0.215897797399558	0.620631723228144	0.794390053406969\\
0.57	0.226257505677663	0.633500238737009	0.784523046062799\\
0.57	0.236837158915675	0.646206219397571	0.774703982861589\\
0.57	0.247632903032949	0.658744071245709	0.765065275211016\\
0.57	0.258640695523528	0.671108323347402	0.755719307772964\\
0.57	0.269856310421543	0.683293635685828	0.746756214005914\\
0.57	0.28127534366066	0.695294806804925	0.738242834604294\\
0.57	0.292893218813452	0.707106781186547	0.730222816007835\\
0.57	0.304705193195076	0.71872465633934	0.722717741087767\\
0.57	0.316706364314172	0.730143689578457	0.715729131143266\\
0.57	0.328891676652598	0.741359304476472	0.709241120561857\\
0.57	0.341255928754291	0.752367096967051	0.703223584571265\\
0.57	0.353793780602429	0.763162841084325	0.697635496597417\\
0.57	0.366499761262991	0.773742494322337	0.692428303594619\\
0.57	0.379368276771857	0.784102202600443	0.687549132887069\\
0.57	0.392393618241769	0.794238304822092	0.682943679235328\\
0.57	0.405569970164763	0.804147337016097	0.678558662193046\\
0.57	0.418891418885081	0.813826036051075	0.674343787412872\\
0.57	0.432351961217133	0.823271342915545	0.670253187715552\\
0.57	0.445945513182735	0.832480405557787	0.666246357336158\\
0.57	0.459665918841652	0.841450581281375	0.662288623474185\\
0.57	0.473506959189401	0.850179438693979	0.65835122167319\\
0.57	0.487462361096317	0.85866475920876	0.654411055137886\\
0.57	0.501525806262096	0.866904538099407	0.650450223206468\\
0.57	0.515690940160279	0.874896985111485	0.646455401874959\\
0.57	0.529951380947623	0.882640524634437	0.642417151055155\\
0.57	0.544300728313782	0.890133795440129	0.638329210947585\\
0.57	0.558732572247415	0.897375649995373	0.634187835387499\\
0.57	0.57324050169555	0.904365153357279	0.629991194998376\\
0.57	0.587818113093873	0.911101581661729	0.625738868900923\\
0.57	0.602459018746568	0.917584420216504	0.62143143163314\\
0.57	0.617156855035314	0.923813361211857	0.617070132480554\\
0.57	0.631905290438127	0.92978830106243	0.612656657833838\\
0.57	0.6466980333399	0.935509337395467	0.608192963377087\\
0.57	0.661528839617671	0.940976765701173	0.603681161495104\\
0.57	0.676391519984904	0.94619107566195	0.599123449738607\\
0.57	0.691279947080357	0.951152947177946	0.59452206790159\\
0.57	0.706188062288412	0.955863246106974	0.589879273662388\\
0.57	0.721109882279076	0.960323019737426	0.585197329319937\\
0.57	0.736039505257225	0.964533492013186	0.580478494543978\\
0.57	0.75097111691199	0.968496058529893	0.575725022018314\\
0.57	0.765898996058536	0.972212281322139	0.570939154289558\\
0.57	0.780817519965826	0.975683883461272	0.566123121053176\\
0.57	0.795721169365275	0.97891274348356	0.561279136605558\\
0.57	0.810604533136476	0.981900889668376	0.556409397399198\\
0.57	0.825462312667457	0.984650494185936	0.551516079695736\\
0.57	0.840289325888138	0.987163867133889	0.546601337319641\\
0.57	0.855080510976839	0.989443450481768	0.541667299515139\\
0.57	0.86983092974082	0.991491811941914	0.536716068908598\\
0.57	0.884535770672926	0.993311638785082	0.531749719578141\\
0.57	0.899190351687411	0.994905731618374	0.526770295231796\\
0.57	0.913790122539012	0.996276998142666	0.521779807495105\\
0.57	0.928330666930242	0.997428446906011	0.516780234308682\\
0.57	0.942807704312699	0.998363181068902	0.511773518435869\\
0.57	0.95721709138901	0.999084392196567	0.506761566080168\\
0.57	0.971554823322703	0.999595354092743	0.501746245611875\\
0.57	0.985817034663989	0.999899416688637	0.496729386403004\\
0.57	1	1	0.491669087346906\\
0.585	0	0	0.710195569525497\\
0.585	0.000100583311362513	0.0141829653360114	0.71240460418299\\
0.585	0.000404645907256436	0.0284451766772965	0.714630690536905\\
0.585	0.000915607803433	0.0427829086109896	0.716916312064106\\
0.585	0.00163681893109844	0.057192295687301	0.719289019360234\\
0.585	0.00257155309398959	0.0716693330697585	0.721780367061016\\
0.585	0.00372300185733414	0.086209877460988	0.724425655492835\\
0.585	0.00509426838162598	0.100809648312589	0.727263413850304\\
0.585	0.00668836121491816	0.115464229327074	0.73033458997498\\
0.585	0.00850818805808555	0.13016907025918	0.733681422514709\\
0.585	0.0105565495182326	0.144919489023162	0.737345986077549\\
0.585	0.0128361328661109	0.159710674111862	0.741368418705642\\
0.585	0.0153495058140643	0.174537687332543	0.745784862958201\\
0.585	0.0180991103316243	0.189395466863524	0.750625176092311\\
0.585	0.0210872565164405	0.204278830634725	0.755910489868998\\
0.585	0.0243161165387281	0.219182480034174	0.761650724664453\\
0.585	0.0277877186778607	0.234101003941464	0.767842183867616\\
0.585	0.0315039414701067	0.24902888308801	0.77446537092041\\
0.585	0.0354665079868145	0.263960494742775	0.781483180780606\\
0.585	0.0396769802625738	0.278890117720924	0.788839618268073\\
0.585	0.0441367538930258	0.293811937711588	0.796459186325544\\
0.585	0.0488470528220537	0.308720052919643	0.804247066923454\\
0.585	0.0538089243380495	0.323608480015096	0.812090186161949\\
0.585	0.0590232342988274	0.338471160382329	0.819859213932694\\
0.585	0.064490662604533	0.3533019666601	0.827411499063639\\
0.585	0.0702116989375697	0.368094709561873	0.834594885813505\\
0.585	0.0761866387881432	0.382843144964686	0.841252300287178\\
0.585	0.0824155797834956	0.397540981253432	0.847226939729081\\
0.585	0.0888984183382709	0.412181886906127	0.852367847905041\\
0.585	0.0956348466427213	0.42675949830445	0.856535620027829\\
0.585	0.102624350004627	0.441267427752585	0.859607954619618\\
0.585	0.109866204559871	0.455699271686219	0.861484760266623\\
0.585	0.117359475365564	0.470048619052377	0.862092534251541\\
0.585	0.125103014888515	0.484309059839721	0.861387758068368\\
0.585	0.133095461900593	0.498474193737904	0.859359100890819\\
0.585	0.14133524079124	0.512537638903683	0.856028283765736\\
0.585	0.149820561306021	0.526493040810599	0.851449530870182\\
0.585	0.158549418718625	0.540334081158348	0.845707614739507\\
0.585	0.167519594442214	0.554054486817265	0.838914584343154\\
0.585	0.176728657084455	0.567648038782868	0.831205342370836\\
0.585	0.186173963948925	0.581108581114919	0.822732305417527\\
0.585	0.195852662983903	0.594430029835237	0.813659432944309\\
0.585	0.205761695177907	0.607606381758231	0.804155944104869\\
0.585	0.215897797399558	0.620631723228144	0.794390053406969\\
0.585	0.226257505677663	0.633500238737009	0.784523046062799\\
0.585	0.236837158915675	0.646206219397571	0.774703982861589\\
0.585	0.247632903032949	0.658744071245709	0.765065275211016\\
0.585	0.258640695523528	0.671108323347402	0.755719307772964\\
0.585	0.269856310421543	0.683293635685828	0.746756214005914\\
0.585	0.28127534366066	0.695294806804925	0.738242834604294\\
0.585	0.292893218813452	0.707106781186547	0.730222816007836\\
0.585	0.304705193195075	0.71872465633934	0.722717741087767\\
0.585	0.316706364314172	0.730143689578457	0.715729131143265\\
0.585	0.328891676652598	0.741359304476472	0.709241120561857\\
0.585	0.341255928754291	0.752367096967051	0.703223584571265\\
0.585	0.353793780602429	0.763162841084325	0.697635496597417\\
0.585	0.366499761262991	0.773742494322337	0.692428303594619\\
0.585	0.379368276771857	0.784102202600443	0.687549132887069\\
0.585	0.392393618241769	0.794238304822092	0.682943679235328\\
0.585	0.405569970164763	0.804147337016097	0.678558662193046\\
0.585	0.418891418885081	0.813826036051075	0.674343787412872\\
0.585	0.432351961217133	0.823271342915545	0.670253187715552\\
0.585	0.445945513182735	0.832480405557787	0.666246357336157\\
0.585	0.459665918841652	0.841450581281375	0.662288623474185\\
0.585	0.473506959189401	0.850179438693979	0.658351221673192\\
0.585	0.487462361096317	0.85866475920876	0.654411055137886\\
0.585	0.501525806262096	0.866904538099407	0.650450223206468\\
0.585	0.515690940160279	0.874896985111485	0.646455401874959\\
0.585	0.529951380947623	0.882640524634437	0.642417151055155\\
0.585	0.544300728313782	0.890133795440129	0.638329210947585\\
0.585	0.558732572247415	0.897375649995373	0.634187835387498\\
0.585	0.57324050169555	0.904365153357279	0.629991194998376\\
0.585	0.587818113093873	0.911101581661729	0.625738868900924\\
0.585	0.602459018746568	0.917584420216504	0.62143143163314\\
0.585	0.617156855035314	0.923813361211857	0.617070132480554\\
0.585	0.631905290438127	0.92978830106243	0.612656657833838\\
0.585	0.6466980333399	0.935509337395467	0.608192963377087\\
0.585	0.661528839617671	0.940976765701173	0.603681161495105\\
0.585	0.676391519984904	0.946191075661951	0.599123449738606\\
0.585	0.691279947080357	0.951152947177946	0.594522067901592\\
0.585	0.706188062288412	0.955863246106974	0.589879273662388\\
0.585	0.721109882279076	0.960323019737426	0.585197329319934\\
0.585	0.736039505257225	0.964533492013186	0.580478494543975\\
0.585	0.75097111691199	0.968496058529893	0.575725022018313\\
0.585	0.765898996058536	0.972212281322139	0.570939154289561\\
0.585	0.780817519965826	0.975683883461272	0.566123121053175\\
0.585	0.795721169365275	0.97891274348356	0.561279136605558\\
0.585	0.810604533136476	0.981900889668376	0.556409397399199\\
0.585	0.825462312667457	0.984650494185936	0.551516079695737\\
0.585	0.840289325888138	0.987163867133889	0.546601337319639\\
0.585	0.855080510976839	0.989443450481768	0.541667299515138\\
0.585	0.86983092974082	0.991491811941914	0.536716068908598\\
0.585	0.884535770672926	0.993311638785082	0.531749719578143\\
0.585	0.899190351687411	0.994905731618374	0.526770295231795\\
0.585	0.913790122539012	0.996276998142666	0.521779807495105\\
0.585	0.928330666930241	0.997428446906011	0.516780234308685\\
0.585	0.942807704312699	0.998363181068902	0.511773518435871\\
0.585	0.95721709138901	0.999084392196567	0.506761566080167\\
0.585	0.971554823322704	0.999595354092744	0.501746245611876\\
0.585	0.985817034663989	0.999899416688637	0.496729386403002\\
0.585	1	1	0.491669087346908\\
0.6	0	0	0.710195569525497\\
0.6	0.000100583311362513	0.0141829653360114	0.71240460418299\\
0.6	0.000404645907256436	0.0284451766772965	0.714630690536905\\
0.6	0.000915607803433	0.0427829086109896	0.716916312064106\\
0.6	0.00163681893109844	0.057192295687301	0.719289019360234\\
0.6	0.00257155309398959	0.0716693330697585	0.721780367061016\\
0.6	0.00372300185733414	0.086209877460988	0.724425655492835\\
0.6	0.00509426838162598	0.100809648312589	0.727263413850304\\
0.6	0.00668836121491816	0.115464229327074	0.73033458997498\\
0.6	0.00850818805808555	0.13016907025918	0.733681422514709\\
0.6	0.0105565495182326	0.144919489023162	0.737345986077549\\
0.6	0.0128361328661109	0.159710674111862	0.741368418705642\\
0.6	0.0153495058140643	0.174537687332543	0.745784862958201\\
0.6	0.0180991103316243	0.189395466863524	0.750625176092311\\
0.6	0.0210872565164405	0.204278830634725	0.755910489868998\\
0.6	0.0243161165387281	0.219182480034174	0.761650724664453\\
0.6	0.0277877186778607	0.234101003941464	0.767842183867616\\
0.6	0.0315039414701067	0.24902888308801	0.77446537092041\\
0.6	0.0354665079868145	0.263960494742775	0.781483180780606\\
0.6	0.0396769802625738	0.278890117720924	0.788839618268073\\
0.6	0.0441367538930258	0.293811937711588	0.796459186325544\\
0.6	0.0488470528220537	0.308720052919643	0.804247066923454\\
0.6	0.0538089243380495	0.323608480015096	0.81209018616195\\
0.6	0.0590232342988274	0.338471160382329	0.819859213932694\\
0.6	0.064490662604533	0.3533019666601	0.82741149906364\\
0.6	0.0702116989375697	0.368094709561873	0.834594885813504\\
0.6	0.0761866387881432	0.382843144964686	0.84125230028718\\
0.6	0.0824155797834956	0.397540981253432	0.847226939729081\\
0.6	0.0888984183382709	0.412181886906127	0.852367847905042\\
0.6	0.0956348466427212	0.42675949830445	0.856535620027828\\
0.6	0.102624350004627	0.441267427752584	0.859607954619617\\
0.6	0.109866204559871	0.455699271686219	0.86148476026662\\
0.6	0.117359475365564	0.470048619052377	0.862092534251541\\
0.6	0.125103014888515	0.484309059839721	0.86138775806837\\
0.6	0.133095461900593	0.498474193737904	0.859359100890821\\
0.6	0.14133524079124	0.512537638903683	0.856028283765736\\
0.6	0.149820561306021	0.526493040810599	0.851449530870183\\
0.6	0.158549418718625	0.540334081158348	0.845707614739509\\
0.6	0.167519594442214	0.554054486817265	0.838914584343148\\
0.6	0.176728657084455	0.567648038782868	0.831205342370834\\
0.6	0.186173963948925	0.581108581114919	0.822732305417528\\
0.6	0.195852662983903	0.594430029835237	0.813659432944309\\
0.6	0.205761695177907	0.607606381758231	0.804155944104868\\
0.6	0.215897797399558	0.620631723228144	0.794390053406971\\
0.6	0.226257505677663	0.633500238737009	0.784523046062799\\
0.6	0.236837158915675	0.646206219397571	0.774703982861589\\
0.6	0.247632903032949	0.658744071245709	0.765065275211016\\
0.6	0.258640695523528	0.671108323347402	0.755719307772964\\
0.6	0.269856310421543	0.683293635685828	0.746756214005914\\
0.6	0.28127534366066	0.695294806804925	0.738242834604294\\
0.6	0.292893218813452	0.707106781186547	0.730222816007836\\
0.6	0.304705193195075	0.71872465633934	0.722717741087767\\
0.6	0.316706364314172	0.730143689578457	0.715729131143265\\
0.6	0.328891676652598	0.741359304476472	0.709241120561857\\
0.6	0.341255928754291	0.752367096967051	0.703223584571265\\
0.6	0.353793780602429	0.763162841084325	0.697635496597417\\
0.6	0.366499761262991	0.773742494322337	0.692428303594619\\
0.6	0.379368276771857	0.784102202600443	0.68754913288707\\
0.6	0.392393618241769	0.794238304822092	0.682943679235328\\
0.6	0.405569970164763	0.804147337016097	0.678558662193046\\
0.6	0.418891418885081	0.813826036051075	0.674343787412872\\
0.6	0.432351961217133	0.823271342915545	0.670253187715552\\
0.6	0.445945513182735	0.832480405557787	0.666246357336158\\
0.6	0.459665918841652	0.841450581281375	0.662288623474184\\
0.6	0.473506959189401	0.850179438693979	0.658351221673192\\
0.6	0.487462361096317	0.85866475920876	0.654411055137887\\
0.6	0.501525806262096	0.866904538099407	0.650450223206467\\
0.6	0.515690940160279	0.874896985111485	0.646455401874959\\
0.6	0.529951380947623	0.882640524634437	0.642417151055155\\
0.6	0.544300728313782	0.890133795440129	0.638329210947585\\
0.6	0.558732572247415	0.897375649995373	0.634187835387499\\
0.6	0.57324050169555	0.904365153357279	0.629991194998376\\
0.6	0.587818113093873	0.911101581661729	0.625738868900924\\
0.6	0.602459018746568	0.917584420216504	0.62143143163314\\
0.6	0.617156855035314	0.923813361211857	0.617070132480554\\
0.6	0.631905290438127	0.92978830106243	0.612656657833838\\
0.6	0.6466980333399	0.935509337395467	0.608192963377087\\
0.6	0.661528839617671	0.940976765701173	0.603681161495105\\
0.6	0.676391519984904	0.946191075661951	0.599123449738607\\
0.6	0.691279947080357	0.951152947177946	0.594522067901591\\
0.6	0.706188062288412	0.955863246106974	0.589879273662389\\
0.6	0.721109882279076	0.960323019737426	0.585197329319934\\
0.6	0.736039505257225	0.964533492013186	0.580478494543978\\
0.6	0.75097111691199	0.968496058529893	0.575725022018311\\
0.6	0.765898996058536	0.972212281322139	0.570939154289557\\
0.6	0.780817519965826	0.975683883461272	0.566123121053178\\
0.6	0.795721169365275	0.97891274348356	0.561279136605558\\
0.6	0.810604533136476	0.981900889668376	0.556409397399196\\
0.6	0.825462312667457	0.984650494185936	0.551516079695736\\
0.6	0.840289325888138	0.987163867133889	0.546601337319642\\
0.6	0.855080510976839	0.989443450481768	0.541667299515139\\
0.6	0.86983092974082	0.991491811941914	0.536716068908595\\
0.6	0.884535770672926	0.993311638785082	0.531749719578142\\
0.6	0.899190351687411	0.994905731618374	0.526770295231796\\
0.6	0.913790122539012	0.996276998142666	0.521779807495103\\
0.6	0.928330666930242	0.997428446906011	0.516780234308682\\
0.6	0.942807704312699	0.998363181068902	0.511773518435873\\
0.6	0.95721709138901	0.999084392196567	0.506761566080169\\
0.6	0.971554823322703	0.999595354092743	0.501746245611875\\
0.6	0.985817034663989	0.999899416688637	0.496729386403004\\
0.6	1	1	0.491669087346906\\
0.615	0	0	0.710195569525497\\
0.615	0.000100583311362513	0.0141829653360114	0.71240460418299\\
0.615	0.000404645907256436	0.0284451766772965	0.714630690536905\\
0.615	0.000915607803433	0.0427829086109896	0.716916312064106\\
0.615	0.00163681893109844	0.057192295687301	0.719289019360234\\
0.615	0.00257155309398959	0.0716693330697585	0.721780367061016\\
0.615	0.00372300185733414	0.0862098774609879	0.724425655492835\\
0.615	0.00509426838162599	0.100809648312589	0.727263413850304\\
0.615	0.00668836121491816	0.115464229327074	0.73033458997498\\
0.615	0.00850818805808555	0.13016907025918	0.733681422514709\\
0.615	0.0105565495182326	0.144919489023162	0.737345986077549\\
0.615	0.0128361328661109	0.159710674111862	0.741368418705642\\
0.615	0.0153495058140643	0.174537687332543	0.745784862958201\\
0.615	0.0180991103316243	0.189395466863524	0.750625176092311\\
0.615	0.0210872565164405	0.204278830634725	0.755910489868998\\
0.615	0.0243161165387281	0.219182480034174	0.761650724664453\\
0.615	0.0277877186778607	0.234101003941464	0.767842183867616\\
0.615	0.0315039414701067	0.24902888308801	0.77446537092041\\
0.615	0.0354665079868145	0.263960494742775	0.781483180780606\\
0.615	0.0396769802625737	0.278890117720924	0.788839618268073\\
0.615	0.0441367538930258	0.293811937711588	0.796459186325544\\
0.615	0.0488470528220538	0.308720052919643	0.804247066923455\\
0.615	0.0538089243380495	0.323608480015096	0.81209018616195\\
0.615	0.0590232342988274	0.338471160382329	0.819859213932693\\
0.615	0.064490662604533	0.3533019666601	0.827411499063639\\
0.615	0.0702116989375697	0.368094709561873	0.834594885813505\\
0.615	0.0761866387881432	0.382843144964686	0.841252300287179\\
0.615	0.0824155797834956	0.397540981253432	0.847226939729081\\
0.615	0.0888984183382709	0.412181886906127	0.85236784790504\\
0.615	0.0956348466427212	0.42675949830445	0.85653562002783\\
0.615	0.102624350004627	0.441267427752584	0.859607954619617\\
0.615	0.109866204559871	0.455699271686219	0.861484760266621\\
0.615	0.117359475365564	0.470048619052377	0.862092534251542\\
0.615	0.125103014888515	0.484309059839721	0.861387758068371\\
0.615	0.133095461900593	0.498474193737904	0.859359100890821\\
0.615	0.14133524079124	0.512537638903683	0.856028283765738\\
0.615	0.149820561306021	0.526493040810599	0.851449530870183\\
0.615	0.158549418718625	0.540334081158348	0.845707614739502\\
0.615	0.167519594442214	0.554054486817265	0.838914584343148\\
0.615	0.176728657084455	0.567648038782867	0.831205342370836\\
0.615	0.186173963948925	0.581108581114919	0.822732305417529\\
0.615	0.195852662983903	0.594430029835237	0.813659432944308\\
0.615	0.205761695177907	0.607606381758231	0.804155944104869\\
0.615	0.215897797399558	0.620631723228144	0.794390053406971\\
0.615	0.226257505677663	0.633500238737009	0.784523046062798\\
0.615	0.236837158915675	0.646206219397571	0.774703982861588\\
0.615	0.247632903032949	0.658744071245709	0.765065275211016\\
0.615	0.258640695523528	0.671108323347402	0.755719307772964\\
0.615	0.269856310421543	0.683293635685828	0.746756214005914\\
0.615	0.28127534366066	0.695294806804925	0.738242834604295\\
0.615	0.292893218813452	0.707106781186547	0.730222816007836\\
0.615	0.304705193195075	0.71872465633934	0.722717741087767\\
0.615	0.316706364314172	0.730143689578457	0.715729131143265\\
0.615	0.328891676652598	0.741359304476472	0.709241120561859\\
0.615	0.341255928754291	0.752367096967051	0.703223584571265\\
0.615	0.353793780602429	0.763162841084325	0.697635496597416\\
0.615	0.366499761262991	0.773742494322337	0.692428303594618\\
0.615	0.379368276771857	0.784102202600442	0.687549132887069\\
0.615	0.392393618241769	0.794238304822092	0.682943679235329\\
0.615	0.405569970164763	0.804147337016097	0.678558662193047\\
0.615	0.418891418885081	0.813826036051075	0.674343787412872\\
0.615	0.432351961217133	0.823271342915545	0.670253187715551\\
0.615	0.445945513182735	0.832480405557787	0.666246357336159\\
0.615	0.459665918841652	0.841450581281375	0.662288623474185\\
0.615	0.473506959189401	0.850179438693979	0.658351221673189\\
0.615	0.487462361096317	0.85866475920876	0.654411055137887\\
0.615	0.501525806262096	0.866904538099407	0.650450223206468\\
0.615	0.515690940160279	0.874896985111485	0.646455401874959\\
0.615	0.529951380947623	0.882640524634437	0.642417151055156\\
0.615	0.544300728313782	0.890133795440129	0.638329210947585\\
0.615	0.558732572247415	0.897375649995372	0.634187835387498\\
0.615	0.57324050169555	0.904365153357279	0.629991194998377\\
0.615	0.587818113093873	0.911101581661729	0.625738868900924\\
0.615	0.602459018746568	0.917584420216504	0.62143143163314\\
0.615	0.617156855035314	0.923813361211857	0.617070132480554\\
0.615	0.631905290438127	0.92978830106243	0.612656657833838\\
0.615	0.6466980333399	0.935509337395467	0.608192963377087\\
0.615	0.661528839617671	0.940976765701173	0.603681161495105\\
0.615	0.676391519984904	0.946191075661951	0.599123449738608\\
0.615	0.691279947080357	0.951152947177946	0.59452206790159\\
0.615	0.706188062288412	0.955863246106974	0.589879273662388\\
0.615	0.721109882279076	0.960323019737426	0.585197329319935\\
0.615	0.736039505257225	0.964533492013186	0.580478494543978\\
0.615	0.75097111691199	0.968496058529893	0.575725022018316\\
0.615	0.765898996058536	0.972212281322139	0.570939154289556\\
0.615	0.780817519965825	0.975683883461272	0.566123121053173\\
0.615	0.795721169365275	0.97891274348356	0.56127913660556\\
0.615	0.810604533136476	0.981900889668376	0.556409397399198\\
0.615	0.825462312667457	0.984650494185936	0.551516079695736\\
0.615	0.840289325888138	0.987163867133889	0.546601337319642\\
0.615	0.855080510976839	0.989443450481768	0.541667299515139\\
0.615	0.86983092974082	0.991491811941914	0.536716068908595\\
0.615	0.884535770672926	0.993311638785082	0.531749719578141\\
0.615	0.899190351687411	0.994905731618374	0.526770295231795\\
0.615	0.913790122539012	0.996276998142666	0.521779807495104\\
0.615	0.928330666930242	0.997428446906011	0.516780234308685\\
0.615	0.942807704312699	0.998363181068902	0.511773518435872\\
0.615	0.95721709138901	0.999084392196567	0.506761566080166\\
0.615	0.971554823322703	0.999595354092743	0.501746245611875\\
0.615	0.985817034663989	0.999899416688637	0.496729386403005\\
0.615	1	1	0.491669087346904\\
0.63	0	0	0.710195569525497\\
0.63	0.000100583311362513	0.0141829653360114	0.71240460418299\\
0.63	0.000404645907256436	0.0284451766772965	0.714630690536905\\
0.63	0.000915607803433	0.0427829086109896	0.716916312064106\\
0.63	0.00163681893109844	0.057192295687301	0.719289019360234\\
0.63	0.00257155309398959	0.0716693330697584	0.721780367061016\\
0.63	0.00372300185733414	0.086209877460988	0.724425655492835\\
0.63	0.00509426838162599	0.100809648312589	0.727263413850304\\
0.63	0.00668836121491816	0.115464229327074	0.73033458997498\\
0.63	0.00850818805808555	0.13016907025918	0.733681422514709\\
0.63	0.0105565495182326	0.144919489023162	0.737345986077549\\
0.63	0.0128361328661109	0.159710674111862	0.741368418705642\\
0.63	0.0153495058140643	0.174537687332543	0.745784862958202\\
0.63	0.0180991103316243	0.189395466863524	0.750625176092311\\
0.63	0.0210872565164405	0.204278830634725	0.755910489868999\\
0.63	0.0243161165387281	0.219182480034174	0.761650724664453\\
0.63	0.0277877186778607	0.234101003941464	0.767842183867616\\
0.63	0.0315039414701067	0.24902888308801	0.77446537092041\\
0.63	0.0354665079868145	0.263960494742775	0.781483180780606\\
0.63	0.0396769802625737	0.278890117720924	0.788839618268074\\
0.63	0.0441367538930258	0.293811937711588	0.796459186325544\\
0.63	0.0488470528220538	0.308720052919643	0.804247066923455\\
0.63	0.0538089243380495	0.323608480015096	0.812090186161949\\
0.63	0.0590232342988274	0.338471160382329	0.819859213932694\\
0.63	0.064490662604533	0.3533019666601	0.827411499063638\\
0.63	0.0702116989375697	0.368094709561873	0.834594885813505\\
0.63	0.0761866387881432	0.382843144964686	0.84125230028718\\
0.63	0.0824155797834956	0.397540981253432	0.847226939729081\\
0.63	0.0888984183382709	0.412181886906127	0.85236784790504\\
0.63	0.0956348466427212	0.42675949830445	0.85653562002783\\
0.63	0.102624350004627	0.441267427752584	0.859607954619617\\
0.63	0.109866204559871	0.455699271686219	0.861484760266621\\
0.63	0.117359475365564	0.470048619052377	0.862092534251541\\
0.63	0.125103014888515	0.484309059839721	0.86138775806837\\
0.63	0.133095461900593	0.498474193737904	0.859359100890818\\
0.63	0.14133524079124	0.512537638903683	0.856028283765736\\
0.63	0.149820561306021	0.526493040810599	0.851449530870183\\
0.63	0.158549418718625	0.540334081158348	0.845707614739505\\
0.63	0.167519594442213	0.554054486817265	0.838914584343151\\
0.63	0.176728657084455	0.567648038782867	0.831205342370838\\
0.63	0.186173963948925	0.581108581114919	0.822732305417528\\
0.63	0.195852662983903	0.594430029835237	0.813659432944308\\
0.63	0.205761695177907	0.607606381758231	0.80415594410487\\
0.63	0.215897797399558	0.620631723228144	0.79439005340697\\
0.63	0.226257505677663	0.633500238737009	0.7845230460628\\
0.63	0.236837158915675	0.646206219397571	0.774703982861588\\
0.63	0.247632903032949	0.658744071245709	0.765065275211016\\
0.63	0.258640695523528	0.671108323347402	0.755719307772964\\
0.63	0.269856310421543	0.683293635685828	0.746756214005914\\
0.63	0.28127534366066	0.695294806804925	0.738242834604294\\
0.63	0.292893218813452	0.707106781186547	0.730222816007835\\
0.63	0.304705193195075	0.71872465633934	0.722717741087767\\
0.63	0.316706364314172	0.730143689578457	0.715729131143265\\
0.63	0.328891676652598	0.741359304476472	0.709241120561857\\
0.63	0.341255928754291	0.752367096967051	0.703223584571265\\
0.63	0.353793780602429	0.763162841084325	0.697635496597417\\
0.63	0.366499761262991	0.773742494322337	0.692428303594619\\
0.63	0.379368276771856	0.784102202600442	0.687549132887069\\
0.63	0.392393618241769	0.794238304822092	0.682943679235328\\
0.63	0.405569970164763	0.804147337016097	0.678558662193047\\
0.63	0.418891418885081	0.813826036051075	0.674343787412872\\
0.63	0.432351961217133	0.823271342915545	0.670253187715551\\
0.63	0.445945513182735	0.832480405557787	0.666246357336157\\
0.63	0.459665918841652	0.841450581281375	0.662288623474186\\
0.63	0.473506959189401	0.850179438693979	0.658351221673192\\
0.63	0.487462361096317	0.85866475920876	0.654411055137886\\
0.63	0.501525806262096	0.866904538099407	0.650450223206467\\
0.63	0.51569094016028	0.874896985111485	0.646455401874959\\
0.63	0.529951380947623	0.882640524634437	0.642417151055156\\
0.63	0.544300728313782	0.890133795440129	0.638329210947585\\
0.63	0.558732572247415	0.897375649995372	0.634187835387497\\
0.63	0.57324050169555	0.904365153357279	0.629991194998377\\
0.63	0.587818113093873	0.911101581661729	0.625738868900924\\
0.63	0.602459018746568	0.917584420216504	0.621431431633139\\
0.63	0.617156855035314	0.923813361211857	0.617070132480554\\
0.63	0.631905290438127	0.92978830106243	0.612656657833839\\
0.63	0.6466980333399	0.935509337395467	0.608192963377087\\
0.63	0.661528839617671	0.940976765701173	0.603681161495105\\
0.63	0.676391519984904	0.946191075661951	0.599123449738608\\
0.63	0.691279947080357	0.951152947177946	0.594522067901592\\
0.63	0.706188062288412	0.955863246106974	0.589879273662387\\
0.63	0.721109882279076	0.960323019737426	0.585197329319934\\
0.63	0.736039505257225	0.964533492013186	0.580478494543975\\
0.63	0.75097111691199	0.968496058529893	0.575725022018315\\
0.63	0.765898996058536	0.972212281322139	0.570939154289562\\
0.63	0.780817519965826	0.975683883461272	0.566123121053175\\
0.63	0.795721169365275	0.97891274348356	0.561279136605558\\
0.63	0.810604533136476	0.981900889668376	0.556409397399195\\
0.63	0.825462312667457	0.984650494185936	0.551516079695734\\
0.63	0.840289325888138	0.987163867133889	0.546601337319643\\
0.63	0.855080510976839	0.989443450481768	0.541667299515141\\
0.63	0.86983092974082	0.991491811941915	0.536716068908596\\
0.63	0.884535770672926	0.993311638785082	0.531749719578139\\
0.63	0.899190351687411	0.994905731618374	0.526770295231796\\
0.63	0.913790122539012	0.996276998142666	0.521779807495104\\
0.63	0.928330666930242	0.997428446906011	0.516780234308682\\
0.63	0.942807704312699	0.998363181068902	0.511773518435872\\
0.63	0.95721709138901	0.999084392196567	0.506761566080168\\
0.63	0.971554823322703	0.999595354092743	0.501746245611874\\
0.63	0.985817034663989	0.999899416688637	0.496729386403005\\
0.63	1	1	0.491669087346904\\
0.645	0	0	0.710195569525497\\
0.645	0.000100583311362513	0.0141829653360114	0.71240460418299\\
0.645	0.000404645907256436	0.0284451766772965	0.714630690536905\\
0.645	0.000915607803433	0.0427829086109896	0.716916312064106\\
0.645	0.00163681893109844	0.057192295687301	0.719289019360234\\
0.645	0.00257155309398959	0.0716693330697585	0.721780367061016\\
0.645	0.00372300185733414	0.0862098774609879	0.724425655492835\\
0.645	0.00509426838162599	0.100809648312589	0.727263413850304\\
0.645	0.00668836121491816	0.115464229327074	0.73033458997498\\
0.645	0.00850818805808555	0.13016907025918	0.733681422514709\\
0.645	0.0105565495182326	0.144919489023162	0.737345986077549\\
0.645	0.0128361328661109	0.159710674111862	0.741368418705642\\
0.645	0.0153495058140643	0.174537687332543	0.745784862958202\\
0.645	0.0180991103316243	0.189395466863524	0.750625176092311\\
0.645	0.0210872565164405	0.204278830634725	0.755910489868998\\
0.645	0.0243161165387281	0.219182480034174	0.761650724664453\\
0.645	0.0277877186778607	0.234101003941464	0.767842183867616\\
0.645	0.0315039414701067	0.24902888308801	0.77446537092041\\
0.645	0.0354665079868145	0.263960494742775	0.781483180780606\\
0.645	0.0396769802625738	0.278890117720924	0.788839618268073\\
0.645	0.0441367538930258	0.293811937711588	0.796459186325545\\
0.645	0.0488470528220538	0.308720052919643	0.804247066923455\\
0.645	0.0538089243380495	0.323608480015096	0.812090186161949\\
0.645	0.0590232342988274	0.338471160382329	0.819859213932694\\
0.645	0.064490662604533	0.3533019666601	0.827411499063639\\
0.645	0.0702116989375697	0.368094709561873	0.834594885813505\\
0.645	0.0761866387881433	0.382843144964686	0.84125230028718\\
0.645	0.0824155797834956	0.397540981253432	0.847226939729081\\
0.645	0.0888984183382709	0.412181886906127	0.852367847905042\\
0.645	0.0956348466427212	0.42675949830445	0.856535620027831\\
0.645	0.102624350004627	0.441267427752584	0.859607954619617\\
0.645	0.109866204559871	0.455699271686219	0.861484760266622\\
0.645	0.117359475365564	0.470048619052377	0.862092534251541\\
0.645	0.125103014888515	0.484309059839721	0.861387758068367\\
0.645	0.133095461900593	0.498474193737904	0.85935910089082\\
0.645	0.14133524079124	0.512537638903683	0.856028283765735\\
0.645	0.149820561306021	0.526493040810599	0.851449530870183\\
0.645	0.158549418718625	0.540334081158348	0.845707614739504\\
0.645	0.167519594442213	0.554054486817265	0.83891458434315\\
0.645	0.176728657084455	0.567648038782867	0.831205342370837\\
0.645	0.186173963948925	0.581108581114919	0.822732305417528\\
0.645	0.195852662983903	0.594430029835237	0.813659432944308\\
0.645	0.205761695177907	0.607606381758231	0.804155944104871\\
0.645	0.215897797399558	0.620631723228144	0.794390053406972\\
0.645	0.226257505677663	0.633500238737009	0.784523046062798\\
0.645	0.236837158915675	0.646206219397571	0.774703982861587\\
0.645	0.247632903032949	0.658744071245709	0.765065275211016\\
0.645	0.258640695523528	0.671108323347402	0.755719307772964\\
0.645	0.269856310421543	0.683293635685828	0.746756214005914\\
0.645	0.28127534366066	0.695294806804925	0.738242834604294\\
0.645	0.292893218813452	0.707106781186547	0.730222816007837\\
0.645	0.304705193195075	0.71872465633934	0.722717741087766\\
0.645	0.316706364314172	0.730143689578457	0.715729131143264\\
0.645	0.328891676652598	0.741359304476472	0.709241120561859\\
0.645	0.341255928754291	0.752367096967051	0.703223584571265\\
0.645	0.353793780602429	0.763162841084325	0.697635496597416\\
0.645	0.366499761262991	0.773742494322337	0.692428303594619\\
0.645	0.379368276771856	0.784102202600442	0.68754913288707\\
0.645	0.392393618241769	0.794238304822092	0.682943679235328\\
0.645	0.405569970164763	0.804147337016097	0.678558662193046\\
0.645	0.418891418885081	0.813826036051075	0.674343787412872\\
0.645	0.432351961217133	0.823271342915545	0.670253187715552\\
0.645	0.445945513182735	0.832480405557787	0.666246357336158\\
0.645	0.459665918841652	0.841450581281375	0.662288623474184\\
0.645	0.473506959189401	0.850179438693979	0.658351221673191\\
0.645	0.487462361096317	0.85866475920876	0.654411055137887\\
0.645	0.501525806262096	0.866904538099407	0.650450223206468\\
0.645	0.515690940160279	0.874896985111485	0.646455401874959\\
0.645	0.529951380947623	0.882640524634437	0.642417151055155\\
0.645	0.544300728313782	0.890133795440129	0.638329210947585\\
0.645	0.558732572247415	0.897375649995373	0.634187835387498\\
0.645	0.57324050169555	0.904365153357279	0.629991194998376\\
0.645	0.587818113093873	0.911101581661729	0.625738868900924\\
0.645	0.602459018746568	0.917584420216504	0.62143143163314\\
0.645	0.617156855035314	0.923813361211857	0.617070132480553\\
0.645	0.631905290438127	0.92978830106243	0.612656657833838\\
0.645	0.6466980333399	0.935509337395467	0.608192963377087\\
0.645	0.661528839617671	0.940976765701173	0.603681161495105\\
0.645	0.676391519984904	0.946191075661951	0.599123449738608\\
0.645	0.691279947080357	0.951152947177946	0.594522067901592\\
0.645	0.706188062288412	0.955863246106974	0.589879273662388\\
0.645	0.721109882279076	0.960323019737426	0.585197329319933\\
0.645	0.736039505257225	0.964533492013185	0.580478494543975\\
0.645	0.75097111691199	0.968496058529893	0.575725022018315\\
0.645	0.765898996058536	0.972212281322139	0.57093915428956\\
0.645	0.780817519965826	0.975683883461272	0.566123121053176\\
0.645	0.795721169365275	0.97891274348356	0.56127913660556\\
0.645	0.810604533136476	0.981900889668376	0.556409397399197\\
0.645	0.825462312667457	0.984650494185936	0.551516079695735\\
0.645	0.840289325888138	0.987163867133889	0.546601337319642\\
0.645	0.855080510976839	0.989443450481768	0.541667299515138\\
0.645	0.86983092974082	0.991491811941914	0.536716068908598\\
0.645	0.884535770672926	0.993311638785082	0.53174971957814\\
0.645	0.899190351687411	0.994905731618374	0.526770295231793\\
0.645	0.913790122539012	0.996276998142666	0.521779807495105\\
0.645	0.928330666930242	0.997428446906011	0.516780234308685\\
0.645	0.942807704312699	0.998363181068902	0.511773518435872\\
0.645	0.95721709138901	0.999084392196567	0.506761566080168\\
0.645	0.971554823322703	0.999595354092743	0.501746245611874\\
0.645	0.985817034663989	0.999899416688637	0.496729386403004\\
0.645	1	1	0.491669087346906\\
0.66	0	0	0.710195569525497\\
0.66	0.000100583311362513	0.0141829653360114	0.71240460418299\\
0.66	0.000404645907256436	0.0284451766772965	0.714630690536905\\
0.66	0.000915607803433	0.0427829086109896	0.716916312064106\\
0.66	0.00163681893109844	0.057192295687301	0.719289019360234\\
0.66	0.00257155309398959	0.0716693330697584	0.721780367061016\\
0.66	0.00372300185733414	0.0862098774609879	0.724425655492835\\
0.66	0.00509426838162598	0.100809648312589	0.727263413850304\\
0.66	0.00668836121491816	0.115464229327074	0.73033458997498\\
0.66	0.00850818805808555	0.13016907025918	0.733681422514709\\
0.66	0.0105565495182326	0.144919489023162	0.737345986077549\\
0.66	0.0128361328661109	0.159710674111862	0.741368418705642\\
0.66	0.0153495058140643	0.174537687332543	0.745784862958201\\
0.66	0.0180991103316243	0.189395466863524	0.750625176092311\\
0.66	0.0210872565164405	0.204278830634725	0.755910489868998\\
0.66	0.0243161165387281	0.219182480034174	0.761650724664453\\
0.66	0.0277877186778607	0.234101003941464	0.767842183867616\\
0.66	0.0315039414701067	0.24902888308801	0.77446537092041\\
0.66	0.0354665079868145	0.263960494742775	0.781483180780606\\
0.66	0.0396769802625737	0.278890117720924	0.788839618268073\\
0.66	0.0441367538930258	0.293811937711588	0.796459186325544\\
0.66	0.0488470528220538	0.308720052919643	0.804247066923454\\
0.66	0.0538089243380495	0.323608480015096	0.812090186161949\\
0.66	0.0590232342988274	0.338471160382329	0.819859213932694\\
0.66	0.064490662604533	0.3533019666601	0.827411499063639\\
0.66	0.0702116989375697	0.368094709561873	0.834594885813505\\
0.66	0.0761866387881432	0.382843144964686	0.841252300287178\\
0.66	0.0824155797834956	0.397540981253432	0.847226939729081\\
0.66	0.0888984183382709	0.412181886906127	0.852367847905042\\
0.66	0.0956348466427212	0.42675949830445	0.856535620027829\\
0.66	0.102624350004627	0.441267427752584	0.859607954619618\\
0.66	0.109866204559871	0.455699271686219	0.861484760266622\\
0.66	0.117359475365564	0.470048619052377	0.862092534251539\\
0.66	0.125103014888515	0.484309059839721	0.861387758068367\\
0.66	0.133095461900593	0.498474193737904	0.859359100890822\\
0.66	0.14133524079124	0.512537638903683	0.85602828376574\\
0.66	0.149820561306021	0.526493040810599	0.85144953087018\\
0.66	0.158549418718625	0.540334081158348	0.845707614739505\\
0.66	0.167519594442213	0.554054486817265	0.838914584343152\\
0.66	0.176728657084455	0.567648038782867	0.831205342370837\\
0.66	0.186173963948925	0.581108581114919	0.822732305417528\\
0.66	0.195852662983903	0.594430029835237	0.813659432944309\\
0.66	0.205761695177907	0.607606381758231	0.80415594410487\\
0.66	0.215897797399558	0.620631723228144	0.79439005340697\\
0.66	0.226257505677663	0.633500238737009	0.784523046062796\\
0.66	0.236837158915675	0.646206219397571	0.774703982861589\\
0.66	0.247632903032949	0.658744071245709	0.765065275211016\\
0.66	0.258640695523528	0.671108323347402	0.755719307772964\\
0.66	0.269856310421543	0.683293635685828	0.746756214005914\\
0.66	0.28127534366066	0.695294806804925	0.738242834604295\\
0.66	0.292893218813452	0.707106781186547	0.730222816007834\\
0.66	0.304705193195075	0.71872465633934	0.722717741087766\\
0.66	0.316706364314172	0.730143689578457	0.715729131143267\\
0.66	0.328891676652598	0.741359304476472	0.709241120561859\\
0.66	0.341255928754291	0.752367096967051	0.703223584571265\\
0.66	0.353793780602429	0.763162841084325	0.697635496597416\\
0.66	0.366499761262991	0.773742494322337	0.692428303594619\\
0.66	0.379368276771856	0.784102202600442	0.68754913288707\\
0.66	0.392393618241769	0.794238304822092	0.682943679235329\\
0.66	0.405569970164763	0.804147337016097	0.678558662193046\\
0.66	0.418891418885081	0.813826036051075	0.674343787412872\\
0.66	0.432351961217133	0.823271342915545	0.670253187715551\\
0.66	0.445945513182735	0.832480405557787	0.666246357336157\\
0.66	0.459665918841652	0.841450581281375	0.662288623474184\\
0.66	0.473506959189401	0.850179438693979	0.658351221673192\\
0.66	0.487462361096317	0.85866475920876	0.654411055137887\\
0.66	0.501525806262096	0.866904538099407	0.650450223206468\\
0.66	0.515690940160279	0.874896985111485	0.646455401874959\\
0.66	0.529951380947623	0.882640524634437	0.642417151055156\\
0.66	0.544300728313782	0.890133795440129	0.638329210947585\\
0.66	0.558732572247415	0.897375649995373	0.634187835387497\\
0.66	0.57324050169555	0.904365153357279	0.629991194998376\\
0.66	0.587818113093873	0.911101581661729	0.625738868900924\\
0.66	0.602459018746568	0.917584420216504	0.621431431633139\\
0.66	0.617156855035314	0.923813361211857	0.617070132480554\\
0.66	0.631905290438127	0.92978830106243	0.612656657833839\\
0.66	0.6466980333399	0.935509337395467	0.608192963377087\\
0.66	0.661528839617671	0.940976765701173	0.603681161495105\\
0.66	0.676391519984904	0.946191075661951	0.599123449738607\\
0.66	0.691279947080357	0.951152947177946	0.594522067901592\\
0.66	0.706188062288412	0.955863246106974	0.589879273662389\\
0.66	0.721109882279076	0.960323019737426	0.585197329319934\\
0.66	0.736039505257225	0.964533492013186	0.580478494543977\\
0.66	0.75097111691199	0.968496058529893	0.575725022018315\\
0.66	0.765898996058536	0.972212281322139	0.570939154289558\\
0.66	0.780817519965826	0.975683883461272	0.566123121053175\\
0.66	0.795721169365275	0.97891274348356	0.56127913660556\\
0.66	0.810604533136476	0.981900889668376	0.556409397399197\\
0.66	0.825462312667457	0.984650494185936	0.551516079695735\\
0.66	0.840289325888138	0.987163867133889	0.546601337319641\\
0.66	0.855080510976838	0.989443450481768	0.541667299515141\\
0.66	0.86983092974082	0.991491811941915	0.536716068908599\\
0.66	0.884535770672926	0.993311638785082	0.53174971957814\\
0.66	0.899190351687411	0.994905731618374	0.526770295231793\\
0.66	0.913790122539012	0.996276998142666	0.521779807495104\\
0.66	0.928330666930242	0.997428446906011	0.516780234308684\\
0.66	0.942807704312699	0.998363181068902	0.511773518435873\\
0.66	0.95721709138901	0.999084392196567	0.50676156608017\\
0.66	0.971554823322703	0.999595354092743	0.501746245611875\\
0.66	0.985817034663989	0.999899416688637	0.496729386403004\\
0.66	1	1	0.491669087346906\\
0.675	0	0	0.710195569525497\\
0.675	0.000100583311362513	0.0141829653360114	0.71240460418299\\
0.675	0.000404645907256436	0.0284451766772965	0.714630690536905\\
0.675	0.000915607803433	0.0427829086109896	0.716916312064106\\
0.675	0.00163681893109844	0.057192295687301	0.719289019360234\\
0.675	0.00257155309398959	0.0716693330697584	0.721780367061016\\
0.675	0.00372300185733414	0.0862098774609879	0.724425655492835\\
0.675	0.00509426838162599	0.100809648312589	0.727263413850304\\
0.675	0.00668836121491816	0.115464229327074	0.73033458997498\\
0.675	0.00850818805808555	0.13016907025918	0.733681422514709\\
0.675	0.0105565495182326	0.144919489023162	0.737345986077549\\
0.675	0.0128361328661109	0.159710674111862	0.741368418705642\\
0.675	0.0153495058140643	0.174537687332543	0.745784862958202\\
0.675	0.0180991103316243	0.189395466863524	0.750625176092311\\
0.675	0.0210872565164405	0.204278830634725	0.755910489868999\\
0.675	0.0243161165387281	0.219182480034174	0.761650724664453\\
0.675	0.0277877186778607	0.234101003941464	0.767842183867616\\
0.675	0.0315039414701067	0.24902888308801	0.77446537092041\\
0.675	0.0354665079868145	0.263960494742775	0.781483180780606\\
0.675	0.0396769802625738	0.278890117720924	0.788839618268074\\
0.675	0.0441367538930258	0.293811937711588	0.796459186325544\\
0.675	0.0488470528220538	0.308720052919643	0.804247066923455\\
0.675	0.0538089243380495	0.323608480015096	0.812090186161949\\
0.675	0.0590232342988274	0.338471160382329	0.819859213932694\\
0.675	0.064490662604533	0.3533019666601	0.827411499063639\\
0.675	0.0702116989375697	0.368094709561873	0.834594885813505\\
0.675	0.0761866387881432	0.382843144964686	0.84125230028718\\
0.675	0.0824155797834956	0.397540981253432	0.847226939729081\\
0.675	0.0888984183382709	0.412181886906127	0.852367847905042\\
0.675	0.0956348466427212	0.42675949830445	0.856535620027829\\
0.675	0.102624350004627	0.441267427752584	0.859607954619618\\
0.675	0.109866204559871	0.455699271686219	0.861484760266621\\
0.675	0.117359475365564	0.470048619052377	0.862092534251539\\
0.675	0.125103014888515	0.484309059839721	0.861387758068368\\
0.675	0.133095461900593	0.498474193737904	0.859359100890821\\
0.675	0.14133524079124	0.512537638903683	0.856028283765736\\
0.675	0.149820561306021	0.526493040810599	0.851449530870182\\
0.675	0.158549418718625	0.540334081158348	0.845707614739505\\
0.675	0.167519594442213	0.554054486817265	0.83891458434315\\
0.675	0.176728657084455	0.567648038782867	0.831205342370837\\
0.675	0.186173963948925	0.581108581114919	0.822732305417528\\
0.675	0.195852662983903	0.594430029835237	0.813659432944308\\
0.675	0.205761695177907	0.607606381758231	0.80415594410487\\
0.675	0.215897797399558	0.620631723228144	0.794390053406971\\
0.675	0.226257505677663	0.633500238737009	0.784523046062798\\
0.675	0.236837158915675	0.646206219397571	0.774703982861589\\
0.675	0.247632903032949	0.658744071245709	0.765065275211016\\
0.675	0.258640695523528	0.671108323347402	0.755719307772964\\
0.675	0.269856310421543	0.683293635685828	0.746756214005914\\
0.675	0.28127534366066	0.695294806804925	0.738242834604294\\
0.675	0.292893218813452	0.707106781186547	0.730222816007835\\
0.675	0.304705193195075	0.71872465633934	0.722717741087767\\
0.675	0.316706364314172	0.730143689578457	0.715729131143265\\
0.675	0.328891676652598	0.741359304476472	0.709241120561857\\
0.675	0.341255928754291	0.752367096967051	0.703223584571265\\
0.675	0.353793780602429	0.763162841084325	0.697635496597417\\
0.675	0.366499761262991	0.773742494322337	0.692428303594619\\
0.675	0.379368276771856	0.784102202600442	0.68754913288707\\
0.675	0.392393618241769	0.794238304822092	0.682943679235329\\
0.675	0.405569970164763	0.804147337016097	0.678558662193046\\
0.675	0.418891418885081	0.813826036051075	0.674343787412872\\
0.675	0.432351961217133	0.823271342915545	0.670253187715551\\
0.675	0.445945513182735	0.832480405557787	0.666246357336158\\
0.675	0.459665918841652	0.841450581281375	0.662288623474184\\
0.675	0.473506959189401	0.850179438693979	0.65835122167319\\
0.675	0.487462361096317	0.85866475920876	0.654411055137887\\
0.675	0.501525806262096	0.866904538099407	0.650450223206468\\
0.675	0.515690940160279	0.874896985111485	0.646455401874959\\
0.675	0.529951380947623	0.882640524634437	0.642417151055156\\
0.675	0.544300728313782	0.890133795440129	0.638329210947586\\
0.675	0.558732572247415	0.897375649995373	0.634187835387497\\
0.675	0.57324050169555	0.904365153357279	0.629991194998376\\
0.675	0.587818113093873	0.911101581661729	0.625738868900924\\
0.675	0.602459018746568	0.917584420216504	0.62143143163314\\
0.675	0.617156855035314	0.923813361211857	0.617070132480554\\
0.675	0.631905290438127	0.92978830106243	0.612656657833838\\
0.675	0.6466980333399	0.935509337395467	0.608192963377087\\
0.675	0.661528839617671	0.940976765701173	0.603681161495105\\
0.675	0.676391519984904	0.946191075661951	0.599123449738608\\
0.675	0.691279947080357	0.951152947177946	0.594522067901591\\
0.675	0.706188062288412	0.955863246106974	0.589879273662388\\
0.675	0.721109882279076	0.960323019737426	0.585197329319934\\
0.675	0.736039505257225	0.964533492013186	0.580478494543975\\
0.675	0.75097111691199	0.968496058529893	0.575725022018316\\
0.675	0.765898996058536	0.972212281322139	0.57093915428956\\
0.675	0.780817519965826	0.975683883461272	0.566123121053174\\
0.675	0.795721169365275	0.97891274348356	0.561279136605558\\
0.675	0.810604533136476	0.981900889668376	0.556409397399197\\
0.675	0.825462312667457	0.984650494185936	0.551516079695737\\
0.675	0.840289325888138	0.987163867133889	0.546601337319644\\
0.675	0.855080510976839	0.989443450481768	0.541667299515137\\
0.675	0.86983092974082	0.991491811941914	0.536716068908596\\
0.675	0.884535770672926	0.993311638785082	0.531749719578141\\
0.675	0.899190351687411	0.994905731618374	0.526770295231795\\
0.675	0.913790122539012	0.996276998142666	0.521779807495102\\
0.675	0.928330666930242	0.99742844690601	0.516780234308683\\
0.675	0.942807704312699	0.998363181068902	0.511773518435873\\
0.675	0.95721709138901	0.999084392196567	0.506761566080168\\
0.675	0.971554823322703	0.999595354092743	0.501746245611874\\
0.675	0.985817034663989	0.999899416688637	0.496729386403005\\
0.675	1	1	0.491669087346904\\
0.69	0	0	0.710195569525497\\
0.69	0.000100583311362513	0.0141829653360114	0.71240460418299\\
0.69	0.000404645907256436	0.0284451766772965	0.714630690536905\\
0.69	0.000915607803433	0.0427829086109896	0.716916312064106\\
0.69	0.00163681893109844	0.057192295687301	0.719289019360234\\
0.69	0.00257155309398959	0.0716693330697585	0.721780367061016\\
0.69	0.00372300185733414	0.086209877460988	0.724425655492835\\
0.69	0.00509426838162598	0.100809648312589	0.727263413850304\\
0.69	0.00668836121491816	0.115464229327074	0.73033458997498\\
0.69	0.00850818805808555	0.13016907025918	0.733681422514709\\
0.69	0.0105565495182326	0.144919489023162	0.737345986077549\\
0.69	0.0128361328661109	0.159710674111862	0.741368418705642\\
0.69	0.0153495058140643	0.174537687332543	0.745784862958202\\
0.69	0.0180991103316243	0.189395466863524	0.750625176092311\\
0.69	0.0210872565164405	0.204278830634725	0.755910489868998\\
0.69	0.0243161165387281	0.219182480034174	0.761650724664453\\
0.69	0.0277877186778607	0.234101003941464	0.767842183867616\\
0.69	0.0315039414701067	0.24902888308801	0.77446537092041\\
0.69	0.0354665079868145	0.263960494742775	0.781483180780606\\
0.69	0.0396769802625737	0.278890117720924	0.788839618268073\\
0.69	0.0441367538930258	0.293811937711588	0.796459186325544\\
0.69	0.0488470528220538	0.308720052919643	0.804247066923455\\
0.69	0.0538089243380495	0.323608480015096	0.81209018616195\\
0.69	0.0590232342988274	0.338471160382329	0.819859213932694\\
0.69	0.064490662604533	0.3533019666601	0.827411499063638\\
0.69	0.0702116989375697	0.368094709561873	0.834594885813504\\
0.69	0.0761866387881433	0.382843144964686	0.841252300287181\\
0.69	0.0824155797834956	0.397540981253432	0.84722693972908\\
0.69	0.0888984183382709	0.412181886906127	0.85236784790504\\
0.69	0.0956348466427212	0.42675949830445	0.856535620027831\\
0.69	0.102624350004627	0.441267427752584	0.859607954619618\\
0.69	0.109866204559871	0.455699271686219	0.861484760266621\\
0.69	0.117359475365564	0.470048619052377	0.86209253425154\\
0.69	0.125103014888515	0.484309059839721	0.861387758068368\\
0.69	0.133095461900593	0.498474193737904	0.859359100890819\\
0.69	0.14133524079124	0.512537638903683	0.856028283765733\\
0.69	0.149820561306021	0.526493040810599	0.851449530870182\\
0.69	0.158549418718625	0.540334081158348	0.845707614739505\\
0.69	0.167519594442213	0.554054486817265	0.838914584343152\\
0.69	0.176728657084455	0.567648038782867	0.831205342370836\\
0.69	0.186173963948925	0.581108581114919	0.822732305417528\\
0.69	0.195852662983903	0.594430029835237	0.813659432944309\\
0.69	0.205761695177907	0.607606381758231	0.804155944104871\\
0.69	0.215897797399558	0.620631723228144	0.794390053406966\\
0.69	0.226257505677663	0.633500238737009	0.784523046062798\\
0.69	0.236837158915675	0.646206219397571	0.774703982861589\\
0.69	0.247632903032949	0.658744071245709	0.765065275211016\\
0.69	0.258640695523528	0.671108323347402	0.755719307772964\\
0.69	0.269856310421543	0.683293635685828	0.746756214005914\\
0.69	0.28127534366066	0.695294806804925	0.738242834604294\\
0.69	0.292893218813452	0.707106781186547	0.730222816007834\\
0.69	0.304705193195075	0.71872465633934	0.722717741087767\\
0.69	0.316706364314172	0.730143689578457	0.715729131143267\\
0.69	0.328891676652598	0.741359304476472	0.709241120561859\\
0.69	0.341255928754291	0.752367096967051	0.703223584571265\\
0.69	0.353793780602429	0.763162841084325	0.697635496597416\\
0.69	0.366499761262991	0.773742494322337	0.692428303594619\\
0.69	0.379368276771856	0.784102202600442	0.687549132887069\\
0.69	0.392393618241769	0.794238304822092	0.682943679235329\\
0.69	0.405569970164763	0.804147337016097	0.678558662193047\\
0.69	0.418891418885081	0.813826036051075	0.674343787412872\\
0.69	0.432351961217133	0.823271342915545	0.670253187715552\\
0.69	0.445945513182735	0.832480405557787	0.666246357336158\\
0.69	0.459665918841652	0.841450581281375	0.662288623474184\\
0.69	0.473506959189401	0.850179438693979	0.65835122167319\\
0.69	0.487462361096317	0.85866475920876	0.654411055137887\\
0.69	0.501525806262096	0.866904538099407	0.650450223206468\\
0.69	0.515690940160279	0.874896985111485	0.646455401874959\\
0.69	0.529951380947623	0.882640524634437	0.642417151055155\\
0.69	0.544300728313782	0.890133795440129	0.638329210947586\\
0.69	0.558732572247415	0.897375649995373	0.634187835387498\\
0.69	0.57324050169555	0.904365153357279	0.629991194998376\\
0.69	0.587818113093873	0.911101581661729	0.625738868900924\\
0.69	0.602459018746568	0.917584420216504	0.621431431633139\\
0.69	0.617156855035314	0.923813361211857	0.617070132480554\\
0.69	0.631905290438127	0.92978830106243	0.612656657833839\\
0.69	0.6466980333399	0.935509337395467	0.608192963377087\\
0.69	0.661528839617671	0.940976765701173	0.603681161495105\\
0.69	0.676391519984904	0.946191075661951	0.599123449738608\\
0.69	0.691279947080357	0.951152947177946	0.594522067901592\\
0.69	0.706188062288412	0.955863246106974	0.589879273662388\\
0.69	0.721109882279076	0.960323019737426	0.585197329319934\\
0.69	0.736039505257225	0.964533492013186	0.580478494543975\\
0.69	0.75097111691199	0.968496058529893	0.575725022018314\\
0.69	0.765898996058536	0.972212281322139	0.570939154289561\\
0.69	0.780817519965826	0.975683883461272	0.566123121053175\\
0.69	0.795721169365275	0.97891274348356	0.561279136605558\\
0.69	0.810604533136476	0.981900889668376	0.556409397399198\\
0.69	0.825462312667457	0.984650494185936	0.551516079695736\\
0.69	0.840289325888138	0.987163867133889	0.546601337319641\\
0.69	0.855080510976839	0.989443450481768	0.54166729951514\\
0.69	0.86983092974082	0.991491811941914	0.536716068908598\\
0.69	0.884535770672926	0.993311638785082	0.531749719578141\\
0.69	0.899190351687411	0.994905731618374	0.526770295231794\\
0.69	0.913790122539012	0.996276998142666	0.521779807495104\\
0.69	0.928330666930242	0.997428446906011	0.516780234308683\\
0.69	0.942807704312699	0.998363181068902	0.511773518435872\\
0.69	0.95721709138901	0.999084392196567	0.506761566080168\\
0.69	0.971554823322703	0.999595354092743	0.501746245611873\\
0.69	0.985817034663989	0.999899416688637	0.496729386403004\\
0.69	1	1	0.491669087346906\\
0.705	0	0	0.710195569525497\\
0.705	0.000100583311362513	0.0141829653360114	0.71240460418299\\
0.705	0.000404645907256436	0.0284451766772965	0.714630690536905\\
0.705	0.000915607803433	0.0427829086109896	0.716916312064106\\
0.705	0.00163681893109844	0.057192295687301	0.719289019360234\\
0.705	0.00257155309398959	0.0716693330697585	0.721780367061016\\
0.705	0.00372300185733414	0.086209877460988	0.724425655492835\\
0.705	0.00509426838162598	0.100809648312589	0.727263413850304\\
0.705	0.00668836121491816	0.115464229327074	0.73033458997498\\
0.705	0.00850818805808555	0.13016907025918	0.733681422514709\\
0.705	0.0105565495182326	0.144919489023162	0.737345986077549\\
0.705	0.0128361328661109	0.159710674111862	0.741368418705642\\
0.705	0.0153495058140643	0.174537687332543	0.745784862958201\\
0.705	0.0180991103316243	0.189395466863524	0.750625176092311\\
0.705	0.0210872565164405	0.204278830634725	0.755910489868998\\
0.705	0.0243161165387281	0.219182480034174	0.761650724664453\\
0.705	0.0277877186778607	0.234101003941464	0.767842183867616\\
0.705	0.0315039414701067	0.24902888308801	0.77446537092041\\
0.705	0.0354665079868145	0.263960494742775	0.781483180780606\\
0.705	0.0396769802625738	0.278890117720924	0.788839618268073\\
0.705	0.0441367538930258	0.293811937711588	0.796459186325544\\
0.705	0.0488470528220538	0.308720052919643	0.804247066923454\\
0.705	0.0538089243380495	0.323608480015096	0.81209018616195\\
0.705	0.0590232342988274	0.338471160382329	0.819859213932694\\
0.705	0.064490662604533	0.3533019666601	0.827411499063639\\
0.705	0.0702116989375697	0.368094709561873	0.834594885813504\\
0.705	0.0761866387881432	0.382843144964686	0.841252300287179\\
0.705	0.0824155797834956	0.397540981253432	0.847226939729081\\
0.705	0.0888984183382709	0.412181886906127	0.85236784790504\\
0.705	0.0956348466427212	0.42675949830445	0.85653562002783\\
0.705	0.102624350004627	0.441267427752584	0.859607954619617\\
0.705	0.109866204559871	0.455699271686219	0.861484760266622\\
0.705	0.117359475365564	0.470048619052377	0.86209253425154\\
0.705	0.125103014888515	0.484309059839721	0.861387758068368\\
0.705	0.133095461900593	0.498474193737904	0.85935910089082\\
0.705	0.14133524079124	0.512537638903683	0.856028283765736\\
0.705	0.149820561306021	0.526493040810599	0.851449530870183\\
0.705	0.158549418718625	0.540334081158348	0.845707614739509\\
0.705	0.167519594442214	0.554054486817265	0.83891458434315\\
0.705	0.176728657084455	0.567648038782867	0.831205342370836\\
0.705	0.186173963948925	0.581108581114919	0.822732305417527\\
0.705	0.195852662983903	0.594430029835237	0.813659432944308\\
0.705	0.205761695177907	0.607606381758231	0.804155944104867\\
0.705	0.215897797399558	0.620631723228143	0.794390053406966\\
0.705	0.226257505677663	0.633500238737009	0.7845230460628\\
0.705	0.236837158915675	0.646206219397571	0.774703982861589\\
0.705	0.247632903032949	0.658744071245709	0.765065275211016\\
0.705	0.258640695523528	0.671108323347402	0.755719307772964\\
0.705	0.269856310421543	0.683293635685828	0.746756214005914\\
0.705	0.28127534366066	0.695294806804925	0.738242834604294\\
0.705	0.292893218813452	0.707106781186547	0.730222816007836\\
0.705	0.304705193195075	0.71872465633934	0.722717741087767\\
0.705	0.316706364314172	0.730143689578457	0.715729131143266\\
0.705	0.328891676652598	0.741359304476472	0.709241120561857\\
0.705	0.341255928754291	0.752367096967051	0.703223584571265\\
0.705	0.353793780602429	0.763162841084325	0.697635496597417\\
0.705	0.366499761262991	0.773742494322337	0.692428303594618\\
0.705	0.379368276771857	0.784102202600442	0.687549132887069\\
0.705	0.392393618241769	0.794238304822092	0.682943679235329\\
0.705	0.405569970164763	0.804147337016097	0.678558662193046\\
0.705	0.418891418885081	0.813826036051075	0.674343787412872\\
0.705	0.432351961217133	0.823271342915545	0.670253187715552\\
0.705	0.445945513182735	0.832480405557787	0.666246357336158\\
0.705	0.459665918841652	0.841450581281375	0.662288623474185\\
0.705	0.473506959189401	0.850179438693979	0.658351221673191\\
0.705	0.487462361096317	0.85866475920876	0.654411055137886\\
0.705	0.501525806262096	0.866904538099407	0.650450223206467\\
0.705	0.515690940160279	0.874896985111485	0.646455401874959\\
0.705	0.529951380947623	0.882640524634437	0.642417151055155\\
0.705	0.544300728313782	0.890133795440129	0.638329210947585\\
0.705	0.558732572247415	0.897375649995373	0.634187835387498\\
0.705	0.57324050169555	0.904365153357279	0.629991194998375\\
0.705	0.587818113093873	0.911101581661729	0.625738868900924\\
0.705	0.602459018746568	0.917584420216504	0.62143143163314\\
0.705	0.617156855035314	0.923813361211857	0.617070132480554\\
0.705	0.631905290438127	0.92978830106243	0.612656657833838\\
0.705	0.6466980333399	0.935509337395467	0.608192963377087\\
0.705	0.661528839617671	0.940976765701173	0.603681161495105\\
0.705	0.676391519984904	0.946191075661951	0.599123449738607\\
0.705	0.691279947080357	0.951152947177946	0.594522067901591\\
0.705	0.706188062288412	0.955863246106974	0.589879273662389\\
0.705	0.721109882279076	0.960323019737426	0.585197329319936\\
0.705	0.736039505257225	0.964533492013186	0.580478494543978\\
0.705	0.75097111691199	0.968496058529893	0.575725022018312\\
0.705	0.765898996058536	0.972212281322139	0.570939154289558\\
0.705	0.780817519965826	0.975683883461272	0.566123121053175\\
0.705	0.795721169365275	0.97891274348356	0.561279136605559\\
0.705	0.810604533136476	0.981900889668376	0.556409397399198\\
0.705	0.825462312667457	0.984650494185936	0.551516079695737\\
0.705	0.840289325888138	0.987163867133889	0.546601337319641\\
0.705	0.855080510976839	0.989443450481768	0.54166729951514\\
0.705	0.86983092974082	0.991491811941914	0.536716068908598\\
0.705	0.884535770672926	0.993311638785082	0.531749719578141\\
0.705	0.899190351687411	0.994905731618374	0.526770295231796\\
0.705	0.913790122539012	0.996276998142666	0.521779807495104\\
0.705	0.928330666930242	0.997428446906011	0.516780234308682\\
0.705	0.942807704312699	0.998363181068902	0.511773518435872\\
0.705	0.95721709138901	0.999084392196567	0.506761566080171\\
0.705	0.971554823322704	0.999595354092744	0.501746245611874\\
0.705	0.985817034663989	0.999899416688637	0.496729386403002\\
0.705	1	1	0.491669087346908\\
0.72	0	0	0.710195569525497\\
0.72	0.000100583311362513	0.0141829653360114	0.71240460418299\\
0.72	0.000404645907256436	0.0284451766772965	0.714630690536905\\
0.72	0.000915607803433	0.0427829086109896	0.716916312064106\\
0.72	0.00163681893109844	0.057192295687301	0.719289019360234\\
0.72	0.00257155309398959	0.0716693330697584	0.721780367061016\\
0.72	0.00372300185733414	0.086209877460988	0.724425655492835\\
0.72	0.00509426838162598	0.100809648312589	0.727263413850304\\
0.72	0.00668836121491816	0.115464229327074	0.73033458997498\\
0.72	0.00850818805808555	0.13016907025918	0.733681422514709\\
0.72	0.0105565495182326	0.144919489023162	0.737345986077549\\
0.72	0.0128361328661109	0.159710674111862	0.741368418705642\\
0.72	0.0153495058140643	0.174537687332543	0.745784862958202\\
0.72	0.0180991103316243	0.189395466863524	0.750625176092311\\
0.72	0.0210872565164405	0.204278830634725	0.755910489868998\\
0.72	0.0243161165387281	0.219182480034174	0.761650724664453\\
0.72	0.0277877186778607	0.234101003941464	0.767842183867616\\
0.72	0.0315039414701067	0.24902888308801	0.77446537092041\\
0.72	0.0354665079868145	0.263960494742775	0.781483180780606\\
0.72	0.0396769802625738	0.278890117720924	0.788839618268073\\
0.72	0.0441367538930258	0.293811937711588	0.796459186325544\\
0.72	0.0488470528220538	0.308720052919643	0.804247066923454\\
0.72	0.0538089243380495	0.323608480015096	0.81209018616195\\
0.72	0.0590232342988274	0.338471160382329	0.819859213932694\\
0.72	0.064490662604533	0.3533019666601	0.827411499063639\\
0.72	0.0702116989375697	0.368094709561873	0.834594885813504\\
0.72	0.0761866387881432	0.382843144964686	0.841252300287179\\
0.72	0.0824155797834956	0.397540981253432	0.847226939729081\\
0.72	0.0888984183382709	0.412181886906127	0.852367847905041\\
0.72	0.0956348466427212	0.42675949830445	0.856535620027829\\
0.72	0.102624350004627	0.441267427752584	0.859607954619617\\
0.72	0.109866204559871	0.455699271686219	0.861484760266622\\
0.72	0.117359475365564	0.470048619052377	0.86209253425154\\
0.72	0.125103014888515	0.484309059839721	0.861387758068368\\
0.72	0.133095461900593	0.498474193737904	0.859359100890821\\
0.72	0.14133524079124	0.512537638903683	0.856028283765736\\
0.72	0.149820561306021	0.526493040810599	0.851449530870183\\
0.72	0.158549418718625	0.540334081158348	0.845707614739505\\
0.72	0.167519594442214	0.554054486817265	0.83891458434315\\
0.72	0.176728657084455	0.567648038782867	0.831205342370836\\
0.72	0.186173963948925	0.581108581114919	0.822732305417527\\
0.72	0.195852662983903	0.594430029835237	0.813659432944309\\
0.72	0.205761695177907	0.607606381758231	0.804155944104868\\
0.72	0.215897797399558	0.620631723228143	0.794390053406969\\
0.72	0.226257505677663	0.633500238737009	0.7845230460628\\
0.72	0.236837158915675	0.646206219397571	0.774703982861589\\
0.72	0.247632903032949	0.658744071245709	0.765065275211016\\
0.72	0.258640695523528	0.671108323347402	0.755719307772964\\
0.72	0.269856310421543	0.683293635685828	0.746756214005914\\
0.72	0.28127534366066	0.695294806804925	0.738242834604294\\
0.72	0.292893218813452	0.707106781186547	0.730222816007836\\
0.72	0.304705193195075	0.71872465633934	0.722717741087767\\
0.72	0.316706364314172	0.730143689578457	0.715729131143265\\
0.72	0.328891676652598	0.741359304476472	0.709241120561857\\
0.72	0.341255928754291	0.752367096967051	0.703223584571265\\
0.72	0.353793780602429	0.763162841084325	0.697635496597417\\
0.72	0.366499761262991	0.773742494322337	0.692428303594618\\
0.72	0.379368276771857	0.784102202600442	0.687549132887069\\
0.72	0.392393618241769	0.794238304822092	0.682943679235329\\
0.72	0.405569970164763	0.804147337016097	0.678558662193046\\
0.72	0.418891418885081	0.813826036051075	0.674343787412872\\
0.72	0.432351961217133	0.823271342915545	0.670253187715552\\
0.72	0.445945513182735	0.832480405557787	0.666246357336158\\
0.72	0.459665918841652	0.841450581281375	0.662288623474185\\
0.72	0.473506959189401	0.850179438693979	0.658351221673191\\
0.72	0.487462361096317	0.85866475920876	0.654411055137886\\
0.72	0.501525806262096	0.866904538099407	0.650450223206467\\
0.72	0.515690940160279	0.874896985111485	0.646455401874959\\
0.72	0.529951380947623	0.882640524634437	0.642417151055155\\
0.72	0.544300728313782	0.890133795440129	0.638329210947585\\
0.72	0.558732572247415	0.897375649995373	0.634187835387498\\
0.72	0.57324050169555	0.904365153357279	0.629991194998376\\
0.72	0.587818113093873	0.911101581661729	0.625738868900924\\
0.72	0.602459018746568	0.917584420216504	0.62143143163314\\
0.72	0.617156855035314	0.923813361211857	0.617070132480554\\
0.72	0.631905290438127	0.92978830106243	0.612656657833838\\
0.72	0.6466980333399	0.935509337395467	0.608192963377087\\
0.72	0.661528839617671	0.940976765701173	0.603681161495105\\
0.72	0.676391519984904	0.946191075661951	0.599123449738607\\
0.72	0.691279947080357	0.951152947177946	0.59452206790159\\
0.72	0.706188062288412	0.955863246106974	0.589879273662389\\
0.72	0.721109882279076	0.960323019737426	0.585197329319936\\
0.72	0.736039505257225	0.964533492013186	0.580478494543978\\
0.72	0.75097111691199	0.968496058529893	0.575725022018313\\
0.72	0.765898996058536	0.972212281322139	0.570939154289558\\
0.72	0.780817519965826	0.975683883461272	0.566123121053175\\
0.72	0.795721169365275	0.97891274348356	0.561279136605558\\
0.72	0.810604533136476	0.981900889668376	0.556409397399198\\
0.72	0.825462312667457	0.984650494185936	0.551516079695736\\
0.72	0.840289325888138	0.987163867133889	0.546601337319641\\
0.72	0.855080510976839	0.989443450481768	0.541667299515139\\
0.72	0.86983092974082	0.991491811941914	0.536716068908598\\
0.72	0.884535770672926	0.993311638785082	0.531749719578141\\
0.72	0.899190351687411	0.994905731618374	0.526770295231796\\
0.72	0.913790122539012	0.996276998142666	0.521779807495104\\
0.72	0.928330666930242	0.997428446906011	0.516780234308684\\
0.72	0.942807704312699	0.998363181068902	0.511773518435871\\
0.72	0.95721709138901	0.999084392196567	0.506761566080168\\
0.72	0.971554823322703	0.999595354092743	0.501746245611875\\
0.72	0.985817034663989	0.999899416688637	0.496729386403004\\
0.72	1	1	0.491669087346906\\
0.735	0	0	0.710195569525497\\
0.735	0.000100583311362513	0.0141829653360114	0.71240460418299\\
0.735	0.000404645907256436	0.0284451766772965	0.714630690536905\\
0.735	0.000915607803433	0.0427829086109896	0.716916312064106\\
0.735	0.00163681893109844	0.057192295687301	0.719289019360234\\
0.735	0.00257155309398959	0.0716693330697585	0.721780367061016\\
0.735	0.00372300185733414	0.086209877460988	0.724425655492835\\
0.735	0.00509426838162598	0.100809648312589	0.727263413850304\\
0.735	0.00668836121491816	0.115464229327074	0.73033458997498\\
0.735	0.00850818805808555	0.13016907025918	0.733681422514709\\
0.735	0.0105565495182326	0.144919489023162	0.737345986077549\\
0.735	0.0128361328661109	0.159710674111862	0.741368418705642\\
0.735	0.0153495058140643	0.174537687332543	0.745784862958202\\
0.735	0.0180991103316243	0.189395466863524	0.750625176092311\\
0.735	0.0210872565164405	0.204278830634725	0.755910489868998\\
0.735	0.0243161165387281	0.219182480034174	0.761650724664453\\
0.735	0.0277877186778607	0.234101003941464	0.767842183867616\\
0.735	0.0315039414701067	0.24902888308801	0.77446537092041\\
0.735	0.0354665079868145	0.263960494742775	0.781483180780606\\
0.735	0.0396769802625738	0.278890117720924	0.788839618268073\\
0.735	0.0441367538930258	0.293811937711588	0.796459186325544\\
0.735	0.0488470528220538	0.308720052919643	0.804247066923454\\
0.735	0.0538089243380495	0.323608480015096	0.81209018616195\\
0.735	0.0590232342988274	0.338471160382329	0.819859213932694\\
0.735	0.064490662604533	0.3533019666601	0.82741149906364\\
0.735	0.0702116989375697	0.368094709561873	0.834594885813504\\
0.735	0.0761866387881432	0.382843144964686	0.841252300287179\\
0.735	0.0824155797834956	0.397540981253432	0.847226939729081\\
0.735	0.0888984183382709	0.412181886906127	0.852367847905041\\
0.735	0.0956348466427212	0.42675949830445	0.856535620027829\\
0.735	0.102624350004627	0.441267427752584	0.859607954619617\\
0.735	0.109866204559871	0.455699271686219	0.861484760266622\\
0.735	0.117359475365564	0.470048619052377	0.86209253425154\\
0.735	0.125103014888515	0.484309059839721	0.861387758068368\\
0.735	0.133095461900593	0.498474193737904	0.85935910089082\\
0.735	0.14133524079124	0.512537638903683	0.856028283765736\\
0.735	0.149820561306021	0.526493040810599	0.851449530870183\\
0.735	0.158549418718625	0.540334081158348	0.845707614739505\\
0.735	0.167519594442214	0.554054486817265	0.83891458434315\\
0.735	0.176728657084455	0.567648038782867	0.831205342370836\\
0.735	0.186173963948925	0.581108581114919	0.822732305417529\\
0.735	0.195852662983903	0.594430029835237	0.813659432944309\\
0.735	0.205761695177907	0.607606381758231	0.804155944104868\\
0.735	0.215897797399558	0.620631723228143	0.794390053406969\\
0.735	0.226257505677663	0.633500238737009	0.7845230460628\\
0.735	0.236837158915675	0.646206219397571	0.774703982861589\\
0.735	0.247632903032949	0.658744071245709	0.765065275211016\\
0.735	0.258640695523528	0.671108323347402	0.755719307772964\\
0.735	0.269856310421543	0.683293635685828	0.746756214005914\\
0.735	0.28127534366066	0.695294806804925	0.738242834604294\\
0.735	0.292893218813452	0.707106781186547	0.730222816007836\\
0.735	0.304705193195075	0.71872465633934	0.722717741087767\\
0.735	0.316706364314172	0.730143689578457	0.715729131143265\\
0.735	0.328891676652598	0.741359304476472	0.709241120561857\\
0.735	0.341255928754291	0.752367096967051	0.703223584571265\\
0.735	0.353793780602429	0.763162841084325	0.697635496597417\\
0.735	0.366499761262991	0.773742494322337	0.692428303594618\\
0.735	0.379368276771857	0.784102202600442	0.687549132887069\\
0.735	0.392393618241769	0.794238304822092	0.682943679235329\\
0.735	0.405569970164763	0.804147337016097	0.678558662193046\\
0.735	0.418891418885081	0.813826036051075	0.674343787412872\\
0.735	0.432351961217133	0.823271342915545	0.670253187715552\\
0.735	0.445945513182735	0.832480405557787	0.666246357336158\\
0.735	0.459665918841652	0.841450581281375	0.662288623474185\\
0.735	0.473506959189401	0.850179438693979	0.658351221673191\\
0.735	0.487462361096317	0.85866475920876	0.654411055137886\\
0.735	0.501525806262096	0.866904538099407	0.650450223206467\\
0.735	0.515690940160279	0.874896985111485	0.646455401874959\\
0.735	0.529951380947623	0.882640524634437	0.642417151055155\\
0.735	0.544300728313782	0.890133795440129	0.638329210947585\\
0.735	0.558732572247415	0.897375649995373	0.634187835387498\\
0.735	0.57324050169555	0.904365153357279	0.629991194998376\\
0.735	0.587818113093873	0.911101581661729	0.625738868900924\\
0.735	0.602459018746568	0.917584420216504	0.62143143163314\\
0.735	0.617156855035314	0.923813361211857	0.617070132480554\\
0.735	0.631905290438127	0.92978830106243	0.612656657833838\\
0.735	0.6466980333399	0.935509337395467	0.608192963377087\\
0.735	0.661528839617671	0.940976765701173	0.603681161495105\\
0.735	0.676391519984904	0.946191075661951	0.599123449738607\\
0.735	0.691279947080357	0.951152947177946	0.59452206790159\\
0.735	0.706188062288412	0.955863246106974	0.589879273662389\\
0.735	0.721109882279076	0.960323019737426	0.585197329319936\\
0.735	0.736039505257225	0.964533492013186	0.580478494543978\\
0.735	0.75097111691199	0.968496058529893	0.575725022018313\\
0.735	0.765898996058536	0.972212281322139	0.570939154289558\\
0.735	0.780817519965826	0.975683883461272	0.566123121053175\\
0.735	0.795721169365275	0.97891274348356	0.561279136605558\\
0.735	0.810604533136476	0.981900889668376	0.556409397399198\\
0.735	0.825462312667457	0.984650494185936	0.551516079695736\\
0.735	0.840289325888138	0.987163867133889	0.546601337319641\\
0.735	0.855080510976839	0.989443450481768	0.541667299515139\\
0.735	0.86983092974082	0.991491811941914	0.536716068908598\\
0.735	0.884535770672926	0.993311638785082	0.531749719578141\\
0.735	0.899190351687411	0.994905731618374	0.526770295231796\\
0.735	0.913790122539012	0.996276998142666	0.521779807495104\\
0.735	0.928330666930242	0.997428446906011	0.516780234308684\\
0.735	0.942807704312699	0.998363181068902	0.511773518435872\\
0.735	0.95721709138901	0.999084392196567	0.506761566080168\\
0.735	0.971554823322703	0.999595354092743	0.501746245611874\\
0.735	0.985817034663989	0.999899416688637	0.496729386403004\\
0.735	1	1	0.491669087346906\\
0.75	0	0	0.710195569525497\\
0.75	0.000100583311362513	0.0141829653360114	0.71240460418299\\
0.75	0.000404645907256436	0.0284451766772965	0.714630690536905\\
0.75	0.000915607803433	0.0427829086109896	0.716916312064106\\
0.75	0.00163681893109844	0.057192295687301	0.719289019360234\\
0.75	0.00257155309398959	0.0716693330697585	0.721780367061016\\
0.75	0.00372300185733414	0.086209877460988	0.724425655492835\\
0.75	0.00509426838162598	0.100809648312589	0.727263413850304\\
0.75	0.00668836121491816	0.115464229327074	0.73033458997498\\
0.75	0.00850818805808555	0.13016907025918	0.733681422514709\\
0.75	0.0105565495182326	0.144919489023162	0.737345986077549\\
0.75	0.0128361328661109	0.159710674111862	0.741368418705642\\
0.75	0.0153495058140643	0.174537687332543	0.745784862958202\\
0.75	0.0180991103316243	0.189395466863524	0.750625176092311\\
0.75	0.0210872565164405	0.204278830634725	0.755910489868999\\
0.75	0.0243161165387281	0.219182480034174	0.761650724664453\\
0.75	0.0277877186778607	0.234101003941464	0.767842183867616\\
0.75	0.0315039414701067	0.24902888308801	0.77446537092041\\
0.75	0.0354665079868145	0.263960494742775	0.781483180780606\\
0.75	0.0396769802625738	0.278890117720924	0.788839618268073\\
0.75	0.0441367538930258	0.293811937711588	0.796459186325544\\
0.75	0.0488470528220538	0.308720052919643	0.804247066923454\\
0.75	0.0538089243380495	0.323608480015096	0.81209018616195\\
0.75	0.0590232342988274	0.338471160382329	0.819859213932694\\
0.75	0.064490662604533	0.3533019666601	0.827411499063639\\
0.75	0.0702116989375697	0.368094709561873	0.834594885813504\\
0.75	0.0761866387881432	0.382843144964686	0.841252300287179\\
0.75	0.0824155797834956	0.397540981253432	0.847226939729081\\
0.75	0.0888984183382709	0.412181886906127	0.852367847905041\\
0.75	0.0956348466427212	0.42675949830445	0.856535620027829\\
0.75	0.102624350004627	0.441267427752584	0.859607954619617\\
0.75	0.109866204559871	0.455699271686219	0.861484760266622\\
0.75	0.117359475365564	0.470048619052377	0.86209253425154\\
0.75	0.125103014888515	0.484309059839721	0.861387758068368\\
0.75	0.133095461900593	0.498474193737904	0.85935910089082\\
0.75	0.14133524079124	0.512537638903683	0.856028283765736\\
0.75	0.149820561306021	0.526493040810599	0.851449530870183\\
0.75	0.158549418718625	0.540334081158348	0.845707614739505\\
0.75	0.167519594442214	0.554054486817265	0.83891458434315\\
0.75	0.176728657084455	0.567648038782867	0.831205342370836\\
0.75	0.186173963948925	0.581108581114919	0.822732305417528\\
0.75	0.195852662983903	0.594430029835237	0.813659432944309\\
0.75	0.205761695177907	0.607606381758231	0.804155944104868\\
0.75	0.215897797399558	0.620631723228143	0.794390053406969\\
0.75	0.226257505677663	0.633500238737009	0.7845230460628\\
0.75	0.236837158915675	0.646206219397571	0.774703982861589\\
0.75	0.247632903032949	0.658744071245709	0.765065275211016\\
0.75	0.258640695523528	0.671108323347402	0.755719307772964\\
0.75	0.269856310421543	0.683293635685828	0.746756214005914\\
0.75	0.28127534366066	0.695294806804925	0.738242834604294\\
0.75	0.292893218813452	0.707106781186547	0.730222816007836\\
0.75	0.304705193195075	0.71872465633934	0.722717741087767\\
0.75	0.316706364314172	0.730143689578457	0.715729131143265\\
0.75	0.328891676652598	0.741359304476472	0.709241120561857\\
0.75	0.341255928754291	0.752367096967051	0.703223584571265\\
0.75	0.353793780602429	0.763162841084325	0.697635496597417\\
0.75	0.366499761262991	0.773742494322337	0.692428303594618\\
0.75	0.379368276771857	0.784102202600442	0.687549132887069\\
0.75	0.392393618241769	0.794238304822092	0.682943679235329\\
0.75	0.405569970164763	0.804147337016097	0.678558662193046\\
0.75	0.418891418885081	0.813826036051075	0.674343787412872\\
0.75	0.432351961217133	0.823271342915545	0.670253187715552\\
0.75	0.445945513182735	0.832480405557787	0.666246357336158\\
0.75	0.459665918841652	0.841450581281375	0.662288623474185\\
0.75	0.473506959189401	0.850179438693979	0.658351221673191\\
0.75	0.487462361096317	0.85866475920876	0.654411055137886\\
0.75	0.501525806262096	0.866904538099407	0.650450223206467\\
0.75	0.515690940160279	0.874896985111485	0.646455401874959\\
0.75	0.529951380947623	0.882640524634437	0.642417151055155\\
0.75	0.544300728313782	0.890133795440129	0.638329210947585\\
0.75	0.558732572247415	0.897375649995373	0.634187835387498\\
0.75	0.57324050169555	0.904365153357279	0.629991194998376\\
0.75	0.587818113093873	0.911101581661729	0.625738868900924\\
0.75	0.602459018746568	0.917584420216504	0.62143143163314\\
0.75	0.617156855035314	0.923813361211857	0.617070132480554\\
0.75	0.631905290438127	0.92978830106243	0.612656657833838\\
0.75	0.6466980333399	0.935509337395467	0.608192963377087\\
0.75	0.661528839617671	0.940976765701173	0.603681161495105\\
0.75	0.676391519984904	0.946191075661951	0.599123449738607\\
0.75	0.691279947080357	0.951152947177946	0.59452206790159\\
0.75	0.706188062288412	0.955863246106974	0.589879273662389\\
0.75	0.721109882279076	0.960323019737426	0.585197329319936\\
0.75	0.736039505257225	0.964533492013186	0.580478494543978\\
0.75	0.75097111691199	0.968496058529893	0.575725022018313\\
0.75	0.765898996058536	0.972212281322139	0.570939154289558\\
0.75	0.780817519965826	0.975683883461272	0.566123121053175\\
0.75	0.795721169365275	0.97891274348356	0.561279136605558\\
0.75	0.810604533136476	0.981900889668376	0.556409397399198\\
0.75	0.825462312667457	0.984650494185936	0.551516079695736\\
0.75	0.840289325888138	0.987163867133889	0.546601337319641\\
0.75	0.855080510976839	0.989443450481768	0.541667299515139\\
0.75	0.86983092974082	0.991491811941914	0.536716068908598\\
0.75	0.884535770672926	0.993311638785082	0.531749719578141\\
0.75	0.899190351687411	0.994905731618374	0.526770295231796\\
0.75	0.913790122539012	0.996276998142666	0.521779807495104\\
0.75	0.928330666930242	0.997428446906011	0.516780234308684\\
0.75	0.942807704312699	0.998363181068902	0.511773518435872\\
0.75	0.95721709138901	0.999084392196567	0.506761566080168\\
0.75	0.971554823322703	0.999595354092743	0.501746245611874\\
0.75	0.985817034663989	0.999899416688637	0.496729386403004\\
0.75	1	1	0.491669087346906\\
0.765	0	0	0.710195569525497\\
0.765	0.000100583311362513	0.0141829653360114	0.71240460418299\\
0.765	0.000404645907256436	0.0284451766772965	0.714630690536905\\
0.765	0.000915607803433	0.0427829086109896	0.716916312064106\\
0.765	0.00163681893109844	0.057192295687301	0.719289019360234\\
0.765	0.00257155309398959	0.0716693330697585	0.721780367061016\\
0.765	0.00372300185733414	0.086209877460988	0.724425655492835\\
0.765	0.00509426838162598	0.100809648312589	0.727263413850304\\
0.765	0.00668836121491816	0.115464229327074	0.73033458997498\\
0.765	0.00850818805808555	0.13016907025918	0.733681422514709\\
0.765	0.0105565495182326	0.144919489023162	0.737345986077549\\
0.765	0.0128361328661109	0.159710674111862	0.741368418705642\\
0.765	0.0153495058140643	0.174537687332543	0.745784862958202\\
0.765	0.0180991103316243	0.189395466863524	0.750625176092311\\
0.765	0.0210872565164405	0.204278830634725	0.755910489868999\\
0.765	0.0243161165387281	0.219182480034174	0.761650724664453\\
0.765	0.0277877186778607	0.234101003941464	0.767842183867616\\
0.765	0.0315039414701067	0.24902888308801	0.77446537092041\\
0.765	0.0354665079868145	0.263960494742775	0.781483180780606\\
0.765	0.0396769802625738	0.278890117720924	0.788839618268073\\
0.765	0.0441367538930258	0.293811937711588	0.796459186325544\\
0.765	0.0488470528220538	0.308720052919643	0.804247066923454\\
0.765	0.0538089243380495	0.323608480015096	0.81209018616195\\
0.765	0.0590232342988274	0.338471160382329	0.819859213932694\\
0.765	0.064490662604533	0.3533019666601	0.827411499063639\\
0.765	0.0702116989375697	0.368094709561873	0.834594885813504\\
0.765	0.0761866387881432	0.382843144964686	0.841252300287179\\
0.765	0.0824155797834956	0.397540981253432	0.847226939729081\\
0.765	0.0888984183382709	0.412181886906127	0.852367847905041\\
0.765	0.0956348466427212	0.42675949830445	0.856535620027829\\
0.765	0.102624350004627	0.441267427752584	0.859607954619617\\
0.765	0.109866204559871	0.455699271686219	0.861484760266622\\
0.765	0.117359475365564	0.470048619052377	0.86209253425154\\
0.765	0.125103014888515	0.484309059839721	0.861387758068368\\
0.765	0.133095461900593	0.498474193737904	0.85935910089082\\
0.765	0.14133524079124	0.512537638903683	0.856028283765736\\
0.765	0.149820561306021	0.526493040810599	0.851449530870183\\
0.765	0.158549418718625	0.540334081158348	0.845707614739505\\
0.765	0.167519594442214	0.554054486817265	0.83891458434315\\
0.765	0.176728657084455	0.567648038782867	0.831205342370836\\
0.765	0.186173963948925	0.581108581114919	0.822732305417528\\
0.765	0.195852662983903	0.594430029835237	0.813659432944309\\
0.765	0.205761695177907	0.607606381758231	0.804155944104868\\
0.765	0.215897797399558	0.620631723228143	0.794390053406969\\
0.765	0.226257505677663	0.633500238737009	0.7845230460628\\
0.765	0.236837158915675	0.646206219397571	0.774703982861589\\
0.765	0.247632903032949	0.658744071245709	0.765065275211016\\
0.765	0.258640695523528	0.671108323347402	0.755719307772964\\
0.765	0.269856310421543	0.683293635685828	0.746756214005914\\
0.765	0.28127534366066	0.695294806804925	0.738242834604294\\
0.765	0.292893218813452	0.707106781186547	0.730222816007836\\
0.765	0.304705193195075	0.71872465633934	0.722717741087767\\
0.765	0.316706364314172	0.730143689578457	0.715729131143265\\
0.765	0.328891676652598	0.741359304476472	0.709241120561857\\
0.765	0.341255928754291	0.752367096967051	0.703223584571265\\
0.765	0.353793780602429	0.763162841084325	0.697635496597417\\
0.765	0.366499761262991	0.773742494322337	0.692428303594618\\
0.765	0.379368276771857	0.784102202600442	0.687549132887069\\
0.765	0.392393618241769	0.794238304822092	0.682943679235329\\
0.765	0.405569970164763	0.804147337016097	0.678558662193046\\
0.765	0.418891418885081	0.813826036051075	0.674343787412872\\
0.765	0.432351961217133	0.823271342915545	0.670253187715552\\
0.765	0.445945513182735	0.832480405557787	0.666246357336158\\
0.765	0.459665918841652	0.841450581281375	0.662288623474185\\
0.765	0.473506959189401	0.850179438693979	0.658351221673191\\
0.765	0.487462361096317	0.85866475920876	0.654411055137886\\
0.765	0.501525806262096	0.866904538099407	0.650450223206467\\
0.765	0.515690940160279	0.874896985111485	0.646455401874959\\
0.765	0.529951380947623	0.882640524634437	0.642417151055155\\
0.765	0.544300728313782	0.890133795440129	0.638329210947585\\
0.765	0.558732572247415	0.897375649995373	0.634187835387498\\
0.765	0.57324050169555	0.904365153357279	0.629991194998376\\
0.765	0.587818113093873	0.911101581661729	0.625738868900924\\
0.765	0.602459018746568	0.917584420216504	0.62143143163314\\
0.765	0.617156855035314	0.923813361211857	0.617070132480554\\
0.765	0.631905290438127	0.92978830106243	0.612656657833838\\
0.765	0.6466980333399	0.935509337395467	0.608192963377087\\
0.765	0.661528839617671	0.940976765701173	0.603681161495105\\
0.765	0.676391519984904	0.946191075661951	0.599123449738607\\
0.765	0.691279947080357	0.951152947177946	0.59452206790159\\
0.765	0.706188062288412	0.955863246106974	0.589879273662389\\
0.765	0.721109882279076	0.960323019737426	0.585197329319936\\
0.765	0.736039505257225	0.964533492013186	0.580478494543978\\
0.765	0.75097111691199	0.968496058529893	0.575725022018313\\
0.765	0.765898996058536	0.972212281322139	0.570939154289558\\
0.765	0.780817519965826	0.975683883461272	0.566123121053175\\
0.765	0.795721169365275	0.97891274348356	0.561279136605558\\
0.765	0.810604533136476	0.981900889668376	0.556409397399198\\
0.765	0.825462312667457	0.984650494185936	0.551516079695736\\
0.765	0.840289325888138	0.987163867133889	0.546601337319641\\
0.765	0.855080510976839	0.989443450481768	0.541667299515139\\
0.765	0.86983092974082	0.991491811941914	0.536716068908598\\
0.765	0.884535770672926	0.993311638785082	0.531749719578141\\
0.765	0.899190351687411	0.994905731618374	0.526770295231796\\
0.765	0.913790122539012	0.996276998142666	0.521779807495104\\
0.765	0.928330666930242	0.997428446906011	0.516780234308684\\
0.765	0.942807704312699	0.998363181068902	0.511773518435872\\
0.765	0.95721709138901	0.999084392196567	0.506761566080168\\
0.765	0.971554823322703	0.999595354092743	0.501746245611874\\
0.765	0.985817034663989	0.999899416688637	0.496729386403004\\
0.765	1	1	0.491669087346906\\
0.78	0	0	0.710195569525497\\
0.78	0.000100583311362513	0.0141829653360114	0.71240460418299\\
0.78	0.000404645907256436	0.0284451766772965	0.714630690536905\\
0.78	0.000915607803433	0.0427829086109896	0.716916312064106\\
0.78	0.00163681893109844	0.057192295687301	0.719289019360234\\
0.78	0.00257155309398959	0.0716693330697584	0.721780367061016\\
0.78	0.00372300185733414	0.086209877460988	0.724425655492835\\
0.78	0.00509426838162598	0.100809648312589	0.727263413850304\\
0.78	0.00668836121491816	0.115464229327074	0.73033458997498\\
0.78	0.00850818805808555	0.13016907025918	0.733681422514709\\
0.78	0.0105565495182326	0.144919489023162	0.737345986077549\\
0.78	0.0128361328661109	0.159710674111862	0.741368418705642\\
0.78	0.0153495058140643	0.174537687332543	0.745784862958201\\
0.78	0.0180991103316243	0.189395466863524	0.750625176092311\\
0.78	0.0210872565164405	0.204278830634725	0.755910489868998\\
0.78	0.0243161165387281	0.219182480034174	0.761650724664453\\
0.78	0.0277877186778607	0.234101003941464	0.767842183867616\\
0.78	0.0315039414701067	0.24902888308801	0.77446537092041\\
0.78	0.0354665079868145	0.263960494742775	0.781483180780606\\
0.78	0.0396769802625738	0.278890117720924	0.788839618268073\\
0.78	0.0441367538930258	0.293811937711588	0.796459186325544\\
0.78	0.0488470528220538	0.308720052919643	0.804247066923454\\
0.78	0.0538089243380495	0.323608480015096	0.81209018616195\\
0.78	0.0590232342988274	0.338471160382329	0.819859213932694\\
0.78	0.064490662604533	0.3533019666601	0.827411499063639\\
0.78	0.0702116989375697	0.368094709561873	0.834594885813504\\
0.78	0.0761866387881432	0.382843144964686	0.841252300287179\\
0.78	0.0824155797834956	0.397540981253432	0.847226939729081\\
0.78	0.0888984183382709	0.412181886906127	0.852367847905041\\
0.78	0.0956348466427212	0.42675949830445	0.856535620027829\\
0.78	0.102624350004627	0.441267427752584	0.859607954619618\\
0.78	0.109866204559871	0.455699271686219	0.861484760266622\\
0.78	0.117359475365564	0.470048619052377	0.86209253425154\\
0.78	0.125103014888515	0.484309059839721	0.861387758068368\\
0.78	0.133095461900593	0.498474193737904	0.85935910089082\\
0.78	0.14133524079124	0.512537638903683	0.856028283765736\\
0.78	0.149820561306021	0.526493040810599	0.851449530870183\\
0.78	0.158549418718625	0.540334081158348	0.845707614739505\\
0.78	0.167519594442214	0.554054486817265	0.83891458434315\\
0.78	0.176728657084455	0.567648038782867	0.831205342370836\\
0.78	0.186173963948925	0.581108581114919	0.82273230541753\\
0.78	0.195852662983903	0.594430029835237	0.813659432944309\\
0.78	0.205761695177907	0.607606381758231	0.804155944104868\\
0.78	0.215897797399558	0.620631723228143	0.794390053406969\\
0.78	0.226257505677663	0.633500238737009	0.7845230460628\\
0.78	0.236837158915675	0.646206219397571	0.774703982861589\\
0.78	0.247632903032949	0.658744071245709	0.765065275211016\\
0.78	0.258640695523528	0.671108323347402	0.755719307772964\\
0.78	0.269856310421543	0.683293635685828	0.746756214005914\\
0.78	0.28127534366066	0.695294806804925	0.738242834604294\\
0.78	0.292893218813452	0.707106781186547	0.730222816007836\\
0.78	0.304705193195075	0.71872465633934	0.722717741087767\\
0.78	0.316706364314172	0.730143689578457	0.715729131143265\\
0.78	0.328891676652598	0.741359304476472	0.709241120561857\\
0.78	0.341255928754291	0.752367096967051	0.703223584571265\\
0.78	0.353793780602429	0.763162841084325	0.697635496597417\\
0.78	0.366499761262991	0.773742494322337	0.692428303594618\\
0.78	0.379368276771857	0.784102202600442	0.687549132887069\\
0.78	0.392393618241769	0.794238304822092	0.682943679235329\\
0.78	0.405569970164763	0.804147337016097	0.678558662193046\\
0.78	0.418891418885081	0.813826036051075	0.674343787412872\\
0.78	0.432351961217133	0.823271342915545	0.670253187715552\\
0.78	0.445945513182735	0.832480405557787	0.666246357336158\\
0.78	0.459665918841652	0.841450581281375	0.662288623474185\\
0.78	0.473506959189401	0.850179438693979	0.658351221673191\\
0.78	0.487462361096317	0.85866475920876	0.654411055137886\\
0.78	0.501525806262096	0.866904538099407	0.650450223206467\\
0.78	0.515690940160279	0.874896985111485	0.646455401874959\\
0.78	0.529951380947623	0.882640524634437	0.642417151055155\\
0.78	0.544300728313782	0.890133795440129	0.638329210947585\\
0.78	0.558732572247415	0.897375649995373	0.634187835387498\\
0.78	0.57324050169555	0.904365153357279	0.629991194998376\\
0.78	0.587818113093873	0.911101581661729	0.625738868900924\\
0.78	0.602459018746568	0.917584420216504	0.62143143163314\\
0.78	0.617156855035314	0.923813361211857	0.617070132480554\\
0.78	0.631905290438127	0.92978830106243	0.612656657833838\\
0.78	0.6466980333399	0.935509337395467	0.608192963377087\\
0.78	0.661528839617671	0.940976765701173	0.603681161495105\\
0.78	0.676391519984904	0.946191075661951	0.599123449738607\\
0.78	0.691279947080357	0.951152947177946	0.59452206790159\\
0.78	0.706188062288412	0.955863246106974	0.589879273662389\\
0.78	0.721109882279076	0.960323019737426	0.585197329319936\\
0.78	0.736039505257225	0.964533492013186	0.580478494543978\\
0.78	0.75097111691199	0.968496058529893	0.575725022018313\\
0.78	0.765898996058536	0.972212281322139	0.570939154289558\\
0.78	0.780817519965826	0.975683883461272	0.566123121053175\\
0.78	0.795721169365275	0.97891274348356	0.561279136605558\\
0.78	0.810604533136476	0.981900889668376	0.556409397399198\\
0.78	0.825462312667457	0.984650494185936	0.551516079695736\\
0.78	0.840289325888138	0.987163867133889	0.546601337319641\\
0.78	0.855080510976839	0.989443450481768	0.541667299515139\\
0.78	0.86983092974082	0.991491811941914	0.536716068908598\\
0.78	0.884535770672926	0.993311638785082	0.531749719578141\\
0.78	0.899190351687411	0.994905731618374	0.526770295231796\\
0.78	0.913790122539012	0.996276998142666	0.521779807495104\\
0.78	0.928330666930242	0.997428446906011	0.516780234308684\\
0.78	0.942807704312699	0.998363181068902	0.511773518435873\\
0.78	0.95721709138901	0.999084392196567	0.506761566080168\\
0.78	0.971554823322703	0.999595354092743	0.501746245611874\\
0.78	0.985817034663989	0.999899416688637	0.496729386403004\\
0.78	1	1	0.491669087346906\\
0.795	0	0	0.710195569525497\\
0.795	0.000100583311362513	0.0141829653360114	0.71240460418299\\
0.795	0.000404645907256436	0.0284451766772965	0.714630690536905\\
0.795	0.000915607803433	0.0427829086109896	0.716916312064106\\
0.795	0.00163681893109844	0.057192295687301	0.719289019360234\\
0.795	0.00257155309398959	0.0716693330697585	0.721780367061016\\
0.795	0.00372300185733414	0.086209877460988	0.724425655492835\\
0.795	0.00509426838162598	0.100809648312589	0.727263413850304\\
0.795	0.00668836121491816	0.115464229327074	0.73033458997498\\
0.795	0.00850818805808555	0.13016907025918	0.733681422514709\\
0.795	0.0105565495182326	0.144919489023162	0.737345986077549\\
0.795	0.0128361328661109	0.159710674111862	0.741368418705642\\
0.795	0.0153495058140643	0.174537687332543	0.745784862958202\\
0.795	0.0180991103316243	0.189395466863524	0.750625176092311\\
0.795	0.0210872565164405	0.204278830634725	0.755910489868998\\
0.795	0.0243161165387281	0.219182480034174	0.761650724664453\\
0.795	0.0277877186778607	0.234101003941464	0.767842183867616\\
0.795	0.0315039414701067	0.24902888308801	0.77446537092041\\
0.795	0.0354665079868145	0.263960494742775	0.781483180780606\\
0.795	0.0396769802625738	0.278890117720924	0.788839618268073\\
0.795	0.0441367538930258	0.293811937711588	0.796459186325544\\
0.795	0.0488470528220538	0.308720052919643	0.804247066923454\\
0.795	0.0538089243380495	0.323608480015096	0.81209018616195\\
0.795	0.0590232342988274	0.338471160382329	0.819859213932694\\
0.795	0.064490662604533	0.3533019666601	0.827411499063639\\
0.795	0.0702116989375697	0.368094709561873	0.834594885813504\\
0.795	0.0761866387881432	0.382843144964686	0.841252300287179\\
0.795	0.0824155797834956	0.397540981253432	0.847226939729081\\
0.795	0.0888984183382709	0.412181886906127	0.852367847905041\\
0.795	0.0956348466427212	0.42675949830445	0.856535620027829\\
0.795	0.102624350004627	0.441267427752584	0.859607954619618\\
0.795	0.109866204559871	0.455699271686219	0.861484760266622\\
0.795	0.117359475365564	0.470048619052377	0.86209253425154\\
0.795	0.125103014888515	0.484309059839721	0.861387758068368\\
0.795	0.133095461900593	0.498474193737904	0.85935910089082\\
0.795	0.14133524079124	0.512537638903683	0.856028283765735\\
0.795	0.149820561306021	0.526493040810599	0.851449530870183\\
0.795	0.158549418718625	0.540334081158348	0.845707614739505\\
0.795	0.167519594442214	0.554054486817265	0.83891458434315\\
0.795	0.176728657084455	0.567648038782867	0.831205342370837\\
0.795	0.186173963948925	0.581108581114919	0.822732305417528\\
0.795	0.195852662983903	0.594430029835237	0.813659432944308\\
0.795	0.205761695177907	0.607606381758231	0.804155944104868\\
0.795	0.215897797399558	0.620631723228143	0.794390053406969\\
0.795	0.226257505677663	0.633500238737009	0.7845230460628\\
0.795	0.236837158915675	0.646206219397571	0.774703982861589\\
0.795	0.247632903032949	0.658744071245709	0.765065275211016\\
0.795	0.258640695523528	0.671108323347402	0.755719307772964\\
0.795	0.269856310421543	0.683293635685828	0.746756214005914\\
0.795	0.28127534366066	0.695294806804925	0.738242834604294\\
0.795	0.292893218813452	0.707106781186547	0.730222816007836\\
0.795	0.304705193195075	0.71872465633934	0.722717741087767\\
0.795	0.316706364314172	0.730143689578457	0.715729131143265\\
0.795	0.328891676652598	0.741359304476472	0.709241120561857\\
0.795	0.341255928754291	0.752367096967051	0.703223584571265\\
0.795	0.353793780602429	0.763162841084325	0.697635496597417\\
0.795	0.366499761262991	0.773742494322337	0.692428303594618\\
0.795	0.379368276771857	0.784102202600442	0.687549132887069\\
0.795	0.392393618241769	0.794238304822092	0.682943679235329\\
0.795	0.405569970164763	0.804147337016097	0.678558662193046\\
0.795	0.418891418885081	0.813826036051075	0.674343787412872\\
0.795	0.432351961217133	0.823271342915545	0.670253187715552\\
0.795	0.445945513182735	0.832480405557787	0.666246357336158\\
0.795	0.459665918841652	0.841450581281375	0.662288623474185\\
0.795	0.473506959189401	0.850179438693979	0.658351221673191\\
0.795	0.487462361096317	0.85866475920876	0.654411055137886\\
0.795	0.501525806262096	0.866904538099407	0.650450223206467\\
0.795	0.515690940160279	0.874896985111485	0.646455401874959\\
0.795	0.529951380947623	0.882640524634437	0.642417151055155\\
0.795	0.544300728313782	0.890133795440129	0.638329210947586\\
0.795	0.558732572247415	0.897375649995373	0.634187835387499\\
0.795	0.57324050169555	0.904365153357279	0.629991194998376\\
0.795	0.587818113093873	0.911101581661729	0.625738868900924\\
0.795	0.602459018746568	0.917584420216504	0.62143143163314\\
0.795	0.617156855035314	0.923813361211857	0.617070132480554\\
0.795	0.631905290438127	0.92978830106243	0.612656657833838\\
0.795	0.6466980333399	0.935509337395467	0.608192963377087\\
0.795	0.661528839617671	0.940976765701173	0.603681161495105\\
0.795	0.676391519984904	0.946191075661951	0.599123449738607\\
0.795	0.691279947080357	0.951152947177946	0.59452206790159\\
0.795	0.706188062288412	0.955863246106974	0.589879273662389\\
0.795	0.721109882279076	0.960323019737426	0.585197329319936\\
0.795	0.736039505257225	0.964533492013186	0.580478494543978\\
0.795	0.75097111691199	0.968496058529893	0.575725022018313\\
0.795	0.765898996058536	0.972212281322139	0.570939154289558\\
0.795	0.780817519965826	0.975683883461272	0.566123121053175\\
0.795	0.795721169365275	0.97891274348356	0.561279136605558\\
0.795	0.810604533136476	0.981900889668376	0.556409397399198\\
0.795	0.825462312667457	0.984650494185936	0.551516079695735\\
0.795	0.840289325888138	0.987163867133889	0.546601337319641\\
0.795	0.855080510976839	0.989443450481768	0.541667299515139\\
0.795	0.86983092974082	0.991491811941914	0.536716068908598\\
0.795	0.884535770672926	0.993311638785082	0.53174971957814\\
0.795	0.899190351687411	0.994905731618374	0.526770295231796\\
0.795	0.913790122539012	0.996276998142666	0.521779807495104\\
0.795	0.928330666930242	0.997428446906011	0.516780234308682\\
0.795	0.942807704312699	0.998363181068902	0.511773518435872\\
0.795	0.95721709138901	0.999084392196567	0.506761566080171\\
0.795	0.971554823322704	0.999595354092744	0.501746245611874\\
0.795	0.985817034663989	0.999899416688637	0.496729386403002\\
0.795	1	1	0.491669087346908\\
0.81	0	0	0.710195569525497\\
0.81	0.000100583311362513	0.0141829653360114	0.71240460418299\\
0.81	0.000404645907256436	0.0284451766772965	0.714630690536905\\
0.81	0.000915607803433	0.0427829086109896	0.716916312064106\\
0.81	0.00163681893109844	0.057192295687301	0.719289019360234\\
0.81	0.00257155309398959	0.0716693330697584	0.721780367061016\\
0.81	0.00372300185733414	0.086209877460988	0.724425655492835\\
0.81	0.00509426838162598	0.100809648312589	0.727263413850304\\
0.81	0.00668836121491816	0.115464229327074	0.73033458997498\\
0.81	0.00850818805808555	0.13016907025918	0.733681422514709\\
0.81	0.0105565495182326	0.144919489023162	0.737345986077549\\
0.81	0.0128361328661109	0.159710674111862	0.741368418705642\\
0.81	0.0153495058140643	0.174537687332543	0.745784862958201\\
0.81	0.0180991103316243	0.189395466863524	0.750625176092311\\
0.81	0.0210872565164405	0.204278830634725	0.755910489868998\\
0.81	0.0243161165387281	0.219182480034174	0.761650724664453\\
0.81	0.0277877186778607	0.234101003941464	0.767842183867616\\
0.81	0.0315039414701067	0.24902888308801	0.77446537092041\\
0.81	0.0354665079868145	0.263960494742775	0.781483180780606\\
0.81	0.0396769802625738	0.278890117720924	0.788839618268074\\
0.81	0.0441367538930258	0.293811937711588	0.796459186325544\\
0.81	0.0488470528220538	0.308720052919643	0.804247066923454\\
0.81	0.0538089243380495	0.323608480015096	0.81209018616195\\
0.81	0.0590232342988274	0.338471160382329	0.819859213932694\\
0.81	0.064490662604533	0.3533019666601	0.827411499063639\\
0.81	0.0702116989375697	0.368094709561873	0.834594885813504\\
0.81	0.0761866387881432	0.382843144964686	0.841252300287179\\
0.81	0.0824155797834956	0.397540981253432	0.847226939729081\\
0.81	0.0888984183382709	0.412181886906127	0.852367847905041\\
0.81	0.0956348466427212	0.42675949830445	0.856535620027829\\
0.81	0.102624350004627	0.441267427752584	0.859607954619617\\
0.81	0.109866204559871	0.455699271686219	0.861484760266622\\
0.81	0.117359475365564	0.470048619052377	0.86209253425154\\
0.81	0.125103014888515	0.484309059839721	0.861387758068368\\
0.81	0.133095461900593	0.498474193737904	0.859359100890822\\
0.81	0.14133524079124	0.512537638903683	0.856028283765736\\
0.81	0.149820561306021	0.526493040810599	0.851449530870183\\
0.81	0.158549418718625	0.540334081158348	0.845707614739505\\
0.81	0.167519594442214	0.554054486817265	0.83891458434315\\
0.81	0.176728657084455	0.567648038782867	0.831205342370839\\
0.81	0.186173963948925	0.581108581114919	0.822732305417527\\
0.81	0.195852662983903	0.594430029835237	0.813659432944309\\
0.81	0.205761695177907	0.607606381758231	0.804155944104868\\
0.81	0.215897797399558	0.620631723228143	0.794390053406969\\
0.81	0.226257505677663	0.633500238737009	0.7845230460628\\
0.81	0.236837158915675	0.646206219397571	0.774703982861589\\
0.81	0.247632903032949	0.658744071245709	0.765065275211016\\
0.81	0.258640695523528	0.671108323347402	0.755719307772964\\
0.81	0.269856310421543	0.683293635685828	0.746756214005914\\
0.81	0.28127534366066	0.695294806804925	0.738242834604294\\
0.81	0.292893218813452	0.707106781186547	0.730222816007836\\
0.81	0.304705193195075	0.71872465633934	0.722717741087767\\
0.81	0.316706364314172	0.730143689578457	0.715729131143265\\
0.81	0.328891676652598	0.741359304476472	0.709241120561857\\
0.81	0.341255928754291	0.752367096967051	0.703223584571265\\
0.81	0.353793780602429	0.763162841084325	0.697635496597417\\
0.81	0.366499761262991	0.773742494322337	0.692428303594618\\
0.81	0.379368276771857	0.784102202600442	0.687549132887069\\
0.81	0.392393618241769	0.794238304822092	0.682943679235329\\
0.81	0.405569970164763	0.804147337016097	0.678558662193046\\
0.81	0.418891418885081	0.813826036051075	0.674343787412872\\
0.81	0.432351961217133	0.823271342915545	0.670253187715552\\
0.81	0.445945513182735	0.832480405557787	0.666246357336158\\
0.81	0.459665918841652	0.841450581281375	0.662288623474185\\
0.81	0.473506959189401	0.850179438693979	0.658351221673191\\
0.81	0.487462361096317	0.85866475920876	0.654411055137886\\
0.81	0.501525806262096	0.866904538099407	0.650450223206467\\
0.81	0.515690940160279	0.874896985111485	0.646455401874959\\
0.81	0.529951380947623	0.882640524634437	0.642417151055155\\
0.81	0.544300728313782	0.890133795440129	0.638329210947585\\
0.81	0.558732572247415	0.897375649995373	0.634187835387499\\
0.81	0.57324050169555	0.904365153357279	0.629991194998376\\
0.81	0.587818113093873	0.911101581661729	0.625738868900924\\
0.81	0.602459018746568	0.917584420216504	0.62143143163314\\
0.81	0.617156855035314	0.923813361211857	0.617070132480554\\
0.81	0.631905290438127	0.92978830106243	0.612656657833838\\
0.81	0.6466980333399	0.935509337395467	0.608192963377087\\
0.81	0.661528839617671	0.940976765701173	0.603681161495105\\
0.81	0.676391519984904	0.946191075661951	0.599123449738607\\
0.81	0.691279947080357	0.951152947177946	0.59452206790159\\
0.81	0.706188062288412	0.955863246106974	0.589879273662389\\
0.81	0.721109882279076	0.960323019737426	0.585197329319936\\
0.81	0.736039505257225	0.964533492013186	0.580478494543978\\
0.81	0.75097111691199	0.968496058529893	0.575725022018314\\
0.81	0.765898996058536	0.972212281322139	0.570939154289558\\
0.81	0.780817519965826	0.975683883461272	0.566123121053175\\
0.81	0.795721169365275	0.97891274348356	0.561279136605558\\
0.81	0.810604533136476	0.981900889668376	0.556409397399199\\
0.81	0.825462312667457	0.984650494185936	0.551516079695736\\
0.81	0.840289325888138	0.987163867133889	0.546601337319641\\
0.81	0.855080510976839	0.989443450481768	0.541667299515139\\
0.81	0.86983092974082	0.991491811941914	0.536716068908597\\
0.81	0.884535770672926	0.993311638785082	0.531749719578142\\
0.81	0.899190351687411	0.994905731618374	0.526770295231795\\
0.81	0.913790122539012	0.996276998142666	0.521779807495105\\
0.81	0.928330666930242	0.997428446906011	0.516780234308684\\
0.81	0.942807704312699	0.998363181068902	0.511773518435871\\
0.81	0.95721709138901	0.999084392196567	0.506761566080166\\
0.81	0.971554823322703	0.999595354092743	0.501746245611875\\
0.81	0.985817034663989	0.999899416688637	0.496729386403004\\
0.81	1	1	0.491669087346906\\
0.825	0	0	0.710195569525497\\
0.825	0.000100583311362513	0.0141829653360114	0.71240460418299\\
0.825	0.000404645907256436	0.0284451766772965	0.714630690536905\\
0.825	0.000915607803433	0.0427829086109896	0.716916312064106\\
0.825	0.00163681893109844	0.057192295687301	0.719289019360234\\
0.825	0.00257155309398959	0.0716693330697585	0.721780367061016\\
0.825	0.00372300185733414	0.0862098774609879	0.724425655492835\\
0.825	0.00509426838162598	0.100809648312589	0.727263413850304\\
0.825	0.00668836121491816	0.115464229327074	0.73033458997498\\
0.825	0.00850818805808555	0.13016907025918	0.733681422514709\\
0.825	0.0105565495182326	0.144919489023162	0.737345986077549\\
0.825	0.0128361328661109	0.159710674111862	0.741368418705642\\
0.825	0.0153495058140643	0.174537687332543	0.745784862958201\\
0.825	0.0180991103316243	0.189395466863524	0.750625176092311\\
0.825	0.0210872565164405	0.204278830634725	0.755910489868999\\
0.825	0.0243161165387281	0.219182480034174	0.761650724664453\\
0.825	0.0277877186778607	0.234101003941464	0.767842183867616\\
0.825	0.0315039414701067	0.24902888308801	0.77446537092041\\
0.825	0.0354665079868145	0.263960494742775	0.781483180780606\\
0.825	0.0396769802625738	0.278890117720924	0.788839618268073\\
0.825	0.0441367538930258	0.293811937711588	0.796459186325544\\
0.825	0.0488470528220538	0.308720052919643	0.804247066923454\\
0.825	0.0538089243380495	0.323608480015096	0.812090186161949\\
0.825	0.0590232342988274	0.338471160382329	0.819859213932694\\
0.825	0.064490662604533	0.3533019666601	0.827411499063639\\
0.825	0.0702116989375697	0.368094709561873	0.834594885813504\\
0.825	0.0761866387881432	0.382843144964686	0.841252300287179\\
0.825	0.0824155797834956	0.397540981253432	0.847226939729081\\
0.825	0.0888984183382709	0.412181886906127	0.852367847905041\\
0.825	0.0956348466427212	0.42675949830445	0.856535620027829\\
0.825	0.102624350004627	0.441267427752584	0.859607954619618\\
0.825	0.109866204559871	0.455699271686219	0.861484760266621\\
0.825	0.117359475365564	0.470048619052377	0.86209253425154\\
0.825	0.125103014888515	0.484309059839721	0.861387758068368\\
0.825	0.133095461900593	0.498474193737904	0.85935910089082\\
0.825	0.14133524079124	0.512537638903683	0.856028283765738\\
0.825	0.149820561306021	0.526493040810599	0.851449530870183\\
0.825	0.158549418718625	0.540334081158348	0.845707614739505\\
0.825	0.167519594442214	0.554054486817265	0.838914584343151\\
0.825	0.176728657084455	0.567648038782868	0.831205342370836\\
0.825	0.186173963948925	0.581108581114919	0.822732305417528\\
0.825	0.195852662983903	0.594430029835237	0.813659432944309\\
0.825	0.205761695177907	0.607606381758231	0.804155944104868\\
0.825	0.215897797399558	0.620631723228143	0.794390053406969\\
0.825	0.226257505677663	0.633500238737009	0.7845230460628\\
0.825	0.236837158915675	0.646206219397571	0.774703982861589\\
0.825	0.247632903032949	0.658744071245709	0.765065275211016\\
0.825	0.258640695523528	0.671108323347402	0.755719307772964\\
0.825	0.269856310421543	0.683293635685828	0.746756214005914\\
0.825	0.28127534366066	0.695294806804925	0.738242834604294\\
0.825	0.292893218813452	0.707106781186547	0.730222816007836\\
0.825	0.304705193195075	0.71872465633934	0.722717741087767\\
0.825	0.316706364314172	0.730143689578457	0.715729131143265\\
0.825	0.328891676652598	0.741359304476472	0.709241120561857\\
0.825	0.341255928754291	0.752367096967051	0.703223584571265\\
0.825	0.353793780602429	0.763162841084325	0.697635496597417\\
0.825	0.366499761262991	0.773742494322337	0.692428303594618\\
0.825	0.379368276771857	0.784102202600442	0.687549132887069\\
0.825	0.392393618241769	0.794238304822092	0.682943679235329\\
0.825	0.405569970164763	0.804147337016097	0.678558662193046\\
0.825	0.418891418885081	0.813826036051075	0.674343787412872\\
0.825	0.432351961217133	0.823271342915545	0.670253187715552\\
0.825	0.445945513182735	0.832480405557787	0.666246357336158\\
0.825	0.459665918841652	0.841450581281375	0.662288623474185\\
0.825	0.473506959189401	0.850179438693979	0.658351221673191\\
0.825	0.487462361096317	0.85866475920876	0.654411055137886\\
0.825	0.501525806262096	0.866904538099407	0.650450223206467\\
0.825	0.515690940160279	0.874896985111485	0.646455401874959\\
0.825	0.529951380947623	0.882640524634437	0.642417151055155\\
0.825	0.544300728313782	0.890133795440129	0.638329210947585\\
0.825	0.558732572247415	0.897375649995373	0.634187835387498\\
0.825	0.57324050169555	0.904365153357279	0.629991194998376\\
0.825	0.587818113093873	0.911101581661729	0.625738868900924\\
0.825	0.602459018746568	0.917584420216504	0.62143143163314\\
0.825	0.617156855035314	0.923813361211857	0.617070132480554\\
0.825	0.631905290438127	0.92978830106243	0.612656657833838\\
0.825	0.6466980333399	0.935509337395467	0.608192963377087\\
0.825	0.661528839617671	0.940976765701173	0.603681161495105\\
0.825	0.676391519984904	0.946191075661951	0.599123449738607\\
0.825	0.691279947080357	0.951152947177946	0.59452206790159\\
0.825	0.706188062288412	0.955863246106974	0.589879273662389\\
0.825	0.721109882279076	0.960323019737426	0.585197329319936\\
0.825	0.736039505257225	0.964533492013186	0.580478494543975\\
0.825	0.75097111691199	0.968496058529893	0.575725022018313\\
0.825	0.765898996058536	0.972212281322139	0.57093915428956\\
0.825	0.780817519965826	0.975683883461272	0.566123121053175\\
0.825	0.795721169365275	0.97891274348356	0.561279136605558\\
0.825	0.810604533136476	0.981900889668376	0.556409397399196\\
0.825	0.825462312667457	0.984650494185936	0.551516079695737\\
0.825	0.840289325888138	0.987163867133889	0.546601337319642\\
0.825	0.855080510976839	0.989443450481768	0.541667299515141\\
0.825	0.86983092974082	0.991491811941915	0.536716068908595\\
0.825	0.884535770672926	0.993311638785082	0.531749719578141\\
0.825	0.899190351687411	0.994905731618374	0.526770295231796\\
0.825	0.913790122539012	0.996276998142666	0.521779807495103\\
0.825	0.928330666930242	0.997428446906011	0.516780234308685\\
0.825	0.942807704312699	0.998363181068902	0.511773518435873\\
0.825	0.95721709138901	0.999084392196567	0.506761566080166\\
0.825	0.971554823322703	0.999595354092743	0.501746245611874\\
0.825	0.985817034663989	0.999899416688637	0.496729386403005\\
0.825	1	1	0.491669087346904\\
0.84	0	0	0.710195569525497\\
0.84	0.000100583311362513	0.0141829653360114	0.71240460418299\\
0.84	0.000404645907256436	0.0284451766772965	0.714630690536905\\
0.84	0.000915607803433	0.0427829086109896	0.716916312064106\\
0.84	0.00163681893109844	0.057192295687301	0.719289019360234\\
0.84	0.00257155309398959	0.0716693330697584	0.721780367061016\\
0.84	0.00372300185733414	0.086209877460988	0.724425655492835\\
0.84	0.00509426838162598	0.100809648312589	0.727263413850304\\
0.84	0.00668836121491816	0.115464229327074	0.73033458997498\\
0.84	0.00850818805808555	0.13016907025918	0.733681422514709\\
0.84	0.0105565495182326	0.144919489023162	0.737345986077549\\
0.84	0.0128361328661109	0.159710674111862	0.741368418705642\\
0.84	0.0153495058140643	0.174537687332543	0.745784862958201\\
0.84	0.0180991103316243	0.189395466863524	0.750625176092311\\
0.84	0.0210872565164405	0.204278830634725	0.755910489868998\\
0.84	0.0243161165387281	0.219182480034174	0.761650724664453\\
0.84	0.0277877186778607	0.234101003941464	0.767842183867616\\
0.84	0.0315039414701067	0.24902888308801	0.77446537092041\\
0.84	0.0354665079868145	0.263960494742775	0.781483180780606\\
0.84	0.0396769802625738	0.278890117720924	0.788839618268073\\
0.84	0.0441367538930258	0.293811937711588	0.796459186325544\\
0.84	0.0488470528220538	0.308720052919643	0.804247066923455\\
0.84	0.0538089243380495	0.323608480015096	0.81209018616195\\
0.84	0.0590232342988274	0.338471160382329	0.819859213932693\\
0.84	0.064490662604533	0.3533019666601	0.827411499063639\\
0.84	0.0702116989375697	0.368094709561873	0.834594885813504\\
0.84	0.0761866387881432	0.382843144964686	0.84125230028718\\
0.84	0.0824155797834956	0.397540981253432	0.847226939729081\\
0.84	0.0888984183382709	0.412181886906127	0.852367847905041\\
0.84	0.0956348466427212	0.42675949830445	0.856535620027828\\
0.84	0.102624350004627	0.441267427752584	0.859607954619618\\
0.84	0.109866204559871	0.455699271686219	0.861484760266622\\
0.84	0.117359475365564	0.470048619052377	0.86209253425154\\
0.84	0.125103014888515	0.484309059839721	0.861387758068368\\
0.84	0.133095461900593	0.498474193737904	0.859359100890819\\
0.84	0.14133524079124	0.512537638903683	0.856028283765736\\
0.84	0.149820561306021	0.526493040810599	0.851449530870183\\
0.84	0.158549418718625	0.540334081158348	0.845707614739505\\
0.84	0.167519594442214	0.554054486817265	0.83891458434315\\
0.84	0.176728657084455	0.567648038782867	0.831205342370837\\
0.84	0.186173963948925	0.581108581114919	0.822732305417528\\
0.84	0.195852662983903	0.594430029835237	0.813659432944308\\
0.84	0.205761695177907	0.607606381758231	0.804155944104868\\
0.84	0.215897797399558	0.620631723228143	0.794390053406969\\
0.84	0.226257505677663	0.633500238737009	0.7845230460628\\
0.84	0.236837158915675	0.646206219397571	0.774703982861589\\
0.84	0.247632903032949	0.658744071245709	0.765065275211016\\
0.84	0.258640695523528	0.671108323347402	0.755719307772964\\
0.84	0.269856310421543	0.683293635685828	0.746756214005914\\
0.84	0.28127534366066	0.695294806804925	0.738242834604294\\
0.84	0.292893218813452	0.707106781186547	0.730222816007836\\
0.84	0.304705193195075	0.71872465633934	0.722717741087767\\
0.84	0.316706364314172	0.730143689578457	0.715729131143265\\
0.84	0.328891676652598	0.741359304476472	0.709241120561857\\
0.84	0.341255928754291	0.752367096967051	0.703223584571265\\
0.84	0.353793780602429	0.763162841084325	0.697635496597417\\
0.84	0.366499761262991	0.773742494322337	0.692428303594618\\
0.84	0.379368276771857	0.784102202600442	0.687549132887069\\
0.84	0.392393618241769	0.794238304822092	0.682943679235329\\
0.84	0.405569970164763	0.804147337016097	0.678558662193046\\
0.84	0.418891418885081	0.813826036051075	0.674343787412872\\
0.84	0.432351961217133	0.823271342915545	0.670253187715552\\
0.84	0.445945513182735	0.832480405557787	0.666246357336158\\
0.84	0.459665918841652	0.841450581281375	0.662288623474185\\
0.84	0.473506959189401	0.850179438693979	0.658351221673191\\
0.84	0.487462361096317	0.85866475920876	0.654411055137886\\
0.84	0.501525806262096	0.866904538099407	0.650450223206467\\
0.84	0.515690940160279	0.874896985111485	0.646455401874959\\
0.84	0.529951380947623	0.882640524634437	0.642417151055155\\
0.84	0.544300728313782	0.890133795440129	0.638329210947585\\
0.84	0.558732572247415	0.897375649995373	0.634187835387498\\
0.84	0.57324050169555	0.904365153357279	0.629991194998376\\
0.84	0.587818113093873	0.911101581661729	0.625738868900924\\
0.84	0.602459018746568	0.917584420216504	0.62143143163314\\
0.84	0.617156855035314	0.923813361211857	0.617070132480554\\
0.84	0.631905290438127	0.92978830106243	0.612656657833838\\
0.84	0.6466980333399	0.935509337395467	0.608192963377087\\
0.84	0.661528839617671	0.940976765701173	0.603681161495105\\
0.84	0.676391519984904	0.946191075661951	0.599123449738607\\
0.84	0.691279947080357	0.951152947177946	0.59452206790159\\
0.84	0.706188062288412	0.955863246106974	0.589879273662389\\
0.84	0.721109882279076	0.960323019737426	0.585197329319936\\
0.84	0.736039505257225	0.964533492013186	0.580478494543978\\
0.84	0.75097111691199	0.968496058529893	0.575725022018313\\
0.84	0.765898996058536	0.972212281322139	0.570939154289558\\
0.84	0.780817519965826	0.975683883461272	0.566123121053176\\
0.84	0.795721169365275	0.97891274348356	0.561279136605558\\
0.84	0.810604533136476	0.981900889668376	0.556409397399196\\
0.84	0.825462312667457	0.984650494185936	0.551516079695736\\
0.84	0.840289325888138	0.987163867133889	0.546601337319642\\
0.84	0.855080510976839	0.989443450481768	0.541667299515141\\
0.84	0.86983092974082	0.991491811941915	0.536716068908596\\
0.84	0.884535770672926	0.993311638785082	0.531749719578139\\
0.84	0.899190351687411	0.994905731618374	0.526770295231797\\
0.84	0.913790122539012	0.996276998142666	0.521779807495104\\
0.84	0.928330666930242	0.997428446906011	0.516780234308682\\
0.84	0.942807704312699	0.998363181068902	0.511773518435872\\
0.84	0.95721709138901	0.999084392196567	0.506761566080171\\
0.84	0.971554823322704	0.999595354092744	0.501746245611874\\
0.84	0.985817034663989	0.999899416688637	0.496729386403002\\
0.84	1	1	0.491669087346908\\
0.855	0	0	0.710195569525497\\
0.855	0.000100583311362513	0.0141829653360114	0.71240460418299\\
0.855	0.000404645907256436	0.0284451766772965	0.714630690536905\\
0.855	0.000915607803433	0.0427829086109896	0.716916312064106\\
0.855	0.00163681893109844	0.057192295687301	0.719289019360234\\
0.855	0.00257155309398959	0.0716693330697585	0.721780367061016\\
0.855	0.00372300185733414	0.086209877460988	0.724425655492835\\
0.855	0.00509426838162598	0.100809648312589	0.727263413850304\\
0.855	0.00668836121491816	0.115464229327074	0.73033458997498\\
0.855	0.00850818805808555	0.13016907025918	0.733681422514709\\
0.855	0.0105565495182326	0.144919489023162	0.737345986077549\\
0.855	0.0128361328661109	0.159710674111862	0.741368418705642\\
0.855	0.0153495058140643	0.174537687332543	0.745784862958202\\
0.855	0.0180991103316243	0.189395466863524	0.750625176092311\\
0.855	0.0210872565164405	0.204278830634725	0.755910489868998\\
0.855	0.0243161165387281	0.219182480034174	0.761650724664453\\
0.855	0.0277877186778607	0.234101003941464	0.767842183867616\\
0.855	0.0315039414701067	0.24902888308801	0.77446537092041\\
0.855	0.0354665079868145	0.263960494742775	0.781483180780606\\
0.855	0.0396769802625737	0.278890117720924	0.788839618268073\\
0.855	0.0441367538930258	0.293811937711588	0.796459186325544\\
0.855	0.0488470528220538	0.308720052919643	0.804247066923454\\
0.855	0.0538089243380495	0.323608480015096	0.81209018616195\\
0.855	0.0590232342988274	0.338471160382329	0.819859213932693\\
0.855	0.064490662604533	0.3533019666601	0.82741149906364\\
0.855	0.0702116989375697	0.368094709561873	0.834594885813504\\
0.855	0.0761866387881433	0.382843144964686	0.841252300287179\\
0.855	0.0824155797834956	0.397540981253432	0.847226939729081\\
0.855	0.0888984183382709	0.412181886906127	0.852367847905041\\
0.855	0.0956348466427212	0.42675949830445	0.85653562002783\\
0.855	0.102624350004627	0.441267427752584	0.859607954619617\\
0.855	0.109866204559871	0.455699271686219	0.86148476026662\\
0.855	0.117359475365564	0.470048619052377	0.86209253425154\\
0.855	0.125103014888515	0.484309059839721	0.86138775806837\\
0.855	0.133095461900593	0.498474193737904	0.85935910089082\\
0.855	0.14133524079124	0.512537638903683	0.856028283765736\\
0.855	0.149820561306021	0.526493040810599	0.851449530870183\\
0.855	0.158549418718625	0.540334081158348	0.845707614739502\\
0.855	0.167519594442214	0.554054486817265	0.838914584343151\\
0.855	0.176728657084455	0.567648038782868	0.831205342370836\\
0.855	0.186173963948925	0.581108581114919	0.822732305417525\\
0.855	0.195852662983903	0.594430029835237	0.813659432944309\\
0.855	0.205761695177907	0.607606381758231	0.804155944104868\\
0.855	0.215897797399558	0.620631723228143	0.794390053406969\\
0.855	0.226257505677663	0.633500238737009	0.7845230460628\\
0.855	0.236837158915675	0.646206219397571	0.774703982861589\\
0.855	0.247632903032949	0.658744071245709	0.765065275211016\\
0.855	0.258640695523528	0.671108323347402	0.755719307772964\\
0.855	0.269856310421543	0.683293635685828	0.746756214005914\\
0.855	0.28127534366066	0.695294806804925	0.738242834604294\\
0.855	0.292893218813452	0.707106781186547	0.730222816007836\\
0.855	0.304705193195075	0.71872465633934	0.722717741087767\\
0.855	0.316706364314172	0.730143689578457	0.715729131143265\\
0.855	0.328891676652598	0.741359304476472	0.709241120561857\\
0.855	0.341255928754291	0.752367096967051	0.703223584571265\\
0.855	0.353793780602429	0.763162841084325	0.697635496597417\\
0.855	0.366499761262991	0.773742494322337	0.692428303594618\\
0.855	0.379368276771857	0.784102202600442	0.687549132887069\\
0.855	0.392393618241769	0.794238304822092	0.682943679235329\\
0.855	0.405569970164763	0.804147337016097	0.678558662193046\\
0.855	0.418891418885081	0.813826036051075	0.674343787412872\\
0.855	0.432351961217133	0.823271342915545	0.670253187715552\\
0.855	0.445945513182735	0.832480405557787	0.666246357336158\\
0.855	0.459665918841652	0.841450581281375	0.662288623474185\\
0.855	0.473506959189401	0.850179438693979	0.658351221673191\\
0.855	0.487462361096317	0.85866475920876	0.654411055137887\\
0.855	0.501525806262096	0.866904538099407	0.650450223206467\\
0.855	0.515690940160279	0.874896985111485	0.646455401874959\\
0.855	0.529951380947623	0.882640524634437	0.642417151055155\\
0.855	0.544300728313782	0.890133795440129	0.638329210947585\\
0.855	0.558732572247415	0.897375649995373	0.634187835387498\\
0.855	0.57324050169555	0.904365153357279	0.629991194998375\\
0.855	0.587818113093873	0.911101581661729	0.625738868900924\\
0.855	0.602459018746568	0.917584420216504	0.62143143163314\\
0.855	0.617156855035314	0.923813361211857	0.617070132480554\\
0.855	0.631905290438127	0.92978830106243	0.612656657833838\\
0.855	0.6466980333399	0.935509337395467	0.608192963377087\\
0.855	0.661528839617671	0.940976765701173	0.603681161495105\\
0.855	0.676391519984904	0.946191075661951	0.599123449738607\\
0.855	0.691279947080357	0.951152947177946	0.59452206790159\\
0.855	0.706188062288412	0.955863246106974	0.589879273662389\\
0.855	0.721109882279076	0.960323019737426	0.585197329319936\\
0.855	0.736039505257225	0.964533492013186	0.580478494543978\\
0.855	0.75097111691199	0.968496058529893	0.575725022018314\\
0.855	0.765898996058536	0.972212281322139	0.570939154289558\\
0.855	0.780817519965826	0.975683883461272	0.566123121053175\\
0.855	0.795721169365275	0.97891274348356	0.561279136605558\\
0.855	0.810604533136476	0.981900889668376	0.556409397399196\\
0.855	0.825462312667457	0.984650494185936	0.551516079695735\\
0.855	0.840289325888138	0.987163867133889	0.546601337319642\\
0.855	0.855080510976839	0.989443450481768	0.541667299515141\\
0.855	0.86983092974082	0.991491811941915	0.536716068908598\\
0.855	0.884535770672926	0.993311638785082	0.531749719578139\\
0.855	0.899190351687411	0.994905731618374	0.526770295231796\\
0.855	0.913790122539012	0.996276998142666	0.521779807495104\\
0.855	0.928330666930241	0.997428446906011	0.516780234308685\\
0.855	0.942807704312699	0.998363181068902	0.511773518435871\\
0.855	0.95721709138901	0.999084392196567	0.506761566080166\\
0.855	0.971554823322703	0.999595354092743	0.501746245611875\\
0.855	0.985817034663989	0.999899416688637	0.496729386403004\\
0.855	1	1	0.491669087346906\\
0.87	0	0	0.710195569525497\\
0.87	0.000100583311362513	0.0141829653360114	0.71240460418299\\
0.87	0.000404645907256436	0.0284451766772965	0.714630690536905\\
0.87	0.000915607803433	0.0427829086109896	0.716916312064106\\
0.87	0.00163681893109844	0.057192295687301	0.719289019360234\\
0.87	0.00257155309398959	0.0716693330697585	0.721780367061016\\
0.87	0.00372300185733414	0.086209877460988	0.724425655492835\\
0.87	0.00509426838162598	0.100809648312589	0.727263413850304\\
0.87	0.00668836121491816	0.115464229327074	0.73033458997498\\
0.87	0.00850818805808555	0.13016907025918	0.733681422514709\\
0.87	0.0105565495182326	0.144919489023162	0.737345986077549\\
0.87	0.0128361328661109	0.159710674111862	0.741368418705642\\
0.87	0.0153495058140643	0.174537687332543	0.745784862958202\\
0.87	0.0180991103316243	0.189395466863524	0.750625176092311\\
0.87	0.0210872565164405	0.204278830634725	0.755910489868998\\
0.87	0.0243161165387281	0.219182480034174	0.761650724664453\\
0.87	0.0277877186778607	0.234101003941464	0.767842183867616\\
0.87	0.0315039414701067	0.24902888308801	0.77446537092041\\
0.87	0.0354665079868145	0.263960494742775	0.781483180780606\\
0.87	0.0396769802625738	0.278890117720924	0.788839618268073\\
0.87	0.0441367538930258	0.293811937711588	0.796459186325544\\
0.87	0.0488470528220538	0.308720052919643	0.804247066923454\\
0.87	0.0538089243380495	0.323608480015097	0.81209018616195\\
0.87	0.0590232342988274	0.338471160382329	0.819859213932693\\
0.87	0.064490662604533	0.3533019666601	0.82741149906364\\
0.87	0.0702116989375697	0.368094709561873	0.834594885813505\\
0.87	0.0761866387881432	0.382843144964686	0.841252300287178\\
0.87	0.0824155797834956	0.397540981253432	0.847226939729081\\
0.87	0.0888984183382709	0.412181886906127	0.852367847905041\\
0.87	0.0956348466427212	0.42675949830445	0.85653562002783\\
0.87	0.102624350004627	0.441267427752584	0.859607954619618\\
0.87	0.109866204559871	0.455699271686218	0.861484760266622\\
0.87	0.117359475365564	0.470048619052377	0.86209253425154\\
0.87	0.125103014888515	0.484309059839721	0.861387758068368\\
0.87	0.133095461900593	0.498474193737904	0.859359100890821\\
0.87	0.14133524079124	0.512537638903683	0.856028283765737\\
0.87	0.149820561306021	0.526493040810599	0.851449530870182\\
0.87	0.158549418718625	0.540334081158348	0.845707614739504\\
0.87	0.167519594442214	0.554054486817265	0.838914584343151\\
0.87	0.176728657084455	0.567648038782867	0.831205342370832\\
0.87	0.186173963948925	0.581108581114919	0.82273230541753\\
0.87	0.195852662983903	0.594430029835237	0.813659432944309\\
0.87	0.205761695177907	0.607606381758231	0.804155944104868\\
0.87	0.215897797399558	0.620631723228143	0.794390053406971\\
0.87	0.226257505677663	0.633500238737009	0.7845230460628\\
0.87	0.236837158915675	0.646206219397571	0.774703982861588\\
0.87	0.247632903032949	0.658744071245709	0.765065275211016\\
0.87	0.258640695523528	0.671108323347402	0.755719307772965\\
0.87	0.269856310421543	0.683293635685828	0.746756214005914\\
0.87	0.28127534366066	0.695294806804925	0.738242834604294\\
0.87	0.292893218813452	0.707106781186547	0.730222816007836\\
0.87	0.304705193195075	0.71872465633934	0.722717741087766\\
0.87	0.316706364314172	0.730143689578457	0.715729131143265\\
0.87	0.328891676652598	0.741359304476472	0.709241120561857\\
0.87	0.341255928754291	0.752367096967051	0.703223584571265\\
0.87	0.353793780602429	0.763162841084325	0.697635496597417\\
0.87	0.366499761262991	0.773742494322337	0.692428303594618\\
0.87	0.379368276771857	0.784102202600442	0.687549132887069\\
0.87	0.392393618241769	0.794238304822092	0.682943679235329\\
0.87	0.405569970164763	0.804147337016097	0.678558662193046\\
0.87	0.418891418885081	0.813826036051075	0.674343787412872\\
0.87	0.432351961217133	0.823271342915545	0.670253187715552\\
0.87	0.445945513182735	0.832480405557787	0.666246357336157\\
0.87	0.459665918841652	0.841450581281375	0.662288623474184\\
0.87	0.473506959189401	0.850179438693979	0.658351221673191\\
0.87	0.487462361096317	0.85866475920876	0.654411055137886\\
0.87	0.501525806262096	0.866904538099407	0.650450223206468\\
0.87	0.515690940160279	0.874896985111485	0.646455401874959\\
0.87	0.529951380947623	0.882640524634437	0.642417151055155\\
0.87	0.544300728313782	0.890133795440129	0.638329210947586\\
0.87	0.558732572247415	0.897375649995373	0.634187835387499\\
0.87	0.57324050169555	0.904365153357279	0.629991194998375\\
0.87	0.587818113093873	0.911101581661729	0.625738868900923\\
0.87	0.602459018746568	0.917584420216504	0.62143143163314\\
0.87	0.617156855035314	0.923813361211857	0.617070132480554\\
0.87	0.631905290438127	0.92978830106243	0.612656657833838\\
0.87	0.6466980333399	0.935509337395467	0.608192963377087\\
0.87	0.661528839617671	0.940976765701173	0.603681161495105\\
0.87	0.676391519984904	0.946191075661951	0.599123449738607\\
0.87	0.691279947080357	0.951152947177946	0.59452206790159\\
0.87	0.706188062288412	0.955863246106974	0.589879273662388\\
0.87	0.721109882279076	0.960323019737426	0.585197329319936\\
0.87	0.736039505257225	0.964533492013186	0.580478494543975\\
0.87	0.75097111691199	0.968496058529893	0.575725022018313\\
0.87	0.765898996058536	0.972212281322139	0.57093915428956\\
0.87	0.780817519965826	0.975683883461272	0.566123121053176\\
0.87	0.795721169365275	0.97891274348356	0.561279136605558\\
0.87	0.810604533136476	0.981900889668376	0.556409397399198\\
0.87	0.825462312667457	0.984650494185936	0.551516079695737\\
0.87	0.840289325888138	0.987163867133889	0.546601337319641\\
0.87	0.855080510976839	0.989443450481768	0.541667299515141\\
0.87	0.86983092974082	0.991491811941915	0.536716068908598\\
0.87	0.884535770672926	0.993311638785082	0.53174971957814\\
0.87	0.899190351687411	0.994905731618374	0.526770295231796\\
0.87	0.913790122539012	0.996276998142666	0.521779807495105\\
0.87	0.928330666930242	0.997428446906011	0.516780234308682\\
0.87	0.942807704312699	0.998363181068902	0.511773518435873\\
0.87	0.95721709138901	0.999084392196567	0.506761566080166\\
0.87	0.971554823322703	0.999595354092743	0.501746245611874\\
0.87	0.985817034663989	0.999899416688637	0.496729386403005\\
0.87	1	1	0.491669087346904\\
0.885	0	0	0.710195569525497\\
0.885	0.000100583311362513	0.0141829653360114	0.71240460418299\\
0.885	0.000404645907256436	0.0284451766772965	0.714630690536905\\
0.885	0.000915607803433	0.0427829086109896	0.716916312064106\\
0.885	0.00163681893109844	0.057192295687301	0.719289019360234\\
0.885	0.00257155309398959	0.0716693330697585	0.721780367061016\\
0.885	0.00372300185733414	0.0862098774609879	0.724425655492835\\
0.885	0.00509426838162598	0.100809648312589	0.727263413850304\\
0.885	0.00668836121491816	0.115464229327074	0.73033458997498\\
0.885	0.00850818805808555	0.13016907025918	0.733681422514709\\
0.885	0.0105565495182326	0.144919489023162	0.737345986077549\\
0.885	0.0128361328661109	0.159710674111862	0.741368418705642\\
0.885	0.0153495058140643	0.174537687332543	0.745784862958202\\
0.885	0.0180991103316243	0.189395466863524	0.750625176092311\\
0.885	0.0210872565164405	0.204278830634725	0.755910489868998\\
0.885	0.0243161165387281	0.219182480034174	0.761650724664453\\
0.885	0.0277877186778607	0.234101003941464	0.767842183867616\\
0.885	0.0315039414701067	0.24902888308801	0.77446537092041\\
0.885	0.0354665079868145	0.263960494742775	0.781483180780606\\
0.885	0.0396769802625737	0.278890117720924	0.788839618268073\\
0.885	0.0441367538930258	0.293811937711588	0.796459186325544\\
0.885	0.0488470528220537	0.308720052919643	0.804247066923454\\
0.885	0.0538089243380495	0.323608480015096	0.812090186161949\\
0.885	0.0590232342988274	0.338471160382329	0.819859213932694\\
0.885	0.064490662604533	0.3533019666601	0.827411499063638\\
0.885	0.0702116989375697	0.368094709561873	0.834594885813505\\
0.885	0.0761866387881432	0.382843144964686	0.841252300287179\\
0.885	0.0824155797834956	0.397540981253432	0.847226939729081\\
0.885	0.0888984183382709	0.412181886906127	0.852367847905041\\
0.885	0.0956348466427212	0.42675949830445	0.856535620027828\\
0.885	0.102624350004627	0.441267427752584	0.859607954619618\\
0.885	0.109866204559871	0.455699271686219	0.861484760266623\\
0.885	0.117359475365564	0.470048619052377	0.86209253425154\\
0.885	0.125103014888515	0.484309059839721	0.861387758068367\\
0.885	0.133095461900593	0.498474193737904	0.85935910089082\\
0.885	0.14133524079124	0.512537638903683	0.856028283765735\\
0.885	0.149820561306021	0.526493040810599	0.851449530870182\\
0.885	0.158549418718625	0.540334081158348	0.845707614739509\\
0.885	0.167519594442213	0.554054486817265	0.838914584343151\\
0.885	0.176728657084455	0.567648038782867	0.831205342370839\\
0.885	0.186173963948925	0.581108581114919	0.822732305417529\\
0.885	0.195852662983903	0.594430029835237	0.813659432944308\\
0.885	0.205761695177907	0.607606381758231	0.804155944104869\\
0.885	0.215897797399558	0.620631723228144	0.794390053406971\\
0.885	0.226257505677663	0.633500238737009	0.784523046062799\\
0.885	0.236837158915675	0.646206219397571	0.774703982861589\\
0.885	0.247632903032949	0.658744071245709	0.765065275211017\\
0.885	0.258640695523528	0.671108323347402	0.755719307772964\\
0.885	0.269856310421543	0.683293635685828	0.746756214005912\\
0.885	0.28127534366066	0.695294806804924	0.738242834604294\\
0.885	0.292893218813452	0.707106781186547	0.730222816007835\\
0.885	0.304705193195075	0.71872465633934	0.722717741087767\\
0.885	0.316706364314172	0.730143689578457	0.715729131143266\\
0.885	0.328891676652598	0.741359304476472	0.709241120561857\\
0.885	0.341255928754291	0.752367096967051	0.703223584571265\\
0.885	0.353793780602429	0.763162841084325	0.697635496597417\\
0.885	0.366499761262991	0.773742494322337	0.692428303594619\\
0.885	0.379368276771857	0.784102202600443	0.687549132887069\\
0.885	0.392393618241769	0.794238304822092	0.682943679235329\\
0.885	0.405569970164763	0.804147337016097	0.678558662193046\\
0.885	0.418891418885081	0.813826036051075	0.674343787412872\\
0.885	0.432351961217133	0.823271342915545	0.670253187715552\\
0.885	0.445945513182735	0.832480405557787	0.666246357336158\\
0.885	0.459665918841652	0.841450581281375	0.662288623474184\\
0.885	0.473506959189401	0.850179438693979	0.658351221673191\\
0.885	0.487462361096317	0.85866475920876	0.654411055137886\\
0.885	0.501525806262096	0.866904538099407	0.650450223206467\\
0.885	0.515690940160279	0.874896985111485	0.646455401874959\\
0.885	0.529951380947623	0.882640524634437	0.642417151055155\\
0.885	0.544300728313782	0.890133795440129	0.638329210947585\\
0.885	0.558732572247415	0.897375649995373	0.634187835387499\\
0.885	0.57324050169555	0.904365153357279	0.629991194998377\\
0.885	0.587818113093873	0.911101581661729	0.625738868900924\\
0.885	0.602459018746568	0.917584420216504	0.62143143163314\\
0.885	0.617156855035314	0.923813361211857	0.617070132480554\\
0.885	0.631905290438127	0.92978830106243	0.612656657833838\\
0.885	0.6466980333399	0.935509337395467	0.608192963377087\\
0.885	0.661528839617671	0.940976765701173	0.603681161495105\\
0.885	0.676391519984904	0.946191075661951	0.599123449738608\\
0.885	0.691279947080357	0.951152947177946	0.594522067901592\\
0.885	0.706188062288412	0.955863246106974	0.589879273662387\\
0.885	0.721109882279076	0.960323019737426	0.585197329319934\\
0.885	0.736039505257225	0.964533492013186	0.580478494543978\\
0.885	0.75097111691199	0.968496058529893	0.575725022018314\\
0.885	0.765898996058536	0.972212281322139	0.570939154289558\\
0.885	0.780817519965826	0.975683883461272	0.566123121053174\\
0.885	0.795721169365275	0.97891274348356	0.561279136605558\\
0.885	0.810604533136476	0.981900889668376	0.556409397399195\\
0.885	0.825462312667457	0.984650494185936	0.551516079695737\\
0.885	0.840289325888138	0.987163867133889	0.546601337319644\\
0.885	0.855080510976839	0.989443450481768	0.541667299515139\\
0.885	0.86983092974082	0.991491811941914	0.536716068908596\\
0.885	0.884535770672926	0.993311638785082	0.531749719578142\\
0.885	0.899190351687411	0.994905731618374	0.526770295231796\\
0.885	0.913790122539012	0.996276998142666	0.521779807495105\\
0.885	0.928330666930241	0.997428446906011	0.516780234308685\\
0.885	0.942807704312699	0.998363181068902	0.511773518435873\\
0.885	0.95721709138901	0.999084392196567	0.506761566080168\\
0.885	0.971554823322703	0.999595354092743	0.501746245611874\\
0.885	0.985817034663989	0.999899416688637	0.496729386403004\\
0.885	1	1	0.491669087346906\\
0.9	0	0	0.710195569525497\\
0.9	0.000100583311362513	0.0141829653360114	0.71240460418299\\
0.9	0.000404645907256436	0.0284451766772965	0.714630690536905\\
0.9	0.000915607803433	0.0427829086109896	0.716916312064106\\
0.9	0.00163681893109844	0.057192295687301	0.719289019360234\\
0.9	0.00257155309398959	0.0716693330697585	0.721780367061016\\
0.9	0.00372300185733414	0.086209877460988	0.724425655492835\\
0.9	0.00509426838162598	0.100809648312589	0.727263413850304\\
0.9	0.00668836121491816	0.115464229327074	0.73033458997498\\
0.9	0.00850818805808555	0.13016907025918	0.733681422514709\\
0.9	0.0105565495182326	0.144919489023162	0.737345986077549\\
0.9	0.0128361328661109	0.159710674111862	0.741368418705642\\
0.9	0.0153495058140643	0.174537687332543	0.745784862958202\\
0.9	0.0180991103316243	0.189395466863524	0.750625176092311\\
0.9	0.0210872565164405	0.204278830634725	0.755910489868998\\
0.9	0.0243161165387281	0.219182480034174	0.761650724664453\\
0.9	0.0277877186778607	0.234101003941464	0.767842183867616\\
0.9	0.0315039414701067	0.24902888308801	0.77446537092041\\
0.9	0.0354665079868145	0.263960494742775	0.781483180780606\\
0.9	0.0396769802625738	0.278890117720924	0.788839618268073\\
0.9	0.0441367538930258	0.293811937711588	0.796459186325544\\
0.9	0.0488470528220537	0.308720052919643	0.804247066923454\\
0.9	0.0538089243380495	0.323608480015096	0.812090186161949\\
0.9	0.0590232342988274	0.338471160382329	0.819859213932694\\
0.9	0.064490662604533	0.3533019666601	0.827411499063638\\
0.9	0.0702116989375697	0.368094709561873	0.834594885813504\\
0.9	0.0761866387881432	0.382843144964686	0.841252300287178\\
0.9	0.0824155797834956	0.397540981253432	0.847226939729081\\
0.9	0.0888984183382709	0.412181886906127	0.852367847905041\\
0.9	0.0956348466427212	0.42675949830445	0.856535620027829\\
0.9	0.102624350004627	0.441267427752584	0.859607954619617\\
0.9	0.109866204559871	0.455699271686219	0.861484760266622\\
0.9	0.117359475365564	0.470048619052377	0.86209253425154\\
0.9	0.125103014888515	0.484309059839721	0.861387758068367\\
0.9	0.133095461900593	0.498474193737904	0.859359100890821\\
0.9	0.14133524079124	0.512537638903683	0.856028283765734\\
0.9	0.149820561306021	0.526493040810599	0.851449530870183\\
0.9	0.158549418718625	0.540334081158348	0.845707614739503\\
0.9	0.167519594442214	0.554054486817265	0.838914584343152\\
0.9	0.176728657084455	0.567648038782868	0.831205342370839\\
0.9	0.186173963948925	0.581108581114919	0.822732305417527\\
0.9	0.195852662983903	0.594430029835237	0.813659432944309\\
0.9	0.205761695177907	0.607606381758231	0.804155944104869\\
0.9	0.215897797399558	0.620631723228144	0.79439005340697\\
0.9	0.226257505677663	0.633500238737009	0.784523046062799\\
0.9	0.236837158915675	0.646206219397571	0.774703982861589\\
0.9	0.247632903032949	0.658744071245709	0.765065275211016\\
0.9	0.258640695523528	0.671108323347402	0.755719307772964\\
0.9	0.269856310421543	0.683293635685828	0.746756214005914\\
0.9	0.28127534366066	0.695294806804925	0.738242834604295\\
0.9	0.292893218813452	0.707106781186547	0.730222816007836\\
0.9	0.304705193195075	0.71872465633934	0.722717741087767\\
0.9	0.316706364314172	0.730143689578457	0.715729131143265\\
0.9	0.328891676652598	0.741359304476472	0.709241120561858\\
0.9	0.341255928754291	0.752367096967051	0.703223584571265\\
0.9	0.353793780602429	0.763162841084325	0.697635496597417\\
0.9	0.366499761262991	0.773742494322337	0.692428303594619\\
0.9	0.379368276771857	0.784102202600443	0.687549132887069\\
0.9	0.392393618241769	0.794238304822092	0.682943679235328\\
0.9	0.405569970164763	0.804147337016097	0.678558662193046\\
0.9	0.418891418885081	0.813826036051075	0.674343787412872\\
0.9	0.432351961217133	0.823271342915545	0.670253187715552\\
0.9	0.445945513182735	0.832480405557787	0.666246357336158\\
0.9	0.459665918841652	0.841450581281375	0.662288623474185\\
0.9	0.473506959189401	0.850179438693979	0.658351221673191\\
0.9	0.487462361096317	0.85866475920876	0.654411055137887\\
0.9	0.501525806262096	0.866904538099407	0.650450223206467\\
0.9	0.515690940160279	0.874896985111485	0.646455401874959\\
0.9	0.529951380947623	0.882640524634437	0.642417151055155\\
0.9	0.544300728313782	0.890133795440129	0.638329210947585\\
0.9	0.558732572247415	0.897375649995373	0.634187835387498\\
0.9	0.57324050169555	0.904365153357279	0.629991194998376\\
0.9	0.587818113093873	0.911101581661729	0.625738868900924\\
0.9	0.602459018746568	0.917584420216504	0.62143143163314\\
0.9	0.617156855035314	0.923813361211857	0.617070132480554\\
0.9	0.631905290438127	0.92978830106243	0.612656657833838\\
0.9	0.6466980333399	0.935509337395467	0.608192963377087\\
0.9	0.661528839617671	0.940976765701173	0.603681161495105\\
0.9	0.676391519984904	0.946191075661951	0.599123449738607\\
0.9	0.691279947080357	0.951152947177946	0.594522067901592\\
0.9	0.706188062288412	0.955863246106974	0.589879273662389\\
0.9	0.721109882279076	0.960323019737426	0.585197329319934\\
0.9	0.736039505257225	0.964533492013186	0.580478494543978\\
0.9	0.75097111691199	0.968496058529893	0.575725022018313\\
0.9	0.765898996058536	0.972212281322139	0.570939154289558\\
0.9	0.780817519965826	0.975683883461272	0.566123121053178\\
0.9	0.795721169365275	0.97891274348356	0.561279136605558\\
0.9	0.810604533136476	0.981900889668376	0.556409397399196\\
0.9	0.825462312667457	0.984650494185936	0.551516079695734\\
0.9	0.840289325888138	0.987163867133889	0.546601337319642\\
0.9	0.855080510976839	0.989443450481768	0.541667299515139\\
0.9	0.86983092974082	0.991491811941914	0.536716068908596\\
0.9	0.884535770672926	0.993311638785082	0.531749719578142\\
0.9	0.899190351687411	0.994905731618374	0.526770295231796\\
0.9	0.913790122539012	0.996276998142666	0.521779807495102\\
0.9	0.928330666930242	0.997428446906011	0.516780234308682\\
0.9	0.942807704312699	0.998363181068902	0.511773518435872\\
0.9	0.95721709138901	0.999084392196567	0.50676156608017\\
0.9	0.971554823322703	0.999595354092743	0.501746245611874\\
0.9	0.985817034663989	0.999899416688637	0.496729386403004\\
0.9	1	1	0.491669087346906\\
0.915	0	0	0.710195569525497\\
0.915	0.000100583311362513	0.0141829653360114	0.71240460418299\\
0.915	0.000404645907256436	0.0284451766772965	0.714630690536905\\
0.915	0.000915607803433	0.0427829086109896	0.716916312064106\\
0.915	0.00163681893109844	0.057192295687301	0.719289019360234\\
0.915	0.00257155309398959	0.0716693330697585	0.721780367061016\\
0.915	0.00372300185733414	0.086209877460988	0.724425655492835\\
0.915	0.00509426838162598	0.100809648312589	0.727263413850304\\
0.915	0.00668836121491816	0.115464229327074	0.73033458997498\\
0.915	0.00850818805808555	0.13016907025918	0.733681422514709\\
0.915	0.0105565495182326	0.144919489023162	0.737345986077549\\
0.915	0.0128361328661109	0.159710674111862	0.741368418705642\\
0.915	0.0153495058140643	0.174537687332543	0.745784862958201\\
0.915	0.0180991103316243	0.189395466863524	0.750625176092311\\
0.915	0.0210872565164405	0.204278830634725	0.755910489868998\\
0.915	0.0243161165387281	0.219182480034174	0.761650724664453\\
0.915	0.0277877186778607	0.234101003941464	0.767842183867616\\
0.915	0.0315039414701067	0.24902888308801	0.77446537092041\\
0.915	0.0354665079868145	0.263960494742775	0.781483180780606\\
0.915	0.0396769802625738	0.278890117720924	0.788839618268074\\
0.915	0.0441367538930258	0.293811937711588	0.796459186325544\\
0.915	0.0488470528220537	0.308720052919643	0.804247066923455\\
0.915	0.0538089243380495	0.323608480015096	0.812090186161949\\
0.915	0.0590232342988274	0.338471160382329	0.819859213932694\\
0.915	0.064490662604533	0.3533019666601	0.827411499063639\\
0.915	0.0702116989375697	0.368094709561873	0.834594885813505\\
0.915	0.0761866387881432	0.382843144964686	0.84125230028718\\
0.915	0.0824155797834956	0.397540981253432	0.847226939729081\\
0.915	0.0888984183382709	0.412181886906127	0.852367847905041\\
0.915	0.0956348466427213	0.42675949830445	0.856535620027831\\
0.915	0.102624350004627	0.441267427752585	0.859607954619616\\
0.915	0.109866204559871	0.455699271686219	0.86148476026662\\
0.915	0.117359475365564	0.470048619052377	0.86209253425154\\
0.915	0.125103014888515	0.484309059839721	0.861387758068369\\
0.915	0.133095461900593	0.498474193737904	0.859359100890822\\
0.915	0.14133524079124	0.512537638903683	0.856028283765736\\
0.915	0.149820561306021	0.526493040810599	0.851449530870182\\
0.915	0.158549418718625	0.540334081158348	0.845707614739505\\
0.915	0.167519594442214	0.554054486817265	0.838914584343149\\
0.915	0.176728657084455	0.567648038782868	0.831205342370836\\
0.915	0.186173963948925	0.581108581114919	0.822732305417529\\
0.915	0.195852662983903	0.594430029835237	0.813659432944309\\
0.915	0.205761695177907	0.607606381758231	0.804155944104869\\
0.915	0.215897797399558	0.620631723228144	0.794390053406969\\
0.915	0.226257505677663	0.633500238737009	0.784523046062799\\
0.915	0.236837158915675	0.646206219397571	0.774703982861588\\
0.915	0.247632903032949	0.658744071245709	0.765065275211016\\
0.915	0.258640695523528	0.671108323347402	0.755719307772964\\
0.915	0.269856310421543	0.683293635685828	0.746756214005914\\
0.915	0.28127534366066	0.695294806804925	0.738242834604294\\
0.915	0.292893218813452	0.707106781186547	0.730222816007836\\
0.915	0.304705193195075	0.71872465633934	0.722717741087766\\
0.915	0.316706364314172	0.730143689578457	0.715729131143265\\
0.915	0.328891676652598	0.741359304476472	0.709241120561857\\
0.915	0.341255928754291	0.752367096967051	0.703223584571265\\
0.915	0.353793780602429	0.763162841084325	0.697635496597417\\
0.915	0.366499761262991	0.773742494322337	0.692428303594619\\
0.915	0.379368276771857	0.784102202600443	0.687549132887069\\
0.915	0.392393618241769	0.794238304822092	0.682943679235328\\
0.915	0.405569970164763	0.804147337016097	0.678558662193046\\
0.915	0.418891418885081	0.813826036051075	0.674343787412872\\
0.915	0.432351961217133	0.823271342915545	0.670253187715552\\
0.915	0.445945513182735	0.832480405557787	0.666246357336157\\
0.915	0.459665918841652	0.841450581281375	0.662288623474184\\
0.915	0.473506959189401	0.850179438693979	0.658351221673192\\
0.915	0.487462361096317	0.85866475920876	0.654411055137887\\
0.915	0.501525806262096	0.866904538099407	0.650450223206468\\
0.915	0.515690940160279	0.874896985111485	0.646455401874959\\
0.915	0.529951380947623	0.882640524634437	0.642417151055155\\
0.915	0.544300728313782	0.890133795440129	0.638329210947585\\
0.915	0.558732572247415	0.897375649995373	0.634187835387498\\
0.915	0.57324050169555	0.904365153357279	0.629991194998375\\
0.915	0.587818113093873	0.911101581661729	0.625738868900924\\
0.915	0.602459018746568	0.917584420216504	0.62143143163314\\
0.915	0.617156855035314	0.923813361211857	0.617070132480554\\
0.915	0.631905290438127	0.92978830106243	0.612656657833838\\
0.915	0.6466980333399	0.935509337395467	0.608192963377087\\
0.915	0.661528839617671	0.940976765701173	0.603681161495105\\
0.915	0.676391519984904	0.946191075661951	0.599123449738608\\
0.915	0.691279947080357	0.951152947177946	0.594522067901592\\
0.915	0.706188062288412	0.955863246106974	0.58987927366239\\
0.915	0.721109882279076	0.960323019737426	0.585197329319934\\
0.915	0.736039505257225	0.964533492013186	0.580478494543975\\
0.915	0.75097111691199	0.968496058529893	0.575725022018313\\
0.915	0.765898996058536	0.972212281322139	0.57093915428956\\
0.915	0.780817519965826	0.975683883461272	0.566123121053175\\
0.915	0.795721169365275	0.97891274348356	0.561279136605558\\
0.915	0.810604533136476	0.981900889668376	0.556409397399199\\
0.915	0.825462312667457	0.984650494185936	0.551516079695735\\
0.915	0.840289325888138	0.987163867133889	0.546601337319639\\
0.915	0.855080510976839	0.989443450481768	0.541667299515138\\
0.915	0.86983092974082	0.991491811941914	0.536716068908598\\
0.915	0.884535770672926	0.993311638785082	0.531749719578142\\
0.915	0.899190351687411	0.994905731618374	0.526770295231797\\
0.915	0.913790122539012	0.996276998142666	0.521779807495104\\
0.915	0.928330666930241	0.997428446906011	0.516780234308685\\
0.915	0.942807704312699	0.998363181068902	0.51177351843587\\
0.915	0.95721709138901	0.999084392196567	0.506761566080168\\
0.915	0.971554823322704	0.999595354092744	0.501746245611876\\
0.915	0.985817034663989	0.999899416688637	0.496729386403002\\
0.915	1	1	0.491669087346908\\
0.93	0	0	0.710195569525497\\
0.93	0.000100583311362513	0.0141829653360114	0.71240460418299\\
0.93	0.000404645907256436	0.0284451766772965	0.714630690536905\\
0.93	0.000915607803433	0.0427829086109896	0.716916312064106\\
0.93	0.00163681893109844	0.057192295687301	0.719289019360234\\
0.93	0.00257155309398959	0.0716693330697585	0.721780367061016\\
0.93	0.00372300185733414	0.086209877460988	0.724425655492835\\
0.93	0.00509426838162598	0.100809648312589	0.727263413850304\\
0.93	0.00668836121491816	0.115464229327074	0.73033458997498\\
0.93	0.00850818805808555	0.13016907025918	0.733681422514709\\
0.93	0.0105565495182326	0.144919489023162	0.737345986077549\\
0.93	0.0128361328661109	0.159710674111862	0.741368418705642\\
0.93	0.0153495058140643	0.174537687332543	0.745784862958201\\
0.93	0.0180991103316243	0.189395466863524	0.750625176092311\\
0.93	0.0210872565164405	0.204278830634725	0.755910489868998\\
0.93	0.0243161165387281	0.219182480034174	0.761650724664453\\
0.93	0.0277877186778607	0.234101003941464	0.767842183867616\\
0.93	0.0315039414701067	0.24902888308801	0.77446537092041\\
0.93	0.0354665079868145	0.263960494742775	0.781483180780606\\
0.93	0.0396769802625738	0.278890117720924	0.788839618268073\\
0.93	0.0441367538930258	0.293811937711588	0.796459186325544\\
0.93	0.0488470528220538	0.308720052919643	0.804247066923455\\
0.93	0.0538089243380495	0.323608480015096	0.81209018616195\\
0.93	0.0590232342988274	0.338471160382329	0.819859213932694\\
0.93	0.064490662604533	0.3533019666601	0.82741149906364\\
0.93	0.0702116989375697	0.368094709561873	0.834594885813504\\
0.93	0.0761866387881433	0.382843144964686	0.841252300287181\\
0.93	0.0824155797834956	0.397540981253432	0.847226939729081\\
0.93	0.0888984183382709	0.412181886906127	0.852367847905041\\
0.93	0.0956348466427212	0.42675949830445	0.856535620027829\\
0.93	0.102624350004627	0.441267427752585	0.859607954619618\\
0.93	0.109866204559871	0.455699271686218	0.861484760266622\\
0.93	0.117359475365564	0.470048619052377	0.862092534251541\\
0.93	0.125103014888515	0.484309059839721	0.861387758068373\\
0.93	0.133095461900593	0.498474193737904	0.859359100890821\\
0.93	0.14133524079124	0.512537638903683	0.856028283765736\\
0.93	0.149820561306021	0.526493040810599	0.851449530870182\\
0.93	0.158549418718625	0.540334081158348	0.845707614739509\\
0.93	0.167519594442214	0.554054486817265	0.838914584343148\\
0.93	0.176728657084455	0.567648038782868	0.831205342370837\\
0.93	0.186173963948925	0.581108581114919	0.822732305417528\\
0.93	0.195852662983903	0.594430029835237	0.813659432944308\\
0.93	0.205761695177907	0.607606381758231	0.804155944104869\\
0.93	0.215897797399558	0.620631723228144	0.794390053406969\\
0.93	0.226257505677663	0.633500238737009	0.784523046062799\\
0.93	0.236837158915675	0.646206219397571	0.774703982861589\\
0.93	0.247632903032949	0.658744071245709	0.765065275211016\\
0.93	0.258640695523528	0.671108323347402	0.755719307772964\\
0.93	0.269856310421543	0.683293635685828	0.746756214005914\\
0.93	0.28127534366066	0.695294806804925	0.738242834604294\\
0.93	0.292893218813452	0.707106781186547	0.730222816007835\\
0.93	0.304705193195076	0.71872465633934	0.722717741087767\\
0.93	0.316706364314172	0.730143689578457	0.715729131143266\\
0.93	0.328891676652598	0.741359304476472	0.709241120561857\\
0.93	0.341255928754291	0.752367096967051	0.703223584571265\\
0.93	0.353793780602429	0.763162841084325	0.697635496597417\\
0.93	0.366499761262991	0.773742494322337	0.692428303594619\\
0.93	0.379368276771857	0.784102202600443	0.687549132887069\\
0.93	0.392393618241769	0.794238304822092	0.682943679235328\\
0.93	0.405569970164763	0.804147337016097	0.678558662193046\\
0.93	0.418891418885081	0.813826036051075	0.674343787412872\\
0.93	0.432351961217133	0.823271342915545	0.670253187715552\\
0.93	0.445945513182735	0.832480405557787	0.666246357336158\\
0.93	0.459665918841652	0.841450581281375	0.662288623474184\\
0.93	0.473506959189401	0.850179438693979	0.65835122167319\\
0.93	0.487462361096317	0.85866475920876	0.654411055137886\\
0.93	0.501525806262096	0.866904538099407	0.650450223206468\\
0.93	0.515690940160279	0.874896985111485	0.646455401874959\\
0.93	0.529951380947623	0.882640524634437	0.642417151055155\\
0.93	0.544300728313782	0.890133795440129	0.638329210947585\\
0.93	0.558732572247415	0.897375649995373	0.634187835387499\\
0.93	0.57324050169555	0.904365153357279	0.629991194998375\\
0.93	0.587818113093873	0.911101581661729	0.625738868900923\\
0.93	0.602459018746568	0.917584420216504	0.62143143163314\\
0.93	0.617156855035314	0.923813361211857	0.617070132480554\\
0.93	0.631905290438127	0.92978830106243	0.612656657833838\\
0.93	0.6466980333399	0.935509337395467	0.608192963377087\\
0.93	0.661528839617671	0.940976765701173	0.603681161495104\\
0.93	0.676391519984904	0.94619107566195	0.599123449738607\\
0.93	0.691279947080357	0.951152947177946	0.59452206790159\\
0.93	0.706188062288412	0.955863246106974	0.58987927366239\\
0.93	0.721109882279076	0.960323019737426	0.585197329319937\\
0.93	0.736039505257225	0.964533492013186	0.580478494543978\\
0.93	0.75097111691199	0.968496058529893	0.575725022018313\\
0.93	0.765898996058536	0.972212281322139	0.570939154289558\\
0.93	0.780817519965826	0.975683883461272	0.566123121053176\\
0.93	0.795721169365275	0.97891274348356	0.561279136605558\\
0.93	0.810604533136476	0.981900889668376	0.556409397399198\\
0.93	0.825462312667457	0.984650494185936	0.551516079695736\\
0.93	0.840289325888138	0.987163867133889	0.546601337319641\\
0.93	0.855080510976839	0.989443450481768	0.541667299515139\\
0.93	0.86983092974082	0.991491811941914	0.536716068908598\\
0.93	0.884535770672926	0.993311638785082	0.53174971957814\\
0.93	0.899190351687411	0.994905731618374	0.526770295231796\\
0.93	0.913790122539012	0.996276998142666	0.521779807495104\\
0.93	0.928330666930242	0.997428446906011	0.516780234308682\\
0.93	0.942807704312699	0.998363181068902	0.511773518435871\\
0.93	0.95721709138901	0.999084392196567	0.506761566080168\\
0.93	0.971554823322703	0.999595354092743	0.501746245611877\\
0.93	0.985817034663989	0.999899416688637	0.496729386403004\\
0.93	1	1	0.491669087346906\\
0.945	0	0	0.710195569525497\\
0.945	0.000100583311362513	0.0141829653360114	0.71240460418299\\
0.945	0.000404645907256436	0.0284451766772965	0.714630690536905\\
0.945	0.000915607803433	0.0427829086109896	0.716916312064106\\
0.945	0.00163681893109844	0.057192295687301	0.719289019360234\\
0.945	0.00257155309398959	0.0716693330697585	0.721780367061016\\
0.945	0.00372300185733414	0.086209877460988	0.724425655492835\\
0.945	0.00509426838162598	0.100809648312589	0.727263413850304\\
0.945	0.00668836121491816	0.115464229327074	0.73033458997498\\
0.945	0.00850818805808555	0.13016907025918	0.733681422514709\\
0.945	0.0105565495182326	0.144919489023162	0.737345986077549\\
0.945	0.0128361328661109	0.159710674111862	0.741368418705642\\
0.945	0.0153495058140643	0.174537687332543	0.745784862958202\\
0.945	0.0180991103316243	0.189395466863524	0.750625176092311\\
0.945	0.0210872565164405	0.204278830634725	0.755910489868998\\
0.945	0.0243161165387281	0.219182480034174	0.761650724664453\\
0.945	0.0277877186778607	0.234101003941464	0.767842183867616\\
0.945	0.0315039414701067	0.24902888308801	0.77446537092041\\
0.945	0.0354665079868145	0.263960494742775	0.781483180780606\\
0.945	0.0396769802625738	0.278890117720924	0.788839618268073\\
0.945	0.0441367538930258	0.293811937711588	0.796459186325544\\
0.945	0.0488470528220538	0.308720052919643	0.804247066923455\\
0.945	0.0538089243380495	0.323608480015097	0.812090186161949\\
0.945	0.0590232342988274	0.338471160382329	0.819859213932693\\
0.945	0.064490662604533	0.3533019666601	0.827411499063639\\
0.945	0.0702116989375697	0.368094709561873	0.834594885813504\\
0.945	0.0761866387881433	0.382843144964686	0.841252300287178\\
0.945	0.0824155797834956	0.397540981253432	0.847226939729081\\
0.945	0.0888984183382709	0.412181886906127	0.852367847905041\\
0.945	0.0956348466427212	0.42675949830445	0.856535620027829\\
0.945	0.102624350004627	0.441267427752585	0.859607954619617\\
0.945	0.109866204559871	0.455699271686219	0.86148476026662\\
0.945	0.117359475365564	0.470048619052377	0.86209253425154\\
0.945	0.125103014888515	0.484309059839721	0.861387758068368\\
0.945	0.133095461900593	0.498474193737904	0.859359100890819\\
0.945	0.14133524079124	0.512537638903683	0.856028283765734\\
0.945	0.149820561306021	0.526493040810599	0.851449530870183\\
0.945	0.158549418718625	0.540334081158348	0.845707614739502\\
0.945	0.167519594442214	0.554054486817265	0.83891458434315\\
0.945	0.176728657084455	0.567648038782868	0.831205342370836\\
0.945	0.186173963948925	0.581108581114919	0.822732305417526\\
0.945	0.195852662983903	0.594430029835237	0.813659432944309\\
0.945	0.205761695177907	0.607606381758231	0.804155944104869\\
0.945	0.215897797399558	0.620631723228144	0.794390053406969\\
0.945	0.226257505677663	0.633500238737009	0.784523046062799\\
0.945	0.236837158915675	0.646206219397571	0.774703982861589\\
0.945	0.247632903032949	0.658744071245709	0.765065275211016\\
0.945	0.258640695523528	0.671108323347402	0.755719307772965\\
0.945	0.269856310421543	0.683293635685828	0.746756214005914\\
0.945	0.28127534366066	0.695294806804925	0.738242834604294\\
0.945	0.292893218813452	0.707106781186547	0.730222816007836\\
0.945	0.304705193195075	0.71872465633934	0.722717741087767\\
0.945	0.316706364314172	0.730143689578457	0.715729131143265\\
0.945	0.328891676652598	0.741359304476472	0.709241120561857\\
0.945	0.341255928754291	0.752367096967051	0.703223584571265\\
0.945	0.353793780602429	0.763162841084325	0.697635496597417\\
0.945	0.366499761262991	0.773742494322337	0.692428303594619\\
0.945	0.379368276771857	0.784102202600443	0.687549132887069\\
0.945	0.392393618241769	0.794238304822092	0.682943679235328\\
0.945	0.405569970164763	0.804147337016097	0.678558662193046\\
0.945	0.418891418885081	0.813826036051075	0.674343787412872\\
0.945	0.432351961217133	0.823271342915545	0.670253187715552\\
0.945	0.445945513182735	0.832480405557787	0.666246357336158\\
0.945	0.459665918841652	0.841450581281375	0.662288623474185\\
0.945	0.473506959189401	0.850179438693979	0.658351221673191\\
0.945	0.487462361096317	0.85866475920876	0.654411055137886\\
0.945	0.501525806262096	0.866904538099407	0.650450223206468\\
0.945	0.515690940160279	0.874896985111485	0.646455401874959\\
0.945	0.529951380947623	0.882640524634437	0.642417151055155\\
0.945	0.544300728313782	0.890133795440129	0.638329210947586\\
0.945	0.558732572247415	0.897375649995373	0.634187835387499\\
0.945	0.57324050169555	0.904365153357279	0.629991194998376\\
0.945	0.587818113093873	0.911101581661729	0.625738868900924\\
0.945	0.602459018746568	0.917584420216504	0.62143143163314\\
0.945	0.617156855035314	0.923813361211857	0.617070132480554\\
0.945	0.631905290438127	0.92978830106243	0.612656657833838\\
0.945	0.6466980333399	0.935509337395467	0.608192963377087\\
0.945	0.661528839617671	0.940976765701173	0.603681161495105\\
0.945	0.676391519984904	0.946191075661951	0.599123449738606\\
0.945	0.691279947080357	0.951152947177946	0.594522067901589\\
0.945	0.706188062288412	0.955863246106974	0.589879273662388\\
0.945	0.721109882279076	0.960323019737426	0.585197329319937\\
0.945	0.736039505257225	0.964533492013186	0.580478494543978\\
0.945	0.75097111691199	0.968496058529893	0.575725022018314\\
0.945	0.765898996058536	0.972212281322139	0.570939154289557\\
0.945	0.780817519965826	0.975683883461272	0.566123121053175\\
0.945	0.795721169365275	0.97891274348356	0.561279136605558\\
0.945	0.810604533136476	0.981900889668376	0.556409397399199\\
0.945	0.825462312667457	0.984650494185936	0.551516079695737\\
0.945	0.840289325888138	0.987163867133889	0.546601337319639\\
0.945	0.855080510976839	0.989443450481768	0.541667299515138\\
0.945	0.86983092974082	0.991491811941914	0.536716068908599\\
0.945	0.884535770672926	0.993311638785082	0.531749719578143\\
0.945	0.899190351687411	0.994905731618374	0.526770295231796\\
0.945	0.913790122539012	0.996276998142666	0.521779807495105\\
0.945	0.928330666930241	0.997428446906011	0.51678023430868\\
0.945	0.942807704312699	0.998363181068902	0.51177351843587\\
0.945	0.95721709138901	0.999084392196567	0.506761566080168\\
0.945	0.971554823322703	0.999595354092743	0.501746245611876\\
0.945	0.985817034663989	0.999899416688637	0.496729386403005\\
0.945	1	1	0.491669087346904\\
0.96	0	0	0.710195569525497\\
0.96	0.000100583311362513	0.0141829653360114	0.71240460418299\\
0.96	0.000404645907256436	0.0284451766772965	0.714630690536905\\
0.96	0.000915607803433	0.0427829086109896	0.716916312064106\\
0.96	0.00163681893109844	0.057192295687301	0.719289019360234\\
0.96	0.00257155309398959	0.0716693330697585	0.721780367061016\\
0.96	0.00372300185733414	0.0862098774609879	0.724425655492835\\
0.96	0.00509426838162598	0.100809648312589	0.727263413850304\\
0.96	0.00668836121491816	0.115464229327074	0.73033458997498\\
0.96	0.00850818805808555	0.13016907025918	0.733681422514709\\
0.96	0.0105565495182326	0.144919489023162	0.737345986077549\\
0.96	0.0128361328661109	0.159710674111862	0.741368418705642\\
0.96	0.0153495058140643	0.174537687332543	0.745784862958201\\
0.96	0.0180991103316243	0.189395466863524	0.750625176092311\\
0.96	0.0210872565164405	0.204278830634725	0.755910489868999\\
0.96	0.0243161165387281	0.219182480034174	0.761650724664453\\
0.96	0.0277877186778607	0.234101003941464	0.767842183867616\\
0.96	0.0315039414701067	0.24902888308801	0.77446537092041\\
0.96	0.0354665079868145	0.263960494742775	0.781483180780606\\
0.96	0.0396769802625738	0.278890117720924	0.788839618268073\\
0.96	0.0441367538930258	0.293811937711588	0.796459186325544\\
0.96	0.0488470528220538	0.308720052919643	0.804247066923455\\
0.96	0.0538089243380495	0.323608480015096	0.81209018616195\\
0.96	0.0590232342988274	0.338471160382329	0.819859213932694\\
0.96	0.064490662604533	0.3533019666601	0.827411499063639\\
0.96	0.0702116989375697	0.368094709561873	0.834594885813505\\
0.96	0.0761866387881432	0.382843144964686	0.841252300287178\\
0.96	0.0824155797834956	0.397540981253432	0.847226939729081\\
0.96	0.0888984183382709	0.412181886906127	0.852367847905041\\
0.96	0.0956348466427213	0.42675949830445	0.856535620027831\\
0.96	0.102624350004627	0.441267427752585	0.859607954619618\\
0.96	0.109866204559871	0.455699271686218	0.861484760266618\\
0.96	0.117359475365564	0.470048619052377	0.862092534251541\\
0.96	0.125103014888515	0.484309059839721	0.861387758068366\\
0.96	0.133095461900593	0.498474193737904	0.85935910089082\\
0.96	0.14133524079124	0.512537638903683	0.856028283765737\\
0.96	0.149820561306021	0.526493040810599	0.851449530870182\\
0.96	0.158549418718625	0.540334081158348	0.845707614739505\\
0.96	0.167519594442214	0.554054486817265	0.838914584343152\\
0.96	0.176728657084455	0.567648038782868	0.831205342370836\\
0.96	0.186173963948925	0.581108581114919	0.822732305417528\\
0.96	0.195852662983903	0.594430029835237	0.813659432944309\\
0.96	0.205761695177907	0.607606381758231	0.804155944104869\\
0.96	0.215897797399558	0.620631723228144	0.794390053406969\\
0.96	0.226257505677663	0.633500238737009	0.784523046062799\\
0.96	0.236837158915675	0.646206219397571	0.774703982861589\\
0.96	0.247632903032949	0.658744071245709	0.765065275211016\\
0.96	0.258640695523528	0.671108323347402	0.755719307772964\\
0.96	0.269856310421543	0.683293635685828	0.746756214005913\\
0.96	0.28127534366066	0.695294806804925	0.738242834604294\\
0.96	0.292893218813452	0.707106781186547	0.730222816007836\\
0.96	0.304705193195075	0.71872465633934	0.722717741087766\\
0.96	0.316706364314172	0.730143689578457	0.715729131143265\\
0.96	0.328891676652598	0.741359304476472	0.709241120561857\\
0.96	0.341255928754291	0.752367096967051	0.703223584571265\\
0.96	0.353793780602429	0.763162841084325	0.697635496597417\\
0.96	0.366499761262991	0.773742494322337	0.692428303594619\\
0.96	0.379368276771857	0.784102202600443	0.687549132887069\\
0.96	0.392393618241769	0.794238304822092	0.682943679235329\\
0.96	0.405569970164763	0.804147337016097	0.678558662193046\\
0.96	0.418891418885081	0.813826036051075	0.674343787412872\\
0.96	0.432351961217133	0.823271342915545	0.670253187715552\\
0.96	0.445945513182735	0.832480405557787	0.666246357336157\\
0.96	0.459665918841652	0.841450581281375	0.662288623474184\\
0.96	0.473506959189401	0.850179438693979	0.658351221673192\\
0.96	0.487462361096317	0.85866475920876	0.654411055137887\\
0.96	0.501525806262096	0.866904538099407	0.650450223206468\\
0.96	0.515690940160279	0.874896985111485	0.646455401874959\\
0.96	0.529951380947623	0.882640524634437	0.642417151055155\\
0.96	0.544300728313782	0.890133795440129	0.638329210947585\\
0.96	0.558732572247415	0.897375649995373	0.634187835387498\\
0.96	0.57324050169555	0.904365153357279	0.629991194998376\\
0.96	0.587818113093873	0.911101581661729	0.625738868900924\\
0.96	0.602459018746568	0.917584420216504	0.62143143163314\\
0.96	0.617156855035314	0.923813361211857	0.617070132480554\\
0.96	0.631905290438127	0.92978830106243	0.612656657833838\\
0.96	0.6466980333399	0.935509337395467	0.608192963377087\\
0.96	0.661528839617671	0.940976765701173	0.603681161495105\\
0.96	0.676391519984904	0.946191075661951	0.599123449738608\\
0.96	0.691279947080357	0.951152947177946	0.59452206790159\\
0.96	0.706188062288412	0.955863246106974	0.589879273662388\\
0.96	0.721109882279076	0.960323019737426	0.585197329319936\\
0.96	0.736039505257225	0.964533492013186	0.580478494543975\\
0.96	0.75097111691199	0.968496058529893	0.575725022018315\\
0.96	0.765898996058536	0.972212281322139	0.570939154289561\\
0.96	0.780817519965826	0.975683883461272	0.566123121053175\\
0.96	0.795721169365275	0.97891274348356	0.561279136605558\\
0.96	0.810604533136476	0.981900889668376	0.556409397399196\\
0.96	0.825462312667457	0.984650494185936	0.551516079695736\\
0.96	0.840289325888138	0.987163867133889	0.546601337319642\\
0.96	0.855080510976839	0.989443450481768	0.541667299515139\\
0.96	0.86983092974082	0.991491811941914	0.536716068908594\\
0.96	0.884535770672926	0.993311638785082	0.531749719578141\\
0.96	0.899190351687411	0.994905731618374	0.526770295231796\\
0.96	0.913790122539012	0.996276998142666	0.521779807495105\\
0.96	0.928330666930242	0.997428446906011	0.516780234308687\\
0.96	0.942807704312699	0.998363181068902	0.51177351843587\\
0.96	0.95721709138901	0.999084392196567	0.506761566080167\\
0.96	0.971554823322704	0.999595354092744	0.501746245611876\\
0.96	0.985817034663989	0.999899416688637	0.496729386403002\\
0.96	1	1	0.491669087346908\\
0.975	0	0	0.710195569525497\\
0.975	0.000100583311362513	0.0141829653360114	0.71240460418299\\
0.975	0.000404645907256436	0.0284451766772965	0.714630690536905\\
0.975	0.000915607803433	0.0427829086109896	0.716916312064106\\
0.975	0.00163681893109844	0.057192295687301	0.719289019360234\\
0.975	0.00257155309398959	0.0716693330697585	0.721780367061016\\
0.975	0.00372300185733414	0.086209877460988	0.724425655492835\\
0.975	0.00509426838162598	0.100809648312589	0.727263413850304\\
0.975	0.00668836121491816	0.115464229327074	0.73033458997498\\
0.975	0.00850818805808555	0.13016907025918	0.733681422514709\\
0.975	0.0105565495182326	0.144919489023162	0.737345986077549\\
0.975	0.0128361328661109	0.159710674111862	0.741368418705642\\
0.975	0.0153495058140643	0.174537687332543	0.745784862958201\\
0.975	0.0180991103316243	0.189395466863524	0.750625176092311\\
0.975	0.0210872565164405	0.204278830634725	0.755910489868998\\
0.975	0.0243161165387281	0.219182480034174	0.761650724664453\\
0.975	0.0277877186778607	0.234101003941464	0.767842183867616\\
0.975	0.0315039414701067	0.24902888308801	0.77446537092041\\
0.975	0.0354665079868145	0.263960494742775	0.781483180780606\\
0.975	0.0396769802625738	0.278890117720924	0.788839618268073\\
0.975	0.0441367538930258	0.293811937711588	0.796459186325544\\
0.975	0.0488470528220538	0.308720052919643	0.804247066923454\\
0.975	0.0538089243380495	0.323608480015096	0.81209018616195\\
0.975	0.0590232342988274	0.338471160382329	0.819859213932693\\
0.975	0.064490662604533	0.3533019666601	0.82741149906364\\
0.975	0.0702116989375697	0.368094709561873	0.834594885813505\\
0.975	0.0761866387881432	0.382843144964686	0.841252300287178\\
0.975	0.0824155797834956	0.397540981253432	0.847226939729081\\
0.975	0.0888984183382709	0.412181886906127	0.852367847905041\\
0.975	0.0956348466427212	0.42675949830445	0.85653562002783\\
0.975	0.102624350004627	0.441267427752584	0.859607954619618\\
0.975	0.109866204559871	0.455699271686218	0.861484760266623\\
0.975	0.117359475365564	0.470048619052377	0.862092534251542\\
0.975	0.125103014888515	0.484309059839721	0.861387758068371\\
0.975	0.133095461900593	0.498474193737904	0.85935910089082\\
0.975	0.14133524079124	0.512537638903683	0.856028283765739\\
0.975	0.149820561306021	0.526493040810599	0.851449530870183\\
0.975	0.158549418718625	0.540334081158348	0.845707614739508\\
0.975	0.167519594442214	0.554054486817265	0.83891458434315\\
0.975	0.176728657084455	0.567648038782868	0.831205342370834\\
0.975	0.186173963948925	0.581108581114919	0.822732305417528\\
0.975	0.195852662983903	0.594430029835237	0.813659432944309\\
0.975	0.205761695177907	0.607606381758231	0.804155944104869\\
0.975	0.215897797399558	0.620631723228144	0.794390053406969\\
0.975	0.226257505677663	0.633500238737009	0.784523046062799\\
0.975	0.236837158915675	0.646206219397571	0.774703982861589\\
0.975	0.247632903032949	0.658744071245709	0.765065275211016\\
0.975	0.258640695523528	0.671108323347402	0.755719307772964\\
0.975	0.269856310421543	0.683293635685828	0.746756214005914\\
0.975	0.28127534366066	0.695294806804925	0.738242834604294\\
0.975	0.292893218813452	0.707106781186547	0.730222816007835\\
0.975	0.304705193195076	0.71872465633934	0.722717741087766\\
0.975	0.316706364314172	0.730143689578457	0.715729131143266\\
0.975	0.328891676652598	0.741359304476472	0.709241120561858\\
0.975	0.341255928754291	0.752367096967051	0.703223584571265\\
0.975	0.353793780602429	0.763162841084325	0.697635496597417\\
0.975	0.366499761262991	0.773742494322337	0.692428303594619\\
0.975	0.379368276771857	0.784102202600443	0.687549132887069\\
0.975	0.392393618241769	0.794238304822092	0.682943679235328\\
0.975	0.405569970164763	0.804147337016097	0.678558662193046\\
0.975	0.418891418885081	0.813826036051075	0.674343787412872\\
0.975	0.432351961217133	0.823271342915545	0.670253187715552\\
0.975	0.445945513182735	0.832480405557787	0.666246357336158\\
0.975	0.459665918841652	0.841450581281375	0.662288623474184\\
0.975	0.473506959189401	0.850179438693979	0.65835122167319\\
0.975	0.487462361096317	0.85866475920876	0.654411055137886\\
0.975	0.501525806262096	0.866904538099407	0.650450223206468\\
0.975	0.515690940160279	0.874896985111485	0.646455401874959\\
0.975	0.529951380947623	0.882640524634437	0.642417151055155\\
0.975	0.544300728313782	0.890133795440129	0.638329210947585\\
0.975	0.558732572247415	0.897375649995373	0.634187835387499\\
0.975	0.57324050169555	0.904365153357279	0.629991194998375\\
0.975	0.587818113093873	0.911101581661729	0.625738868900923\\
0.975	0.602459018746568	0.917584420216504	0.62143143163314\\
0.975	0.617156855035314	0.923813361211857	0.617070132480554\\
0.975	0.631905290438127	0.92978830106243	0.612656657833838\\
0.975	0.6466980333399	0.935509337395467	0.608192963377087\\
0.975	0.661528839617671	0.940976765701173	0.603681161495104\\
0.975	0.676391519984904	0.94619107566195	0.599123449738607\\
0.975	0.691279947080357	0.951152947177946	0.594522067901591\\
0.975	0.706188062288412	0.955863246106974	0.589879273662389\\
0.975	0.721109882279076	0.960323019737426	0.585197329319936\\
0.975	0.736039505257225	0.964533492013186	0.580478494543978\\
0.975	0.75097111691199	0.968496058529893	0.575725022018311\\
0.975	0.765898996058536	0.972212281322139	0.570939154289558\\
0.975	0.780817519965826	0.975683883461272	0.566123121053177\\
0.975	0.795721169365275	0.97891274348356	0.56127913660556\\
0.975	0.810604533136476	0.981900889668376	0.556409397399199\\
0.975	0.825462312667457	0.984650494185936	0.551516079695735\\
0.975	0.840289325888138	0.987163867133889	0.546601337319639\\
0.975	0.855080510976839	0.989443450481768	0.541667299515139\\
0.975	0.86983092974082	0.991491811941914	0.536716068908597\\
0.975	0.884535770672926	0.993311638785082	0.531749719578142\\
0.975	0.899190351687411	0.994905731618374	0.526770295231796\\
0.975	0.913790122539012	0.996276998142666	0.521779807495102\\
0.975	0.928330666930241	0.997428446906011	0.516780234308684\\
0.975	0.942807704312699	0.998363181068902	0.511773518435872\\
0.975	0.95721709138901	0.999084392196567	0.506761566080167\\
0.975	0.971554823322703	0.999595354092743	0.501746245611875\\
0.975	0.985817034663989	0.999899416688637	0.496729386403004\\
0.975	1	1	0.491669087346906\\
0.99	0	0	0.710195569525497\\
0.99	0.000100583311362513	0.0141829653360114	0.71240460418299\\
0.99	0.000404645907256436	0.0284451766772965	0.714630690536905\\
0.99	0.000915607803433	0.0427829086109896	0.716916312064106\\
0.99	0.00163681893109844	0.057192295687301	0.719289019360234\\
0.99	0.00257155309398959	0.0716693330697585	0.721780367061016\\
0.99	0.00372300185733414	0.086209877460988	0.724425655492835\\
0.99	0.00509426838162598	0.100809648312589	0.727263413850304\\
0.99	0.00668836121491816	0.115464229327074	0.73033458997498\\
0.99	0.00850818805808555	0.13016907025918	0.733681422514709\\
0.99	0.0105565495182326	0.144919489023162	0.737345986077549\\
0.99	0.0128361328661109	0.159710674111862	0.741368418705642\\
0.99	0.0153495058140643	0.174537687332543	0.745784862958202\\
0.99	0.0180991103316243	0.189395466863524	0.750625176092311\\
0.99	0.0210872565164405	0.204278830634725	0.755910489868998\\
0.99	0.0243161165387281	0.219182480034174	0.761650724664453\\
0.99	0.0277877186778607	0.234101003941464	0.767842183867616\\
0.99	0.0315039414701067	0.24902888308801	0.77446537092041\\
0.99	0.0354665079868145	0.263960494742775	0.781483180780606\\
0.99	0.0396769802625737	0.278890117720924	0.788839618268073\\
0.99	0.0441367538930258	0.293811937711588	0.796459186325544\\
0.99	0.0488470528220538	0.308720052919643	0.804247066923454\\
0.99	0.0538089243380495	0.323608480015097	0.81209018616195\\
0.99	0.0590232342988274	0.338471160382329	0.819859213932693\\
0.99	0.064490662604533	0.3533019666601	0.82741149906364\\
0.99	0.0702116989375697	0.368094709561873	0.834594885813505\\
0.99	0.0761866387881432	0.382843144964686	0.841252300287178\\
0.99	0.0824155797834956	0.397540981253432	0.847226939729081\\
0.99	0.0888984183382709	0.412181886906127	0.852367847905041\\
0.99	0.0956348466427212	0.42675949830445	0.856535620027828\\
0.99	0.102624350004627	0.441267427752584	0.859607954619617\\
0.99	0.109866204559871	0.455699271686219	0.861484760266624\\
0.99	0.117359475365564	0.470048619052377	0.862092534251541\\
0.99	0.125103014888515	0.484309059839721	0.86138775806837\\
0.99	0.133095461900593	0.498474193737904	0.859359100890819\\
0.99	0.14133524079124	0.512537638903683	0.856028283765735\\
0.99	0.149820561306021	0.526493040810599	0.851449530870182\\
0.99	0.158549418718625	0.540334081158348	0.845707614739502\\
0.99	0.167519594442213	0.554054486817265	0.83891458434315\\
0.99	0.176728657084455	0.567648038782867	0.831205342370837\\
0.99	0.186173963948925	0.581108581114919	0.822732305417529\\
0.99	0.195852662983903	0.594430029835237	0.813659432944309\\
0.99	0.205761695177907	0.607606381758231	0.804155944104869\\
0.99	0.215897797399558	0.620631723228144	0.794390053406969\\
0.99	0.226257505677663	0.633500238737009	0.784523046062799\\
0.99	0.236837158915675	0.646206219397571	0.774703982861589\\
0.99	0.247632903032949	0.658744071245709	0.765065275211016\\
0.99	0.258640695523528	0.671108323347402	0.755719307772964\\
0.99	0.269856310421543	0.683293635685828	0.746756214005913\\
0.99	0.28127534366066	0.695294806804924	0.738242834604294\\
0.99	0.292893218813452	0.707106781186547	0.730222816007835\\
0.99	0.304705193195075	0.71872465633934	0.722717741087767\\
0.99	0.316706364314172	0.730143689578457	0.715729131143267\\
0.99	0.328891676652598	0.741359304476472	0.709241120561857\\
0.99	0.341255928754291	0.752367096967051	0.703223584571265\\
0.99	0.353793780602429	0.763162841084325	0.697635496597417\\
0.99	0.366499761262991	0.773742494322337	0.692428303594619\\
0.99	0.379368276771857	0.784102202600443	0.687549132887069\\
0.99	0.392393618241769	0.794238304822092	0.682943679235328\\
0.99	0.405569970164763	0.804147337016097	0.678558662193046\\
0.99	0.418891418885081	0.813826036051075	0.674343787412872\\
0.99	0.432351961217133	0.823271342915545	0.670253187715551\\
0.99	0.445945513182735	0.832480405557787	0.666246357336158\\
0.99	0.459665918841652	0.841450581281375	0.662288623474186\\
0.99	0.473506959189401	0.850179438693979	0.658351221673192\\
0.99	0.487462361096317	0.85866475920876	0.654411055137886\\
0.99	0.501525806262096	0.866904538099407	0.650450223206467\\
0.99	0.515690940160279	0.874896985111485	0.646455401874959\\
0.99	0.529951380947623	0.882640524634437	0.642417151055155\\
0.99	0.544300728313782	0.890133795440129	0.638329210947585\\
0.99	0.558732572247415	0.897375649995373	0.634187835387498\\
0.99	0.57324050169555	0.904365153357279	0.629991194998376\\
0.99	0.587818113093873	0.911101581661729	0.625738868900924\\
0.99	0.602459018746568	0.917584420216504	0.62143143163314\\
0.99	0.617156855035314	0.923813361211857	0.617070132480554\\
0.99	0.631905290438127	0.92978830106243	0.612656657833838\\
0.99	0.6466980333399	0.935509337395467	0.608192963377087\\
0.99	0.661528839617671	0.940976765701173	0.603681161495105\\
0.99	0.676391519984904	0.946191075661951	0.599123449738608\\
0.99	0.691279947080357	0.951152947177946	0.59452206790159\\
0.99	0.706188062288412	0.955863246106974	0.589879273662388\\
0.99	0.721109882279076	0.960323019737426	0.585197329319936\\
0.99	0.736039505257225	0.964533492013186	0.580478494543978\\
0.99	0.75097111691199	0.968496058529893	0.575725022018314\\
0.99	0.765898996058536	0.972212281322139	0.570939154289557\\
0.99	0.780817519965826	0.975683883461272	0.566123121053176\\
0.99	0.795721169365275	0.97891274348356	0.56127913660556\\
0.99	0.810604533136476	0.981900889668376	0.556409397399193\\
0.99	0.825462312667457	0.984650494185936	0.551516079695735\\
0.99	0.840289325888138	0.987163867133889	0.546601337319645\\
0.99	0.855080510976839	0.989443450481768	0.541667299515141\\
0.99	0.86983092974082	0.991491811941915	0.536716068908595\\
0.99	0.884535770672926	0.993311638785082	0.531749719578139\\
0.99	0.899190351687411	0.994905731618374	0.526770295231797\\
0.99	0.913790122539012	0.996276998142666	0.521779807495104\\
0.99	0.928330666930242	0.997428446906011	0.516780234308684\\
0.99	0.942807704312699	0.998363181068902	0.511773518435872\\
0.99	0.95721709138901	0.999084392196567	0.506761566080166\\
0.99	0.971554823322703	0.999595354092743	0.501746245611874\\
0.99	0.985817034663989	0.999899416688637	0.496729386403005\\
0.99	1	1	0.491669087346904\\
1.005	0	0	0.710195569525497\\
1.005	0.000100583311362513	0.0141829653360114	0.71240460418299\\
1.005	0.000404645907256436	0.0284451766772965	0.714630690536905\\
1.005	0.000915607803433	0.0427829086109896	0.716916312064106\\
1.005	0.00163681893109844	0.057192295687301	0.719289019360234\\
1.005	0.00257155309398959	0.0716693330697585	0.721780367061016\\
1.005	0.00372300185733414	0.086209877460988	0.724425655492835\\
1.005	0.00509426838162598	0.100809648312589	0.727263413850304\\
1.005	0.00668836121491816	0.115464229327074	0.73033458997498\\
1.005	0.00850818805808555	0.13016907025918	0.733681422514709\\
1.005	0.0105565495182326	0.144919489023162	0.737345986077549\\
1.005	0.0128361328661109	0.159710674111862	0.741368418705642\\
1.005	0.0153495058140643	0.174537687332543	0.745784862958201\\
1.005	0.0180991103316243	0.189395466863524	0.750625176092311\\
1.005	0.0210872565164405	0.204278830634725	0.755910489868999\\
1.005	0.0243161165387281	0.219182480034174	0.761650724664453\\
1.005	0.0277877186778607	0.234101003941464	0.767842183867616\\
1.005	0.0315039414701067	0.24902888308801	0.77446537092041\\
1.005	0.0354665079868145	0.263960494742775	0.781483180780606\\
1.005	0.0396769802625738	0.278890117720924	0.788839618268073\\
1.005	0.0441367538930258	0.293811937711588	0.796459186325544\\
1.005	0.0488470528220538	0.308720052919643	0.804247066923454\\
1.005	0.0538089243380495	0.323608480015096	0.812090186161949\\
1.005	0.0590232342988274	0.338471160382329	0.819859213932694\\
1.005	0.064490662604533	0.3533019666601	0.827411499063638\\
1.005	0.0702116989375697	0.368094709561873	0.834594885813505\\
1.005	0.0761866387881432	0.382843144964686	0.841252300287181\\
1.005	0.0824155797834956	0.397540981253432	0.847226939729081\\
1.005	0.0888984183382709	0.412181886906127	0.852367847905042\\
1.005	0.0956348466427212	0.42675949830445	0.856535620027829\\
1.005	0.102624350004627	0.441267427752584	0.859607954619617\\
1.005	0.109866204559871	0.455699271686219	0.861484760266619\\
1.005	0.117359475365564	0.470048619052377	0.86209253425154\\
1.005	0.125103014888515	0.484309059839721	0.861387758068365\\
1.005	0.133095461900593	0.498474193737904	0.85935910089082\\
1.005	0.14133524079124	0.512537638903683	0.856028283765733\\
1.005	0.149820561306021	0.526493040810599	0.851449530870182\\
1.005	0.158549418718625	0.540334081158348	0.845707614739505\\
1.005	0.167519594442213	0.554054486817265	0.838914584343154\\
1.005	0.176728657084455	0.567648038782868	0.83120534237084\\
1.005	0.186173963948925	0.581108581114919	0.822732305417527\\
1.005	0.195852662983903	0.594430029835237	0.813659432944309\\
1.005	0.205761695177907	0.607606381758231	0.804155944104869\\
1.005	0.215897797399558	0.620631723228144	0.79439005340697\\
1.005	0.226257505677663	0.633500238737009	0.784523046062799\\
1.005	0.236837158915675	0.646206219397571	0.774703982861591\\
1.005	0.247632903032949	0.658744071245709	0.765065275211016\\
1.005	0.258640695523528	0.671108323347402	0.755719307772965\\
1.005	0.269856310421543	0.683293635685828	0.746756214005914\\
1.005	0.28127534366066	0.695294806804925	0.738242834604295\\
1.005	0.292893218813452	0.707106781186547	0.730222816007836\\
1.005	0.304705193195075	0.71872465633934	0.722717741087767\\
1.005	0.316706364314172	0.730143689578457	0.715729131143265\\
1.005	0.328891676652598	0.741359304476472	0.709241120561857\\
1.005	0.341255928754291	0.752367096967051	0.703223584571265\\
1.005	0.353793780602429	0.763162841084325	0.697635496597417\\
1.005	0.366499761262991	0.773742494322337	0.692428303594619\\
1.005	0.379368276771857	0.784102202600443	0.687549132887069\\
1.005	0.392393618241769	0.794238304822092	0.682943679235328\\
1.005	0.405569970164763	0.804147337016097	0.678558662193046\\
1.005	0.418891418885081	0.813826036051075	0.674343787412872\\
1.005	0.432351961217133	0.823271342915545	0.670253187715551\\
1.005	0.445945513182735	0.832480405557787	0.666246357336157\\
1.005	0.459665918841652	0.841450581281375	0.662288623474185\\
1.005	0.473506959189401	0.850179438693979	0.658351221673193\\
1.005	0.487462361096317	0.85866475920876	0.654411055137887\\
1.005	0.501525806262096	0.866904538099407	0.650450223206467\\
1.005	0.515690940160279	0.874896985111485	0.646455401874959\\
1.005	0.529951380947623	0.882640524634437	0.642417151055156\\
1.005	0.544300728313782	0.890133795440129	0.638329210947586\\
1.005	0.558732572247415	0.897375649995373	0.634187835387498\\
1.005	0.57324050169555	0.904365153357279	0.629991194998376\\
1.005	0.587818113093873	0.911101581661729	0.625738868900924\\
1.005	0.602459018746568	0.917584420216504	0.62143143163314\\
1.005	0.617156855035314	0.923813361211857	0.617070132480554\\
1.005	0.631905290438127	0.92978830106243	0.612656657833838\\
1.005	0.6466980333399	0.935509337395467	0.608192963377087\\
1.005	0.661528839617671	0.940976765701173	0.603681161495105\\
1.005	0.676391519984904	0.946191075661951	0.599123449738608\\
1.005	0.691279947080357	0.951152947177946	0.594522067901591\\
1.005	0.706188062288412	0.955863246106974	0.589879273662388\\
1.005	0.721109882279076	0.960323019737426	0.585197329319936\\
1.005	0.736039505257225	0.964533492013186	0.580478494543975\\
1.005	0.75097111691199	0.968496058529893	0.575725022018313\\
1.005	0.765898996058536	0.972212281322139	0.57093915428956\\
1.005	0.780817519965826	0.975683883461272	0.566123121053175\\
1.005	0.795721169365275	0.97891274348356	0.56127913660556\\
1.005	0.810604533136476	0.981900889668376	0.556409397399198\\
1.005	0.825462312667457	0.984650494185936	0.551516079695734\\
1.005	0.840289325888138	0.987163867133889	0.546601337319641\\
1.005	0.855080510976839	0.989443450481768	0.541667299515141\\
1.005	0.86983092974082	0.991491811941915	0.536716068908599\\
1.005	0.884535770672926	0.993311638785082	0.53174971957814\\
1.005	0.899190351687411	0.994905731618374	0.526770295231796\\
1.005	0.913790122539012	0.996276998142666	0.521779807495105\\
1.005	0.928330666930241	0.997428446906011	0.51678023430868\\
1.005	0.942807704312699	0.998363181068902	0.511773518435871\\
1.005	0.95721709138901	0.999084392196567	0.50676156608017\\
1.005	0.971554823322703	0.999595354092743	0.501746245611874\\
1.005	0.985817034663989	0.999899416688637	0.496729386403004\\
1.005	1	1	0.491669087346906\\
1.02	0	0	0.710195569525497\\
1.02	0.000100583311362513	0.0141829653360114	0.71240460418299\\
1.02	0.000404645907256436	0.0284451766772965	0.714630690536905\\
1.02	0.000915607803433	0.0427829086109896	0.716916312064106\\
1.02	0.00163681893109844	0.057192295687301	0.719289019360234\\
1.02	0.00257155309398959	0.0716693330697585	0.721780367061016\\
1.02	0.00372300185733414	0.0862098774609879	0.724425655492835\\
1.02	0.00509426838162598	0.100809648312589	0.727263413850304\\
1.02	0.00668836121491816	0.115464229327074	0.73033458997498\\
1.02	0.00850818805808555	0.13016907025918	0.733681422514709\\
1.02	0.0105565495182326	0.144919489023162	0.737345986077549\\
1.02	0.0128361328661109	0.159710674111862	0.741368418705642\\
1.02	0.0153495058140643	0.174537687332543	0.745784862958201\\
1.02	0.0180991103316243	0.189395466863524	0.750625176092311\\
1.02	0.0210872565164405	0.204278830634725	0.755910489868998\\
1.02	0.0243161165387281	0.219182480034174	0.761650724664453\\
1.02	0.0277877186778607	0.234101003941464	0.767842183867616\\
1.02	0.0315039414701067	0.24902888308801	0.77446537092041\\
1.02	0.0354665079868145	0.263960494742775	0.781483180780606\\
1.02	0.0396769802625737	0.278890117720924	0.788839618268074\\
1.02	0.0441367538930258	0.293811937711588	0.796459186325544\\
1.02	0.0488470528220538	0.308720052919643	0.804247066923454\\
1.02	0.0538089243380495	0.323608480015096	0.812090186161949\\
1.02	0.0590232342988274	0.338471160382329	0.819859213932694\\
1.02	0.064490662604533	0.3533019666601	0.827411499063638\\
1.02	0.0702116989375697	0.368094709561873	0.834594885813504\\
1.02	0.0761866387881433	0.382843144964686	0.841252300287181\\
1.02	0.0824155797834956	0.397540981253432	0.847226939729081\\
1.02	0.0888984183382709	0.412181886906127	0.852367847905041\\
1.02	0.0956348466427212	0.42675949830445	0.856535620027831\\
1.02	0.102624350004627	0.441267427752584	0.859607954619619\\
1.02	0.109866204559871	0.455699271686218	0.86148476026662\\
1.02	0.117359475365564	0.470048619052377	0.86209253425154\\
1.02	0.125103014888515	0.484309059839721	0.861387758068366\\
1.02	0.133095461900593	0.498474193737904	0.859359100890821\\
1.02	0.14133524079124	0.512537638903683	0.856028283765735\\
1.02	0.149820561306021	0.526493040810599	0.851449530870183\\
1.02	0.158549418718625	0.540334081158348	0.845707614739509\\
1.02	0.167519594442214	0.554054486817265	0.838914584343152\\
1.02	0.176728657084455	0.567648038782868	0.831205342370836\\
1.02	0.186173963948925	0.581108581114919	0.822732305417528\\
1.02	0.195852662983903	0.594430029835237	0.813659432944309\\
1.02	0.205761695177907	0.607606381758231	0.80415594410487\\
1.02	0.215897797399558	0.620631723228144	0.794390053406969\\
1.02	0.226257505677663	0.633500238737009	0.7845230460628\\
1.02	0.236837158915675	0.646206219397571	0.774703982861591\\
1.02	0.247632903032949	0.658744071245709	0.765065275211015\\
1.02	0.258640695523528	0.671108323347402	0.755719307772964\\
1.02	0.269856310421543	0.683293635685828	0.746756214005913\\
1.02	0.28127534366066	0.695294806804925	0.738242834604294\\
1.02	0.292893218813452	0.707106781186547	0.730222816007835\\
1.02	0.304705193195076	0.71872465633934	0.722717741087767\\
1.02	0.316706364314172	0.730143689578457	0.715729131143265\\
1.02	0.328891676652598	0.741359304476472	0.709241120561857\\
1.02	0.341255928754291	0.752367096967051	0.703223584571265\\
1.02	0.353793780602429	0.763162841084325	0.697635496597417\\
1.02	0.366499761262991	0.773742494322337	0.692428303594619\\
1.02	0.379368276771857	0.784102202600443	0.687549132887069\\
1.02	0.392393618241769	0.794238304822092	0.682943679235329\\
1.02	0.405569970164763	0.804147337016097	0.678558662193046\\
1.02	0.418891418885081	0.813826036051075	0.674343787412872\\
1.02	0.432351961217133	0.823271342915545	0.670253187715552\\
1.02	0.445945513182735	0.832480405557787	0.666246357336157\\
1.02	0.459665918841652	0.841450581281375	0.662288623474184\\
1.02	0.473506959189401	0.850179438693979	0.658351221673191\\
1.02	0.487462361096317	0.85866475920876	0.654411055137887\\
1.02	0.501525806262096	0.866904538099407	0.650450223206467\\
1.02	0.515690940160279	0.874896985111485	0.646455401874959\\
1.02	0.529951380947623	0.882640524634437	0.642417151055155\\
1.02	0.544300728313782	0.890133795440129	0.638329210947586\\
1.02	0.558732572247415	0.897375649995373	0.634187835387498\\
1.02	0.57324050169555	0.904365153357279	0.629991194998376\\
1.02	0.587818113093873	0.911101581661729	0.625738868900924\\
1.02	0.602459018746568	0.917584420216504	0.62143143163314\\
1.02	0.617156855035314	0.923813361211857	0.617070132480554\\
1.02	0.631905290438127	0.92978830106243	0.612656657833838\\
1.02	0.6466980333399	0.935509337395467	0.608192963377087\\
1.02	0.661528839617671	0.940976765701173	0.603681161495105\\
1.02	0.676391519984904	0.946191075661951	0.599123449738608\\
1.02	0.691279947080357	0.951152947177946	0.594522067901591\\
1.02	0.706188062288412	0.955863246106974	0.589879273662389\\
1.02	0.721109882279076	0.960323019737426	0.585197329319936\\
1.02	0.736039505257225	0.964533492013186	0.580478494543978\\
1.02	0.75097111691199	0.968496058529893	0.575725022018314\\
1.02	0.765898996058536	0.972212281322139	0.570939154289557\\
1.02	0.780817519965826	0.975683883461272	0.566123121053175\\
1.02	0.795721169365275	0.97891274348356	0.56127913660556\\
1.02	0.810604533136476	0.981900889668376	0.556409397399195\\
1.02	0.825462312667457	0.984650494185936	0.551516079695734\\
1.02	0.840289325888138	0.987163867133889	0.546601337319644\\
1.02	0.855080510976839	0.989443450481768	0.541667299515141\\
1.02	0.86983092974082	0.991491811941915	0.536716068908594\\
1.02	0.884535770672926	0.993311638785082	0.53174971957814\\
1.02	0.899190351687411	0.994905731618374	0.526770295231796\\
1.02	0.913790122539012	0.996276998142666	0.521779807495105\\
1.02	0.928330666930242	0.997428446906011	0.516780234308687\\
1.02	0.942807704312699	0.998363181068902	0.511773518435871\\
1.02	0.95721709138901	0.999084392196567	0.506761566080166\\
1.02	0.971554823322703	0.999595354092743	0.501746245611874\\
1.02	0.985817034663989	0.999899416688637	0.496729386403004\\
1.02	1	1	0.491669087346906\\
1.035	0	0	0.710195569525497\\
1.035	0.000100583311362513	0.0141829653360114	0.71240460418299\\
1.035	0.000404645907256436	0.0284451766772965	0.714630690536905\\
1.035	0.000915607803433	0.0427829086109896	0.716916312064106\\
1.035	0.00163681893109844	0.057192295687301	0.719289019360234\\
1.035	0.00257155309398959	0.0716693330697584	0.721780367061016\\
1.035	0.00372300185733414	0.086209877460988	0.724425655492835\\
1.035	0.00509426838162598	0.100809648312589	0.727263413850304\\
1.035	0.00668836121491816	0.115464229327074	0.73033458997498\\
1.035	0.00850818805808555	0.13016907025918	0.733681422514709\\
1.035	0.0105565495182326	0.144919489023162	0.737345986077549\\
1.035	0.0128361328661109	0.159710674111862	0.741368418705642\\
1.035	0.0153495058140643	0.174537687332543	0.745784862958201\\
1.035	0.0180991103316243	0.189395466863524	0.750625176092311\\
1.035	0.0210872565164405	0.204278830634725	0.755910489868998\\
1.035	0.0243161165387281	0.219182480034174	0.761650724664453\\
1.035	0.0277877186778607	0.234101003941464	0.767842183867616\\
1.035	0.0315039414701067	0.24902888308801	0.77446537092041\\
1.035	0.0354665079868145	0.263960494742775	0.781483180780606\\
1.035	0.0396769802625738	0.278890117720924	0.788839618268073\\
1.035	0.0441367538930258	0.293811937711588	0.796459186325544\\
1.035	0.0488470528220538	0.308720052919643	0.804247066923454\\
1.035	0.0538089243380495	0.323608480015096	0.81209018616195\\
1.035	0.0590232342988274	0.338471160382329	0.819859213932693\\
1.035	0.064490662604533	0.3533019666601	0.827411499063639\\
1.035	0.0702116989375697	0.368094709561873	0.834594885813504\\
1.035	0.0761866387881433	0.382843144964686	0.841252300287178\\
1.035	0.0824155797834956	0.397540981253432	0.847226939729081\\
1.035	0.0888984183382709	0.412181886906127	0.85236784790504\\
1.035	0.0956348466427212	0.42675949830445	0.856535620027828\\
1.035	0.102624350004627	0.441267427752584	0.859607954619618\\
1.035	0.109866204559871	0.455699271686218	0.861484760266623\\
1.035	0.117359475365564	0.470048619052377	0.862092534251541\\
1.035	0.125103014888515	0.484309059839721	0.861387758068369\\
1.035	0.133095461900593	0.498474193737904	0.85935910089082\\
1.035	0.14133524079124	0.512537638903683	0.856028283765738\\
1.035	0.149820561306021	0.526493040810599	0.851449530870183\\
1.035	0.158549418718625	0.540334081158348	0.845707614739503\\
1.035	0.167519594442214	0.554054486817265	0.838914584343151\\
1.035	0.176728657084455	0.567648038782868	0.831205342370836\\
1.035	0.186173963948925	0.581108581114919	0.822732305417528\\
1.035	0.195852662983903	0.594430029835237	0.813659432944309\\
1.035	0.205761695177907	0.607606381758231	0.804155944104869\\
1.035	0.215897797399558	0.620631723228144	0.794390053406968\\
1.035	0.226257505677663	0.633500238737009	0.784523046062801\\
1.035	0.236837158915675	0.646206219397571	0.774703982861587\\
1.035	0.247632903032949	0.658744071245709	0.765065275211014\\
1.035	0.258640695523528	0.671108323347402	0.755719307772964\\
1.035	0.269856310421543	0.683293635685828	0.746756214005914\\
1.035	0.28127534366066	0.695294806804925	0.738242834604294\\
1.035	0.292893218813452	0.707106781186547	0.730222816007836\\
1.035	0.304705193195075	0.71872465633934	0.722717741087767\\
1.035	0.316706364314172	0.730143689578457	0.715729131143265\\
1.035	0.328891676652598	0.741359304476472	0.709241120561857\\
1.035	0.341255928754291	0.752367096967051	0.703223584571265\\
1.035	0.353793780602429	0.763162841084325	0.697635496597417\\
1.035	0.366499761262991	0.773742494322337	0.692428303594619\\
1.035	0.379368276771857	0.784102202600443	0.687549132887069\\
1.035	0.392393618241769	0.794238304822092	0.682943679235328\\
1.035	0.405569970164763	0.804147337016097	0.678558662193046\\
1.035	0.418891418885081	0.813826036051075	0.674343787412872\\
1.035	0.432351961217133	0.823271342915545	0.670253187715552\\
1.035	0.445945513182735	0.832480405557787	0.666246357336158\\
1.035	0.459665918841652	0.841450581281375	0.662288623474185\\
1.035	0.473506959189401	0.850179438693979	0.658351221673191\\
1.035	0.487462361096317	0.85866475920876	0.654411055137887\\
1.035	0.501525806262096	0.866904538099407	0.650450223206468\\
1.035	0.515690940160279	0.874896985111485	0.646455401874959\\
1.035	0.529951380947623	0.882640524634437	0.642417151055155\\
1.035	0.544300728313782	0.890133795440129	0.638329210947585\\
1.035	0.558732572247415	0.897375649995373	0.634187835387499\\
1.035	0.57324050169555	0.904365153357279	0.629991194998376\\
1.035	0.587818113093873	0.911101581661729	0.625738868900923\\
1.035	0.602459018746568	0.917584420216504	0.62143143163314\\
1.035	0.617156855035314	0.923813361211857	0.617070132480554\\
1.035	0.631905290438127	0.92978830106243	0.612656657833838\\
1.035	0.6466980333399	0.935509337395467	0.608192963377087\\
1.035	0.661528839617671	0.940976765701173	0.603681161495104\\
1.035	0.676391519984904	0.94619107566195	0.599123449738607\\
1.035	0.691279947080357	0.951152947177946	0.594522067901592\\
1.035	0.706188062288412	0.955863246106974	0.589879273662389\\
1.035	0.721109882279076	0.960323019737426	0.585197329319936\\
1.035	0.736039505257225	0.964533492013186	0.580478494543978\\
1.035	0.75097111691199	0.968496058529893	0.575725022018316\\
1.035	0.765898996058536	0.972212281322139	0.570939154289559\\
1.035	0.780817519965826	0.975683883461272	0.566123121053174\\
1.035	0.795721169365275	0.97891274348356	0.56127913660556\\
1.035	0.810604533136476	0.981900889668376	0.556409397399198\\
1.035	0.825462312667457	0.984650494185936	0.551516079695734\\
1.035	0.840289325888138	0.987163867133889	0.546601337319641\\
1.035	0.855080510976839	0.989443450481768	0.541667299515141\\
1.035	0.86983092974082	0.991491811941915	0.536716068908598\\
1.035	0.884535770672926	0.993311638785082	0.531749719578139\\
1.035	0.899190351687411	0.994905731618374	0.526770295231796\\
1.035	0.913790122539012	0.996276998142666	0.521779807495103\\
1.035	0.928330666930242	0.997428446906011	0.516780234308684\\
1.035	0.942807704312699	0.998363181068902	0.511773518435873\\
1.035	0.95721709138901	0.999084392196567	0.50676156608017\\
1.035	0.971554823322704	0.999595354092744	0.501746245611874\\
1.035	0.985817034663989	0.999899416688637	0.496729386403002\\
1.035	1	1	0.491669087346908\\
1.05	0	0	0.710195569525497\\
1.05	0.000100583311362513	0.0141829653360114	0.71240460418299\\
1.05	0.000404645907256436	0.0284451766772965	0.714630690536905\\
1.05	0.000915607803433	0.0427829086109896	0.716916312064106\\
1.05	0.00163681893109844	0.057192295687301	0.719289019360234\\
1.05	0.00257155309398959	0.0716693330697585	0.721780367061016\\
1.05	0.00372300185733414	0.086209877460988	0.724425655492835\\
1.05	0.00509426838162598	0.100809648312589	0.727263413850304\\
1.05	0.00668836121491816	0.115464229327074	0.73033458997498\\
1.05	0.00850818805808555	0.13016907025918	0.733681422514709\\
1.05	0.0105565495182326	0.144919489023162	0.737345986077549\\
1.05	0.0128361328661109	0.159710674111862	0.741368418705642\\
1.05	0.0153495058140643	0.174537687332543	0.745784862958202\\
1.05	0.0180991103316243	0.189395466863524	0.750625176092311\\
1.05	0.0210872565164405	0.204278830634725	0.755910489868998\\
1.05	0.0243161165387281	0.219182480034174	0.761650724664453\\
1.05	0.0277877186778607	0.234101003941464	0.767842183867616\\
1.05	0.0315039414701067	0.24902888308801	0.77446537092041\\
1.05	0.0354665079868145	0.263960494742775	0.781483180780606\\
1.05	0.0396769802625738	0.278890117720924	0.788839618268073\\
1.05	0.0441367538930258	0.293811937711588	0.796459186325544\\
1.05	0.0488470528220537	0.308720052919643	0.804247066923454\\
1.05	0.0538089243380495	0.323608480015096	0.81209018616195\\
1.05	0.0590232342988274	0.338471160382329	0.819859213932694\\
1.05	0.064490662604533	0.3533019666601	0.827411499063639\\
1.05	0.0702116989375697	0.368094709561873	0.834594885813504\\
1.05	0.0761866387881432	0.382843144964686	0.841252300287178\\
1.05	0.0824155797834956	0.397540981253432	0.847226939729081\\
1.05	0.0888984183382709	0.412181886906127	0.852367847905041\\
1.05	0.0956348466427212	0.42675949830445	0.856535620027828\\
1.05	0.102624350004627	0.441267427752584	0.859607954619617\\
1.05	0.109866204559871	0.455699271686219	0.861484760266621\\
1.05	0.117359475365564	0.470048619052377	0.86209253425154\\
1.05	0.125103014888515	0.484309059839721	0.86138775806837\\
1.05	0.133095461900593	0.498474193737904	0.859359100890821\\
1.05	0.14133524079124	0.512537638903683	0.85602828376574\\
1.05	0.149820561306021	0.526493040810599	0.851449530870182\\
1.05	0.158549418718625	0.540334081158348	0.845707614739506\\
1.05	0.167519594442214	0.554054486817265	0.83891458434315\\
1.05	0.176728657084455	0.567648038782868	0.831205342370837\\
1.05	0.186173963948925	0.581108581114919	0.82273230541753\\
1.05	0.195852662983903	0.594430029835237	0.813659432944309\\
1.05	0.205761695177907	0.607606381758231	0.804155944104869\\
1.05	0.215897797399558	0.620631723228144	0.794390053406968\\
1.05	0.226257505677663	0.633500238737009	0.7845230460628\\
1.05	0.236837158915675	0.646206219397571	0.774703982861587\\
1.05	0.247632903032949	0.658744071245709	0.765065275211016\\
1.05	0.258640695523528	0.671108323347402	0.755719307772964\\
1.05	0.269856310421543	0.683293635685828	0.746756214005914\\
1.05	0.28127534366066	0.695294806804925	0.738242834604294\\
1.05	0.292893218813452	0.707106781186547	0.730222816007836\\
1.05	0.304705193195075	0.71872465633934	0.722717741087767\\
1.05	0.316706364314172	0.730143689578457	0.715729131143265\\
1.05	0.328891676652598	0.741359304476472	0.709241120561857\\
1.05	0.341255928754291	0.752367096967051	0.703223584571265\\
1.05	0.353793780602429	0.763162841084325	0.697635496597417\\
1.05	0.366499761262991	0.773742494322337	0.692428303594619\\
1.05	0.379368276771857	0.784102202600443	0.68754913288707\\
1.05	0.392393618241769	0.794238304822092	0.682943679235328\\
1.05	0.405569970164763	0.804147337016097	0.678558662193046\\
1.05	0.418891418885081	0.813826036051075	0.674343787412872\\
1.05	0.432351961217133	0.823271342915545	0.670253187715552\\
1.05	0.445945513182735	0.832480405557787	0.666246357336157\\
1.05	0.459665918841652	0.841450581281375	0.662288623474185\\
1.05	0.473506959189401	0.850179438693979	0.658351221673192\\
1.05	0.487462361096317	0.85866475920876	0.654411055137886\\
1.05	0.501525806262096	0.866904538099407	0.650450223206468\\
1.05	0.515690940160279	0.874896985111485	0.646455401874959\\
1.05	0.529951380947623	0.882640524634437	0.642417151055155\\
1.05	0.544300728313782	0.890133795440129	0.638329210947585\\
1.05	0.558732572247415	0.897375649995373	0.634187835387498\\
1.05	0.57324050169555	0.904365153357279	0.629991194998376\\
1.05	0.587818113093873	0.911101581661729	0.625738868900924\\
1.05	0.602459018746568	0.917584420216504	0.62143143163314\\
1.05	0.617156855035314	0.923813361211857	0.617070132480554\\
1.05	0.631905290438127	0.92978830106243	0.612656657833838\\
1.05	0.6466980333399	0.935509337395467	0.608192963377087\\
1.05	0.661528839617671	0.940976765701173	0.603681161495105\\
1.05	0.676391519984904	0.946191075661951	0.599123449738606\\
1.05	0.691279947080357	0.951152947177946	0.59452206790159\\
1.05	0.706188062288412	0.955863246106974	0.589879273662388\\
1.05	0.721109882279076	0.960323019737426	0.585197329319936\\
1.05	0.736039505257225	0.964533492013186	0.580478494543975\\
1.05	0.75097111691199	0.968496058529893	0.575725022018312\\
1.05	0.765898996058536	0.972212281322139	0.570939154289561\\
1.05	0.780817519965826	0.975683883461272	0.566123121053178\\
1.05	0.795721169365275	0.97891274348356	0.56127913660556\\
1.05	0.810604533136476	0.981900889668376	0.556409397399198\\
1.05	0.825462312667457	0.984650494185936	0.551516079695735\\
1.05	0.840289325888138	0.987163867133889	0.546601337319642\\
1.05	0.855080510976839	0.989443450481768	0.541667299515141\\
1.05	0.86983092974082	0.991491811941915	0.536716068908596\\
1.05	0.884535770672926	0.993311638785082	0.53174971957814\\
1.05	0.899190351687411	0.994905731618374	0.526770295231796\\
1.05	0.913790122539012	0.996276998142666	0.521779807495102\\
1.05	0.928330666930242	0.997428446906011	0.516780234308682\\
1.05	0.942807704312699	0.998363181068902	0.511773518435872\\
1.05	0.95721709138901	0.999084392196567	0.50676156608017\\
1.05	0.971554823322703	0.999595354092743	0.501746245611875\\
1.05	0.985817034663989	0.999899416688637	0.496729386403004\\
1.05	1	1	0.491669087346906\\
1.065	0	0	0.710195569525497\\
1.065	0.000100583311362513	0.0141829653360114	0.71240460418299\\
1.065	0.000404645907256436	0.0284451766772965	0.714630690536905\\
1.065	0.000915607803433	0.0427829086109896	0.716916312064106\\
1.065	0.00163681893109844	0.057192295687301	0.719289019360234\\
1.065	0.00257155309398959	0.0716693330697584	0.721780367061016\\
1.065	0.00372300185733414	0.086209877460988	0.724425655492835\\
1.065	0.00509426838162598	0.100809648312589	0.727263413850304\\
1.065	0.00668836121491816	0.115464229327074	0.73033458997498\\
1.065	0.00850818805808555	0.13016907025918	0.733681422514709\\
1.065	0.0105565495182326	0.144919489023162	0.737345986077549\\
1.065	0.0128361328661109	0.159710674111862	0.741368418705642\\
1.065	0.0153495058140643	0.174537687332543	0.745784862958202\\
1.065	0.0180991103316243	0.189395466863524	0.750625176092311\\
1.065	0.0210872565164405	0.204278830634725	0.755910489868999\\
1.065	0.0243161165387281	0.219182480034174	0.761650724664453\\
1.065	0.0277877186778607	0.234101003941464	0.767842183867616\\
1.065	0.0315039414701067	0.24902888308801	0.77446537092041\\
1.065	0.0354665079868145	0.263960494742775	0.781483180780606\\
1.065	0.0396769802625738	0.278890117720924	0.788839618268073\\
1.065	0.0441367538930258	0.293811937711588	0.796459186325545\\
1.065	0.0488470528220537	0.308720052919643	0.804247066923454\\
1.065	0.0538089243380495	0.323608480015097	0.812090186161949\\
1.065	0.0590232342988274	0.338471160382329	0.819859213932693\\
1.065	0.064490662604533	0.3533019666601	0.827411499063639\\
1.065	0.0702116989375697	0.368094709561873	0.834594885813504\\
1.065	0.0761866387881432	0.382843144964686	0.841252300287179\\
1.065	0.0824155797834956	0.397540981253432	0.847226939729081\\
1.065	0.0888984183382709	0.412181886906127	0.852367847905042\\
1.065	0.0956348466427213	0.42675949830445	0.856535620027831\\
1.065	0.102624350004627	0.441267427752584	0.859607954619617\\
1.065	0.109866204559871	0.455699271686219	0.861484760266618\\
1.065	0.117359475365564	0.470048619052377	0.862092534251537\\
1.065	0.125103014888515	0.484309059839721	0.861387758068369\\
1.065	0.133095461900593	0.498474193737904	0.859359100890822\\
1.065	0.14133524079124	0.512537638903683	0.856028283765735\\
1.065	0.149820561306021	0.526493040810599	0.851449530870182\\
1.065	0.158549418718625	0.540334081158348	0.845707614739508\\
1.065	0.167519594442214	0.554054486817265	0.838914584343149\\
1.065	0.176728657084455	0.567648038782868	0.831205342370836\\
1.065	0.186173963948925	0.581108581114919	0.822732305417526\\
1.065	0.195852662983903	0.594430029835237	0.813659432944307\\
1.065	0.205761695177907	0.607606381758231	0.804155944104868\\
1.065	0.215897797399558	0.620631723228143	0.794390053406969\\
1.065	0.226257505677663	0.633500238737009	0.7845230460628\\
1.065	0.236837158915675	0.646206219397571	0.774703982861589\\
1.065	0.247632903032949	0.658744071245709	0.765065275211016\\
1.065	0.258640695523528	0.671108323347402	0.755719307772965\\
1.065	0.269856310421543	0.683293635685828	0.746756214005914\\
1.065	0.28127534366066	0.695294806804925	0.738242834604294\\
1.065	0.292893218813452	0.707106781186547	0.730222816007835\\
1.065	0.304705193195076	0.71872465633934	0.722717741087767\\
1.065	0.316706364314172	0.730143689578457	0.715729131143266\\
1.065	0.328891676652598	0.741359304476472	0.709241120561857\\
1.065	0.341255928754291	0.752367096967051	0.703223584571265\\
1.065	0.353793780602429	0.763162841084325	0.697635496597417\\
1.065	0.366499761262991	0.773742494322337	0.692428303594618\\
1.065	0.379368276771857	0.784102202600442	0.68754913288707\\
1.065	0.392393618241769	0.794238304822092	0.682943679235329\\
1.065	0.405569970164763	0.804147337016097	0.678558662193046\\
1.065	0.418891418885081	0.813826036051075	0.674343787412872\\
1.065	0.432351961217133	0.823271342915545	0.670253187715552\\
1.065	0.445945513182735	0.832480405557787	0.666246357336157\\
1.065	0.459665918841652	0.841450581281375	0.662288623474184\\
1.065	0.473506959189401	0.850179438693979	0.658351221673193\\
1.065	0.487462361096317	0.85866475920876	0.654411055137886\\
1.065	0.501525806262096	0.866904538099407	0.650450223206466\\
1.065	0.515690940160279	0.874896985111485	0.646455401874959\\
1.065	0.529951380947623	0.882640524634437	0.642417151055156\\
1.065	0.544300728313782	0.890133795440129	0.638329210947585\\
1.065	0.558732572247415	0.897375649995373	0.634187835387497\\
1.065	0.57324050169555	0.904365153357279	0.629991194998376\\
1.065	0.587818113093873	0.911101581661729	0.625738868900924\\
1.065	0.602459018746568	0.917584420216504	0.62143143163314\\
1.065	0.617156855035314	0.923813361211857	0.617070132480554\\
1.065	0.631905290438127	0.92978830106243	0.612656657833838\\
1.065	0.6466980333399	0.935509337395467	0.608192963377087\\
1.065	0.661528839617671	0.940976765701173	0.603681161495105\\
1.065	0.676391519984904	0.946191075661951	0.599123449738607\\
1.065	0.691279947080357	0.951152947177946	0.594522067901592\\
1.065	0.706188062288412	0.955863246106974	0.589879273662388\\
1.065	0.721109882279076	0.960323019737426	0.585197329319934\\
1.065	0.736039505257225	0.964533492013186	0.580478494543978\\
1.065	0.75097111691199	0.968496058529893	0.575725022018313\\
1.065	0.765898996058536	0.972212281322139	0.570939154289557\\
1.065	0.780817519965826	0.975683883461272	0.566123121053175\\
1.065	0.795721169365275	0.97891274348356	0.561279136605556\\
1.065	0.810604533136476	0.981900889668376	0.556409397399197\\
1.065	0.825462312667457	0.984650494185936	0.551516079695737\\
1.065	0.840289325888138	0.987163867133889	0.546601337319642\\
1.065	0.855080510976839	0.989443450481768	0.541667299515141\\
1.065	0.86983092974082	0.991491811941915	0.5367160689086\\
1.065	0.884535770672926	0.993311638785082	0.53174971957814\\
1.065	0.899190351687411	0.994905731618374	0.526770295231796\\
1.065	0.913790122539012	0.996276998142666	0.521779807495105\\
1.065	0.928330666930242	0.997428446906011	0.516780234308684\\
1.065	0.942807704312699	0.998363181068902	0.511773518435872\\
1.065	0.95721709138901	0.999084392196567	0.506761566080164\\
1.065	0.971554823322703	0.999595354092743	0.501746245611874\\
1.065	0.985817034663989	0.999899416688637	0.496729386403005\\
1.065	1	1	0.491669087346904\\
1.08	0	0	0.710195569525497\\
1.08	0.000100583311362513	0.0141829653360114	0.71240460418299\\
1.08	0.000404645907256436	0.0284451766772965	0.714630690536905\\
1.08	0.000915607803433	0.0427829086109896	0.716916312064106\\
1.08	0.00163681893109844	0.057192295687301	0.719289019360234\\
1.08	0.00257155309398959	0.0716693330697585	0.721780367061016\\
1.08	0.00372300185733414	0.0862098774609879	0.724425655492835\\
1.08	0.00509426838162598	0.100809648312589	0.727263413850304\\
1.08	0.00668836121491816	0.115464229327074	0.73033458997498\\
1.08	0.00850818805808555	0.13016907025918	0.733681422514709\\
1.08	0.0105565495182326	0.144919489023162	0.737345986077549\\
1.08	0.0128361328661109	0.159710674111862	0.741368418705642\\
1.08	0.0153495058140643	0.174537687332543	0.745784862958202\\
1.08	0.0180991103316243	0.189395466863524	0.750625176092311\\
1.08	0.0210872565164405	0.204278830634725	0.755910489868998\\
1.08	0.0243161165387281	0.219182480034174	0.761650724664453\\
1.08	0.0277877186778607	0.234101003941464	0.767842183867616\\
1.08	0.0315039414701067	0.24902888308801	0.77446537092041\\
1.08	0.0354665079868145	0.263960494742775	0.781483180780606\\
1.08	0.0396769802625738	0.278890117720924	0.788839618268073\\
1.08	0.0441367538930258	0.293811937711588	0.796459186325544\\
1.08	0.0488470528220537	0.308720052919643	0.804247066923455\\
1.08	0.0538089243380495	0.323608480015096	0.812090186161949\\
1.08	0.0590232342988274	0.338471160382329	0.819859213932694\\
1.08	0.064490662604533	0.3533019666601	0.827411499063639\\
1.08	0.0702116989375697	0.368094709561873	0.834594885813505\\
1.08	0.0761866387881432	0.382843144964686	0.841252300287178\\
1.08	0.0824155797834956	0.397540981253432	0.847226939729081\\
1.08	0.0888984183382709	0.412181886906127	0.852367847905041\\
1.08	0.0956348466427212	0.42675949830445	0.85653562002783\\
1.08	0.102624350004627	0.441267427752584	0.859607954619618\\
1.08	0.109866204559871	0.455699271686218	0.861484760266621\\
1.08	0.117359475365564	0.470048619052377	0.862092534251537\\
1.08	0.125103014888515	0.484309059839721	0.861387758068368\\
1.08	0.133095461900593	0.498474193737904	0.85935910089082\\
1.08	0.14133524079124	0.512537638903683	0.856028283765732\\
1.08	0.149820561306021	0.526493040810599	0.851449530870183\\
1.08	0.158549418718625	0.540334081158348	0.845707614739504\\
1.08	0.167519594442214	0.554054486817265	0.83891458434315\\
1.08	0.176728657084455	0.567648038782867	0.831205342370834\\
1.08	0.186173963948925	0.581108581114919	0.822732305417525\\
1.08	0.195852662983903	0.594430029835237	0.813659432944309\\
1.08	0.205761695177907	0.607606381758231	0.804155944104868\\
1.08	0.215897797399558	0.620631723228143	0.794390053406969\\
1.08	0.226257505677663	0.633500238737009	0.7845230460628\\
1.08	0.236837158915675	0.646206219397571	0.774703982861589\\
1.08	0.247632903032949	0.658744071245709	0.765065275211016\\
1.08	0.258640695523528	0.671108323347402	0.755719307772964\\
1.08	0.269856310421543	0.683293635685828	0.746756214005913\\
1.08	0.28127534366066	0.695294806804925	0.738242834604294\\
1.08	0.292893218813452	0.707106781186547	0.730222816007836\\
1.08	0.304705193195075	0.71872465633934	0.722717741087767\\
1.08	0.316706364314172	0.730143689578457	0.715729131143265\\
1.08	0.328891676652598	0.741359304476472	0.709241120561857\\
1.08	0.341255928754291	0.752367096967051	0.703223584571265\\
1.08	0.353793780602429	0.763162841084325	0.697635496597417\\
1.08	0.366499761262991	0.773742494322337	0.692428303594618\\
1.08	0.379368276771857	0.784102202600442	0.687549132887069\\
1.08	0.392393618241769	0.794238304822092	0.682943679235329\\
1.08	0.405569970164763	0.804147337016097	0.678558662193046\\
1.08	0.418891418885081	0.813826036051075	0.674343787412872\\
1.08	0.432351961217133	0.823271342915545	0.670253187715552\\
1.08	0.445945513182735	0.832480405557787	0.666246357336158\\
1.08	0.459665918841652	0.841450581281375	0.662288623474184\\
1.08	0.473506959189401	0.850179438693979	0.658351221673191\\
1.08	0.487462361096317	0.85866475920876	0.654411055137887\\
1.08	0.501525806262096	0.866904538099407	0.650450223206468\\
1.08	0.515690940160279	0.874896985111485	0.646455401874959\\
1.08	0.529951380947623	0.882640524634437	0.642417151055155\\
1.08	0.544300728313782	0.890133795440129	0.638329210947586\\
1.08	0.558732572247415	0.897375649995373	0.634187835387498\\
1.08	0.57324050169555	0.904365153357279	0.629991194998376\\
1.08	0.587818113093873	0.911101581661729	0.625738868900924\\
1.08	0.602459018746568	0.917584420216504	0.62143143163314\\
1.08	0.617156855035314	0.923813361211857	0.617070132480554\\
1.08	0.631905290438127	0.92978830106243	0.612656657833838\\
1.08	0.6466980333399	0.935509337395467	0.608192963377087\\
1.08	0.661528839617671	0.940976765701173	0.603681161495105\\
1.08	0.676391519984904	0.946191075661951	0.599123449738607\\
1.08	0.691279947080357	0.951152947177946	0.594522067901591\\
1.08	0.706188062288412	0.955863246106974	0.58987927366239\\
1.08	0.721109882279076	0.960323019737426	0.585197329319934\\
1.08	0.736039505257225	0.964533492013186	0.580478494543978\\
1.08	0.75097111691199	0.968496058529893	0.575725022018313\\
1.08	0.765898996058536	0.972212281322139	0.570939154289558\\
1.08	0.780817519965826	0.975683883461272	0.566123121053177\\
1.08	0.795721169365275	0.97891274348356	0.561279136605557\\
1.08	0.810604533136476	0.981900889668376	0.556409397399197\\
1.08	0.825462312667457	0.984650494185936	0.551516079695737\\
1.08	0.840289325888138	0.987163867133889	0.546601337319641\\
1.08	0.855080510976839	0.989443450481768	0.541667299515139\\
1.08	0.86983092974082	0.991491811941914	0.536716068908594\\
1.08	0.884535770672926	0.993311638785082	0.531749719578141\\
1.08	0.899190351687411	0.994905731618374	0.526770295231796\\
1.08	0.913790122539012	0.996276998142666	0.521779807495105\\
1.08	0.928330666930242	0.997428446906011	0.516780234308685\\
1.08	0.942807704312699	0.998363181068902	0.511773518435873\\
1.08	0.95721709138901	0.999084392196567	0.50676156608017\\
1.08	0.971554823322704	0.999595354092744	0.501746245611874\\
1.08	0.985817034663989	0.999899416688637	0.496729386403002\\
1.08	1	1	0.491669087346908\\
1.095	0	0	0.710195569525497\\
1.095	0.000100583311362513	0.0141829653360114	0.71240460418299\\
1.095	0.000404645907256436	0.0284451766772965	0.714630690536905\\
1.095	0.000915607803433	0.0427829086109896	0.716916312064106\\
1.095	0.00163681893109844	0.057192295687301	0.719289019360234\\
1.095	0.00257155309398959	0.0716693330697584	0.721780367061016\\
1.095	0.00372300185733414	0.086209877460988	0.724425655492835\\
1.095	0.00509426838162598	0.100809648312589	0.727263413850304\\
1.095	0.00668836121491816	0.115464229327074	0.73033458997498\\
1.095	0.00850818805808555	0.13016907025918	0.733681422514709\\
1.095	0.0105565495182326	0.144919489023162	0.737345986077549\\
1.095	0.0128361328661109	0.159710674111862	0.741368418705642\\
1.095	0.0153495058140643	0.174537687332543	0.745784862958201\\
1.095	0.0180991103316243	0.189395466863524	0.750625176092311\\
1.095	0.0210872565164405	0.204278830634725	0.755910489868998\\
1.095	0.0243161165387281	0.219182480034174	0.761650724664453\\
1.095	0.0277877186778607	0.234101003941464	0.767842183867616\\
1.095	0.0315039414701067	0.24902888308801	0.77446537092041\\
1.095	0.0354665079868145	0.263960494742775	0.781483180780606\\
1.095	0.0396769802625737	0.278890117720924	0.788839618268073\\
1.095	0.0441367538930258	0.293811937711588	0.796459186325544\\
1.095	0.0488470528220538	0.308720052919643	0.804247066923455\\
1.095	0.0538089243380495	0.323608480015096	0.81209018616195\\
1.095	0.0590232342988274	0.338471160382329	0.819859213932693\\
1.095	0.064490662604533	0.3533019666601	0.82741149906364\\
1.095	0.0702116989375697	0.368094709561873	0.834594885813505\\
1.095	0.0761866387881432	0.382843144964686	0.84125230028718\\
1.095	0.0824155797834956	0.397540981253432	0.847226939729081\\
1.095	0.0888984183382709	0.412181886906127	0.85236784790504\\
1.095	0.0956348466427212	0.42675949830445	0.856535620027828\\
1.095	0.102624350004627	0.441267427752584	0.859607954619617\\
1.095	0.109866204559871	0.455699271686219	0.861484760266624\\
1.095	0.117359475365564	0.470048619052377	0.862092534251542\\
1.095	0.125103014888515	0.484309059839721	0.86138775806837\\
1.095	0.133095461900593	0.498474193737904	0.85935910089082\\
1.095	0.14133524079124	0.512537638903683	0.856028283765737\\
1.095	0.149820561306021	0.526493040810599	0.851449530870183\\
1.095	0.158549418718625	0.540334081158348	0.845707614739505\\
1.095	0.167519594442214	0.554054486817265	0.83891458434315\\
1.095	0.176728657084455	0.567648038782867	0.831205342370836\\
1.095	0.186173963948925	0.581108581114919	0.822732305417529\\
1.095	0.195852662983903	0.594430029835237	0.813659432944309\\
1.095	0.205761695177907	0.607606381758231	0.804155944104868\\
1.095	0.215897797399558	0.620631723228143	0.794390053406969\\
1.095	0.226257505677663	0.633500238737009	0.7845230460628\\
1.095	0.236837158915675	0.646206219397571	0.774703982861588\\
1.095	0.247632903032949	0.658744071245709	0.765065275211016\\
1.095	0.258640695523528	0.671108323347402	0.755719307772964\\
1.095	0.269856310421543	0.683293635685828	0.746756214005914\\
1.095	0.28127534366066	0.695294806804925	0.738242834604294\\
1.095	0.292893218813452	0.707106781186547	0.730222816007836\\
1.095	0.304705193195075	0.71872465633934	0.722717741087767\\
1.095	0.316706364314172	0.730143689578457	0.715729131143265\\
1.095	0.328891676652598	0.741359304476472	0.709241120561857\\
1.095	0.341255928754291	0.752367096967051	0.703223584571265\\
1.095	0.353793780602429	0.763162841084325	0.697635496597417\\
1.095	0.366499761262991	0.773742494322337	0.692428303594618\\
1.095	0.379368276771857	0.784102202600442	0.687549132887069\\
1.095	0.392393618241769	0.794238304822092	0.682943679235329\\
1.095	0.405569970164763	0.804147337016097	0.678558662193046\\
1.095	0.418891418885081	0.813826036051075	0.674343787412872\\
1.095	0.432351961217133	0.823271342915545	0.670253187715552\\
1.095	0.445945513182735	0.832480405557787	0.666246357336158\\
1.095	0.459665918841652	0.841450581281375	0.662288623474185\\
1.095	0.473506959189401	0.850179438693979	0.658351221673191\\
1.095	0.487462361096317	0.85866475920876	0.654411055137887\\
1.095	0.501525806262096	0.866904538099407	0.650450223206467\\
1.095	0.515690940160279	0.874896985111485	0.646455401874959\\
1.095	0.529951380947623	0.882640524634437	0.642417151055155\\
1.095	0.544300728313782	0.890133795440129	0.638329210947585\\
1.095	0.558732572247415	0.897375649995373	0.634187835387499\\
1.095	0.57324050169555	0.904365153357279	0.629991194998376\\
1.095	0.587818113093873	0.911101581661729	0.625738868900924\\
1.095	0.602459018746568	0.917584420216504	0.62143143163314\\
1.095	0.617156855035314	0.923813361211857	0.617070132480554\\
1.095	0.631905290438127	0.92978830106243	0.612656657833838\\
1.095	0.6466980333399	0.935509337395467	0.608192963377087\\
1.095	0.661528839617671	0.940976765701173	0.603681161495105\\
1.095	0.676391519984904	0.946191075661951	0.599123449738608\\
1.095	0.691279947080357	0.951152947177946	0.594522067901589\\
1.095	0.706188062288412	0.955863246106974	0.589879273662388\\
1.095	0.721109882279076	0.960323019737426	0.585197329319937\\
1.095	0.736039505257225	0.964533492013186	0.580478494543978\\
1.095	0.75097111691199	0.968496058529893	0.575725022018313\\
1.095	0.765898996058536	0.972212281322139	0.570939154289558\\
1.095	0.780817519965826	0.975683883461272	0.566123121053175\\
1.095	0.795721169365275	0.97891274348356	0.561279136605559\\
1.095	0.810604533136476	0.981900889668376	0.556409397399197\\
1.095	0.825462312667457	0.984650494185936	0.551516079695736\\
1.095	0.840289325888138	0.987163867133889	0.546601337319641\\
1.095	0.855080510976839	0.989443450481768	0.541667299515141\\
1.095	0.86983092974082	0.991491811941915	0.536716068908598\\
1.095	0.884535770672926	0.993311638785082	0.531749719578139\\
1.095	0.899190351687411	0.994905731618374	0.526770295231796\\
1.095	0.913790122539012	0.996276998142666	0.521779807495104\\
1.095	0.928330666930242	0.997428446906011	0.516780234308684\\
1.095	0.942807704312699	0.998363181068902	0.511773518435873\\
1.095	0.95721709138901	0.999084392196567	0.506761566080168\\
1.095	0.971554823322703	0.999595354092743	0.501746245611875\\
1.095	0.985817034663989	0.999899416688637	0.496729386403004\\
1.095	1	1	0.491669087346906\\
1.11	0	0	0.710195569525497\\
1.11	0.000100583311362513	0.0141829653360114	0.71240460418299\\
1.11	0.000404645907256436	0.0284451766772965	0.714630690536905\\
1.11	0.000915607803433	0.0427829086109896	0.716916312064106\\
1.11	0.00163681893109844	0.057192295687301	0.719289019360234\\
1.11	0.00257155309398959	0.0716693330697585	0.721780367061016\\
1.11	0.00372300185733414	0.086209877460988	0.724425655492835\\
1.11	0.00509426838162598	0.100809648312589	0.727263413850304\\
1.11	0.00668836121491816	0.115464229327074	0.73033458997498\\
1.11	0.00850818805808555	0.13016907025918	0.733681422514709\\
1.11	0.0105565495182326	0.144919489023162	0.737345986077549\\
1.11	0.0128361328661109	0.159710674111862	0.741368418705642\\
1.11	0.0153495058140643	0.174537687332543	0.745784862958201\\
1.11	0.0180991103316243	0.189395466863524	0.750625176092311\\
1.11	0.0210872565164405	0.204278830634725	0.755910489868998\\
1.11	0.0243161165387281	0.219182480034174	0.761650724664453\\
1.11	0.0277877186778607	0.234101003941464	0.767842183867616\\
1.11	0.0315039414701067	0.24902888308801	0.77446537092041\\
1.11	0.0354665079868145	0.263960494742775	0.781483180780606\\
1.11	0.0396769802625738	0.278890117720924	0.788839618268073\\
1.11	0.0441367538930258	0.293811937711588	0.796459186325544\\
1.11	0.0488470528220538	0.308720052919643	0.804247066923454\\
1.11	0.0538089243380495	0.323608480015097	0.81209018616195\\
1.11	0.0590232342988274	0.338471160382329	0.819859213932693\\
1.11	0.064490662604533	0.3533019666601	0.827411499063639\\
1.11	0.0702116989375697	0.368094709561873	0.834594885813505\\
1.11	0.0761866387881433	0.382843144964686	0.84125230028718\\
1.11	0.0824155797834956	0.397540981253432	0.847226939729081\\
1.11	0.0888984183382709	0.412181886906127	0.852367847905041\\
1.11	0.0956348466427212	0.42675949830445	0.856535620027829\\
1.11	0.102624350004627	0.441267427752584	0.859607954619617\\
1.11	0.109866204559871	0.455699271686219	0.861484760266622\\
1.11	0.117359475365564	0.470048619052377	0.862092534251542\\
1.11	0.125103014888515	0.484309059839721	0.86138775806837\\
1.11	0.133095461900593	0.498474193737904	0.859359100890819\\
1.11	0.14133524079124	0.512537638903683	0.856028283765737\\
1.11	0.149820561306021	0.526493040810599	0.851449530870183\\
1.11	0.158549418718625	0.540334081158348	0.845707614739506\\
1.11	0.167519594442214	0.554054486817265	0.83891458434315\\
1.11	0.176728657084455	0.567648038782867	0.831205342370837\\
1.11	0.186173963948925	0.581108581114919	0.822732305417528\\
1.11	0.195852662983903	0.594430029835237	0.813659432944308\\
1.11	0.205761695177907	0.607606381758231	0.804155944104868\\
1.11	0.215897797399558	0.620631723228143	0.794390053406969\\
1.11	0.226257505677663	0.633500238737009	0.7845230460628\\
1.11	0.236837158915675	0.646206219397571	0.774703982861589\\
1.11	0.247632903032949	0.658744071245709	0.765065275211016\\
1.11	0.258640695523528	0.671108323347402	0.755719307772964\\
1.11	0.269856310421543	0.683293635685828	0.746756214005914\\
1.11	0.28127534366066	0.695294806804925	0.738242834604294\\
1.11	0.292893218813452	0.707106781186547	0.730222816007835\\
1.11	0.304705193195076	0.71872465633934	0.722717741087767\\
1.11	0.316706364314172	0.730143689578457	0.715729131143266\\
1.11	0.328891676652598	0.741359304476472	0.709241120561857\\
1.11	0.341255928754291	0.752367096967051	0.703223584571265\\
1.11	0.353793780602429	0.763162841084325	0.697635496597417\\
1.11	0.366499761262991	0.773742494322337	0.692428303594618\\
1.11	0.379368276771857	0.784102202600442	0.687549132887069\\
1.11	0.392393618241769	0.794238304822092	0.682943679235329\\
1.11	0.405569970164763	0.804147337016097	0.678558662193046\\
1.11	0.418891418885081	0.813826036051075	0.674343787412872\\
1.11	0.432351961217133	0.823271342915545	0.670253187715552\\
1.11	0.445945513182735	0.832480405557787	0.666246357336158\\
1.11	0.459665918841652	0.841450581281375	0.662288623474185\\
1.11	0.473506959189401	0.850179438693979	0.658351221673191\\
1.11	0.487462361096317	0.85866475920876	0.654411055137886\\
1.11	0.501525806262096	0.866904538099407	0.650450223206468\\
1.11	0.515690940160279	0.874896985111485	0.646455401874959\\
1.11	0.529951380947623	0.882640524634437	0.642417151055155\\
1.11	0.544300728313782	0.890133795440129	0.638329210947585\\
1.11	0.558732572247415	0.897375649995373	0.634187835387498\\
1.11	0.57324050169555	0.904365153357279	0.629991194998376\\
1.11	0.587818113093873	0.911101581661729	0.625738868900924\\
1.11	0.602459018746568	0.917584420216504	0.62143143163314\\
1.11	0.617156855035314	0.923813361211857	0.617070132480554\\
1.11	0.631905290438127	0.92978830106243	0.612656657833838\\
1.11	0.6466980333399	0.935509337395467	0.608192963377087\\
1.11	0.661528839617671	0.940976765701173	0.603681161495105\\
1.11	0.676391519984904	0.946191075661951	0.599123449738607\\
1.11	0.691279947080357	0.951152947177946	0.594522067901591\\
1.11	0.706188062288412	0.955863246106974	0.589879273662388\\
1.11	0.721109882279076	0.960323019737426	0.585197329319936\\
1.11	0.736039505257225	0.964533492013186	0.580478494543978\\
1.11	0.75097111691199	0.968496058529893	0.575725022018312\\
1.11	0.765898996058536	0.972212281322139	0.570939154289558\\
1.11	0.780817519965826	0.975683883461272	0.566123121053177\\
1.11	0.795721169365275	0.97891274348356	0.561279136605558\\
1.11	0.810604533136476	0.981900889668376	0.556409397399199\\
1.11	0.825462312667457	0.984650494185936	0.551516079695736\\
1.11	0.840289325888138	0.987163867133889	0.546601337319639\\
1.11	0.855080510976839	0.989443450481768	0.541667299515139\\
1.11	0.86983092974082	0.991491811941914	0.536716068908598\\
1.11	0.884535770672926	0.993311638785082	0.531749719578142\\
1.11	0.899190351687411	0.994905731618374	0.526770295231796\\
1.11	0.913790122539012	0.996276998142666	0.521779807495103\\
1.11	0.928330666930242	0.997428446906011	0.516780234308682\\
1.11	0.942807704312699	0.998363181068902	0.511773518435872\\
1.11	0.95721709138901	0.999084392196567	0.506761566080168\\
1.11	0.971554823322703	0.999595354092743	0.501746245611874\\
1.11	0.985817034663989	0.999899416688637	0.496729386403005\\
1.11	1	1	0.491669087346904\\
1.125	0	0	0.710195569525497\\
1.125	0.000100583311362513	0.0141829653360114	0.71240460418299\\
1.125	0.000404645907256436	0.0284451766772965	0.714630690536905\\
1.125	0.000915607803433	0.0427829086109896	0.716916312064106\\
1.125	0.00163681893109844	0.057192295687301	0.719289019360234\\
1.125	0.00257155309398959	0.0716693330697585	0.721780367061016\\
1.125	0.00372300185733414	0.086209877460988	0.724425655492835\\
1.125	0.00509426838162598	0.100809648312589	0.727263413850304\\
1.125	0.00668836121491816	0.115464229327074	0.73033458997498\\
1.125	0.00850818805808555	0.13016907025918	0.733681422514709\\
1.125	0.0105565495182326	0.144919489023162	0.737345986077549\\
1.125	0.0128361328661109	0.159710674111862	0.741368418705642\\
1.125	0.0153495058140643	0.174537687332543	0.745784862958201\\
1.125	0.0180991103316243	0.189395466863524	0.750625176092311\\
1.125	0.0210872565164405	0.204278830634725	0.755910489868998\\
1.125	0.0243161165387281	0.219182480034174	0.761650724664453\\
1.125	0.0277877186778607	0.234101003941464	0.767842183867616\\
1.125	0.0315039414701067	0.24902888308801	0.77446537092041\\
1.125	0.0354665079868145	0.263960494742775	0.781483180780606\\
1.125	0.0396769802625738	0.278890117720924	0.788839618268073\\
1.125	0.0441367538930258	0.293811937711588	0.796459186325544\\
1.125	0.0488470528220538	0.308720052919643	0.804247066923454\\
1.125	0.0538089243380495	0.323608480015096	0.812090186161949\\
1.125	0.0590232342988274	0.338471160382329	0.819859213932694\\
1.125	0.064490662604533	0.3533019666601	0.827411499063638\\
1.125	0.0702116989375697	0.368094709561873	0.834594885813504\\
1.125	0.0761866387881432	0.382843144964686	0.841252300287178\\
1.125	0.0824155797834956	0.397540981253432	0.847226939729081\\
1.125	0.0888984183382709	0.412181886906127	0.852367847905042\\
1.125	0.0956348466427212	0.42675949830445	0.85653562002783\\
1.125	0.102624350004627	0.441267427752584	0.859607954619618\\
1.125	0.109866204559871	0.455699271686219	0.861484760266621\\
1.125	0.117359475365564	0.470048619052377	0.86209253425154\\
1.125	0.125103014888515	0.484309059839721	0.861387758068368\\
1.125	0.133095461900593	0.498474193737904	0.859359100890819\\
1.125	0.14133524079124	0.512537638903683	0.856028283765735\\
1.125	0.149820561306021	0.526493040810599	0.851449530870183\\
1.125	0.158549418718625	0.540334081158348	0.845707614739504\\
1.125	0.167519594442214	0.554054486817265	0.83891458434315\\
1.125	0.176728657084455	0.567648038782867	0.831205342370836\\
1.125	0.186173963948925	0.581108581114919	0.822732305417527\\
1.125	0.195852662983903	0.594430029835237	0.813659432944309\\
1.125	0.205761695177907	0.607606381758231	0.804155944104868\\
1.125	0.215897797399558	0.620631723228143	0.794390053406969\\
1.125	0.226257505677663	0.633500238737009	0.7845230460628\\
1.125	0.236837158915675	0.646206219397571	0.774703982861589\\
1.125	0.247632903032949	0.658744071245709	0.765065275211016\\
1.125	0.258640695523528	0.671108323347402	0.755719307772964\\
1.125	0.269856310421543	0.683293635685828	0.746756214005914\\
1.125	0.28127534366066	0.695294806804925	0.738242834604294\\
1.125	0.292893218813452	0.707106781186547	0.730222816007836\\
1.125	0.304705193195075	0.71872465633934	0.722717741087767\\
1.125	0.316706364314172	0.730143689578457	0.715729131143265\\
1.125	0.328891676652598	0.741359304476472	0.709241120561857\\
1.125	0.341255928754291	0.752367096967051	0.703223584571265\\
1.125	0.353793780602429	0.763162841084325	0.697635496597417\\
1.125	0.366499761262991	0.773742494322337	0.692428303594618\\
1.125	0.379368276771857	0.784102202600442	0.687549132887069\\
1.125	0.392393618241769	0.794238304822092	0.682943679235329\\
1.125	0.405569970164763	0.804147337016097	0.678558662193046\\
1.125	0.418891418885081	0.813826036051075	0.674343787412872\\
1.125	0.432351961217133	0.823271342915545	0.670253187715552\\
1.125	0.445945513182735	0.832480405557787	0.666246357336158\\
1.125	0.459665918841652	0.841450581281375	0.662288623474185\\
1.125	0.473506959189401	0.850179438693979	0.658351221673191\\
1.125	0.487462361096317	0.85866475920876	0.654411055137886\\
1.125	0.501525806262096	0.866904538099407	0.650450223206467\\
1.125	0.515690940160279	0.874896985111485	0.646455401874959\\
1.125	0.529951380947623	0.882640524634437	0.642417151055155\\
1.125	0.544300728313782	0.890133795440129	0.638329210947585\\
1.125	0.558732572247415	0.897375649995373	0.634187835387498\\
1.125	0.57324050169555	0.904365153357279	0.629991194998375\\
1.125	0.587818113093873	0.911101581661729	0.625738868900924\\
1.125	0.602459018746568	0.917584420216504	0.62143143163314\\
1.125	0.617156855035314	0.923813361211857	0.617070132480554\\
1.125	0.631905290438127	0.92978830106243	0.612656657833838\\
1.125	0.6466980333399	0.935509337395467	0.608192963377087\\
1.125	0.661528839617671	0.940976765701173	0.603681161495105\\
1.125	0.676391519984904	0.946191075661951	0.599123449738608\\
1.125	0.691279947080357	0.951152947177946	0.59452206790159\\
1.125	0.706188062288412	0.955863246106974	0.589879273662389\\
1.125	0.721109882279076	0.960323019737426	0.585197329319936\\
1.125	0.736039505257225	0.964533492013186	0.580478494543978\\
1.125	0.75097111691199	0.968496058529893	0.575725022018313\\
1.125	0.765898996058536	0.972212281322139	0.570939154289557\\
1.125	0.780817519965826	0.975683883461272	0.566123121053175\\
1.125	0.795721169365275	0.97891274348356	0.561279136605558\\
1.125	0.810604533136476	0.981900889668376	0.556409397399198\\
1.125	0.825462312667457	0.984650494185936	0.551516079695737\\
1.125	0.840289325888138	0.987163867133889	0.546601337319641\\
1.125	0.855080510976839	0.989443450481768	0.541667299515139\\
1.125	0.86983092974082	0.991491811941914	0.536716068908598\\
1.125	0.884535770672926	0.993311638785082	0.53174971957814\\
1.125	0.899190351687411	0.994905731618374	0.526770295231797\\
1.125	0.913790122539012	0.996276998142666	0.521779807495104\\
1.125	0.928330666930242	0.997428446906011	0.516780234308684\\
1.125	0.942807704312699	0.998363181068902	0.511773518435871\\
1.125	0.95721709138901	0.999084392196567	0.506761566080168\\
1.125	0.971554823322703	0.999595354092743	0.501746245611874\\
1.125	0.985817034663989	0.999899416688637	0.496729386403004\\
1.125	1	1	0.491669087346906\\
1.14	0	0	0.710195569525497\\
1.14	0.000100583311362513	0.0141829653360114	0.71240460418299\\
1.14	0.000404645907256436	0.0284451766772965	0.714630690536905\\
1.14	0.000915607803433	0.0427829086109896	0.716916312064106\\
1.14	0.00163681893109844	0.057192295687301	0.719289019360234\\
1.14	0.00257155309398959	0.0716693330697585	0.721780367061016\\
1.14	0.00372300185733414	0.086209877460988	0.724425655492835\\
1.14	0.00509426838162598	0.100809648312589	0.727263413850304\\
1.14	0.00668836121491816	0.115464229327074	0.73033458997498\\
1.14	0.00850818805808555	0.13016907025918	0.733681422514709\\
1.14	0.0105565495182326	0.144919489023162	0.737345986077549\\
1.14	0.0128361328661109	0.159710674111862	0.741368418705642\\
1.14	0.0153495058140643	0.174537687332543	0.745784862958202\\
1.14	0.0180991103316243	0.189395466863524	0.750625176092311\\
1.14	0.0210872565164405	0.204278830634725	0.755910489868998\\
1.14	0.0243161165387281	0.219182480034174	0.761650724664453\\
1.14	0.0277877186778607	0.234101003941464	0.767842183867616\\
1.14	0.0315039414701067	0.24902888308801	0.77446537092041\\
1.14	0.0354665079868145	0.263960494742775	0.781483180780606\\
1.14	0.0396769802625738	0.278890117720924	0.788839618268073\\
1.14	0.0441367538930258	0.293811937711588	0.796459186325544\\
1.14	0.0488470528220538	0.308720052919643	0.804247066923454\\
1.14	0.0538089243380495	0.323608480015096	0.81209018616195\\
1.14	0.0590232342988274	0.338471160382329	0.819859213932694\\
1.14	0.064490662604533	0.3533019666601	0.827411499063639\\
1.14	0.0702116989375697	0.368094709561873	0.834594885813504\\
1.14	0.0761866387881432	0.382843144964686	0.841252300287179\\
1.14	0.0824155797834956	0.397540981253432	0.847226939729081\\
1.14	0.0888984183382709	0.412181886906127	0.852367847905041\\
1.14	0.0956348466427212	0.42675949830445	0.85653562002783\\
1.14	0.102624350004627	0.441267427752584	0.859607954619617\\
1.14	0.109866204559871	0.455699271686219	0.861484760266622\\
1.14	0.117359475365564	0.470048619052377	0.86209253425154\\
1.14	0.125103014888515	0.484309059839721	0.86138775806837\\
1.14	0.133095461900593	0.498474193737904	0.85935910089082\\
1.14	0.14133524079124	0.512537638903683	0.856028283765736\\
1.14	0.149820561306021	0.526493040810599	0.851449530870182\\
1.14	0.158549418718625	0.540334081158348	0.845707614739505\\
1.14	0.167519594442214	0.554054486817265	0.83891458434315\\
1.14	0.176728657084455	0.567648038782867	0.831205342370836\\
1.14	0.186173963948925	0.581108581114919	0.82273230541753\\
1.14	0.195852662983903	0.594430029835237	0.813659432944309\\
1.14	0.205761695177907	0.607606381758231	0.804155944104867\\
1.14	0.215897797399558	0.620631723228143	0.794390053406969\\
1.14	0.226257505677663	0.633500238737009	0.7845230460628\\
1.14	0.236837158915675	0.646206219397571	0.774703982861589\\
1.14	0.247632903032949	0.658744071245709	0.765065275211016\\
1.14	0.258640695523528	0.671108323347402	0.755719307772964\\
1.14	0.269856310421543	0.683293635685828	0.746756214005914\\
1.14	0.28127534366066	0.695294806804925	0.738242834604294\\
1.14	0.292893218813452	0.707106781186547	0.730222816007836\\
1.14	0.304705193195075	0.71872465633934	0.722717741087767\\
1.14	0.316706364314172	0.730143689578457	0.715729131143265\\
1.14	0.328891676652598	0.741359304476472	0.709241120561857\\
1.14	0.341255928754291	0.752367096967051	0.703223584571265\\
1.14	0.353793780602429	0.763162841084325	0.697635496597417\\
1.14	0.366499761262991	0.773742494322337	0.692428303594618\\
1.14	0.379368276771857	0.784102202600442	0.687549132887069\\
1.14	0.392393618241769	0.794238304822092	0.682943679235329\\
1.14	0.405569970164763	0.804147337016097	0.678558662193046\\
1.14	0.418891418885081	0.813826036051075	0.674343787412872\\
1.14	0.432351961217133	0.823271342915545	0.670253187715552\\
1.14	0.445945513182735	0.832480405557787	0.666246357336156\\
1.14	0.459665918841652	0.841450581281375	0.662288623474185\\
1.14	0.473506959189401	0.850179438693979	0.658351221673193\\
1.14	0.487462361096317	0.85866475920876	0.654411055137886\\
1.14	0.501525806262096	0.866904538099407	0.650450223206467\\
1.14	0.515690940160279	0.874896985111485	0.646455401874959\\
1.14	0.529951380947623	0.882640524634437	0.642417151055155\\
1.14	0.544300728313782	0.890133795440129	0.638329210947585\\
1.14	0.558732572247415	0.897375649995373	0.634187835387499\\
1.14	0.57324050169555	0.904365153357279	0.629991194998376\\
1.14	0.587818113093873	0.911101581661729	0.625738868900923\\
1.14	0.602459018746568	0.917584420216504	0.62143143163314\\
1.14	0.617156855035314	0.923813361211857	0.617070132480554\\
1.14	0.631905290438127	0.92978830106243	0.612656657833838\\
1.14	0.6466980333399	0.935509337395467	0.608192963377087\\
1.14	0.661528839617671	0.940976765701173	0.603681161495104\\
1.14	0.676391519984904	0.94619107566195	0.599123449738607\\
1.14	0.691279947080357	0.951152947177946	0.594522067901592\\
1.14	0.706188062288412	0.955863246106974	0.589879273662389\\
1.14	0.721109882279076	0.960323019737426	0.585197329319936\\
1.14	0.736039505257225	0.964533492013186	0.580478494543978\\
1.14	0.75097111691199	0.968496058529893	0.575725022018313\\
1.14	0.765898996058536	0.972212281322139	0.570939154289558\\
1.14	0.780817519965826	0.975683883461272	0.566123121053175\\
1.14	0.795721169365275	0.97891274348356	0.561279136605558\\
1.14	0.810604533136476	0.981900889668376	0.556409397399198\\
1.14	0.825462312667457	0.984650494185936	0.551516079695736\\
1.14	0.840289325888138	0.987163867133889	0.546601337319641\\
1.14	0.855080510976839	0.989443450481768	0.541667299515139\\
1.14	0.86983092974082	0.991491811941914	0.536716068908599\\
1.14	0.884535770672926	0.993311638785082	0.531749719578141\\
1.14	0.899190351687411	0.994905731618374	0.526770295231796\\
1.14	0.913790122539012	0.996276998142666	0.521779807495104\\
1.14	0.928330666930242	0.997428446906011	0.516780234308682\\
1.14	0.942807704312699	0.998363181068902	0.511773518435872\\
1.14	0.95721709138901	0.999084392196567	0.506761566080168\\
1.14	0.971554823322703	0.999595354092743	0.501746245611875\\
1.14	0.985817034663989	0.999899416688637	0.496729386403004\\
1.14	1	1	0.491669087346906\\
1.155	0	0	0.710195569525497\\
1.155	0.000100583311362513	0.0141829653360114	0.71240460418299\\
1.155	0.000404645907256436	0.0284451766772965	0.714630690536905\\
1.155	0.000915607803433	0.0427829086109896	0.716916312064106\\
1.155	0.00163681893109844	0.057192295687301	0.719289019360234\\
1.155	0.00257155309398959	0.0716693330697585	0.721780367061016\\
1.155	0.00372300185733414	0.086209877460988	0.724425655492835\\
1.155	0.00509426838162598	0.100809648312589	0.727263413850304\\
1.155	0.00668836121491816	0.115464229327074	0.73033458997498\\
1.155	0.00850818805808555	0.13016907025918	0.733681422514709\\
1.155	0.0105565495182326	0.144919489023162	0.737345986077549\\
1.155	0.0128361328661109	0.159710674111862	0.741368418705642\\
1.155	0.0153495058140643	0.174537687332543	0.745784862958201\\
1.155	0.0180991103316243	0.189395466863524	0.750625176092311\\
1.155	0.0210872565164405	0.204278830634725	0.755910489868999\\
1.155	0.0243161165387281	0.219182480034174	0.761650724664453\\
1.155	0.0277877186778607	0.234101003941464	0.767842183867616\\
1.155	0.0315039414701067	0.24902888308801	0.77446537092041\\
1.155	0.0354665079868145	0.263960494742775	0.781483180780606\\
1.155	0.0396769802625738	0.278890117720924	0.788839618268074\\
1.155	0.0441367538930258	0.293811937711588	0.796459186325544\\
1.155	0.0488470528220538	0.308720052919643	0.804247066923454\\
1.155	0.0538089243380495	0.323608480015097	0.81209018616195\\
1.155	0.0590232342988274	0.338471160382329	0.819859213932694\\
1.155	0.064490662604533	0.3533019666601	0.827411499063639\\
1.155	0.0702116989375697	0.368094709561873	0.834594885813503\\
1.155	0.0761866387881432	0.382843144964686	0.841252300287178\\
1.155	0.0824155797834956	0.397540981253432	0.847226939729081\\
1.155	0.0888984183382709	0.412181886906127	0.852367847905043\\
1.155	0.0956348466427212	0.42675949830445	0.856535620027828\\
1.155	0.102624350004627	0.441267427752584	0.859607954619617\\
1.155	0.109866204559871	0.455699271686219	0.861484760266622\\
1.155	0.117359475365564	0.470048619052377	0.862092534251541\\
1.155	0.125103014888515	0.484309059839721	0.861387758068365\\
1.155	0.133095461900593	0.498474193737904	0.85935910089082\\
1.155	0.14133524079124	0.512537638903683	0.856028283765737\\
1.155	0.149820561306021	0.526493040810599	0.851449530870183\\
1.155	0.158549418718625	0.540334081158348	0.845707614739506\\
1.155	0.167519594442213	0.554054486817265	0.838914584343152\\
1.155	0.176728657084455	0.567648038782867	0.83120534237084\\
1.155	0.186173963948925	0.581108581114919	0.822732305417528\\
1.155	0.195852662983903	0.594430029835237	0.813659432944308\\
1.155	0.205761695177907	0.607606381758231	0.804155944104867\\
1.155	0.215897797399558	0.620631723228143	0.794390053406969\\
1.155	0.226257505677663	0.633500238737009	0.7845230460628\\
1.155	0.236837158915675	0.646206219397571	0.774703982861589\\
1.155	0.247632903032949	0.658744071245709	0.765065275211016\\
1.155	0.258640695523528	0.671108323347402	0.755719307772964\\
1.155	0.269856310421543	0.683293635685828	0.746756214005914\\
1.155	0.28127534366066	0.695294806804925	0.738242834604294\\
1.155	0.292893218813452	0.707106781186547	0.730222816007835\\
1.155	0.304705193195076	0.71872465633934	0.722717741087767\\
1.155	0.316706364314172	0.730143689578457	0.715729131143266\\
1.155	0.328891676652598	0.741359304476472	0.709241120561857\\
1.155	0.341255928754291	0.752367096967051	0.703223584571265\\
1.155	0.353793780602429	0.763162841084325	0.697635496597417\\
1.155	0.366499761262991	0.773742494322337	0.692428303594618\\
1.155	0.379368276771857	0.784102202600442	0.687549132887069\\
1.155	0.392393618241769	0.794238304822092	0.682943679235329\\
1.155	0.405569970164763	0.804147337016097	0.678558662193046\\
1.155	0.418891418885081	0.813826036051075	0.674343787412872\\
1.155	0.432351961217133	0.823271342915545	0.670253187715552\\
1.155	0.445945513182735	0.832480405557787	0.666246357336158\\
1.155	0.459665918841652	0.841450581281375	0.662288623474184\\
1.155	0.473506959189401	0.850179438693979	0.658351221673192\\
1.155	0.487462361096317	0.85866475920876	0.654411055137886\\
1.155	0.501525806262096	0.866904538099407	0.650450223206467\\
1.155	0.515690940160279	0.874896985111485	0.646455401874959\\
1.155	0.529951380947623	0.882640524634437	0.642417151055156\\
1.155	0.544300728313782	0.890133795440129	0.638329210947586\\
1.155	0.558732572247415	0.897375649995373	0.634187835387498\\
1.155	0.57324050169555	0.904365153357279	0.629991194998376\\
1.155	0.587818113093873	0.911101581661729	0.625738868900924\\
1.155	0.602459018746568	0.917584420216504	0.62143143163314\\
1.155	0.617156855035314	0.923813361211857	0.617070132480554\\
1.155	0.631905290438127	0.92978830106243	0.612656657833838\\
1.155	0.6466980333399	0.935509337395467	0.608192963377087\\
1.155	0.661528839617671	0.940976765701173	0.603681161495105\\
1.155	0.676391519984904	0.946191075661951	0.599123449738606\\
1.155	0.691279947080357	0.951152947177946	0.59452206790159\\
1.155	0.706188062288412	0.955863246106974	0.589879273662389\\
1.155	0.721109882279076	0.960323019737426	0.585197329319936\\
1.155	0.736039505257225	0.964533492013186	0.580478494543978\\
1.155	0.75097111691199	0.968496058529893	0.575725022018313\\
1.155	0.765898996058536	0.972212281322139	0.570939154289557\\
1.155	0.780817519965826	0.975683883461272	0.566123121053175\\
1.155	0.795721169365275	0.97891274348356	0.561279136605558\\
1.155	0.810604533136476	0.981900889668376	0.556409397399198\\
1.155	0.825462312667457	0.984650494185936	0.551516079695737\\
1.155	0.840289325888138	0.987163867133889	0.546601337319641\\
1.155	0.855080510976839	0.989443450481768	0.541667299515139\\
1.155	0.86983092974082	0.991491811941914	0.536716068908598\\
1.155	0.884535770672926	0.993311638785082	0.531749719578141\\
1.155	0.899190351687411	0.994905731618374	0.526770295231796\\
1.155	0.913790122539012	0.996276998142666	0.521779807495105\\
1.155	0.928330666930242	0.997428446906011	0.516780234308684\\
1.155	0.942807704312699	0.998363181068902	0.51177351843587\\
1.155	0.95721709138901	0.999084392196567	0.506761566080167\\
1.155	0.971554823322703	0.999595354092743	0.501746245611876\\
1.155	0.985817034663989	0.999899416688637	0.496729386403005\\
1.155	1	1	0.491669087346904\\
1.17	0	0	0.710195569525497\\
1.17	0.000100583311362513	0.0141829653360114	0.71240460418299\\
1.17	0.000404645907256436	0.0284451766772965	0.714630690536905\\
1.17	0.000915607803433	0.0427829086109896	0.716916312064106\\
1.17	0.00163681893109844	0.057192295687301	0.719289019360234\\
1.17	0.00257155309398959	0.0716693330697585	0.721780367061016\\
1.17	0.00372300185733414	0.086209877460988	0.724425655492835\\
1.17	0.00509426838162598	0.100809648312589	0.727263413850304\\
1.17	0.00668836121491816	0.115464229327074	0.73033458997498\\
1.17	0.00850818805808555	0.13016907025918	0.733681422514709\\
1.17	0.0105565495182326	0.144919489023162	0.737345986077549\\
1.17	0.0128361328661109	0.159710674111862	0.741368418705642\\
1.17	0.0153495058140643	0.174537687332543	0.745784862958201\\
1.17	0.0180991103316243	0.189395466863524	0.750625176092311\\
1.17	0.0210872565164405	0.204278830634725	0.755910489868998\\
1.17	0.0243161165387281	0.219182480034174	0.761650724664453\\
1.17	0.0277877186778607	0.234101003941464	0.767842183867616\\
1.17	0.0315039414701067	0.24902888308801	0.77446537092041\\
1.17	0.0354665079868145	0.263960494742775	0.781483180780606\\
1.17	0.0396769802625738	0.278890117720924	0.788839618268073\\
1.17	0.0441367538930258	0.293811937711588	0.796459186325544\\
1.17	0.0488470528220538	0.308720052919643	0.804247066923454\\
1.17	0.0538089243380495	0.323608480015096	0.81209018616195\\
1.17	0.0590232342988274	0.338471160382329	0.819859213932694\\
1.17	0.064490662604533	0.3533019666601	0.82741149906364\\
1.17	0.0702116989375697	0.368094709561873	0.834594885813505\\
1.17	0.0761866387881432	0.382843144964686	0.841252300287177\\
1.17	0.0824155797834956	0.397540981253432	0.847226939729081\\
1.17	0.0888984183382709	0.412181886906127	0.852367847905041\\
1.17	0.0956348466427213	0.42675949830445	0.85653562002783\\
1.17	0.102624350004627	0.441267427752585	0.859607954619617\\
1.17	0.109866204559871	0.455699271686219	0.861484760266622\\
1.17	0.117359475365564	0.470048619052377	0.86209253425154\\
1.17	0.125103014888515	0.484309059839721	0.861387758068367\\
1.17	0.133095461900593	0.498474193737904	0.859359100890824\\
1.17	0.14133524079124	0.512537638903683	0.856028283765736\\
1.17	0.149820561306021	0.526493040810599	0.851449530870182\\
1.17	0.158549418718625	0.540334081158348	0.845707614739506\\
1.17	0.167519594442214	0.554054486817265	0.838914584343151\\
1.17	0.176728657084455	0.567648038782868	0.831205342370836\\
1.17	0.186173963948925	0.581108581114919	0.822732305417525\\
1.17	0.195852662983903	0.594430029835237	0.813659432944309\\
1.17	0.205761695177907	0.607606381758231	0.804155944104868\\
1.17	0.215897797399558	0.620631723228143	0.794390053406971\\
1.17	0.226257505677663	0.633500238737009	0.7845230460628\\
1.17	0.236837158915675	0.646206219397571	0.774703982861589\\
1.17	0.247632903032949	0.658744071245709	0.765065275211016\\
1.17	0.258640695523528	0.671108323347402	0.755719307772964\\
1.17	0.269856310421543	0.683293635685828	0.746756214005914\\
1.17	0.28127534366066	0.695294806804925	0.738242834604294\\
1.17	0.292893218813452	0.707106781186547	0.730222816007836\\
1.17	0.304705193195075	0.71872465633934	0.722717741087767\\
1.17	0.316706364314172	0.730143689578457	0.715729131143265\\
1.17	0.328891676652598	0.741359304476472	0.709241120561857\\
1.17	0.341255928754291	0.752367096967051	0.703223584571265\\
1.17	0.353793780602429	0.763162841084325	0.697635496597417\\
1.17	0.366499761262991	0.773742494322337	0.692428303594618\\
1.17	0.379368276771857	0.784102202600442	0.687549132887069\\
1.17	0.392393618241769	0.794238304822092	0.682943679235329\\
1.17	0.405569970164763	0.804147337016097	0.678558662193047\\
1.17	0.418891418885081	0.813826036051075	0.674343787412872\\
1.17	0.432351961217133	0.823271342915545	0.670253187715552\\
1.17	0.445945513182735	0.832480405557787	0.666246357336158\\
1.17	0.459665918841652	0.841450581281375	0.662288623474184\\
1.17	0.473506959189401	0.850179438693979	0.658351221673192\\
1.17	0.487462361096317	0.85866475920876	0.654411055137887\\
1.17	0.501525806262096	0.866904538099407	0.650450223206467\\
1.17	0.515690940160279	0.874896985111485	0.646455401874959\\
1.17	0.529951380947623	0.882640524634437	0.642417151055155\\
1.17	0.544300728313782	0.890133795440129	0.638329210947586\\
1.17	0.558732572247415	0.897375649995373	0.634187835387499\\
1.17	0.57324050169555	0.904365153357279	0.629991194998376\\
1.17	0.587818113093873	0.911101581661729	0.625738868900924\\
1.17	0.602459018746568	0.917584420216504	0.62143143163314\\
1.17	0.617156855035314	0.923813361211857	0.617070132480554\\
1.17	0.631905290438127	0.92978830106243	0.612656657833838\\
1.17	0.6466980333399	0.935509337395467	0.608192963377087\\
1.17	0.661528839617671	0.940976765701173	0.603681161495105\\
1.17	0.676391519984904	0.946191075661951	0.599123449738607\\
1.17	0.691279947080357	0.951152947177946	0.59452206790159\\
1.17	0.706188062288412	0.955863246106974	0.589879273662389\\
1.17	0.721109882279076	0.960323019737426	0.585197329319936\\
1.17	0.736039505257225	0.964533492013186	0.580478494543978\\
1.17	0.75097111691199	0.968496058529893	0.575725022018315\\
1.17	0.765898996058536	0.972212281322139	0.570939154289558\\
1.17	0.780817519965826	0.975683883461272	0.566123121053174\\
1.17	0.795721169365275	0.97891274348356	0.561279136605558\\
1.17	0.810604533136476	0.981900889668376	0.556409397399196\\
1.17	0.825462312667457	0.984650494185936	0.551516079695736\\
1.17	0.840289325888138	0.987163867133889	0.546601337319642\\
1.17	0.855080510976839	0.989443450481768	0.541667299515139\\
1.17	0.86983092974082	0.991491811941914	0.536716068908597\\
1.17	0.884535770672926	0.993311638785082	0.531749719578141\\
1.17	0.899190351687411	0.994905731618374	0.526770295231795\\
1.17	0.913790122539012	0.996276998142666	0.521779807495104\\
1.17	0.928330666930242	0.997428446906011	0.516780234308684\\
1.17	0.942807704312699	0.998363181068902	0.51177351843587\\
1.17	0.95721709138901	0.999084392196567	0.506761566080167\\
1.17	0.971554823322703	0.999595354092743	0.501746245611876\\
1.17	0.985817034663989	0.999899416688637	0.496729386403004\\
1.17	1	1	0.491669087346906\\
1.185	0	0	0.710195569525497\\
1.185	0.000100583311362513	0.0141829653360114	0.71240460418299\\
1.185	0.000404645907256436	0.0284451766772965	0.714630690536905\\
1.185	0.000915607803433	0.0427829086109896	0.716916312064106\\
1.185	0.00163681893109844	0.057192295687301	0.719289019360234\\
1.185	0.00257155309398959	0.0716693330697585	0.721780367061016\\
1.185	0.00372300185733414	0.086209877460988	0.724425655492835\\
1.185	0.00509426838162598	0.100809648312589	0.727263413850304\\
1.185	0.00668836121491816	0.115464229327074	0.73033458997498\\
1.185	0.00850818805808555	0.13016907025918	0.733681422514709\\
1.185	0.0105565495182326	0.144919489023162	0.737345986077549\\
1.185	0.0128361328661109	0.159710674111862	0.741368418705642\\
1.185	0.0153495058140643	0.174537687332543	0.745784862958202\\
1.185	0.0180991103316243	0.189395466863524	0.750625176092311\\
1.185	0.0210872565164405	0.204278830634725	0.755910489868998\\
1.185	0.0243161165387281	0.219182480034174	0.761650724664453\\
1.185	0.0277877186778607	0.234101003941464	0.767842183867616\\
1.185	0.0315039414701067	0.24902888308801	0.77446537092041\\
1.185	0.0354665079868145	0.263960494742775	0.781483180780606\\
1.185	0.0396769802625738	0.278890117720924	0.788839618268073\\
1.185	0.0441367538930258	0.293811937711588	0.796459186325544\\
1.185	0.0488470528220538	0.308720052919643	0.804247066923454\\
1.185	0.0538089243380495	0.323608480015096	0.812090186161949\\
1.185	0.0590232342988274	0.338471160382329	0.819859213932694\\
1.185	0.064490662604533	0.3533019666601	0.827411499063639\\
1.185	0.0702116989375697	0.368094709561873	0.834594885813507\\
1.185	0.0761866387881432	0.382843144964686	0.84125230028718\\
1.185	0.0824155797834956	0.397540981253432	0.847226939729081\\
1.185	0.0888984183382709	0.412181886906127	0.85236784790504\\
1.185	0.0956348466427212	0.42675949830445	0.856535620027831\\
1.185	0.102624350004627	0.441267427752585	0.859607954619617\\
1.185	0.109866204559871	0.455699271686219	0.861484760266622\\
1.185	0.117359475365564	0.470048619052377	0.86209253425154\\
1.185	0.125103014888515	0.484309059839721	0.861387758068372\\
1.185	0.133095461900593	0.498474193737904	0.85935910089082\\
1.185	0.14133524079124	0.512537638903683	0.856028283765734\\
1.185	0.149820561306021	0.526493040810599	0.851449530870183\\
1.185	0.158549418718625	0.540334081158348	0.845707614739505\\
1.185	0.167519594442214	0.554054486817265	0.838914584343148\\
1.185	0.176728657084455	0.567648038782867	0.831205342370833\\
1.185	0.186173963948925	0.581108581114919	0.82273230541753\\
1.185	0.195852662983903	0.594430029835237	0.813659432944309\\
1.185	0.205761695177907	0.607606381758231	0.804155944104868\\
1.185	0.215897797399558	0.620631723228143	0.794390053406969\\
1.185	0.226257505677663	0.633500238737009	0.7845230460628\\
1.185	0.236837158915675	0.646206219397571	0.774703982861589\\
1.185	0.247632903032949	0.658744071245709	0.765065275211016\\
1.185	0.258640695523528	0.671108323347402	0.755719307772964\\
1.185	0.269856310421543	0.683293635685828	0.746756214005914\\
1.185	0.28127534366066	0.695294806804925	0.738242834604294\\
1.185	0.292893218813452	0.707106781186547	0.730222816007836\\
1.185	0.304705193195075	0.71872465633934	0.722717741087767\\
1.185	0.316706364314172	0.730143689578457	0.715729131143265\\
1.185	0.328891676652598	0.741359304476472	0.709241120561857\\
1.185	0.341255928754291	0.752367096967051	0.703223584571265\\
1.185	0.353793780602429	0.763162841084325	0.697635496597417\\
1.185	0.366499761262991	0.773742494322337	0.692428303594618\\
1.185	0.379368276771857	0.784102202600442	0.687549132887069\\
1.185	0.392393618241769	0.794238304822092	0.682943679235329\\
1.185	0.405569970164763	0.804147337016097	0.678558662193046\\
1.185	0.418891418885081	0.813826036051075	0.674343787412872\\
1.185	0.432351961217133	0.823271342915545	0.670253187715552\\
1.185	0.445945513182735	0.832480405557787	0.666246357336158\\
1.185	0.459665918841652	0.841450581281375	0.662288623474184\\
1.185	0.473506959189401	0.850179438693979	0.658351221673191\\
1.185	0.487462361096317	0.85866475920876	0.654411055137887\\
1.185	0.501525806262096	0.866904538099407	0.650450223206467\\
1.185	0.515690940160279	0.874896985111485	0.646455401874959\\
1.185	0.529951380947623	0.882640524634437	0.642417151055155\\
1.185	0.544300728313782	0.890133795440129	0.638329210947585\\
1.185	0.558732572247415	0.897375649995373	0.634187835387498\\
1.185	0.57324050169555	0.904365153357279	0.629991194998376\\
1.185	0.587818113093873	0.911101581661729	0.625738868900924\\
1.185	0.602459018746568	0.917584420216504	0.62143143163314\\
1.185	0.617156855035314	0.923813361211857	0.617070132480554\\
1.185	0.631905290438127	0.92978830106243	0.612656657833838\\
1.185	0.6466980333399	0.935509337395467	0.608192963377087\\
1.185	0.661528839617671	0.940976765701173	0.603681161495105\\
1.185	0.676391519984904	0.946191075661951	0.599123449738607\\
1.185	0.691279947080357	0.951152947177946	0.59452206790159\\
1.185	0.706188062288412	0.955863246106974	0.589879273662389\\
1.185	0.721109882279076	0.960323019737426	0.585197329319936\\
1.185	0.736039505257225	0.964533492013186	0.580478494543978\\
1.185	0.75097111691199	0.968496058529893	0.575725022018312\\
1.185	0.765898996058536	0.972212281322139	0.570939154289558\\
1.185	0.780817519965826	0.975683883461272	0.566123121053177\\
1.185	0.795721169365275	0.97891274348356	0.561279136605558\\
1.185	0.810604533136476	0.981900889668376	0.556409397399196\\
1.185	0.825462312667457	0.984650494185936	0.551516079695735\\
1.185	0.840289325888138	0.987163867133889	0.546601337319642\\
1.185	0.855080510976839	0.989443450481768	0.541667299515141\\
1.185	0.86983092974082	0.991491811941915	0.536716068908598\\
1.185	0.884535770672926	0.993311638785082	0.53174971957814\\
1.185	0.899190351687411	0.994905731618374	0.526770295231796\\
1.185	0.913790122539012	0.996276998142666	0.521779807495102\\
1.185	0.928330666930242	0.997428446906011	0.516780234308684\\
1.185	0.942807704312699	0.998363181068902	0.511773518435872\\
1.185	0.95721709138901	0.999084392196567	0.506761566080167\\
1.185	0.971554823322703	0.999595354092743	0.501746245611874\\
1.185	0.985817034663989	0.999899416688637	0.496729386403004\\
1.185	1	1	0.491669087346906\\
1.2	0	0	0.710195569525497\\
1.2	0.000100583311362513	0.0141829653360114	0.71240460418299\\
1.2	0.000404645907256436	0.0284451766772965	0.714630690536905\\
1.2	0.000915607803433	0.0427829086109896	0.716916312064106\\
1.2	0.00163681893109844	0.057192295687301	0.719289019360234\\
1.2	0.00257155309398959	0.0716693330697585	0.721780367061016\\
1.2	0.00372300185733414	0.086209877460988	0.724425655492835\\
1.2	0.00509426838162598	0.100809648312589	0.727263413850304\\
1.2	0.00668836121491816	0.115464229327074	0.73033458997498\\
1.2	0.00850818805808555	0.13016907025918	0.733681422514709\\
1.2	0.0105565495182326	0.144919489023162	0.737345986077549\\
1.2	0.0128361328661109	0.159710674111862	0.741368418705642\\
1.2	0.0153495058140643	0.174537687332543	0.745784862958202\\
1.2	0.0180991103316243	0.189395466863524	0.750625176092311\\
1.2	0.0210872565164405	0.204278830634725	0.755910489868998\\
1.2	0.0243161165387281	0.219182480034174	0.761650724664453\\
1.2	0.0277877186778607	0.234101003941464	0.767842183867616\\
1.2	0.0315039414701067	0.24902888308801	0.77446537092041\\
1.2	0.0354665079868145	0.263960494742775	0.781483180780606\\
1.2	0.0396769802625737	0.278890117720924	0.788839618268073\\
1.2	0.0441367538930258	0.293811937711588	0.796459186325544\\
1.2	0.0488470528220538	0.308720052919643	0.804247066923455\\
1.2	0.0538089243380495	0.323608480015096	0.81209018616195\\
1.2	0.0590232342988274	0.338471160382329	0.819859213932694\\
1.2	0.064490662604533	0.3533019666601	0.82741149906364\\
1.2	0.0702116989375697	0.368094709561873	0.834594885813505\\
1.2	0.0761866387881432	0.382843144964686	0.84125230028718\\
1.2	0.0824155797834956	0.397540981253432	0.847226939729081\\
1.2	0.0888984183382709	0.412181886906127	0.852367847905041\\
1.2	0.0956348466427212	0.42675949830445	0.856535620027829\\
1.2	0.102624350004627	0.441267427752584	0.859607954619617\\
1.2	0.109866204559871	0.455699271686219	0.86148476026662\\
1.2	0.117359475365564	0.470048619052377	0.862092534251541\\
1.2	0.125103014888515	0.484309059839721	0.861387758068369\\
1.2	0.133095461900593	0.498474193737904	0.859359100890818\\
1.2	0.14133524079124	0.512537638903683	0.856028283765736\\
1.2	0.149820561306021	0.526493040810599	0.851449530870183\\
1.2	0.158549418718625	0.540334081158348	0.845707614739502\\
1.2	0.167519594442214	0.554054486817265	0.83891458434315\\
1.2	0.176728657084455	0.567648038782867	0.831205342370837\\
1.2	0.186173963948925	0.581108581114919	0.822732305417528\\
1.2	0.195852662983903	0.594430029835237	0.813659432944308\\
1.2	0.205761695177907	0.607606381758231	0.804155944104868\\
1.2	0.215897797399558	0.620631723228143	0.794390053406969\\
1.2	0.226257505677663	0.633500238737009	0.7845230460628\\
1.2	0.236837158915675	0.646206219397571	0.774703982861589\\
1.2	0.247632903032949	0.658744071245709	0.765065275211016\\
1.2	0.258640695523528	0.671108323347402	0.755719307772964\\
1.2	0.269856310421543	0.683293635685828	0.746756214005914\\
1.2	0.28127534366066	0.695294806804925	0.738242834604294\\
1.2	0.292893218813452	0.707106781186547	0.730222816007836\\
1.2	0.304705193195075	0.71872465633934	0.722717741087767\\
1.2	0.316706364314172	0.730143689578457	0.715729131143265\\
1.2	0.328891676652598	0.741359304476472	0.709241120561857\\
1.2	0.341255928754291	0.752367096967051	0.703223584571265\\
1.2	0.353793780602429	0.763162841084325	0.697635496597417\\
1.2	0.366499761262991	0.773742494322337	0.692428303594618\\
1.2	0.379368276771857	0.784102202600442	0.68754913288707\\
1.2	0.392393618241769	0.794238304822092	0.682943679235329\\
1.2	0.405569970164763	0.804147337016097	0.678558662193046\\
1.2	0.418891418885081	0.813826036051075	0.674343787412872\\
1.2	0.432351961217133	0.823271342915545	0.670253187715552\\
1.2	0.445945513182735	0.832480405557787	0.666246357336158\\
1.2	0.459665918841652	0.841450581281375	0.662288623474185\\
1.2	0.473506959189401	0.850179438693979	0.658351221673191\\
1.2	0.487462361096317	0.85866475920876	0.654411055137886\\
1.2	0.501525806262096	0.866904538099407	0.650450223206468\\
1.2	0.515690940160279	0.874896985111485	0.646455401874959\\
1.2	0.529951380947623	0.882640524634437	0.642417151055155\\
1.2	0.544300728313782	0.890133795440129	0.638329210947585\\
1.2	0.558732572247415	0.897375649995373	0.634187835387498\\
1.2	0.57324050169555	0.904365153357279	0.629991194998376\\
1.2	0.587818113093873	0.911101581661729	0.625738868900924\\
1.2	0.602459018746568	0.917584420216504	0.62143143163314\\
1.2	0.617156855035314	0.923813361211857	0.617070132480554\\
1.2	0.631905290438127	0.92978830106243	0.612656657833838\\
1.2	0.6466980333399	0.935509337395467	0.608192963377087\\
1.2	0.661528839617671	0.940976765701173	0.603681161495105\\
1.2	0.676391519984904	0.946191075661951	0.599123449738607\\
1.2	0.691279947080357	0.951152947177946	0.59452206790159\\
1.2	0.706188062288412	0.955863246106974	0.589879273662389\\
1.2	0.721109882279076	0.960323019737426	0.585197329319936\\
1.2	0.736039505257225	0.964533492013186	0.580478494543978\\
1.2	0.75097111691199	0.968496058529893	0.575725022018314\\
1.2	0.765898996058536	0.972212281322139	0.570939154289557\\
1.2	0.780817519965826	0.975683883461272	0.566123121053175\\
1.2	0.795721169365275	0.97891274348356	0.561279136605558\\
1.2	0.810604533136476	0.981900889668376	0.556409397399198\\
1.2	0.825462312667457	0.984650494185936	0.551516079695735\\
1.2	0.840289325888138	0.987163867133889	0.546601337319641\\
1.2	0.855080510976839	0.989443450481768	0.541667299515139\\
1.2	0.86983092974082	0.991491811941914	0.5367160689086\\
1.2	0.884535770672926	0.993311638785082	0.531749719578141\\
1.2	0.899190351687411	0.994905731618374	0.526770295231796\\
1.2	0.913790122539012	0.996276998142666	0.521779807495105\\
1.2	0.928330666930242	0.997428446906011	0.516780234308682\\
1.2	0.942807704312699	0.998363181068902	0.511773518435871\\
1.2	0.95721709138901	0.999084392196567	0.50676156608017\\
1.2	0.971554823322704	0.999595354092744	0.501746245611874\\
1.2	0.985817034663989	0.999899416688637	0.496729386403002\\
1.2	1	1	0.491669087346908\\
1.215	0	0	0.710195569525497\\
1.215	0.000100583311362513	0.0141829653360114	0.71240460418299\\
1.215	0.000404645907256436	0.0284451766772965	0.714630690536905\\
1.215	0.000915607803433	0.0427829086109896	0.716916312064106\\
1.215	0.00163681893109844	0.057192295687301	0.719289019360234\\
1.215	0.00257155309398959	0.0716693330697585	0.721780367061016\\
1.215	0.00372300185733414	0.0862098774609879	0.724425655492835\\
1.215	0.00509426838162598	0.100809648312589	0.727263413850304\\
1.215	0.00668836121491816	0.115464229327074	0.73033458997498\\
1.215	0.00850818805808555	0.13016907025918	0.733681422514709\\
1.215	0.0105565495182326	0.144919489023162	0.737345986077549\\
1.215	0.0128361328661109	0.159710674111862	0.741368418705642\\
1.215	0.0153495058140643	0.174537687332543	0.745784862958201\\
1.215	0.0180991103316243	0.189395466863524	0.750625176092311\\
1.215	0.0210872565164405	0.204278830634725	0.755910489868998\\
1.215	0.0243161165387281	0.219182480034174	0.761650724664453\\
1.215	0.0277877186778607	0.234101003941464	0.767842183867616\\
1.215	0.0315039414701067	0.24902888308801	0.77446537092041\\
1.215	0.0354665079868145	0.263960494742775	0.781483180780606\\
1.215	0.0396769802625738	0.278890117720924	0.788839618268073\\
1.215	0.0441367538930258	0.293811937711588	0.796459186325544\\
1.215	0.0488470528220538	0.308720052919643	0.804247066923454\\
1.215	0.0538089243380495	0.323608480015096	0.81209018616195\\
1.215	0.0590232342988274	0.338471160382329	0.819859213932694\\
1.215	0.064490662604533	0.3533019666601	0.82741149906364\\
1.215	0.0702116989375697	0.368094709561873	0.834594885813505\\
1.215	0.0761866387881432	0.382843144964686	0.841252300287179\\
1.215	0.0824155797834956	0.397540981253432	0.847226939729081\\
1.215	0.0888984183382709	0.412181886906127	0.85236784790504\\
1.215	0.0956348466427213	0.42675949830445	0.85653562002783\\
1.215	0.102624350004627	0.441267427752584	0.859607954619619\\
1.215	0.109866204559871	0.455699271686219	0.861484760266621\\
1.215	0.117359475365564	0.470048619052377	0.862092534251541\\
1.215	0.125103014888515	0.484309059839721	0.861387758068369\\
1.215	0.133095461900593	0.498474193737904	0.85935910089082\\
1.215	0.14133524079124	0.512537638903683	0.856028283765736\\
1.215	0.149820561306021	0.526493040810599	0.851449530870182\\
1.215	0.158549418718625	0.540334081158348	0.845707614739505\\
1.215	0.167519594442214	0.554054486817265	0.838914584343151\\
1.215	0.176728657084455	0.567648038782867	0.831205342370836\\
1.215	0.186173963948925	0.581108581114919	0.822732305417527\\
1.215	0.195852662983903	0.594430029835237	0.813659432944309\\
1.215	0.205761695177907	0.607606381758231	0.804155944104868\\
1.215	0.215897797399558	0.620631723228143	0.794390053406969\\
1.215	0.226257505677663	0.633500238737009	0.7845230460628\\
1.215	0.236837158915675	0.646206219397571	0.774703982861589\\
1.215	0.247632903032949	0.658744071245709	0.765065275211016\\
1.215	0.258640695523528	0.671108323347402	0.755719307772964\\
1.215	0.269856310421543	0.683293635685828	0.746756214005914\\
1.215	0.28127534366066	0.695294806804925	0.738242834604294\\
1.215	0.292893218813452	0.707106781186547	0.730222816007836\\
1.215	0.304705193195075	0.71872465633934	0.722717741087767\\
1.215	0.316706364314172	0.730143689578457	0.715729131143265\\
1.215	0.328891676652598	0.741359304476472	0.709241120561857\\
1.215	0.341255928754291	0.752367096967051	0.703223584571265\\
1.215	0.353793780602429	0.763162841084325	0.697635496597417\\
1.215	0.366499761262991	0.773742494322337	0.692428303594618\\
1.215	0.379368276771857	0.784102202600442	0.687549132887069\\
1.215	0.392393618241769	0.794238304822092	0.682943679235329\\
1.215	0.405569970164763	0.804147337016097	0.678558662193046\\
1.215	0.418891418885081	0.813826036051075	0.674343787412872\\
1.215	0.432351961217133	0.823271342915545	0.670253187715552\\
1.215	0.445945513182735	0.832480405557787	0.666246357336156\\
1.215	0.459665918841652	0.841450581281375	0.662288623474185\\
1.215	0.473506959189401	0.850179438693979	0.658351221673193\\
1.215	0.487462361096317	0.85866475920876	0.654411055137887\\
1.215	0.501525806262096	0.866904538099407	0.650450223206467\\
1.215	0.515690940160279	0.874896985111485	0.646455401874959\\
1.215	0.529951380947623	0.882640524634437	0.642417151055155\\
1.215	0.544300728313782	0.890133795440129	0.638329210947585\\
1.215	0.558732572247415	0.897375649995373	0.634187835387498\\
1.215	0.57324050169555	0.904365153357279	0.629991194998376\\
1.215	0.587818113093873	0.911101581661729	0.625738868900924\\
1.215	0.602459018746568	0.917584420216504	0.62143143163314\\
1.215	0.617156855035314	0.923813361211857	0.617070132480554\\
1.215	0.631905290438127	0.92978830106243	0.612656657833838\\
1.215	0.6466980333399	0.935509337395467	0.608192963377087\\
1.215	0.661528839617671	0.940976765701173	0.603681161495105\\
1.215	0.676391519984904	0.946191075661951	0.599123449738607\\
1.215	0.691279947080357	0.951152947177946	0.59452206790159\\
1.215	0.706188062288412	0.955863246106974	0.589879273662389\\
1.215	0.721109882279076	0.960323019737426	0.585197329319936\\
1.215	0.736039505257225	0.964533492013186	0.580478494543975\\
1.215	0.75097111691199	0.968496058529893	0.575725022018314\\
1.215	0.765898996058536	0.972212281322139	0.57093915428956\\
1.215	0.780817519965826	0.975683883461272	0.566123121053175\\
1.215	0.795721169365275	0.97891274348356	0.561279136605558\\
1.215	0.810604533136476	0.981900889668376	0.556409397399199\\
1.215	0.825462312667457	0.984650494185936	0.551516079695735\\
1.215	0.840289325888138	0.987163867133889	0.546601337319639\\
1.215	0.855080510976839	0.989443450481768	0.541667299515139\\
1.215	0.86983092974082	0.991491811941914	0.536716068908598\\
1.215	0.884535770672926	0.993311638785082	0.531749719578141\\
1.215	0.899190351687411	0.994905731618374	0.526770295231795\\
1.215	0.913790122539012	0.996276998142666	0.521779807495105\\
1.215	0.928330666930242	0.997428446906011	0.516780234308687\\
1.215	0.942807704312699	0.998363181068902	0.511773518435872\\
1.215	0.95721709138901	0.999084392196567	0.506761566080166\\
1.215	0.971554823322703	0.999595354092743	0.501746245611875\\
1.215	0.985817034663989	0.999899416688637	0.496729386403004\\
1.215	1	1	0.491669087346906\\
1.23	0	0	0.710195569525497\\
1.23	0.000100583311362513	0.0141829653360114	0.71240460418299\\
1.23	0.000404645907256436	0.0284451766772965	0.714630690536905\\
1.23	0.000915607803433	0.0427829086109896	0.716916312064106\\
1.23	0.00163681893109844	0.057192295687301	0.719289019360234\\
1.23	0.00257155309398959	0.0716693330697585	0.721780367061016\\
1.23	0.00372300185733414	0.086209877460988	0.724425655492835\\
1.23	0.00509426838162599	0.100809648312589	0.727263413850304\\
1.23	0.00668836121491816	0.115464229327074	0.73033458997498\\
1.23	0.00850818805808555	0.13016907025918	0.733681422514709\\
1.23	0.0105565495182326	0.144919489023162	0.737345986077549\\
1.23	0.0128361328661109	0.159710674111862	0.741368418705642\\
1.23	0.0153495058140643	0.174537687332543	0.745784862958201\\
1.23	0.0180991103316243	0.189395466863524	0.750625176092311\\
1.23	0.0210872565164405	0.204278830634725	0.755910489868998\\
1.23	0.0243161165387281	0.219182480034174	0.761650724664453\\
1.23	0.0277877186778607	0.234101003941464	0.767842183867616\\
1.23	0.0315039414701067	0.24902888308801	0.77446537092041\\
1.23	0.0354665079868145	0.263960494742775	0.781483180780606\\
1.23	0.0396769802625738	0.278890117720924	0.788839618268073\\
1.23	0.0441367538930258	0.293811937711588	0.796459186325544\\
1.23	0.0488470528220538	0.308720052919643	0.804247066923454\\
1.23	0.0538089243380495	0.323608480015097	0.81209018616195\\
1.23	0.0590232342988274	0.338471160382329	0.819859213932693\\
1.23	0.064490662604533	0.3533019666601	0.827411499063638\\
1.23	0.0702116989375697	0.368094709561873	0.834594885813505\\
1.23	0.0761866387881432	0.382843144964686	0.841252300287179\\
1.23	0.0824155797834956	0.397540981253432	0.847226939729081\\
1.23	0.0888984183382709	0.412181886906127	0.852367847905041\\
1.23	0.0956348466427212	0.42675949830445	0.85653562002783\\
1.23	0.102624350004627	0.441267427752584	0.859607954619619\\
1.23	0.109866204559871	0.455699271686219	0.861484760266622\\
1.23	0.117359475365564	0.470048619052377	0.862092534251539\\
1.23	0.125103014888515	0.484309059839721	0.861387758068368\\
1.23	0.133095461900593	0.498474193737904	0.85935910089082\\
1.23	0.14133524079124	0.512537638903683	0.856028283765738\\
1.23	0.149820561306021	0.526493040810599	0.851449530870183\\
1.23	0.158549418718625	0.540334081158348	0.845707614739508\\
1.23	0.167519594442214	0.554054486817265	0.83891458434315\\
1.23	0.176728657084455	0.567648038782867	0.831205342370836\\
1.23	0.186173963948925	0.581108581114919	0.82273230541753\\
1.23	0.195852662983903	0.594430029835237	0.813659432944309\\
1.23	0.205761695177907	0.607606381758231	0.804155944104868\\
1.23	0.215897797399558	0.620631723228143	0.79439005340697\\
1.23	0.226257505677663	0.633500238737009	0.7845230460628\\
1.23	0.236837158915675	0.646206219397571	0.774703982861589\\
1.23	0.247632903032949	0.658744071245709	0.765065275211016\\
1.23	0.258640695523528	0.671108323347402	0.755719307772964\\
1.23	0.269856310421543	0.683293635685828	0.746756214005914\\
1.23	0.28127534366066	0.695294806804925	0.738242834604294\\
1.23	0.292893218813452	0.707106781186547	0.730222816007836\\
1.23	0.304705193195075	0.71872465633934	0.722717741087767\\
1.23	0.316706364314172	0.730143689578457	0.715729131143265\\
1.23	0.328891676652598	0.741359304476472	0.709241120561857\\
1.23	0.341255928754291	0.752367096967051	0.703223584571265\\
1.23	0.353793780602429	0.763162841084325	0.697635496597417\\
1.23	0.366499761262991	0.773742494322337	0.692428303594618\\
1.23	0.379368276771857	0.784102202600442	0.68754913288707\\
1.23	0.392393618241769	0.794238304822092	0.682943679235329\\
1.23	0.405569970164763	0.804147337016097	0.678558662193046\\
1.23	0.418891418885081	0.813826036051075	0.674343787412872\\
1.23	0.432351961217133	0.823271342915545	0.670253187715552\\
1.23	0.445945513182735	0.832480405557787	0.666246357336158\\
1.23	0.459665918841652	0.841450581281375	0.662288623474184\\
1.23	0.473506959189401	0.850179438693979	0.65835122167319\\
1.23	0.487462361096317	0.85866475920876	0.654411055137886\\
1.23	0.501525806262096	0.866904538099407	0.650450223206469\\
1.23	0.515690940160279	0.874896985111485	0.646455401874959\\
1.23	0.529951380947623	0.882640524634437	0.642417151055155\\
1.23	0.544300728313782	0.890133795440129	0.638329210947586\\
1.23	0.558732572247415	0.897375649995373	0.634187835387499\\
1.23	0.57324050169555	0.904365153357279	0.629991194998376\\
1.23	0.587818113093873	0.911101581661729	0.625738868900924\\
1.23	0.602459018746568	0.917584420216504	0.62143143163314\\
1.23	0.617156855035314	0.923813361211857	0.617070132480554\\
1.23	0.631905290438127	0.92978830106243	0.612656657833838\\
1.23	0.6466980333399	0.935509337395467	0.608192963377087\\
1.23	0.661528839617671	0.940976765701173	0.603681161495105\\
1.23	0.676391519984904	0.946191075661951	0.599123449738607\\
1.23	0.691279947080357	0.951152947177946	0.59452206790159\\
1.23	0.706188062288412	0.955863246106974	0.589879273662388\\
1.23	0.721109882279076	0.960323019737426	0.585197329319936\\
1.23	0.736039505257225	0.964533492013186	0.580478494543978\\
1.23	0.75097111691199	0.968496058529893	0.575725022018313\\
1.23	0.765898996058536	0.972212281322139	0.570939154289557\\
1.23	0.780817519965826	0.975683883461272	0.566123121053175\\
1.23	0.795721169365275	0.97891274348356	0.561279136605558\\
1.23	0.810604533136476	0.981900889668376	0.556409397399199\\
1.23	0.825462312667457	0.984650494185936	0.551516079695736\\
1.23	0.840289325888138	0.987163867133889	0.546601337319639\\
1.23	0.855080510976839	0.989443450481768	0.541667299515139\\
1.23	0.86983092974082	0.991491811941914	0.536716068908598\\
1.23	0.884535770672926	0.993311638785082	0.531749719578141\\
1.23	0.899190351687411	0.994905731618374	0.526770295231796\\
1.23	0.913790122539012	0.996276998142666	0.521779807495104\\
1.23	0.928330666930242	0.997428446906011	0.516780234308684\\
1.23	0.942807704312699	0.998363181068902	0.511773518435873\\
1.23	0.95721709138901	0.999084392196567	0.506761566080168\\
1.23	0.971554823322703	0.999595354092743	0.501746245611874\\
1.23	0.985817034663989	0.999899416688637	0.496729386403004\\
1.23	1	1	0.491669087346906\\
1.245	0	0	0.710195569525497\\
1.245	0.000100583311362513	0.0141829653360114	0.71240460418299\\
1.245	0.000404645907256436	0.0284451766772965	0.714630690536905\\
1.245	0.000915607803433	0.0427829086109896	0.716916312064106\\
1.245	0.00163681893109844	0.057192295687301	0.719289019360234\\
1.245	0.00257155309398959	0.0716693330697585	0.721780367061016\\
1.245	0.00372300185733414	0.086209877460988	0.724425655492835\\
1.245	0.00509426838162598	0.100809648312589	0.727263413850304\\
1.245	0.00668836121491816	0.115464229327074	0.73033458997498\\
1.245	0.00850818805808555	0.13016907025918	0.733681422514709\\
1.245	0.0105565495182326	0.144919489023162	0.737345986077549\\
1.245	0.0128361328661109	0.159710674111862	0.741368418705642\\
1.245	0.0153495058140643	0.174537687332543	0.745784862958202\\
1.245	0.0180991103316243	0.189395466863524	0.750625176092311\\
1.245	0.0210872565164405	0.204278830634725	0.755910489868998\\
1.245	0.0243161165387281	0.219182480034174	0.761650724664453\\
1.245	0.0277877186778607	0.234101003941464	0.767842183867616\\
1.245	0.0315039414701067	0.24902888308801	0.77446537092041\\
1.245	0.0354665079868145	0.263960494742775	0.781483180780606\\
1.245	0.0396769802625738	0.278890117720924	0.788839618268073\\
1.245	0.0441367538930258	0.293811937711588	0.796459186325544\\
1.245	0.0488470528220537	0.308720052919643	0.804247066923454\\
1.245	0.0538089243380495	0.323608480015096	0.812090186161949\\
1.245	0.0590232342988274	0.338471160382329	0.819859213932694\\
1.245	0.064490662604533	0.3533019666601	0.827411499063638\\
1.245	0.0702116989375697	0.368094709561873	0.834594885813504\\
1.245	0.0761866387881432	0.382843144964686	0.841252300287179\\
1.245	0.0824155797834956	0.397540981253432	0.847226939729081\\
1.245	0.0888984183382709	0.412181886906127	0.852367847905042\\
1.245	0.0956348466427212	0.42675949830445	0.85653562002783\\
1.245	0.102624350004627	0.441267427752584	0.859607954619617\\
1.245	0.109866204559871	0.455699271686219	0.861484760266622\\
1.245	0.117359475365564	0.470048619052377	0.862092534251541\\
1.245	0.125103014888515	0.484309059839721	0.861387758068369\\
1.245	0.133095461900593	0.498474193737904	0.859359100890823\\
1.245	0.14133524079124	0.512537638903683	0.85602828376574\\
1.245	0.149820561306021	0.526493040810599	0.851449530870182\\
1.245	0.158549418718625	0.540334081158348	0.845707614739505\\
1.245	0.167519594442214	0.554054486817265	0.83891458434315\\
1.245	0.176728657084455	0.567648038782867	0.831205342370837\\
1.245	0.186173963948925	0.581108581114919	0.822732305417528\\
1.245	0.195852662983903	0.594430029835237	0.813659432944308\\
1.245	0.205761695177907	0.607606381758231	0.804155944104868\\
1.245	0.215897797399558	0.620631723228143	0.794390053406969\\
1.245	0.226257505677663	0.633500238737009	0.7845230460628\\
1.245	0.236837158915675	0.646206219397571	0.774703982861589\\
1.245	0.247632903032949	0.658744071245709	0.765065275211016\\
1.245	0.258640695523528	0.671108323347402	0.755719307772964\\
1.245	0.269856310421543	0.683293635685828	0.746756214005914\\
1.245	0.28127534366066	0.695294806804925	0.738242834604294\\
1.245	0.292893218813452	0.707106781186547	0.730222816007835\\
1.245	0.304705193195076	0.71872465633934	0.722717741087767\\
1.245	0.316706364314172	0.730143689578457	0.715729131143266\\
1.245	0.328891676652598	0.741359304476472	0.709241120561857\\
1.245	0.341255928754291	0.752367096967051	0.703223584571265\\
1.245	0.353793780602429	0.763162841084325	0.697635496597417\\
1.245	0.366499761262991	0.773742494322337	0.692428303594618\\
1.245	0.379368276771857	0.784102202600442	0.687549132887069\\
1.245	0.392393618241769	0.794238304822092	0.682943679235329\\
1.245	0.405569970164763	0.804147337016097	0.678558662193046\\
1.245	0.418891418885081	0.813826036051075	0.674343787412872\\
1.245	0.432351961217133	0.823271342915545	0.670253187715552\\
1.245	0.445945513182735	0.832480405557787	0.666246357336158\\
1.245	0.459665918841652	0.841450581281375	0.662288623474185\\
1.245	0.473506959189401	0.850179438693979	0.658351221673191\\
1.245	0.487462361096317	0.85866475920876	0.654411055137885\\
1.245	0.501525806262096	0.866904538099407	0.650450223206467\\
1.245	0.515690940160279	0.874896985111485	0.64645540187496\\
1.245	0.529951380947623	0.882640524634437	0.642417151055155\\
1.245	0.544300728313782	0.890133795440129	0.638329210947585\\
1.245	0.558732572247415	0.897375649995373	0.634187835387499\\
1.245	0.57324050169555	0.904365153357279	0.629991194998376\\
1.245	0.587818113093873	0.911101581661729	0.625738868900924\\
1.245	0.602459018746568	0.917584420216504	0.62143143163314\\
1.245	0.617156855035314	0.923813361211857	0.617070132480554\\
1.245	0.631905290438127	0.92978830106243	0.612656657833838\\
1.245	0.6466980333399	0.935509337395467	0.608192963377087\\
1.245	0.661528839617671	0.940976765701173	0.603681161495105\\
1.245	0.676391519984904	0.946191075661951	0.599123449738607\\
1.245	0.691279947080357	0.951152947177946	0.594522067901592\\
1.245	0.706188062288412	0.955863246106974	0.589879273662389\\
1.245	0.721109882279076	0.960323019737426	0.585197329319934\\
1.245	0.736039505257225	0.964533492013186	0.580478494543978\\
1.245	0.75097111691199	0.968496058529893	0.575725022018316\\
1.245	0.765898996058536	0.972212281322139	0.570939154289558\\
1.245	0.780817519965826	0.975683883461272	0.566123121053174\\
1.245	0.795721169365275	0.97891274348356	0.561279136605558\\
1.245	0.810604533136476	0.981900889668376	0.556409397399198\\
1.245	0.825462312667457	0.984650494185936	0.551516079695737\\
1.245	0.840289325888138	0.987163867133889	0.546601337319641\\
1.245	0.855080510976839	0.989443450481768	0.541667299515139\\
1.245	0.86983092974082	0.991491811941914	0.536716068908597\\
1.245	0.884535770672926	0.993311638785082	0.531749719578141\\
1.245	0.899190351687411	0.994905731618374	0.526770295231796\\
1.245	0.913790122539012	0.996276998142666	0.521779807495104\\
1.245	0.928330666930242	0.997428446906011	0.516780234308684\\
1.245	0.942807704312699	0.998363181068902	0.511773518435872\\
1.245	0.95721709138901	0.999084392196567	0.506761566080168\\
1.245	0.971554823322703	0.999595354092743	0.501746245611874\\
1.245	0.985817034663989	0.999899416688637	0.496729386403004\\
1.245	1	1	0.491669087346906\\
1.26	0	0	0.710195569525497\\
1.26	0.000100583311362513	0.0141829653360114	0.71240460418299\\
1.26	0.000404645907256436	0.0284451766772965	0.714630690536905\\
1.26	0.000915607803433	0.0427829086109896	0.716916312064106\\
1.26	0.00163681893109844	0.057192295687301	0.719289019360234\\
1.26	0.00257155309398959	0.0716693330697584	0.721780367061016\\
1.26	0.00372300185733414	0.086209877460988	0.724425655492835\\
1.26	0.00509426838162598	0.100809648312589	0.727263413850304\\
1.26	0.00668836121491816	0.115464229327074	0.73033458997498\\
1.26	0.00850818805808555	0.13016907025918	0.733681422514709\\
1.26	0.0105565495182326	0.144919489023162	0.737345986077549\\
1.26	0.0128361328661109	0.159710674111862	0.741368418705642\\
1.26	0.0153495058140643	0.174537687332543	0.745784862958202\\
1.26	0.0180991103316243	0.189395466863524	0.750625176092311\\
1.26	0.0210872565164405	0.204278830634725	0.755910489868999\\
1.26	0.0243161165387281	0.219182480034174	0.761650724664453\\
1.26	0.0277877186778607	0.234101003941464	0.767842183867616\\
1.26	0.0315039414701067	0.24902888308801	0.77446537092041\\
1.26	0.0354665079868145	0.263960494742775	0.781483180780606\\
1.26	0.0396769802625738	0.278890117720924	0.788839618268074\\
1.26	0.0441367538930258	0.293811937711588	0.796459186325544\\
1.26	0.0488470528220538	0.308720052919643	0.804247066923455\\
1.26	0.0538089243380495	0.323608480015096	0.81209018616195\\
1.26	0.0590232342988274	0.338471160382329	0.819859213932694\\
1.26	0.064490662604533	0.3533019666601	0.827411499063639\\
1.26	0.0702116989375697	0.368094709561873	0.834594885813504\\
1.26	0.0761866387881432	0.382843144964686	0.841252300287178\\
1.26	0.0824155797834956	0.397540981253432	0.847226939729081\\
1.26	0.0888984183382709	0.412181886906127	0.852367847905041\\
1.26	0.0956348466427212	0.42675949830445	0.856535620027829\\
1.26	0.102624350004627	0.441267427752584	0.859607954619618\\
1.26	0.109866204559871	0.455699271686219	0.861484760266619\\
1.26	0.117359475365564	0.470048619052377	0.862092534251541\\
1.26	0.125103014888515	0.484309059839721	0.861387758068368\\
1.26	0.133095461900593	0.498474193737904	0.859359100890821\\
1.26	0.14133524079124	0.512537638903683	0.856028283765737\\
1.26	0.149820561306021	0.526493040810599	0.851449530870182\\
1.26	0.158549418718625	0.540334081158348	0.845707614739505\\
1.26	0.167519594442214	0.554054486817265	0.83891458434315\\
1.26	0.176728657084455	0.567648038782867	0.831205342370839\\
1.26	0.186173963948925	0.581108581114919	0.822732305417527\\
1.26	0.195852662983903	0.594430029835237	0.813659432944309\\
1.26	0.205761695177907	0.607606381758231	0.804155944104868\\
1.26	0.215897797399558	0.620631723228143	0.794390053406969\\
1.26	0.226257505677663	0.633500238737009	0.7845230460628\\
1.26	0.236837158915675	0.646206219397571	0.774703982861589\\
1.26	0.247632903032949	0.658744071245709	0.765065275211016\\
1.26	0.258640695523528	0.671108323347402	0.755719307772965\\
1.26	0.269856310421543	0.683293635685828	0.746756214005914\\
1.26	0.28127534366066	0.695294806804925	0.738242834604294\\
1.26	0.292893218813452	0.707106781186547	0.730222816007836\\
1.26	0.304705193195075	0.71872465633934	0.722717741087767\\
1.26	0.316706364314172	0.730143689578457	0.715729131143265\\
1.26	0.328891676652598	0.741359304476472	0.709241120561856\\
1.26	0.341255928754291	0.752367096967051	0.703223584571265\\
1.26	0.353793780602429	0.763162841084325	0.697635496597419\\
1.26	0.366499761262991	0.773742494322337	0.692428303594618\\
1.26	0.379368276771857	0.784102202600442	0.687549132887069\\
1.26	0.392393618241769	0.794238304822092	0.682943679235329\\
1.26	0.405569970164763	0.804147337016097	0.678558662193046\\
1.26	0.418891418885081	0.813826036051075	0.674343787412872\\
1.26	0.432351961217133	0.823271342915545	0.670253187715552\\
1.26	0.445945513182735	0.832480405557787	0.666246357336158\\
1.26	0.459665918841652	0.841450581281375	0.662288623474185\\
1.26	0.473506959189401	0.850179438693979	0.658351221673191\\
1.26	0.487462361096317	0.85866475920876	0.654411055137886\\
1.26	0.501525806262096	0.866904538099407	0.650450223206467\\
1.26	0.515690940160279	0.874896985111485	0.646455401874959\\
1.26	0.529951380947623	0.882640524634437	0.642417151055155\\
1.26	0.544300728313782	0.890133795440129	0.638329210947585\\
1.26	0.558732572247415	0.897375649995373	0.634187835387498\\
1.26	0.57324050169555	0.904365153357279	0.629991194998376\\
1.26	0.587818113093873	0.911101581661729	0.625738868900924\\
1.26	0.602459018746568	0.917584420216504	0.62143143163314\\
1.26	0.617156855035314	0.923813361211857	0.617070132480554\\
1.26	0.631905290438127	0.92978830106243	0.612656657833838\\
1.26	0.6466980333399	0.935509337395467	0.608192963377087\\
1.26	0.661528839617671	0.940976765701173	0.603681161495105\\
1.26	0.676391519984904	0.946191075661951	0.599123449738607\\
1.26	0.691279947080357	0.951152947177946	0.594522067901589\\
1.26	0.706188062288412	0.955863246106974	0.58987927366239\\
1.26	0.721109882279076	0.960323019737426	0.585197329319937\\
1.26	0.736039505257225	0.964533492013186	0.580478494543975\\
1.26	0.75097111691199	0.968496058529893	0.575725022018312\\
1.26	0.765898996058536	0.972212281322139	0.570939154289561\\
1.26	0.780817519965826	0.975683883461272	0.566123121053177\\
1.26	0.795721169365275	0.97891274348356	0.561279136605558\\
1.26	0.810604533136476	0.981900889668376	0.556409397399198\\
1.26	0.825462312667457	0.984650494185936	0.551516079695736\\
1.26	0.840289325888138	0.987163867133889	0.546601337319641\\
1.26	0.855080510976839	0.989443450481768	0.541667299515141\\
1.26	0.86983092974082	0.991491811941915	0.536716068908598\\
1.26	0.884535770672926	0.993311638785082	0.53174971957814\\
1.26	0.899190351687411	0.994905731618374	0.526770295231796\\
1.26	0.913790122539012	0.996276998142666	0.521779807495105\\
1.26	0.928330666930242	0.997428446906011	0.516780234308684\\
1.26	0.942807704312699	0.998363181068902	0.511773518435872\\
1.26	0.95721709138901	0.999084392196567	0.506761566080168\\
1.26	0.971554823322703	0.999595354092743	0.501746245611875\\
1.26	0.985817034663989	0.999899416688637	0.496729386403004\\
1.26	1	1	0.491669087346906\\
1.275	0	0	0.710195569525497\\
1.275	0.000100583311362513	0.0141829653360114	0.71240460418299\\
1.275	0.000404645907256436	0.0284451766772965	0.714630690536905\\
1.275	0.000915607803433	0.0427829086109896	0.716916312064106\\
1.275	0.00163681893109844	0.057192295687301	0.719289019360234\\
1.275	0.00257155309398959	0.0716693330697585	0.721780367061016\\
1.275	0.00372300185733414	0.0862098774609879	0.724425655492835\\
1.275	0.00509426838162598	0.100809648312589	0.727263413850304\\
1.275	0.00668836121491816	0.115464229327074	0.73033458997498\\
1.275	0.00850818805808555	0.13016907025918	0.733681422514709\\
1.275	0.0105565495182326	0.144919489023162	0.737345986077549\\
1.275	0.0128361328661109	0.159710674111862	0.741368418705642\\
1.275	0.0153495058140643	0.174537687332543	0.745784862958202\\
1.275	0.0180991103316243	0.189395466863524	0.750625176092311\\
1.275	0.0210872565164405	0.204278830634725	0.755910489868998\\
1.275	0.0243161165387281	0.219182480034174	0.761650724664453\\
1.275	0.0277877186778607	0.234101003941464	0.767842183867616\\
1.275	0.0315039414701067	0.24902888308801	0.77446537092041\\
1.275	0.0354665079868145	0.263960494742775	0.781483180780606\\
1.275	0.0396769802625738	0.278890117720924	0.788839618268073\\
1.275	0.0441367538930258	0.293811937711588	0.796459186325544\\
1.275	0.0488470528220538	0.308720052919643	0.804247066923455\\
1.275	0.0538089243380495	0.323608480015097	0.81209018616195\\
1.275	0.0590232342988274	0.338471160382329	0.819859213932694\\
1.275	0.064490662604533	0.3533019666601	0.827411499063639\\
1.275	0.0702116989375697	0.368094709561873	0.834594885813506\\
1.275	0.0761866387881432	0.382843144964686	0.84125230028718\\
1.275	0.0824155797834956	0.397540981253432	0.847226939729081\\
1.275	0.0888984183382709	0.412181886906127	0.852367847905041\\
1.275	0.0956348466427212	0.42675949830445	0.856535620027829\\
1.275	0.102624350004627	0.441267427752584	0.859607954619618\\
1.275	0.109866204559871	0.455699271686218	0.861484760266622\\
1.275	0.117359475365564	0.470048619052377	0.862092534251539\\
1.275	0.125103014888515	0.484309059839721	0.861387758068371\\
1.275	0.133095461900593	0.498474193737904	0.859359100890819\\
1.275	0.14133524079124	0.512537638903683	0.856028283765733\\
1.275	0.149820561306021	0.526493040810599	0.851449530870182\\
1.275	0.158549418718625	0.540334081158348	0.845707614739505\\
1.275	0.167519594442214	0.554054486817265	0.838914584343151\\
1.275	0.176728657084455	0.567648038782868	0.831205342370836\\
1.275	0.186173963948925	0.581108581114919	0.822732305417528\\
1.275	0.195852662983903	0.594430029835237	0.813659432944309\\
1.275	0.205761695177907	0.607606381758231	0.804155944104867\\
1.275	0.215897797399558	0.620631723228143	0.794390053406969\\
1.275	0.226257505677663	0.633500238737009	0.7845230460628\\
1.275	0.236837158915675	0.646206219397571	0.774703982861589\\
1.275	0.247632903032949	0.658744071245709	0.765065275211016\\
1.275	0.258640695523528	0.671108323347402	0.755719307772964\\
1.275	0.269856310421543	0.683293635685828	0.746756214005913\\
1.275	0.28127534366066	0.695294806804925	0.738242834604294\\
1.275	0.292893218813452	0.707106781186547	0.730222816007836\\
1.275	0.304705193195075	0.71872465633934	0.722717741087767\\
1.275	0.316706364314172	0.730143689578457	0.715729131143265\\
1.275	0.328891676652598	0.741359304476472	0.709241120561857\\
1.275	0.341255928754291	0.752367096967051	0.703223584571265\\
1.275	0.353793780602429	0.763162841084325	0.697635496597417\\
1.275	0.366499761262991	0.773742494322337	0.692428303594618\\
1.275	0.379368276771857	0.784102202600442	0.687549132887069\\
1.275	0.392393618241769	0.794238304822092	0.682943679235329\\
1.275	0.405569970164763	0.804147337016097	0.678558662193046\\
1.275	0.418891418885081	0.813826036051075	0.674343787412872\\
1.275	0.432351961217133	0.823271342915545	0.670253187715552\\
1.275	0.445945513182735	0.832480405557787	0.666246357336158\\
1.275	0.459665918841652	0.841450581281375	0.662288623474185\\
1.275	0.473506959189401	0.850179438693979	0.658351221673191\\
1.275	0.487462361096317	0.85866475920876	0.654411055137886\\
1.275	0.501525806262096	0.866904538099407	0.650450223206467\\
1.275	0.515690940160279	0.874896985111485	0.646455401874959\\
1.275	0.529951380947623	0.882640524634437	0.642417151055155\\
1.275	0.544300728313782	0.890133795440129	0.638329210947585\\
1.275	0.558732572247415	0.897375649995373	0.634187835387498\\
1.275	0.57324050169555	0.904365153357279	0.629991194998376\\
1.275	0.587818113093873	0.911101581661729	0.625738868900924\\
1.275	0.602459018746568	0.917584420216504	0.62143143163314\\
1.275	0.617156855035314	0.923813361211857	0.617070132480554\\
1.275	0.631905290438127	0.92978830106243	0.612656657833838\\
1.275	0.6466980333399	0.935509337395467	0.608192963377087\\
1.275	0.661528839617671	0.940976765701173	0.603681161495105\\
1.275	0.676391519984904	0.946191075661951	0.599123449738607\\
1.275	0.691279947080357	0.951152947177946	0.594522067901589\\
1.275	0.706188062288412	0.955863246106974	0.589879273662388\\
1.275	0.721109882279076	0.960323019737426	0.585197329319937\\
1.275	0.736039505257225	0.964533492013186	0.580478494543978\\
1.275	0.75097111691199	0.968496058529893	0.575725022018313\\
1.275	0.765898996058536	0.972212281322139	0.570939154289557\\
1.275	0.780817519965826	0.975683883461272	0.566123121053175\\
1.275	0.795721169365275	0.97891274348356	0.561279136605558\\
1.275	0.810604533136476	0.981900889668376	0.556409397399198\\
1.275	0.825462312667457	0.984650494185936	0.551516079695735\\
1.275	0.840289325888138	0.987163867133889	0.546601337319641\\
1.275	0.855080510976839	0.989443450481768	0.541667299515139\\
1.275	0.86983092974082	0.991491811941914	0.536716068908596\\
1.275	0.884535770672926	0.993311638785082	0.531749719578141\\
1.275	0.899190351687411	0.994905731618374	0.526770295231796\\
1.275	0.913790122539012	0.996276998142666	0.521779807495104\\
1.275	0.928330666930242	0.997428446906011	0.516780234308685\\
1.275	0.942807704312699	0.998363181068902	0.511773518435872\\
1.275	0.95721709138901	0.999084392196567	0.506761566080166\\
1.275	0.971554823322703	0.999595354092743	0.501746245611874\\
1.275	0.985817034663989	0.999899416688637	0.496729386403005\\
1.275	1	1	0.491669087346904\\
1.29	0	0	0.710195569525497\\
1.29	0.000100583311362513	0.0141829653360114	0.71240460418299\\
1.29	0.000404645907256436	0.0284451766772965	0.714630690536905\\
1.29	0.000915607803433	0.0427829086109896	0.716916312064106\\
1.29	0.00163681893109844	0.057192295687301	0.719289019360234\\
1.29	0.00257155309398959	0.0716693330697584	0.721780367061016\\
1.29	0.00372300185733414	0.086209877460988	0.724425655492835\\
1.29	0.00509426838162598	0.100809648312589	0.727263413850304\\
1.29	0.00668836121491816	0.115464229327074	0.73033458997498\\
1.29	0.00850818805808555	0.13016907025918	0.733681422514709\\
1.29	0.0105565495182326	0.144919489023162	0.737345986077549\\
1.29	0.0128361328661109	0.159710674111862	0.741368418705642\\
1.29	0.0153495058140643	0.174537687332543	0.745784862958201\\
1.29	0.0180991103316243	0.189395466863524	0.750625176092311\\
1.29	0.0210872565164405	0.204278830634725	0.755910489868998\\
1.29	0.0243161165387281	0.219182480034174	0.761650724664453\\
1.29	0.0277877186778607	0.234101003941464	0.767842183867616\\
1.29	0.0315039414701067	0.24902888308801	0.77446537092041\\
1.29	0.0354665079868145	0.263960494742775	0.781483180780606\\
1.29	0.0396769802625738	0.278890117720924	0.788839618268073\\
1.29	0.0441367538930258	0.293811937711588	0.796459186325544\\
1.29	0.0488470528220538	0.308720052919643	0.804247066923454\\
1.29	0.0538089243380495	0.323608480015096	0.812090186161949\\
1.29	0.0590232342988274	0.338471160382329	0.819859213932694\\
1.29	0.064490662604533	0.3533019666601	0.82741149906364\\
1.29	0.0702116989375697	0.368094709561873	0.834594885813504\\
1.29	0.0761866387881432	0.382843144964686	0.841252300287181\\
1.29	0.0824155797834956	0.397540981253432	0.847226939729081\\
1.29	0.0888984183382709	0.412181886906127	0.852367847905041\\
1.29	0.0956348466427212	0.42675949830445	0.85653562002783\\
1.29	0.102624350004627	0.441267427752584	0.859607954619617\\
1.29	0.109866204559871	0.455699271686219	0.861484760266624\\
1.29	0.117359475365564	0.470048619052377	0.86209253425154\\
1.29	0.125103014888515	0.484309059839721	0.861387758068369\\
1.29	0.133095461900593	0.498474193737904	0.85935910089082\\
1.29	0.14133524079124	0.512537638903683	0.856028283765733\\
1.29	0.149820561306021	0.526493040810599	0.851449530870183\\
1.29	0.158549418718625	0.540334081158348	0.845707614739502\\
1.29	0.167519594442213	0.554054486817265	0.83891458434315\\
1.29	0.176728657084455	0.567648038782867	0.831205342370834\\
1.29	0.186173963948925	0.581108581114919	0.822732305417528\\
1.29	0.195852662983903	0.594430029835237	0.813659432944308\\
1.29	0.205761695177907	0.607606381758231	0.804155944104867\\
1.29	0.215897797399558	0.620631723228143	0.794390053406969\\
1.29	0.226257505677663	0.633500238737009	0.7845230460628\\
1.29	0.236837158915675	0.646206219397571	0.774703982861589\\
1.29	0.247632903032949	0.658744071245709	0.765065275211016\\
1.29	0.258640695523528	0.671108323347402	0.755719307772964\\
1.29	0.269856310421543	0.683293635685828	0.746756214005914\\
1.29	0.28127534366066	0.695294806804925	0.738242834604294\\
1.29	0.292893218813452	0.707106781186547	0.730222816007836\\
1.29	0.304705193195075	0.71872465633934	0.722717741087767\\
1.29	0.316706364314172	0.730143689578457	0.715729131143265\\
1.29	0.328891676652598	0.741359304476472	0.709241120561857\\
1.29	0.341255928754291	0.752367096967051	0.703223584571265\\
1.29	0.353793780602429	0.763162841084325	0.697635496597417\\
1.29	0.366499761262991	0.773742494322337	0.692428303594618\\
1.29	0.379368276771857	0.784102202600442	0.68754913288707\\
1.29	0.392393618241769	0.794238304822092	0.682943679235329\\
1.29	0.405569970164763	0.804147337016097	0.678558662193046\\
1.29	0.418891418885081	0.813826036051075	0.674343787412872\\
1.29	0.432351961217133	0.823271342915545	0.670253187715552\\
1.29	0.445945513182735	0.832480405557787	0.666246357336158\\
1.29	0.459665918841652	0.841450581281375	0.662288623474185\\
1.29	0.473506959189401	0.850179438693979	0.658351221673191\\
1.29	0.487462361096317	0.85866475920876	0.654411055137887\\
1.29	0.501525806262096	0.866904538099407	0.650450223206468\\
1.29	0.515690940160279	0.874896985111485	0.646455401874959\\
1.29	0.529951380947623	0.882640524634437	0.642417151055155\\
1.29	0.544300728313782	0.890133795440129	0.638329210947585\\
1.29	0.558732572247415	0.897375649995373	0.634187835387498\\
1.29	0.57324050169555	0.904365153357279	0.629991194998376\\
1.29	0.587818113093873	0.911101581661729	0.625738868900924\\
1.29	0.602459018746568	0.917584420216504	0.62143143163314\\
1.29	0.617156855035314	0.923813361211857	0.617070132480554\\
1.29	0.631905290438127	0.92978830106243	0.612656657833838\\
1.29	0.6466980333399	0.935509337395467	0.608192963377087\\
1.29	0.661528839617671	0.940976765701173	0.603681161495105\\
1.29	0.676391519984904	0.946191075661951	0.599123449738607\\
1.29	0.691279947080357	0.951152947177946	0.59452206790159\\
1.29	0.706188062288412	0.955863246106974	0.589879273662388\\
1.29	0.721109882279076	0.960323019737426	0.585197329319936\\
1.29	0.736039505257225	0.964533492013186	0.580478494543978\\
1.29	0.75097111691199	0.968496058529893	0.575725022018314\\
1.29	0.765898996058536	0.972212281322139	0.570939154289558\\
1.29	0.780817519965826	0.975683883461272	0.566123121053175\\
1.29	0.795721169365275	0.97891274348356	0.561279136605558\\
1.29	0.810604533136476	0.981900889668376	0.556409397399199\\
1.29	0.825462312667457	0.984650494185936	0.551516079695736\\
1.29	0.840289325888138	0.987163867133889	0.546601337319639\\
1.29	0.855080510976839	0.989443450481768	0.541667299515141\\
1.29	0.86983092974082	0.991491811941915	0.536716068908598\\
1.29	0.884535770672926	0.993311638785082	0.531749719578139\\
1.29	0.899190351687411	0.994905731618374	0.526770295231796\\
1.29	0.913790122539012	0.996276998142666	0.521779807495104\\
1.29	0.928330666930242	0.997428446906011	0.516780234308684\\
1.29	0.942807704312699	0.998363181068902	0.511773518435872\\
1.29	0.95721709138901	0.999084392196567	0.506761566080168\\
1.29	0.971554823322703	0.999595354092743	0.501746245611874\\
1.29	0.985817034663989	0.999899416688637	0.496729386403004\\
1.29	1	1	0.491669087346906\\
1.305	0	0	0.710195569525497\\
1.305	0.000100583311362513	0.0141829653360114	0.71240460418299\\
1.305	0.000404645907256436	0.0284451766772965	0.714630690536905\\
1.305	0.000915607803433	0.0427829086109896	0.716916312064106\\
1.305	0.00163681893109844	0.057192295687301	0.719289019360234\\
1.305	0.00257155309398959	0.0716693330697585	0.721780367061016\\
1.305	0.00372300185733414	0.086209877460988	0.724425655492835\\
1.305	0.00509426838162598	0.100809648312589	0.727263413850304\\
1.305	0.00668836121491816	0.115464229327074	0.73033458997498\\
1.305	0.00850818805808555	0.13016907025918	0.733681422514709\\
1.305	0.0105565495182326	0.144919489023162	0.737345986077549\\
1.305	0.0128361328661109	0.159710674111862	0.741368418705642\\
1.305	0.0153495058140643	0.174537687332543	0.745784862958201\\
1.305	0.0180991103316243	0.189395466863524	0.750625176092311\\
1.305	0.0210872565164405	0.204278830634725	0.755910489868998\\
1.305	0.0243161165387281	0.219182480034174	0.761650724664453\\
1.305	0.0277877186778607	0.234101003941464	0.767842183867616\\
1.305	0.0315039414701067	0.24902888308801	0.77446537092041\\
1.305	0.0354665079868145	0.263960494742775	0.781483180780606\\
1.305	0.0396769802625738	0.278890117720924	0.788839618268073\\
1.305	0.0441367538930258	0.293811937711588	0.796459186325544\\
1.305	0.0488470528220538	0.308720052919643	0.804247066923454\\
1.305	0.0538089243380495	0.323608480015096	0.812090186161949\\
1.305	0.0590232342988274	0.338471160382329	0.819859213932694\\
1.305	0.064490662604533	0.3533019666601	0.827411499063639\\
1.305	0.0702116989375697	0.368094709561873	0.834594885813504\\
1.305	0.0761866387881432	0.382843144964686	0.841252300287178\\
1.305	0.0824155797834956	0.397540981253432	0.847226939729081\\
1.305	0.0888984183382709	0.412181886906127	0.852367847905043\\
1.305	0.0956348466427212	0.42675949830445	0.856535620027829\\
1.305	0.102624350004627	0.441267427752584	0.859607954619617\\
1.305	0.109866204559871	0.455699271686219	0.861484760266622\\
1.305	0.117359475365564	0.470048619052377	0.86209253425154\\
1.305	0.125103014888515	0.484309059839721	0.861387758068365\\
1.305	0.133095461900593	0.498474193737904	0.85935910089082\\
1.305	0.14133524079124	0.512537638903683	0.856028283765735\\
1.305	0.149820561306021	0.526493040810599	0.851449530870182\\
1.305	0.158549418718625	0.540334081158348	0.845707614739505\\
1.305	0.167519594442214	0.554054486817265	0.838914584343151\\
1.305	0.176728657084455	0.567648038782867	0.831205342370836\\
1.305	0.186173963948925	0.581108581114919	0.822732305417527\\
1.305	0.195852662983903	0.594430029835237	0.813659432944309\\
1.305	0.205761695177907	0.607606381758231	0.804155944104868\\
1.305	0.215897797399558	0.620631723228143	0.794390053406969\\
1.305	0.226257505677663	0.633500238737009	0.7845230460628\\
1.305	0.236837158915675	0.646206219397571	0.774703982861589\\
1.305	0.247632903032949	0.658744071245709	0.765065275211016\\
1.305	0.258640695523528	0.671108323347402	0.755719307772964\\
1.305	0.269856310421543	0.683293635685828	0.746756214005914\\
1.305	0.28127534366066	0.695294806804925	0.738242834604294\\
1.305	0.292893218813452	0.707106781186547	0.730222816007836\\
1.305	0.304705193195075	0.71872465633934	0.722717741087766\\
1.305	0.316706364314172	0.730143689578457	0.715729131143265\\
1.305	0.328891676652598	0.741359304476472	0.709241120561857\\
1.305	0.341255928754291	0.752367096967051	0.703223584571265\\
1.305	0.353793780602429	0.763162841084325	0.697635496597417\\
1.305	0.366499761262991	0.773742494322337	0.692428303594618\\
1.305	0.379368276771857	0.784102202600442	0.687549132887069\\
1.305	0.392393618241769	0.794238304822092	0.682943679235329\\
1.305	0.405569970164763	0.804147337016097	0.678558662193046\\
1.305	0.418891418885081	0.813826036051075	0.674343787412872\\
1.305	0.432351961217133	0.823271342915545	0.670253187715552\\
1.305	0.445945513182735	0.832480405557787	0.666246357336157\\
1.305	0.459665918841652	0.841450581281375	0.662288623474185\\
1.305	0.473506959189401	0.850179438693979	0.658351221673192\\
1.305	0.487462361096317	0.85866475920876	0.654411055137886\\
1.305	0.501525806262096	0.866904538099407	0.650450223206468\\
1.305	0.515690940160279	0.874896985111485	0.646455401874959\\
1.305	0.529951380947623	0.882640524634437	0.642417151055156\\
1.305	0.544300728313782	0.890133795440129	0.638329210947586\\
1.305	0.558732572247415	0.897375649995373	0.634187835387497\\
1.305	0.57324050169555	0.904365153357279	0.629991194998376\\
1.305	0.587818113093873	0.911101581661729	0.625738868900924\\
1.305	0.602459018746568	0.917584420216504	0.62143143163314\\
1.305	0.617156855035314	0.923813361211857	0.617070132480554\\
1.305	0.631905290438127	0.92978830106243	0.612656657833838\\
1.305	0.6466980333399	0.935509337395467	0.608192963377087\\
1.305	0.661528839617671	0.940976765701173	0.603681161495105\\
1.305	0.676391519984904	0.946191075661951	0.599123449738607\\
1.305	0.691279947080357	0.951152947177946	0.59452206790159\\
1.305	0.706188062288412	0.955863246106974	0.589879273662389\\
1.305	0.721109882279076	0.960323019737426	0.585197329319936\\
1.305	0.736039505257225	0.964533492013186	0.580478494543975\\
1.305	0.75097111691199	0.968496058529893	0.575725022018313\\
1.305	0.765898996058536	0.972212281322139	0.57093915428956\\
1.305	0.780817519965826	0.975683883461272	0.566123121053175\\
1.305	0.795721169365275	0.97891274348356	0.561279136605558\\
1.305	0.810604533136476	0.981900889668376	0.556409397399198\\
1.305	0.825462312667457	0.984650494185936	0.551516079695737\\
1.305	0.840289325888138	0.987163867133889	0.546601337319641\\
1.305	0.855080510976839	0.989443450481768	0.541667299515139\\
1.305	0.86983092974082	0.991491811941914	0.536716068908598\\
1.305	0.884535770672926	0.993311638785082	0.531749719578141\\
1.305	0.899190351687411	0.994905731618374	0.526770295231796\\
1.305	0.913790122539012	0.996276998142666	0.521779807495104\\
1.305	0.928330666930242	0.997428446906011	0.516780234308684\\
1.305	0.942807704312699	0.998363181068902	0.511773518435872\\
1.305	0.95721709138901	0.999084392196567	0.506761566080168\\
1.305	0.971554823322703	0.999595354092743	0.501746245611874\\
1.305	0.985817034663989	0.999899416688637	0.496729386403004\\
1.305	1	1	0.491669087346906\\
1.32	0	0	0.710195569525497\\
1.32	0.000100583311362513	0.0141829653360114	0.71240460418299\\
1.32	0.000404645907256436	0.0284451766772965	0.714630690536905\\
1.32	0.000915607803433	0.0427829086109896	0.716916312064106\\
1.32	0.00163681893109844	0.057192295687301	0.719289019360234\\
1.32	0.00257155309398959	0.0716693330697585	0.721780367061016\\
1.32	0.00372300185733414	0.086209877460988	0.724425655492835\\
1.32	0.00509426838162598	0.100809648312589	0.727263413850304\\
1.32	0.00668836121491816	0.115464229327074	0.73033458997498\\
1.32	0.00850818805808555	0.13016907025918	0.733681422514709\\
1.32	0.0105565495182326	0.144919489023162	0.737345986077549\\
1.32	0.0128361328661109	0.159710674111862	0.741368418705642\\
1.32	0.0153495058140643	0.174537687332543	0.745784862958202\\
1.32	0.0180991103316243	0.189395466863524	0.750625176092311\\
1.32	0.0210872565164405	0.204278830634725	0.755910489868999\\
1.32	0.0243161165387281	0.219182480034174	0.761650724664453\\
1.32	0.0277877186778607	0.234101003941464	0.767842183867616\\
1.32	0.0315039414701067	0.24902888308801	0.77446537092041\\
1.32	0.0354665079868145	0.263960494742775	0.781483180780606\\
1.32	0.0396769802625738	0.278890117720924	0.788839618268073\\
1.32	0.0441367538930258	0.293811937711588	0.796459186325544\\
1.32	0.0488470528220538	0.308720052919643	0.804247066923454\\
1.32	0.0538089243380495	0.323608480015096	0.81209018616195\\
1.32	0.0590232342988274	0.338471160382329	0.819859213932693\\
1.32	0.064490662604533	0.3533019666601	0.827411499063638\\
1.32	0.0702116989375697	0.368094709561873	0.834594885813504\\
1.32	0.0761866387881432	0.382843144964686	0.841252300287179\\
1.32	0.0824155797834956	0.397540981253432	0.847226939729081\\
1.32	0.0888984183382709	0.412181886906127	0.852367847905041\\
1.32	0.0956348466427213	0.42675949830445	0.85653562002783\\
1.32	0.102624350004627	0.441267427752584	0.859607954619617\\
1.32	0.109866204559871	0.455699271686219	0.861484760266622\\
1.32	0.117359475365564	0.470048619052377	0.862092534251541\\
1.32	0.125103014888515	0.484309059839721	0.861387758068368\\
1.32	0.133095461900593	0.498474193737904	0.85935910089082\\
1.32	0.14133524079124	0.512537638903683	0.856028283765736\\
1.32	0.149820561306021	0.526493040810599	0.851449530870183\\
1.32	0.158549418718625	0.540334081158348	0.845707614739508\\
1.32	0.167519594442214	0.554054486817265	0.83891458434315\\
1.32	0.176728657084455	0.567648038782867	0.831205342370836\\
1.32	0.186173963948925	0.581108581114919	0.822732305417529\\
1.32	0.195852662983903	0.594430029835237	0.813659432944309\\
1.32	0.205761695177907	0.607606381758231	0.804155944104868\\
1.32	0.215897797399558	0.620631723228143	0.794390053406969\\
1.32	0.226257505677663	0.633500238737009	0.7845230460628\\
1.32	0.236837158915675	0.646206219397571	0.774703982861589\\
1.32	0.247632903032949	0.658744071245709	0.765065275211016\\
1.32	0.258640695523528	0.671108323347402	0.755719307772964\\
1.32	0.269856310421543	0.683293635685828	0.746756214005914\\
1.32	0.28127534366066	0.695294806804925	0.738242834604294\\
1.32	0.292893218813452	0.707106781186547	0.730222816007833\\
1.32	0.304705193195075	0.71872465633934	0.722717741087767\\
1.32	0.316706364314172	0.730143689578457	0.715729131143268\\
1.32	0.328891676652598	0.741359304476472	0.709241120561857\\
1.32	0.341255928754291	0.752367096967051	0.703223584571265\\
1.32	0.353793780602429	0.763162841084325	0.697635496597417\\
1.32	0.366499761262991	0.773742494322337	0.692428303594618\\
1.32	0.379368276771857	0.784102202600442	0.687549132887069\\
1.32	0.392393618241769	0.794238304822092	0.682943679235329\\
1.32	0.405569970164763	0.804147337016097	0.678558662193046\\
1.32	0.418891418885081	0.813826036051075	0.674343787412872\\
1.32	0.432351961217133	0.823271342915545	0.670253187715552\\
1.32	0.445945513182735	0.832480405557787	0.666246357336158\\
1.32	0.459665918841652	0.841450581281375	0.662288623474184\\
1.32	0.473506959189401	0.850179438693979	0.658351221673191\\
1.32	0.487462361096317	0.85866475920876	0.654411055137887\\
1.32	0.501525806262096	0.866904538099407	0.650450223206467\\
1.32	0.515690940160279	0.874896985111485	0.646455401874959\\
1.32	0.529951380947623	0.882640524634437	0.642417151055155\\
1.32	0.544300728313782	0.890133795440129	0.638329210947586\\
1.32	0.558732572247415	0.897375649995373	0.634187835387499\\
1.32	0.57324050169555	0.904365153357279	0.629991194998376\\
1.32	0.587818113093873	0.911101581661729	0.625738868900924\\
1.32	0.602459018746568	0.917584420216504	0.62143143163314\\
1.32	0.617156855035314	0.923813361211857	0.617070132480554\\
1.32	0.631905290438127	0.92978830106243	0.612656657833838\\
1.32	0.6466980333399	0.935509337395467	0.608192963377087\\
1.32	0.661528839617671	0.940976765701173	0.603681161495105\\
1.32	0.676391519984904	0.946191075661951	0.599123449738607\\
1.32	0.691279947080357	0.951152947177946	0.59452206790159\\
1.32	0.706188062288412	0.955863246106974	0.589879273662389\\
1.32	0.721109882279076	0.960323019737426	0.585197329319936\\
1.32	0.736039505257225	0.964533492013186	0.580478494543978\\
1.32	0.75097111691199	0.968496058529893	0.575725022018314\\
1.32	0.765898996058536	0.972212281322139	0.570939154289558\\
1.32	0.780817519965826	0.975683883461272	0.566123121053174\\
1.32	0.795721169365275	0.97891274348356	0.561279136605558\\
1.32	0.810604533136476	0.981900889668376	0.556409397399197\\
1.32	0.825462312667457	0.984650494185936	0.551516079695736\\
1.32	0.840289325888138	0.987163867133889	0.546601337319641\\
1.32	0.855080510976839	0.989443450481768	0.541667299515139\\
1.32	0.86983092974082	0.991491811941914	0.536716068908598\\
1.32	0.884535770672926	0.993311638785082	0.531749719578141\\
1.32	0.899190351687411	0.994905731618374	0.526770295231796\\
1.32	0.913790122539012	0.996276998142666	0.521779807495104\\
1.32	0.928330666930242	0.997428446906011	0.516780234308685\\
1.32	0.942807704312699	0.998363181068902	0.511773518435872\\
1.32	0.95721709138901	0.999084392196567	0.506761566080168\\
1.32	0.971554823322704	0.999595354092744	0.501746245611874\\
1.32	0.985817034663989	0.999899416688637	0.496729386403002\\
1.32	1	1	0.491669087346908\\
1.335	0	0	0.710195569525497\\
1.335	0.000100583311362513	0.0141829653360114	0.71240460418299\\
1.335	0.000404645907256436	0.0284451766772965	0.714630690536905\\
1.335	0.000915607803433	0.0427829086109896	0.716916312064106\\
1.335	0.00163681893109844	0.057192295687301	0.719289019360234\\
1.335	0.00257155309398959	0.0716693330697585	0.721780367061016\\
1.335	0.00372300185733414	0.086209877460988	0.724425655492835\\
1.335	0.00509426838162598	0.100809648312589	0.727263413850304\\
1.335	0.00668836121491816	0.115464229327074	0.73033458997498\\
1.335	0.00850818805808555	0.13016907025918	0.733681422514709\\
1.335	0.0105565495182326	0.144919489023162	0.737345986077549\\
1.335	0.0128361328661109	0.159710674111862	0.741368418705642\\
1.335	0.0153495058140643	0.174537687332543	0.745784862958202\\
1.335	0.0180991103316243	0.189395466863524	0.750625176092311\\
1.335	0.0210872565164405	0.204278830634725	0.755910489868998\\
1.335	0.0243161165387281	0.219182480034174	0.761650724664453\\
1.335	0.0277877186778607	0.234101003941464	0.767842183867616\\
1.335	0.0315039414701067	0.24902888308801	0.77446537092041\\
1.335	0.0354665079868145	0.263960494742775	0.781483180780606\\
1.335	0.0396769802625738	0.278890117720924	0.788839618268073\\
1.335	0.0441367538930258	0.293811937711588	0.796459186325544\\
1.335	0.0488470528220538	0.308720052919643	0.804247066923454\\
1.335	0.0538089243380495	0.323608480015096	0.81209018616195\\
1.335	0.0590232342988274	0.338471160382329	0.819859213932694\\
1.335	0.064490662604533	0.3533019666601	0.827411499063639\\
1.335	0.0702116989375697	0.368094709561873	0.834594885813504\\
1.335	0.0761866387881432	0.382843144964686	0.841252300287179\\
1.335	0.0824155797834956	0.397540981253432	0.847226939729081\\
1.335	0.0888984183382709	0.412181886906127	0.85236784790504\\
1.335	0.0956348466427212	0.42675949830445	0.85653562002783\\
1.335	0.102624350004627	0.441267427752584	0.859607954619617\\
1.335	0.109866204559871	0.455699271686219	0.861484760266621\\
1.335	0.117359475365564	0.470048619052377	0.862092534251539\\
1.335	0.125103014888515	0.484309059839721	0.861387758068369\\
1.335	0.133095461900593	0.498474193737904	0.85935910089082\\
1.335	0.14133524079124	0.512537638903683	0.85602828376574\\
1.335	0.149820561306021	0.526493040810599	0.851449530870183\\
1.335	0.158549418718625	0.540334081158348	0.845707614739504\\
1.335	0.167519594442214	0.554054486817265	0.83891458434315\\
1.335	0.176728657084455	0.567648038782867	0.831205342370837\\
1.335	0.186173963948925	0.581108581114919	0.822732305417528\\
1.335	0.195852662983903	0.594430029835237	0.813659432944309\\
1.335	0.205761695177907	0.607606381758231	0.804155944104868\\
1.335	0.215897797399558	0.620631723228143	0.79439005340697\\
1.335	0.226257505677663	0.633500238737009	0.7845230460628\\
1.335	0.236837158915675	0.646206219397571	0.774703982861589\\
1.335	0.247632903032949	0.658744071245709	0.765065275211016\\
1.335	0.258640695523528	0.671108323347402	0.755719307772964\\
1.335	0.269856310421543	0.683293635685828	0.746756214005914\\
1.335	0.28127534366066	0.695294806804925	0.738242834604294\\
1.335	0.292893218813452	0.707106781186547	0.730222816007836\\
1.335	0.304705193195075	0.71872465633934	0.722717741087768\\
1.335	0.316706364314172	0.730143689578457	0.715729131143265\\
1.335	0.328891676652598	0.741359304476472	0.709241120561857\\
1.335	0.341255928754291	0.752367096967051	0.703223584571265\\
1.335	0.353793780602429	0.763162841084325	0.697635496597418\\
1.335	0.366499761262991	0.773742494322337	0.692428303594618\\
1.335	0.379368276771857	0.784102202600442	0.687549132887069\\
1.335	0.392393618241769	0.794238304822092	0.682943679235329\\
1.335	0.405569970164763	0.804147337016097	0.678558662193046\\
1.335	0.418891418885081	0.813826036051075	0.674343787412872\\
1.335	0.432351961217133	0.823271342915545	0.670253187715552\\
1.335	0.445945513182735	0.832480405557787	0.666246357336158\\
1.335	0.459665918841652	0.841450581281375	0.662288623474184\\
1.335	0.473506959189401	0.850179438693979	0.658351221673191\\
1.335	0.487462361096317	0.85866475920876	0.654411055137886\\
1.335	0.501525806262096	0.866904538099407	0.650450223206467\\
1.335	0.515690940160279	0.874896985111485	0.646455401874959\\
1.335	0.529951380947623	0.882640524634437	0.642417151055155\\
1.335	0.544300728313782	0.890133795440129	0.638329210947585\\
1.335	0.558732572247415	0.897375649995373	0.634187835387498\\
1.335	0.57324050169555	0.904365153357279	0.629991194998376\\
1.335	0.587818113093873	0.911101581661729	0.625738868900924\\
1.335	0.602459018746568	0.917584420216504	0.62143143163314\\
1.335	0.617156855035314	0.923813361211857	0.617070132480554\\
1.335	0.631905290438127	0.92978830106243	0.612656657833838\\
1.335	0.6466980333399	0.935509337395467	0.608192963377087\\
1.335	0.661528839617671	0.940976765701173	0.603681161495105\\
1.335	0.676391519984904	0.946191075661951	0.599123449738608\\
1.335	0.691279947080357	0.951152947177946	0.59452206790159\\
1.335	0.706188062288412	0.955863246106974	0.589879273662388\\
1.335	0.721109882279076	0.960323019737426	0.585197329319936\\
1.335	0.736039505257225	0.964533492013186	0.580478494543978\\
1.335	0.75097111691199	0.968496058529893	0.575725022018313\\
1.335	0.765898996058536	0.972212281322139	0.570939154289558\\
1.335	0.780817519965826	0.975683883461272	0.566123121053175\\
1.335	0.795721169365275	0.97891274348356	0.56127913660556\\
1.335	0.810604533136476	0.981900889668376	0.556409397399198\\
1.335	0.825462312667457	0.984650494185936	0.551516079695734\\
1.335	0.840289325888138	0.987163867133889	0.546601337319641\\
1.335	0.855080510976839	0.989443450481768	0.541667299515139\\
1.335	0.86983092974082	0.991491811941914	0.536716068908599\\
1.335	0.884535770672926	0.993311638785082	0.531749719578141\\
1.335	0.899190351687411	0.994905731618374	0.526770295231796\\
1.335	0.913790122539012	0.996276998142666	0.521779807495104\\
1.335	0.928330666930242	0.997428446906011	0.516780234308682\\
1.335	0.942807704312699	0.998363181068902	0.511773518435873\\
1.335	0.95721709138901	0.999084392196567	0.506761566080168\\
1.335	0.971554823322703	0.999595354092743	0.501746245611875\\
1.335	0.985817034663989	0.999899416688637	0.496729386403004\\
1.335	1	1	0.491669087346906\\
1.35	0	0	0.710195569525497\\
1.35	0.000100583311362513	0.0141829653360114	0.71240460418299\\
1.35	0.000404645907256436	0.0284451766772965	0.714630690536905\\
1.35	0.000915607803433	0.0427829086109896	0.716916312064106\\
1.35	0.00163681893109844	0.057192295687301	0.719289019360234\\
1.35	0.00257155309398959	0.0716693330697585	0.721780367061016\\
1.35	0.00372300185733414	0.086209877460988	0.724425655492835\\
1.35	0.00509426838162598	0.100809648312589	0.727263413850304\\
1.35	0.00668836121491816	0.115464229327074	0.73033458997498\\
1.35	0.00850818805808555	0.13016907025918	0.733681422514709\\
1.35	0.0105565495182326	0.144919489023162	0.737345986077549\\
1.35	0.0128361328661109	0.159710674111862	0.741368418705642\\
1.35	0.0153495058140643	0.174537687332543	0.745784862958202\\
1.35	0.0180991103316243	0.189395466863524	0.750625176092311\\
1.35	0.0210872565164405	0.204278830634725	0.755910489868998\\
1.35	0.0243161165387281	0.219182480034174	0.761650724664453\\
1.35	0.0277877186778607	0.234101003941464	0.767842183867616\\
1.35	0.0315039414701067	0.24902888308801	0.77446537092041\\
1.35	0.0354665079868145	0.263960494742775	0.781483180780606\\
1.35	0.0396769802625738	0.278890117720924	0.788839618268073\\
1.35	0.0441367538930258	0.293811937711588	0.796459186325544\\
1.35	0.0488470528220538	0.308720052919643	0.804247066923454\\
1.35	0.0538089243380495	0.323608480015096	0.81209018616195\\
1.35	0.0590232342988274	0.338471160382329	0.819859213932694\\
1.35	0.064490662604533	0.3533019666601	0.827411499063639\\
1.35	0.0702116989375697	0.368094709561873	0.834594885813505\\
1.35	0.0761866387881432	0.382843144964686	0.84125230028718\\
1.35	0.0824155797834956	0.397540981253432	0.847226939729081\\
1.35	0.0888984183382709	0.412181886906127	0.852367847905041\\
1.35	0.0956348466427212	0.42675949830445	0.856535620027829\\
1.35	0.102624350004627	0.441267427752584	0.859607954619617\\
1.35	0.109866204559871	0.455699271686219	0.86148476026662\\
1.35	0.117359475365564	0.470048619052377	0.862092534251539\\
1.35	0.125103014888515	0.484309059839721	0.86138775806837\\
1.35	0.133095461900593	0.498474193737904	0.85935910089082\\
1.35	0.14133524079124	0.512537638903683	0.856028283765736\\
1.35	0.149820561306021	0.526493040810599	0.851449530870182\\
1.35	0.158549418718625	0.540334081158348	0.845707614739505\\
1.35	0.167519594442214	0.554054486817265	0.83891458434315\\
1.35	0.176728657084455	0.567648038782867	0.831205342370837\\
1.35	0.186173963948925	0.581108581114919	0.822732305417527\\
1.35	0.195852662983903	0.594430029835237	0.813659432944309\\
1.35	0.205761695177907	0.607606381758231	0.804155944104868\\
1.35	0.215897797399558	0.620631723228143	0.794390053406969\\
1.35	0.226257505677663	0.633500238737009	0.7845230460628\\
1.35	0.236837158915675	0.646206219397571	0.774703982861589\\
1.35	0.247632903032949	0.658744071245709	0.765065275211016\\
1.35	0.258640695523528	0.671108323347402	0.755719307772964\\
1.35	0.269856310421543	0.683293635685828	0.746756214005914\\
1.35	0.28127534366066	0.695294806804925	0.738242834604294\\
1.35	0.292893218813452	0.707106781186547	0.730222816007836\\
1.35	0.304705193195075	0.71872465633934	0.722717741087767\\
1.35	0.316706364314172	0.730143689578457	0.715729131143265\\
1.35	0.328891676652598	0.741359304476472	0.709241120561857\\
1.35	0.341255928754291	0.752367096967051	0.703223584571265\\
1.35	0.353793780602429	0.763162841084325	0.697635496597417\\
1.35	0.366499761262991	0.773742494322337	0.692428303594618\\
1.35	0.379368276771857	0.784102202600442	0.687549132887069\\
1.35	0.392393618241769	0.794238304822092	0.682943679235329\\
1.35	0.405569970164763	0.804147337016097	0.678558662193046\\
1.35	0.418891418885081	0.813826036051075	0.674343787412872\\
1.35	0.432351961217133	0.823271342915545	0.670253187715552\\
1.35	0.445945513182735	0.832480405557787	0.666246357336158\\
1.35	0.459665918841652	0.841450581281375	0.662288623474185\\
1.35	0.473506959189401	0.850179438693979	0.658351221673191\\
1.35	0.487462361096317	0.85866475920876	0.654411055137886\\
1.35	0.501525806262096	0.866904538099407	0.650450223206467\\
1.35	0.515690940160279	0.874896985111485	0.646455401874959\\
1.35	0.529951380947623	0.882640524634437	0.642417151055155\\
1.35	0.544300728313782	0.890133795440129	0.638329210947585\\
1.35	0.558732572247415	0.897375649995373	0.634187835387498\\
1.35	0.57324050169555	0.904365153357279	0.629991194998376\\
1.35	0.587818113093873	0.911101581661729	0.625738868900924\\
1.35	0.602459018746568	0.917584420216504	0.62143143163314\\
1.35	0.617156855035314	0.923813361211857	0.617070132480554\\
1.35	0.631905290438127	0.92978830106243	0.612656657833838\\
1.35	0.6466980333399	0.935509337395467	0.608192963377087\\
1.35	0.661528839617671	0.940976765701173	0.603681161495105\\
1.35	0.676391519984904	0.946191075661951	0.599123449738607\\
1.35	0.691279947080357	0.951152947177946	0.594522067901591\\
1.35	0.706188062288412	0.955863246106974	0.589879273662389\\
1.35	0.721109882279076	0.960323019737426	0.585197329319936\\
1.35	0.736039505257225	0.964533492013186	0.580478494543978\\
1.35	0.75097111691199	0.968496058529893	0.575725022018314\\
1.35	0.765898996058536	0.972212281322139	0.570939154289558\\
1.35	0.780817519965826	0.975683883461272	0.566123121053175\\
1.35	0.795721169365275	0.97891274348356	0.561279136605557\\
1.35	0.810604533136476	0.981900889668376	0.556409397399198\\
1.35	0.825462312667457	0.984650494185936	0.551516079695737\\
1.35	0.840289325888138	0.987163867133889	0.546601337319641\\
1.35	0.855080510976839	0.989443450481768	0.541667299515139\\
1.35	0.86983092974082	0.991491811941914	0.536716068908598\\
1.35	0.884535770672926	0.993311638785082	0.531749719578141\\
1.35	0.899190351687411	0.994905731618374	0.526770295231796\\
1.35	0.913790122539012	0.996276998142666	0.521779807495105\\
1.35	0.928330666930242	0.997428446906011	0.516780234308684\\
1.35	0.942807704312699	0.998363181068902	0.511773518435872\\
1.35	0.95721709138901	0.999084392196567	0.506761566080168\\
1.35	0.971554823322703	0.999595354092743	0.501746245611874\\
1.35	0.985817034663989	0.999899416688637	0.496729386403004\\
1.35	1	1	0.491669087346906\\
1.365	0	0	0.710195569525497\\
1.365	0.000100583311362513	0.0141829653360114	0.71240460418299\\
1.365	0.000404645907256436	0.0284451766772965	0.714630690536905\\
1.365	0.000915607803433	0.0427829086109896	0.716916312064106\\
1.365	0.00163681893109844	0.057192295687301	0.719289019360234\\
1.365	0.00257155309398959	0.0716693330697585	0.721780367061016\\
1.365	0.00372300185733414	0.086209877460988	0.724425655492835\\
1.365	0.00509426838162598	0.100809648312589	0.727263413850304\\
1.365	0.00668836121491816	0.115464229327074	0.73033458997498\\
1.365	0.00850818805808555	0.13016907025918	0.733681422514709\\
1.365	0.0105565495182326	0.144919489023162	0.737345986077549\\
1.365	0.0128361328661109	0.159710674111862	0.741368418705642\\
1.365	0.0153495058140643	0.174537687332543	0.745784862958201\\
1.365	0.0180991103316243	0.189395466863524	0.750625176092311\\
1.365	0.0210872565164405	0.204278830634725	0.755910489868998\\
1.365	0.0243161165387281	0.219182480034174	0.761650724664453\\
1.365	0.0277877186778607	0.234101003941464	0.767842183867616\\
1.365	0.0315039414701067	0.24902888308801	0.77446537092041\\
1.365	0.0354665079868145	0.263960494742775	0.781483180780606\\
1.365	0.0396769802625738	0.278890117720924	0.788839618268073\\
1.365	0.0441367538930258	0.293811937711588	0.796459186325544\\
1.365	0.0488470528220538	0.308720052919643	0.804247066923454\\
1.365	0.0538089243380495	0.323608480015096	0.81209018616195\\
1.365	0.0590232342988274	0.338471160382329	0.819859213932694\\
1.365	0.064490662604533	0.3533019666601	0.827411499063639\\
1.365	0.0702116989375697	0.368094709561873	0.834594885813504\\
1.365	0.0761866387881433	0.382843144964686	0.841252300287179\\
1.365	0.0824155797834956	0.397540981253432	0.847226939729081\\
1.365	0.0888984183382709	0.412181886906127	0.852367847905041\\
1.365	0.0956348466427212	0.42675949830445	0.85653562002783\\
1.365	0.102624350004627	0.441267427752584	0.859607954619618\\
1.365	0.109866204559871	0.455699271686218	0.861484760266622\\
1.365	0.117359475365564	0.470048619052377	0.862092534251542\\
1.365	0.125103014888515	0.484309059839721	0.861387758068369\\
1.365	0.133095461900593	0.498474193737904	0.85935910089082\\
1.365	0.14133524079124	0.512537638903683	0.856028283765734\\
1.365	0.149820561306021	0.526493040810599	0.851449530870183\\
1.365	0.158549418718625	0.540334081158348	0.845707614739506\\
1.365	0.167519594442214	0.554054486817265	0.83891458434315\\
1.365	0.176728657084455	0.567648038782867	0.831205342370839\\
1.365	0.186173963948925	0.581108581114919	0.822732305417528\\
1.365	0.195852662983903	0.594430029835237	0.813659432944309\\
1.365	0.205761695177907	0.607606381758231	0.804155944104868\\
1.365	0.215897797399558	0.620631723228143	0.794390053406969\\
1.365	0.226257505677663	0.633500238737009	0.7845230460628\\
1.365	0.236837158915675	0.646206219397571	0.774703982861589\\
1.365	0.247632903032949	0.658744071245709	0.765065275211016\\
1.365	0.258640695523528	0.671108323347402	0.755719307772964\\
1.365	0.269856310421543	0.683293635685828	0.746756214005914\\
1.365	0.28127534366066	0.695294806804925	0.738242834604294\\
1.365	0.292893218813452	0.707106781186547	0.730222816007836\\
1.365	0.304705193195075	0.71872465633934	0.722717741087767\\
1.365	0.316706364314172	0.730143689578457	0.715729131143265\\
1.365	0.328891676652598	0.741359304476472	0.709241120561857\\
1.365	0.341255928754291	0.752367096967051	0.703223584571265\\
1.365	0.353793780602429	0.763162841084325	0.697635496597417\\
1.365	0.366499761262991	0.773742494322337	0.692428303594618\\
1.365	0.379368276771857	0.784102202600442	0.687549132887069\\
1.365	0.392393618241769	0.794238304822092	0.682943679235329\\
1.365	0.405569970164763	0.804147337016097	0.678558662193046\\
1.365	0.418891418885081	0.813826036051075	0.674343787412872\\
1.365	0.432351961217133	0.823271342915545	0.670253187715552\\
1.365	0.445945513182735	0.832480405557787	0.666246357336158\\
1.365	0.459665918841652	0.841450581281375	0.662288623474185\\
1.365	0.473506959189401	0.850179438693979	0.658351221673191\\
1.365	0.487462361096317	0.85866475920876	0.654411055137887\\
1.365	0.501525806262096	0.866904538099407	0.650450223206467\\
1.365	0.515690940160279	0.874896985111485	0.646455401874959\\
1.365	0.529951380947623	0.882640524634437	0.642417151055155\\
1.365	0.544300728313782	0.890133795440129	0.638329210947585\\
1.365	0.558732572247415	0.897375649995373	0.634187835387498\\
1.365	0.57324050169555	0.904365153357279	0.629991194998376\\
1.365	0.587818113093873	0.911101581661729	0.625738868900924\\
1.365	0.602459018746568	0.917584420216504	0.62143143163314\\
1.365	0.617156855035314	0.923813361211857	0.617070132480554\\
1.365	0.631905290438127	0.92978830106243	0.612656657833838\\
1.365	0.6466980333399	0.935509337395467	0.608192963377087\\
1.365	0.661528839617671	0.940976765701173	0.603681161495105\\
1.365	0.676391519984904	0.946191075661951	0.599123449738607\\
1.365	0.691279947080357	0.951152947177946	0.59452206790159\\
1.365	0.706188062288412	0.955863246106974	0.589879273662389\\
1.365	0.721109882279076	0.960323019737426	0.585197329319936\\
1.365	0.736039505257225	0.964533492013186	0.580478494543978\\
1.365	0.75097111691199	0.968496058529893	0.575725022018313\\
1.365	0.765898996058536	0.972212281322139	0.570939154289558\\
1.365	0.780817519965826	0.975683883461272	0.566123121053175\\
1.365	0.795721169365275	0.97891274348356	0.561279136605558\\
1.365	0.810604533136476	0.981900889668376	0.556409397399197\\
1.365	0.825462312667457	0.984650494185936	0.551516079695736\\
1.365	0.840289325888138	0.987163867133889	0.546601337319641\\
1.365	0.855080510976839	0.989443450481768	0.541667299515139\\
1.365	0.86983092974082	0.991491811941914	0.536716068908598\\
1.365	0.884535770672926	0.993311638785082	0.531749719578141\\
1.365	0.899190351687411	0.994905731618374	0.526770295231796\\
1.365	0.913790122539012	0.996276998142666	0.521779807495104\\
1.365	0.928330666930242	0.997428446906011	0.516780234308684\\
1.365	0.942807704312699	0.998363181068902	0.511773518435872\\
1.365	0.95721709138901	0.999084392196567	0.506761566080168\\
1.365	0.971554823322703	0.999595354092743	0.501746245611874\\
1.365	0.985817034663989	0.999899416688637	0.496729386403004\\
1.365	1	1	0.491669087346906\\
1.38	0	0	0.710195569525497\\
1.38	0.000100583311362513	0.0141829653360114	0.71240460418299\\
1.38	0.000404645907256436	0.0284451766772965	0.714630690536905\\
1.38	0.000915607803433	0.0427829086109896	0.716916312064106\\
1.38	0.00163681893109844	0.057192295687301	0.719289019360234\\
1.38	0.00257155309398959	0.0716693330697585	0.721780367061016\\
1.38	0.00372300185733414	0.0862098774609879	0.724425655492835\\
1.38	0.00509426838162598	0.100809648312589	0.727263413850304\\
1.38	0.00668836121491816	0.115464229327074	0.73033458997498\\
1.38	0.00850818805808555	0.13016907025918	0.733681422514709\\
1.38	0.0105565495182326	0.144919489023162	0.737345986077549\\
1.38	0.0128361328661109	0.159710674111862	0.741368418705642\\
1.38	0.0153495058140643	0.174537687332543	0.745784862958202\\
1.38	0.0180991103316243	0.189395466863524	0.750625176092311\\
1.38	0.0210872565164405	0.204278830634725	0.755910489868998\\
1.38	0.0243161165387281	0.219182480034174	0.761650724664453\\
1.38	0.0277877186778607	0.234101003941464	0.767842183867616\\
1.38	0.0315039414701067	0.24902888308801	0.77446537092041\\
1.38	0.0354665079868145	0.263960494742775	0.781483180780606\\
1.38	0.0396769802625738	0.278890117720924	0.788839618268073\\
1.38	0.0441367538930258	0.293811937711588	0.796459186325544\\
1.38	0.0488470528220538	0.308720052919643	0.804247066923454\\
1.38	0.0538089243380495	0.323608480015096	0.81209018616195\\
1.38	0.0590232342988274	0.338471160382329	0.819859213932694\\
1.38	0.064490662604533	0.3533019666601	0.827411499063639\\
1.38	0.0702116989375697	0.368094709561873	0.834594885813504\\
1.38	0.0761866387881432	0.382843144964686	0.841252300287178\\
1.38	0.0824155797834956	0.397540981253432	0.847226939729081\\
1.38	0.0888984183382709	0.412181886906127	0.852367847905041\\
1.38	0.0956348466427212	0.42675949830445	0.856535620027829\\
1.38	0.102624350004627	0.441267427752584	0.859607954619617\\
1.38	0.109866204559871	0.455699271686219	0.861484760266623\\
1.38	0.117359475365564	0.470048619052377	0.86209253425154\\
1.38	0.125103014888515	0.484309059839721	0.861387758068367\\
1.38	0.133095461900593	0.498474193737904	0.85935910089082\\
1.38	0.14133524079124	0.512537638903683	0.856028283765736\\
1.38	0.149820561306021	0.526493040810599	0.851449530870183\\
1.38	0.158549418718625	0.540334081158348	0.845707614739505\\
1.38	0.167519594442214	0.554054486817265	0.838914584343151\\
1.38	0.176728657084455	0.567648038782868	0.831205342370836\\
1.38	0.186173963948925	0.581108581114919	0.822732305417527\\
1.38	0.195852662983903	0.594430029835237	0.813659432944309\\
1.38	0.205761695177907	0.607606381758231	0.804155944104868\\
1.38	0.215897797399558	0.620631723228143	0.794390053406969\\
1.38	0.226257505677663	0.633500238737009	0.7845230460628\\
1.38	0.236837158915675	0.646206219397571	0.774703982861589\\
1.38	0.247632903032949	0.658744071245709	0.765065275211016\\
1.38	0.258640695523528	0.671108323347402	0.755719307772964\\
1.38	0.269856310421543	0.683293635685828	0.746756214005914\\
1.38	0.28127534366066	0.695294806804925	0.738242834604294\\
1.38	0.292893218813452	0.707106781186547	0.730222816007836\\
1.38	0.304705193195075	0.71872465633934	0.722717741087767\\
1.38	0.316706364314172	0.730143689578457	0.715729131143265\\
1.38	0.328891676652598	0.741359304476472	0.709241120561857\\
1.38	0.341255928754291	0.752367096967051	0.703223584571265\\
1.38	0.353793780602429	0.763162841084325	0.697635496597417\\
1.38	0.366499761262991	0.773742494322337	0.692428303594618\\
1.38	0.379368276771857	0.784102202600442	0.687549132887069\\
1.38	0.392393618241769	0.794238304822092	0.682943679235329\\
1.38	0.405569970164763	0.804147337016097	0.678558662193046\\
1.38	0.418891418885081	0.813826036051075	0.674343787412872\\
1.38	0.432351961217133	0.823271342915545	0.670253187715552\\
1.38	0.445945513182735	0.832480405557787	0.666246357336158\\
1.38	0.459665918841652	0.841450581281375	0.662288623474185\\
1.38	0.473506959189401	0.850179438693979	0.658351221673191\\
1.38	0.487462361096317	0.85866475920876	0.654411055137886\\
1.38	0.501525806262096	0.866904538099407	0.650450223206468\\
1.38	0.515690940160279	0.874896985111485	0.646455401874959\\
1.38	0.529951380947623	0.882640524634437	0.642417151055155\\
1.38	0.544300728313782	0.890133795440129	0.638329210947585\\
1.38	0.558732572247415	0.897375649995373	0.634187835387498\\
1.38	0.57324050169555	0.904365153357279	0.629991194998376\\
1.38	0.587818113093873	0.911101581661729	0.625738868900924\\
1.38	0.602459018746568	0.917584420216504	0.62143143163314\\
1.38	0.617156855035314	0.923813361211857	0.617070132480554\\
1.38	0.631905290438127	0.92978830106243	0.612656657833838\\
1.38	0.6466980333399	0.935509337395467	0.608192963377087\\
1.38	0.661528839617671	0.940976765701173	0.603681161495105\\
1.38	0.676391519984904	0.946191075661951	0.599123449738607\\
1.38	0.691279947080357	0.951152947177946	0.59452206790159\\
1.38	0.706188062288412	0.955863246106974	0.589879273662389\\
1.38	0.721109882279076	0.960323019737426	0.585197329319936\\
1.38	0.736039505257225	0.964533492013186	0.580478494543978\\
1.38	0.75097111691199	0.968496058529893	0.575725022018313\\
1.38	0.765898996058536	0.972212281322139	0.570939154289558\\
1.38	0.780817519965826	0.975683883461272	0.566123121053175\\
1.38	0.795721169365275	0.97891274348356	0.561279136605558\\
1.38	0.810604533136476	0.981900889668376	0.556409397399198\\
1.38	0.825462312667457	0.984650494185936	0.551516079695736\\
1.38	0.840289325888138	0.987163867133889	0.546601337319641\\
1.38	0.855080510976839	0.989443450481768	0.541667299515139\\
1.38	0.86983092974082	0.991491811941914	0.536716068908598\\
1.38	0.884535770672926	0.993311638785082	0.531749719578141\\
1.38	0.899190351687411	0.994905731618374	0.526770295231796\\
1.38	0.913790122539012	0.996276998142666	0.521779807495104\\
1.38	0.928330666930242	0.997428446906011	0.516780234308684\\
1.38	0.942807704312699	0.998363181068902	0.511773518435872\\
1.38	0.95721709138901	0.999084392196567	0.506761566080168\\
1.38	0.971554823322703	0.999595354092743	0.501746245611874\\
1.38	0.985817034663989	0.999899416688637	0.496729386403004\\
1.38	1	1	0.491669087346906\\
1.395	0	0	0.710195569525497\\
1.395	0.000100583311362513	0.0141829653360114	0.71240460418299\\
1.395	0.000404645907256436	0.0284451766772965	0.714630690536905\\
1.395	0.000915607803433	0.0427829086109896	0.716916312064106\\
1.395	0.00163681893109844	0.057192295687301	0.719289019360234\\
1.395	0.00257155309398959	0.0716693330697585	0.721780367061016\\
1.395	0.00372300185733414	0.086209877460988	0.724425655492835\\
1.395	0.00509426838162598	0.100809648312589	0.727263413850304\\
1.395	0.00668836121491816	0.115464229327074	0.73033458997498\\
1.395	0.00850818805808555	0.13016907025918	0.733681422514709\\
1.395	0.0105565495182326	0.144919489023162	0.737345986077549\\
1.395	0.0128361328661109	0.159710674111862	0.741368418705642\\
1.395	0.0153495058140643	0.174537687332543	0.745784862958202\\
1.395	0.0180991103316243	0.189395466863524	0.750625176092311\\
1.395	0.0210872565164405	0.204278830634725	0.755910489868998\\
1.395	0.0243161165387281	0.219182480034174	0.761650724664453\\
1.395	0.0277877186778607	0.234101003941464	0.767842183867616\\
1.395	0.0315039414701067	0.24902888308801	0.77446537092041\\
1.395	0.0354665079868145	0.263960494742775	0.781483180780606\\
1.395	0.0396769802625738	0.278890117720924	0.788839618268073\\
1.395	0.0441367538930258	0.293811937711588	0.796459186325544\\
1.395	0.0488470528220538	0.308720052919643	0.804247066923454\\
1.395	0.0538089243380495	0.323608480015096	0.81209018616195\\
1.395	0.0590232342988274	0.338471160382329	0.819859213932694\\
1.395	0.064490662604533	0.3533019666601	0.827411499063639\\
1.395	0.0702116989375697	0.368094709561873	0.834594885813506\\
1.395	0.0761866387881432	0.382843144964686	0.841252300287179\\
1.395	0.0824155797834956	0.397540981253432	0.847226939729081\\
1.395	0.0888984183382709	0.412181886906127	0.852367847905041\\
1.395	0.0956348466427212	0.42675949830445	0.85653562002783\\
1.395	0.102624350004627	0.441267427752584	0.859607954619617\\
1.395	0.109866204559871	0.455699271686219	0.86148476026662\\
1.395	0.117359475365564	0.470048619052377	0.86209253425154\\
1.395	0.125103014888515	0.484309059839721	0.861387758068368\\
1.395	0.133095461900593	0.498474193737904	0.859359100890821\\
1.395	0.14133524079124	0.512537638903683	0.856028283765736\\
1.395	0.149820561306021	0.526493040810599	0.851449530870183\\
1.395	0.158549418718625	0.540334081158348	0.845707614739505\\
1.395	0.167519594442214	0.554054486817265	0.83891458434315\\
1.395	0.176728657084455	0.567648038782867	0.831205342370833\\
1.395	0.186173963948925	0.581108581114919	0.822732305417528\\
1.395	0.195852662983903	0.594430029835237	0.813659432944309\\
1.395	0.205761695177907	0.607606381758231	0.804155944104868\\
1.395	0.215897797399558	0.620631723228143	0.794390053406969\\
1.395	0.226257505677663	0.633500238737009	0.7845230460628\\
1.395	0.236837158915675	0.646206219397571	0.774703982861589\\
1.395	0.247632903032949	0.658744071245709	0.765065275211016\\
1.395	0.258640695523528	0.671108323347402	0.755719307772964\\
1.395	0.269856310421543	0.683293635685828	0.746756214005914\\
1.395	0.28127534366066	0.695294806804925	0.738242834604294\\
1.395	0.292893218813452	0.707106781186547	0.730222816007836\\
1.395	0.304705193195075	0.71872465633934	0.722717741087767\\
1.395	0.316706364314172	0.730143689578457	0.715729131143265\\
1.395	0.328891676652598	0.741359304476472	0.709241120561857\\
1.395	0.341255928754291	0.752367096967051	0.703223584571265\\
1.395	0.353793780602429	0.763162841084325	0.697635496597417\\
1.395	0.366499761262991	0.773742494322337	0.692428303594618\\
1.395	0.379368276771857	0.784102202600442	0.687549132887069\\
1.395	0.392393618241769	0.794238304822092	0.682943679235329\\
1.395	0.405569970164763	0.804147337016097	0.678558662193046\\
1.395	0.418891418885081	0.813826036051075	0.674343787412872\\
1.395	0.432351961217133	0.823271342915545	0.670253187715552\\
1.395	0.445945513182735	0.832480405557787	0.666246357336158\\
1.395	0.459665918841652	0.841450581281375	0.662288623474185\\
1.395	0.473506959189401	0.850179438693979	0.658351221673191\\
1.395	0.487462361096317	0.85866475920876	0.654411055137886\\
1.395	0.501525806262096	0.866904538099407	0.650450223206467\\
1.395	0.515690940160279	0.874896985111485	0.646455401874959\\
1.395	0.529951380947623	0.882640524634437	0.642417151055155\\
1.395	0.544300728313782	0.890133795440129	0.638329210947585\\
1.395	0.558732572247415	0.897375649995373	0.634187835387498\\
1.395	0.57324050169555	0.904365153357279	0.629991194998376\\
1.395	0.587818113093873	0.911101581661729	0.625738868900924\\
1.395	0.602459018746568	0.917584420216504	0.62143143163314\\
1.395	0.617156855035314	0.923813361211857	0.617070132480554\\
1.395	0.631905290438127	0.92978830106243	0.612656657833838\\
1.395	0.6466980333399	0.935509337395467	0.608192963377087\\
1.395	0.661528839617671	0.940976765701173	0.603681161495105\\
1.395	0.676391519984904	0.946191075661951	0.599123449738607\\
1.395	0.691279947080357	0.951152947177946	0.59452206790159\\
1.395	0.706188062288412	0.955863246106974	0.589879273662389\\
1.395	0.721109882279076	0.960323019737426	0.585197329319936\\
1.395	0.736039505257225	0.964533492013186	0.580478494543978\\
1.395	0.75097111691199	0.968496058529893	0.575725022018313\\
1.395	0.765898996058536	0.972212281322139	0.570939154289557\\
1.395	0.780817519965826	0.975683883461272	0.566123121053175\\
1.395	0.795721169365275	0.97891274348356	0.561279136605558\\
1.395	0.810604533136476	0.981900889668376	0.556409397399198\\
1.395	0.825462312667457	0.984650494185936	0.551516079695736\\
1.395	0.840289325888138	0.987163867133889	0.546601337319641\\
1.395	0.855080510976839	0.989443450481768	0.541667299515139\\
1.395	0.86983092974082	0.991491811941914	0.536716068908598\\
1.395	0.884535770672926	0.993311638785082	0.531749719578142\\
1.395	0.899190351687411	0.994905731618374	0.526770295231796\\
1.395	0.913790122539012	0.996276998142666	0.521779807495103\\
1.395	0.928330666930242	0.997428446906011	0.516780234308684\\
1.395	0.942807704312699	0.998363181068902	0.511773518435872\\
1.395	0.95721709138901	0.999084392196567	0.506761566080168\\
1.395	0.971554823322703	0.999595354092743	0.501746245611874\\
1.395	0.985817034663989	0.999899416688637	0.496729386403004\\
1.395	1	1	0.491669087346906\\
1.41	0	0	0.710195569525497\\
1.41	0.000100583311362513	0.0141829653360114	0.71240460418299\\
1.41	0.000404645907256436	0.0284451766772965	0.714630690536905\\
1.41	0.000915607803433	0.0427829086109896	0.716916312064106\\
1.41	0.00163681893109844	0.057192295687301	0.719289019360234\\
1.41	0.00257155309398959	0.0716693330697585	0.721780367061016\\
1.41	0.00372300185733414	0.086209877460988	0.724425655492835\\
1.41	0.00509426838162598	0.100809648312589	0.727263413850304\\
1.41	0.00668836121491816	0.115464229327074	0.73033458997498\\
1.41	0.00850818805808555	0.13016907025918	0.733681422514709\\
1.41	0.0105565495182326	0.144919489023162	0.737345986077549\\
1.41	0.0128361328661109	0.159710674111862	0.741368418705642\\
1.41	0.0153495058140643	0.174537687332543	0.745784862958201\\
1.41	0.0180991103316243	0.189395466863524	0.750625176092311\\
1.41	0.0210872565164405	0.204278830634725	0.755910489868998\\
1.41	0.0243161165387281	0.219182480034174	0.761650724664453\\
1.41	0.0277877186778607	0.234101003941464	0.767842183867616\\
1.41	0.0315039414701067	0.24902888308801	0.77446537092041\\
1.41	0.0354665079868145	0.263960494742775	0.781483180780606\\
1.41	0.0396769802625738	0.278890117720924	0.788839618268073\\
1.41	0.0441367538930258	0.293811937711588	0.796459186325544\\
1.41	0.0488470528220538	0.308720052919643	0.804247066923454\\
1.41	0.0538089243380495	0.323608480015096	0.812090186161949\\
1.41	0.0590232342988274	0.338471160382329	0.819859213932694\\
1.41	0.064490662604533	0.3533019666601	0.827411499063639\\
1.41	0.0702116989375697	0.368094709561873	0.834594885813504\\
1.41	0.0761866387881432	0.382843144964686	0.84125230028718\\
1.41	0.0824155797834956	0.397540981253432	0.847226939729081\\
1.41	0.0888984183382709	0.412181886906127	0.852367847905041\\
1.41	0.0956348466427212	0.42675949830445	0.856535620027829\\
1.41	0.102624350004627	0.441267427752584	0.859607954619618\\
1.41	0.109866204559871	0.455699271686218	0.86148476026662\\
1.41	0.117359475365564	0.470048619052377	0.86209253425154\\
1.41	0.125103014888515	0.484309059839721	0.861387758068368\\
1.41	0.133095461900593	0.498474193737904	0.85935910089082\\
1.41	0.14133524079124	0.512537638903683	0.856028283765736\\
1.41	0.149820561306021	0.526493040810599	0.851449530870183\\
1.41	0.158549418718625	0.540334081158348	0.845707614739505\\
1.41	0.167519594442214	0.554054486817265	0.83891458434315\\
1.41	0.176728657084455	0.567648038782867	0.831205342370836\\
1.41	0.186173963948925	0.581108581114919	0.822732305417528\\
1.41	0.195852662983903	0.594430029835237	0.813659432944309\\
1.41	0.205761695177907	0.607606381758231	0.804155944104868\\
1.41	0.215897797399558	0.620631723228143	0.794390053406969\\
1.41	0.226257505677663	0.633500238737009	0.7845230460628\\
1.41	0.236837158915675	0.646206219397571	0.774703982861589\\
1.41	0.247632903032949	0.658744071245709	0.765065275211016\\
1.41	0.258640695523528	0.671108323347402	0.755719307772964\\
1.41	0.269856310421543	0.683293635685828	0.746756214005914\\
1.41	0.28127534366066	0.695294806804925	0.738242834604294\\
1.41	0.292893218813452	0.707106781186547	0.730222816007836\\
1.41	0.304705193195075	0.71872465633934	0.722717741087767\\
1.41	0.316706364314172	0.730143689578457	0.715729131143265\\
1.41	0.328891676652598	0.741359304476472	0.709241120561857\\
1.41	0.341255928754291	0.752367096967051	0.703223584571265\\
1.41	0.353793780602429	0.763162841084325	0.697635496597417\\
1.41	0.366499761262991	0.773742494322337	0.692428303594618\\
1.41	0.379368276771857	0.784102202600442	0.687549132887069\\
1.41	0.392393618241769	0.794238304822092	0.682943679235329\\
1.41	0.405569970164763	0.804147337016097	0.678558662193046\\
1.41	0.418891418885081	0.813826036051075	0.674343787412872\\
1.41	0.432351961217133	0.823271342915545	0.670253187715552\\
1.41	0.445945513182735	0.832480405557787	0.666246357336158\\
1.41	0.459665918841652	0.841450581281375	0.662288623474185\\
1.41	0.473506959189401	0.850179438693979	0.658351221673191\\
1.41	0.487462361096317	0.85866475920876	0.654411055137886\\
1.41	0.501525806262096	0.866904538099407	0.650450223206467\\
1.41	0.515690940160279	0.874896985111485	0.64645540187496\\
1.41	0.529951380947623	0.882640524634437	0.642417151055155\\
1.41	0.544300728313782	0.890133795440129	0.638329210947585\\
1.41	0.558732572247415	0.897375649995373	0.634187835387498\\
1.41	0.57324050169555	0.904365153357279	0.629991194998376\\
1.41	0.587818113093873	0.911101581661729	0.625738868900924\\
1.41	0.602459018746568	0.917584420216504	0.62143143163314\\
1.41	0.617156855035314	0.923813361211857	0.617070132480554\\
1.41	0.631905290438127	0.92978830106243	0.612656657833838\\
1.41	0.6466980333399	0.935509337395467	0.608192963377087\\
1.41	0.661528839617671	0.940976765701173	0.603681161495105\\
1.41	0.676391519984904	0.946191075661951	0.599123449738607\\
1.41	0.691279947080357	0.951152947177946	0.59452206790159\\
1.41	0.706188062288412	0.955863246106974	0.589879273662388\\
1.41	0.721109882279076	0.960323019737426	0.585197329319936\\
1.41	0.736039505257225	0.964533492013186	0.580478494543978\\
1.41	0.75097111691199	0.968496058529893	0.575725022018314\\
1.41	0.765898996058536	0.972212281322139	0.570939154289558\\
1.41	0.780817519965826	0.975683883461272	0.566123121053175\\
1.41	0.795721169365275	0.97891274348356	0.561279136605559\\
1.41	0.810604533136476	0.981900889668376	0.556409397399198\\
1.41	0.825462312667457	0.984650494185936	0.551516079695736\\
1.41	0.840289325888138	0.987163867133889	0.546601337319641\\
1.41	0.855080510976839	0.989443450481768	0.541667299515139\\
1.41	0.86983092974082	0.991491811941914	0.536716068908598\\
1.41	0.884535770672926	0.993311638785082	0.531749719578141\\
1.41	0.899190351687411	0.994905731618374	0.526770295231797\\
1.41	0.913790122539012	0.996276998142666	0.521779807495104\\
1.41	0.928330666930242	0.997428446906011	0.516780234308684\\
1.41	0.942807704312699	0.998363181068902	0.511773518435872\\
1.41	0.95721709138901	0.999084392196567	0.506761566080168\\
1.41	0.971554823322703	0.999595354092743	0.501746245611874\\
1.41	0.985817034663989	0.999899416688637	0.496729386403004\\
1.41	1	1	0.491669087346906\\
1.425	0	0	0.710195569525497\\
1.425	0.000100583311362513	0.0141829653360114	0.71240460418299\\
1.425	0.000404645907256436	0.0284451766772965	0.714630690536905\\
1.425	0.000915607803433	0.0427829086109896	0.716916312064106\\
1.425	0.00163681893109844	0.057192295687301	0.719289019360234\\
1.425	0.00257155309398959	0.0716693330697585	0.721780367061016\\
1.425	0.00372300185733414	0.086209877460988	0.724425655492835\\
1.425	0.00509426838162598	0.100809648312589	0.727263413850304\\
1.425	0.00668836121491816	0.115464229327074	0.73033458997498\\
1.425	0.00850818805808555	0.13016907025918	0.733681422514709\\
1.425	0.0105565495182326	0.144919489023162	0.737345986077549\\
1.425	0.0128361328661109	0.159710674111862	0.741368418705642\\
1.425	0.0153495058140643	0.174537687332543	0.745784862958201\\
1.425	0.0180991103316243	0.189395466863524	0.750625176092311\\
1.425	0.0210872565164405	0.204278830634725	0.755910489868998\\
1.425	0.0243161165387281	0.219182480034174	0.761650724664453\\
1.425	0.0277877186778607	0.234101003941464	0.767842183867616\\
1.425	0.0315039414701067	0.24902888308801	0.77446537092041\\
1.425	0.0354665079868145	0.263960494742775	0.781483180780606\\
1.425	0.0396769802625738	0.278890117720924	0.788839618268073\\
1.425	0.0441367538930258	0.293811937711588	0.796459186325544\\
1.425	0.0488470528220538	0.308720052919643	0.804247066923454\\
1.425	0.0538089243380495	0.323608480015096	0.81209018616195\\
1.425	0.0590232342988274	0.338471160382329	0.819859213932694\\
1.425	0.064490662604533	0.3533019666601	0.827411499063639\\
1.425	0.0702116989375697	0.368094709561873	0.834594885813504\\
1.425	0.0761866387881432	0.382843144964686	0.841252300287179\\
1.425	0.0824155797834956	0.397540981253432	0.847226939729081\\
1.425	0.0888984183382709	0.412181886906127	0.852367847905041\\
1.425	0.0956348466427213	0.42675949830445	0.856535620027829\\
1.425	0.102624350004627	0.441267427752584	0.859607954619617\\
1.425	0.109866204559871	0.455699271686219	0.861484760266623\\
1.425	0.117359475365564	0.470048619052377	0.86209253425154\\
1.425	0.125103014888515	0.484309059839721	0.861387758068367\\
1.425	0.133095461900593	0.498474193737904	0.859359100890819\\
1.425	0.14133524079124	0.512537638903683	0.856028283765736\\
1.425	0.149820561306021	0.526493040810599	0.851449530870183\\
1.425	0.158549418718625	0.540334081158348	0.845707614739505\\
1.425	0.167519594442214	0.554054486817265	0.838914584343152\\
1.425	0.176728657084455	0.567648038782867	0.831205342370836\\
1.425	0.186173963948925	0.581108581114919	0.822732305417528\\
1.425	0.195852662983903	0.594430029835237	0.813659432944309\\
1.425	0.205761695177907	0.607606381758231	0.804155944104868\\
1.425	0.215897797399558	0.620631723228143	0.794390053406969\\
1.425	0.226257505677663	0.633500238737009	0.7845230460628\\
1.425	0.236837158915675	0.646206219397571	0.774703982861589\\
1.425	0.247632903032949	0.658744071245709	0.765065275211016\\
1.425	0.258640695523528	0.671108323347402	0.755719307772964\\
1.425	0.269856310421543	0.683293635685828	0.746756214005914\\
1.425	0.28127534366066	0.695294806804925	0.738242834604294\\
1.425	0.292893218813452	0.707106781186547	0.730222816007836\\
1.425	0.304705193195075	0.71872465633934	0.722717741087767\\
1.425	0.316706364314172	0.730143689578457	0.715729131143265\\
1.425	0.328891676652598	0.741359304476472	0.709241120561857\\
1.425	0.341255928754291	0.752367096967051	0.703223584571265\\
1.425	0.353793780602429	0.763162841084325	0.697635496597417\\
1.425	0.366499761262991	0.773742494322337	0.692428303594618\\
1.425	0.379368276771857	0.784102202600442	0.687549132887069\\
1.425	0.392393618241769	0.794238304822092	0.682943679235329\\
1.425	0.405569970164763	0.804147337016097	0.678558662193046\\
1.425	0.418891418885081	0.813826036051075	0.674343787412872\\
1.425	0.432351961217133	0.823271342915545	0.670253187715552\\
1.425	0.445945513182735	0.832480405557787	0.666246357336158\\
1.425	0.459665918841652	0.841450581281375	0.662288623474185\\
1.425	0.473506959189401	0.850179438693979	0.658351221673191\\
1.425	0.487462361096317	0.85866475920876	0.654411055137886\\
1.425	0.501525806262096	0.866904538099407	0.650450223206467\\
1.425	0.515690940160279	0.874896985111485	0.646455401874959\\
1.425	0.529951380947623	0.882640524634437	0.642417151055155\\
1.425	0.544300728313782	0.890133795440129	0.638329210947585\\
1.425	0.558732572247415	0.897375649995373	0.634187835387498\\
1.425	0.57324050169555	0.904365153357279	0.629991194998376\\
1.425	0.587818113093873	0.911101581661729	0.625738868900924\\
1.425	0.602459018746568	0.917584420216504	0.62143143163314\\
1.425	0.617156855035314	0.923813361211857	0.617070132480554\\
1.425	0.631905290438127	0.92978830106243	0.612656657833838\\
1.425	0.6466980333399	0.935509337395467	0.608192963377087\\
1.425	0.661528839617671	0.940976765701173	0.603681161495105\\
1.425	0.676391519984904	0.946191075661951	0.599123449738607\\
1.425	0.691279947080357	0.951152947177946	0.594522067901591\\
1.425	0.706188062288412	0.955863246106974	0.589879273662389\\
1.425	0.721109882279076	0.960323019737426	0.585197329319934\\
1.425	0.736039505257225	0.964533492013186	0.580478494543978\\
1.425	0.75097111691199	0.968496058529893	0.575725022018313\\
1.425	0.765898996058536	0.972212281322139	0.570939154289558\\
1.425	0.780817519965826	0.975683883461272	0.566123121053174\\
1.425	0.795721169365275	0.97891274348356	0.561279136605558\\
1.425	0.810604533136476	0.981900889668376	0.556409397399199\\
1.425	0.825462312667457	0.984650494185936	0.551516079695736\\
1.425	0.840289325888138	0.987163867133889	0.546601337319641\\
1.425	0.855080510976839	0.989443450481767	0.541667299515139\\
1.425	0.86983092974082	0.991491811941914	0.536716068908598\\
1.425	0.884535770672926	0.993311638785082	0.531749719578141\\
1.425	0.899190351687411	0.994905731618374	0.526770295231796\\
1.425	0.913790122539012	0.996276998142666	0.521779807495104\\
1.425	0.928330666930242	0.997428446906011	0.516780234308684\\
1.425	0.942807704312699	0.998363181068902	0.511773518435872\\
1.425	0.95721709138901	0.999084392196567	0.506761566080168\\
1.425	0.971554823322703	0.999595354092743	0.501746245611874\\
1.425	0.985817034663989	0.999899416688637	0.496729386403004\\
1.425	1	1	0.491669087346906\\
1.44	0	0	0.710195569525497\\
1.44	0.000100583311362513	0.0141829653360114	0.71240460418299\\
1.44	0.000404645907256436	0.0284451766772965	0.714630690536905\\
1.44	0.000915607803433	0.0427829086109896	0.716916312064106\\
1.44	0.00163681893109844	0.057192295687301	0.719289019360234\\
1.44	0.00257155309398959	0.0716693330697585	0.721780367061016\\
1.44	0.00372300185733414	0.086209877460988	0.724425655492835\\
1.44	0.00509426838162598	0.100809648312589	0.727263413850304\\
1.44	0.00668836121491816	0.115464229327074	0.73033458997498\\
1.44	0.00850818805808555	0.13016907025918	0.733681422514709\\
1.44	0.0105565495182326	0.144919489023162	0.737345986077549\\
1.44	0.0128361328661109	0.159710674111862	0.741368418705642\\
1.44	0.0153495058140643	0.174537687332543	0.745784862958202\\
1.44	0.0180991103316243	0.189395466863524	0.750625176092311\\
1.44	0.0210872565164405	0.204278830634725	0.755910489868998\\
1.44	0.0243161165387281	0.219182480034174	0.761650724664453\\
1.44	0.0277877186778607	0.234101003941464	0.767842183867616\\
1.44	0.0315039414701067	0.24902888308801	0.77446537092041\\
1.44	0.0354665079868145	0.263960494742775	0.781483180780606\\
1.44	0.0396769802625738	0.278890117720924	0.788839618268073\\
1.44	0.0441367538930258	0.293811937711588	0.796459186325544\\
1.44	0.0488470528220538	0.308720052919643	0.804247066923455\\
1.44	0.0538089243380495	0.323608480015096	0.81209018616195\\
1.44	0.0590232342988274	0.338471160382329	0.819859213932694\\
1.44	0.064490662604533	0.3533019666601	0.827411499063639\\
1.44	0.0702116989375697	0.368094709561873	0.834594885813505\\
1.44	0.0761866387881432	0.382843144964686	0.841252300287179\\
1.44	0.0824155797834956	0.397540981253432	0.847226939729081\\
1.44	0.0888984183382709	0.412181886906127	0.852367847905041\\
1.44	0.0956348466427212	0.42675949830445	0.85653562002783\\
1.44	0.102624350004627	0.441267427752584	0.859607954619617\\
1.44	0.109866204559871	0.455699271686219	0.861484760266622\\
1.44	0.117359475365564	0.470048619052377	0.86209253425154\\
1.44	0.125103014888515	0.484309059839721	0.861387758068368\\
1.44	0.133095461900593	0.498474193737904	0.85935910089082\\
1.44	0.14133524079124	0.512537638903683	0.856028283765736\\
1.44	0.149820561306021	0.526493040810599	0.851449530870183\\
1.44	0.158549418718625	0.540334081158348	0.845707614739506\\
1.44	0.167519594442214	0.554054486817265	0.83891458434315\\
1.44	0.176728657084455	0.567648038782867	0.831205342370836\\
1.44	0.186173963948925	0.581108581114919	0.822732305417528\\
1.44	0.195852662983903	0.594430029835237	0.813659432944309\\
1.44	0.205761695177907	0.607606381758231	0.804155944104868\\
1.44	0.215897797399558	0.620631723228143	0.794390053406969\\
1.44	0.226257505677663	0.633500238737009	0.7845230460628\\
1.44	0.236837158915675	0.646206219397571	0.774703982861589\\
1.44	0.247632903032949	0.658744071245709	0.765065275211016\\
1.44	0.258640695523528	0.671108323347402	0.755719307772964\\
1.44	0.269856310421543	0.683293635685828	0.746756214005914\\
1.44	0.28127534366066	0.695294806804925	0.738242834604294\\
1.44	0.292893218813452	0.707106781186547	0.730222816007836\\
1.44	0.304705193195075	0.71872465633934	0.722717741087767\\
1.44	0.316706364314172	0.730143689578457	0.715729131143265\\
1.44	0.328891676652598	0.741359304476472	0.709241120561857\\
1.44	0.341255928754291	0.752367096967051	0.703223584571265\\
1.44	0.353793780602429	0.763162841084325	0.697635496597417\\
1.44	0.366499761262991	0.773742494322337	0.692428303594618\\
1.44	0.379368276771857	0.784102202600442	0.687549132887069\\
1.44	0.392393618241769	0.794238304822092	0.682943679235329\\
1.44	0.405569970164763	0.804147337016097	0.678558662193046\\
1.44	0.418891418885081	0.813826036051075	0.674343787412872\\
1.44	0.432351961217133	0.823271342915545	0.670253187715552\\
1.44	0.445945513182735	0.832480405557787	0.666246357336158\\
1.44	0.459665918841652	0.841450581281375	0.662288623474185\\
1.44	0.473506959189401	0.850179438693979	0.658351221673191\\
1.44	0.487462361096317	0.85866475920876	0.654411055137886\\
1.44	0.501525806262096	0.866904538099407	0.650450223206467\\
1.44	0.515690940160279	0.874896985111485	0.646455401874959\\
1.44	0.529951380947623	0.882640524634437	0.642417151055155\\
1.44	0.544300728313782	0.890133795440129	0.638329210947585\\
1.44	0.558732572247415	0.897375649995373	0.634187835387498\\
1.44	0.57324050169555	0.904365153357279	0.629991194998376\\
1.44	0.587818113093873	0.911101581661729	0.625738868900924\\
1.44	0.602459018746568	0.917584420216504	0.62143143163314\\
1.44	0.617156855035314	0.923813361211857	0.617070132480554\\
1.44	0.631905290438127	0.92978830106243	0.612656657833838\\
1.44	0.6466980333399	0.935509337395467	0.608192963377087\\
1.44	0.661528839617671	0.940976765701173	0.603681161495105\\
1.44	0.676391519984904	0.946191075661951	0.599123449738608\\
1.44	0.691279947080357	0.951152947177946	0.59452206790159\\
1.44	0.706188062288412	0.955863246106974	0.589879273662388\\
1.44	0.721109882279076	0.960323019737426	0.585197329319936\\
1.44	0.736039505257225	0.964533492013186	0.580478494543978\\
1.44	0.75097111691199	0.968496058529893	0.575725022018313\\
1.44	0.765898996058536	0.972212281322139	0.570939154289558\\
1.44	0.780817519965826	0.975683883461272	0.566123121053175\\
1.44	0.795721169365275	0.97891274348356	0.561279136605558\\
1.44	0.810604533136476	0.981900889668376	0.556409397399198\\
1.44	0.825462312667457	0.984650494185936	0.551516079695736\\
1.44	0.840289325888138	0.987163867133889	0.546601337319641\\
1.44	0.855080510976839	0.989443450481768	0.541667299515139\\
1.44	0.86983092974082	0.991491811941914	0.536716068908598\\
1.44	0.884535770672926	0.993311638785082	0.531749719578141\\
1.44	0.899190351687411	0.994905731618374	0.526770295231796\\
1.44	0.913790122539012	0.996276998142666	0.521779807495104\\
1.44	0.928330666930242	0.997428446906011	0.516780234308684\\
1.44	0.942807704312699	0.998363181068902	0.511773518435872\\
1.44	0.95721709138901	0.999084392196567	0.50676156608017\\
1.44	0.971554823322704	0.999595354092744	0.501746245611874\\
1.44	0.985817034663989	0.999899416688637	0.496729386403002\\
1.44	1	1	0.491669087346908\\
1.455	0	0	0.710195569525497\\
1.455	0.000100583311362513	0.0141829653360114	0.71240460418299\\
1.455	0.000404645907256436	0.0284451766772965	0.714630690536905\\
1.455	0.000915607803433	0.0427829086109896	0.716916312064106\\
1.455	0.00163681893109844	0.057192295687301	0.719289019360234\\
1.455	0.00257155309398959	0.0716693330697585	0.721780367061016\\
1.455	0.00372300185733414	0.086209877460988	0.724425655492835\\
1.455	0.00509426838162598	0.100809648312589	0.727263413850304\\
1.455	0.00668836121491816	0.115464229327074	0.73033458997498\\
1.455	0.00850818805808555	0.13016907025918	0.733681422514709\\
1.455	0.0105565495182326	0.144919489023162	0.737345986077549\\
1.455	0.0128361328661109	0.159710674111862	0.741368418705642\\
1.455	0.0153495058140643	0.174537687332543	0.745784862958201\\
1.455	0.0180991103316243	0.189395466863524	0.750625176092311\\
1.455	0.0210872565164405	0.204278830634725	0.755910489868999\\
1.455	0.0243161165387281	0.219182480034174	0.761650724664453\\
1.455	0.0277877186778607	0.234101003941464	0.767842183867616\\
1.455	0.0315039414701067	0.24902888308801	0.77446537092041\\
1.455	0.0354665079868145	0.263960494742775	0.781483180780606\\
1.455	0.0396769802625738	0.278890117720924	0.788839618268073\\
1.455	0.0441367538930258	0.293811937711588	0.796459186325544\\
1.455	0.0488470528220538	0.308720052919643	0.804247066923454\\
1.455	0.0538089243380495	0.323608480015096	0.81209018616195\\
1.455	0.0590232342988274	0.338471160382329	0.819859213932694\\
1.455	0.064490662604533	0.3533019666601	0.827411499063639\\
1.455	0.0702116989375697	0.368094709561873	0.834594885813504\\
1.455	0.0761866387881432	0.382843144964686	0.841252300287179\\
1.455	0.0824155797834956	0.397540981253432	0.847226939729081\\
1.455	0.0888984183382709	0.412181886906127	0.852367847905041\\
1.455	0.0956348466427212	0.42675949830445	0.85653562002783\\
1.455	0.102624350004627	0.441267427752584	0.859607954619617\\
1.455	0.109866204559871	0.455699271686219	0.86148476026662\\
1.455	0.117359475365564	0.470048619052377	0.86209253425154\\
1.455	0.125103014888515	0.484309059839721	0.86138775806837\\
1.455	0.133095461900593	0.498474193737904	0.85935910089082\\
1.455	0.14133524079124	0.512537638903683	0.856028283765736\\
1.455	0.149820561306021	0.526493040810599	0.851449530870183\\
1.455	0.158549418718625	0.540334081158348	0.845707614739505\\
1.455	0.167519594442214	0.554054486817265	0.838914584343148\\
1.455	0.176728657084455	0.567648038782867	0.831205342370836\\
1.455	0.186173963948925	0.581108581114919	0.822732305417528\\
1.455	0.195852662983903	0.594430029835237	0.813659432944309\\
1.455	0.205761695177907	0.607606381758231	0.804155944104868\\
1.455	0.215897797399558	0.620631723228143	0.794390053406969\\
1.455	0.226257505677663	0.633500238737009	0.7845230460628\\
1.455	0.236837158915675	0.646206219397571	0.774703982861589\\
1.455	0.247632903032949	0.658744071245709	0.765065275211016\\
1.455	0.258640695523528	0.671108323347402	0.755719307772964\\
1.455	0.269856310421543	0.683293635685828	0.746756214005914\\
1.455	0.28127534366066	0.695294806804925	0.738242834604294\\
1.455	0.292893218813452	0.707106781186547	0.730222816007836\\
1.455	0.304705193195075	0.71872465633934	0.722717741087767\\
1.455	0.316706364314172	0.730143689578457	0.715729131143265\\
1.455	0.328891676652598	0.741359304476472	0.709241120561857\\
1.455	0.341255928754291	0.752367096967051	0.703223584571265\\
1.455	0.353793780602429	0.763162841084325	0.697635496597417\\
1.455	0.366499761262991	0.773742494322337	0.692428303594618\\
1.455	0.379368276771857	0.784102202600442	0.687549132887069\\
1.455	0.392393618241769	0.794238304822092	0.682943679235329\\
1.455	0.405569970164763	0.804147337016097	0.678558662193046\\
1.455	0.418891418885081	0.813826036051075	0.674343787412872\\
1.455	0.432351961217133	0.823271342915545	0.670253187715552\\
1.455	0.445945513182735	0.832480405557787	0.666246357336158\\
1.455	0.459665918841652	0.841450581281375	0.662288623474185\\
1.455	0.473506959189401	0.850179438693979	0.658351221673191\\
1.455	0.487462361096317	0.85866475920876	0.654411055137886\\
1.455	0.501525806262096	0.866904538099407	0.650450223206467\\
1.455	0.515690940160279	0.874896985111485	0.646455401874959\\
1.455	0.529951380947623	0.882640524634437	0.642417151055155\\
1.455	0.544300728313782	0.890133795440129	0.638329210947585\\
1.455	0.558732572247415	0.897375649995373	0.634187835387498\\
1.455	0.57324050169555	0.904365153357279	0.629991194998376\\
1.455	0.587818113093873	0.911101581661729	0.625738868900924\\
1.455	0.602459018746568	0.917584420216504	0.62143143163314\\
1.455	0.617156855035314	0.923813361211857	0.617070132480554\\
1.455	0.631905290438127	0.92978830106243	0.612656657833838\\
1.455	0.6466980333399	0.935509337395467	0.608192963377087\\
1.455	0.661528839617671	0.940976765701173	0.603681161495105\\
1.455	0.676391519984904	0.946191075661951	0.599123449738607\\
1.455	0.691279947080357	0.951152947177946	0.594522067901591\\
1.455	0.706188062288412	0.955863246106974	0.589879273662389\\
1.455	0.721109882279076	0.960323019737426	0.585197329319936\\
1.455	0.736039505257225	0.964533492013186	0.580478494543978\\
1.455	0.75097111691199	0.968496058529893	0.575725022018313\\
1.455	0.765898996058536	0.972212281322139	0.570939154289558\\
1.455	0.780817519965826	0.975683883461272	0.566123121053174\\
1.455	0.795721169365275	0.97891274348356	0.561279136605558\\
1.455	0.810604533136476	0.981900889668376	0.556409397399199\\
1.455	0.825462312667457	0.984650494185936	0.551516079695735\\
1.455	0.840289325888138	0.987163867133889	0.546601337319641\\
1.455	0.855080510976839	0.989443450481767	0.541667299515139\\
1.455	0.86983092974082	0.991491811941914	0.536716068908598\\
1.455	0.884535770672926	0.993311638785082	0.531749719578141\\
1.455	0.899190351687411	0.994905731618374	0.526770295231796\\
1.455	0.913790122539012	0.996276998142666	0.521779807495104\\
1.455	0.928330666930242	0.997428446906011	0.516780234308684\\
1.455	0.942807704312699	0.998363181068902	0.511773518435872\\
1.455	0.95721709138901	0.999084392196567	0.506761566080168\\
1.455	0.971554823322703	0.999595354092743	0.501746245611875\\
1.455	0.985817034663989	0.999899416688637	0.496729386403004\\
1.455	1	1	0.491669087346906\\
1.47	0	0	0.710195569525497\\
1.47	0.000100583311362513	0.0141829653360114	0.71240460418299\\
1.47	0.000404645907256436	0.0284451766772965	0.714630690536905\\
1.47	0.000915607803433	0.0427829086109896	0.716916312064106\\
1.47	0.00163681893109844	0.057192295687301	0.719289019360234\\
1.47	0.00257155309398959	0.0716693330697585	0.721780367061016\\
1.47	0.00372300185733414	0.086209877460988	0.724425655492835\\
1.47	0.00509426838162598	0.100809648312589	0.727263413850304\\
1.47	0.00668836121491816	0.115464229327074	0.73033458997498\\
1.47	0.00850818805808555	0.13016907025918	0.733681422514709\\
1.47	0.0105565495182326	0.144919489023162	0.737345986077549\\
1.47	0.0128361328661109	0.159710674111862	0.741368418705642\\
1.47	0.0153495058140643	0.174537687332543	0.745784862958201\\
1.47	0.0180991103316243	0.189395466863524	0.750625176092311\\
1.47	0.0210872565164405	0.204278830634725	0.755910489868998\\
1.47	0.0243161165387281	0.219182480034174	0.761650724664453\\
1.47	0.0277877186778607	0.234101003941464	0.767842183867616\\
1.47	0.0315039414701067	0.24902888308801	0.77446537092041\\
1.47	0.0354665079868145	0.263960494742775	0.781483180780606\\
1.47	0.0396769802625738	0.278890117720924	0.788839618268073\\
1.47	0.0441367538930258	0.293811937711588	0.796459186325544\\
1.47	0.0488470528220538	0.308720052919643	0.804247066923454\\
1.47	0.0538089243380495	0.323608480015096	0.81209018616195\\
1.47	0.0590232342988274	0.338471160382329	0.819859213932694\\
1.47	0.064490662604533	0.3533019666601	0.827411499063639\\
1.47	0.0702116989375697	0.368094709561873	0.834594885813504\\
1.47	0.0761866387881432	0.382843144964686	0.841252300287179\\
1.47	0.0824155797834956	0.397540981253432	0.847226939729081\\
1.47	0.0888984183382709	0.412181886906127	0.852367847905041\\
1.47	0.0956348466427212	0.42675949830445	0.85653562002783\\
1.47	0.102624350004627	0.441267427752584	0.859607954619618\\
1.47	0.109866204559871	0.455699271686218	0.861484760266622\\
1.47	0.117359475365564	0.470048619052377	0.86209253425154\\
1.47	0.125103014888515	0.484309059839721	0.861387758068369\\
1.47	0.133095461900593	0.498474193737904	0.85935910089082\\
1.47	0.14133524079124	0.512537638903683	0.856028283765736\\
1.47	0.149820561306021	0.526493040810599	0.851449530870183\\
1.47	0.158549418718625	0.540334081158348	0.845707614739505\\
1.47	0.167519594442214	0.554054486817265	0.83891458434315\\
1.47	0.176728657084455	0.567648038782867	0.831205342370836\\
1.47	0.186173963948925	0.581108581114919	0.822732305417528\\
1.47	0.195852662983903	0.594430029835237	0.813659432944309\\
1.47	0.205761695177907	0.607606381758231	0.804155944104868\\
1.47	0.215897797399558	0.620631723228143	0.794390053406969\\
1.47	0.226257505677663	0.633500238737009	0.7845230460628\\
1.47	0.236837158915675	0.646206219397571	0.774703982861589\\
1.47	0.247632903032949	0.658744071245709	0.765065275211016\\
1.47	0.258640695523528	0.671108323347402	0.755719307772964\\
1.47	0.269856310421543	0.683293635685828	0.746756214005914\\
1.47	0.28127534366066	0.695294806804925	0.738242834604294\\
1.47	0.292893218813452	0.707106781186547	0.730222816007836\\
1.47	0.304705193195075	0.71872465633934	0.722717741087767\\
1.47	0.316706364314172	0.730143689578457	0.715729131143265\\
1.47	0.328891676652598	0.741359304476472	0.709241120561857\\
1.47	0.341255928754291	0.752367096967051	0.703223584571265\\
1.47	0.353793780602429	0.763162841084325	0.697635496597417\\
1.47	0.366499761262991	0.773742494322337	0.692428303594618\\
1.47	0.379368276771857	0.784102202600442	0.687549132887069\\
1.47	0.392393618241769	0.794238304822092	0.682943679235329\\
1.47	0.405569970164763	0.804147337016097	0.678558662193046\\
1.47	0.418891418885081	0.813826036051075	0.674343787412872\\
1.47	0.432351961217133	0.823271342915545	0.670253187715552\\
1.47	0.445945513182735	0.832480405557787	0.666246357336158\\
1.47	0.459665918841652	0.841450581281375	0.662288623474185\\
1.47	0.473506959189401	0.850179438693979	0.658351221673191\\
1.47	0.487462361096317	0.85866475920876	0.654411055137886\\
1.47	0.501525806262096	0.866904538099407	0.650450223206467\\
1.47	0.515690940160279	0.874896985111485	0.646455401874959\\
1.47	0.529951380947623	0.882640524634437	0.642417151055155\\
1.47	0.544300728313782	0.890133795440129	0.638329210947585\\
1.47	0.558732572247415	0.897375649995373	0.634187835387498\\
1.47	0.57324050169555	0.904365153357279	0.629991194998376\\
1.47	0.587818113093873	0.911101581661729	0.625738868900924\\
1.47	0.602459018746568	0.917584420216504	0.62143143163314\\
1.47	0.617156855035314	0.923813361211857	0.617070132480554\\
1.47	0.631905290438127	0.92978830106243	0.612656657833838\\
1.47	0.6466980333399	0.935509337395467	0.608192963377087\\
1.47	0.661528839617671	0.940976765701173	0.603681161495105\\
1.47	0.676391519984904	0.946191075661951	0.599123449738607\\
1.47	0.691279947080357	0.951152947177946	0.59452206790159\\
1.47	0.706188062288412	0.955863246106974	0.589879273662389\\
1.47	0.721109882279076	0.960323019737426	0.585197329319936\\
1.47	0.736039505257225	0.964533492013186	0.580478494543978\\
1.47	0.75097111691199	0.968496058529893	0.575725022018313\\
1.47	0.765898996058536	0.972212281322139	0.570939154289558\\
1.47	0.780817519965826	0.975683883461272	0.566123121053175\\
1.47	0.795721169365275	0.97891274348356	0.561279136605558\\
1.47	0.810604533136476	0.981900889668376	0.556409397399199\\
1.47	0.825462312667457	0.984650494185936	0.551516079695736\\
1.47	0.840289325888138	0.987163867133889	0.546601337319639\\
1.47	0.855080510976839	0.989443450481768	0.541667299515139\\
1.47	0.86983092974082	0.991491811941914	0.536716068908598\\
1.47	0.884535770672926	0.993311638785082	0.531749719578141\\
1.47	0.899190351687411	0.994905731618374	0.526770295231796\\
1.47	0.913790122539012	0.996276998142666	0.521779807495104\\
1.47	0.928330666930242	0.997428446906011	0.516780234308684\\
1.47	0.942807704312699	0.998363181068902	0.511773518435872\\
1.47	0.95721709138901	0.999084392196567	0.506761566080168\\
1.47	0.971554823322703	0.999595354092743	0.501746245611874\\
1.47	0.985817034663989	0.999899416688637	0.496729386403004\\
1.47	1	1	0.491669087346906\\
1.485	0	0	0.710195569525497\\
1.485	0.000100583311362513	0.0141829653360114	0.71240460418299\\
1.485	0.000404645907256436	0.0284451766772965	0.714630690536905\\
1.485	0.000915607803433	0.0427829086109896	0.716916312064106\\
1.485	0.00163681893109844	0.057192295687301	0.719289019360234\\
1.485	0.00257155309398959	0.0716693330697585	0.721780367061016\\
1.485	0.00372300185733414	0.086209877460988	0.724425655492835\\
1.485	0.00509426838162598	0.100809648312589	0.727263413850304\\
1.485	0.00668836121491816	0.115464229327074	0.73033458997498\\
1.485	0.00850818805808555	0.13016907025918	0.733681422514709\\
1.485	0.0105565495182326	0.144919489023162	0.737345986077549\\
1.485	0.0128361328661109	0.159710674111862	0.741368418705642\\
1.485	0.0153495058140643	0.174537687332543	0.745784862958202\\
1.485	0.0180991103316243	0.189395466863524	0.750625176092311\\
1.485	0.0210872565164405	0.204278830634725	0.755910489868998\\
1.485	0.0243161165387281	0.219182480034174	0.761650724664453\\
1.485	0.0277877186778607	0.234101003941464	0.767842183867616\\
1.485	0.0315039414701067	0.24902888308801	0.77446537092041\\
1.485	0.0354665079868145	0.263960494742775	0.781483180780606\\
1.485	0.0396769802625738	0.278890117720924	0.788839618268073\\
1.485	0.0441367538930258	0.293811937711588	0.796459186325544\\
1.485	0.0488470528220538	0.308720052919643	0.804247066923454\\
1.485	0.0538089243380495	0.323608480015096	0.81209018616195\\
1.485	0.0590232342988274	0.338471160382329	0.819859213932694\\
1.485	0.064490662604533	0.3533019666601	0.827411499063639\\
1.485	0.0702116989375697	0.368094709561873	0.834594885813505\\
1.485	0.0761866387881432	0.382843144964686	0.841252300287179\\
1.485	0.0824155797834956	0.397540981253432	0.847226939729081\\
1.485	0.0888984183382709	0.412181886906127	0.852367847905041\\
1.485	0.0956348466427212	0.42675949830445	0.856535620027829\\
1.485	0.102624350004627	0.441267427752584	0.859607954619617\\
1.485	0.109866204559871	0.455699271686219	0.861484760266623\\
1.485	0.117359475365564	0.470048619052377	0.86209253425154\\
1.485	0.125103014888515	0.484309059839721	0.861387758068369\\
1.485	0.133095461900593	0.498474193737904	0.85935910089082\\
1.485	0.14133524079124	0.512537638903683	0.856028283765736\\
1.485	0.149820561306021	0.526493040810599	0.851449530870183\\
1.485	0.158549418718625	0.540334081158348	0.845707614739505\\
1.485	0.167519594442214	0.554054486817265	0.83891458434315\\
1.485	0.176728657084455	0.567648038782867	0.831205342370836\\
1.485	0.186173963948925	0.581108581114919	0.822732305417528\\
1.485	0.195852662983903	0.594430029835237	0.813659432944309\\
1.485	0.205761695177907	0.607606381758231	0.804155944104868\\
1.485	0.215897797399558	0.620631723228143	0.794390053406969\\
1.485	0.226257505677663	0.633500238737009	0.7845230460628\\
1.485	0.236837158915675	0.646206219397571	0.774703982861589\\
1.485	0.247632903032949	0.658744071245709	0.765065275211016\\
1.485	0.258640695523528	0.671108323347402	0.755719307772964\\
1.485	0.269856310421543	0.683293635685828	0.746756214005914\\
1.485	0.28127534366066	0.695294806804925	0.738242834604294\\
1.485	0.292893218813452	0.707106781186547	0.730222816007836\\
1.485	0.304705193195075	0.71872465633934	0.722717741087767\\
1.485	0.316706364314172	0.730143689578457	0.715729131143265\\
1.485	0.328891676652598	0.741359304476472	0.709241120561857\\
1.485	0.341255928754291	0.752367096967051	0.703223584571265\\
1.485	0.353793780602429	0.763162841084325	0.697635496597417\\
1.485	0.366499761262991	0.773742494322337	0.692428303594618\\
1.485	0.379368276771857	0.784102202600442	0.687549132887069\\
1.485	0.392393618241769	0.794238304822092	0.682943679235329\\
1.485	0.405569970164763	0.804147337016097	0.678558662193046\\
1.485	0.418891418885081	0.813826036051075	0.674343787412872\\
1.485	0.432351961217133	0.823271342915545	0.670253187715552\\
1.485	0.445945513182735	0.832480405557787	0.666246357336158\\
1.485	0.459665918841652	0.841450581281375	0.662288623474185\\
1.485	0.473506959189401	0.850179438693979	0.658351221673191\\
1.485	0.487462361096317	0.85866475920876	0.654411055137886\\
1.485	0.501525806262096	0.866904538099407	0.650450223206467\\
1.485	0.515690940160279	0.874896985111485	0.646455401874959\\
1.485	0.529951380947623	0.882640524634437	0.642417151055155\\
1.485	0.544300728313782	0.890133795440129	0.638329210947585\\
1.485	0.558732572247415	0.897375649995373	0.634187835387498\\
1.485	0.57324050169555	0.904365153357279	0.629991194998376\\
1.485	0.587818113093873	0.911101581661729	0.625738868900924\\
1.485	0.602459018746568	0.917584420216504	0.62143143163314\\
1.485	0.617156855035314	0.923813361211857	0.617070132480554\\
1.485	0.631905290438127	0.92978830106243	0.612656657833838\\
1.485	0.6466980333399	0.935509337395467	0.608192963377087\\
1.485	0.661528839617671	0.940976765701173	0.603681161495105\\
1.485	0.676391519984904	0.946191075661951	0.599123449738607\\
1.485	0.691279947080357	0.951152947177946	0.59452206790159\\
1.485	0.706188062288412	0.955863246106974	0.589879273662389\\
1.485	0.721109882279076	0.960323019737426	0.585197329319936\\
1.485	0.736039505257225	0.964533492013186	0.580478494543978\\
1.485	0.75097111691199	0.968496058529893	0.575725022018313\\
1.485	0.765898996058536	0.972212281322139	0.570939154289558\\
1.485	0.780817519965826	0.975683883461272	0.566123121053174\\
1.485	0.795721169365275	0.97891274348356	0.561279136605558\\
1.485	0.810604533136476	0.981900889668376	0.556409397399199\\
1.485	0.825462312667457	0.984650494185936	0.551516079695737\\
1.485	0.840289325888138	0.987163867133889	0.546601337319641\\
1.485	0.855080510976839	0.989443450481767	0.541667299515139\\
1.485	0.86983092974082	0.991491811941914	0.536716068908598\\
1.485	0.884535770672926	0.993311638785082	0.531749719578141\\
1.485	0.899190351687411	0.994905731618374	0.526770295231796\\
1.485	0.913790122539012	0.996276998142666	0.521779807495104\\
1.485	0.928330666930242	0.997428446906011	0.516780234308684\\
1.485	0.942807704312699	0.998363181068902	0.511773518435872\\
1.485	0.95721709138901	0.999084392196567	0.506761566080168\\
1.485	0.971554823322703	0.999595354092743	0.501746245611874\\
1.485	0.985817034663989	0.999899416688637	0.496729386403004\\
1.485	1	1	0.491669087346906\\
1.5	0	0	0.710195569525497\\
1.5	0.000100583311362513	0.0141829653360114	0.71240460418299\\
1.5	0.000404645907256436	0.0284451766772965	0.714630690536905\\
1.5	0.000915607803433	0.0427829086109896	0.716916312064106\\
1.5	0.00163681893109844	0.057192295687301	0.719289019360234\\
1.5	0.00257155309398959	0.0716693330697585	0.721780367061016\\
1.5	0.00372300185733414	0.086209877460988	0.724425655492835\\
1.5	0.00509426838162598	0.100809648312589	0.727263413850304\\
1.5	0.00668836121491816	0.115464229327074	0.73033458997498\\
1.5	0.00850818805808555	0.13016907025918	0.733681422514709\\
1.5	0.0105565495182326	0.144919489023162	0.737345986077549\\
1.5	0.0128361328661109	0.159710674111862	0.741368418705642\\
1.5	0.0153495058140643	0.174537687332543	0.745784862958202\\
1.5	0.0180991103316243	0.189395466863524	0.750625176092311\\
1.5	0.0210872565164405	0.204278830634725	0.755910489868998\\
1.5	0.0243161165387281	0.219182480034174	0.761650724664453\\
1.5	0.0277877186778607	0.234101003941464	0.767842183867616\\
1.5	0.0315039414701067	0.24902888308801	0.77446537092041\\
1.5	0.0354665079868145	0.263960494742775	0.781483180780606\\
1.5	0.0396769802625738	0.278890117720924	0.788839618268073\\
1.5	0.0441367538930258	0.293811937711588	0.796459186325544\\
1.5	0.0488470528220538	0.308720052919643	0.804247066923454\\
1.5	0.0538089243380495	0.323608480015096	0.81209018616195\\
1.5	0.0590232342988274	0.338471160382329	0.819859213932694\\
1.5	0.064490662604533	0.3533019666601	0.827411499063639\\
1.5	0.0702116989375697	0.368094709561873	0.834594885813504\\
1.5	0.0761866387881432	0.382843144964686	0.841252300287179\\
1.5	0.0824155797834956	0.397540981253432	0.847226939729081\\
1.5	0.0888984183382709	0.412181886906127	0.852367847905041\\
1.5	0.0956348466427212	0.42675949830445	0.856535620027829\\
1.5	0.102624350004627	0.441267427752584	0.859607954619617\\
1.5	0.109866204559871	0.455699271686219	0.86148476026662\\
1.5	0.117359475365564	0.470048619052377	0.86209253425154\\
1.5	0.125103014888515	0.484309059839721	0.861387758068368\\
1.5	0.133095461900593	0.498474193737904	0.859359100890821\\
1.5	0.14133524079124	0.512537638903683	0.856028283765736\\
1.5	0.149820561306021	0.526493040810599	0.851449530870183\\
1.5	0.158549418718625	0.540334081158348	0.845707614739505\\
1.5	0.167519594442214	0.554054486817265	0.83891458434315\\
1.5	0.176728657084455	0.567648038782867	0.831205342370836\\
1.5	0.186173963948925	0.581108581114919	0.822732305417528\\
1.5	0.195852662983903	0.594430029835237	0.813659432944309\\
1.5	0.205761695177907	0.607606381758231	0.804155944104868\\
1.5	0.215897797399558	0.620631723228143	0.794390053406969\\
1.5	0.226257505677663	0.633500238737009	0.7845230460628\\
1.5	0.236837158915675	0.646206219397571	0.774703982861589\\
1.5	0.247632903032949	0.658744071245709	0.765065275211016\\
1.5	0.258640695523528	0.671108323347402	0.755719307772964\\
1.5	0.269856310421543	0.683293635685828	0.746756214005914\\
1.5	0.28127534366066	0.695294806804925	0.738242834604294\\
1.5	0.292893218813452	0.707106781186547	0.730222816007836\\
1.5	0.304705193195075	0.71872465633934	0.722717741087767\\
1.5	0.316706364314172	0.730143689578457	0.715729131143265\\
1.5	0.328891676652598	0.741359304476472	0.709241120561857\\
1.5	0.341255928754291	0.752367096967051	0.703223584571265\\
1.5	0.353793780602429	0.763162841084325	0.697635496597417\\
1.5	0.366499761262991	0.773742494322337	0.692428303594618\\
1.5	0.379368276771857	0.784102202600442	0.687549132887069\\
1.5	0.392393618241769	0.794238304822092	0.682943679235329\\
1.5	0.405569970164763	0.804147337016097	0.678558662193046\\
1.5	0.418891418885081	0.813826036051075	0.674343787412872\\
1.5	0.432351961217133	0.823271342915545	0.670253187715552\\
1.5	0.445945513182735	0.832480405557787	0.666246357336158\\
1.5	0.459665918841652	0.841450581281375	0.662288623474185\\
1.5	0.473506959189401	0.850179438693979	0.658351221673191\\
1.5	0.487462361096317	0.85866475920876	0.654411055137886\\
1.5	0.501525806262096	0.866904538099407	0.650450223206467\\
1.5	0.515690940160279	0.874896985111485	0.646455401874959\\
1.5	0.529951380947623	0.882640524634437	0.642417151055155\\
1.5	0.544300728313782	0.890133795440129	0.638329210947585\\
1.5	0.558732572247415	0.897375649995373	0.634187835387498\\
1.5	0.57324050169555	0.904365153357279	0.629991194998376\\
1.5	0.587818113093873	0.911101581661729	0.625738868900924\\
1.5	0.602459018746568	0.917584420216504	0.62143143163314\\
1.5	0.617156855035314	0.923813361211857	0.617070132480554\\
1.5	0.631905290438127	0.92978830106243	0.612656657833838\\
1.5	0.6466980333399	0.935509337395467	0.608192963377087\\
1.5	0.661528839617671	0.940976765701173	0.603681161495105\\
1.5	0.676391519984904	0.946191075661951	0.599123449738607\\
1.5	0.691279947080357	0.951152947177946	0.59452206790159\\
1.5	0.706188062288412	0.955863246106974	0.589879273662389\\
1.5	0.721109882279076	0.960323019737426	0.585197329319936\\
1.5	0.736039505257225	0.964533492013186	0.580478494543978\\
1.5	0.75097111691199	0.968496058529893	0.575725022018313\\
1.5	0.765898996058536	0.972212281322139	0.570939154289558\\
1.5	0.780817519965826	0.975683883461272	0.566123121053175\\
1.5	0.795721169365275	0.97891274348356	0.561279136605555\\
1.5	0.810604533136476	0.981900889668376	0.556409397399198\\
1.5	0.825462312667457	0.984650494185936	0.551516079695735\\
1.5	0.840289325888138	0.987163867133889	0.546601337319641\\
1.5	0.855080510976839	0.989443450481768	0.541667299515139\\
1.5	0.86983092974082	0.991491811941914	0.536716068908598\\
1.5	0.884535770672926	0.993311638785082	0.531749719578141\\
1.5	0.899190351687411	0.994905731618374	0.526770295231796\\
1.5	0.913790122539012	0.996276998142666	0.521779807495104\\
1.5	0.928330666930242	0.997428446906011	0.516780234308684\\
1.5	0.942807704312699	0.998363181068902	0.511773518435872\\
1.5	0.95721709138901	0.999084392196567	0.506761566080168\\
1.5	0.971554823322703	0.999595354092743	0.501746245611874\\
1.5	0.985817034663989	0.999899416688637	0.496729386403004\\
1.5	1	1	0.491669087346906\\
};
\addplot3 [color=mycolor1,solid,line width=1.5pt]
 table[row sep=crcr] {%
0	0	0\\
1.5	0	0\\
};
 \addplot3 [color=mycolor1,solid,line width=1.5pt]
 table[row sep=crcr] {%
0	1	1\\
0.015	1	1\\
0.03	1	1\\
0.045	1	1\\
0.06	1	1\\
0.075	1	1\\
0.09	1	1\\
0.105	1	1\\
0.12	1	1\\
0.135	1	1\\
0.15	1	1\\
0.165	1	1\\
0.18	1	1\\
0.195	1	1\\
0.21	1	1\\
0.225	1	1\\
0.24	1	1\\
0.255	1	1\\
0.27	1	1\\
0.285	1	1\\
0.3	1	1\\
0.315	1	1\\
0.33	1	1\\
0.345	1	1\\
0.36	1	1\\
0.375	1	1\\
0.39	1	1\\
0.405	1	1\\
0.42	1	1\\
0.435	1	1\\
0.45	1	1\\
0.465	1	1\\
0.48	1	1\\
0.495	1	1\\
0.51	1	1\\
0.525	1	1\\
0.54	1	1\\
0.555	1	1\\
0.57	1	1\\
0.585	1	1\\
0.6	1	1\\
0.615	1	1\\
0.63	1	1\\
0.645	1	1\\
0.66	1	1\\
0.675	1	1\\
0.69	1	1\\
0.705	1	1\\
0.72	1	1\\
0.735	1	1\\
0.75	1	1\\
0.765	1	1\\
0.78	1	1\\
0.795	1	1\\
0.81	1	1\\
0.825	1	1\\
0.84	1	1\\
0.855	1	1\\
0.87	1	1\\
0.885	1	1\\
0.9	1	1\\
0.915	1	1\\
0.93	1	1\\
0.945	1	1\\
0.96	1	1\\
0.975	1	1\\
0.99	1	1\\
1.005	1	1\\
1.02	1	1\\
1.035	1	1\\
1.05	1	1\\
1.065	1	1\\
1.08	1	1\\
1.095	1	1\\
1.11	1	1\\
1.125	1	1\\
1.14	1	1\\
1.155	1	1\\
1.17	1	1\\
1.185	1	1\\
1.2	1	1\\
1.215	1	1\\
1.23	1	1\\
1.245	1	1\\
1.26	1	1\\
1.275	1	1\\
1.29	1	1\\
1.305	1	1\\
1.32	1	1\\
1.335	1	1\\
1.35	1	1\\
1.365	1	1\\
1.38	1	1\\
1.395	1	1\\
1.41	1	1\\
1.425	1	1\\
1.44	1	1\\
1.455	1	1\\
1.47	1	1\\
1.485	1	1\\
1.5	1	1\\
};
 \addplot3 [color=mycolor1,solid,line width=1.5pt]
 table[row sep=crcr] {%
0	0	0\\
0	0.000100583311362513	0.0141829653360114\\
0	0.000404645907256436	0.0284451766772965\\
0	0.000915607803433	0.0427829086109896\\
0	0.00163681893109844	0.057192295687301\\
0	0.00257155309398959	0.0716693330697585\\
0	0.00372300185733414	0.086209877460988\\
0	0.00509426838162598	0.100809648312589\\
0	0.00668836121491816	0.115464229327074\\
0	0.00850818805808555	0.13016907025918\\
0	0.0105565495182326	0.144919489023162\\
0	0.0128361328661109	0.159710674111862\\
0	0.0153495058140643	0.174537687332543\\
0	0.0180991103316243	0.189395466863524\\
0	0.0210872565164405	0.204278830634725\\
0	0.0243161165387281	0.219182480034174\\
0	0.0277877186778607	0.234101003941464\\
0	0.0315039414701067	0.24902888308801\\
0	0.0354665079868145	0.263960494742775\\
0	0.0396769802625738	0.278890117720924\\
0	0.0441367538930258	0.293811937711588\\
0	0.0488470528220538	0.308720052919643\\
0	0.0538089243380495	0.323608480015096\\
0	0.0590232342988274	0.338471160382329\\
0	0.064490662604533	0.3533019666601\\
0	0.0702116989375697	0.368094709561873\\
0	0.0761866387881432	0.382843144964686\\
0	0.0824155797834956	0.397540981253432\\
0	0.0888984183382709	0.412181886906127\\
0	0.0956348466427212	0.42675949830445\\
0	0.102624350004627	0.441267427752584\\
0	0.109866204559871	0.455699271686219\\
0	0.117359475365564	0.470048619052377\\
0	0.125103014888515	0.484309059839721\\
0	0.133095461900593	0.498474193737904\\
0	0.14133524079124	0.512537638903683\\
0	0.149820561306021	0.526493040810599\\
0	0.158549418718625	0.540334081158348\\
0	0.167519594442214	0.554054486817265\\
0	0.176728657084455	0.567648038782867\\
0	0.186173963948925	0.581108581114919\\
0	0.195852662983903	0.594430029835237\\
0	0.205761695177907	0.607606381758231\\
0	0.215897797399558	0.620631723228143\\
0	0.226257505677663	0.633500238737009\\
0	0.236837158915675	0.646206219397571\\
0	0.247632903032949	0.658744071245709\\
0	0.258640695523528	0.671108323347402\\
0	0.269856310421543	0.683293635685828\\
0	0.28127534366066	0.695294806804925\\
0	0.292893218813452	0.707106781186547\\
0	0.304705193195075	0.71872465633934\\
0	0.316706364314172	0.730143689578457\\
0	0.328891676652598	0.741359304476472\\
0	0.341255928754291	0.752367096967051\\
0	0.353793780602429	0.763162841084325\\
0	0.366499761262991	0.773742494322337\\
0	0.379368276771857	0.784102202600442\\
0	0.392393618241769	0.794238304822092\\
0	0.405569970164763	0.804147337016097\\
0	0.418891418885081	0.813826036051075\\
0	0.432351961217133	0.823271342915545\\
0	0.445945513182735	0.832480405557787\\
0	0.459665918841652	0.841450581281375\\
0	0.473506959189401	0.850179438693979\\
0	0.487462361096317	0.85866475920876\\
0	0.501525806262096	0.866904538099407\\
0	0.515690940160279	0.874896985111485\\
0	0.529951380947623	0.882640524634437\\
0	0.544300728313782	0.890133795440129\\
0	0.558732572247415	0.897375649995373\\
0	0.57324050169555	0.904365153357279\\
0	0.587818113093873	0.911101581661729\\
0	0.602459018746568	0.917584420216504\\
0	0.617156855035314	0.923813361211857\\
0	0.631905290438127	0.92978830106243\\
0	0.6466980333399	0.935509337395467\\
0	0.661528839617671	0.940976765701173\\
0	0.676391519984904	0.946191075661951\\
0	0.691279947080357	0.951152947177946\\
0	0.706188062288412	0.955863246106974\\
0	0.721109882279076	0.960323019737426\\
0	0.736039505257225	0.964533492013186\\
0	0.75097111691199	0.968496058529893\\
0	0.765898996058536	0.972212281322139\\
0	0.780817519965826	0.975683883461272\\
0	0.795721169365275	0.97891274348356\\
0	0.810604533136476	0.981900889668376\\
0	0.825462312667457	0.984650494185936\\
0	0.840289325888138	0.987163867133889\\
0	0.855080510976839	0.989443450481768\\
0	0.86983092974082	0.991491811941914\\
0	0.884535770672926	0.993311638785082\\
0	0.899190351687411	0.994905731618374\\
0	0.913790122539012	0.996276998142666\\
0	0.928330666930242	0.997428446906011\\
0	0.942807704312699	0.998363181068902\\
0	0.95721709138901	0.999084392196567\\
0	0.971554823322703	0.999595354092743\\
0	0.985817034663989	0.999899416688637\\
0	1	1\\
};
 \addplot3 [color=mycolor1,solid,line width=1.5pt]
 table[row sep=crcr] {%
1.5	0	0\\
1.5	0.000100583311362513	0.0141829653360114\\
1.5	0.000404645907256436	0.0284451766772965\\
1.5	0.000915607803433	0.0427829086109896\\
1.5	0.00163681893109844	0.057192295687301\\
1.5	0.00257155309398959	0.0716693330697585\\
1.5	0.00372300185733414	0.086209877460988\\
1.5	0.00509426838162598	0.100809648312589\\
1.5	0.00668836121491816	0.115464229327074\\
1.5	0.00850818805808555	0.13016907025918\\
1.5	0.0105565495182326	0.144919489023162\\
1.5	0.0128361328661109	0.159710674111862\\
1.5	0.0153495058140643	0.174537687332543\\
1.5	0.0180991103316243	0.189395466863524\\
1.5	0.0210872565164405	0.204278830634725\\
1.5	0.0243161165387281	0.219182480034174\\
1.5	0.0277877186778607	0.234101003941464\\
1.5	0.0315039414701067	0.24902888308801\\
1.5	0.0354665079868145	0.263960494742775\\
1.5	0.0396769802625738	0.278890117720924\\
1.5	0.0441367538930258	0.293811937711588\\
1.5	0.0488470528220538	0.308720052919643\\
1.5	0.0538089243380495	0.323608480015096\\
1.5	0.0590232342988274	0.338471160382329\\
1.5	0.064490662604533	0.3533019666601\\
1.5	0.0702116989375697	0.368094709561873\\
1.5	0.0761866387881432	0.382843144964686\\
1.5	0.0824155797834956	0.397540981253432\\
1.5	0.0888984183382709	0.412181886906127\\
1.5	0.0956348466427212	0.42675949830445\\
1.5	0.102624350004627	0.441267427752584\\
1.5	0.109866204559871	0.455699271686219\\
1.5	0.117359475365564	0.470048619052377\\
1.5	0.125103014888515	0.484309059839721\\
1.5	0.133095461900593	0.498474193737904\\
1.5	0.14133524079124	0.512537638903683\\
1.5	0.149820561306021	0.526493040810599\\
1.5	0.158549418718625	0.540334081158348\\
1.5	0.167519594442214	0.554054486817265\\
1.5	0.176728657084455	0.567648038782867\\
1.5	0.186173963948925	0.581108581114919\\
1.5	0.195852662983903	0.594430029835237\\
1.5	0.205761695177907	0.607606381758231\\
1.5	0.215897797399558	0.620631723228143\\
1.5	0.226257505677663	0.633500238737009\\
1.5	0.236837158915675	0.646206219397571\\
1.5	0.247632903032949	0.658744071245709\\
1.5	0.258640695523528	0.671108323347402\\
1.5	0.269856310421543	0.683293635685828\\
1.5	0.28127534366066	0.695294806804925\\
1.5	0.292893218813452	0.707106781186547\\
1.5	0.304705193195075	0.71872465633934\\
1.5	0.316706364314172	0.730143689578457\\
1.5	0.328891676652598	0.741359304476472\\
1.5	0.341255928754291	0.752367096967051\\
1.5	0.353793780602429	0.763162841084325\\
1.5	0.366499761262991	0.773742494322337\\
1.5	0.379368276771857	0.784102202600442\\
1.5	0.392393618241769	0.794238304822092\\
1.5	0.405569970164763	0.804147337016097\\
1.5	0.418891418885081	0.813826036051075\\
1.5	0.432351961217133	0.823271342915545\\
1.5	0.445945513182735	0.832480405557787\\
1.5	0.459665918841652	0.841450581281375\\
1.5	0.473506959189401	0.850179438693979\\
1.5	0.487462361096317	0.85866475920876\\
1.5	0.501525806262096	0.866904538099407\\
1.5	0.515690940160279	0.874896985111485\\
1.5	0.529951380947623	0.882640524634437\\
1.5	0.544300728313782	0.890133795440129\\
1.5	0.558732572247415	0.897375649995373\\
1.5	0.57324050169555	0.904365153357279\\
1.5	0.587818113093873	0.911101581661729\\
1.5	0.602459018746568	0.917584420216504\\
1.5	0.617156855035314	0.923813361211857\\
1.5	0.631905290438127	0.92978830106243\\
1.5	0.6466980333399	0.935509337395467\\
1.5	0.661528839617671	0.940976765701173\\
1.5	0.676391519984904	0.946191075661951\\
1.5	0.691279947080357	0.951152947177946\\
1.5	0.706188062288412	0.955863246106974\\
1.5	0.721109882279076	0.960323019737426\\
1.5	0.736039505257225	0.964533492013186\\
1.5	0.75097111691199	0.968496058529893\\
1.5	0.765898996058536	0.972212281322139\\
1.5	0.780817519965826	0.975683883461272\\
1.5	0.795721169365275	0.97891274348356\\
1.5	0.810604533136476	0.981900889668376\\
1.5	0.825462312667457	0.984650494185936\\
1.5	0.840289325888138	0.987163867133889\\
1.5	0.855080510976839	0.989443450481768\\
1.5	0.86983092974082	0.991491811941914\\
1.5	0.884535770672926	0.993311638785082\\
1.5	0.899190351687411	0.994905731618374\\
1.5	0.913790122539012	0.996276998142666\\
1.5	0.928330666930242	0.997428446906011\\
1.5	0.942807704312699	0.998363181068902\\
1.5	0.95721709138901	0.999084392196567\\
1.5	0.971554823322703	0.999595354092743\\
1.5	0.985817034663989	0.999899416688637\\
1.5	1	1\\
};
\addplot3 [draw=none, mark size=3.3pt, scatter,mark=ball,scatter/use mapped color={ball color=red},scatter src=rand,only marks,z buffer=sort]
 table[row sep=crcr] {%
0	0	0\\
0	0	1\\
0	1	1\\
1.5	0	0\\
1.5	0	1\\
1.5	1	1\\
};

 \addplot3 [color=black,dashed]
 table[row sep=crcr] {%
0	0	0\\
1.5	0	0\\
};
 \addplot3 [color=black,dashed]
 table[row sep=crcr] {%
0	0	1\\
1.5	0	1\\
};
 \addplot3 [color=black,dashed]
 table[row sep=crcr] {%
0	1	1\\
1.5	1	1\\
};
 \addplot3 [color=black,dashed]
 table[row sep=crcr] {%
0	0	0\\
0	0	1\\
0	1	1\\
};
 \addplot3 [color=black,dashed]
 table[row sep=crcr] {%
1.5	0	0\\
1.5	0	1\\
1.5	1	1\\
};
\addplot3 [color=red, dashed, -stealth,line width=1.5pt, postaction={decorate,decoration={text along path,
              text={$\xi$} {--} direction, raise=1ex, text align={center}, text color={red},
          }}]
 table[row sep=crcr] {%
0	0	0\\
0	0.000100583311362513	0.0141829653360114\\
0	0.000404645907256436	0.0284451766772965\\
0	0.000915607803433	0.0427829086109896\\
0	0.00163681893109844	0.057192295687301\\
0	0.00257155309398959	0.0716693330697585\\
0	0.00372300185733414	0.086209877460988\\
0	0.00509426838162598	0.100809648312589\\
0	0.00668836121491816	0.115464229327074\\
0	0.00850818805808555	0.13016907025918\\
0	0.0105565495182326	0.144919489023162\\
0	0.0128361328661109	0.159710674111862\\
0	0.0153495058140643	0.174537687332543\\
0	0.0180991103316243	0.189395466863524\\
0	0.0210872565164405	0.204278830634725\\
0	0.0243161165387281	0.219182480034174\\
0	0.0277877186778607	0.234101003941464\\
0	0.0315039414701067	0.24902888308801\\
0	0.0354665079868145	0.263960494742775\\
0	0.0396769802625738	0.278890117720924\\
0	0.0441367538930258	0.293811937711588\\
0	0.0488470528220538	0.308720052919643\\
0	0.0538089243380495	0.323608480015096\\
0	0.0590232342988274	0.338471160382329\\
0	0.064490662604533	0.3533019666601\\
0	0.0702116989375697	0.368094709561873\\
0	0.0761866387881432	0.382843144964686\\
0	0.0824155797834956	0.397540981253432\\
0	0.0888984183382709	0.412181886906127\\
0	0.0956348466427212	0.42675949830445\\
0	0.102624350004627	0.441267427752584\\
0	0.109866204559871	0.455699271686219\\
0	0.117359475365564	0.470048619052377\\
0	0.125103014888515	0.484309059839721\\
0	0.133095461900593	0.498474193737904\\
0	0.14133524079124	0.512537638903683\\
0	0.149820561306021	0.526493040810599\\
0	0.158549418718625	0.540334081158348\\
0	0.167519594442214	0.554054486817265\\
0	0.176728657084455	0.567648038782867\\
0	0.186173963948925	0.581108581114919\\
0	0.195852662983903	0.594430029835237\\
0	0.205761695177907	0.607606381758231\\
0	0.215897797399558	0.620631723228143\\
0	0.226257505677663	0.633500238737009\\
0	0.236837158915675	0.646206219397571\\
0	0.247632903032949	0.658744071245709\\
0	0.258640695523528	0.671108323347402\\
0	0.269856310421543	0.683293635685828\\
0	0.28127534366066	0.695294806804925\\
0	0.292893218813452	0.707106781186547\\
0	0.304705193195075	0.71872465633934\\
%0	0.316706364314172	0.730143689578457\\
%0	0.328891676652598	0.741359304476472\\
%0	0.341255928754291	0.752367096967051\\
%0	0.353793780602429	0.763162841084325\\
%0	0.366499761262991	0.773742494322337\\
%0	0.379368276771857	0.784102202600442\\
%0	0.392393618241769	0.794238304822092\\
%0	0.405569970164763	0.804147337016097\\
%0	0.418891418885081	0.813826036051075\\
%0	0.432351961217133	0.823271342915545\\
%0	0.445945513182735	0.832480405557787\\
%0	0.459665918841652	0.841450581281375\\
%0	0.473506959189401	0.850179438693979\\
%0	0.487462361096317	0.85866475920876\\
%0	0.501525806262096	0.866904538099407\\
%0	0.515690940160279	0.874896985111485\\
%0	0.529951380947623	0.882640524634437\\
%0	0.544300728313782	0.890133795440129\\
%0	0.558732572247415	0.897375649995373\\
%0	0.57324050169555	0.904365153357279\\
%0	0.587818113093873	0.911101581661729\\
%0	0.602459018746568	0.917584420216504\\
%0	0.617156855035314	0.923813361211857\\
%0	0.631905290438127	0.92978830106243\\
%0	0.6466980333399	0.935509337395467\\
%0	0.661528839617671	0.940976765701173\\
%0	0.676391519984904	0.946191075661951\\
%0	0.691279947080357	0.951152947177946\\
%0	0.706188062288412	0.955863246106974\\
%0	0.721109882279076	0.960323019737426\\
%0	0.736039505257225	0.964533492013186\\
%0	0.75097111691199	0.968496058529893\\
%0	0.765898996058536	0.972212281322139\\
%0	0.780817519965826	0.975683883461272\\
%0	0.795721169365275	0.97891274348356\\
%0	0.810604533136476	0.981900889668376\\
%0	0.825462312667457	0.984650494185936\\
%0	0.840289325888138	0.987163867133889\\
%0	0.855080510976839	0.989443450481768\\
%0	0.86983092974082	0.991491811941914\\
%0	0.884535770672926	0.993311638785082\\
%0	0.899190351687411	0.994905731618374\\
%0	0.913790122539012	0.996276998142666\\
%0	0.928330666930242	0.997428446906011\\
%0	0.942807704312699	0.998363181068902\\
%0	0.95721709138901	0.999084392196567\\
%0	0.971554823322703	0.999595354092743\\
%0	0.985817034663989	0.999899416688637\\
%0	1	1\\
};
%----------------
%\shade [ball color = red] (axis cs:0, 0, 0) circle (3.3pt) node [right] {$\mathbf{P}_{1}(0, 0, 0)$};
%\shade [ball color = red] (axis cs:0, 0, 1) circle (3.3pt) node [right] {$\mathbf{P}_{2}(0, 0, 1)$};
%\shade [ball color = red] (axis cs:0, 1, 1) circle (3.3pt) node [right] {$\mathbf{P}_{3}(0, 1, 1)$};
%\shade [ball color = red] (axis cs:1.5, 0, 0) circle (3.3pt) node [right] {$\mathbf{P}_{4}(1.5, 0, 0)$};
%\shade [ball color = red] (axis cs:1.5, 0, 1) circle (3.3pt) node [right] {$\mathbf{P}_{5}(1.5, 0, 1)$};
%\shade [ball color = red] (axis cs:1.5, 1, 1) circle (3.3pt) node [right] {$\mathbf{P}_{6}(1.5, 1, 1)$};
\coordinate (A) at (axis cs: 0, 0, 0);

\coordinate (X) at (axis cs: 0.75, 0, 0);

\coordinate (eta) at (axis cs: 0.3750, 0, 0);

\coordinate (C2End) at (axis cs: 0.75, 1, 1);

\coordinate (C1End) at (axis cs: 0.75, 0, 0);

\coordinate (C3End) at (axis cs: 0, 0.501525806262096, 0.866904538099407);

\coordinate (C4End) at (axis cs: 1.5, 0.501525806262096, 0.866904538099407);
\end{axis}
\draw[-stealth, dashed, line width=1.5pt, color=red] (A) -- node[above, near end, sloped] {$\eta$ -- direction} (X);
%\node[above] at (eta) {$\eta$--direction};
%\draw[help lines,xstep=1,ystep=1] (0,0) grid (10, 10);
%\foreach \x in {0,1,...,10} { \node [anchor=north] at (\x, 0) {\x}; }
%\foreach \y in {0,1,...,10} { \node [anchor=east] at (0, \y) {\y}; }

\draw[-latex,thick](4, 8)node[right]{Boundary Reference $ = 2$}
        to[out=180,in=90] (C2End);
        
\draw[-latex,thick](7, 1)node[right]{Boundary Reference $ = 1$}
        to[out=180,in=90] (C1End);
        
\draw[-latex,thick](1.5, 6.5)node[left]{Boundary Reference $ = 3$}
        to[out=0,in=45] (C3End);
        
\draw[-latex,thick](8, 6.0)node[right]{Boundary Reference $ = 4$}
        to[out=180,in=90] (C4End);
        
\node[above] at (A) {$\mathbf{P}_{1}(0, 0, 0)$};
%\node[above] at (2.6, 0, 0) {$\mathbf{P}_{1}(0, 0, 0)$};
\node[right] at (2.6, 3.7, 0) {$\mathbf{P}_{2}(0, 0, 1)$};
\node[right] at (0.0, 5.3, 0) {$\mathbf{P}_{3}(0, 1, 1)$};
\node[right] at (7.6, 2.0, 0) {$\mathbf{P}_{4}(1.5, 0, 0)$};
\node[right] at (7.6, 5.5, 0) {$\mathbf{P}_{5}(1.5, 0, 1)$};
\node[above] at (5.0, 7.2, 0) {$\mathbf{P}_{6}(1.5, 1, 1)$};
\end{tikzpicture}% \\
                \begin{subfigure}[b]{0.3\textwidth}
                        \tikzsetnextfilename{Ch3FrontalSlicesCylinder}
                        \normalsize
                        \begin{tikzpicture}
    [x={(-0.5cm,-0.5cm)}, y={(1cm,0cm)}, z={(0cm,1cm)},
    scale=1, thin, double, every node/.append style={transform shape}, on grid]
    \newcommand\drawface{\draw[fill=gray!10] (0, 0) rectangle (6, 2.5)}
    % face #1
    {\fontencoding{T1}\selectfont \ttfamily
    \begin{scope}[canvas is yz plane at x=-1]
        \drawface;
%        \foreach \x in {0, 2, ..., 6}
%        {
%            \draw (\x, 0) -- (\x, 4);
%        }
%        \foreach \y in {0, 1, ..., 4}
%        {
%            \draw (0, \y) -- (6, \y);
%        }
        \node at (1, 2) {1.5000};
        \node at (1, 1.5) {0.0000};
        \node at (1, 1) {0.0000};
        \node at (1, 0.5) {1.0000};

        \node at (3, 2) {1.0607};
        \node at (3, 1.5) {0.0000};
        \node at (3, 1) {0.7071};
        \node at (3, 0.5) {0.7071};

        \node at (5, 2) {1.5000};
        \node at (5, 1.5) {1.0000};
        \node at (5, 1) {1.0000};
        \node at (5, 0.5) {1.0000};
        
        \node [rectangle, anchor=west, fill=red!50, fill opacity=0.70, align=center, above] at (1, 2.5) {$\mathbf{P}_4$};
        \node [rectangle, anchor=west, fill=red!50, fill opacity=0.70, align=center, above] at (3, 2.5) {$\mathbf{P}_5$};
        \node [rectangle, anchor=west, fill=red!50, fill opacity=0.70, align=center, above] at (5, 2.5) {$\mathbf{P}_6$};
    \end{scope}

    \begin{scope}[canvas is yz plane at x=1]
    \drawface;
%    \foreach \x in {0, 2, ..., 6}
%    {
%        \draw (\x, 0) -- (\x, 4);
%    }
%    \foreach \y in {0, 1, ..., 4}
%    {
%        \draw (0, \y) -- (6, \y);
%    }
    \node at (1, 2) {0.0000};
    \node at (1, 1.5) {0.0000};
    \node at (1, 1) {0.0000};
    \node at (1, 0.5) {1.0000};

    \node at (3, 2) {0.0000};
    \node at (3, 1.5) {0.0000};
    \node at (3, 1) {0.7071};
    \node at (3, 0.5) {0.7071};

    \node at (5, 2) {0.0000};
    \node at (5, 1.5) {1.0000};
    \node at (5, 1) {1.0000};
    \node at (5, 0.5) {1.0000};

    \node [rectangle, anchor=west, fill=red!50, fill opacity=0.70, align=center, text width=1cm, left] at (0, 2.0) {x w};
    \node [rectangle, anchor=west, fill=red!50, fill opacity=0.70, align=center, text width=1cm, left] at (0, 1.5) {y w};
    \node [rectangle, anchor=west, fill=red!50, fill opacity=0.70, align=center, text width=1cm, left] at (0, 1.0) {z w};
    \node [rectangle, anchor=west, fill=red!50, fill opacity=0.70, align=center, text width=1cm, left] at (0, 0.5) {w};
    
    \node [rectangle, anchor=west, fill=red!50, fill opacity=0.70, align=center, below] at (1, 0) {$\mathbf{P}_1$};
    \node [rectangle, anchor=west, fill=red!50, fill opacity=0.70, align=center, below] at (3, 0) {$\mathbf{P}_2$};
    \node [rectangle, anchor=west, fill=red!50, fill opacity=0.70, align=center, below] at (5, 0) {$\mathbf{P}_3$};
    \end{scope}
    }
\end{tikzpicture} 
                        \caption{Each frontal slice store data for forming the corresponding control polyline in $\xi$ -- direction.}
                        \label{Ch3FrontalSlicesCylinder}
                    \end{subfigure}
                    \qquad \qquad \qquad \qquad
                    \begin{subfigure}[b]{0.3\textwidth}
                        \tikzsetnextfilename{Ch3LateralSlicesCylinder}
                        \begin{tikzpicture}
    [x={(-0.5cm,-0.5cm)}, y={(1cm,0cm)}, z={(0cm,1cm)},
    scale=1, thin, double, every node/.append style={transform shape}, on grid]
    \newcommand\drawface{\draw[fill=gray!10] (0,0) rectangle (2.5, 4)}
    {\fontencoding{T1}\selectfont \ttfamily
    \begin{scope}[canvas is zx plane at y=-1]
       \drawface;
        \node[rotate=-90] at (2, 3) {0.0000};
        \node[rotate=-90] at (1.5, 3) {0.0000};
        \node[rotate=-90] at (1, 3) {0.0000};
        \node[rotate=-90] at (0.5, 3) {1.0000};

        \node[rotate=-90] at (2, 1) {1.5000};
        \node[rotate=-90] at (1.5, 1) {0.0000};
        \node[rotate=-90] at (1, 1) {0.0000};
        \node[rotate=-90] at (0.5, 1) {1.0000};
        \node [rectangle, anchor=west, fill=red!50, fill opacity=0.70, align=center, text width=1cm, left, rotate=-90] at (2.0, 4.0) {x w};
        \node [rectangle, anchor=west, fill=red!50, fill opacity=0.70, align=center, text width=1cm, left, rotate=-90] at (1.5, 4.0) {y w};
        \node [rectangle, anchor=west, fill=red!50, fill opacity=0.70, align=center, text width=1cm, left, rotate=-90] at (1.0, 4.0) {z w};
        \node [rectangle, anchor=west, fill=red!50, fill opacity=0.70, align=center, text width=1cm, left, rotate=-90] at (0.5, 4.0) {w};
        
        \node [rectangle, anchor=west, fill=red!50, fill opacity=0.70, align=center, below, rotate=-90] at (0.0, 3.0) {$\mathbf{P}_1$};
        \node [rectangle, anchor=west, fill=red!50, fill opacity=0.70, align=center, above, rotate=-90] at (2.5, 1.0) {$\mathbf{P}_4$};
    \end{scope}
    \begin{scope}[canvas is zx plane at y=0]
       \drawface;
       \node[rotate=-90] at (2, 1) {1.0607};
       \node[rotate=-90] at (1.5, 1) {0.0000};
       \node[rotate=-90] at (1, 1) {0.7071};
       \node[rotate=-90] at (0.5, 1) {0.7071};

       \node[rotate=-90] at (2, 3) {0.0000};
       \node[rotate=-90] at (1.5, 3) {0.0000};
       \node[rotate=-90] at (1, 3) {0.7071};
       \node[rotate=-90] at (0.5, 3) {0.7071};
       
       \node [rectangle, anchor=west, fill=red!50, fill opacity=0.70, align=center, below, rotate=-90] at (0.0, 3.0) {$\mathbf{P}_2$};
       \node [rectangle, anchor=west, fill=red!50, fill opacity=0.70, align=center, above, rotate=-90] at (2.5, 1.0) {$\mathbf{P}_5$};
    \end{scope}
    \begin{scope}[canvas is zx plane at y=1]
       \drawface;
       \node[rotate=-90] at (2, 1) {1.5000};
       \node[rotate=-90] at (1.5, 1) {1.0000};
       \node[rotate=-90] at (1, 1) {1.0000};
       \node[rotate=-90] at (0.5, 1) {1.0000};

       \node[rotate=-90] at (2, 3) {0.0000};
       \node[rotate=-90] at (1.5, 3) {1.0000};
       \node[rotate=-90] at (1, 3) {1.0000};
       \node[rotate=-90] at (0.5, 3) {1.0000};
       
       \node [rectangle, anchor=west, fill=red!50, fill opacity=0.70, align=center, below, rotate=-90] at (0.0, 3.0) {$\mathbf{P}_3$};
       \node [rectangle, anchor=west, fill=red!50, fill opacity=0.70, align=center, above, rotate=-90] at (2.5, 1.0) {$\mathbf{P}_6$};
    \end{scope}
    }
\end{tikzpicture} 
                        \caption{Each lateral slice store data for forming the corresponding control polyline in $\eta$ -- direction.}
                        \label{fig:Ch3LateralSlicesCylinder}
                    \end{subfigure}
                \caption{Demonstration of data storage for contructing a NURBS surface.}
                \label{fig:Ch3SurfAQuarterOfACylinder}
            \end{figure}
            Similar to curve, the $\xi$ and $\eta$ represent the parameter value from one end of a surface to the other. These identifications play a same role as aforementioned functionality for curve. In addition, these information are very important when we model a multiple patches geometry since it requires two patches must have the same parameter direction at the interface of them (in the case that adjacent patches do not have the same parameter direction, a reverse of direction is needed).

            In the Fig.~\ref{fig:Ch3SurfAQuarterOfACylinder}, ``Boundary Reference'' is a label which is used to indicate the boundary of the geomtry in SIMO Package. Each label keep an integer value, these values are enumerated based on parameter values. Beginning from $\xi$ and then $\eta$ direction, the enumeration specified as $\xi = 0$ and $\xi = 1$ corresponds to Boundary Reference = 1 and 2, $\eta = 0$ and $\eta = 1$ corresponds to Boundary Reference = 3 and 4.
            \item For solid: control point's coordinates are stored in a four-dimensional array where each frontal slice of this array return a three-dimensional array and each frontal slice of this three-dimensional array return a two-dimensional array. So this is an extension of data structure for surface, therefore all conventions are the same as in surface case. An example code is showed below to illustrate this argument.  To avoid confusion, we should fill control point's coordinate to the 4-D array in the order from the first frontal slice {\fontencoding{T1}\selectfont \ttfamily CtrlPts(:, :, :, 1)} to the second frontal slice {\fontencoding{T1}\selectfont \ttfamily CtrlPts(:, :, :, 2)} and so on.
                \begin{lstlisting}
                    r = 2; % inner radius of hollow cylinder
                    t = 0.5; % thickness of hollow cylinder
                    h = 6; % height of hollow cylinder

                    % control points
                    CtrlPts = zeros(4, 9, 2, 2);

                    CtrlPts(1 : 3, 1, 1, 1) = [ r;  0; 0];
                    CtrlPts(1 : 3, 2, 1, 1) = [ r;  r; 0];
                    CtrlPts(1 : 3, 3, 1, 1) = [ 0;  r; 0];
                    CtrlPts(1 : 3, 4, 1, 1) = [-r;  r; 0];
                    CtrlPts(1 : 3, 5, 1, 1) = [-r;  0; 0];
                    CtrlPts(1 : 3, 6, 1, 1) = [-r; -r; 0];
                    CtrlPts(1 : 3, 7, 1, 1) = [ 0; -r; 0];
                    CtrlPts(1 : 3, 8, 1, 1) = [ r; -r; 0];
                    CtrlPts(1 : 3, 9, 1, 1) = [ r;  0; 0];

                    CtrlPts(1 : 3, 1, 2, 1) = [ r + t;      0; 0];
                    CtrlPts(1 : 3, 2, 2, 1) = [ r + t;  r + t; 0];
                    CtrlPts(1 : 3, 3, 2, 1) = [     0;  r + t; 0];
                    CtrlPts(1 : 3, 4, 2, 1) = [-(r + t);    r + t; 0];
                    CtrlPts(1 : 3, 5, 2, 1) = [-(r + t);        0; 0];
                    CtrlPts(1 : 3, 6, 2, 1) = [-(r + t); -(r + t); 0];
                    CtrlPts(1 : 3, 7, 2, 1) = [     0; -(r + t); 0];
                    CtrlPts(1 : 3, 8, 2, 1) = [ r + t; -(r + t); 0];
                    CtrlPts(1 : 3, 9, 2, 1) = [ r + t;        0; 0];

                    CtrlPts(1 : 3, 1, 1, 2) = [ r;  0; h];
                    CtrlPts(1 : 3, 2, 1, 2) = [ r;  r; h];
                    CtrlPts(1 : 3, 3, 1, 2) = [ 0;  r; h];
                    CtrlPts(1 : 3, 4, 1, 2) = [-r;  r; h];
                    CtrlPts(1 : 3, 5, 1, 2) = [-r;  0; h];
                    CtrlPts(1 : 3, 6, 1, 2) = [-r; -r; h];
                    CtrlPts(1 : 3, 7, 1, 2) = [ 0; -r; h];
                    CtrlPts(1 : 3, 8, 1, 2) = [ r; -r; h];
                    CtrlPts(1 : 3, 9, 1, 2) = [ r;  0; h];

                    CtrlPts(1 : 3, 1, 2, 2) = [ r + t;      0; h];
                    CtrlPts(1 : 3, 2, 2, 2) = [ r + t;  r + t; h];
                    CtrlPts(1 : 3, 3, 2, 2) = [     0;  r + t; h];
                    CtrlPts(1 : 3, 4, 2, 2) = [-(r + t);    r + t; h];
                    CtrlPts(1 : 3, 5, 2, 2) = [-(r + t);        0; h];
                    CtrlPts(1 : 3, 6, 2, 2) = [-(r + t); -(r + t); h];
                    CtrlPts(1 : 3, 7, 2, 2) = [     0; -(r + t); h];
                    CtrlPts(1 : 3, 8, 2, 2) = [ r + t; -(r + t); h];
                    CtrlPts(1 : 3, 9, 2, 2) = [ r + t;        0; h];
                    % weights
                    CtrlPts(4, :, :, :) = 1;
                    fac = 1 / sqrt(2);
                    CtrlPts(:, 2 : 2 : 8, :, :) = CtrlPts(:, 2 : 2 : 8, :, :) * fac;
                \end{lstlisting}
        \end{itemize}
\end{itemize}
These two parameterization data (knot vector(s) and control points) are then collected and putted in a structure data type of MATLAB for later using in computation. To do this in SIMO Package, we need to call the funtion named ``CreateNURBS'' ({\fontencoding{T1}\selectfont \ttfamily Object = CreateNURBS(KntVect, CtrlPts);}), the return value is a structure array contains the following fields:
\begin{itemize}
    \item KntVect: knot vector(s) of the B-spline (NURBS) patch.
    \item uqKntVect: unique knot values of the B-spline (NURBS) patch.
    \item CtrlPts4D: control point's coordinates in homogeneous space.
    \item CtrlPts3D: control point's coordinates projected into Cartesian space.
    \item Weights: weights of control points stored in 1D array.
    \item Dim: dimension of the B-spline (NURBS) patch (Dim $= 1$ for curve, Dim $= 2$ for surface, Dim $= 3$ for volume).
    \item NCtrlPts: number of control point in each direction stored as a 1D array.
    \item Order: order of B-spline (NURBS) patch per each direction stored as a 1D array.
    \item NNP: number of total control points on the B-spline (NURBS) patch.
    \item Orientation: orientation of the B-spline (NURBS) patch (sign of Jacobian of mapping), usually in computation we use absolute value of this Jacobian determinant. This infomation is needed in some cases such as to determine the orientation of the normal vector of the boundary, etc.
\end{itemize}
\subsection{Basis functions}
There are two primary routines implemented in SIMO Package which are intended to evaluate the non-zero values of univariate basis function at parametric points, one is named ``BasisFuns.m'' and the other is ``DersBasisFuns.m'', while the former is mainly used in generating geometry, the latter is used in analysis as it simultaneously evaluates the corresponding derivative values of basis functions. The syntax to use these functions is described as follows
\begin{itemize}
    \item For the routine ``BasisFuns.m''
    \begin{lstlisting}
        N0 = BasisFuns(Idx, Pts, p, KntVect);
    \end{lstlisting}
    Input parameters:
    \begin{itemize}
        \item \lstinline{Idx}: span indices of the corresponding parametric points, it can be a single value or an array (the number element of this array have to be equal to the number element of \lstinline{Pts} array). This array is usually generated by the ``FindSpan.m'' routine.
        \item \lstinline{Pts}: parametric points at which the values of basis functions are computed
        \item \lstinline{p}: degree of the basis functions
        \item \lstinline{KntVect}: knot vector defining the parameterization
    \end{itemize}
    Output parameter:
    \begin{itemize}
        \item \lstinline{N0}: a 2D array which stores evaluated basis functions values of the input parametric points, where each row contains $p + 1$ values computed at a parametric point (as at each parametric point there are always exist $p + 1$ basis functions).
    \end{itemize}
    \item For the routine ``DersBasisFuns.m''
    \begin{lstlisting}
        N0n = DersBasisFuns(Idx, Pts, p, n, KntVect);
    \end{lstlisting}
    Input parameters:
    \begin{itemize}
        \item \lstinline{Idx}, \lstinline{Pts}, \lstinline{p}, \lstinline{KntVect}: similar as aforementioned routine.
        \item \lstinline{n}: order of derivatives we would like to compute
    \end{itemize}
    Output parameter:
    \begin{itemize}
        \item N0n: a 3D array holding values of evaluated basis functions and its derivatives, the size of this array in the third direction depends on the maximum order of derivatives, starting from the zeroth-order derivatives (basis functions) in the first slice, then the first-order derivatives in the second slice, etc., e.g
            \begin{itemize}
                \item To retrieve the basis functions, we can simply use the command
                \begin{lstlisting}
                    N0 = N0n(:, :, 1);
                \end{lstlisting}
                \item To retrieve the first derivatives of basis functions, we invoke
                 \begin{lstlisting}
                    N0 = N0n(:, :, 2);
                \end{lstlisting}
            \end{itemize}
    \end{itemize}
\end{itemize}
\section{Enumeration and the Local-to-Global Connectivity Array}
For the purpose of calculating elemental matrices as well as performing matrix assembly, we need to specify which data for an Isogeometric structured mesh of a patch should be generated. These data are listed below
\begin{enumerate}
    \item Global control point numbers
    \item Element numbers
    \item The element-to-control point connectivity array
    \item Local control point numbers
\end{enumerate}
\begin{figure}[H]
    \centering
    \tikzsetnextfilename{Ch3SurfAQuarterOfACylinderEnum}
    \normalsize
    % This file was created by matlab2tikz.
%
\definecolor{mycolor1}{rgb}{0.38000,0.54800,0.24000}%

\definecolor{mycolor2}{rgb}{1,0.4,0}
%
\begin{tikzpicture}

\begin{axis}[%
width=10cm,
scale only axis,
plot box ratio=1 1 1,
point meta min=0,
point meta max=1,
xmin=0,
xmax=1.5,
tick align=outside,
%xlabel={x},
ymin=0,
ymax=1,
%ylabel={y},
zmin=0,
zmax=1,
%zlabel={z},
ticks=none,
%yticklabels={,,},
view={-37.5}{30},
axis background/.style={fill=white},
axis x line*=bottom,
axis y line*=left,
axis z line*=left,
unit vector ratio=1 1 1,%
]

\addplot3[%
surf,
shader=interp,colormap={mymap}{[1pt] rgb(0pt)=(0.692,0.936,0.936); rgb(2pt)=(0.692,0.936,0.936)},mesh/rows=101]
table[row sep=crcr, point meta=\thisrow{c}] {%
%
x	y	z	c\\
0	0	0	0.710195569525497\\
0	0.000100583311362513	0.0141829653360114	0.71240460418299\\
0	0.000404645907256436	0.0284451766772965	0.714630690536905\\
0	0.000915607803432999	0.0427829086109896	0.716916312064106\\
0	0.00163681893109844	0.057192295687301	0.719289019360234\\
0	0.00257155309398959	0.0716693330697585	0.721780367061016\\
0	0.00372300185733413	0.0862098774609879	0.724425655492835\\
0	0.00509426838162598	0.100809648312589	0.727263413850304\\
0	0.00668836121491815	0.115464229327074	0.73033458997498\\
0	0.00850818805808554	0.13016907025918	0.733681422514709\\
0	0.0105565495182326	0.144919489023162	0.737345986077549\\
0	0.0128361328661109	0.159710674111862	0.741368418705642\\
0	0.0153495058140643	0.174537687332543	0.745784862958201\\
0	0.0180991103316243	0.189395466863524	0.750625176092311\\
0	0.0210872565164405	0.204278830634725	0.755910489868999\\
0	0.0243161165387281	0.219182480034174	0.761650724664453\\
0	0.0277877186778607	0.234101003941464	0.767842183867616\\
0	0.0315039414701067	0.24902888308801	0.774465370920409\\
0	0.0354665079868145	0.263960494742775	0.781483180780605\\
0	0.0396769802625738	0.278890117720924	0.788839618268074\\
0	0.0441367538930258	0.293811937711588	0.796459186325545\\
0	0.0488470528220538	0.308720052919643	0.804247066923457\\
0	0.0538089243380495	0.323608480015096	0.812090186161949\\
0	0.0590232342988274	0.338471160382329	0.819859213932693\\
0	0.064490662604533	0.3533019666601	0.82741149906364\\
0	0.0702116989375697	0.368094709561873	0.834594885813506\\
0	0.0761866387881432	0.382843144964686	0.841252300287183\\
0	0.0824155797834956	0.397540981253432	0.847226939729085\\
0	0.0888984183382709	0.412181886906127	0.852367847905041\\
0	0.0956348466427212	0.42675949830445	0.856535620027837\\
0	0.102624350004627	0.441267427752584	0.85960795461962\\
0	0.109866204559871	0.455699271686218	0.861484760266613\\
0	0.117359475365564	0.470048619052377	0.862092534251544\\
0	0.125103014888515	0.484309059839721	0.861387758068377\\
0	0.133095461900593	0.498474193737904	0.85935910089082\\
0	0.14133524079124	0.512537638903683	0.856028283765731\\
0	0.149820561306021	0.526493040810599	0.851449530870176\\
0	0.158549418718625	0.540334081158348	0.845707614739506\\
0	0.167519594442213	0.554054486817265	0.83891458434315\\
0	0.176728657084455	0.567648038782867	0.831205342370843\\
0	0.186173963948925	0.581108581114919	0.822732305417527\\
0	0.195852662983903	0.594430029835236	0.813659432944298\\
0	0.205761695177907	0.607606381758231	0.804155944104871\\
0	0.215897797399558	0.620631723228143	0.794390053406975\\
0	0.226257505677663	0.633500238737009	0.784523046062794\\
0	0.236837158915675	0.646206219397571	0.774703982861581\\
0	0.247632903032949	0.658744071245709	0.765065275211018\\
0	0.258640695523527	0.671108323347402	0.755719307772962\\
0	0.269856310421543	0.683293635685828	0.746756214005913\\
0	0.28127534366066	0.695294806804925	0.738242834604293\\
0	0.292893218813452	0.707106781186547	0.730222816007833\\
0	0.304705193195075	0.71872465633934	0.722717741087772\\
0	0.316706364314172	0.730143689578457	0.715729131143268\\
0	0.328891676652598	0.741359304476472	0.709241120561855\\
0	0.341255928754291	0.752367096967051	0.703223584571265\\
0	0.353793780602429	0.763162841084325	0.697635496597419\\
0	0.366499761262991	0.773742494322337	0.692428303594618\\
0	0.379368276771857	0.784102202600442	0.687549132887069\\
0	0.392393618241769	0.794238304822092	0.682943679235329\\
0	0.405569970164763	0.804147337016097	0.678558662193045\\
0	0.418891418885081	0.813826036051075	0.674343787412869\\
0	0.432351961217132	0.823271342915544	0.670253187715552\\
0	0.445945513182735	0.832480405557786	0.66624635733616\\
0	0.459665918841652	0.841450581281375	0.662288623474186\\
0	0.473506959189401	0.850179438693979	0.658351221673189\\
0	0.487462361096317	0.85866475920876	0.654411055137884\\
0	0.501525806262096	0.866904538099407	0.650450223206468\\
0	0.515690940160279	0.874896985111485	0.646455401874957\\
0	0.529951380947623	0.882640524634436	0.642417151055153\\
0	0.544300728313782	0.890133795440129	0.638329210947586\\
0	0.558732572247415	0.897375649995373	0.634187835387498\\
0	0.57324050169555	0.904365153357279	0.629991194998378\\
0	0.587818113093873	0.911101581661729	0.625738868900924\\
0	0.602459018746568	0.917584420216504	0.621431431633141\\
0	0.617156855035314	0.923813361211857	0.617070132480553\\
0	0.631905290438127	0.92978830106243	0.612656657833839\\
0	0.6466980333399	0.935509337395467	0.608192963377089\\
0	0.661528839617671	0.940976765701173	0.603681161495104\\
0	0.676391519984904	0.94619107566195	0.599123449738604\\
0	0.691279947080357	0.951152947177946	0.59452206790159\\
0	0.706188062288412	0.955863246106974	0.589879273662388\\
0	0.721109882279076	0.960323019737426	0.585197329319935\\
0	0.736039505257226	0.964533492013186	0.580478494543978\\
0	0.75097111691199	0.968496058529893	0.575725022018315\\
0	0.765898996058536	0.972212281322139	0.570939154289557\\
0	0.780817519965826	0.975683883461272	0.566123121053178\\
0	0.795721169365275	0.97891274348356	0.56127913660556\\
0	0.810604533136476	0.981900889668376	0.556409397399202\\
0	0.825462312667457	0.984650494185936	0.551516079695734\\
0	0.840289325888138	0.987163867133889	0.546601337319641\\
0	0.855080510976839	0.989443450481767	0.541667299515142\\
0	0.86983092974082	0.991491811941914	0.536716068908599\\
0	0.884535770672926	0.993311638785082	0.531749719578143\\
0	0.899190351687411	0.994905731618374	0.526770295231792\\
0	0.913790122539012	0.996276998142666	0.5217798074951\\
0	0.928330666930242	0.997428446906011	0.516780234308683\\
0	0.942807704312699	0.998363181068902	0.511773518435873\\
0	0.95721709138901	0.999084392196567	0.50676156608017\\
0	0.971554823322703	0.999595354092743	0.501746245611877\\
0	0.985817034663989	0.999899416688637	0.496729386403004\\
0	1	1	0.491669087346906\\
0.015	0	0	0.710195569525497\\
0.015	0.000100583311362513	0.0141829653360114	0.71240460418299\\
0.015	0.000404645907256436	0.0284451766772965	0.714630690536905\\
0.015	0.000915607803432999	0.0427829086109896	0.716916312064106\\
0.015	0.00163681893109844	0.057192295687301	0.719289019360234\\
0.015	0.00257155309398959	0.0716693330697584	0.721780367061016\\
0.015	0.00372300185733413	0.0862098774609879	0.724425655492835\\
0.015	0.00509426838162598	0.100809648312589	0.727263413850304\\
0.015	0.00668836121491815	0.115464229327074	0.73033458997498\\
0.015	0.00850818805808555	0.13016907025918	0.733681422514709\\
0.015	0.0105565495182326	0.144919489023162	0.737345986077549\\
0.015	0.0128361328661109	0.159710674111862	0.741368418705642\\
0.015	0.0153495058140643	0.174537687332543	0.745784862958201\\
0.015	0.0180991103316243	0.189395466863524	0.750625176092311\\
0.015	0.0210872565164405	0.204278830634725	0.755910489868998\\
0.015	0.0243161165387281	0.219182480034174	0.761650724664453\\
0.015	0.0277877186778607	0.234101003941464	0.767842183867616\\
0.015	0.0315039414701067	0.24902888308801	0.77446537092041\\
0.015	0.0354665079868145	0.263960494742775	0.781483180780606\\
0.015	0.0396769802625738	0.278890117720924	0.788839618268073\\
0.015	0.0441367538930258	0.293811937711588	0.796459186325543\\
0.015	0.0488470528220538	0.308720052919643	0.804247066923455\\
0.015	0.0538089243380495	0.323608480015096	0.812090186161951\\
0.015	0.0590232342988274	0.338471160382329	0.819859213932693\\
0.015	0.064490662604533	0.3533019666601	0.827411499063639\\
0.015	0.0702116989375697	0.368094709561873	0.834594885813506\\
0.015	0.0761866387881432	0.382843144964686	0.841252300287178\\
0.015	0.0824155797834956	0.397540981253432	0.847226939729082\\
0.015	0.0888984183382709	0.412181886906127	0.852367847905041\\
0.015	0.0956348466427212	0.42675949830445	0.856535620027827\\
0.015	0.102624350004627	0.441267427752584	0.859607954619619\\
0.015	0.109866204559871	0.455699271686218	0.861484760266622\\
0.015	0.117359475365564	0.470048619052377	0.862092534251539\\
0.015	0.125103014888515	0.484309059839721	0.861387758068368\\
0.015	0.133095461900593	0.498474193737904	0.85935910089082\\
0.015	0.14133524079124	0.512537638903683	0.856028283765736\\
0.015	0.149820561306021	0.526493040810599	0.851449530870183\\
0.015	0.158549418718625	0.540334081158348	0.845707614739506\\
0.015	0.167519594442213	0.554054486817265	0.838914584343151\\
0.015	0.176728657084455	0.567648038782868	0.831205342370837\\
0.015	0.186173963948925	0.581108581114919	0.822732305417523\\
0.015	0.195852662983903	0.594430029835236	0.813659432944306\\
0.015	0.205761695177907	0.607606381758231	0.804155944104874\\
0.015	0.215897797399558	0.620631723228143	0.794390053406967\\
0.015	0.226257505677663	0.633500238737009	0.784523046062799\\
0.015	0.236837158915675	0.646206219397571	0.774703982861595\\
0.015	0.247632903032949	0.658744071245709	0.765065275211018\\
0.015	0.258640695523527	0.671108323347402	0.755719307772961\\
0.015	0.269856310421543	0.683293635685828	0.746756214005912\\
0.015	0.28127534366066	0.695294806804925	0.738242834604295\\
0.015	0.292893218813452	0.707106781186547	0.730222816007835\\
0.015	0.304705193195075	0.71872465633934	0.722717741087768\\
0.015	0.316706364314172	0.730143689578457	0.715729131143267\\
0.015	0.328891676652598	0.741359304476472	0.709241120561857\\
0.015	0.341255928754291	0.752367096967051	0.703223584571265\\
0.015	0.353793780602429	0.763162841084325	0.697635496597416\\
0.015	0.366499761262991	0.773742494322337	0.692428303594618\\
0.015	0.379368276771857	0.784102202600442	0.68754913288707\\
0.015	0.392393618241769	0.794238304822092	0.682943679235329\\
0.015	0.405569970164763	0.804147337016097	0.678558662193045\\
0.015	0.418891418885081	0.813826036051075	0.674343787412871\\
0.015	0.432351961217133	0.823271342915545	0.67025318771555\\
0.015	0.445945513182735	0.832480405557786	0.666246357336158\\
0.015	0.459665918841652	0.841450581281375	0.662288623474187\\
0.015	0.473506959189401	0.850179438693979	0.658351221673192\\
0.015	0.487462361096317	0.85866475920876	0.654411055137885\\
0.015	0.501525806262096	0.866904538099407	0.650450223206468\\
0.015	0.515690940160279	0.874896985111485	0.64645540187496\\
0.015	0.529951380947623	0.882640524634437	0.642417151055155\\
0.015	0.544300728313781	0.890133795440129	0.638329210947585\\
0.015	0.558732572247415	0.897375649995373	0.634187835387497\\
0.015	0.57324050169555	0.904365153357279	0.629991194998376\\
0.015	0.587818113093873	0.911101581661729	0.625738868900926\\
0.015	0.602459018746568	0.917584420216505	0.621431431633141\\
0.015	0.617156855035314	0.923813361211857	0.617070132480552\\
0.015	0.631905290438127	0.92978830106243	0.612656657833836\\
0.015	0.6466980333399	0.935509337395467	0.608192963377089\\
0.015	0.661528839617671	0.940976765701173	0.603681161495107\\
0.015	0.676391519984904	0.946191075661951	0.599123449738607\\
0.015	0.691279947080357	0.951152947177946	0.594522067901589\\
0.015	0.706188062288412	0.955863246106974	0.589879273662391\\
0.015	0.721109882279076	0.960323019737426	0.585197329319937\\
0.015	0.736039505257226	0.964533492013186	0.580478494543973\\
0.015	0.75097111691199	0.968496058529893	0.575725022018313\\
0.015	0.765898996058536	0.972212281322139	0.570939154289559\\
0.015	0.780817519965826	0.975683883461272	0.566123121053175\\
0.015	0.795721169365275	0.97891274348356	0.56127913660556\\
0.015	0.810604533136476	0.981900889668376	0.556409397399199\\
0.015	0.825462312667457	0.984650494185936	0.551516079695735\\
0.015	0.840289325888138	0.987163867133889	0.546601337319639\\
0.015	0.855080510976839	0.989443450481767	0.541667299515142\\
0.015	0.86983092974082	0.991491811941915	0.536716068908601\\
0.015	0.884535770672926	0.993311638785082	0.531749719578143\\
0.015	0.899190351687411	0.994905731618374	0.526770295231794\\
0.015	0.913790122539012	0.996276998142666	0.521779807495101\\
0.015	0.928330666930242	0.997428446906011	0.516780234308682\\
0.015	0.942807704312699	0.998363181068902	0.511773518435872\\
0.015	0.95721709138901	0.999084392196567	0.506761566080168\\
0.015	0.971554823322703	0.999595354092743	0.501746245611874\\
0.015	0.985817034663988	0.999899416688637	0.496729386403005\\
0.015	1	1	0.491669087346904\\
0.03	0	0	0.710195569525497\\
0.03	0.000100583311362513	0.0141829653360114	0.71240460418299\\
0.03	0.000404645907256436	0.0284451766772965	0.714630690536905\\
0.03	0.000915607803432999	0.0427829086109896	0.716916312064106\\
0.03	0.00163681893109844	0.057192295687301	0.719289019360234\\
0.03	0.00257155309398959	0.0716693330697584	0.721780367061016\\
0.03	0.00372300185733414	0.0862098774609879	0.724425655492835\\
0.03	0.00509426838162598	0.100809648312589	0.727263413850304\\
0.03	0.00668836121491816	0.115464229327074	0.73033458997498\\
0.03	0.00850818805808554	0.13016907025918	0.733681422514709\\
0.03	0.0105565495182326	0.144919489023162	0.737345986077549\\
0.03	0.0128361328661109	0.159710674111862	0.741368418705642\\
0.03	0.0153495058140643	0.174537687332543	0.745784862958202\\
0.03	0.0180991103316243	0.189395466863524	0.750625176092311\\
0.03	0.0210872565164405	0.204278830634725	0.755910489868998\\
0.03	0.0243161165387281	0.219182480034174	0.761650724664453\\
0.03	0.0277877186778607	0.234101003941464	0.767842183867616\\
0.03	0.0315039414701067	0.24902888308801	0.77446537092041\\
0.03	0.0354665079868145	0.263960494742775	0.781483180780606\\
0.03	0.0396769802625738	0.278890117720924	0.788839618268073\\
0.03	0.0441367538930258	0.293811937711588	0.796459186325544\\
0.03	0.0488470528220538	0.308720052919643	0.804247066923454\\
0.03	0.0538089243380495	0.323608480015096	0.81209018616195\\
0.03	0.0590232342988274	0.338471160382329	0.819859213932694\\
0.03	0.064490662604533	0.3533019666601	0.827411499063639\\
0.03	0.0702116989375697	0.368094709561873	0.834594885813505\\
0.03	0.0761866387881432	0.382843144964686	0.841252300287177\\
0.03	0.0824155797834956	0.397540981253432	0.847226939729079\\
0.03	0.0888984183382709	0.412181886906127	0.852367847905042\\
0.03	0.0956348466427212	0.42675949830445	0.856535620027829\\
0.03	0.102624350004627	0.441267427752584	0.859607954619617\\
0.03	0.109866204559871	0.455699271686218	0.861484760266622\\
0.03	0.117359475365564	0.470048619052377	0.862092534251539\\
0.03	0.125103014888515	0.484309059839721	0.861387758068369\\
0.03	0.133095461900593	0.498474193737904	0.85935910089082\\
0.03	0.14133524079124	0.512537638903683	0.856028283765736\\
0.03	0.149820561306021	0.526493040810599	0.851449530870183\\
0.03	0.158549418718625	0.540334081158348	0.845707614739501\\
0.03	0.167519594442213	0.554054486817265	0.83891458434315\\
0.03	0.176728657084455	0.567648038782867	0.831205342370836\\
0.03	0.186173963948925	0.581108581114919	0.822732305417527\\
0.03	0.195852662983903	0.594430029835236	0.813659432944309\\
0.03	0.205761695177907	0.607606381758231	0.804155944104868\\
0.03	0.215897797399558	0.620631723228143	0.794390053406968\\
0.03	0.226257505677663	0.633500238737009	0.784523046062806\\
0.03	0.236837158915675	0.646206219397571	0.774703982861591\\
0.03	0.247632903032949	0.658744071245709	0.765065275211014\\
0.03	0.258640695523527	0.671108323347402	0.755719307772961\\
0.03	0.269856310421543	0.683293635685828	0.746756214005913\\
0.03	0.28127534366066	0.695294806804925	0.738242834604297\\
0.03	0.292893218813452	0.707106781186547	0.730222816007833\\
0.03	0.304705193195075	0.71872465633934	0.722717741087768\\
0.03	0.316706364314172	0.730143689578457	0.715729131143268\\
0.03	0.328891676652598	0.741359304476472	0.709241120561855\\
0.03	0.341255928754291	0.752367096967051	0.703223584571265\\
0.03	0.353793780602429	0.763162841084325	0.697635496597419\\
0.03	0.366499761262991	0.773742494322337	0.692428303594617\\
0.03	0.379368276771856	0.784102202600442	0.687549132887069\\
0.03	0.392393618241769	0.794238304822092	0.68294367923533\\
0.03	0.405569970164763	0.804147337016097	0.678558662193045\\
0.03	0.418891418885081	0.813826036051075	0.674343787412872\\
0.03	0.432351961217133	0.823271342915545	0.670253187715553\\
0.03	0.445945513182735	0.832480405557786	0.666246357336158\\
0.03	0.459665918841652	0.841450581281375	0.662288623474185\\
0.03	0.473506959189401	0.850179438693979	0.658351221673191\\
0.03	0.487462361096317	0.85866475920876	0.654411055137887\\
0.03	0.501525806262096	0.866904538099407	0.650450223206468\\
0.03	0.515690940160279	0.874896985111485	0.646455401874957\\
0.03	0.529951380947623	0.882640524634436	0.642417151055156\\
0.03	0.544300728313782	0.890133795440129	0.638329210947589\\
0.03	0.558732572247415	0.897375649995373	0.634187835387497\\
0.03	0.57324050169555	0.904365153357279	0.629991194998376\\
0.03	0.587818113093873	0.911101581661729	0.625738868900925\\
0.03	0.602459018746568	0.917584420216504	0.621431431633141\\
0.03	0.617156855035314	0.923813361211857	0.617070132480552\\
0.03	0.631905290438127	0.92978830106243	0.612656657833836\\
0.03	0.6466980333399	0.935509337395467	0.608192963377089\\
0.03	0.661528839617671	0.940976765701173	0.603681161495105\\
0.03	0.676391519984903	0.94619107566195	0.599123449738607\\
0.03	0.691279947080357	0.951152947177946	0.59452206790159\\
0.03	0.706188062288412	0.955863246106974	0.589879273662389\\
0.03	0.721109882279076	0.960323019737426	0.585197329319938\\
0.03	0.736039505257226	0.964533492013186	0.580478494543977\\
0.03	0.75097111691199	0.968496058529893	0.575725022018312\\
0.03	0.765898996058536	0.972212281322139	0.570939154289559\\
0.03	0.780817519965826	0.975683883461272	0.566123121053176\\
0.03	0.795721169365275	0.97891274348356	0.56127913660556\\
0.03	0.810604533136476	0.981900889668376	0.556409397399197\\
0.03	0.825462312667457	0.984650494185936	0.551516079695734\\
0.03	0.840289325888138	0.987163867133889	0.546601337319643\\
0.03	0.855080510976838	0.989443450481768	0.541667299515139\\
0.03	0.86983092974082	0.991491811941914	0.536716068908599\\
0.03	0.884535770672926	0.993311638785082	0.531749719578144\\
0.03	0.899190351687411	0.994905731618374	0.526770295231793\\
0.03	0.913790122539012	0.996276998142666	0.521779807495103\\
0.03	0.928330666930242	0.997428446906011	0.516780234308686\\
0.03	0.942807704312699	0.998363181068902	0.511773518435872\\
0.03	0.95721709138901	0.999084392196567	0.506761566080168\\
0.03	0.971554823322703	0.999595354092743	0.501746245611874\\
0.03	0.985817034663989	0.999899416688637	0.496729386403004\\
0.03	1	1	0.491669087346906\\
0.045	0	0	0.710195569525497\\
0.045	0.000100583311362513	0.0141829653360114	0.71240460418299\\
0.045	0.000404645907256436	0.0284451766772965	0.714630690536905\\
0.045	0.000915607803432999	0.0427829086109896	0.716916312064106\\
0.045	0.00163681893109843	0.057192295687301	0.719289019360234\\
0.045	0.00257155309398959	0.0716693330697584	0.721780367061016\\
0.045	0.00372300185733413	0.0862098774609879	0.724425655492835\\
0.045	0.00509426838162598	0.100809648312589	0.727263413850304\\
0.045	0.00668836121491815	0.115464229327074	0.73033458997498\\
0.045	0.00850818805808554	0.13016907025918	0.733681422514709\\
0.045	0.0105565495182326	0.144919489023162	0.737345986077549\\
0.045	0.0128361328661109	0.159710674111862	0.741368418705642\\
0.045	0.0153495058140643	0.174537687332543	0.745784862958202\\
0.045	0.0180991103316243	0.189395466863524	0.750625176092311\\
0.045	0.0210872565164405	0.204278830634725	0.755910489868999\\
0.045	0.0243161165387281	0.219182480034174	0.761650724664453\\
0.045	0.0277877186778607	0.234101003941464	0.767842183867616\\
0.045	0.0315039414701067	0.24902888308801	0.77446537092041\\
0.045	0.0354665079868145	0.263960494742775	0.781483180780606\\
0.045	0.0396769802625738	0.278890117720924	0.788839618268073\\
0.045	0.0441367538930258	0.293811937711588	0.796459186325544\\
0.045	0.0488470528220538	0.308720052919643	0.804247066923454\\
0.045	0.0538089243380495	0.323608480015096	0.812090186161949\\
0.045	0.0590232342988274	0.338471160382329	0.819859213932692\\
0.045	0.064490662604533	0.3533019666601	0.82741149906364\\
0.045	0.0702116989375697	0.368094709561873	0.834594885813504\\
0.045	0.0761866387881432	0.382843144964686	0.841252300287181\\
0.045	0.0824155797834956	0.397540981253432	0.847226939729081\\
0.045	0.0888984183382709	0.412181886906127	0.852367847905041\\
0.045	0.0956348466427212	0.42675949830445	0.85653562002783\\
0.045	0.102624350004627	0.441267427752584	0.859607954619617\\
0.045	0.109866204559871	0.455699271686218	0.861484760266623\\
0.045	0.117359475365564	0.470048619052377	0.862092534251542\\
0.045	0.125103014888515	0.484309059839721	0.861387758068371\\
0.045	0.133095461900593	0.498474193737904	0.859359100890818\\
0.045	0.14133524079124	0.512537638903683	0.856028283765736\\
0.045	0.149820561306021	0.526493040810599	0.851449530870182\\
0.045	0.158549418718625	0.540334081158348	0.845707614739503\\
0.045	0.167519594442213	0.554054486817265	0.838914584343151\\
0.045	0.176728657084455	0.567648038782867	0.831205342370837\\
0.045	0.186173963948925	0.581108581114919	0.822732305417526\\
0.045	0.195852662983903	0.594430029835236	0.813659432944307\\
0.045	0.205761695177907	0.607606381758231	0.804155944104868\\
0.045	0.215897797399558	0.620631723228143	0.794390053406972\\
0.045	0.226257505677663	0.633500238737009	0.784523046062799\\
0.045	0.236837158915675	0.646206219397571	0.774703982861588\\
0.045	0.247632903032949	0.658744071245709	0.765065275211018\\
0.045	0.258640695523527	0.671108323347402	0.755719307772962\\
0.045	0.269856310421543	0.683293635685828	0.746756214005913\\
0.045	0.28127534366066	0.695294806804925	0.738242834604295\\
0.045	0.292893218813452	0.707106781186547	0.730222816007832\\
0.045	0.304705193195075	0.71872465633934	0.722717741087768\\
0.045	0.316706364314172	0.730143689578457	0.715729131143267\\
0.045	0.328891676652598	0.741359304476472	0.709241120561856\\
0.045	0.341255928754291	0.752367096967051	0.703223584571266\\
0.045	0.353793780602429	0.763162841084325	0.697635496597416\\
0.045	0.366499761262991	0.773742494322337	0.692428303594618\\
0.045	0.379368276771857	0.784102202600442	0.687549132887071\\
0.045	0.392393618241769	0.794238304822092	0.682943679235329\\
0.045	0.405569970164763	0.804147337016097	0.678558662193043\\
0.045	0.418891418885081	0.813826036051075	0.67434378741287\\
0.045	0.432351961217132	0.823271342915544	0.670253187715553\\
0.045	0.445945513182735	0.832480405557786	0.666246357336159\\
0.045	0.459665918841652	0.841450581281375	0.662288623474187\\
0.045	0.473506959189401	0.850179438693979	0.658351221673191\\
0.045	0.487462361096317	0.85866475920876	0.654411055137886\\
0.045	0.501525806262096	0.866904538099407	0.650450223206468\\
0.045	0.515690940160279	0.874896985111485	0.646455401874959\\
0.045	0.529951380947623	0.882640524634437	0.642417151055156\\
0.045	0.544300728313781	0.890133795440129	0.638329210947585\\
0.045	0.558732572247415	0.897375649995373	0.634187835387497\\
0.045	0.57324050169555	0.904365153357279	0.629991194998375\\
0.045	0.587818113093873	0.911101581661729	0.625738868900925\\
0.045	0.602459018746568	0.917584420216504	0.621431431633141\\
0.045	0.617156855035314	0.923813361211857	0.617070132480553\\
0.045	0.631905290438127	0.92978830106243	0.612656657833835\\
0.045	0.6466980333399	0.935509337395467	0.608192963377088\\
0.045	0.661528839617671	0.940976765701173	0.603681161495108\\
0.045	0.676391519984904	0.946191075661951	0.599123449738606\\
0.045	0.691279947080358	0.951152947177946	0.594522067901589\\
0.045	0.706188062288412	0.955863246106974	0.58987927366239\\
0.045	0.721109882279076	0.960323019737426	0.585197329319934\\
0.045	0.736039505257226	0.964533492013185	0.580478494543975\\
0.045	0.75097111691199	0.968496058529893	0.575725022018315\\
0.045	0.765898996058536	0.972212281322139	0.570939154289561\\
0.045	0.780817519965826	0.975683883461272	0.566123121053174\\
0.045	0.795721169365275	0.978912743483559	0.561279136605557\\
0.045	0.810604533136476	0.981900889668376	0.556409397399201\\
0.045	0.825462312667457	0.984650494185936	0.551516079695733\\
0.045	0.840289325888138	0.987163867133889	0.546601337319641\\
0.045	0.855080510976839	0.989443450481767	0.541667299515141\\
0.045	0.86983092974082	0.991491811941914	0.536716068908597\\
0.045	0.884535770672926	0.993311638785082	0.531749719578144\\
0.045	0.899190351687411	0.994905731618374	0.526770295231795\\
0.045	0.913790122539012	0.996276998142666	0.521779807495099\\
0.045	0.928330666930241	0.99742844690601	0.516780234308683\\
0.045	0.942807704312699	0.998363181068901	0.511773518435875\\
0.045	0.95721709138901	0.999084392196567	0.50676156608017\\
0.045	0.971554823322703	0.999595354092743	0.501746245611874\\
0.045	0.985817034663989	0.999899416688637	0.496729386403004\\
0.045	1	1	0.491669087346906\\
0.06	0	0	0.710195569525497\\
0.06	0.000100583311362513	0.0141829653360114	0.71240460418299\\
0.06	0.000404645907256436	0.0284451766772965	0.714630690536905\\
0.06	0.000915607803432999	0.0427829086109896	0.716916312064106\\
0.06	0.00163681893109844	0.057192295687301	0.719289019360234\\
0.06	0.00257155309398959	0.0716693330697585	0.721780367061016\\
0.06	0.00372300185733413	0.0862098774609879	0.724425655492835\\
0.06	0.00509426838162598	0.100809648312589	0.727263413850304\\
0.06	0.00668836121491815	0.115464229327074	0.73033458997498\\
0.06	0.00850818805808554	0.13016907025918	0.733681422514709\\
0.06	0.0105565495182326	0.144919489023162	0.737345986077549\\
0.06	0.0128361328661109	0.159710674111862	0.741368418705642\\
0.06	0.0153495058140643	0.174537687332543	0.745784862958201\\
0.06	0.0180991103316243	0.189395466863524	0.750625176092311\\
0.06	0.0210872565164405	0.204278830634725	0.755910489868999\\
0.06	0.0243161165387281	0.219182480034174	0.761650724664453\\
0.06	0.0277877186778607	0.234101003941464	0.767842183867616\\
0.06	0.0315039414701067	0.24902888308801	0.77446537092041\\
0.06	0.0354665079868145	0.263960494742775	0.781483180780606\\
0.06	0.0396769802625737	0.278890117720924	0.788839618268073\\
0.06	0.0441367538930258	0.293811937711588	0.796459186325544\\
0.06	0.0488470528220538	0.308720052919643	0.804247066923455\\
0.06	0.0538089243380495	0.323608480015096	0.81209018616195\\
0.06	0.0590232342988274	0.338471160382329	0.819859213932692\\
0.06	0.064490662604533	0.3533019666601	0.827411499063639\\
0.06	0.0702116989375697	0.368094709561873	0.834594885813506\\
0.06	0.0761866387881432	0.382843144964686	0.841252300287181\\
0.06	0.0824155797834956	0.397540981253432	0.847226939729083\\
0.06	0.0888984183382709	0.412181886906127	0.852367847905042\\
0.06	0.0956348466427212	0.42675949830445	0.856535620027826\\
0.06	0.102624350004627	0.441267427752584	0.859607954619617\\
0.06	0.109866204559871	0.455699271686218	0.861484760266622\\
0.06	0.117359475365564	0.470048619052377	0.862092534251542\\
0.06	0.125103014888515	0.484309059839721	0.861387758068366\\
0.06	0.133095461900593	0.498474193737904	0.859359100890822\\
0.06	0.14133524079124	0.512537638903683	0.856028283765738\\
0.06	0.149820561306021	0.526493040810599	0.851449530870184\\
0.06	0.158549418718625	0.540334081158348	0.845707614739505\\
0.06	0.167519594442213	0.554054486817265	0.83891458434315\\
0.06	0.176728657084455	0.567648038782867	0.831205342370836\\
0.06	0.186173963948925	0.581108581114919	0.822732305417526\\
0.06	0.195852662983903	0.594430029835236	0.813659432944306\\
0.06	0.205761695177907	0.607606381758231	0.804155944104869\\
0.06	0.215897797399558	0.620631723228143	0.794390053406971\\
0.06	0.226257505677663	0.633500238737009	0.784523046062801\\
0.06	0.236837158915675	0.646206219397571	0.774703982861595\\
0.06	0.247632903032949	0.658744071245709	0.765065275211015\\
0.06	0.258640695523527	0.671108323347402	0.755719307772962\\
0.06	0.269856310421543	0.683293635685828	0.746756214005911\\
0.06	0.28127534366066	0.695294806804924	0.738242834604296\\
0.06	0.292893218813452	0.707106781186547	0.730222816007834\\
0.06	0.304705193195075	0.71872465633934	0.722717741087768\\
0.06	0.316706364314172	0.730143689578457	0.715729131143267\\
0.06	0.328891676652598	0.741359304476472	0.709241120561856\\
0.06	0.341255928754291	0.752367096967051	0.703223584571266\\
0.06	0.353793780602429	0.763162841084325	0.697635496597418\\
0.06	0.366499761262991	0.773742494322337	0.692428303594617\\
0.06	0.379368276771857	0.784102202600442	0.68754913288707\\
0.06	0.392393618241769	0.794238304822092	0.682943679235329\\
0.06	0.405569970164763	0.804147337016097	0.678558662193046\\
0.06	0.418891418885081	0.813826036051075	0.674343787412871\\
0.06	0.432351961217132	0.823271342915545	0.67025318771555\\
0.06	0.445945513182735	0.832480405557786	0.666246357336157\\
0.06	0.459665918841652	0.841450581281375	0.662288623474186\\
0.06	0.473506959189401	0.850179438693979	0.658351221673192\\
0.06	0.487462361096317	0.85866475920876	0.654411055137885\\
0.06	0.501525806262096	0.866904538099407	0.650450223206468\\
0.06	0.515690940160279	0.874896985111485	0.646455401874958\\
0.06	0.529951380947623	0.882640524634436	0.642417151055155\\
0.06	0.544300728313782	0.890133795440129	0.638329210947589\\
0.06	0.558732572247415	0.897375649995373	0.6341878353875\\
0.06	0.57324050169555	0.904365153357279	0.629991194998375\\
0.06	0.587818113093873	0.911101581661729	0.625738868900923\\
0.06	0.602459018746568	0.917584420216504	0.621431431633139\\
0.06	0.617156855035314	0.923813361211857	0.617070132480554\\
0.06	0.631905290438127	0.92978830106243	0.612656657833838\\
0.06	0.6466980333399	0.935509337395467	0.608192963377086\\
0.06	0.661528839617671	0.940976765701173	0.603681161495104\\
0.06	0.676391519984904	0.94619107566195	0.599123449738609\\
0.06	0.691279947080357	0.951152947177946	0.594522067901591\\
0.06	0.706188062288412	0.955863246106974	0.589879273662389\\
0.06	0.721109882279076	0.960323019737426	0.585197329319936\\
0.06	0.736039505257225	0.964533492013186	0.580478494543974\\
0.06	0.75097111691199	0.968496058529893	0.575725022018311\\
0.06	0.765898996058536	0.972212281322139	0.570939154289562\\
0.06	0.780817519965826	0.975683883461272	0.566123121053176\\
0.06	0.795721169365275	0.97891274348356	0.561279136605557\\
0.06	0.810604533136476	0.981900889668376	0.556409397399201\\
0.06	0.825462312667457	0.984650494185936	0.551516079695735\\
0.06	0.840289325888138	0.987163867133889	0.546601337319637\\
0.06	0.855080510976839	0.989443450481767	0.541667299515141\\
0.06	0.86983092974082	0.991491811941915	0.536716068908601\\
0.06	0.884535770672926	0.993311638785082	0.531749719578139\\
0.06	0.899190351687411	0.994905731618374	0.526770295231793\\
0.06	0.913790122539012	0.996276998142666	0.521779807495104\\
0.06	0.928330666930242	0.997428446906011	0.516780234308683\\
0.06	0.942807704312699	0.998363181068902	0.511773518435872\\
0.06	0.95721709138901	0.999084392196567	0.50676156608017\\
0.06	0.971554823322703	0.999595354092743	0.501746245611874\\
0.06	0.985817034663988	0.999899416688637	0.496729386403004\\
0.06	1	1	0.491669087346906\\
0.075	0	0	0.710195569525497\\
0.075	0.000100583311362513	0.0141829653360114	0.71240460418299\\
0.075	0.000404645907256436	0.0284451766772965	0.714630690536905\\
0.075	0.000915607803432999	0.0427829086109896	0.716916312064106\\
0.075	0.00163681893109844	0.057192295687301	0.719289019360234\\
0.075	0.00257155309398959	0.0716693330697584	0.721780367061016\\
0.075	0.00372300185733413	0.086209877460988	0.724425655492835\\
0.075	0.00509426838162598	0.100809648312589	0.727263413850304\\
0.075	0.00668836121491816	0.115464229327074	0.73033458997498\\
0.075	0.00850818805808554	0.13016907025918	0.733681422514709\\
0.075	0.0105565495182326	0.144919489023162	0.737345986077549\\
0.075	0.0128361328661109	0.159710674111862	0.741368418705642\\
0.075	0.0153495058140643	0.174537687332543	0.745784862958201\\
0.075	0.0180991103316243	0.189395466863524	0.750625176092311\\
0.075	0.0210872565164405	0.204278830634725	0.755910489868999\\
0.075	0.0243161165387281	0.219182480034174	0.761650724664453\\
0.075	0.0277877186778607	0.234101003941464	0.767842183867616\\
0.075	0.0315039414701067	0.24902888308801	0.77446537092041\\
0.075	0.0354665079868145	0.263960494742775	0.781483180780606\\
0.075	0.0396769802625738	0.278890117720924	0.788839618268074\\
0.075	0.0441367538930258	0.293811937711588	0.796459186325544\\
0.075	0.0488470528220538	0.308720052919643	0.804247066923454\\
0.075	0.0538089243380495	0.323608480015096	0.812090186161951\\
0.075	0.0590232342988274	0.338471160382329	0.819859213932693\\
0.075	0.064490662604533	0.3533019666601	0.827411499063639\\
0.075	0.0702116989375697	0.368094709561873	0.834594885813505\\
0.075	0.0761866387881432	0.382843144964686	0.841252300287178\\
0.075	0.0824155797834956	0.397540981253432	0.847226939729081\\
0.075	0.0888984183382709	0.412181886906127	0.852367847905042\\
0.075	0.0956348466427212	0.42675949830445	0.856535620027828\\
0.075	0.102624350004627	0.441267427752585	0.859607954619616\\
0.075	0.109866204559871	0.455699271686218	0.86148476026662\\
0.075	0.117359475365564	0.470048619052377	0.862092534251542\\
0.075	0.125103014888515	0.484309059839721	0.861387758068366\\
0.075	0.133095461900593	0.498474193737904	0.859359100890821\\
0.075	0.14133524079124	0.512537638903683	0.856028283765736\\
0.075	0.149820561306021	0.526493040810599	0.851449530870183\\
0.075	0.158549418718625	0.540334081158348	0.845707614739501\\
0.075	0.167519594442213	0.554054486817265	0.83891458434315\\
0.075	0.176728657084455	0.567648038782867	0.831205342370839\\
0.075	0.186173963948925	0.581108581114919	0.822732305417527\\
0.075	0.195852662983903	0.594430029835236	0.813659432944308\\
0.075	0.205761695177907	0.60760638175823	0.804155944104874\\
0.075	0.215897797399558	0.620631723228143	0.794390053406974\\
0.075	0.226257505677663	0.633500238737009	0.784523046062803\\
0.075	0.236837158915675	0.646206219397571	0.77470398286159\\
0.075	0.247632903032949	0.658744071245709	0.765065275211015\\
0.075	0.258640695523527	0.671108323347402	0.755719307772963\\
0.075	0.269856310421543	0.683293635685828	0.746756214005913\\
0.075	0.28127534366066	0.695294806804925	0.738242834604296\\
0.075	0.292893218813452	0.707106781186547	0.730222816007833\\
0.075	0.304705193195075	0.71872465633934	0.722717741087768\\
0.075	0.316706364314172	0.730143689578457	0.715729131143266\\
0.075	0.328891676652598	0.741359304476472	0.709241120561855\\
0.075	0.341255928754291	0.752367096967051	0.703223584571267\\
0.075	0.353793780602429	0.763162841084325	0.697635496597419\\
0.075	0.366499761262991	0.773742494322337	0.692428303594618\\
0.075	0.379368276771856	0.784102202600442	0.68754913288707\\
0.075	0.392393618241769	0.794238304822092	0.682943679235328\\
0.075	0.405569970164763	0.804147337016097	0.678558662193044\\
0.075	0.418891418885081	0.813826036051075	0.674343787412872\\
0.075	0.432351961217132	0.823271342915545	0.670253187715552\\
0.075	0.445945513182735	0.832480405557786	0.666246357336158\\
0.075	0.459665918841652	0.841450581281375	0.662288623474187\\
0.075	0.473506959189401	0.850179438693979	0.65835122167319\\
0.075	0.487462361096317	0.85866475920876	0.654411055137885\\
0.075	0.501525806262096	0.866904538099407	0.650450223206468\\
0.075	0.515690940160279	0.874896985111485	0.646455401874959\\
0.075	0.529951380947623	0.882640524634437	0.642417151055153\\
0.075	0.544300728313782	0.890133795440129	0.638329210947585\\
0.075	0.558732572247415	0.897375649995372	0.6341878353875\\
0.075	0.57324050169555	0.904365153357279	0.629991194998376\\
0.075	0.587818113093873	0.911101581661729	0.625738868900925\\
0.075	0.602459018746568	0.917584420216505	0.621431431633141\\
0.075	0.617156855035314	0.923813361211857	0.617070132480553\\
0.075	0.631905290438127	0.92978830106243	0.612656657833838\\
0.075	0.6466980333399	0.935509337395467	0.608192963377088\\
0.075	0.661528839617671	0.940976765701173	0.603681161495104\\
0.075	0.676391519984904	0.94619107566195	0.599123449738608\\
0.075	0.691279947080357	0.951152947177946	0.594522067901591\\
0.075	0.706188062288412	0.955863246106974	0.589879273662391\\
0.075	0.721109882279076	0.960323019737426	0.585197329319935\\
0.075	0.736039505257226	0.964533492013186	0.580478494543976\\
0.075	0.75097111691199	0.968496058529893	0.575725022018313\\
0.075	0.765898996058536	0.972212281322139	0.570939154289558\\
0.075	0.780817519965826	0.975683883461272	0.566123121053175\\
0.075	0.795721169365275	0.97891274348356	0.561279136605558\\
0.075	0.810604533136476	0.981900889668376	0.556409397399199\\
0.075	0.825462312667457	0.984650494185936	0.551516079695736\\
0.075	0.840289325888138	0.987163867133889	0.546601337319641\\
0.075	0.855080510976839	0.989443450481768	0.541667299515139\\
0.075	0.86983092974082	0.991491811941914	0.536716068908599\\
0.075	0.884535770672926	0.993311638785082	0.531749719578144\\
0.075	0.899190351687411	0.994905731618374	0.526770295231793\\
0.075	0.913790122539012	0.996276998142666	0.521779807495099\\
0.075	0.928330666930242	0.99742844690601	0.516780234308685\\
0.075	0.942807704312699	0.998363181068902	0.511773518435874\\
0.075	0.95721709138901	0.999084392196567	0.50676156608017\\
0.075	0.971554823322704	0.999595354092744	0.501746245611874\\
0.075	0.985817034663989	0.999899416688637	0.496729386403002\\
0.075	1	1	0.491669087346908\\
0.09	0	0	0.710195569525497\\
0.09	0.000100583311362513	0.0141829653360114	0.71240460418299\\
0.09	0.000404645907256436	0.0284451766772965	0.714630690536905\\
0.09	0.000915607803432999	0.0427829086109896	0.716916312064106\\
0.09	0.00163681893109844	0.057192295687301	0.719289019360234\\
0.09	0.00257155309398959	0.0716693330697584	0.721780367061016\\
0.09	0.00372300185733414	0.0862098774609879	0.724425655492835\\
0.09	0.00509426838162598	0.100809648312589	0.727263413850304\\
0.09	0.00668836121491816	0.115464229327074	0.73033458997498\\
0.09	0.00850818805808555	0.13016907025918	0.733681422514709\\
0.09	0.0105565495182326	0.144919489023162	0.737345986077549\\
0.09	0.0128361328661109	0.159710674111862	0.741368418705642\\
0.09	0.0153495058140643	0.174537687332543	0.745784862958201\\
0.09	0.0180991103316243	0.189395466863524	0.750625176092311\\
0.09	0.0210872565164405	0.204278830634725	0.755910489868999\\
0.09	0.0243161165387281	0.219182480034174	0.761650724664453\\
0.09	0.0277877186778607	0.234101003941464	0.767842183867616\\
0.09	0.0315039414701067	0.24902888308801	0.77446537092041\\
0.09	0.0354665079868145	0.263960494742775	0.781483180780605\\
0.09	0.0396769802625738	0.278890117720924	0.788839618268073\\
0.09	0.0441367538930258	0.293811937711588	0.796459186325544\\
0.09	0.0488470528220538	0.308720052919643	0.804247066923455\\
0.09	0.0538089243380495	0.323608480015096	0.812090186161951\\
0.09	0.0590232342988274	0.338471160382329	0.819859213932692\\
0.09	0.064490662604533	0.3533019666601	0.827411499063638\\
0.09	0.0702116989375697	0.368094709561873	0.834594885813504\\
0.09	0.0761866387881432	0.382843144964686	0.841252300287178\\
0.09	0.0824155797834956	0.397540981253432	0.847226939729081\\
0.09	0.0888984183382709	0.412181886906127	0.852367847905041\\
0.09	0.0956348466427212	0.42675949830445	0.856535620027831\\
0.09	0.102624350004627	0.441267427752584	0.859607954619615\\
0.09	0.109866204559871	0.455699271686218	0.86148476026662\\
0.09	0.117359475365564	0.470048619052377	0.862092534251541\\
0.09	0.125103014888515	0.484309059839721	0.861387758068369\\
0.09	0.133095461900593	0.498474193737904	0.859359100890818\\
0.09	0.14133524079124	0.512537638903683	0.856028283765736\\
0.09	0.149820561306021	0.526493040810599	0.851449530870182\\
0.09	0.158549418718625	0.540334081158348	0.845707614739508\\
0.09	0.167519594442213	0.554054486817265	0.838914584343155\\
0.09	0.176728657084455	0.567648038782868	0.831205342370839\\
0.09	0.186173963948925	0.581108581114919	0.822732305417527\\
0.09	0.195852662983903	0.594430029835237	0.813659432944308\\
0.09	0.205761695177907	0.607606381758231	0.804155944104874\\
0.09	0.215897797399558	0.620631723228144	0.794390053406969\\
0.09	0.226257505677663	0.633500238737009	0.784523046062798\\
0.09	0.236837158915675	0.646206219397571	0.774703982861589\\
0.09	0.247632903032949	0.658744071245709	0.765065275211017\\
0.09	0.258640695523527	0.671108323347402	0.755719307772961\\
0.09	0.269856310421543	0.683293635685828	0.746756214005911\\
0.09	0.28127534366066	0.695294806804924	0.738242834604296\\
0.09	0.292893218813452	0.707106781186547	0.730222816007834\\
0.09	0.304705193195075	0.71872465633934	0.722717741087768\\
0.09	0.316706364314172	0.730143689578457	0.715729131143268\\
0.09	0.328891676652598	0.741359304476472	0.709241120561857\\
0.09	0.341255928754291	0.752367096967051	0.703223584571265\\
0.09	0.353793780602429	0.763162841084325	0.697635496597418\\
0.09	0.366499761262991	0.773742494322337	0.692428303594618\\
0.09	0.379368276771856	0.784102202600442	0.687549132887069\\
0.09	0.392393618241769	0.794238304822092	0.682943679235329\\
0.09	0.405569970164763	0.804147337016097	0.678558662193045\\
0.09	0.418891418885081	0.813826036051075	0.674343787412871\\
0.09	0.432351961217133	0.823271342915545	0.670253187715553\\
0.09	0.445945513182735	0.832480405557786	0.666246357336158\\
0.09	0.459665918841652	0.841450581281375	0.662288623474186\\
0.09	0.473506959189401	0.850179438693979	0.658351221673191\\
0.09	0.487462361096317	0.85866475920876	0.654411055137886\\
0.09	0.501525806262096	0.866904538099407	0.65045022320647\\
0.09	0.515690940160279	0.874896985111485	0.646455401874959\\
0.09	0.529951380947623	0.882640524634437	0.642417151055155\\
0.09	0.544300728313781	0.890133795440129	0.638329210947584\\
0.09	0.558732572247415	0.897375649995372	0.634187835387496\\
0.09	0.57324050169555	0.904365153357279	0.629991194998376\\
0.09	0.587818113093873	0.911101581661729	0.625738868900923\\
0.09	0.602459018746568	0.917584420216504	0.62143143163314\\
0.09	0.617156855035314	0.923813361211857	0.617070132480553\\
0.09	0.631905290438127	0.92978830106243	0.612656657833839\\
0.09	0.6466980333399	0.935509337395467	0.608192963377089\\
0.09	0.661528839617671	0.940976765701173	0.603681161495103\\
0.09	0.676391519984904	0.94619107566195	0.599123449738607\\
0.09	0.691279947080357	0.951152947177946	0.594522067901591\\
0.09	0.706188062288412	0.955863246106974	0.589879273662391\\
0.09	0.721109882279076	0.960323019737426	0.585197329319937\\
0.09	0.736039505257226	0.964533492013186	0.580478494543973\\
0.09	0.75097111691199	0.968496058529893	0.575725022018313\\
0.09	0.765898996058536	0.972212281322139	0.570939154289562\\
0.09	0.780817519965826	0.975683883461272	0.566123121053176\\
0.09	0.795721169365275	0.97891274348356	0.561279136605558\\
0.09	0.810604533136476	0.981900889668376	0.556409397399199\\
0.09	0.825462312667457	0.984650494185936	0.551516079695733\\
0.09	0.840289325888138	0.987163867133889	0.546601337319641\\
0.09	0.855080510976838	0.989443450481768	0.541667299515142\\
0.09	0.86983092974082	0.991491811941915	0.536716068908599\\
0.09	0.884535770672926	0.993311638785082	0.53174971957814\\
0.09	0.899190351687411	0.994905731618374	0.526770295231795\\
0.09	0.913790122539012	0.996276998142666	0.521779807495103\\
0.09	0.928330666930242	0.997428446906011	0.516780234308681\\
0.09	0.942807704312699	0.998363181068902	0.511773518435872\\
0.09	0.95721709138901	0.999084392196567	0.50676156608017\\
0.09	0.971554823322704	0.999595354092743	0.501746245611874\\
0.09	0.985817034663989	0.999899416688637	0.496729386403004\\
0.09	1	1	0.491669087346906\\
0.105	0	0	0.710195569525497\\
0.105	0.000100583311362513	0.0141829653360114	0.71240460418299\\
0.105	0.000404645907256436	0.0284451766772965	0.714630690536905\\
0.105	0.000915607803432999	0.0427829086109896	0.716916312064106\\
0.105	0.00163681893109844	0.057192295687301	0.719289019360234\\
0.105	0.00257155309398959	0.0716693330697585	0.721780367061016\\
0.105	0.00372300185733414	0.0862098774609879	0.724425655492835\\
0.105	0.00509426838162598	0.100809648312589	0.727263413850304\\
0.105	0.00668836121491816	0.115464229327074	0.73033458997498\\
0.105	0.00850818805808555	0.13016907025918	0.733681422514709\\
0.105	0.0105565495182326	0.144919489023162	0.737345986077549\\
0.105	0.0128361328661109	0.159710674111862	0.741368418705642\\
0.105	0.0153495058140643	0.174537687332543	0.745784862958202\\
0.105	0.0180991103316243	0.189395466863524	0.750625176092311\\
0.105	0.0210872565164405	0.204278830634725	0.755910489868998\\
0.105	0.0243161165387281	0.219182480034174	0.761650724664453\\
0.105	0.0277877186778607	0.234101003941464	0.767842183867616\\
0.105	0.0315039414701067	0.24902888308801	0.77446537092041\\
0.105	0.0354665079868145	0.263960494742775	0.781483180780605\\
0.105	0.0396769802625738	0.278890117720924	0.788839618268073\\
0.105	0.0441367538930258	0.293811937711588	0.796459186325544\\
0.105	0.0488470528220538	0.308720052919643	0.804247066923455\\
0.105	0.0538089243380495	0.323608480015097	0.812090186161951\\
0.105	0.0590232342988274	0.338471160382329	0.819859213932694\\
0.105	0.064490662604533	0.3533019666601	0.827411499063639\\
0.105	0.0702116989375697	0.368094709561873	0.834594885813505\\
0.105	0.0761866387881432	0.382843144964686	0.841252300287178\\
0.105	0.0824155797834956	0.397540981253432	0.847226939729081\\
0.105	0.0888984183382709	0.412181886906127	0.852367847905041\\
0.105	0.0956348466427212	0.42675949830445	0.85653562002783\\
0.105	0.102624350004627	0.441267427752584	0.859607954619619\\
0.105	0.109866204559871	0.455699271686218	0.861484760266622\\
0.105	0.117359475365564	0.470048619052377	0.862092534251539\\
0.105	0.125103014888515	0.484309059839721	0.861387758068367\\
0.105	0.133095461900593	0.498474193737904	0.85935910089082\\
0.105	0.14133524079124	0.512537638903683	0.856028283765738\\
0.105	0.149820561306021	0.526493040810599	0.851449530870183\\
0.105	0.158549418718625	0.540334081158348	0.845707614739508\\
0.105	0.167519594442214	0.554054486817265	0.838914584343153\\
0.105	0.176728657084455	0.567648038782867	0.831205342370838\\
0.105	0.186173963948925	0.581108581114919	0.822732305417526\\
0.105	0.195852662983903	0.594430029835236	0.813659432944305\\
0.105	0.205761695177907	0.607606381758231	0.804155944104866\\
0.105	0.215897797399558	0.620631723228143	0.794390053406968\\
0.105	0.226257505677663	0.633500238737009	0.784523046062802\\
0.105	0.236837158915675	0.646206219397571	0.77470398286159\\
0.105	0.247632903032949	0.658744071245709	0.765065275211016\\
0.105	0.258640695523527	0.671108323347402	0.755719307772961\\
0.105	0.269856310421543	0.683293635685828	0.746756214005913\\
0.105	0.28127534366066	0.695294806804925	0.738242834604296\\
0.105	0.292893218813452	0.707106781186547	0.730222816007834\\
0.105	0.304705193195075	0.71872465633934	0.722717741087769\\
0.105	0.316706364314172	0.730143689578457	0.715729131143267\\
0.105	0.328891676652598	0.741359304476472	0.709241120561856\\
0.105	0.341255928754291	0.752367096967051	0.703223584571264\\
0.105	0.353793780602429	0.763162841084325	0.697635496597419\\
0.105	0.366499761262991	0.773742494322337	0.692428303594617\\
0.105	0.379368276771857	0.784102202600442	0.687549132887069\\
0.105	0.392393618241769	0.794238304822092	0.68294367923533\\
0.105	0.405569970164763	0.804147337016097	0.678558662193044\\
0.105	0.418891418885081	0.813826036051075	0.67434378741287\\
0.105	0.432351961217133	0.823271342915544	0.670253187715552\\
0.105	0.445945513182735	0.832480405557786	0.666246357336159\\
0.105	0.459665918841652	0.841450581281375	0.662288623474186\\
0.105	0.473506959189401	0.850179438693979	0.658351221673191\\
0.105	0.487462361096317	0.85866475920876	0.654411055137886\\
0.105	0.501525806262096	0.866904538099407	0.650450223206468\\
0.105	0.515690940160279	0.874896985111485	0.646455401874959\\
0.105	0.529951380947623	0.882640524634436	0.642417151055155\\
0.105	0.544300728313781	0.890133795440129	0.638329210947587\\
0.105	0.558732572247415	0.897375649995373	0.634187835387499\\
0.105	0.57324050169555	0.904365153357279	0.629991194998376\\
0.105	0.587818113093873	0.911101581661729	0.625738868900924\\
0.105	0.602459018746568	0.917584420216505	0.621431431633138\\
0.105	0.617156855035314	0.923813361211857	0.617070132480551\\
0.105	0.631905290438127	0.92978830106243	0.612656657833835\\
0.105	0.6466980333399	0.935509337395467	0.608192963377089\\
0.105	0.661528839617671	0.940976765701173	0.603681161495107\\
0.105	0.676391519984904	0.94619107566195	0.599123449738606\\
0.105	0.691279947080357	0.951152947177946	0.59452206790159\\
0.105	0.706188062288412	0.955863246106974	0.58987927366239\\
0.105	0.721109882279076	0.960323019737426	0.585197329319938\\
0.105	0.736039505257226	0.964533492013186	0.580478494543973\\
0.105	0.75097111691199	0.968496058529893	0.575725022018312\\
0.105	0.765898996058536	0.972212281322139	0.57093915428956\\
0.105	0.780817519965826	0.975683883461272	0.566123121053175\\
0.105	0.795721169365275	0.97891274348356	0.561279136605559\\
0.105	0.810604533136476	0.981900889668376	0.556409397399199\\
0.105	0.825462312667457	0.984650494185936	0.551516079695737\\
0.105	0.840289325888138	0.987163867133889	0.546601337319641\\
0.105	0.855080510976839	0.989443450481767	0.541667299515139\\
0.105	0.86983092974082	0.991491811941914	0.536716068908598\\
0.105	0.884535770672926	0.993311638785082	0.531749719578144\\
0.105	0.899190351687411	0.994905731618374	0.526770295231796\\
0.105	0.913790122539012	0.996276998142666	0.521779807495102\\
0.105	0.928330666930242	0.997428446906011	0.516780234308683\\
0.105	0.942807704312699	0.998363181068902	0.511773518435872\\
0.105	0.95721709138901	0.999084392196567	0.506761566080172\\
0.105	0.971554823322704	0.999595354092744	0.501746245611874\\
0.105	0.985817034663988	0.999899416688637	0.496729386402999\\
0.105	1	1	0.49166908734691\\
0.12	0	0	0.710195569525497\\
0.12	0.000100583311362513	0.0141829653360114	0.71240460418299\\
0.12	0.000404645907256436	0.0284451766772965	0.714630690536905\\
0.12	0.000915607803432999	0.0427829086109896	0.716916312064106\\
0.12	0.00163681893109844	0.057192295687301	0.719289019360234\\
0.12	0.00257155309398959	0.0716693330697584	0.721780367061016\\
0.12	0.00372300185733413	0.0862098774609879	0.724425655492835\\
0.12	0.00509426838162598	0.100809648312589	0.727263413850304\\
0.12	0.00668836121491816	0.115464229327074	0.73033458997498\\
0.12	0.00850818805808555	0.13016907025918	0.733681422514709\\
0.12	0.0105565495182326	0.144919489023162	0.737345986077549\\
0.12	0.0128361328661109	0.159710674111862	0.741368418705642\\
0.12	0.0153495058140643	0.174537687332543	0.745784862958201\\
0.12	0.0180991103316243	0.189395466863524	0.750625176092311\\
0.12	0.0210872565164405	0.204278830634725	0.755910489868999\\
0.12	0.0243161165387281	0.219182480034174	0.761650724664453\\
0.12	0.0277877186778607	0.234101003941464	0.767842183867616\\
0.12	0.0315039414701067	0.24902888308801	0.77446537092041\\
0.12	0.0354665079868146	0.263960494742775	0.781483180780607\\
0.12	0.0396769802625738	0.278890117720924	0.788839618268073\\
0.12	0.0441367538930258	0.293811937711588	0.796459186325544\\
0.12	0.0488470528220538	0.308720052919643	0.804247066923454\\
0.12	0.0538089243380495	0.323608480015097	0.81209018616195\\
0.12	0.0590232342988274	0.338471160382329	0.819859213932694\\
0.12	0.064490662604533	0.3533019666601	0.82741149906364\\
0.12	0.0702116989375697	0.368094709561873	0.834594885813505\\
0.12	0.0761866387881432	0.382843144964686	0.841252300287178\\
0.12	0.0824155797834956	0.397540981253432	0.847226939729081\\
0.12	0.0888984183382709	0.412181886906127	0.852367847905041\\
0.12	0.0956348466427212	0.42675949830445	0.85653562002783\\
0.12	0.102624350004627	0.441267427752585	0.85960795461962\\
0.12	0.109866204559871	0.455699271686218	0.861484760266624\\
0.12	0.117359475365564	0.470048619052377	0.86209253425154\\
0.12	0.125103014888515	0.484309059839721	0.861387758068369\\
0.12	0.133095461900593	0.498474193737904	0.859359100890822\\
0.12	0.14133524079124	0.512537638903683	0.856028283765735\\
0.12	0.149820561306021	0.526493040810599	0.851449530870182\\
0.12	0.158549418718625	0.540334081158348	0.845707614739502\\
0.12	0.167519594442213	0.554054486817265	0.838914584343153\\
0.12	0.176728657084455	0.567648038782868	0.831205342370836\\
0.12	0.186173963948925	0.581108581114919	0.822732305417524\\
0.12	0.195852662983903	0.594430029835236	0.813659432944309\\
0.12	0.205761695177907	0.607606381758231	0.804155944104868\\
0.12	0.215897797399558	0.620631723228144	0.794390053406972\\
0.12	0.226257505677663	0.633500238737009	0.7845230460628\\
0.12	0.236837158915675	0.646206219397571	0.774703982861591\\
0.12	0.247632903032949	0.658744071245709	0.765065275211016\\
0.12	0.258640695523527	0.671108323347402	0.755719307772961\\
0.12	0.269856310421543	0.683293635685828	0.746756214005913\\
0.12	0.28127534366066	0.695294806804925	0.738242834604296\\
0.12	0.292893218813452	0.707106781186547	0.730222816007832\\
0.12	0.304705193195075	0.71872465633934	0.722717741087768\\
0.12	0.316706364314172	0.730143689578457	0.71572913114327\\
0.12	0.328891676652598	0.741359304476472	0.709241120561856\\
0.12	0.341255928754291	0.752367096967051	0.703223584571264\\
0.12	0.353793780602429	0.763162841084325	0.697635496597418\\
0.12	0.366499761262991	0.773742494322337	0.692428303594619\\
0.12	0.379368276771856	0.784102202600442	0.68754913288707\\
0.12	0.392393618241769	0.794238304822092	0.682943679235329\\
0.12	0.405569970164763	0.804147337016097	0.678558662193044\\
0.12	0.418891418885081	0.813826036051075	0.674343787412869\\
0.12	0.432351961217132	0.823271342915544	0.670253187715554\\
0.12	0.445945513182735	0.832480405557787	0.666246357336159\\
0.12	0.459665918841652	0.841450581281375	0.662288623474185\\
0.12	0.473506959189401	0.850179438693979	0.658351221673192\\
0.12	0.487462361096317	0.85866475920876	0.654411055137886\\
0.12	0.501525806262096	0.866904538099407	0.650450223206468\\
0.12	0.515690940160279	0.874896985111485	0.646455401874958\\
0.12	0.529951380947623	0.882640524634436	0.642417151055155\\
0.12	0.544300728313782	0.890133795440129	0.638329210947587\\
0.12	0.558732572247415	0.897375649995373	0.634187835387497\\
0.12	0.57324050169555	0.904365153357279	0.629991194998376\\
0.12	0.587818113093873	0.911101581661729	0.625738868900926\\
0.12	0.602459018746568	0.917584420216505	0.621431431633142\\
0.12	0.617156855035314	0.923813361211857	0.617070132480551\\
0.12	0.631905290438127	0.92978830106243	0.612656657833836\\
0.12	0.6466980333399	0.935509337395467	0.608192963377087\\
0.12	0.661528839617671	0.940976765701173	0.603681161495106\\
0.12	0.676391519984904	0.946191075661951	0.599123449738607\\
0.12	0.691279947080357	0.951152947177946	0.594522067901589\\
0.12	0.706188062288412	0.955863246106974	0.589879273662389\\
0.12	0.721109882279076	0.960323019737426	0.585197329319938\\
0.12	0.736039505257226	0.964533492013186	0.580478494543976\\
0.12	0.75097111691199	0.968496058529893	0.575725022018313\\
0.12	0.765898996058536	0.972212281322139	0.570939154289562\\
0.12	0.780817519965826	0.975683883461272	0.566123121053172\\
0.12	0.795721169365275	0.978912743483559	0.561279136605557\\
0.12	0.810604533136476	0.981900889668376	0.556409397399198\\
0.12	0.825462312667457	0.984650494185936	0.551516079695733\\
0.12	0.840289325888138	0.987163867133889	0.546601337319643\\
0.12	0.855080510976838	0.989443450481767	0.541667299515141\\
0.12	0.86983092974082	0.991491811941914	0.536716068908599\\
0.12	0.884535770672926	0.993311638785082	0.531749719578142\\
0.12	0.899190351687411	0.994905731618374	0.526770295231794\\
0.12	0.913790122539012	0.996276998142666	0.521779807495104\\
0.12	0.928330666930242	0.997428446906011	0.516780234308683\\
0.12	0.942807704312699	0.998363181068901	0.511773518435872\\
0.12	0.95721709138901	0.999084392196567	0.50676156608017\\
0.12	0.971554823322704	0.999595354092743	0.501746245611877\\
0.12	0.985817034663989	0.999899416688638	0.496729386403004\\
0.12	1	1	0.491669087346897\\
0.135	0	0	0.710195569525497\\
0.135	0.000100583311362513	0.0141829653360114	0.71240460418299\\
0.135	0.000404645907256436	0.0284451766772965	0.714630690536905\\
0.135	0.000915607803432999	0.0427829086109896	0.716916312064106\\
0.135	0.00163681893109844	0.057192295687301	0.719289019360234\\
0.135	0.00257155309398959	0.0716693330697584	0.721780367061016\\
0.135	0.00372300185733414	0.0862098774609879	0.724425655492835\\
0.135	0.00509426838162598	0.100809648312589	0.727263413850304\\
0.135	0.00668836121491815	0.115464229327074	0.73033458997498\\
0.135	0.00850818805808555	0.13016907025918	0.733681422514709\\
0.135	0.0105565495182326	0.144919489023162	0.737345986077549\\
0.135	0.0128361328661109	0.159710674111862	0.741368418705642\\
0.135	0.0153495058140643	0.174537687332543	0.745784862958202\\
0.135	0.0180991103316243	0.189395466863524	0.750625176092311\\
0.135	0.0210872565164405	0.204278830634725	0.755910489868999\\
0.135	0.0243161165387281	0.219182480034174	0.761650724664453\\
0.135	0.0277877186778607	0.234101003941464	0.767842183867616\\
0.135	0.0315039414701067	0.24902888308801	0.77446537092041\\
0.135	0.0354665079868145	0.263960494742775	0.781483180780606\\
0.135	0.0396769802625738	0.278890117720924	0.788839618268073\\
0.135	0.0441367538930258	0.293811937711588	0.796459186325543\\
0.135	0.0488470528220538	0.308720052919643	0.804247066923455\\
0.135	0.0538089243380495	0.323608480015097	0.81209018616195\\
0.135	0.0590232342988274	0.338471160382329	0.819859213932694\\
0.135	0.064490662604533	0.3533019666601	0.827411499063639\\
0.135	0.0702116989375697	0.368094709561873	0.834594885813506\\
0.135	0.0761866387881432	0.382843144964686	0.841252300287181\\
0.135	0.0824155797834956	0.397540981253432	0.847226939729081\\
0.135	0.0888984183382709	0.412181886906127	0.852367847905041\\
0.135	0.0956348466427212	0.42675949830445	0.856535620027829\\
0.135	0.102624350004627	0.441267427752585	0.859607954619617\\
0.135	0.109866204559871	0.455699271686218	0.861484760266621\\
0.135	0.117359475365564	0.470048619052377	0.86209253425154\\
0.135	0.125103014888515	0.484309059839721	0.861387758068371\\
0.135	0.133095461900593	0.498474193737904	0.85935910089082\\
0.135	0.14133524079124	0.512537638903683	0.856028283765736\\
0.135	0.149820561306021	0.526493040810599	0.851449530870182\\
0.135	0.158549418718625	0.540334081158348	0.845707614739506\\
0.135	0.167519594442213	0.554054486817266	0.838914584343151\\
0.135	0.176728657084455	0.567648038782867	0.831205342370833\\
0.135	0.186173963948925	0.581108581114919	0.822732305417527\\
0.135	0.195852662983903	0.594430029835237	0.813659432944307\\
0.135	0.205761695177907	0.60760638175823	0.80415594410487\\
0.135	0.215897797399558	0.620631723228143	0.794390053406968\\
0.135	0.226257505677663	0.633500238737009	0.784523046062802\\
0.135	0.236837158915675	0.646206219397571	0.774703982861592\\
0.135	0.247632903032949	0.658744071245709	0.765065275211015\\
0.135	0.258640695523527	0.671108323347402	0.755719307772962\\
0.135	0.269856310421543	0.683293635685828	0.746756214005914\\
0.135	0.28127534366066	0.695294806804925	0.738242834604294\\
0.135	0.292893218813452	0.707106781186547	0.730222816007833\\
0.135	0.304705193195075	0.71872465633934	0.72271774108777\\
0.135	0.316706364314172	0.730143689578457	0.715729131143266\\
0.135	0.328891676652598	0.741359304476472	0.709241120561855\\
0.135	0.341255928754291	0.752367096967051	0.703223584571266\\
0.135	0.353793780602429	0.763162841084325	0.697635496597419\\
0.135	0.366499761262991	0.773742494322337	0.692428303594617\\
0.135	0.379368276771857	0.784102202600442	0.687549132887069\\
0.135	0.392393618241769	0.794238304822092	0.682943679235329\\
0.135	0.405569970164763	0.804147337016097	0.678558662193045\\
0.135	0.418891418885081	0.813826036051075	0.67434378741287\\
0.135	0.432351961217132	0.823271342915544	0.670253187715551\\
0.135	0.445945513182735	0.832480405557786	0.66624635733616\\
0.135	0.459665918841652	0.841450581281375	0.662288623474186\\
0.135	0.473506959189401	0.850179438693979	0.658351221673192\\
0.135	0.487462361096317	0.85866475920876	0.654411055137887\\
0.135	0.501525806262096	0.866904538099407	0.650450223206467\\
0.135	0.515690940160279	0.874896985111485	0.646455401874959\\
0.135	0.529951380947623	0.882640524634436	0.642417151055155\\
0.135	0.544300728313782	0.890133795440129	0.638329210947587\\
0.135	0.558732572247416	0.897375649995373	0.634187835387498\\
0.135	0.57324050169555	0.904365153357279	0.629991194998375\\
0.135	0.587818113093873	0.911101581661729	0.625738868900924\\
0.135	0.602459018746568	0.917584420216504	0.621431431633141\\
0.135	0.617156855035314	0.923813361211857	0.617070132480554\\
0.135	0.631905290438127	0.92978830106243	0.612656657833837\\
0.135	0.6466980333399	0.935509337395467	0.608192963377087\\
0.135	0.661528839617671	0.940976765701173	0.603681161495105\\
0.135	0.676391519984903	0.94619107566195	0.599123449738607\\
0.135	0.691279947080357	0.951152947177946	0.594522067901589\\
0.135	0.706188062288412	0.955863246106974	0.589879273662389\\
0.135	0.721109882279076	0.960323019737426	0.585197329319937\\
0.135	0.736039505257226	0.964533492013186	0.580478494543976\\
0.135	0.75097111691199	0.968496058529893	0.575725022018313\\
0.135	0.765898996058536	0.972212281322139	0.570939154289561\\
0.135	0.780817519965826	0.975683883461272	0.566123121053175\\
0.135	0.795721169365275	0.978912743483559	0.561279136605558\\
0.135	0.810604533136476	0.981900889668376	0.5564093973992\\
0.135	0.825462312667457	0.984650494185936	0.551516079695734\\
0.135	0.840289325888138	0.987163867133889	0.546601337319642\\
0.135	0.855080510976838	0.989443450481767	0.541667299515139\\
0.135	0.86983092974082	0.991491811941914	0.536716068908595\\
0.135	0.884535770672926	0.993311638785082	0.531749719578142\\
0.135	0.899190351687411	0.994905731618374	0.526770295231797\\
0.135	0.913790122539012	0.996276998142666	0.521779807495104\\
0.135	0.928330666930242	0.997428446906011	0.516780234308684\\
0.135	0.942807704312699	0.998363181068902	0.511773518435872\\
0.135	0.95721709138901	0.999084392196567	0.506761566080168\\
0.135	0.971554823322703	0.999595354092743	0.501746245611874\\
0.135	0.985817034663989	0.999899416688637	0.496729386403004\\
0.135	1	1	0.491669087346906\\
0.15	0	0	0.710195569525497\\
0.15	0.000100583311362513	0.0141829653360114	0.71240460418299\\
0.15	0.000404645907256436	0.0284451766772965	0.714630690536905\\
0.15	0.000915607803432999	0.0427829086109896	0.716916312064106\\
0.15	0.00163681893109844	0.057192295687301	0.719289019360234\\
0.15	0.00257155309398959	0.0716693330697584	0.721780367061016\\
0.15	0.00372300185733413	0.0862098774609879	0.724425655492835\\
0.15	0.00509426838162598	0.100809648312589	0.727263413850304\\
0.15	0.00668836121491815	0.115464229327074	0.73033458997498\\
0.15	0.00850818805808555	0.13016907025918	0.733681422514709\\
0.15	0.0105565495182326	0.144919489023162	0.737345986077549\\
0.15	0.0128361328661109	0.159710674111862	0.741368418705642\\
0.15	0.0153495058140643	0.174537687332543	0.745784862958202\\
0.15	0.0180991103316243	0.189395466863524	0.750625176092311\\
0.15	0.0210872565164405	0.204278830634725	0.755910489868999\\
0.15	0.0243161165387281	0.219182480034174	0.761650724664453\\
0.15	0.0277877186778607	0.234101003941464	0.767842183867616\\
0.15	0.0315039414701067	0.24902888308801	0.77446537092041\\
0.15	0.0354665079868145	0.263960494742775	0.781483180780606\\
0.15	0.0396769802625738	0.278890117720924	0.788839618268073\\
0.15	0.0441367538930258	0.293811937711588	0.796459186325544\\
0.15	0.0488470528220538	0.308720052919643	0.804247066923454\\
0.15	0.0538089243380495	0.323608480015096	0.812090186161949\\
0.15	0.0590232342988274	0.338471160382329	0.819859213932694\\
0.15	0.064490662604533	0.3533019666601	0.827411499063638\\
0.15	0.0702116989375697	0.368094709561873	0.834594885813504\\
0.15	0.0761866387881432	0.382843144964686	0.841252300287181\\
0.15	0.0824155797834956	0.397540981253432	0.847226939729083\\
0.15	0.0888984183382709	0.412181886906127	0.852367847905041\\
0.15	0.0956348466427212	0.42675949830445	0.856535620027826\\
0.15	0.102624350004627	0.441267427752584	0.859607954619617\\
0.15	0.109866204559871	0.455699271686218	0.86148476026662\\
0.15	0.117359475365564	0.470048619052377	0.86209253425154\\
0.15	0.125103014888515	0.484309059839721	0.86138775806837\\
0.15	0.133095461900593	0.498474193737904	0.85935910089082\\
0.15	0.14133524079124	0.512537638903683	0.856028283765737\\
0.15	0.149820561306021	0.526493040810599	0.851449530870184\\
0.15	0.158549418718625	0.540334081158348	0.845707614739508\\
0.15	0.167519594442214	0.554054486817265	0.838914584343148\\
0.15	0.176728657084455	0.567648038782867	0.831205342370835\\
0.15	0.186173963948925	0.581108581114919	0.822732305417528\\
0.15	0.195852662983903	0.594430029835236	0.813659432944304\\
0.15	0.205761695177907	0.607606381758231	0.804155944104874\\
0.15	0.215897797399558	0.620631723228143	0.79439005340697\\
0.15	0.226257505677663	0.633500238737009	0.784523046062803\\
0.15	0.236837158915675	0.646206219397571	0.774703982861589\\
0.15	0.247632903032949	0.658744071245709	0.765065275211015\\
0.15	0.258640695523527	0.671108323347402	0.755719307772963\\
0.15	0.269856310421543	0.683293635685828	0.746756214005913\\
0.15	0.28127534366066	0.695294806804925	0.738242834604293\\
0.15	0.292893218813452	0.707106781186547	0.730222816007832\\
0.15	0.304705193195075	0.71872465633934	0.722717741087768\\
0.15	0.316706364314172	0.730143689578457	0.715729131143267\\
0.15	0.328891676652598	0.741359304476472	0.709241120561857\\
0.15	0.341255928754291	0.752367096967051	0.703223584571265\\
0.15	0.353793780602429	0.763162841084325	0.697635496597417\\
0.15	0.366499761262991	0.773742494322337	0.69242830359462\\
0.15	0.379368276771857	0.784102202600443	0.68754913288707\\
0.15	0.392393618241769	0.794238304822092	0.682943679235328\\
0.15	0.405569970164763	0.804147337016097	0.678558662193044\\
0.15	0.418891418885081	0.813826036051075	0.674343787412871\\
0.15	0.432351961217132	0.823271342915544	0.670253187715551\\
0.15	0.445945513182735	0.832480405557786	0.666246357336158\\
0.15	0.459665918841652	0.841450581281375	0.662288623474186\\
0.15	0.473506959189401	0.850179438693979	0.658351221673191\\
0.15	0.487462361096317	0.85866475920876	0.654411055137887\\
0.15	0.501525806262096	0.866904538099407	0.650450223206468\\
0.15	0.515690940160279	0.874896985111485	0.646455401874958\\
0.15	0.529951380947623	0.882640524634436	0.642417151055155\\
0.15	0.544300728313782	0.890133795440129	0.638329210947586\\
0.15	0.558732572247415	0.897375649995373	0.634187835387498\\
0.15	0.57324050169555	0.904365153357279	0.629991194998374\\
0.15	0.587818113093873	0.911101581661729	0.625738868900923\\
0.15	0.602459018746568	0.917584420216504	0.621431431633141\\
0.15	0.617156855035314	0.923813361211857	0.617070132480553\\
0.15	0.631905290438127	0.92978830106243	0.612656657833839\\
0.15	0.6466980333399	0.935509337395467	0.608192963377089\\
0.15	0.661528839617671	0.940976765701173	0.603681161495105\\
0.15	0.676391519984904	0.946191075661951	0.599123449738607\\
0.15	0.691279947080357	0.951152947177946	0.594522067901588\\
0.15	0.706188062288412	0.955863246106974	0.589879273662389\\
0.15	0.721109882279076	0.960323019737426	0.585197329319936\\
0.15	0.736039505257225	0.964533492013185	0.580478494543975\\
0.15	0.75097111691199	0.968496058529893	0.575725022018313\\
0.15	0.765898996058536	0.972212281322139	0.570939154289559\\
0.15	0.780817519965826	0.975683883461272	0.566123121053174\\
0.15	0.795721169365275	0.978912743483559	0.56127913660556\\
0.15	0.810604533136476	0.981900889668376	0.556409397399201\\
0.15	0.825462312667457	0.984650494185936	0.551516079695734\\
0.15	0.840289325888138	0.987163867133889	0.546601337319641\\
0.15	0.855080510976839	0.989443450481767	0.541667299515139\\
0.15	0.86983092974082	0.991491811941914	0.536716068908597\\
0.15	0.884535770672926	0.993311638785082	0.531749719578141\\
0.15	0.899190351687411	0.994905731618374	0.526770295231794\\
0.15	0.913790122539012	0.996276998142666	0.521779807495102\\
0.15	0.928330666930241	0.99742844690601	0.516780234308683\\
0.15	0.942807704312699	0.998363181068901	0.511773518435874\\
0.15	0.95721709138901	0.999084392196567	0.50676156608017\\
0.15	0.971554823322703	0.999595354092743	0.501746245611875\\
0.15	0.985817034663989	0.999899416688637	0.496729386403004\\
0.15	1	1	0.491669087346906\\
0.165	0	0	0.710195569525497\\
0.165	0.000100583311362513	0.0141829653360114	0.71240460418299\\
0.165	0.000404645907256436	0.0284451766772965	0.714630690536905\\
0.165	0.000915607803432999	0.0427829086109896	0.716916312064106\\
0.165	0.00163681893109844	0.057192295687301	0.719289019360234\\
0.165	0.00257155309398959	0.0716693330697584	0.721780367061016\\
0.165	0.00372300185733414	0.0862098774609879	0.724425655492835\\
0.165	0.00509426838162598	0.100809648312589	0.727263413850304\\
0.165	0.00668836121491815	0.115464229327074	0.73033458997498\\
0.165	0.00850818805808555	0.13016907025918	0.733681422514709\\
0.165	0.0105565495182326	0.144919489023162	0.737345986077549\\
0.165	0.0128361328661109	0.159710674111862	0.741368418705642\\
0.165	0.0153495058140643	0.174537687332543	0.745784862958201\\
0.165	0.0180991103316243	0.189395466863524	0.750625176092311\\
0.165	0.0210872565164405	0.204278830634725	0.755910489868999\\
0.165	0.0243161165387281	0.219182480034174	0.761650724664453\\
0.165	0.0277877186778607	0.234101003941464	0.767842183867616\\
0.165	0.0315039414701067	0.24902888308801	0.77446537092041\\
0.165	0.0354665079868145	0.263960494742775	0.781483180780606\\
0.165	0.0396769802625738	0.278890117720924	0.788839618268074\\
0.165	0.0441367538930258	0.293811937711588	0.796459186325544\\
0.165	0.0488470528220537	0.308720052919643	0.804247066923455\\
0.165	0.0538089243380495	0.323608480015096	0.81209018616195\\
0.165	0.0590232342988274	0.338471160382329	0.819859213932694\\
0.165	0.0644906626045329	0.3533019666601	0.827411499063639\\
0.165	0.0702116989375697	0.368094709561873	0.834594885813502\\
0.165	0.0761866387881432	0.382843144964686	0.841252300287178\\
0.165	0.0824155797834956	0.397540981253432	0.84722693972908\\
0.165	0.0888984183382709	0.412181886906127	0.852367847905042\\
0.165	0.0956348466427212	0.42675949830445	0.856535620027829\\
0.165	0.102624350004627	0.441267427752584	0.859607954619617\\
0.165	0.109866204559871	0.455699271686218	0.861484760266622\\
0.165	0.117359475365564	0.470048619052377	0.862092534251542\\
0.165	0.125103014888515	0.484309059839721	0.86138775806837\\
0.165	0.133095461900593	0.498474193737904	0.859359100890821\\
0.165	0.14133524079124	0.512537638903683	0.856028283765739\\
0.165	0.149820561306021	0.526493040810599	0.851449530870183\\
0.165	0.158549418718625	0.540334081158348	0.845707614739504\\
0.165	0.167519594442214	0.554054486817265	0.83891458434315\\
0.165	0.176728657084455	0.567648038782867	0.831205342370837\\
0.165	0.186173963948925	0.581108581114919	0.822732305417529\\
0.165	0.195852662983903	0.594430029835236	0.813659432944307\\
0.165	0.205761695177907	0.607606381758231	0.804155944104868\\
0.165	0.215897797399558	0.620631723228143	0.794390053406969\\
0.165	0.226257505677663	0.633500238737009	0.7845230460628\\
0.165	0.236837158915675	0.646206219397571	0.774703982861588\\
0.165	0.247632903032949	0.658744071245709	0.765065275211016\\
0.165	0.258640695523527	0.671108323347402	0.755719307772962\\
0.165	0.269856310421543	0.683293635685828	0.746756214005912\\
0.165	0.28127534366066	0.695294806804925	0.738242834604294\\
0.165	0.292893218813452	0.707106781186547	0.730222816007833\\
0.165	0.304705193195075	0.71872465633934	0.722717741087769\\
0.165	0.316706364314172	0.730143689578457	0.715729131143269\\
0.165	0.328891676652598	0.741359304476472	0.709241120561857\\
0.165	0.341255928754291	0.752367096967051	0.703223584571264\\
0.165	0.353793780602429	0.763162841084325	0.697635496597418\\
0.165	0.366499761262991	0.773742494322337	0.692428303594618\\
0.165	0.379368276771857	0.784102202600442	0.68754913288707\\
0.165	0.392393618241769	0.794238304822092	0.682943679235329\\
0.165	0.405569970164763	0.804147337016097	0.678558662193044\\
0.165	0.418891418885081	0.813826036051075	0.67434378741287\\
0.165	0.432351961217132	0.823271342915544	0.670253187715552\\
0.165	0.445945513182735	0.832480405557786	0.666246357336159\\
0.165	0.459665918841652	0.841450581281375	0.662288623474186\\
0.165	0.473506959189401	0.850179438693979	0.658351221673191\\
0.165	0.487462361096317	0.85866475920876	0.654411055137886\\
0.165	0.501525806262096	0.866904538099407	0.650450223206468\\
0.165	0.515690940160279	0.874896985111485	0.646455401874957\\
0.165	0.529951380947623	0.882640524634436	0.642417151055155\\
0.165	0.544300728313782	0.890133795440129	0.638329210947587\\
0.165	0.558732572247415	0.897375649995373	0.634187835387498\\
0.165	0.57324050169555	0.904365153357279	0.629991194998377\\
0.165	0.587818113093873	0.911101581661729	0.625738868900923\\
0.165	0.602459018746568	0.917584420216504	0.621431431633137\\
0.165	0.617156855035314	0.923813361211857	0.617070132480553\\
0.165	0.631905290438127	0.92978830106243	0.612656657833838\\
0.165	0.6466980333399	0.935509337395467	0.608192963377086\\
0.165	0.661528839617671	0.940976765701173	0.603681161495104\\
0.165	0.676391519984904	0.94619107566195	0.599123449738608\\
0.165	0.691279947080357	0.951152947177946	0.59452206790159\\
0.165	0.706188062288412	0.955863246106974	0.589879273662389\\
0.165	0.721109882279076	0.960323019737426	0.585197329319937\\
0.165	0.736039505257226	0.964533492013186	0.580478494543977\\
0.165	0.75097111691199	0.968496058529893	0.575725022018312\\
0.165	0.765898996058536	0.972212281322139	0.570939154289558\\
0.165	0.780817519965825	0.975683883461272	0.566123121053175\\
0.165	0.795721169365275	0.97891274348356	0.561279136605559\\
0.165	0.810604533136476	0.981900889668376	0.556409397399199\\
0.165	0.825462312667457	0.984650494185936	0.551516079695734\\
0.165	0.840289325888138	0.987163867133889	0.546601337319641\\
0.165	0.855080510976839	0.989443450481767	0.541667299515141\\
0.165	0.86983092974082	0.991491811941915	0.536716068908599\\
0.165	0.884535770672926	0.993311638785082	0.531749719578141\\
0.165	0.899190351687411	0.994905731618374	0.526770295231793\\
0.165	0.913790122539012	0.996276998142666	0.521779807495103\\
0.165	0.928330666930242	0.997428446906011	0.516780234308683\\
0.165	0.942807704312699	0.998363181068901	0.511773518435872\\
0.165	0.95721709138901	0.999084392196567	0.506761566080168\\
0.165	0.971554823322703	0.999595354092743	0.501746245611874\\
0.165	0.985817034663989	0.999899416688637	0.496729386403005\\
0.165	1	1	0.491669087346904\\
0.18	0	0	0.710195569525497\\
0.18	0.000100583311362513	0.0141829653360114	0.71240460418299\\
0.18	0.000404645907256436	0.0284451766772965	0.714630690536905\\
0.18	0.000915607803432999	0.0427829086109896	0.716916312064106\\
0.18	0.00163681893109844	0.057192295687301	0.719289019360234\\
0.18	0.00257155309398959	0.0716693330697584	0.721780367061016\\
0.18	0.00372300185733414	0.0862098774609879	0.724425655492835\\
0.18	0.00509426838162598	0.100809648312589	0.727263413850304\\
0.18	0.00668836121491816	0.115464229327074	0.73033458997498\\
0.18	0.00850818805808554	0.13016907025918	0.733681422514709\\
0.18	0.0105565495182326	0.144919489023162	0.737345986077549\\
0.18	0.0128361328661109	0.159710674111862	0.741368418705642\\
0.18	0.0153495058140643	0.174537687332543	0.745784862958201\\
0.18	0.0180991103316243	0.189395466863524	0.750625176092311\\
0.18	0.0210872565164405	0.204278830634725	0.755910489868999\\
0.18	0.0243161165387281	0.219182480034174	0.761650724664453\\
0.18	0.0277877186778607	0.234101003941464	0.767842183867616\\
0.18	0.0315039414701067	0.24902888308801	0.77446537092041\\
0.18	0.0354665079868145	0.263960494742775	0.781483180780606\\
0.18	0.0396769802625738	0.278890117720924	0.788839618268073\\
0.18	0.0441367538930258	0.293811937711588	0.796459186325544\\
0.18	0.0488470528220538	0.308720052919643	0.804247066923454\\
0.18	0.0538089243380495	0.323608480015096	0.812090186161951\\
0.18	0.0590232342988274	0.338471160382329	0.819859213932694\\
0.18	0.064490662604533	0.3533019666601	0.82741149906364\\
0.18	0.0702116989375697	0.368094709561873	0.834594885813504\\
0.18	0.0761866387881432	0.382843144964686	0.841252300287178\\
0.18	0.0824155797834956	0.397540981253432	0.847226939729079\\
0.18	0.0888984183382709	0.412181886906127	0.852367847905041\\
0.18	0.0956348466427212	0.42675949830445	0.856535620027833\\
0.18	0.102624350004627	0.441267427752585	0.859607954619617\\
0.18	0.109866204559871	0.455699271686218	0.86148476026662\\
0.18	0.117359475365564	0.470048619052377	0.862092534251539\\
0.18	0.125103014888515	0.484309059839721	0.861387758068369\\
0.18	0.133095461900593	0.498474193737904	0.85935910089082\\
0.18	0.14133524079124	0.512537638903683	0.856028283765738\\
0.18	0.149820561306021	0.526493040810599	0.851449530870181\\
0.18	0.158549418718625	0.540334081158348	0.845707614739504\\
0.18	0.167519594442213	0.554054486817265	0.838914584343152\\
0.18	0.176728657084455	0.567648038782867	0.831205342370837\\
0.18	0.186173963948925	0.581108581114919	0.822732305417525\\
0.18	0.195852662983903	0.594430029835236	0.813659432944305\\
0.18	0.205761695177907	0.607606381758231	0.804155944104866\\
0.18	0.215897797399558	0.620631723228143	0.794390053406971\\
0.18	0.226257505677663	0.633500238737009	0.784523046062801\\
0.18	0.236837158915675	0.646206219397571	0.774703982861592\\
0.18	0.247632903032949	0.658744071245709	0.765065275211015\\
0.18	0.258640695523527	0.671108323347402	0.755719307772961\\
0.18	0.269856310421543	0.683293635685828	0.746756214005913\\
0.18	0.28127534366066	0.695294806804925	0.738242834604295\\
0.18	0.292893218813452	0.707106781186547	0.730222816007833\\
0.18	0.304705193195075	0.71872465633934	0.722717741087768\\
0.18	0.316706364314172	0.730143689578457	0.715729131143269\\
0.18	0.328891676652598	0.741359304476472	0.709241120561855\\
0.18	0.341255928754291	0.752367096967051	0.703223584571265\\
0.18	0.353793780602429	0.763162841084325	0.697635496597419\\
0.18	0.366499761262991	0.773742494322337	0.692428303594618\\
0.18	0.379368276771856	0.784102202600442	0.687549132887069\\
0.18	0.392393618241769	0.794238304822092	0.682943679235329\\
0.18	0.405569970164763	0.804147337016097	0.678558662193045\\
0.18	0.418891418885081	0.813826036051075	0.67434378741287\\
0.18	0.432351961217132	0.823271342915544	0.670253187715551\\
0.18	0.445945513182735	0.832480405557786	0.666246357336159\\
0.18	0.459665918841652	0.841450581281375	0.662288623474188\\
0.18	0.473506959189401	0.850179438693979	0.65835122167319\\
0.18	0.487462361096317	0.85866475920876	0.654411055137885\\
0.18	0.501525806262096	0.866904538099407	0.65045022320647\\
0.18	0.515690940160279	0.874896985111485	0.646455401874958\\
0.18	0.529951380947623	0.882640524634436	0.642417151055154\\
0.18	0.544300728313781	0.890133795440129	0.638329210947587\\
0.18	0.558732572247415	0.897375649995373	0.634187835387499\\
0.18	0.57324050169555	0.904365153357279	0.629991194998376\\
0.18	0.587818113093873	0.911101581661729	0.625738868900923\\
0.18	0.602459018746568	0.917584420216504	0.62143143163314\\
0.18	0.617156855035314	0.923813361211857	0.617070132480553\\
0.18	0.631905290438127	0.92978830106243	0.612656657833839\\
0.18	0.6466980333399	0.935509337395467	0.608192963377089\\
0.18	0.661528839617671	0.940976765701173	0.603681161495104\\
0.18	0.676391519984904	0.946191075661951	0.599123449738606\\
0.18	0.691279947080358	0.951152947177946	0.59452206790159\\
0.18	0.706188062288412	0.955863246106974	0.58987927366239\\
0.18	0.721109882279076	0.960323019737426	0.585197329319938\\
0.18	0.736039505257226	0.964533492013186	0.580478494543976\\
0.18	0.75097111691199	0.968496058529893	0.575725022018311\\
0.18	0.765898996058536	0.972212281322139	0.57093915428956\\
0.18	0.780817519965826	0.975683883461272	0.566123121053174\\
0.18	0.795721169365275	0.978912743483559	0.561279136605559\\
0.18	0.810604533136476	0.981900889668376	0.556409397399201\\
0.18	0.825462312667457	0.984650494185936	0.551516079695733\\
0.18	0.840289325888138	0.987163867133889	0.546601337319641\\
0.18	0.855080510976838	0.989443450481767	0.541667299515141\\
0.18	0.86983092974082	0.991491811941914	0.5367160689086\\
0.18	0.884535770672926	0.993311638785082	0.531749719578141\\
0.18	0.899190351687411	0.994905731618374	0.526770295231793\\
0.18	0.913790122539012	0.996276998142666	0.521779807495104\\
0.18	0.928330666930242	0.997428446906011	0.516780234308683\\
0.18	0.942807704312699	0.998363181068902	0.511773518435872\\
0.18	0.95721709138901	0.999084392196567	0.506761566080171\\
0.18	0.971554823322703	0.999595354092744	0.501746245611876\\
0.18	0.985817034663989	0.999899416688638	0.496729386403002\\
0.18	1	1	0.491669087346898\\
0.195	0	0	0.710195569525497\\
0.195	0.000100583311362513	0.0141829653360114	0.71240460418299\\
0.195	0.000404645907256436	0.0284451766772965	0.714630690536905\\
0.195	0.000915607803432999	0.0427829086109896	0.716916312064106\\
0.195	0.00163681893109843	0.057192295687301	0.719289019360234\\
0.195	0.00257155309398959	0.0716693330697584	0.721780367061016\\
0.195	0.00372300185733413	0.0862098774609879	0.724425655492835\\
0.195	0.00509426838162598	0.100809648312589	0.727263413850304\\
0.195	0.00668836121491815	0.115464229327074	0.73033458997498\\
0.195	0.00850818805808554	0.13016907025918	0.733681422514709\\
0.195	0.0105565495182326	0.144919489023162	0.737345986077549\\
0.195	0.0128361328661109	0.159710674111862	0.741368418705642\\
0.195	0.0153495058140643	0.174537687332543	0.745784862958202\\
0.195	0.0180991103316243	0.189395466863524	0.750625176092311\\
0.195	0.0210872565164405	0.204278830634725	0.755910489868999\\
0.195	0.0243161165387281	0.219182480034174	0.761650724664453\\
0.195	0.0277877186778607	0.234101003941464	0.767842183867616\\
0.195	0.0315039414701067	0.24902888308801	0.77446537092041\\
0.195	0.0354665079868145	0.263960494742775	0.781483180780606\\
0.195	0.0396769802625738	0.278890117720924	0.788839618268072\\
0.195	0.0441367538930258	0.293811937711588	0.796459186325544\\
0.195	0.0488470528220538	0.308720052919643	0.804247066923454\\
0.195	0.0538089243380495	0.323608480015096	0.81209018616195\\
0.195	0.0590232342988274	0.338471160382329	0.819859213932692\\
0.195	0.064490662604533	0.3533019666601	0.827411499063639\\
0.195	0.0702116989375697	0.368094709561873	0.834594885813505\\
0.195	0.0761866387881432	0.382843144964686	0.841252300287179\\
0.195	0.0824155797834956	0.397540981253432	0.847226939729082\\
0.195	0.0888984183382709	0.412181886906127	0.852367847905041\\
0.195	0.0956348466427212	0.42675949830445	0.856535620027829\\
0.195	0.102624350004627	0.441267427752584	0.859607954619618\\
0.195	0.109866204559871	0.455699271686218	0.861484760266622\\
0.195	0.117359475365564	0.470048619052377	0.86209253425154\\
0.195	0.125103014888515	0.484309059839721	0.861387758068369\\
0.195	0.133095461900593	0.498474193737904	0.85935910089082\\
0.195	0.14133524079124	0.512537638903683	0.856028283765735\\
0.195	0.149820561306021	0.526493040810599	0.851449530870181\\
0.195	0.158549418718625	0.540334081158348	0.845707614739505\\
0.195	0.167519594442213	0.554054486817265	0.838914584343148\\
0.195	0.176728657084455	0.567648038782867	0.831205342370835\\
0.195	0.186173963948925	0.581108581114919	0.822732305417525\\
0.195	0.195852662983903	0.594430029835236	0.813659432944309\\
0.195	0.205761695177907	0.607606381758231	0.804155944104872\\
0.195	0.215897797399558	0.620631723228143	0.794390053406973\\
0.195	0.226257505677663	0.633500238737009	0.784523046062801\\
0.195	0.236837158915675	0.646206219397571	0.77470398286159\\
0.195	0.247632903032949	0.658744071245709	0.765065275211015\\
0.195	0.258640695523528	0.671108323347402	0.755719307772964\\
0.195	0.269856310421543	0.683293635685828	0.746756214005911\\
0.195	0.28127534366066	0.695294806804924	0.738242834604294\\
0.195	0.292893218813452	0.707106781186547	0.730222816007835\\
0.195	0.304705193195075	0.71872465633934	0.722717741087768\\
0.195	0.316706364314172	0.730143689578457	0.715729131143267\\
0.195	0.328891676652598	0.741359304476472	0.709241120561856\\
0.195	0.341255928754291	0.752367096967051	0.703223584571264\\
0.195	0.353793780602429	0.763162841084325	0.697635496597417\\
0.195	0.366499761262991	0.773742494322337	0.692428303594619\\
0.195	0.379368276771856	0.784102202600442	0.68754913288707\\
0.195	0.392393618241769	0.794238304822092	0.682943679235327\\
0.195	0.405569970164763	0.804147337016096	0.678558662193046\\
0.195	0.418891418885081	0.813826036051075	0.674343787412872\\
0.195	0.432351961217133	0.823271342915545	0.670253187715551\\
0.195	0.445945513182735	0.832480405557786	0.666246357336157\\
0.195	0.459665918841652	0.841450581281375	0.662288623474187\\
0.195	0.473506959189401	0.850179438693979	0.658351221673191\\
0.195	0.487462361096317	0.85866475920876	0.654411055137884\\
0.195	0.501525806262096	0.866904538099407	0.650450223206469\\
0.195	0.515690940160279	0.874896985111485	0.646455401874958\\
0.195	0.529951380947623	0.882640524634436	0.642417151055154\\
0.195	0.544300728313781	0.890133795440129	0.638329210947589\\
0.195	0.558732572247415	0.897375649995373	0.634187835387499\\
0.195	0.57324050169555	0.904365153357279	0.629991194998375\\
0.195	0.587818113093873	0.911101581661729	0.625738868900924\\
0.195	0.602459018746568	0.917584420216504	0.621431431633139\\
0.195	0.617156855035314	0.923813361211857	0.617070132480554\\
0.195	0.631905290438127	0.92978830106243	0.612656657833836\\
0.195	0.6466980333399	0.935509337395467	0.608192963377087\\
0.195	0.661528839617671	0.940976765701173	0.603681161495106\\
0.195	0.676391519984903	0.94619107566195	0.599123449738607\\
0.195	0.691279947080357	0.951152947177946	0.59452206790159\\
0.195	0.706188062288412	0.955863246106974	0.589879273662392\\
0.195	0.721109882279076	0.960323019737426	0.585197329319937\\
0.195	0.736039505257226	0.964533492013186	0.580478494543976\\
0.195	0.75097111691199	0.968496058529893	0.575725022018314\\
0.195	0.765898996058536	0.972212281322139	0.570939154289561\\
0.195	0.780817519965826	0.975683883461272	0.566123121053172\\
0.195	0.795721169365275	0.978912743483559	0.561279136605557\\
0.195	0.810604533136476	0.981900889668376	0.556409397399201\\
0.195	0.825462312667457	0.984650494185936	0.551516079695733\\
0.195	0.840289325888138	0.987163867133889	0.546601337319641\\
0.195	0.855080510976839	0.989443450481767	0.541667299515141\\
0.195	0.86983092974082	0.991491811941914	0.536716068908599\\
0.195	0.884535770672926	0.993311638785082	0.531749719578143\\
0.195	0.899190351687411	0.994905731618374	0.526770295231792\\
0.195	0.913790122539012	0.996276998142666	0.521779807495104\\
0.195	0.928330666930242	0.997428446906011	0.516780234308683\\
0.195	0.942807704312699	0.998363181068901	0.511773518435871\\
0.195	0.95721709138901	0.999084392196567	0.506761566080168\\
0.195	0.971554823322703	0.999595354092743	0.501746245611875\\
0.195	0.985817034663988	0.999899416688637	0.496729386403006\\
0.195	1	1	0.491669087346904\\
0.21	0	0	0.710195569525497\\
0.21	0.000100583311362513	0.0141829653360114	0.71240460418299\\
0.21	0.000404645907256436	0.0284451766772965	0.714630690536905\\
0.21	0.000915607803432999	0.0427829086109896	0.716916312064106\\
0.21	0.00163681893109844	0.057192295687301	0.719289019360234\\
0.21	0.00257155309398959	0.0716693330697584	0.721780367061016\\
0.21	0.00372300185733413	0.0862098774609879	0.724425655492835\\
0.21	0.00509426838162598	0.100809648312589	0.727263413850304\\
0.21	0.00668836121491815	0.115464229327074	0.73033458997498\\
0.21	0.00850818805808555	0.13016907025918	0.733681422514709\\
0.21	0.0105565495182326	0.144919489023162	0.737345986077549\\
0.21	0.0128361328661109	0.159710674111862	0.741368418705642\\
0.21	0.0153495058140643	0.174537687332543	0.745784862958201\\
0.21	0.0180991103316243	0.189395466863524	0.750625176092311\\
0.21	0.0210872565164405	0.204278830634725	0.755910489868998\\
0.21	0.0243161165387281	0.219182480034174	0.761650724664453\\
0.21	0.0277877186778607	0.234101003941464	0.767842183867616\\
0.21	0.0315039414701067	0.24902888308801	0.77446537092041\\
0.21	0.0354665079868145	0.263960494742775	0.781483180780606\\
0.21	0.0396769802625738	0.278890117720924	0.788839618268073\\
0.21	0.0441367538930258	0.293811937711588	0.796459186325543\\
0.21	0.0488470528220538	0.308720052919643	0.804247066923454\\
0.21	0.0538089243380495	0.323608480015097	0.812090186161951\\
0.21	0.0590232342988274	0.338471160382329	0.819859213932694\\
0.21	0.064490662604533	0.3533019666601	0.82741149906364\\
0.21	0.0702116989375697	0.368094709561873	0.834594885813505\\
0.21	0.0761866387881432	0.382843144964686	0.841252300287179\\
0.21	0.0824155797834956	0.397540981253432	0.847226939729081\\
0.21	0.0888984183382709	0.412181886906127	0.852367847905042\\
0.21	0.0956348466427212	0.42675949830445	0.856535620027826\\
0.21	0.102624350004627	0.441267427752584	0.859607954619617\\
0.21	0.109866204559871	0.455699271686218	0.86148476026662\\
0.21	0.117359475365564	0.470048619052377	0.862092534251542\\
0.21	0.125103014888515	0.484309059839721	0.861387758068365\\
0.21	0.133095461900593	0.498474193737904	0.859359100890821\\
0.21	0.14133524079124	0.512537638903683	0.856028283765735\\
0.21	0.149820561306021	0.526493040810599	0.851449530870181\\
0.21	0.158549418718625	0.540334081158348	0.845707614739508\\
0.21	0.167519594442214	0.554054486817265	0.83891458434315\\
0.21	0.176728657084455	0.567648038782867	0.831205342370838\\
0.21	0.186173963948925	0.581108581114919	0.822732305417529\\
0.21	0.195852662983903	0.594430029835237	0.81365943294431\\
0.21	0.205761695177907	0.607606381758231	0.80415594410487\\
0.21	0.215897797399558	0.620631723228143	0.794390053406968\\
0.21	0.226257505677663	0.633500238737009	0.784523046062803\\
0.21	0.236837158915675	0.646206219397571	0.77470398286159\\
0.21	0.247632903032949	0.658744071245709	0.765065275211016\\
0.21	0.258640695523527	0.671108323347402	0.755719307772961\\
0.21	0.269856310421543	0.683293635685828	0.746756214005911\\
0.21	0.28127534366066	0.695294806804924	0.738242834604297\\
0.21	0.292893218813452	0.707106781186547	0.730222816007834\\
0.21	0.304705193195075	0.71872465633934	0.722717741087768\\
0.21	0.316706364314172	0.730143689578457	0.715729131143268\\
0.21	0.328891676652598	0.741359304476472	0.709241120561856\\
0.21	0.341255928754291	0.752367096967051	0.703223584571265\\
0.21	0.353793780602429	0.763162841084325	0.697635496597419\\
0.21	0.366499761262991	0.773742494322337	0.692428303594619\\
0.21	0.379368276771856	0.784102202600443	0.687549132887069\\
0.21	0.392393618241769	0.794238304822092	0.682943679235327\\
0.21	0.405569970164763	0.804147337016097	0.678558662193044\\
0.21	0.418891418885081	0.813826036051075	0.674343787412871\\
0.21	0.432351961217132	0.823271342915544	0.670253187715553\\
0.21	0.445945513182735	0.832480405557786	0.666246357336159\\
0.21	0.459665918841652	0.841450581281375	0.662288623474186\\
0.21	0.473506959189401	0.850179438693979	0.658351221673191\\
0.21	0.487462361096317	0.85866475920876	0.654411055137886\\
0.21	0.501525806262096	0.866904538099407	0.650450223206468\\
0.21	0.515690940160279	0.874896985111485	0.646455401874959\\
0.21	0.529951380947623	0.882640524634436	0.642417151055154\\
0.21	0.544300728313782	0.890133795440129	0.638329210947587\\
0.21	0.558732572247415	0.897375649995373	0.6341878353875\\
0.21	0.57324050169555	0.904365153357279	0.629991194998377\\
0.21	0.587818113093873	0.911101581661729	0.625738868900923\\
0.21	0.602459018746568	0.917584420216504	0.621431431633139\\
0.21	0.617156855035314	0.923813361211857	0.617070132480553\\
0.21	0.631905290438127	0.92978830106243	0.612656657833838\\
0.21	0.6466980333399	0.935509337395467	0.608192963377087\\
0.21	0.661528839617671	0.940976765701173	0.603681161495105\\
0.21	0.676391519984904	0.946191075661951	0.599123449738607\\
0.21	0.691279947080357	0.951152947177946	0.594522067901587\\
0.21	0.706188062288412	0.955863246106974	0.589879273662389\\
0.21	0.721109882279076	0.960323019737426	0.585197329319939\\
0.21	0.736039505257226	0.964533492013186	0.580478494543975\\
0.21	0.75097111691199	0.968496058529893	0.575725022018313\\
0.21	0.765898996058536	0.972212281322139	0.570939154289562\\
0.21	0.780817519965826	0.975683883461272	0.566123121053177\\
0.21	0.795721169365275	0.97891274348356	0.561279136605558\\
0.21	0.810604533136476	0.981900889668376	0.556409397399196\\
0.21	0.825462312667457	0.984650494185936	0.551516079695732\\
0.21	0.840289325888138	0.987163867133889	0.546601337319642\\
0.21	0.855080510976838	0.989443450481767	0.541667299515141\\
0.21	0.86983092974082	0.991491811941914	0.536716068908598\\
0.21	0.884535770672926	0.993311638785082	0.531749719578143\\
0.21	0.899190351687411	0.994905731618374	0.526770295231793\\
0.21	0.913790122539012	0.996276998142666	0.521779807495104\\
0.21	0.928330666930242	0.997428446906011	0.516780234308684\\
0.21	0.942807704312699	0.998363181068902	0.51177351843587\\
0.21	0.95721709138901	0.999084392196567	0.50676156608017\\
0.21	0.971554823322703	0.999595354092743	0.501746245611874\\
0.21	0.985817034663989	0.999899416688637	0.496729386403004\\
0.21	1	1	0.491669087346906\\
0.225	0	0	0.710195569525497\\
0.225	0.000100583311362513	0.0141829653360114	0.71240460418299\\
0.225	0.000404645907256436	0.0284451766772965	0.714630690536905\\
0.225	0.000915607803432999	0.0427829086109896	0.716916312064106\\
0.225	0.00163681893109843	0.057192295687301	0.719289019360234\\
0.225	0.00257155309398959	0.0716693330697584	0.721780367061016\\
0.225	0.00372300185733414	0.0862098774609879	0.724425655492835\\
0.225	0.00509426838162598	0.100809648312589	0.727263413850304\\
0.225	0.00668836121491815	0.115464229327074	0.73033458997498\\
0.225	0.00850818805808554	0.13016907025918	0.733681422514709\\
0.225	0.0105565495182326	0.144919489023162	0.737345986077549\\
0.225	0.0128361328661109	0.159710674111862	0.741368418705642\\
0.225	0.0153495058140643	0.174537687332543	0.745784862958201\\
0.225	0.0180991103316243	0.189395466863524	0.750625176092311\\
0.225	0.0210872565164405	0.204278830634725	0.755910489868998\\
0.225	0.0243161165387281	0.219182480034174	0.761650724664453\\
0.225	0.0277877186778607	0.234101003941464	0.767842183867616\\
0.225	0.0315039414701067	0.24902888308801	0.77446537092041\\
0.225	0.0354665079868145	0.263960494742775	0.781483180780606\\
0.225	0.0396769802625738	0.278890117720924	0.788839618268074\\
0.225	0.0441367538930258	0.293811937711588	0.796459186325544\\
0.225	0.0488470528220537	0.308720052919643	0.804247066923454\\
0.225	0.0538089243380495	0.323608480015096	0.81209018616195\\
0.225	0.0590232342988274	0.338471160382329	0.819859213932693\\
0.225	0.064490662604533	0.3533019666601	0.827411499063639\\
0.225	0.0702116989375697	0.368094709561873	0.834594885813505\\
0.225	0.0761866387881432	0.382843144964686	0.841252300287179\\
0.225	0.0824155797834956	0.397540981253432	0.847226939729079\\
0.225	0.0888984183382709	0.412181886906127	0.852367847905041\\
0.225	0.0956348466427212	0.42675949830445	0.856535620027829\\
0.225	0.102624350004627	0.441267427752585	0.859607954619618\\
0.225	0.109866204559871	0.455699271686218	0.86148476026662\\
0.225	0.117359475365564	0.470048619052377	0.86209253425154\\
0.225	0.125103014888515	0.484309059839721	0.861387758068368\\
0.225	0.133095461900593	0.498474193737904	0.859359100890821\\
0.225	0.14133524079124	0.512537638903683	0.856028283765734\\
0.225	0.149820561306021	0.526493040810599	0.851449530870185\\
0.225	0.158549418718625	0.540334081158348	0.845707614739505\\
0.225	0.167519594442213	0.554054486817265	0.838914584343153\\
0.225	0.176728657084455	0.567648038782867	0.831205342370838\\
0.225	0.186173963948925	0.581108581114919	0.822732305417528\\
0.225	0.195852662983903	0.594430029835237	0.813659432944307\\
0.225	0.205761695177907	0.607606381758231	0.804155944104867\\
0.225	0.215897797399558	0.620631723228143	0.794390053406974\\
0.225	0.226257505677663	0.633500238737009	0.784523046062803\\
0.225	0.236837158915675	0.646206219397571	0.77470398286159\\
0.225	0.247632903032949	0.658744071245709	0.765065275211015\\
0.225	0.258640695523527	0.671108323347402	0.755719307772963\\
0.225	0.269856310421543	0.683293635685828	0.746756214005912\\
0.225	0.28127534366066	0.695294806804924	0.738242834604297\\
0.225	0.292893218813452	0.707106781186547	0.730222816007834\\
0.225	0.304705193195075	0.71872465633934	0.722717741087768\\
0.225	0.316706364314172	0.730143689578457	0.715729131143269\\
0.225	0.328891676652598	0.741359304476473	0.709241120561854\\
0.225	0.341255928754291	0.752367096967051	0.703223584571264\\
0.225	0.353793780602429	0.763162841084325	0.69763549659742\\
0.225	0.366499761262991	0.773742494322337	0.692428303594619\\
0.225	0.379368276771856	0.784102202600442	0.687549132887069\\
0.225	0.392393618241769	0.794238304822092	0.682943679235329\\
0.225	0.405569970164763	0.804147337016097	0.678558662193045\\
0.225	0.418891418885081	0.813826036051075	0.67434378741287\\
0.225	0.432351961217132	0.823271342915544	0.670253187715551\\
0.225	0.445945513182735	0.832480405557786	0.66624635733616\\
0.225	0.459665918841652	0.841450581281375	0.662288623474187\\
0.225	0.473506959189401	0.850179438693979	0.65835122167319\\
0.225	0.487462361096317	0.85866475920876	0.654411055137886\\
0.225	0.501525806262096	0.866904538099407	0.650450223206469\\
0.225	0.515690940160279	0.874896985111485	0.646455401874959\\
0.225	0.529951380947623	0.882640524634436	0.642417151055155\\
0.225	0.544300728313781	0.890133795440129	0.638329210947587\\
0.225	0.558732572247415	0.897375649995373	0.634187835387498\\
0.225	0.57324050169555	0.904365153357279	0.629991194998376\\
0.225	0.587818113093873	0.911101581661729	0.625738868900924\\
0.225	0.602459018746568	0.917584420216504	0.62143143163314\\
0.225	0.617156855035314	0.923813361211857	0.617070132480553\\
0.225	0.631905290438127	0.92978830106243	0.612656657833838\\
0.225	0.6466980333399	0.935509337395467	0.608192963377086\\
0.225	0.661528839617671	0.940976765701173	0.603681161495106\\
0.225	0.676391519984903	0.94619107566195	0.599123449738607\\
0.225	0.691279947080357	0.951152947177946	0.59452206790159\\
0.225	0.706188062288412	0.955863246106974	0.589879273662389\\
0.225	0.721109882279076	0.960323019737426	0.585197329319937\\
0.225	0.736039505257226	0.964533492013186	0.580478494543975\\
0.225	0.75097111691199	0.968496058529893	0.575725022018312\\
0.225	0.765898996058536	0.972212281322139	0.570939154289562\\
0.225	0.780817519965826	0.975683883461272	0.566123121053175\\
0.225	0.795721169365275	0.978912743483559	0.561279136605558\\
0.225	0.810604533136476	0.981900889668376	0.556409397399202\\
0.225	0.825462312667457	0.984650494185936	0.551516079695735\\
0.225	0.840289325888138	0.987163867133889	0.546601337319639\\
0.225	0.855080510976838	0.989443450481767	0.541667299515141\\
0.225	0.86983092974082	0.991491811941915	0.536716068908598\\
0.225	0.884535770672926	0.993311638785082	0.531749719578141\\
0.225	0.899190351687411	0.994905731618374	0.526770295231794\\
0.225	0.913790122539012	0.996276998142666	0.521779807495102\\
0.225	0.928330666930242	0.997428446906011	0.516780234308686\\
0.225	0.942807704312699	0.998363181068902	0.51177351843587\\
0.225	0.95721709138901	0.999084392196567	0.506761566080168\\
0.225	0.971554823322704	0.999595354092743	0.501746245611876\\
0.225	0.985817034663989	0.999899416688637	0.496729386403004\\
0.225	1	1	0.491669087346906\\
0.24	0	0	0.710195569525497\\
0.24	0.000100583311362513	0.0141829653360114	0.71240460418299\\
0.24	0.000404645907256436	0.0284451766772965	0.714630690536905\\
0.24	0.000915607803432999	0.0427829086109896	0.716916312064106\\
0.24	0.00163681893109844	0.057192295687301	0.719289019360234\\
0.24	0.00257155309398959	0.0716693330697584	0.721780367061016\\
0.24	0.00372300185733413	0.0862098774609879	0.724425655492835\\
0.24	0.00509426838162598	0.100809648312589	0.727263413850304\\
0.24	0.00668836121491816	0.115464229327074	0.73033458997498\\
0.24	0.00850818805808554	0.13016907025918	0.733681422514709\\
0.24	0.0105565495182326	0.144919489023162	0.737345986077549\\
0.24	0.0128361328661109	0.159710674111862	0.741368418705642\\
0.24	0.0153495058140643	0.174537687332543	0.745784862958202\\
0.24	0.0180991103316243	0.189395466863524	0.750625176092311\\
0.24	0.0210872565164405	0.204278830634725	0.755910489868998\\
0.24	0.0243161165387281	0.219182480034174	0.761650724664453\\
0.24	0.0277877186778607	0.234101003941464	0.767842183867616\\
0.24	0.0315039414701067	0.24902888308801	0.77446537092041\\
0.24	0.0354665079868145	0.263960494742775	0.781483180780606\\
0.24	0.0396769802625738	0.278890117720924	0.788839618268074\\
0.24	0.0441367538930258	0.293811937711588	0.796459186325544\\
0.24	0.0488470528220538	0.308720052919643	0.804247066923455\\
0.24	0.0538089243380495	0.323608480015096	0.81209018616195\\
0.24	0.0590232342988274	0.338471160382329	0.819859213932694\\
0.24	0.064490662604533	0.3533019666601	0.827411499063638\\
0.24	0.0702116989375697	0.368094709561873	0.834594885813506\\
0.24	0.0761866387881432	0.382843144964686	0.84125230028718\\
0.24	0.0824155797834956	0.397540981253432	0.847226939729081\\
0.24	0.0888984183382709	0.412181886906127	0.852367847905043\\
0.24	0.0956348466427212	0.42675949830445	0.856535620027832\\
0.24	0.102624350004627	0.441267427752584	0.859607954619617\\
0.24	0.109866204559871	0.455699271686218	0.861484760266622\\
0.24	0.117359475365564	0.470048619052377	0.86209253425154\\
0.24	0.125103014888515	0.484309059839721	0.861387758068373\\
0.24	0.133095461900593	0.498474193737904	0.85935910089082\\
0.24	0.14133524079124	0.512537638903683	0.856028283765738\\
0.24	0.149820561306021	0.526493040810599	0.851449530870183\\
0.24	0.158549418718625	0.540334081158348	0.845707614739502\\
0.24	0.167519594442213	0.554054486817265	0.838914584343151\\
0.24	0.176728657084455	0.567648038782867	0.831205342370836\\
0.24	0.186173963948925	0.581108581114919	0.822732305417528\\
0.24	0.195852662983903	0.594430029835237	0.813659432944304\\
0.24	0.205761695177907	0.607606381758231	0.80415594410487\\
0.24	0.215897797399558	0.620631723228144	0.79439005340697\\
0.24	0.226257505677663	0.633500238737009	0.7845230460628\\
0.24	0.236837158915675	0.646206219397571	0.774703982861588\\
0.24	0.247632903032949	0.658744071245709	0.765065275211014\\
0.24	0.258640695523527	0.671108323347402	0.755719307772963\\
0.24	0.269856310421543	0.683293635685828	0.746756214005913\\
0.24	0.28127534366066	0.695294806804925	0.738242834604295\\
0.24	0.292893218813452	0.707106781186547	0.730222816007833\\
0.24	0.304705193195075	0.71872465633934	0.722717741087768\\
0.24	0.316706364314172	0.730143689578457	0.715729131143268\\
0.24	0.328891676652598	0.741359304476472	0.709241120561855\\
0.24	0.341255928754291	0.752367096967051	0.703223584571266\\
0.24	0.353793780602429	0.763162841084325	0.697635496597419\\
0.24	0.366499761262991	0.773742494322337	0.692428303594619\\
0.24	0.379368276771856	0.784102202600443	0.687549132887069\\
0.24	0.392393618241769	0.794238304822092	0.682943679235328\\
0.24	0.405569970164763	0.804147337016097	0.678558662193044\\
0.24	0.418891418885081	0.813826036051075	0.67434378741287\\
0.24	0.432351961217133	0.823271342915544	0.670253187715551\\
0.24	0.445945513182734	0.832480405557786	0.666246357336159\\
0.24	0.459665918841652	0.841450581281375	0.662288623474188\\
0.24	0.473506959189401	0.850179438693979	0.658351221673191\\
0.24	0.487462361096317	0.85866475920876	0.654411055137884\\
0.24	0.501525806262096	0.866904538099407	0.650450223206469\\
0.24	0.515690940160279	0.874896985111485	0.646455401874957\\
0.24	0.529951380947623	0.882640524634436	0.642417151055156\\
0.24	0.544300728313781	0.890133795440129	0.638329210947587\\
0.24	0.558732572247415	0.897375649995373	0.6341878353875\\
0.24	0.57324050169555	0.904365153357279	0.629991194998376\\
0.24	0.587818113093873	0.911101581661729	0.625738868900922\\
0.24	0.602459018746568	0.917584420216504	0.621431431633139\\
0.24	0.617156855035314	0.923813361211857	0.617070132480554\\
0.24	0.631905290438127	0.92978830106243	0.612656657833839\\
0.24	0.6466980333399	0.935509337395467	0.608192963377085\\
0.24	0.661528839617671	0.940976765701172	0.603681161495103\\
0.24	0.676391519984903	0.94619107566195	0.59912344973861\\
0.24	0.691279947080357	0.951152947177946	0.594522067901589\\
0.24	0.706188062288412	0.955863246106974	0.589879273662389\\
0.24	0.721109882279076	0.960323019737426	0.585197329319937\\
0.24	0.736039505257226	0.964533492013186	0.580478494543976\\
0.24	0.75097111691199	0.968496058529893	0.575725022018312\\
0.24	0.765898996058536	0.972212281322139	0.570939154289558\\
0.24	0.780817519965825	0.975683883461272	0.566123121053174\\
0.24	0.795721169365275	0.978912743483559	0.561279136605561\\
0.24	0.810604533136476	0.981900889668376	0.5564093973992\\
0.24	0.825462312667457	0.984650494185936	0.551516079695736\\
0.24	0.840289325888138	0.987163867133889	0.546601337319641\\
0.24	0.855080510976839	0.989443450481767	0.54166729951514\\
0.24	0.86983092974082	0.991491811941914	0.5367160689086\\
0.24	0.884535770672926	0.993311638785082	0.531749719578142\\
0.24	0.899190351687411	0.994905731618374	0.526770295231795\\
0.24	0.913790122539012	0.996276998142666	0.5217798074951\\
0.24	0.928330666930241	0.99742844690601	0.516780234308682\\
0.24	0.942807704312699	0.998363181068901	0.511773518435872\\
0.24	0.95721709138901	0.999084392196567	0.506761566080171\\
0.24	0.971554823322703	0.999595354092744	0.501746245611876\\
0.24	0.985817034663989	0.999899416688637	0.496729386403002\\
0.24	1	1	0.491669087346908\\
0.255	0	0	0.710195569525497\\
0.255	0.000100583311362513	0.0141829653360114	0.71240460418299\\
0.255	0.000404645907256436	0.0284451766772965	0.714630690536905\\
0.255	0.000915607803432999	0.0427829086109896	0.716916312064106\\
0.255	0.00163681893109843	0.057192295687301	0.719289019360234\\
0.255	0.00257155309398959	0.0716693330697584	0.721780367061016\\
0.255	0.00372300185733413	0.0862098774609879	0.724425655492835\\
0.255	0.00509426838162598	0.100809648312589	0.727263413850304\\
0.255	0.00668836121491815	0.115464229327074	0.73033458997498\\
0.255	0.00850818805808554	0.13016907025918	0.733681422514709\\
0.255	0.0105565495182326	0.144919489023162	0.737345986077549\\
0.255	0.0128361328661109	0.159710674111862	0.741368418705642\\
0.255	0.0153495058140643	0.174537687332543	0.745784862958201\\
0.255	0.0180991103316243	0.189395466863524	0.750625176092311\\
0.255	0.0210872565164405	0.204278830634725	0.755910489868999\\
0.255	0.0243161165387281	0.219182480034174	0.761650724664453\\
0.255	0.0277877186778607	0.234101003941464	0.767842183867616\\
0.255	0.0315039414701067	0.24902888308801	0.77446537092041\\
0.255	0.0354665079868145	0.263960494742775	0.781483180780606\\
0.255	0.0396769802625738	0.278890117720924	0.788839618268073\\
0.255	0.0441367538930258	0.293811937711588	0.796459186325544\\
0.255	0.0488470528220538	0.308720052919643	0.804247066923456\\
0.255	0.0538089243380495	0.323608480015096	0.81209018616195\\
0.255	0.0590232342988274	0.338471160382329	0.819859213932694\\
0.255	0.064490662604533	0.3533019666601	0.827411499063639\\
0.255	0.0702116989375697	0.368094709561873	0.834594885813506\\
0.255	0.0761866387881432	0.382843144964686	0.841252300287181\\
0.255	0.0824155797834956	0.397540981253432	0.847226939729082\\
0.255	0.0888984183382709	0.412181886906127	0.852367847905041\\
0.255	0.0956348466427212	0.42675949830445	0.85653562002783\\
0.255	0.102624350004627	0.441267427752585	0.859607954619617\\
0.255	0.109866204559871	0.455699271686218	0.861484760266619\\
0.255	0.117359475365564	0.470048619052377	0.862092534251541\\
0.255	0.125103014888515	0.484309059839721	0.861387758068367\\
0.255	0.133095461900593	0.498474193737904	0.859359100890821\\
0.255	0.14133524079124	0.512537638903683	0.856028283765736\\
0.255	0.149820561306021	0.526493040810599	0.851449530870181\\
0.255	0.158549418718625	0.540334081158348	0.845707614739506\\
0.255	0.167519594442213	0.554054486817265	0.838914584343148\\
0.255	0.176728657084455	0.567648038782867	0.831205342370836\\
0.255	0.186173963948925	0.581108581114919	0.822732305417527\\
0.255	0.195852662983903	0.594430029835236	0.813659432944305\\
0.255	0.205761695177907	0.607606381758231	0.804155944104871\\
0.255	0.215897797399558	0.620631723228143	0.794390053406971\\
0.255	0.226257505677663	0.633500238737009	0.7845230460628\\
0.255	0.236837158915675	0.646206219397571	0.774703982861588\\
0.255	0.247632903032949	0.658744071245709	0.765065275211018\\
0.255	0.258640695523527	0.671108323347402	0.755719307772962\\
0.255	0.269856310421543	0.683293635685828	0.74675621400591\\
0.255	0.28127534366066	0.695294806804924	0.738242834604296\\
0.255	0.292893218813452	0.707106781186548	0.730222816007836\\
0.255	0.304705193195075	0.71872465633934	0.722717741087767\\
0.255	0.316706364314172	0.730143689578457	0.715729131143269\\
0.255	0.328891676652598	0.741359304476472	0.709241120561857\\
0.255	0.341255928754291	0.752367096967051	0.703223584571265\\
0.255	0.353793780602429	0.763162841084325	0.697635496597417\\
0.255	0.366499761262991	0.773742494322337	0.692428303594618\\
0.255	0.379368276771856	0.784102202600442	0.687549132887069\\
0.255	0.392393618241769	0.794238304822092	0.682943679235329\\
0.255	0.405569970164763	0.804147337016097	0.678558662193045\\
0.255	0.418891418885081	0.813826036051075	0.67434378741287\\
0.255	0.432351961217132	0.823271342915544	0.67025318771555\\
0.255	0.445945513182735	0.832480405557786	0.666246357336158\\
0.255	0.459665918841652	0.841450581281375	0.662288623474188\\
0.255	0.473506959189401	0.850179438693979	0.658351221673192\\
0.255	0.487462361096317	0.85866475920876	0.654411055137884\\
0.255	0.501525806262096	0.866904538099407	0.650450223206469\\
0.255	0.515690940160279	0.874896985111485	0.646455401874958\\
0.255	0.529951380947623	0.882640524634436	0.642417151055153\\
0.255	0.544300728313782	0.890133795440129	0.638329210947587\\
0.255	0.558732572247415	0.897375649995373	0.634187835387499\\
0.255	0.57324050169555	0.904365153357279	0.629991194998376\\
0.255	0.587818113093873	0.911101581661729	0.625738868900926\\
0.255	0.602459018746568	0.917584420216505	0.62143143163314\\
0.255	0.617156855035314	0.923813361211857	0.61707013248055\\
0.255	0.631905290438127	0.92978830106243	0.612656657833837\\
0.255	0.6466980333399	0.935509337395467	0.60819296337709\\
0.255	0.661528839617671	0.940976765701173	0.603681161495104\\
0.255	0.676391519984903	0.94619107566195	0.599123449738607\\
0.255	0.691279947080357	0.951152947177946	0.594522067901591\\
0.255	0.706188062288412	0.955863246106974	0.589879273662389\\
0.255	0.721109882279076	0.960323019737426	0.585197329319937\\
0.255	0.736039505257226	0.964533492013186	0.580478494543976\\
0.255	0.75097111691199	0.968496058529893	0.575725022018312\\
0.255	0.765898996058536	0.972212281322139	0.570939154289561\\
0.255	0.780817519965826	0.975683883461272	0.566123121053174\\
0.255	0.795721169365274	0.978912743483559	0.561279136605557\\
0.255	0.810604533136476	0.981900889668376	0.556409397399199\\
0.255	0.825462312667457	0.984650494185936	0.551516079695735\\
0.255	0.840289325888138	0.987163867133889	0.546601337319641\\
0.255	0.855080510976839	0.989443450481767	0.541667299515139\\
0.255	0.86983092974082	0.991491811941914	0.536716068908601\\
0.255	0.884535770672926	0.993311638785082	0.531749719578143\\
0.255	0.899190351687411	0.994905731618374	0.526770295231796\\
0.255	0.913790122539012	0.996276998142666	0.521779807495103\\
0.255	0.928330666930242	0.997428446906011	0.516780234308681\\
0.255	0.942807704312699	0.998363181068902	0.511773518435872\\
0.255	0.95721709138901	0.999084392196567	0.506761566080168\\
0.255	0.971554823322703	0.999595354092743	0.501746245611874\\
0.255	0.985817034663988	0.999899416688637	0.496729386403004\\
0.255	1	1	0.491669087346906\\
0.27	0	0	0.710195569525497\\
0.27	0.000100583311362513	0.0141829653360114	0.71240460418299\\
0.27	0.000404645907256436	0.0284451766772965	0.714630690536905\\
0.27	0.000915607803432999	0.0427829086109896	0.716916312064106\\
0.27	0.00163681893109844	0.057192295687301	0.719289019360234\\
0.27	0.00257155309398959	0.0716693330697584	0.721780367061016\\
0.27	0.00372300185733413	0.0862098774609879	0.724425655492835\\
0.27	0.00509426838162598	0.100809648312589	0.727263413850304\\
0.27	0.00668836121491816	0.115464229327074	0.73033458997498\\
0.27	0.00850818805808555	0.13016907025918	0.733681422514709\\
0.27	0.0105565495182326	0.144919489023162	0.737345986077549\\
0.27	0.0128361328661109	0.159710674111862	0.741368418705642\\
0.27	0.0153495058140643	0.174537687332543	0.745784862958201\\
0.27	0.0180991103316243	0.189395466863524	0.750625176092311\\
0.27	0.0210872565164405	0.204278830634725	0.755910489868999\\
0.27	0.0243161165387281	0.219182480034174	0.761650724664453\\
0.27	0.0277877186778607	0.234101003941464	0.767842183867616\\
0.27	0.0315039414701067	0.24902888308801	0.77446537092041\\
0.27	0.0354665079868146	0.263960494742775	0.781483180780606\\
0.27	0.0396769802625738	0.278890117720924	0.788839618268073\\
0.27	0.0441367538930258	0.293811937711588	0.796459186325544\\
0.27	0.0488470528220538	0.308720052919643	0.804247066923455\\
0.27	0.0538089243380495	0.323608480015097	0.812090186161951\\
0.27	0.0590232342988274	0.338471160382329	0.819859213932692\\
0.27	0.064490662604533	0.3533019666601	0.827411499063639\\
0.27	0.0702116989375697	0.368094709561873	0.834594885813505\\
0.27	0.0761866387881432	0.382843144964686	0.841252300287181\\
0.27	0.0824155797834956	0.397540981253432	0.847226939729081\\
0.27	0.0888984183382709	0.412181886906127	0.85236784790504\\
0.27	0.0956348466427212	0.42675949830445	0.856535620027828\\
0.27	0.102624350004627	0.441267427752584	0.859607954619618\\
0.27	0.109866204559871	0.455699271686218	0.861484760266623\\
0.27	0.117359475365564	0.470048619052377	0.862092534251541\\
0.27	0.125103014888515	0.484309059839721	0.861387758068365\\
0.27	0.133095461900593	0.498474193737904	0.859359100890819\\
0.27	0.14133524079124	0.512537638903683	0.856028283765737\\
0.27	0.149820561306021	0.526493040810599	0.851449530870183\\
0.27	0.158549418718625	0.540334081158348	0.845707614739504\\
0.27	0.167519594442213	0.554054486817265	0.838914584343148\\
0.27	0.176728657084455	0.567648038782867	0.831205342370837\\
0.27	0.186173963948925	0.581108581114919	0.822732305417528\\
0.27	0.195852662983903	0.594430029835237	0.813659432944309\\
0.27	0.205761695177907	0.607606381758231	0.804155944104872\\
0.27	0.215897797399558	0.620631723228144	0.794390053406973\\
0.27	0.226257505677663	0.633500238737009	0.7845230460628\\
0.27	0.236837158915675	0.646206219397571	0.77470398286159\\
0.27	0.247632903032949	0.658744071245709	0.765065275211014\\
0.27	0.258640695523527	0.671108323347402	0.75571930777296\\
0.27	0.269856310421543	0.683293635685828	0.746756214005913\\
0.27	0.28127534366066	0.695294806804924	0.738242834604299\\
0.27	0.292893218813452	0.707106781186548	0.730222816007833\\
0.27	0.304705193195075	0.71872465633934	0.722717741087767\\
0.27	0.316706364314172	0.730143689578457	0.715729131143267\\
0.27	0.328891676652598	0.741359304476472	0.709241120561857\\
0.27	0.341255928754291	0.752367096967051	0.703223584571267\\
0.27	0.353793780602429	0.763162841084325	0.697635496597419\\
0.27	0.366499761262991	0.773742494322337	0.692428303594619\\
0.27	0.379368276771856	0.784102202600443	0.687549132887068\\
0.27	0.392393618241769	0.794238304822092	0.682943679235327\\
0.27	0.405569970164763	0.804147337016097	0.678558662193045\\
0.27	0.418891418885081	0.813826036051075	0.674343787412871\\
0.27	0.432351961217133	0.823271342915545	0.670253187715552\\
0.27	0.445945513182735	0.832480405557786	0.666246357336158\\
0.27	0.459665918841652	0.841450581281375	0.662288623474186\\
0.27	0.473506959189401	0.850179438693979	0.658351221673191\\
0.27	0.487462361096317	0.85866475920876	0.654411055137886\\
0.27	0.501525806262096	0.866904538099407	0.650450223206467\\
0.27	0.515690940160279	0.874896985111485	0.646455401874959\\
0.27	0.529951380947623	0.882640524634436	0.642417151055155\\
0.27	0.544300728313781	0.890133795440129	0.638329210947585\\
0.27	0.558732572247415	0.897375649995372	0.634187835387499\\
0.27	0.57324050169555	0.904365153357279	0.629991194998377\\
0.27	0.587818113093873	0.911101581661729	0.625738868900923\\
0.27	0.602459018746568	0.917584420216505	0.621431431633142\\
0.27	0.617156855035314	0.923813361211857	0.617070132480552\\
0.27	0.631905290438127	0.92978830106243	0.612656657833836\\
0.27	0.6466980333399	0.935509337395467	0.608192963377088\\
0.27	0.661528839617671	0.940976765701173	0.603681161495107\\
0.27	0.676391519984903	0.94619107566195	0.599123449738606\\
0.27	0.691279947080357	0.951152947177946	0.594522067901589\\
0.27	0.706188062288412	0.955863246106974	0.58987927366239\\
0.27	0.721109882279076	0.960323019737426	0.585197329319937\\
0.27	0.736039505257226	0.964533492013186	0.580478494543978\\
0.27	0.75097111691199	0.968496058529893	0.575725022018313\\
0.27	0.765898996058536	0.972212281322139	0.570939154289557\\
0.27	0.780817519965826	0.975683883461272	0.566123121053177\\
0.27	0.795721169365275	0.97891274348356	0.561279136605558\\
0.27	0.810604533136476	0.981900889668376	0.556409397399198\\
0.27	0.825462312667457	0.984650494185936	0.551516079695735\\
0.27	0.840289325888138	0.987163867133889	0.546601337319639\\
0.27	0.855080510976839	0.989443450481767	0.541667299515139\\
0.27	0.86983092974082	0.991491811941914	0.536716068908601\\
0.27	0.884535770672926	0.993311638785082	0.531749719578143\\
0.27	0.899190351687411	0.994905731618374	0.526770295231793\\
0.27	0.913790122539012	0.996276998142666	0.521779807495104\\
0.27	0.928330666930242	0.997428446906011	0.516780234308683\\
0.27	0.942807704312699	0.998363181068902	0.51177351843587\\
0.27	0.95721709138901	0.999084392196567	0.506761566080172\\
0.27	0.971554823322704	0.999595354092744	0.501746245611876\\
0.27	0.985817034663989	0.999899416688637	0.496729386403002\\
0.27	1	1	0.491669087346908\\
0.285	0	0	0.710195569525497\\
0.285	0.000100583311362513	0.0141829653360114	0.71240460418299\\
0.285	0.000404645907256436	0.0284451766772965	0.714630690536905\\
0.285	0.000915607803432999	0.0427829086109896	0.716916312064106\\
0.285	0.00163681893109843	0.057192295687301	0.719289019360234\\
0.285	0.00257155309398959	0.0716693330697584	0.721780367061016\\
0.285	0.00372300185733414	0.0862098774609879	0.724425655492835\\
0.285	0.00509426838162598	0.100809648312589	0.727263413850304\\
0.285	0.00668836121491815	0.115464229327074	0.73033458997498\\
0.285	0.00850818805808555	0.13016907025918	0.733681422514709\\
0.285	0.0105565495182326	0.144919489023162	0.737345986077549\\
0.285	0.0128361328661109	0.159710674111862	0.741368418705642\\
0.285	0.0153495058140643	0.174537687332543	0.745784862958201\\
0.285	0.0180991103316243	0.189395466863524	0.750625176092311\\
0.285	0.0210872565164405	0.204278830634725	0.755910489868999\\
0.285	0.0243161165387281	0.219182480034174	0.761650724664453\\
0.285	0.0277877186778607	0.234101003941464	0.767842183867616\\
0.285	0.0315039414701067	0.24902888308801	0.77446537092041\\
0.285	0.0354665079868145	0.263960494742775	0.781483180780606\\
0.285	0.0396769802625738	0.278890117720924	0.788839618268072\\
0.285	0.0441367538930258	0.293811937711588	0.796459186325543\\
0.285	0.0488470528220538	0.308720052919643	0.804247066923454\\
0.285	0.0538089243380495	0.323608480015097	0.812090186161951\\
0.285	0.0590232342988274	0.338471160382329	0.819859213932693\\
0.285	0.064490662604533	0.3533019666601	0.827411499063638\\
0.285	0.0702116989375697	0.368094709561873	0.834594885813504\\
0.285	0.0761866387881432	0.382843144964686	0.84125230028718\\
0.285	0.0824155797834956	0.397540981253432	0.84722693972908\\
0.285	0.0888984183382709	0.412181886906127	0.852367847905041\\
0.285	0.0956348466427212	0.42675949830445	0.856535620027829\\
0.285	0.102624350004627	0.441267427752584	0.859607954619617\\
0.285	0.109866204559871	0.455699271686218	0.861484760266622\\
0.285	0.117359475365563	0.470048619052377	0.86209253425154\\
0.285	0.125103014888515	0.484309059839721	0.861387758068368\\
0.285	0.133095461900593	0.498474193737904	0.859359100890819\\
0.285	0.14133524079124	0.512537638903683	0.856028283765738\\
0.285	0.149820561306021	0.526493040810599	0.851449530870183\\
0.285	0.158549418718625	0.540334081158348	0.845707614739505\\
0.285	0.167519594442213	0.554054486817265	0.838914584343155\\
0.285	0.176728657084455	0.567648038782867	0.831205342370838\\
0.285	0.186173963948925	0.581108581114919	0.82273230541753\\
0.285	0.195852662983903	0.594430029835236	0.813659432944305\\
0.285	0.205761695177907	0.607606381758231	0.804155944104871\\
0.285	0.215897797399558	0.620631723228144	0.794390053406967\\
0.285	0.226257505677663	0.633500238737009	0.784523046062799\\
0.285	0.236837158915675	0.646206219397571	0.774703982861588\\
0.285	0.247632903032949	0.658744071245709	0.765065275211015\\
0.285	0.258640695523527	0.671108323347402	0.755719307772963\\
0.285	0.269856310421543	0.683293635685828	0.746756214005914\\
0.285	0.28127534366066	0.695294806804925	0.738242834604294\\
0.285	0.292893218813452	0.707106781186547	0.730222816007833\\
0.285	0.304705193195076	0.71872465633934	0.72271774108777\\
0.285	0.316706364314172	0.730143689578457	0.715729131143269\\
0.285	0.328891676652598	0.741359304476473	0.709241120561855\\
0.285	0.341255928754291	0.752367096967051	0.703223584571263\\
0.285	0.353793780602429	0.763162841084325	0.697635496597419\\
0.285	0.366499761262991	0.773742494322337	0.69242830359462\\
0.285	0.379368276771856	0.784102202600443	0.687549132887069\\
0.285	0.392393618241769	0.794238304822092	0.682943679235328\\
0.285	0.405569970164763	0.804147337016097	0.678558662193045\\
0.285	0.418891418885081	0.813826036051075	0.674343787412871\\
0.285	0.432351961217133	0.823271342915545	0.670253187715551\\
0.285	0.445945513182735	0.832480405557786	0.666246357336158\\
0.285	0.459665918841652	0.841450581281375	0.662288623474187\\
0.285	0.473506959189401	0.850179438693979	0.658351221673191\\
0.285	0.487462361096317	0.85866475920876	0.654411055137886\\
0.285	0.501525806262096	0.866904538099407	0.650450223206469\\
0.285	0.515690940160279	0.874896985111485	0.646455401874959\\
0.285	0.529951380947623	0.882640524634436	0.642417151055156\\
0.285	0.544300728313782	0.890133795440129	0.638329210947586\\
0.285	0.558732572247415	0.897375649995373	0.634187835387497\\
0.285	0.57324050169555	0.904365153357279	0.629991194998376\\
0.285	0.587818113093873	0.911101581661729	0.625738868900925\\
0.285	0.602459018746568	0.917584420216504	0.62143143163314\\
0.285	0.617156855035314	0.923813361211857	0.617070132480553\\
0.285	0.631905290438127	0.92978830106243	0.612656657833838\\
0.285	0.6466980333399	0.935509337395467	0.608192963377086\\
0.285	0.661528839617671	0.940976765701173	0.603681161495105\\
0.285	0.676391519984904	0.946191075661951	0.599123449738609\\
0.285	0.691279947080357	0.951152947177946	0.594522067901588\\
0.285	0.706188062288412	0.955863246106974	0.589879273662389\\
0.285	0.721109882279076	0.960323019737426	0.585197329319937\\
0.285	0.736039505257226	0.964533492013186	0.580478494543975\\
0.285	0.75097111691199	0.968496058529893	0.575725022018314\\
0.285	0.765898996058536	0.972212281322139	0.570939154289561\\
0.285	0.780817519965826	0.975683883461272	0.566123121053173\\
0.285	0.795721169365275	0.978912743483559	0.56127913660556\\
0.285	0.810604533136476	0.981900889668376	0.5564093973992\\
0.285	0.825462312667457	0.984650494185936	0.551516079695735\\
0.285	0.840289325888138	0.987163867133889	0.546601337319641\\
0.285	0.855080510976839	0.989443450481767	0.541667299515139\\
0.285	0.86983092974082	0.991491811941914	0.536716068908598\\
0.285	0.884535770672926	0.993311638785082	0.531749719578143\\
0.285	0.899190351687411	0.994905731618374	0.526770295231792\\
0.285	0.913790122539012	0.996276998142666	0.521779807495102\\
0.285	0.928330666930242	0.997428446906011	0.516780234308686\\
0.285	0.942807704312699	0.998363181068902	0.511773518435872\\
0.285	0.95721709138901	0.999084392196567	0.506761566080167\\
0.285	0.971554823322703	0.999595354092743	0.501746245611874\\
0.285	0.985817034663988	0.999899416688637	0.496729386403004\\
0.285	1	1	0.491669087346906\\
0.3	0	0	0.710195569525497\\
0.3	0.000100583311362513	0.0141829653360114	0.71240460418299\\
0.3	0.000404645907256436	0.0284451766772965	0.714630690536905\\
0.3	0.000915607803432999	0.0427829086109896	0.716916312064106\\
0.3	0.00163681893109844	0.057192295687301	0.719289019360234\\
0.3	0.00257155309398959	0.0716693330697584	0.721780367061016\\
0.3	0.00372300185733413	0.0862098774609879	0.724425655492835\\
0.3	0.00509426838162598	0.100809648312589	0.727263413850304\\
0.3	0.00668836121491816	0.115464229327074	0.73033458997498\\
0.3	0.00850818805808555	0.13016907025918	0.733681422514709\\
0.3	0.0105565495182326	0.144919489023162	0.737345986077549\\
0.3	0.0128361328661109	0.159710674111862	0.741368418705642\\
0.3	0.0153495058140643	0.174537687332543	0.745784862958201\\
0.3	0.0180991103316243	0.189395466863524	0.750625176092311\\
0.3	0.0210872565164405	0.204278830634725	0.755910489868999\\
0.3	0.0243161165387281	0.219182480034174	0.761650724664453\\
0.3	0.0277877186778607	0.234101003941464	0.767842183867615\\
0.3	0.0315039414701067	0.24902888308801	0.77446537092041\\
0.3	0.0354665079868145	0.263960494742775	0.781483180780606\\
0.3	0.0396769802625738	0.278890117720924	0.788839618268073\\
0.3	0.0441367538930258	0.293811937711588	0.796459186325543\\
0.3	0.0488470528220538	0.308720052919643	0.804247066923454\\
0.3	0.0538089243380495	0.323608480015097	0.812090186161949\\
0.3	0.0590232342988274	0.338471160382329	0.819859213932695\\
0.3	0.064490662604533	0.3533019666601	0.827411499063638\\
0.3	0.0702116989375697	0.368094709561873	0.834594885813502\\
0.3	0.0761866387881432	0.382843144964686	0.84125230028718\\
0.3	0.0824155797834956	0.397540981253432	0.847226939729083\\
0.3	0.0888984183382709	0.412181886906127	0.852367847905041\\
0.3	0.0956348466427212	0.42675949830445	0.85653562002783\\
0.3	0.102624350004627	0.441267427752584	0.859607954619618\\
0.3	0.109866204559871	0.455699271686218	0.861484760266619\\
0.3	0.117359475365564	0.470048619052377	0.862092534251537\\
0.3	0.125103014888515	0.484309059839721	0.861387758068372\\
0.3	0.133095461900593	0.498474193737904	0.859359100890821\\
0.3	0.14133524079124	0.512537638903683	0.856028283765736\\
0.3	0.149820561306021	0.526493040810599	0.851449530870182\\
0.3	0.158549418718625	0.540334081158348	0.84570761473951\\
0.3	0.167519594442214	0.554054486817265	0.838914584343153\\
0.3	0.176728657084455	0.567648038782867	0.831205342370835\\
0.3	0.186173963948925	0.581108581114919	0.822732305417525\\
0.3	0.195852662983903	0.594430029835236	0.813659432944305\\
0.3	0.205761695177907	0.607606381758231	0.804155944104868\\
0.3	0.215897797399558	0.620631723228143	0.794390053406968\\
0.3	0.226257505677663	0.633500238737009	0.784523046062798\\
0.3	0.236837158915675	0.646206219397571	0.77470398286159\\
0.3	0.247632903032949	0.658744071245709	0.765065275211017\\
0.3	0.258640695523527	0.671108323347402	0.755719307772963\\
0.3	0.269856310421543	0.683293635685828	0.746756214005911\\
0.3	0.28127534366066	0.695294806804924	0.738242834604295\\
0.3	0.292893218813452	0.707106781186547	0.730222816007835\\
0.3	0.304705193195075	0.71872465633934	0.722717741087766\\
0.3	0.316706364314172	0.730143689578457	0.715729131143268\\
0.3	0.328891676652598	0.741359304476472	0.709241120561857\\
0.3	0.341255928754291	0.752367096967051	0.703223584571264\\
0.3	0.353793780602429	0.763162841084324	0.697635496597418\\
0.3	0.366499761262991	0.773742494322337	0.69242830359462\\
0.3	0.379368276771857	0.784102202600443	0.687549132887069\\
0.3	0.392393618241769	0.794238304822092	0.682943679235328\\
0.3	0.405569970164763	0.804147337016096	0.678558662193044\\
0.3	0.418891418885081	0.813826036051075	0.674343787412872\\
0.3	0.432351961217133	0.823271342915545	0.670253187715551\\
0.3	0.445945513182735	0.832480405557786	0.666246357336158\\
0.3	0.459665918841652	0.841450581281375	0.662288623474186\\
0.3	0.473506959189401	0.850179438693979	0.658351221673191\\
0.3	0.487462361096317	0.85866475920876	0.654411055137886\\
0.3	0.501525806262096	0.866904538099407	0.650450223206469\\
0.3	0.515690940160279	0.874896985111485	0.646455401874958\\
0.3	0.529951380947623	0.882640524634436	0.642417151055156\\
0.3	0.544300728313781	0.890133795440129	0.638329210947588\\
0.3	0.558732572247415	0.897375649995373	0.634187835387498\\
0.3	0.57324050169555	0.904365153357279	0.629991194998375\\
0.3	0.587818113093873	0.911101581661729	0.625738868900924\\
0.3	0.602459018746568	0.917584420216504	0.62143143163314\\
0.3	0.617156855035314	0.923813361211857	0.617070132480553\\
0.3	0.631905290438127	0.92978830106243	0.612656657833838\\
0.3	0.6466980333399	0.935509337395467	0.608192963377088\\
0.3	0.661528839617671	0.940976765701173	0.603681161495103\\
0.3	0.676391519984904	0.94619107566195	0.599123449738607\\
0.3	0.691279947080357	0.951152947177946	0.59452206790159\\
0.3	0.706188062288412	0.955863246106974	0.589879273662389\\
0.3	0.721109882279076	0.960323019737426	0.585197329319937\\
0.3	0.736039505257226	0.964533492013186	0.580478494543977\\
0.3	0.75097111691199	0.968496058529893	0.575725022018311\\
0.3	0.765898996058536	0.972212281322139	0.570939154289557\\
0.3	0.780817519965826	0.975683883461272	0.566123121053175\\
0.3	0.795721169365275	0.97891274348356	0.561279136605561\\
0.3	0.810604533136476	0.981900889668376	0.556409397399198\\
0.3	0.825462312667457	0.984650494185936	0.551516079695733\\
0.3	0.840289325888138	0.987163867133889	0.546601337319641\\
0.3	0.855080510976838	0.989443450481767	0.541667299515142\\
0.3	0.86983092974082	0.991491811941915	0.536716068908598\\
0.3	0.884535770672926	0.993311638785082	0.531749719578143\\
0.3	0.899190351687411	0.994905731618374	0.526770295231794\\
0.3	0.913790122539012	0.996276998142666	0.521779807495098\\
0.3	0.928330666930241	0.99742844690601	0.516780234308683\\
0.3	0.942807704312699	0.998363181068902	0.511773518435876\\
0.3	0.95721709138901	0.999084392196567	0.506761566080171\\
0.3	0.971554823322703	0.999595354092744	0.501746245611873\\
0.3	0.985817034663989	0.999899416688637	0.496729386403002\\
0.3	1	1	0.491669087346908\\
0.315	0	0	0.710195569525497\\
0.315	0.000100583311362513	0.0141829653360114	0.71240460418299\\
0.315	0.000404645907256436	0.0284451766772965	0.714630690536905\\
0.315	0.000915607803432999	0.0427829086109896	0.716916312064106\\
0.315	0.00163681893109844	0.057192295687301	0.719289019360234\\
0.315	0.00257155309398959	0.0716693330697584	0.721780367061016\\
0.315	0.00372300185733413	0.0862098774609879	0.724425655492835\\
0.315	0.00509426838162598	0.100809648312589	0.727263413850304\\
0.315	0.00668836121491816	0.115464229327074	0.73033458997498\\
0.315	0.00850818805808555	0.13016907025918	0.733681422514709\\
0.315	0.0105565495182326	0.144919489023162	0.737345986077549\\
0.315	0.0128361328661109	0.159710674111862	0.741368418705642\\
0.315	0.0153495058140643	0.174537687332543	0.745784862958201\\
0.315	0.0180991103316243	0.189395466863524	0.750625176092311\\
0.315	0.0210872565164405	0.204278830634725	0.755910489868999\\
0.315	0.0243161165387281	0.219182480034174	0.761650724664453\\
0.315	0.0277877186778607	0.234101003941464	0.767842183867616\\
0.315	0.0315039414701067	0.24902888308801	0.77446537092041\\
0.315	0.0354665079868145	0.263960494742775	0.781483180780607\\
0.315	0.0396769802625738	0.278890117720924	0.788839618268073\\
0.315	0.0441367538930258	0.293811937711588	0.796459186325544\\
0.315	0.0488470528220538	0.308720052919643	0.804247066923455\\
0.315	0.0538089243380495	0.323608480015096	0.81209018616195\\
0.315	0.0590232342988274	0.338471160382329	0.819859213932694\\
0.315	0.064490662604533	0.3533019666601	0.82741149906364\\
0.315	0.0702116989375697	0.368094709561873	0.834594885813503\\
0.315	0.0761866387881432	0.382843144964686	0.84125230028718\\
0.315	0.0824155797834956	0.397540981253432	0.847226939729082\\
0.315	0.0888984183382709	0.412181886906127	0.852367847905041\\
0.315	0.0956348466427212	0.42675949830445	0.856535620027829\\
0.315	0.102624350004627	0.441267427752584	0.859607954619618\\
0.315	0.109866204559871	0.455699271686218	0.861484760266622\\
0.315	0.117359475365564	0.470048619052377	0.86209253425154\\
0.315	0.125103014888515	0.484309059839721	0.861387758068368\\
0.315	0.133095461900593	0.498474193737904	0.859359100890821\\
0.315	0.14133524079124	0.512537638903683	0.856028283765738\\
0.315	0.149820561306021	0.526493040810599	0.85144953087018\\
0.315	0.158549418718625	0.540334081158348	0.845707614739506\\
0.315	0.167519594442214	0.554054486817265	0.838914584343149\\
0.315	0.176728657084455	0.567648038782867	0.831205342370834\\
0.315	0.186173963948925	0.581108581114919	0.822732305417528\\
0.315	0.195852662983903	0.594430029835236	0.813659432944306\\
0.315	0.205761695177907	0.60760638175823	0.804155944104867\\
0.315	0.215897797399558	0.620631723228143	0.79439005340697\\
0.315	0.226257505677663	0.633500238737009	0.7845230460628\\
0.315	0.236837158915675	0.646206219397571	0.774703982861594\\
0.315	0.247632903032949	0.658744071245709	0.765065275211018\\
0.315	0.258640695523527	0.671108323347402	0.755719307772963\\
0.315	0.269856310421543	0.683293635685828	0.746756214005912\\
0.315	0.28127534366066	0.695294806804924	0.738242834604297\\
0.315	0.292893218813452	0.707106781186547	0.730222816007834\\
0.315	0.304705193195075	0.71872465633934	0.722717741087768\\
0.315	0.316706364314172	0.730143689578457	0.715729131143269\\
0.315	0.328891676652598	0.741359304476472	0.709241120561857\\
0.315	0.341255928754291	0.752367096967051	0.703223584571264\\
0.315	0.353793780602429	0.763162841084325	0.697635496597416\\
0.315	0.366499761262991	0.773742494322337	0.69242830359462\\
0.315	0.379368276771857	0.784102202600443	0.68754913288707\\
0.315	0.392393618241769	0.794238304822092	0.682943679235328\\
0.315	0.405569970164763	0.804147337016097	0.678558662193046\\
0.315	0.418891418885081	0.813826036051075	0.674343787412872\\
0.315	0.432351961217132	0.823271342915545	0.670253187715551\\
0.315	0.445945513182735	0.832480405557786	0.666246357336157\\
0.315	0.459665918841652	0.841450581281375	0.662288623474186\\
0.315	0.473506959189401	0.850179438693979	0.65835122167319\\
0.315	0.487462361096317	0.85866475920876	0.654411055137886\\
0.315	0.501525806262096	0.866904538099407	0.650450223206469\\
0.315	0.515690940160279	0.874896985111485	0.646455401874957\\
0.315	0.529951380947623	0.882640524634436	0.642417151055154\\
0.315	0.544300728313781	0.890133795440129	0.638329210947586\\
0.315	0.558732572247415	0.897375649995372	0.6341878353875\\
0.315	0.57324050169555	0.904365153357279	0.629991194998377\\
0.315	0.587818113093873	0.911101581661729	0.625738868900923\\
0.315	0.602459018746568	0.917584420216504	0.621431431633141\\
0.315	0.617156855035314	0.923813361211857	0.617070132480553\\
0.315	0.631905290438127	0.92978830106243	0.612656657833837\\
0.315	0.6466980333399	0.935509337395467	0.608192963377089\\
0.315	0.661528839617671	0.940976765701173	0.603681161495106\\
0.315	0.676391519984904	0.946191075661951	0.599123449738607\\
0.315	0.691279947080357	0.951152947177946	0.594522067901589\\
0.315	0.706188062288412	0.955863246106974	0.58987927366239\\
0.315	0.721109882279076	0.960323019737426	0.585197329319937\\
0.315	0.736039505257225	0.964533492013186	0.580478494543977\\
0.315	0.75097111691199	0.968496058529893	0.575725022018312\\
0.315	0.765898996058536	0.972212281322139	0.570939154289559\\
0.315	0.780817519965826	0.975683883461272	0.566123121053175\\
0.315	0.795721169365275	0.97891274348356	0.561279136605558\\
0.315	0.810604533136476	0.981900889668376	0.5564093973992\\
0.315	0.825462312667457	0.984650494185936	0.551516079695735\\
0.315	0.840289325888138	0.987163867133889	0.54660133731964\\
0.315	0.855080510976839	0.989443450481767	0.541667299515139\\
0.315	0.86983092974082	0.991491811941915	0.5367160689086\\
0.315	0.884535770672926	0.993311638785082	0.531749719578142\\
0.315	0.899190351687411	0.994905731618374	0.526770295231793\\
0.315	0.913790122539012	0.996276998142666	0.521779807495101\\
0.315	0.928330666930242	0.997428446906011	0.516780234308682\\
0.315	0.942807704312699	0.998363181068902	0.511773518435872\\
0.315	0.95721709138901	0.999084392196567	0.506761566080173\\
0.315	0.971554823322704	0.999595354092744	0.501746245611875\\
0.315	0.985817034663988	0.999899416688637	0.496729386403\\
0.315	1	1	0.49166908734691\\
0.33	0	0	0.710195569525497\\
0.33	0.000100583311362513	0.0141829653360114	0.71240460418299\\
0.33	0.000404645907256436	0.0284451766772965	0.714630690536905\\
0.33	0.000915607803432999	0.0427829086109896	0.716916312064106\\
0.33	0.00163681893109844	0.057192295687301	0.719289019360234\\
0.33	0.00257155309398959	0.0716693330697584	0.721780367061016\\
0.33	0.00372300185733413	0.0862098774609879	0.724425655492835\\
0.33	0.00509426838162598	0.100809648312589	0.727263413850304\\
0.33	0.00668836121491816	0.115464229327074	0.73033458997498\\
0.33	0.00850818805808555	0.13016907025918	0.733681422514709\\
0.33	0.0105565495182326	0.144919489023162	0.737345986077549\\
0.33	0.0128361328661109	0.159710674111862	0.741368418705642\\
0.33	0.0153495058140643	0.174537687332543	0.745784862958202\\
0.33	0.0180991103316243	0.189395466863524	0.750625176092311\\
0.33	0.0210872565164405	0.204278830634725	0.755910489868999\\
0.33	0.0243161165387281	0.219182480034174	0.761650724664453\\
0.33	0.0277877186778607	0.234101003941464	0.767842183867616\\
0.33	0.0315039414701067	0.24902888308801	0.77446537092041\\
0.33	0.0354665079868146	0.263960494742775	0.781483180780606\\
0.33	0.0396769802625738	0.278890117720924	0.788839618268074\\
0.33	0.0441367538930258	0.293811937711588	0.796459186325544\\
0.33	0.0488470528220538	0.308720052919643	0.804247066923455\\
0.33	0.0538089243380495	0.323608480015097	0.812090186161951\\
0.33	0.0590232342988274	0.338471160382329	0.819859213932694\\
0.33	0.064490662604533	0.3533019666601	0.82741149906364\\
0.33	0.0702116989375697	0.368094709561873	0.834594885813506\\
0.33	0.0761866387881432	0.382843144964686	0.841252300287178\\
0.33	0.0824155797834956	0.397540981253432	0.84722693972908\\
0.33	0.0888984183382709	0.412181886906127	0.852367847905041\\
0.33	0.0956348466427212	0.42675949830445	0.856535620027828\\
0.33	0.102624350004627	0.441267427752584	0.859607954619615\\
0.33	0.109866204559871	0.455699271686218	0.861484760266624\\
0.33	0.117359475365564	0.470048619052377	0.862092534251541\\
0.33	0.125103014888515	0.484309059839721	0.861387758068369\\
0.33	0.133095461900593	0.498474193737904	0.859359100890819\\
0.33	0.14133524079124	0.512537638903683	0.856028283765734\\
0.33	0.149820561306021	0.526493040810599	0.851449530870184\\
0.33	0.158549418718625	0.540334081158348	0.845707614739505\\
0.33	0.167519594442214	0.554054486817265	0.838914584343147\\
0.33	0.176728657084455	0.567648038782867	0.83120534237084\\
0.33	0.186173963948925	0.581108581114919	0.822732305417531\\
0.33	0.195852662983903	0.594430029835236	0.813659432944304\\
0.33	0.205761695177907	0.607606381758231	0.804155944104873\\
0.33	0.215897797399558	0.620631723228143	0.794390053406972\\
0.33	0.226257505677663	0.633500238737009	0.784523046062805\\
0.33	0.236837158915675	0.646206219397571	0.774703982861592\\
0.33	0.247632903032949	0.658744071245709	0.765065275211015\\
0.33	0.258640695523527	0.671108323347402	0.755719307772961\\
0.33	0.269856310421543	0.683293635685828	0.746756214005912\\
0.33	0.28127534366066	0.695294806804925	0.738242834604295\\
0.33	0.292893218813452	0.707106781186547	0.730222816007831\\
0.33	0.304705193195076	0.71872465633934	0.722717741087768\\
0.33	0.316706364314172	0.730143689578457	0.715729131143269\\
0.33	0.328891676652598	0.741359304476472	0.709241120561856\\
0.33	0.341255928754291	0.752367096967051	0.703223584571264\\
0.33	0.353793780602429	0.763162841084325	0.697635496597417\\
0.33	0.366499761262991	0.773742494322337	0.692428303594619\\
0.33	0.379368276771856	0.784102202600442	0.68754913288707\\
0.33	0.392393618241769	0.794238304822092	0.682943679235329\\
0.33	0.405569970164763	0.804147337016097	0.678558662193045\\
0.33	0.418891418885081	0.813826036051075	0.67434378741287\\
0.33	0.432351961217133	0.823271342915544	0.670253187715553\\
0.33	0.445945513182735	0.832480405557786	0.666246357336159\\
0.33	0.459665918841652	0.841450581281375	0.662288623474186\\
0.33	0.473506959189401	0.850179438693979	0.658351221673192\\
0.33	0.487462361096317	0.85866475920876	0.654411055137886\\
0.33	0.501525806262096	0.866904538099407	0.650450223206468\\
0.33	0.515690940160279	0.874896985111485	0.646455401874959\\
0.33	0.529951380947623	0.882640524634437	0.642417151055154\\
0.33	0.544300728313782	0.890133795440129	0.638329210947586\\
0.33	0.558732572247415	0.897375649995373	0.634187835387498\\
0.33	0.57324050169555	0.904365153357279	0.629991194998376\\
0.33	0.587818113093873	0.911101581661729	0.625738868900923\\
0.33	0.602459018746568	0.917584420216504	0.621431431633139\\
0.33	0.617156855035314	0.923813361211857	0.617070132480553\\
0.33	0.631905290438127	0.92978830106243	0.612656657833837\\
0.33	0.6466980333399	0.935509337395467	0.608192963377089\\
0.33	0.661528839617671	0.940976765701173	0.603681161495107\\
0.33	0.676391519984904	0.946191075661951	0.599123449738606\\
0.33	0.691279947080357	0.951152947177946	0.594522067901588\\
0.33	0.706188062288412	0.955863246106974	0.589879273662391\\
0.33	0.721109882279076	0.960323019737426	0.585197329319937\\
0.33	0.736039505257225	0.964533492013186	0.580478494543977\\
0.33	0.75097111691199	0.968496058529893	0.575725022018313\\
0.33	0.765898996058536	0.972212281322139	0.570939154289558\\
0.33	0.780817519965826	0.975683883461272	0.566123121053174\\
0.33	0.795721169365275	0.97891274348356	0.561279136605558\\
0.33	0.810604533136476	0.981900889668376	0.5564093973992\\
0.33	0.825462312667457	0.984650494185936	0.551516079695736\\
0.33	0.840289325888138	0.987163867133889	0.546601337319641\\
0.33	0.855080510976838	0.989443450481767	0.541667299515139\\
0.33	0.86983092974082	0.991491811941914	0.536716068908599\\
0.33	0.884535770672926	0.993311638785082	0.531749719578144\\
0.33	0.899190351687411	0.994905731618374	0.526770295231793\\
0.33	0.913790122539012	0.996276998142666	0.5217798074951\\
0.33	0.928330666930242	0.997428446906011	0.516780234308684\\
0.33	0.942807704312699	0.998363181068901	0.511773518435872\\
0.33	0.95721709138901	0.999084392196567	0.506761566080168\\
0.33	0.971554823322703	0.999595354092743	0.501746245611875\\
0.33	0.985817034663989	0.999899416688637	0.496729386403005\\
0.33	1	1	0.491669087346904\\
0.345	0	0	0.710195569525497\\
0.345	0.000100583311362513	0.0141829653360114	0.71240460418299\\
0.345	0.000404645907256436	0.0284451766772965	0.714630690536905\\
0.345	0.000915607803432999	0.0427829086109896	0.716916312064106\\
0.345	0.00163681893109844	0.057192295687301	0.719289019360234\\
0.345	0.00257155309398959	0.0716693330697584	0.721780367061016\\
0.345	0.00372300185733414	0.0862098774609879	0.724425655492835\\
0.345	0.00509426838162598	0.100809648312589	0.727263413850304\\
0.345	0.00668836121491816	0.115464229327074	0.73033458997498\\
0.345	0.00850818805808555	0.13016907025918	0.733681422514709\\
0.345	0.0105565495182326	0.144919489023162	0.737345986077549\\
0.345	0.0128361328661109	0.159710674111862	0.741368418705642\\
0.345	0.0153495058140643	0.174537687332543	0.745784862958202\\
0.345	0.0180991103316243	0.189395466863524	0.750625176092311\\
0.345	0.0210872565164405	0.204278830634725	0.755910489868999\\
0.345	0.0243161165387281	0.219182480034174	0.761650724664453\\
0.345	0.0277877186778607	0.234101003941464	0.767842183867616\\
0.345	0.0315039414701067	0.24902888308801	0.77446537092041\\
0.345	0.0354665079868145	0.263960494742775	0.781483180780605\\
0.345	0.0396769802625738	0.278890117720924	0.788839618268073\\
0.345	0.0441367538930258	0.293811937711588	0.796459186325544\\
0.345	0.0488470528220538	0.308720052919643	0.804247066923455\\
0.345	0.0538089243380495	0.323608480015096	0.81209018616195\\
0.345	0.0590232342988274	0.338471160382329	0.819859213932694\\
0.345	0.064490662604533	0.3533019666601	0.827411499063639\\
0.345	0.0702116989375697	0.368094709561873	0.834594885813506\\
0.345	0.0761866387881432	0.382843144964686	0.841252300287179\\
0.345	0.0824155797834956	0.397540981253432	0.847226939729081\\
0.345	0.0888984183382709	0.412181886906127	0.852367847905041\\
0.345	0.0956348466427212	0.42675949830445	0.856535620027828\\
0.345	0.102624350004627	0.441267427752584	0.859607954619617\\
0.345	0.109866204559871	0.455699271686218	0.86148476026662\\
0.345	0.117359475365564	0.470048619052377	0.862092534251539\\
0.345	0.125103014888515	0.484309059839721	0.861387758068368\\
0.345	0.133095461900593	0.498474193737904	0.859359100890817\\
0.345	0.14133524079124	0.512537638903683	0.856028283765733\\
0.345	0.149820561306021	0.526493040810599	0.851449530870182\\
0.345	0.158549418718625	0.540334081158348	0.845707614739501\\
0.345	0.167519594442213	0.554054486817265	0.83891458434315\\
0.345	0.176728657084455	0.567648038782867	0.831205342370839\\
0.345	0.186173963948925	0.581108581114919	0.822732305417525\\
0.345	0.195852662983903	0.594430029835236	0.813659432944309\\
0.345	0.205761695177907	0.607606381758231	0.804155944104873\\
0.345	0.215897797399558	0.620631723228143	0.79439005340697\\
0.345	0.226257505677663	0.633500238737009	0.784523046062801\\
0.345	0.236837158915675	0.646206219397571	0.774703982861588\\
0.345	0.247632903032949	0.658744071245709	0.765065275211014\\
0.345	0.258640695523527	0.671108323347402	0.755719307772961\\
0.345	0.269856310421543	0.683293635685828	0.74675621400591\\
0.345	0.28127534366066	0.695294806804924	0.738242834604294\\
0.345	0.292893218813452	0.707106781186547	0.730222816007835\\
0.345	0.304705193195075	0.71872465633934	0.722717741087769\\
0.345	0.316706364314172	0.730143689578457	0.715729131143268\\
0.345	0.328891676652598	0.741359304476472	0.709241120561855\\
0.345	0.341255928754291	0.752367096967051	0.703223584571265\\
0.345	0.353793780602429	0.763162841084325	0.697635496597418\\
0.345	0.366499761262991	0.773742494322337	0.692428303594618\\
0.345	0.379368276771856	0.784102202600442	0.68754913288707\\
0.345	0.392393618241769	0.794238304822092	0.682943679235329\\
0.345	0.405569970164763	0.804147337016097	0.678558662193044\\
0.345	0.418891418885081	0.813826036051075	0.67434378741287\\
0.345	0.432351961217132	0.823271342915544	0.670253187715552\\
0.345	0.445945513182735	0.832480405557786	0.66624635733616\\
0.345	0.459665918841652	0.841450581281375	0.662288623474186\\
0.345	0.473506959189401	0.850179438693979	0.658351221673191\\
0.345	0.487462361096317	0.85866475920876	0.654411055137887\\
0.345	0.501525806262096	0.866904538099407	0.650450223206468\\
0.345	0.515690940160279	0.874896985111485	0.646455401874959\\
0.345	0.529951380947623	0.882640524634436	0.642417151055155\\
0.345	0.544300728313782	0.890133795440129	0.638329210947586\\
0.345	0.558732572247415	0.897375649995373	0.634187835387499\\
0.345	0.57324050169555	0.904365153357279	0.629991194998375\\
0.345	0.587818113093873	0.911101581661729	0.625738868900924\\
0.345	0.602459018746568	0.917584420216504	0.62143143163314\\
0.345	0.617156855035314	0.923813361211857	0.617070132480554\\
0.345	0.631905290438127	0.92978830106243	0.612656657833836\\
0.345	0.6466980333399	0.935509337395467	0.608192963377088\\
0.345	0.661528839617671	0.940976765701173	0.603681161495106\\
0.345	0.676391519984903	0.94619107566195	0.599123449738608\\
0.345	0.691279947080357	0.951152947177946	0.594522067901589\\
0.345	0.706188062288412	0.955863246106974	0.589879273662391\\
0.345	0.721109882279076	0.960323019737426	0.585197329319937\\
0.345	0.736039505257225	0.964533492013186	0.580478494543976\\
0.345	0.75097111691199	0.968496058529893	0.575725022018312\\
0.345	0.765898996058536	0.972212281322139	0.570939154289558\\
0.345	0.780817519965826	0.975683883461272	0.566123121053175\\
0.345	0.795721169365275	0.97891274348356	0.561279136605559\\
0.345	0.810604533136476	0.981900889668376	0.556409397399199\\
0.345	0.825462312667457	0.984650494185936	0.551516079695738\\
0.345	0.840289325888138	0.987163867133889	0.546601337319641\\
0.345	0.855080510976839	0.989443450481767	0.541667299515138\\
0.345	0.86983092974082	0.991491811941914	0.536716068908599\\
0.345	0.884535770672926	0.993311638785082	0.531749719578143\\
0.345	0.899190351687411	0.994905731618374	0.526770295231795\\
0.345	0.913790122539012	0.996276998142666	0.521779807495101\\
0.345	0.928330666930242	0.99742844690601	0.516780234308682\\
0.345	0.942807704312699	0.998363181068901	0.511773518435874\\
0.345	0.95721709138901	0.999084392196567	0.50676156608017\\
0.345	0.971554823322704	0.999595354092743	0.501746245611873\\
0.345	0.985817034663989	0.999899416688637	0.496729386403004\\
0.345	1	1	0.491669087346906\\
0.36	0	0	0.710195569525497\\
0.36	0.000100583311362513	0.0141829653360114	0.71240460418299\\
0.36	0.000404645907256436	0.0284451766772965	0.714630690536905\\
0.36	0.000915607803432999	0.0427829086109896	0.716916312064106\\
0.36	0.00163681893109844	0.057192295687301	0.719289019360234\\
0.36	0.00257155309398959	0.0716693330697584	0.721780367061016\\
0.36	0.00372300185733413	0.0862098774609879	0.724425655492835\\
0.36	0.00509426838162598	0.100809648312589	0.727263413850304\\
0.36	0.00668836121491816	0.115464229327074	0.73033458997498\\
0.36	0.00850818805808555	0.13016907025918	0.733681422514709\\
0.36	0.0105565495182326	0.144919489023162	0.737345986077549\\
0.36	0.0128361328661109	0.159710674111862	0.741368418705642\\
0.36	0.0153495058140643	0.174537687332543	0.745784862958201\\
0.36	0.0180991103316243	0.189395466863524	0.750625176092311\\
0.36	0.0210872565164405	0.204278830634725	0.755910489868999\\
0.36	0.0243161165387281	0.219182480034174	0.761650724664453\\
0.36	0.0277877186778607	0.234101003941464	0.767842183867616\\
0.36	0.0315039414701067	0.24902888308801	0.77446537092041\\
0.36	0.0354665079868145	0.263960494742775	0.781483180780606\\
0.36	0.0396769802625738	0.278890117720924	0.788839618268073\\
0.36	0.0441367538930258	0.293811937711588	0.796459186325544\\
0.36	0.0488470528220538	0.308720052919643	0.804247066923455\\
0.36	0.0538089243380495	0.323608480015096	0.81209018616195\\
0.36	0.0590232342988274	0.338471160382329	0.819859213932694\\
0.36	0.064490662604533	0.3533019666601	0.827411499063639\\
0.36	0.0702116989375697	0.368094709561873	0.834594885813504\\
0.36	0.0761866387881432	0.382843144964686	0.84125230028718\\
0.36	0.0824155797834956	0.397540981253432	0.847226939729081\\
0.36	0.0888984183382709	0.412181886906127	0.852367847905041\\
0.36	0.0956348466427212	0.42675949830445	0.856535620027829\\
0.36	0.102624350004627	0.441267427752584	0.859607954619619\\
0.36	0.109866204559871	0.455699271686218	0.86148476026662\\
0.36	0.117359475365564	0.470048619052377	0.862092534251543\\
0.36	0.125103014888515	0.484309059839721	0.861387758068369\\
0.36	0.133095461900593	0.498474193737904	0.859359100890822\\
0.36	0.14133524079124	0.512537638903683	0.856028283765739\\
0.36	0.149820561306021	0.526493040810599	0.85144953087018\\
0.36	0.158549418718625	0.540334081158348	0.845707614739505\\
0.36	0.167519594442214	0.554054486817266	0.838914584343156\\
0.36	0.176728657084455	0.567648038782867	0.831205342370837\\
0.36	0.186173963948925	0.581108581114919	0.822732305417525\\
0.36	0.195852662983903	0.594430029835237	0.813659432944306\\
0.36	0.205761695177907	0.607606381758231	0.804155944104869\\
0.36	0.215897797399558	0.620631723228143	0.794390053406968\\
0.36	0.226257505677663	0.633500238737009	0.784523046062802\\
0.36	0.236837158915675	0.646206219397571	0.774703982861589\\
0.36	0.247632903032949	0.658744071245709	0.765065275211016\\
0.36	0.258640695523528	0.671108323347402	0.755719307772963\\
0.36	0.269856310421543	0.683293635685828	0.746756214005914\\
0.36	0.28127534366066	0.695294806804925	0.738242834604295\\
0.36	0.292893218813452	0.707106781186547	0.730222816007834\\
0.36	0.304705193195076	0.71872465633934	0.722717741087767\\
0.36	0.316706364314172	0.730143689578457	0.715729131143268\\
0.36	0.328891676652598	0.741359304476472	0.709241120561856\\
0.36	0.341255928754291	0.752367096967051	0.703223584571266\\
0.36	0.353793780602429	0.763162841084325	0.697635496597419\\
0.36	0.366499761262991	0.773742494322337	0.692428303594618\\
0.36	0.379368276771857	0.784102202600443	0.687549132887069\\
0.36	0.392393618241769	0.794238304822092	0.682943679235329\\
0.36	0.405569970164763	0.804147337016097	0.678558662193044\\
0.36	0.418891418885081	0.813826036051075	0.67434378741287\\
0.36	0.432351961217132	0.823271342915544	0.670253187715552\\
0.36	0.445945513182735	0.832480405557786	0.666246357336159\\
0.36	0.459665918841652	0.841450581281375	0.662288623474186\\
0.36	0.473506959189401	0.850179438693979	0.658351221673191\\
0.36	0.487462361096317	0.85866475920876	0.654411055137885\\
0.36	0.501525806262096	0.866904538099407	0.650450223206468\\
0.36	0.515690940160279	0.874896985111485	0.646455401874959\\
0.36	0.529951380947623	0.882640524634436	0.642417151055156\\
0.36	0.544300728313782	0.890133795440129	0.638329210947585\\
0.36	0.558732572247415	0.897375649995372	0.634187835387499\\
0.36	0.57324050169555	0.904365153357279	0.629991194998378\\
0.36	0.587818113093873	0.911101581661729	0.625738868900923\\
0.36	0.602459018746568	0.917584420216505	0.621431431633139\\
0.36	0.617156855035314	0.923813361211857	0.617070132480553\\
0.36	0.631905290438127	0.92978830106243	0.612656657833837\\
0.36	0.6466980333399	0.935509337395467	0.608192963377088\\
0.36	0.661528839617671	0.940976765701173	0.603681161495106\\
0.36	0.676391519984904	0.94619107566195	0.599123449738607\\
0.36	0.691279947080357	0.951152947177946	0.594522067901589\\
0.36	0.706188062288412	0.955863246106974	0.58987927366239\\
0.36	0.721109882279076	0.960323019737426	0.585197329319939\\
0.36	0.736039505257226	0.964533492013186	0.580478494543976\\
0.36	0.75097111691199	0.968496058529893	0.575725022018313\\
0.36	0.765898996058536	0.972212281322139	0.570939154289558\\
0.36	0.780817519965826	0.975683883461272	0.566123121053173\\
0.36	0.795721169365275	0.978912743483559	0.56127913660556\\
0.36	0.810604533136476	0.981900889668376	0.556409397399201\\
0.36	0.825462312667457	0.984650494185936	0.551516079695734\\
0.36	0.840289325888138	0.987163867133889	0.54660133731964\\
0.36	0.855080510976839	0.989443450481767	0.541667299515141\\
0.36	0.86983092974082	0.991491811941914	0.5367160689086\\
0.36	0.884535770672926	0.993311638785082	0.531749719578143\\
0.36	0.899190351687411	0.994905731618374	0.526770295231793\\
0.36	0.913790122539012	0.996276998142666	0.521779807495103\\
0.36	0.928330666930241	0.997428446906011	0.516780234308683\\
0.36	0.942807704312699	0.998363181068901	0.511773518435871\\
0.36	0.95721709138901	0.999084392196567	0.506761566080172\\
0.36	0.971554823322704	0.999595354092744	0.501746245611874\\
0.36	0.985817034663989	0.999899416688637	0.496729386403001\\
0.36	1	1	0.491669087346908\\
0.375	0	0	0.710195569525497\\
0.375	0.000100583311362513	0.0141829653360114	0.71240460418299\\
0.375	0.000404645907256436	0.0284451766772965	0.714630690536905\\
0.375	0.000915607803432999	0.0427829086109896	0.716916312064106\\
0.375	0.00163681893109844	0.057192295687301	0.719289019360234\\
0.375	0.00257155309398959	0.0716693330697584	0.721780367061016\\
0.375	0.00372300185733413	0.0862098774609879	0.724425655492835\\
0.375	0.00509426838162598	0.100809648312589	0.727263413850304\\
0.375	0.00668836121491816	0.115464229327074	0.73033458997498\\
0.375	0.00850818805808555	0.13016907025918	0.733681422514709\\
0.375	0.0105565495182326	0.144919489023162	0.737345986077549\\
0.375	0.0128361328661109	0.159710674111862	0.741368418705642\\
0.375	0.0153495058140643	0.174537687332543	0.745784862958201\\
0.375	0.0180991103316243	0.189395466863524	0.750625176092311\\
0.375	0.0210872565164405	0.204278830634725	0.755910489868999\\
0.375	0.0243161165387281	0.219182480034174	0.761650724664453\\
0.375	0.0277877186778607	0.234101003941464	0.767842183867616\\
0.375	0.0315039414701067	0.24902888308801	0.77446537092041\\
0.375	0.0354665079868145	0.263960494742775	0.781483180780606\\
0.375	0.0396769802625738	0.278890117720924	0.788839618268073\\
0.375	0.0441367538930258	0.293811937711588	0.796459186325544\\
0.375	0.0488470528220538	0.308720052919643	0.804247066923455\\
0.375	0.0538089243380495	0.323608480015096	0.812090186161951\\
0.375	0.0590232342988274	0.338471160382329	0.819859213932693\\
0.375	0.064490662604533	0.3533019666601	0.827411499063639\\
0.375	0.0702116989375697	0.368094709561873	0.834594885813504\\
0.375	0.0761866387881432	0.382843144964686	0.841252300287179\\
0.375	0.0824155797834956	0.397540981253432	0.847226939729081\\
0.375	0.0888984183382709	0.412181886906127	0.852367847905041\\
0.375	0.0956348466427212	0.42675949830445	0.856535620027829\\
0.375	0.102624350004627	0.441267427752584	0.859607954619617\\
0.375	0.109866204559871	0.455699271686218	0.861484760266621\\
0.375	0.117359475365564	0.470048619052377	0.86209253425154\\
0.375	0.125103014888515	0.484309059839721	0.86138775806837\\
0.375	0.133095461900593	0.498474193737904	0.859359100890822\\
0.375	0.14133524079124	0.512537638903683	0.856028283765738\\
0.375	0.149820561306021	0.526493040810599	0.851449530870186\\
0.375	0.158549418718625	0.540334081158348	0.84570761473951\\
0.375	0.167519594442213	0.554054486817266	0.83891458434315\\
0.375	0.176728657084455	0.567648038782867	0.831205342370837\\
0.375	0.186173963948925	0.581108581114919	0.822732305417524\\
0.375	0.195852662983903	0.594430029835236	0.813659432944306\\
0.375	0.205761695177907	0.607606381758231	0.804155944104868\\
0.375	0.215897797399558	0.620631723228143	0.794390053406969\\
0.375	0.226257505677663	0.633500238737009	0.784523046062802\\
0.375	0.236837158915675	0.646206219397571	0.774703982861589\\
0.375	0.247632903032949	0.658744071245709	0.765065275211016\\
0.375	0.258640695523528	0.671108323347402	0.755719307772962\\
0.375	0.269856310421543	0.683293635685828	0.746756214005909\\
0.375	0.28127534366066	0.695294806804924	0.738242834604294\\
0.375	0.292893218813452	0.707106781186547	0.730222816007836\\
0.375	0.304705193195075	0.71872465633934	0.722717741087768\\
0.375	0.316706364314172	0.730143689578457	0.715729131143268\\
0.375	0.328891676652598	0.741359304476472	0.709241120561856\\
0.375	0.341255928754291	0.752367096967051	0.703223584571265\\
0.375	0.353793780602429	0.763162841084325	0.697635496597419\\
0.375	0.366499761262991	0.773742494322337	0.692428303594618\\
0.375	0.379368276771857	0.784102202600442	0.687549132887069\\
0.375	0.392393618241769	0.794238304822092	0.682943679235329\\
0.375	0.405569970164763	0.804147337016097	0.678558662193044\\
0.375	0.418891418885081	0.813826036051075	0.67434378741287\\
0.375	0.432351961217132	0.823271342915544	0.670253187715551\\
0.375	0.445945513182735	0.832480405557786	0.666246357336158\\
0.375	0.459665918841652	0.841450581281375	0.662288623474188\\
0.375	0.473506959189401	0.850179438693979	0.658351221673192\\
0.375	0.487462361096317	0.85866475920876	0.654411055137886\\
0.375	0.501525806262096	0.866904538099407	0.650450223206466\\
0.375	0.515690940160279	0.874896985111485	0.646455401874956\\
0.375	0.529951380947623	0.882640524634436	0.642417151055156\\
0.375	0.544300728313782	0.890133795440129	0.638329210947589\\
0.375	0.558732572247415	0.897375649995373	0.634187835387497\\
0.375	0.57324050169555	0.904365153357279	0.629991194998375\\
0.375	0.587818113093873	0.911101581661729	0.625738868900925\\
0.375	0.602459018746568	0.917584420216505	0.62143143163314\\
0.375	0.617156855035314	0.923813361211857	0.617070132480553\\
0.375	0.631905290438127	0.92978830106243	0.612656657833837\\
0.375	0.6466980333399	0.935509337395467	0.608192963377086\\
0.375	0.661528839617671	0.940976765701173	0.603681161495106\\
0.375	0.676391519984903	0.94619107566195	0.599123449738606\\
0.375	0.691279947080357	0.951152947177946	0.59452206790159\\
0.375	0.706188062288412	0.955863246106974	0.58987927366239\\
0.375	0.721109882279076	0.960323019737426	0.585197329319937\\
0.375	0.736039505257226	0.964533492013186	0.580478494543977\\
0.375	0.75097111691199	0.968496058529893	0.575725022018312\\
0.375	0.765898996058536	0.972212281322139	0.570939154289559\\
0.375	0.780817519965826	0.975683883461272	0.566123121053173\\
0.375	0.795721169365275	0.978912743483559	0.561279136605557\\
0.375	0.810604533136476	0.981900889668376	0.556409397399201\\
0.375	0.825462312667457	0.984650494185936	0.551516079695736\\
0.375	0.840289325888138	0.987163867133889	0.546601337319637\\
0.375	0.855080510976839	0.989443450481767	0.541667299515141\\
0.375	0.86983092974082	0.991491811941915	0.536716068908601\\
0.375	0.884535770672926	0.993311638785082	0.531749719578139\\
0.375	0.899190351687411	0.994905731618374	0.526770295231796\\
0.375	0.913790122539012	0.996276998142666	0.521779807495103\\
0.375	0.928330666930242	0.997428446906011	0.516780234308683\\
0.375	0.942807704312699	0.998363181068902	0.511773518435872\\
0.375	0.95721709138901	0.999084392196567	0.506761566080168\\
0.375	0.971554823322704	0.999595354092743	0.501746245611876\\
0.375	0.985817034663989	0.999899416688638	0.496729386403004\\
0.375	1	1	0.491669087346897\\
0.39	0	0	0.710195569525497\\
0.39	0.000100583311362513	0.0141829653360114	0.71240460418299\\
0.39	0.000404645907256436	0.0284451766772965	0.714630690536905\\
0.39	0.000915607803432999	0.0427829086109896	0.716916312064106\\
0.39	0.00163681893109844	0.057192295687301	0.719289019360234\\
0.39	0.00257155309398959	0.0716693330697584	0.721780367061016\\
0.39	0.00372300185733414	0.0862098774609879	0.724425655492835\\
0.39	0.00509426838162598	0.100809648312589	0.727263413850304\\
0.39	0.00668836121491816	0.115464229327074	0.73033458997498\\
0.39	0.00850818805808555	0.13016907025918	0.733681422514709\\
0.39	0.0105565495182326	0.144919489023162	0.737345986077549\\
0.39	0.0128361328661109	0.159710674111862	0.741368418705642\\
0.39	0.0153495058140643	0.174537687332543	0.745784862958201\\
0.39	0.0180991103316243	0.189395466863524	0.750625176092311\\
0.39	0.0210872565164405	0.204278830634725	0.755910489868999\\
0.39	0.0243161165387281	0.219182480034174	0.761650724664453\\
0.39	0.0277877186778607	0.234101003941464	0.767842183867616\\
0.39	0.0315039414701067	0.24902888308801	0.77446537092041\\
0.39	0.0354665079868145	0.263960494742775	0.781483180780606\\
0.39	0.0396769802625738	0.278890117720924	0.788839618268073\\
0.39	0.0441367538930258	0.293811937711588	0.796459186325543\\
0.39	0.0488470528220537	0.308720052919643	0.804247066923455\\
0.39	0.0538089243380495	0.323608480015096	0.81209018616195\\
0.39	0.0590232342988274	0.338471160382329	0.819859213932693\\
0.39	0.064490662604533	0.3533019666601	0.82741149906364\\
0.39	0.0702116989375697	0.368094709561873	0.834594885813504\\
0.39	0.0761866387881432	0.382843144964686	0.84125230028718\\
0.39	0.0824155797834956	0.397540981253432	0.847226939729082\\
0.39	0.0888984183382709	0.412181886906127	0.852367847905042\\
0.39	0.0956348466427212	0.42675949830445	0.856535620027828\\
0.39	0.102624350004627	0.441267427752584	0.859607954619617\\
0.39	0.109866204559871	0.455699271686218	0.861484760266622\\
0.39	0.117359475365564	0.470048619052377	0.862092534251539\\
0.39	0.125103014888515	0.484309059839721	0.861387758068369\\
0.39	0.133095461900593	0.498474193737904	0.859359100890818\\
0.39	0.14133524079124	0.512537638903683	0.856028283765736\\
0.39	0.149820561306021	0.526493040810599	0.851449530870182\\
0.39	0.158549418718625	0.540334081158348	0.845707614739505\\
0.39	0.167519594442214	0.554054486817265	0.838914584343147\\
0.39	0.176728657084455	0.567648038782867	0.831205342370835\\
0.39	0.186173963948925	0.581108581114919	0.82273230541753\\
0.39	0.195852662983903	0.594430029835236	0.813659432944307\\
0.39	0.205761695177907	0.607606381758231	0.804155944104869\\
0.39	0.215897797399558	0.620631723228143	0.794390053406972\\
0.39	0.226257505677663	0.633500238737009	0.784523046062801\\
0.39	0.236837158915675	0.646206219397571	0.774703982861589\\
0.39	0.247632903032949	0.658744071245709	0.765065275211013\\
0.39	0.258640695523527	0.671108323347402	0.755719307772962\\
0.39	0.269856310421543	0.683293635685828	0.74675621400591\\
0.39	0.28127534366066	0.695294806804924	0.738242834604298\\
0.39	0.292893218813452	0.707106781186548	0.730222816007835\\
0.39	0.304705193195075	0.71872465633934	0.722717741087766\\
0.39	0.316706364314172	0.730143689578457	0.715729131143268\\
0.39	0.328891676652598	0.741359304476472	0.709241120561855\\
0.39	0.341255928754291	0.752367096967051	0.703223584571265\\
0.39	0.353793780602429	0.763162841084325	0.697635496597419\\
0.39	0.366499761262991	0.773742494322337	0.692428303594618\\
0.39	0.379368276771857	0.784102202600442	0.687549132887069\\
0.39	0.392393618241769	0.794238304822092	0.682943679235329\\
0.39	0.405569970164763	0.804147337016097	0.678558662193045\\
0.39	0.418891418885081	0.813826036051075	0.674343787412871\\
0.39	0.432351961217133	0.823271342915545	0.670253187715552\\
0.39	0.445945513182735	0.832480405557786	0.666246357336157\\
0.39	0.459665918841652	0.841450581281375	0.662288623474186\\
0.39	0.473506959189401	0.850179438693979	0.658351221673192\\
0.39	0.487462361096317	0.85866475920876	0.654411055137886\\
0.39	0.501525806262096	0.866904538099407	0.650450223206468\\
0.39	0.515690940160279	0.874896985111485	0.646455401874958\\
0.39	0.529951380947623	0.882640524634436	0.642417151055154\\
0.39	0.544300728313782	0.890133795440129	0.638329210947586\\
0.39	0.558732572247415	0.897375649995373	0.634187835387499\\
0.39	0.57324050169555	0.904365153357279	0.629991194998376\\
0.39	0.587818113093873	0.911101581661729	0.625738868900923\\
0.39	0.602459018746568	0.917584420216504	0.62143143163314\\
0.39	0.617156855035314	0.923813361211857	0.617070132480553\\
0.39	0.631905290438127	0.92978830106243	0.61265665783384\\
0.39	0.6466980333399	0.935509337395467	0.608192963377086\\
0.39	0.661528839617671	0.940976765701173	0.603681161495104\\
0.39	0.676391519984903	0.94619107566195	0.599123449738607\\
0.39	0.691279947080357	0.951152947177946	0.594522067901588\\
0.39	0.706188062288412	0.955863246106974	0.589879273662389\\
0.39	0.721109882279076	0.960323019737426	0.585197329319936\\
0.39	0.736039505257226	0.964533492013185	0.580478494543978\\
0.39	0.75097111691199	0.968496058529893	0.575725022018313\\
0.39	0.765898996058536	0.972212281322139	0.570939154289559\\
0.39	0.780817519965826	0.975683883461272	0.566123121053177\\
0.39	0.795721169365275	0.97891274348356	0.561279136605558\\
0.39	0.810604533136476	0.981900889668376	0.556409397399197\\
0.39	0.825462312667457	0.984650494185936	0.551516079695737\\
0.39	0.840289325888138	0.987163867133889	0.54660133731964\\
0.39	0.855080510976839	0.989443450481767	0.541667299515135\\
0.39	0.86983092974082	0.991491811941914	0.5367160689086\\
0.39	0.884535770672926	0.993311638785082	0.531749719578142\\
0.39	0.899190351687411	0.994905731618374	0.526770295231793\\
0.39	0.913790122539012	0.996276998142666	0.521779807495103\\
0.39	0.928330666930241	0.997428446906011	0.516780234308683\\
0.39	0.942807704312699	0.998363181068902	0.51177351843587\\
0.39	0.95721709138901	0.999084392196567	0.506761566080172\\
0.39	0.971554823322704	0.999595354092744	0.501746245611876\\
0.39	0.985817034663989	0.999899416688637	0.496729386403002\\
0.39	1	1	0.491669087346908\\
0.405	0	0	0.710195569525497\\
0.405	0.000100583311362513	0.0141829653360114	0.71240460418299\\
0.405	0.000404645907256436	0.0284451766772965	0.714630690536905\\
0.405	0.000915607803432999	0.0427829086109896	0.716916312064106\\
0.405	0.00163681893109844	0.057192295687301	0.719289019360234\\
0.405	0.00257155309398959	0.0716693330697584	0.721780367061016\\
0.405	0.00372300185733413	0.0862098774609879	0.724425655492835\\
0.405	0.00509426838162598	0.100809648312589	0.727263413850304\\
0.405	0.00668836121491816	0.115464229327074	0.73033458997498\\
0.405	0.00850818805808555	0.13016907025918	0.733681422514709\\
0.405	0.0105565495182326	0.144919489023162	0.737345986077549\\
0.405	0.0128361328661109	0.159710674111862	0.741368418705642\\
0.405	0.0153495058140643	0.174537687332543	0.745784862958201\\
0.405	0.0180991103316243	0.189395466863524	0.750625176092311\\
0.405	0.0210872565164405	0.204278830634725	0.755910489868999\\
0.405	0.0243161165387281	0.219182480034174	0.761650724664453\\
0.405	0.0277877186778607	0.234101003941464	0.767842183867616\\
0.405	0.0315039414701067	0.24902888308801	0.77446537092041\\
0.405	0.0354665079868145	0.263960494742775	0.781483180780606\\
0.405	0.0396769802625738	0.278890117720924	0.788839618268073\\
0.405	0.0441367538930258	0.293811937711588	0.796459186325544\\
0.405	0.0488470528220538	0.308720052919643	0.804247066923455\\
0.405	0.0538089243380495	0.323608480015097	0.81209018616195\\
0.405	0.0590232342988274	0.338471160382329	0.819859213932693\\
0.405	0.064490662604533	0.3533019666601	0.827411499063639\\
0.405	0.0702116989375697	0.368094709561873	0.834594885813504\\
0.405	0.0761866387881432	0.382843144964686	0.841252300287179\\
0.405	0.0824155797834956	0.397540981253432	0.847226939729081\\
0.405	0.0888984183382709	0.412181886906127	0.852367847905041\\
0.405	0.0956348466427212	0.42675949830445	0.856535620027829\\
0.405	0.102624350004627	0.441267427752585	0.859607954619618\\
0.405	0.109866204559871	0.455699271686218	0.861484760266622\\
0.405	0.117359475365564	0.470048619052377	0.862092534251542\\
0.405	0.125103014888515	0.484309059839721	0.861387758068369\\
0.405	0.133095461900593	0.498474193737904	0.859359100890819\\
0.405	0.14133524079124	0.512537638903683	0.856028283765738\\
0.405	0.149820561306021	0.526493040810599	0.851449530870183\\
0.405	0.158549418718625	0.540334081158348	0.845707614739503\\
0.405	0.167519594442214	0.554054486817265	0.83891458434315\\
0.405	0.176728657084455	0.567648038782867	0.831205342370838\\
0.405	0.186173963948925	0.581108581114919	0.822732305417529\\
0.405	0.195852662983903	0.594430029835236	0.813659432944306\\
0.405	0.205761695177907	0.607606381758231	0.804155944104871\\
0.405	0.215897797399558	0.620631723228143	0.79439005340697\\
0.405	0.226257505677663	0.633500238737009	0.784523046062802\\
0.405	0.236837158915675	0.646206219397571	0.774703982861588\\
0.405	0.247632903032949	0.658744071245709	0.765065275211014\\
0.405	0.258640695523528	0.671108323347402	0.755719307772963\\
0.405	0.269856310421543	0.683293635685828	0.746756214005914\\
0.405	0.28127534366066	0.695294806804924	0.738242834604297\\
0.405	0.292893218813452	0.707106781186547	0.730222816007834\\
0.405	0.304705193195075	0.71872465633934	0.722717741087768\\
0.405	0.316706364314172	0.730143689578457	0.715729131143267\\
0.405	0.328891676652598	0.741359304476472	0.709241120561855\\
0.405	0.341255928754291	0.752367096967051	0.703223584571266\\
0.405	0.353793780602429	0.763162841084325	0.69763549659742\\
0.405	0.366499761262991	0.773742494322337	0.692428303594619\\
0.405	0.379368276771857	0.784102202600443	0.687549132887068\\
0.405	0.392393618241769	0.794238304822092	0.682943679235329\\
0.405	0.405569970164763	0.804147337016097	0.678558662193044\\
0.405	0.418891418885081	0.813826036051075	0.67434378741287\\
0.405	0.432351961217132	0.823271342915544	0.670253187715552\\
0.405	0.445945513182735	0.832480405557786	0.666246357336159\\
0.405	0.459665918841652	0.841450581281375	0.662288623474186\\
0.405	0.473506959189401	0.850179438693979	0.658351221673191\\
0.405	0.487462361096317	0.85866475920876	0.654411055137886\\
0.405	0.501525806262096	0.866904538099407	0.650450223206468\\
0.405	0.515690940160279	0.874896985111485	0.646455401874959\\
0.405	0.529951380947623	0.882640524634437	0.642417151055155\\
0.405	0.544300728313782	0.890133795440129	0.638329210947586\\
0.405	0.558732572247415	0.897375649995373	0.634187835387496\\
0.405	0.57324050169555	0.904365153357279	0.629991194998374\\
0.405	0.587818113093873	0.911101581661729	0.625738868900925\\
0.405	0.602459018746568	0.917584420216504	0.621431431633141\\
0.405	0.617156855035314	0.923813361211857	0.617070132480553\\
0.405	0.631905290438127	0.92978830106243	0.612656657833839\\
0.405	0.6466980333399	0.935509337395467	0.608192963377087\\
0.405	0.661528839617671	0.940976765701173	0.603681161495103\\
0.405	0.676391519984904	0.94619107566195	0.599123449738609\\
0.405	0.691279947080357	0.951152947177946	0.59452206790159\\
0.405	0.706188062288412	0.955863246106974	0.58987927366239\\
0.405	0.721109882279076	0.960323019737426	0.585197329319936\\
0.405	0.736039505257226	0.964533492013185	0.580478494543976\\
0.405	0.75097111691199	0.968496058529893	0.575725022018314\\
0.405	0.765898996058536	0.972212281322139	0.570939154289558\\
0.405	0.780817519965826	0.975683883461272	0.566123121053176\\
0.405	0.795721169365275	0.97891274348356	0.56127913660556\\
0.405	0.810604533136476	0.981900889668376	0.556409397399199\\
0.405	0.825462312667457	0.984650494185936	0.551516079695736\\
0.405	0.840289325888138	0.987163867133889	0.546601337319642\\
0.405	0.855080510976839	0.989443450481767	0.541667299515138\\
0.405	0.86983092974082	0.991491811941914	0.536716068908601\\
0.405	0.884535770672926	0.993311638785082	0.531749719578143\\
0.405	0.899190351687411	0.994905731618374	0.526770295231791\\
0.405	0.913790122539012	0.996276998142666	0.521779807495102\\
0.405	0.928330666930242	0.997428446906011	0.516780234308687\\
0.405	0.942807704312699	0.998363181068902	0.51177351843587\\
0.405	0.95721709138901	0.999084392196567	0.506761566080167\\
0.405	0.971554823322704	0.999595354092744	0.501746245611877\\
0.405	0.985817034663988	0.999899416688637	0.496729386403002\\
0.405	1	1	0.491669087346908\\
0.42	0	0	0.710195569525497\\
0.42	0.000100583311362513	0.0141829653360114	0.71240460418299\\
0.42	0.000404645907256436	0.0284451766772965	0.714630690536905\\
0.42	0.000915607803432999	0.0427829086109896	0.716916312064106\\
0.42	0.00163681893109844	0.057192295687301	0.719289019360234\\
0.42	0.00257155309398959	0.0716693330697584	0.721780367061016\\
0.42	0.00372300185733414	0.0862098774609879	0.724425655492835\\
0.42	0.00509426838162598	0.100809648312589	0.727263413850304\\
0.42	0.00668836121491816	0.115464229327074	0.73033458997498\\
0.42	0.00850818805808554	0.13016907025918	0.733681422514709\\
0.42	0.0105565495182326	0.144919489023162	0.737345986077549\\
0.42	0.0128361328661109	0.159710674111862	0.741368418705642\\
0.42	0.0153495058140643	0.174537687332543	0.745784862958201\\
0.42	0.0180991103316243	0.189395466863524	0.750625176092311\\
0.42	0.0210872565164405	0.204278830634725	0.755910489868999\\
0.42	0.0243161165387281	0.219182480034174	0.761650724664453\\
0.42	0.0277877186778607	0.234101003941464	0.767842183867616\\
0.42	0.0315039414701067	0.24902888308801	0.77446537092041\\
0.42	0.0354665079868145	0.263960494742775	0.781483180780606\\
0.42	0.0396769802625738	0.278890117720924	0.788839618268073\\
0.42	0.0441367538930258	0.293811937711588	0.796459186325544\\
0.42	0.0488470528220538	0.308720052919643	0.804247066923455\\
0.42	0.0538089243380495	0.323608480015096	0.81209018616195\\
0.42	0.0590232342988274	0.338471160382329	0.819859213932693\\
0.42	0.064490662604533	0.3533019666601	0.827411499063639\\
0.42	0.0702116989375697	0.368094709561873	0.834594885813504\\
0.42	0.0761866387881432	0.382843144964686	0.841252300287178\\
0.42	0.0824155797834956	0.397540981253432	0.847226939729081\\
0.42	0.0888984183382709	0.412181886906127	0.852367847905041\\
0.42	0.0956348466427212	0.42675949830445	0.85653562002783\\
0.42	0.102624350004627	0.441267427752584	0.859607954619617\\
0.42	0.109866204559871	0.455699271686218	0.861484760266619\\
0.42	0.117359475365564	0.470048619052377	0.862092534251542\\
0.42	0.125103014888515	0.484309059839721	0.861387758068371\\
0.42	0.133095461900593	0.498474193737904	0.859359100890821\\
0.42	0.14133524079124	0.512537638903683	0.856028283765738\\
0.42	0.149820561306021	0.526493040810599	0.851449530870181\\
0.42	0.158549418718625	0.540334081158348	0.845707614739507\\
0.42	0.167519594442214	0.554054486817265	0.838914584343153\\
0.42	0.176728657084455	0.567648038782867	0.831205342370839\\
0.42	0.186173963948925	0.581108581114919	0.822732305417524\\
0.42	0.195852662983903	0.594430029835236	0.813659432944307\\
0.42	0.205761695177907	0.607606381758231	0.804155944104868\\
0.42	0.215897797399558	0.620631723228143	0.794390053406972\\
0.42	0.226257505677663	0.633500238737009	0.784523046062802\\
0.42	0.236837158915675	0.646206219397571	0.774703982861587\\
0.42	0.247632903032949	0.658744071245709	0.765065275211016\\
0.42	0.258640695523528	0.671108323347402	0.755719307772966\\
0.42	0.269856310421543	0.683293635685828	0.746756214005912\\
0.42	0.28127534366066	0.695294806804924	0.738242834604296\\
0.42	0.292893218813452	0.707106781186548	0.730222816007834\\
0.42	0.304705193195075	0.71872465633934	0.722717741087767\\
0.42	0.316706364314172	0.730143689578457	0.715729131143268\\
0.42	0.328891676652598	0.741359304476472	0.709241120561856\\
0.42	0.341255928754291	0.752367096967051	0.703223584571265\\
0.42	0.353793780602429	0.763162841084325	0.697635496597419\\
0.42	0.366499761262991	0.773742494322337	0.692428303594619\\
0.42	0.379368276771857	0.784102202600443	0.687549132887068\\
0.42	0.392393618241769	0.794238304822092	0.682943679235329\\
0.42	0.405569970164763	0.804147337016097	0.678558662193045\\
0.42	0.418891418885081	0.813826036051075	0.674343787412871\\
0.42	0.432351961217133	0.823271342915545	0.670253187715552\\
0.42	0.445945513182735	0.832480405557786	0.666246357336158\\
0.42	0.459665918841652	0.841450581281375	0.662288623474186\\
0.42	0.473506959189401	0.850179438693979	0.658351221673191\\
0.42	0.487462361096317	0.85866475920876	0.654411055137887\\
0.42	0.501525806262096	0.866904538099407	0.65045022320647\\
0.42	0.515690940160279	0.874896985111485	0.646455401874958\\
0.42	0.529951380947623	0.882640524634436	0.642417151055154\\
0.42	0.544300728313782	0.890133795440129	0.638329210947587\\
0.42	0.558732572247415	0.897375649995373	0.634187835387498\\
0.42	0.57324050169555	0.904365153357279	0.629991194998376\\
0.42	0.587818113093873	0.911101581661729	0.625738868900925\\
0.42	0.602459018746568	0.917584420216505	0.621431431633139\\
0.42	0.617156855035314	0.923813361211857	0.617070132480552\\
0.42	0.631905290438127	0.92978830106243	0.612656657833838\\
0.42	0.6466980333399	0.935509337395467	0.60819296337709\\
0.42	0.661528839617671	0.940976765701173	0.603681161495104\\
0.42	0.676391519984904	0.94619107566195	0.599123449738606\\
0.42	0.691279947080357	0.951152947177946	0.594522067901591\\
0.42	0.706188062288412	0.955863246106974	0.589879273662389\\
0.42	0.721109882279076	0.960323019737426	0.585197329319937\\
0.42	0.736039505257226	0.964533492013186	0.580478494543975\\
0.42	0.75097111691199	0.968496058529893	0.575725022018313\\
0.42	0.765898996058536	0.972212281322139	0.57093915428956\\
0.42	0.780817519965826	0.975683883461272	0.566123121053175\\
0.42	0.795721169365275	0.97891274348356	0.561279136605559\\
0.42	0.810604533136476	0.981900889668376	0.556409397399199\\
0.42	0.825462312667457	0.984650494185936	0.551516079695734\\
0.42	0.840289325888138	0.987163867133889	0.546601337319641\\
0.42	0.855080510976839	0.989443450481767	0.54166729951514\\
0.42	0.86983092974082	0.991491811941914	0.536716068908599\\
0.42	0.884535770672926	0.993311638785082	0.531749719578143\\
0.42	0.899190351687411	0.994905731618374	0.526770295231792\\
0.42	0.913790122539012	0.996276998142666	0.521779807495102\\
0.42	0.928330666930241	0.997428446906011	0.516780234308684\\
0.42	0.942807704312699	0.998363181068902	0.511773518435873\\
0.42	0.95721709138901	0.999084392196567	0.50676156608017\\
0.42	0.971554823322703	0.999595354092743	0.501746245611874\\
0.42	0.985817034663988	0.999899416688637	0.496729386403004\\
0.42	1	1	0.491669087346906\\
0.435	0	0	0.710195569525497\\
0.435	0.000100583311362513	0.0141829653360114	0.71240460418299\\
0.435	0.000404645907256436	0.0284451766772965	0.714630690536905\\
0.435	0.000915607803432999	0.0427829086109896	0.716916312064106\\
0.435	0.00163681893109844	0.057192295687301	0.719289019360234\\
0.435	0.00257155309398959	0.0716693330697584	0.721780367061016\\
0.435	0.00372300185733413	0.0862098774609879	0.724425655492835\\
0.435	0.00509426838162598	0.100809648312589	0.727263413850304\\
0.435	0.00668836121491816	0.115464229327074	0.73033458997498\\
0.435	0.00850818805808554	0.13016907025918	0.733681422514709\\
0.435	0.0105565495182326	0.144919489023162	0.737345986077549\\
0.435	0.0128361328661109	0.159710674111862	0.741368418705642\\
0.435	0.0153495058140643	0.174537687332543	0.745784862958201\\
0.435	0.0180991103316243	0.189395466863524	0.750625176092311\\
0.435	0.0210872565164405	0.204278830634725	0.755910489868999\\
0.435	0.0243161165387281	0.219182480034174	0.761650724664453\\
0.435	0.0277877186778607	0.234101003941464	0.767842183867616\\
0.435	0.0315039414701067	0.24902888308801	0.77446537092041\\
0.435	0.0354665079868145	0.263960494742775	0.781483180780606\\
0.435	0.0396769802625738	0.278890117720924	0.788839618268073\\
0.435	0.0441367538930258	0.293811937711588	0.796459186325543\\
0.435	0.0488470528220538	0.308720052919643	0.804247066923454\\
0.435	0.0538089243380495	0.323608480015096	0.812090186161952\\
0.435	0.0590232342988274	0.338471160382329	0.819859213932694\\
0.435	0.064490662604533	0.3533019666601	0.82741149906364\\
0.435	0.0702116989375697	0.368094709561873	0.834594885813506\\
0.435	0.0761866387881432	0.382843144964686	0.841252300287179\\
0.435	0.0824155797834956	0.397540981253432	0.847226939729079\\
0.435	0.0888984183382709	0.412181886906127	0.852367847905041\\
0.435	0.0956348466427212	0.42675949830445	0.856535620027828\\
0.435	0.102624350004627	0.441267427752585	0.859607954619618\\
0.435	0.109866204559871	0.455699271686218	0.861484760266621\\
0.435	0.117359475365564	0.470048619052377	0.862092534251541\\
0.435	0.125103014888515	0.484309059839721	0.861387758068371\\
0.435	0.133095461900593	0.498474193737904	0.85935910089082\\
0.435	0.14133524079124	0.512537638903683	0.856028283765732\\
0.435	0.149820561306021	0.526493040810599	0.851449530870182\\
0.435	0.158549418718625	0.540334081158348	0.845707614739506\\
0.435	0.167519594442213	0.554054486817265	0.83891458434315\\
0.435	0.176728657084455	0.567648038782867	0.831205342370836\\
0.435	0.186173963948925	0.581108581114919	0.822732305417527\\
0.435	0.195852662983903	0.594430029835236	0.813659432944306\\
0.435	0.205761695177907	0.607606381758231	0.804155944104872\\
0.435	0.215897797399558	0.620631723228144	0.794390053406973\\
0.435	0.226257505677663	0.633500238737009	0.784523046062799\\
0.435	0.236837158915675	0.646206219397571	0.77470398286159\\
0.435	0.247632903032949	0.658744071245709	0.765065275211018\\
0.435	0.258640695523527	0.671108323347402	0.755719307772962\\
0.435	0.269856310421543	0.683293635685828	0.74675621400591\\
0.435	0.28127534366066	0.695294806804924	0.738242834604296\\
0.435	0.292893218813452	0.707106781186547	0.730222816007835\\
0.435	0.304705193195075	0.71872465633934	0.722717741087767\\
0.435	0.316706364314172	0.730143689578457	0.715729131143267\\
0.435	0.328891676652598	0.741359304476472	0.709241120561858\\
0.435	0.341255928754291	0.752367096967051	0.703223584571265\\
0.435	0.353793780602429	0.763162841084325	0.697635496597419\\
0.435	0.366499761262991	0.773742494322337	0.692428303594618\\
0.435	0.379368276771856	0.784102202600442	0.687549132887068\\
0.435	0.392393618241769	0.794238304822092	0.682943679235329\\
0.435	0.405569970164763	0.804147337016097	0.678558662193044\\
0.435	0.418891418885081	0.813826036051075	0.67434378741287\\
0.435	0.432351961217132	0.823271342915544	0.670253187715552\\
0.435	0.445945513182735	0.832480405557786	0.666246357336159\\
0.435	0.459665918841652	0.841450581281375	0.662288623474186\\
0.435	0.473506959189401	0.850179438693979	0.658351221673189\\
0.435	0.487462361096317	0.85866475920876	0.654411055137886\\
0.435	0.501525806262096	0.866904538099407	0.650450223206469\\
0.435	0.515690940160279	0.874896985111485	0.646455401874957\\
0.435	0.529951380947623	0.882640524634436	0.642417151055154\\
0.435	0.544300728313782	0.890133795440129	0.638329210947589\\
0.435	0.558732572247415	0.897375649995373	0.634187835387498\\
0.435	0.57324050169555	0.904365153357279	0.629991194998375\\
0.435	0.587818113093873	0.911101581661729	0.625738868900925\\
0.435	0.602459018746568	0.917584420216504	0.621431431633141\\
0.435	0.617156855035314	0.923813361211857	0.617070132480553\\
0.435	0.631905290438127	0.92978830106243	0.612656657833835\\
0.435	0.6466980333399	0.935509337395467	0.608192963377088\\
0.435	0.661528839617671	0.940976765701173	0.603681161495107\\
0.435	0.676391519984903	0.946191075661951	0.599123449738605\\
0.435	0.691279947080357	0.951152947177946	0.594522067901588\\
0.435	0.706188062288412	0.955863246106974	0.58987927366239\\
0.435	0.721109882279076	0.960323019737426	0.585197329319937\\
0.435	0.736039505257226	0.964533492013186	0.580478494543977\\
0.435	0.75097111691199	0.968496058529893	0.575725022018313\\
0.435	0.765898996058536	0.972212281322139	0.570939154289561\\
0.435	0.780817519965826	0.975683883461272	0.566123121053176\\
0.435	0.795721169365275	0.97891274348356	0.561279136605559\\
0.435	0.810604533136476	0.981900889668376	0.556409397399198\\
0.435	0.825462312667457	0.984650494185936	0.551516079695733\\
0.435	0.840289325888138	0.987163867133889	0.54660133731964\\
0.435	0.855080510976838	0.989443450481767	0.54166729951514\\
0.435	0.86983092974082	0.991491811941914	0.5367160689086\\
0.435	0.884535770672926	0.993311638785082	0.531749719578144\\
0.435	0.899190351687411	0.994905731618374	0.526770295231794\\
0.435	0.913790122539012	0.996276998142666	0.5217798074951\\
0.435	0.928330666930242	0.997428446906011	0.516780234308682\\
0.435	0.942807704312699	0.998363181068902	0.511773518435871\\
0.435	0.95721709138901	0.999084392196567	0.506761566080172\\
0.435	0.971554823322704	0.999595354092744	0.501746245611876\\
0.435	0.985817034663989	0.999899416688637	0.496729386403001\\
0.435	1	1	0.491669087346908\\
0.45	0	0	0.710195569525497\\
0.45	0.000100583311362513	0.0141829653360114	0.71240460418299\\
0.45	0.000404645907256436	0.0284451766772965	0.714630690536905\\
0.45	0.000915607803432999	0.0427829086109896	0.716916312064106\\
0.45	0.00163681893109844	0.057192295687301	0.719289019360234\\
0.45	0.00257155309398959	0.0716693330697584	0.721780367061016\\
0.45	0.00372300185733413	0.0862098774609879	0.724425655492835\\
0.45	0.00509426838162598	0.100809648312589	0.727263413850304\\
0.45	0.00668836121491816	0.115464229327074	0.73033458997498\\
0.45	0.00850818805808555	0.13016907025918	0.733681422514709\\
0.45	0.0105565495182326	0.144919489023162	0.737345986077549\\
0.45	0.0128361328661109	0.159710674111862	0.741368418705642\\
0.45	0.0153495058140643	0.174537687332543	0.745784862958202\\
0.45	0.0180991103316243	0.189395466863524	0.750625176092311\\
0.45	0.0210872565164405	0.204278830634725	0.755910489868999\\
0.45	0.0243161165387281	0.219182480034174	0.761650724664453\\
0.45	0.0277877186778607	0.234101003941464	0.767842183867616\\
0.45	0.0315039414701067	0.24902888308801	0.77446537092041\\
0.45	0.0354665079868145	0.263960494742775	0.781483180780606\\
0.45	0.0396769802625738	0.278890117720924	0.788839618268073\\
0.45	0.0441367538930258	0.293811937711588	0.796459186325544\\
0.45	0.0488470528220537	0.308720052919643	0.804247066923453\\
0.45	0.0538089243380495	0.323608480015097	0.81209018616195\\
0.45	0.0590232342988274	0.338471160382329	0.819859213932694\\
0.45	0.064490662604533	0.3533019666601	0.827411499063639\\
0.45	0.0702116989375697	0.368094709561873	0.834594885813505\\
0.45	0.0761866387881432	0.382843144964686	0.84125230028718\\
0.45	0.0824155797834956	0.397540981253432	0.847226939729081\\
0.45	0.0888984183382709	0.412181886906127	0.852367847905041\\
0.45	0.0956348466427212	0.42675949830445	0.856535620027829\\
0.45	0.102624350004627	0.441267427752585	0.859607954619618\\
0.45	0.109866204559871	0.455699271686218	0.861484760266622\\
0.45	0.117359475365564	0.470048619052377	0.86209253425154\\
0.45	0.125103014888515	0.484309059839721	0.861387758068367\\
0.45	0.133095461900593	0.498474193737904	0.859359100890821\\
0.45	0.14133524079124	0.512537638903683	0.856028283765736\\
0.45	0.149820561306021	0.526493040810599	0.851449530870184\\
0.45	0.158549418718625	0.540334081158348	0.845707614739505\\
0.45	0.167519594442214	0.554054486817265	0.838914584343151\\
0.45	0.176728657084455	0.567648038782867	0.831205342370837\\
0.45	0.186173963948925	0.581108581114919	0.822732305417527\\
0.45	0.195852662983903	0.594430029835236	0.813659432944304\\
0.45	0.205761695177907	0.607606381758231	0.804155944104872\\
0.45	0.215897797399558	0.620631723228144	0.79439005340697\\
0.45	0.226257505677663	0.633500238737009	0.784523046062801\\
0.45	0.236837158915675	0.646206219397571	0.774703982861592\\
0.45	0.247632903032949	0.658744071245709	0.765065275211014\\
0.45	0.258640695523527	0.671108323347402	0.755719307772961\\
0.45	0.269856310421543	0.683293635685828	0.746756214005912\\
0.45	0.28127534366066	0.695294806804924	0.738242834604299\\
0.45	0.292893218813452	0.707106781186548	0.730222816007834\\
0.45	0.304705193195075	0.71872465633934	0.722717741087766\\
0.45	0.316706364314172	0.730143689578457	0.715729131143268\\
0.45	0.328891676652598	0.741359304476472	0.709241120561856\\
0.45	0.341255928754291	0.752367096967051	0.703223584571265\\
0.45	0.353793780602429	0.763162841084325	0.697635496597418\\
0.45	0.366499761262991	0.773742494322337	0.692428303594618\\
0.45	0.379368276771857	0.784102202600442	0.687549132887069\\
0.45	0.392393618241769	0.794238304822092	0.68294367923533\\
0.45	0.405569970164763	0.804147337016097	0.678558662193045\\
0.45	0.418891418885081	0.813826036051075	0.674343787412871\\
0.45	0.432351961217133	0.823271342915545	0.670253187715552\\
0.45	0.445945513182735	0.832480405557786	0.666246357336158\\
0.45	0.459665918841652	0.841450581281375	0.662288623474187\\
0.45	0.473506959189401	0.850179438693979	0.658351221673191\\
0.45	0.487462361096317	0.85866475920876	0.654411055137885\\
0.45	0.501525806262096	0.866904538099407	0.650450223206469\\
0.45	0.515690940160279	0.874896985111485	0.646455401874959\\
0.45	0.529951380947623	0.882640524634436	0.642417151055154\\
0.45	0.544300728313782	0.890133795440129	0.638329210947586\\
0.45	0.558732572247415	0.897375649995373	0.634187835387498\\
0.45	0.57324050169555	0.904365153357279	0.629991194998375\\
0.45	0.587818113093873	0.911101581661729	0.625738868900924\\
0.45	0.602459018746568	0.917584420216504	0.621431431633141\\
0.45	0.617156855035314	0.923813361211857	0.617070132480553\\
0.45	0.631905290438127	0.92978830106243	0.612656657833837\\
0.45	0.6466980333399	0.935509337395467	0.608192963377086\\
0.45	0.661528839617671	0.940976765701173	0.603681161495105\\
0.45	0.676391519984903	0.94619107566195	0.599123449738607\\
0.45	0.691279947080357	0.951152947177946	0.594522067901589\\
0.45	0.706188062288412	0.955863246106974	0.589879273662389\\
0.45	0.721109882279076	0.960323019737426	0.585197329319936\\
0.45	0.736039505257226	0.964533492013185	0.580478494543975\\
0.45	0.75097111691199	0.968496058529893	0.575725022018313\\
0.45	0.765898996058536	0.972212281322139	0.570939154289558\\
0.45	0.780817519965826	0.975683883461272	0.566123121053174\\
0.45	0.795721169365275	0.978912743483559	0.561279136605561\\
0.45	0.810604533136476	0.981900889668376	0.556409397399202\\
0.45	0.825462312667457	0.984650494185936	0.551516079695734\\
0.45	0.840289325888138	0.987163867133889	0.546601337319639\\
0.45	0.855080510976839	0.989443450481767	0.54166729951514\\
0.45	0.86983092974082	0.991491811941915	0.536716068908599\\
0.45	0.884535770672926	0.993311638785082	0.531749719578142\\
0.45	0.899190351687411	0.994905731618374	0.526770295231795\\
0.45	0.913790122539012	0.996276998142666	0.521779807495104\\
0.45	0.928330666930242	0.997428446906011	0.516780234308682\\
0.45	0.942807704312699	0.998363181068902	0.51177351843587\\
0.45	0.95721709138901	0.999084392196567	0.506761566080171\\
0.45	0.971554823322703	0.999595354092744	0.501746245611878\\
0.45	0.985817034663989	0.999899416688638	0.496729386403002\\
0.45	1	1	0.491669087346898\\
0.465	0	0	0.710195569525497\\
0.465	0.000100583311362513	0.0141829653360114	0.71240460418299\\
0.465	0.000404645907256436	0.0284451766772965	0.714630690536905\\
0.465	0.000915607803432999	0.0427829086109896	0.716916312064106\\
0.465	0.00163681893109844	0.057192295687301	0.719289019360234\\
0.465	0.00257155309398959	0.0716693330697584	0.721780367061016\\
0.465	0.00372300185733413	0.0862098774609879	0.724425655492835\\
0.465	0.00509426838162598	0.100809648312589	0.727263413850304\\
0.465	0.00668836121491815	0.115464229327074	0.73033458997498\\
0.465	0.00850818805808554	0.13016907025918	0.733681422514709\\
0.465	0.0105565495182326	0.144919489023162	0.737345986077549\\
0.465	0.0128361328661109	0.159710674111862	0.741368418705642\\
0.465	0.0153495058140643	0.174537687332543	0.745784862958201\\
0.465	0.0180991103316243	0.189395466863524	0.750625176092311\\
0.465	0.0210872565164405	0.204278830634725	0.755910489868999\\
0.465	0.0243161165387281	0.219182480034174	0.761650724664453\\
0.465	0.0277877186778607	0.234101003941464	0.767842183867616\\
0.465	0.0315039414701067	0.24902888308801	0.77446537092041\\
0.465	0.0354665079868145	0.263960494742775	0.781483180780606\\
0.465	0.0396769802625738	0.278890117720924	0.788839618268073\\
0.465	0.0441367538930258	0.293811937711588	0.796459186325545\\
0.465	0.0488470528220538	0.308720052919643	0.804247066923454\\
0.465	0.0538089243380495	0.323608480015097	0.812090186161949\\
0.465	0.0590232342988274	0.338471160382329	0.819859213932694\\
0.465	0.064490662604533	0.3533019666601	0.827411499063639\\
0.465	0.0702116989375697	0.368094709561873	0.834594885813503\\
0.465	0.0761866387881432	0.382843144964686	0.841252300287178\\
0.465	0.0824155797834956	0.397540981253432	0.847226939729081\\
0.465	0.0888984183382709	0.412181886906127	0.852367847905042\\
0.465	0.0956348466427212	0.42675949830445	0.85653562002783\\
0.465	0.102624350004627	0.441267427752584	0.859607954619618\\
0.465	0.109866204559871	0.455699271686218	0.861484760266622\\
0.465	0.117359475365564	0.470048619052377	0.862092534251539\\
0.465	0.125103014888515	0.484309059839721	0.861387758068366\\
0.465	0.133095461900593	0.498474193737904	0.859359100890822\\
0.465	0.14133524079124	0.512537638903683	0.856028283765736\\
0.465	0.149820561306021	0.526493040810599	0.851449530870182\\
0.465	0.158549418718625	0.540334081158348	0.84570761473951\\
0.465	0.167519594442213	0.554054486817265	0.83891458434315\\
0.465	0.176728657084455	0.567648038782867	0.831205342370836\\
0.465	0.186173963948925	0.581108581114919	0.822732305417524\\
0.465	0.195852662983903	0.594430029835236	0.813659432944305\\
0.465	0.205761695177907	0.607606381758231	0.804155944104873\\
0.465	0.215897797399558	0.620631723228144	0.794390053406969\\
0.465	0.226257505677663	0.633500238737009	0.7845230460628\\
0.465	0.236837158915675	0.646206219397571	0.774703982861588\\
0.465	0.247632903032949	0.658744071245709	0.765065275211013\\
0.465	0.258640695523527	0.671108323347402	0.755719307772962\\
0.465	0.269856310421543	0.683293635685828	0.746756214005913\\
0.465	0.28127534366066	0.695294806804925	0.738242834604295\\
0.465	0.292893218813452	0.707106781186547	0.730222816007831\\
0.465	0.304705193195075	0.71872465633934	0.722717741087768\\
0.465	0.316706364314172	0.730143689578457	0.715729131143268\\
0.465	0.328891676652598	0.741359304476472	0.709241120561856\\
0.465	0.341255928754291	0.752367096967051	0.703223584571265\\
0.465	0.353793780602429	0.763162841084325	0.697635496597419\\
0.465	0.366499761262991	0.773742494322337	0.692428303594618\\
0.465	0.379368276771856	0.784102202600442	0.687549132887069\\
0.465	0.392393618241769	0.794238304822092	0.682943679235328\\
0.465	0.405569970164763	0.804147337016097	0.678558662193044\\
0.465	0.418891418885081	0.813826036051075	0.674343787412871\\
0.465	0.432351961217132	0.823271342915544	0.670253187715552\\
0.465	0.445945513182735	0.832480405557786	0.666246357336158\\
0.465	0.459665918841652	0.841450581281375	0.662288623474186\\
0.465	0.473506959189401	0.850179438693979	0.658351221673191\\
0.465	0.487462361096317	0.85866475920876	0.654411055137885\\
0.465	0.501525806262096	0.866904538099407	0.650450223206468\\
0.465	0.515690940160279	0.874896985111485	0.646455401874959\\
0.465	0.529951380947623	0.882640524634436	0.642417151055155\\
0.465	0.544300728313782	0.890133795440129	0.638329210947586\\
0.465	0.558732572247415	0.897375649995373	0.634187835387498\\
0.465	0.57324050169555	0.904365153357279	0.629991194998376\\
0.465	0.587818113093873	0.911101581661729	0.625738868900925\\
0.465	0.602459018746568	0.917584420216504	0.62143143163314\\
0.465	0.617156855035314	0.923813361211857	0.617070132480553\\
0.465	0.631905290438127	0.92978830106243	0.612656657833838\\
0.465	0.6466980333399	0.935509337395467	0.60819296337709\\
0.465	0.661528839617671	0.940976765701173	0.603681161495104\\
0.465	0.676391519984904	0.946191075661951	0.599123449738605\\
0.465	0.691279947080357	0.951152947177946	0.594522067901589\\
0.465	0.706188062288412	0.955863246106974	0.58987927366239\\
0.465	0.721109882279076	0.960323019737426	0.585197329319937\\
0.465	0.736039505257226	0.964533492013186	0.580478494543977\\
0.465	0.75097111691199	0.968496058529893	0.575725022018315\\
0.465	0.765898996058536	0.972212281322139	0.570939154289558\\
0.465	0.780817519965826	0.975683883461272	0.566123121053173\\
0.465	0.795721169365275	0.97891274348356	0.56127913660556\\
0.465	0.810604533136476	0.981900889668376	0.556409397399198\\
0.465	0.825462312667457	0.984650494185936	0.551516079695735\\
0.465	0.840289325888138	0.987163867133889	0.546601337319641\\
0.465	0.855080510976838	0.989443450481768	0.541667299515141\\
0.465	0.86983092974082	0.991491811941915	0.536716068908598\\
0.465	0.884535770672926	0.993311638785082	0.531749719578141\\
0.465	0.899190351687411	0.994905731618374	0.526770295231793\\
0.465	0.913790122539012	0.996276998142666	0.521779807495102\\
0.465	0.928330666930241	0.99742844690601	0.516780234308685\\
0.465	0.942807704312699	0.998363181068901	0.511773518435872\\
0.465	0.95721709138901	0.999084392196567	0.506761566080169\\
0.465	0.971554823322703	0.999595354092743	0.501746245611877\\
0.465	0.985817034663988	0.999899416688637	0.496729386403004\\
0.465	1	1	0.491669087346906\\
0.48	0	0	0.710195569525497\\
0.48	0.000100583311362513	0.0141829653360114	0.71240460418299\\
0.48	0.000404645907256436	0.0284451766772965	0.714630690536905\\
0.48	0.000915607803432999	0.0427829086109896	0.716916312064106\\
0.48	0.00163681893109843	0.057192295687301	0.719289019360234\\
0.48	0.00257155309398959	0.0716693330697584	0.721780367061016\\
0.48	0.00372300185733413	0.0862098774609879	0.724425655492835\\
0.48	0.00509426838162598	0.100809648312589	0.727263413850304\\
0.48	0.00668836121491815	0.115464229327074	0.73033458997498\\
0.48	0.00850818805808555	0.13016907025918	0.733681422514709\\
0.48	0.0105565495182326	0.144919489023162	0.737345986077549\\
0.48	0.0128361328661109	0.159710674111862	0.741368418705642\\
0.48	0.0153495058140643	0.174537687332543	0.745784862958201\\
0.48	0.0180991103316243	0.189395466863524	0.750625176092311\\
0.48	0.0210872565164405	0.204278830634725	0.755910489868999\\
0.48	0.0243161165387281	0.219182480034174	0.761650724664453\\
0.48	0.0277877186778607	0.234101003941464	0.767842183867616\\
0.48	0.0315039414701067	0.24902888308801	0.77446537092041\\
0.48	0.0354665079868145	0.263960494742775	0.781483180780606\\
0.48	0.0396769802625738	0.278890117720924	0.788839618268073\\
0.48	0.0441367538930258	0.293811937711588	0.796459186325544\\
0.48	0.0488470528220538	0.308720052919643	0.804247066923455\\
0.48	0.0538089243380495	0.323608480015096	0.81209018616195\\
0.48	0.0590232342988274	0.338471160382329	0.819859213932693\\
0.48	0.064490662604533	0.3533019666601	0.827411499063639\\
0.48	0.0702116989375697	0.368094709561873	0.834594885813504\\
0.48	0.0761866387881432	0.382843144964686	0.841252300287179\\
0.48	0.0824155797834956	0.397540981253432	0.847226939729081\\
0.48	0.0888984183382709	0.412181886906127	0.85236784790504\\
0.48	0.0956348466427212	0.42675949830445	0.856535620027829\\
0.48	0.102624350004627	0.441267427752584	0.859607954619618\\
0.48	0.109866204559871	0.455699271686218	0.861484760266622\\
0.48	0.117359475365564	0.470048619052377	0.86209253425154\\
0.48	0.125103014888515	0.484309059839721	0.861387758068365\\
0.48	0.133095461900593	0.498474193737904	0.859359100890821\\
0.48	0.14133524079124	0.512537638903683	0.856028283765737\\
0.48	0.149820561306021	0.526493040810599	0.851449530870183\\
0.48	0.158549418718625	0.540334081158348	0.845707614739506\\
0.48	0.167519594442214	0.554054486817265	0.838914584343149\\
0.48	0.176728657084455	0.567648038782867	0.831205342370836\\
0.48	0.186173963948925	0.581108581114919	0.822732305417526\\
0.48	0.195852662983903	0.594430029835236	0.813659432944311\\
0.48	0.205761695177907	0.607606381758231	0.804155944104871\\
0.48	0.215897797399558	0.620631723228143	0.794390053406969\\
0.48	0.226257505677663	0.633500238737009	0.784523046062802\\
0.48	0.236837158915675	0.646206219397571	0.774703982861589\\
0.48	0.247632903032949	0.658744071245709	0.765065275211016\\
0.48	0.258640695523527	0.671108323347402	0.755719307772961\\
0.48	0.269856310421543	0.683293635685828	0.74675621400591\\
0.48	0.28127534366066	0.695294806804924	0.738242834604294\\
0.48	0.292893218813452	0.707106781186547	0.730222816007835\\
0.48	0.304705193195075	0.71872465633934	0.722717741087769\\
0.48	0.316706364314172	0.730143689578457	0.715729131143269\\
0.48	0.328891676652598	0.741359304476472	0.709241120561855\\
0.48	0.341255928754291	0.752367096967051	0.703223584571264\\
0.48	0.353793780602429	0.763162841084325	0.697635496597419\\
0.48	0.366499761262991	0.773742494322337	0.692428303594619\\
0.48	0.379368276771856	0.784102202600442	0.687549132887069\\
0.48	0.392393618241769	0.794238304822092	0.682943679235329\\
0.48	0.405569970164763	0.804147337016097	0.678558662193043\\
0.48	0.418891418885081	0.813826036051075	0.67434378741287\\
0.48	0.432351961217132	0.823271342915544	0.670253187715553\\
0.48	0.445945513182735	0.832480405557786	0.666246357336159\\
0.48	0.459665918841652	0.841450581281375	0.662288623474186\\
0.48	0.473506959189401	0.850179438693979	0.65835122167319\\
0.48	0.487462361096317	0.85866475920876	0.654411055137886\\
0.48	0.501525806262096	0.866904538099407	0.650450223206467\\
0.48	0.515690940160279	0.874896985111485	0.646455401874958\\
0.48	0.529951380947623	0.882640524634436	0.642417151055155\\
0.48	0.544300728313781	0.890133795440129	0.638329210947585\\
0.48	0.558732572247415	0.897375649995372	0.634187835387497\\
0.48	0.57324050169555	0.904365153357279	0.629991194998377\\
0.48	0.587818113093873	0.911101581661729	0.625738868900926\\
0.48	0.602459018746568	0.917584420216505	0.621431431633141\\
0.48	0.617156855035314	0.923813361211857	0.617070132480552\\
0.48	0.631905290438127	0.92978830106243	0.612656657833835\\
0.48	0.6466980333399	0.935509337395467	0.608192963377088\\
0.48	0.661528839617671	0.940976765701173	0.603681161495108\\
0.48	0.676391519984904	0.946191075661951	0.599123449738607\\
0.48	0.691279947080357	0.951152947177946	0.594522067901589\\
0.48	0.706188062288412	0.955863246106974	0.589879273662389\\
0.48	0.721109882279076	0.960323019737426	0.585197329319937\\
0.48	0.736039505257226	0.964533492013186	0.580478494543975\\
0.48	0.75097111691199	0.968496058529893	0.575725022018313\\
0.48	0.765898996058536	0.972212281322139	0.570939154289562\\
0.48	0.780817519965826	0.975683883461272	0.566123121053174\\
0.48	0.795721169365275	0.978912743483559	0.561279136605561\\
0.48	0.810604533136476	0.981900889668376	0.556409397399201\\
0.48	0.825462312667457	0.984650494185936	0.551516079695734\\
0.48	0.840289325888138	0.987163867133889	0.546601337319641\\
0.48	0.855080510976839	0.989443450481768	0.541667299515139\\
0.48	0.86983092974082	0.991491811941914	0.5367160689086\\
0.48	0.884535770672926	0.993311638785082	0.531749719578142\\
0.48	0.899190351687411	0.994905731618374	0.526770295231792\\
0.48	0.913790122539012	0.996276998142666	0.521779807495102\\
0.48	0.928330666930242	0.997428446906011	0.516780234308685\\
0.48	0.942807704312699	0.998363181068902	0.511773518435872\\
0.48	0.95721709138901	0.999084392196567	0.506761566080166\\
0.48	0.971554823322703	0.999595354092743	0.501746245611875\\
0.48	0.985817034663989	0.999899416688637	0.496729386403005\\
0.48	1	1	0.491669087346904\\
0.495	0	0	0.710195569525497\\
0.495	0.000100583311362513	0.0141829653360114	0.71240460418299\\
0.495	0.000404645907256436	0.0284451766772965	0.714630690536905\\
0.495	0.000915607803432999	0.0427829086109896	0.716916312064106\\
0.495	0.00163681893109844	0.057192295687301	0.719289019360234\\
0.495	0.00257155309398959	0.0716693330697584	0.721780367061016\\
0.495	0.00372300185733413	0.0862098774609879	0.724425655492835\\
0.495	0.00509426838162598	0.100809648312589	0.727263413850304\\
0.495	0.00668836121491816	0.115464229327074	0.73033458997498\\
0.495	0.00850818805808555	0.13016907025918	0.733681422514709\\
0.495	0.0105565495182326	0.144919489023162	0.737345986077549\\
0.495	0.0128361328661109	0.159710674111862	0.741368418705642\\
0.495	0.0153495058140643	0.174537687332543	0.745784862958201\\
0.495	0.0180991103316243	0.189395466863524	0.750625176092311\\
0.495	0.0210872565164405	0.204278830634725	0.755910489868999\\
0.495	0.0243161165387281	0.219182480034174	0.761650724664453\\
0.495	0.0277877186778607	0.234101003941464	0.767842183867616\\
0.495	0.0315039414701067	0.24902888308801	0.77446537092041\\
0.495	0.0354665079868145	0.263960494742775	0.781483180780606\\
0.495	0.0396769802625738	0.278890117720924	0.788839618268074\\
0.495	0.0441367538930258	0.293811937711588	0.796459186325544\\
0.495	0.0488470528220538	0.308720052919643	0.804247066923455\\
0.495	0.0538089243380495	0.323608480015096	0.81209018616195\\
0.495	0.0590232342988274	0.338471160382329	0.819859213932694\\
0.495	0.064490662604533	0.3533019666601	0.827411499063639\\
0.495	0.0702116989375697	0.368094709561873	0.834594885813505\\
0.495	0.0761866387881432	0.382843144964686	0.841252300287179\\
0.495	0.0824155797834956	0.397540981253432	0.847226939729081\\
0.495	0.0888984183382709	0.412181886906127	0.852367847905042\\
0.495	0.0956348466427212	0.42675949830445	0.856535620027827\\
0.495	0.102624350004627	0.441267427752584	0.859607954619617\\
0.495	0.109866204559871	0.455699271686218	0.86148476026662\\
0.495	0.117359475365564	0.470048619052377	0.862092534251539\\
0.495	0.125103014888515	0.484309059839721	0.861387758068369\\
0.495	0.133095461900593	0.498474193737904	0.859359100890822\\
0.495	0.14133524079124	0.512537638903683	0.85602828376574\\
0.495	0.149820561306021	0.526493040810599	0.851449530870184\\
0.495	0.158549418718625	0.540334081158348	0.845707614739506\\
0.495	0.167519594442214	0.554054486817265	0.83891458434315\\
0.495	0.176728657084455	0.567648038782867	0.831205342370836\\
0.495	0.186173963948925	0.581108581114919	0.822732305417527\\
0.495	0.195852662983903	0.594430029835237	0.813659432944309\\
0.495	0.205761695177907	0.607606381758231	0.80415594410487\\
0.495	0.215897797399558	0.620631723228144	0.794390053406971\\
0.495	0.226257505677663	0.633500238737009	0.784523046062801\\
0.495	0.236837158915675	0.646206219397571	0.774703982861586\\
0.495	0.247632903032949	0.658744071245709	0.765065275211015\\
0.495	0.258640695523528	0.671108323347402	0.755719307772961\\
0.495	0.269856310421543	0.683293635685828	0.74675621400591\\
0.495	0.28127534366066	0.695294806804924	0.738242834604296\\
0.495	0.292893218813452	0.707106781186547	0.730222816007835\\
0.495	0.304705193195075	0.71872465633934	0.722717741087768\\
0.495	0.316706364314172	0.730143689578457	0.715729131143267\\
0.495	0.328891676652598	0.741359304476472	0.709241120561857\\
0.495	0.341255928754291	0.752367096967051	0.703223584571265\\
0.495	0.353793780602429	0.763162841084325	0.697635496597418\\
0.495	0.366499761262991	0.773742494322337	0.692428303594618\\
0.495	0.379368276771856	0.784102202600442	0.687549132887069\\
0.495	0.392393618241769	0.794238304822092	0.682943679235329\\
0.495	0.405569970164763	0.804147337016097	0.678558662193044\\
0.495	0.418891418885081	0.813826036051075	0.67434378741287\\
0.495	0.432351961217132	0.823271342915544	0.670253187715553\\
0.495	0.445945513182735	0.832480405557786	0.666246357336159\\
0.495	0.459665918841652	0.841450581281375	0.662288623474186\\
0.495	0.473506959189401	0.850179438693979	0.65835122167319\\
0.495	0.487462361096317	0.85866475920876	0.654411055137886\\
0.495	0.501525806262096	0.866904538099407	0.65045022320647\\
0.495	0.515690940160279	0.874896985111485	0.646455401874959\\
0.495	0.529951380947623	0.882640524634437	0.642417151055154\\
0.495	0.544300728313782	0.890133795440129	0.638329210947586\\
0.495	0.558732572247415	0.897375649995373	0.634187835387497\\
0.495	0.57324050169555	0.904365153357279	0.629991194998376\\
0.495	0.587818113093873	0.911101581661729	0.625738868900925\\
0.495	0.602459018746568	0.917584420216504	0.621431431633141\\
0.495	0.617156855035314	0.923813361211857	0.617070132480553\\
0.495	0.631905290438127	0.92978830106243	0.612656657833836\\
0.495	0.6466980333399	0.935509337395467	0.608192963377087\\
0.495	0.661528839617671	0.940976765701173	0.603681161495106\\
0.495	0.676391519984903	0.94619107566195	0.599123449738607\\
0.495	0.691279947080358	0.951152947177946	0.59452206790159\\
0.495	0.706188062288412	0.955863246106974	0.58987927366239\\
0.495	0.721109882279076	0.960323019737426	0.585197329319937\\
0.495	0.736039505257226	0.964533492013186	0.580478494543976\\
0.495	0.75097111691199	0.968496058529893	0.575725022018311\\
0.495	0.765898996058536	0.972212281322139	0.570939154289559\\
0.495	0.780817519965826	0.975683883461272	0.566123121053174\\
0.495	0.795721169365275	0.978912743483559	0.56127913660556\\
0.495	0.810604533136476	0.981900889668376	0.556409397399201\\
0.495	0.825462312667457	0.984650494185936	0.551516079695734\\
0.495	0.840289325888138	0.987163867133889	0.546601337319641\\
0.495	0.855080510976839	0.989443450481767	0.541667299515139\\
0.495	0.86983092974082	0.991491811941914	0.536716068908599\\
0.495	0.884535770672926	0.993311638785082	0.531749719578144\\
0.495	0.899190351687411	0.994905731618374	0.526770295231792\\
0.495	0.913790122539012	0.996276998142666	0.521779807495101\\
0.495	0.928330666930242	0.997428446906011	0.516780234308683\\
0.495	0.942807704312699	0.998363181068902	0.511773518435871\\
0.495	0.95721709138901	0.999084392196567	0.506761566080168\\
0.495	0.971554823322703	0.999595354092743	0.501746245611877\\
0.495	0.985817034663989	0.999899416688638	0.496729386403005\\
0.495	1	1	0.491669087346895\\
0.51	0	0	0.710195569525497\\
0.51	0.000100583311362513	0.0141829653360114	0.71240460418299\\
0.51	0.000404645907256436	0.0284451766772965	0.714630690536905\\
0.51	0.000915607803432999	0.0427829086109896	0.716916312064106\\
0.51	0.00163681893109844	0.057192295687301	0.719289019360234\\
0.51	0.00257155309398959	0.0716693330697584	0.721780367061016\\
0.51	0.00372300185733414	0.0862098774609879	0.724425655492835\\
0.51	0.00509426838162598	0.100809648312589	0.727263413850304\\
0.51	0.00668836121491816	0.115464229327074	0.73033458997498\\
0.51	0.00850818805808555	0.13016907025918	0.733681422514709\\
0.51	0.0105565495182326	0.144919489023162	0.737345986077549\\
0.51	0.0128361328661109	0.159710674111862	0.741368418705642\\
0.51	0.0153495058140643	0.174537687332543	0.745784862958201\\
0.51	0.0180991103316243	0.189395466863524	0.750625176092311\\
0.51	0.0210872565164405	0.204278830634725	0.755910489868999\\
0.51	0.0243161165387281	0.219182480034174	0.761650724664453\\
0.51	0.0277877186778607	0.234101003941464	0.767842183867616\\
0.51	0.0315039414701067	0.24902888308801	0.77446537092041\\
0.51	0.0354665079868145	0.263960494742775	0.781483180780606\\
0.51	0.0396769802625738	0.278890117720924	0.788839618268073\\
0.51	0.0441367538930258	0.293811937711588	0.796459186325545\\
0.51	0.0488470528220538	0.308720052919643	0.804247066923454\\
0.51	0.0538089243380495	0.323608480015097	0.81209018616195\\
0.51	0.0590232342988274	0.338471160382329	0.819859213932695\\
0.51	0.064490662604533	0.3533019666601	0.82741149906364\\
0.51	0.0702116989375697	0.368094709561873	0.834594885813505\\
0.51	0.0761866387881432	0.382843144964686	0.84125230028718\\
0.51	0.0824155797834956	0.397540981253432	0.847226939729082\\
0.51	0.0888984183382709	0.412181886906127	0.852367847905042\\
0.51	0.0956348466427212	0.42675949830445	0.856535620027829\\
0.51	0.102624350004627	0.441267427752584	0.859607954619619\\
0.51	0.109866204559871	0.455699271686218	0.861484760266622\\
0.51	0.117359475365564	0.470048619052377	0.862092534251538\\
0.51	0.125103014888515	0.484309059839721	0.861387758068372\\
0.51	0.133095461900593	0.498474193737904	0.85935910089082\\
0.51	0.14133524079124	0.512537638903683	0.856028283765735\\
0.51	0.149820561306021	0.526493040810599	0.851449530870185\\
0.51	0.158549418718625	0.540334081158348	0.845707614739502\\
0.51	0.167519594442214	0.554054486817266	0.838914584343148\\
0.51	0.176728657084455	0.567648038782867	0.831205342370836\\
0.51	0.186173963948925	0.581108581114919	0.822732305417528\\
0.51	0.195852662983903	0.594430029835237	0.813659432944305\\
0.51	0.205761695177907	0.607606381758231	0.804155944104871\\
0.51	0.215897797399558	0.620631723228143	0.794390053406968\\
0.51	0.226257505677663	0.633500238737009	0.784523046062799\\
0.51	0.236837158915675	0.646206219397571	0.77470398286159\\
0.51	0.247632903032949	0.658744071245709	0.765065275211015\\
0.51	0.258640695523528	0.671108323347402	0.755719307772961\\
0.51	0.269856310421543	0.683293635685828	0.746756214005911\\
0.51	0.28127534366066	0.695294806804924	0.738242834604298\\
0.51	0.292893218813452	0.707106781186547	0.730222816007836\\
0.51	0.304705193195075	0.71872465633934	0.722717741087767\\
0.51	0.316706364314172	0.730143689578457	0.715729131143269\\
0.51	0.328891676652598	0.741359304476472	0.709241120561858\\
0.51	0.341255928754291	0.752367096967051	0.703223584571264\\
0.51	0.353793780602429	0.763162841084325	0.697635496597419\\
0.51	0.366499761262991	0.773742494322337	0.692428303594619\\
0.51	0.379368276771857	0.784102202600443	0.687549132887068\\
0.51	0.392393618241769	0.794238304822092	0.682943679235329\\
0.51	0.405569970164763	0.804147337016097	0.678558662193045\\
0.51	0.418891418885081	0.813826036051075	0.674343787412869\\
0.51	0.432351961217132	0.823271342915544	0.670253187715552\\
0.51	0.445945513182735	0.832480405557786	0.666246357336161\\
0.51	0.459665918841652	0.841450581281375	0.662288623474186\\
0.51	0.473506959189401	0.850179438693979	0.65835122167319\\
0.51	0.487462361096317	0.85866475920876	0.654411055137885\\
0.51	0.501525806262096	0.866904538099407	0.65045022320647\\
0.51	0.515690940160279	0.874896985111485	0.64645540187496\\
0.51	0.529951380947623	0.882640524634436	0.642417151055155\\
0.51	0.544300728313781	0.890133795440129	0.638329210947586\\
0.51	0.558732572247415	0.897375649995373	0.634187835387497\\
0.51	0.57324050169555	0.904365153357279	0.629991194998375\\
0.51	0.587818113093873	0.911101581661729	0.625738868900925\\
0.51	0.602459018746568	0.917584420216504	0.621431431633141\\
0.51	0.617156855035314	0.923813361211857	0.617070132480553\\
0.51	0.631905290438127	0.92978830106243	0.612656657833836\\
0.51	0.6466980333399	0.935509337395467	0.608192963377087\\
0.51	0.661528839617671	0.940976765701173	0.603681161495106\\
0.51	0.676391519984904	0.94619107566195	0.599123449738607\\
0.51	0.691279947080357	0.951152947177946	0.594522067901589\\
0.51	0.706188062288412	0.955863246106974	0.58987927366239\\
0.51	0.721109882279076	0.960323019737426	0.585197329319937\\
0.51	0.736039505257226	0.964533492013186	0.580478494543978\\
0.51	0.75097111691199	0.968496058529893	0.575725022018314\\
0.51	0.765898996058536	0.972212281322139	0.570939154289558\\
0.51	0.780817519965826	0.975683883461272	0.566123121053172\\
0.51	0.795721169365275	0.978912743483559	0.561279136605559\\
0.51	0.810604533136476	0.981900889668376	0.556409397399202\\
0.51	0.825462312667457	0.984650494185936	0.551516079695732\\
0.51	0.840289325888138	0.987163867133889	0.546601337319641\\
0.51	0.855080510976839	0.989443450481768	0.541667299515141\\
0.51	0.86983092974082	0.991491811941914	0.536716068908599\\
0.51	0.884535770672926	0.993311638785082	0.531749719578144\\
0.51	0.899190351687411	0.994905731618374	0.526770295231794\\
0.51	0.913790122539012	0.996276998142666	0.521779807495101\\
0.51	0.928330666930242	0.997428446906011	0.516780234308684\\
0.51	0.942807704312699	0.998363181068902	0.51177351843587\\
0.51	0.95721709138901	0.999084392196567	0.50676156608017\\
0.51	0.971554823322703	0.999595354092743	0.501746245611876\\
0.51	0.985817034663989	0.999899416688637	0.496729386403004\\
0.51	1	1	0.491669087346906\\
0.525	0	0	0.710195569525497\\
0.525	0.000100583311362513	0.0141829653360114	0.71240460418299\\
0.525	0.000404645907256436	0.0284451766772965	0.714630690536905\\
0.525	0.000915607803432999	0.0427829086109896	0.716916312064106\\
0.525	0.00163681893109844	0.057192295687301	0.719289019360234\\
0.525	0.00257155309398959	0.0716693330697584	0.721780367061016\\
0.525	0.00372300185733414	0.0862098774609879	0.724425655492835\\
0.525	0.00509426838162598	0.100809648312589	0.727263413850304\\
0.525	0.00668836121491816	0.115464229327074	0.73033458997498\\
0.525	0.00850818805808555	0.13016907025918	0.733681422514709\\
0.525	0.0105565495182326	0.144919489023162	0.737345986077549\\
0.525	0.0128361328661109	0.159710674111862	0.741368418705642\\
0.525	0.0153495058140643	0.174537687332543	0.745784862958201\\
0.525	0.0180991103316243	0.189395466863524	0.750625176092311\\
0.525	0.0210872565164405	0.204278830634725	0.755910489868998\\
0.525	0.0243161165387281	0.219182480034174	0.761650724664453\\
0.525	0.0277877186778607	0.234101003941464	0.767842183867616\\
0.525	0.0315039414701067	0.24902888308801	0.77446537092041\\
0.525	0.0354665079868145	0.263960494742775	0.781483180780606\\
0.525	0.0396769802625737	0.278890117720924	0.788839618268073\\
0.525	0.0441367538930258	0.293811937711588	0.796459186325545\\
0.525	0.0488470528220538	0.308720052919643	0.804247066923455\\
0.525	0.0538089243380495	0.323608480015097	0.812090186161949\\
0.525	0.0590232342988274	0.338471160382329	0.819859213932693\\
0.525	0.064490662604533	0.3533019666601	0.827411499063639\\
0.525	0.0702116989375697	0.368094709561873	0.834594885813505\\
0.525	0.0761866387881432	0.382843144964686	0.841252300287181\\
0.525	0.0824155797834956	0.397540981253432	0.847226939729082\\
0.525	0.0888984183382709	0.412181886906127	0.852367847905042\\
0.525	0.0956348466427212	0.42675949830445	0.856535620027831\\
0.525	0.102624350004627	0.441267427752584	0.859607954619618\\
0.525	0.109866204559871	0.455699271686218	0.861484760266622\\
0.525	0.117359475365564	0.470048619052377	0.862092534251541\\
0.525	0.125103014888515	0.484309059839721	0.86138775806837\\
0.525	0.133095461900593	0.498474193737904	0.85935910089082\\
0.525	0.14133524079124	0.512537638903683	0.856028283765733\\
0.525	0.149820561306021	0.526493040810599	0.851449530870183\\
0.525	0.158549418718625	0.540334081158348	0.845707614739502\\
0.525	0.167519594442214	0.554054486817265	0.838914584343151\\
0.525	0.176728657084455	0.567648038782867	0.831205342370839\\
0.525	0.186173963948925	0.581108581114919	0.822732305417528\\
0.525	0.195852662983903	0.594430029835236	0.813659432944308\\
0.525	0.205761695177907	0.607606381758231	0.804155944104867\\
0.525	0.215897797399558	0.620631723228143	0.794390053406972\\
0.525	0.226257505677663	0.633500238737009	0.784523046062799\\
0.525	0.236837158915675	0.646206219397571	0.774703982861589\\
0.525	0.247632903032949	0.658744071245709	0.765065275211015\\
0.525	0.258640695523527	0.671108323347402	0.755719307772964\\
0.525	0.269856310421543	0.683293635685828	0.746756214005914\\
0.525	0.28127534366066	0.695294806804925	0.738242834604296\\
0.525	0.292893218813452	0.707106781186547	0.730222816007832\\
0.525	0.304705193195076	0.71872465633934	0.722717741087768\\
0.525	0.316706364314172	0.730143689578457	0.715729131143268\\
0.525	0.328891676652598	0.741359304476472	0.709241120561856\\
0.525	0.341255928754291	0.752367096967051	0.703223584571266\\
0.525	0.353793780602429	0.763162841084325	0.697635496597419\\
0.525	0.366499761262991	0.773742494322337	0.692428303594618\\
0.525	0.379368276771856	0.784102202600442	0.687549132887069\\
0.525	0.392393618241769	0.794238304822092	0.682943679235328\\
0.525	0.405569970164763	0.804147337016096	0.678558662193045\\
0.525	0.418891418885081	0.813826036051075	0.674343787412871\\
0.525	0.432351961217132	0.823271342915544	0.670253187715552\\
0.525	0.445945513182735	0.832480405557786	0.666246357336159\\
0.525	0.459665918841652	0.841450581281375	0.662288623474186\\
0.525	0.473506959189401	0.850179438693979	0.658351221673192\\
0.525	0.487462361096317	0.85866475920876	0.654411055137886\\
0.525	0.501525806262096	0.866904538099407	0.650450223206467\\
0.525	0.515690940160279	0.874896985111485	0.64645540187496\\
0.525	0.529951380947623	0.882640524634437	0.642417151055156\\
0.525	0.544300728313781	0.890133795440129	0.638329210947586\\
0.525	0.558732572247416	0.897375649995373	0.634187835387497\\
0.525	0.57324050169555	0.904365153357279	0.629991194998375\\
0.525	0.587818113093873	0.911101581661729	0.625738868900924\\
0.525	0.602459018746568	0.917584420216504	0.621431431633141\\
0.525	0.617156855035314	0.923813361211857	0.617070132480554\\
0.525	0.631905290438127	0.92978830106243	0.612656657833837\\
0.525	0.6466980333399	0.935509337395467	0.608192963377087\\
0.525	0.661528839617671	0.940976765701173	0.603681161495104\\
0.525	0.676391519984904	0.94619107566195	0.599123449738607\\
0.525	0.691279947080357	0.951152947177946	0.594522067901589\\
0.525	0.706188062288412	0.955863246106974	0.58987927366239\\
0.525	0.721109882279076	0.960323019737426	0.585197329319936\\
0.525	0.736039505257226	0.964533492013185	0.580478494543975\\
0.525	0.75097111691199	0.968496058529893	0.575725022018314\\
0.525	0.765898996058536	0.972212281322139	0.570939154289562\\
0.525	0.780817519965826	0.975683883461272	0.566123121053177\\
0.525	0.795721169365275	0.97891274348356	0.561279136605555\\
0.525	0.810604533136476	0.981900889668376	0.556409397399199\\
0.525	0.825462312667457	0.984650494185936	0.551516079695737\\
0.525	0.840289325888138	0.987163867133889	0.546601337319638\\
0.525	0.855080510976838	0.989443450481767	0.541667299515141\\
0.525	0.86983092974082	0.991491811941914	0.536716068908599\\
0.525	0.884535770672926	0.993311638785082	0.531749719578142\\
0.525	0.899190351687411	0.994905731618374	0.526770295231793\\
0.525	0.913790122539012	0.996276998142666	0.521779807495103\\
0.525	0.928330666930242	0.99742844690601	0.516780234308683\\
0.525	0.942807704312699	0.998363181068901	0.511773518435872\\
0.525	0.95721709138901	0.999084392196567	0.506761566080167\\
0.525	0.971554823322703	0.999595354092743	0.501746245611877\\
0.525	0.985817034663989	0.999899416688638	0.496729386403005\\
0.525	1	1	0.491669087346895\\
0.54	0	0	0.710195569525497\\
0.54	0.000100583311362513	0.0141829653360114	0.71240460418299\\
0.54	0.000404645907256436	0.0284451766772965	0.714630690536905\\
0.54	0.000915607803432999	0.0427829086109896	0.716916312064106\\
0.54	0.00163681893109844	0.057192295687301	0.719289019360234\\
0.54	0.00257155309398959	0.0716693330697584	0.721780367061016\\
0.54	0.00372300185733413	0.0862098774609879	0.724425655492835\\
0.54	0.00509426838162598	0.100809648312589	0.727263413850304\\
0.54	0.00668836121491815	0.115464229327074	0.73033458997498\\
0.54	0.00850818805808555	0.13016907025918	0.733681422514709\\
0.54	0.0105565495182326	0.144919489023162	0.737345986077549\\
0.54	0.0128361328661109	0.159710674111862	0.741368418705642\\
0.54	0.0153495058140643	0.174537687332543	0.745784862958201\\
0.54	0.0180991103316243	0.189395466863524	0.750625176092311\\
0.54	0.0210872565164405	0.204278830634725	0.755910489868999\\
0.54	0.0243161165387281	0.219182480034174	0.761650724664453\\
0.54	0.0277877186778607	0.234101003941464	0.767842183867616\\
0.54	0.0315039414701067	0.24902888308801	0.77446537092041\\
0.54	0.0354665079868145	0.263960494742775	0.781483180780606\\
0.54	0.0396769802625738	0.278890117720924	0.788839618268073\\
0.54	0.0441367538930258	0.293811937711588	0.796459186325543\\
0.54	0.0488470528220538	0.308720052919643	0.804247066923455\\
0.54	0.0538089243380495	0.323608480015096	0.81209018616195\\
0.54	0.0590232342988274	0.338471160382329	0.819859213932694\\
0.54	0.064490662604533	0.3533019666601	0.82741149906364\\
0.54	0.0702116989375697	0.368094709561873	0.834594885813504\\
0.54	0.0761866387881432	0.382843144964686	0.84125230028718\\
0.54	0.0824155797834956	0.397540981253432	0.847226939729081\\
0.54	0.0888984183382709	0.412181886906127	0.852367847905041\\
0.54	0.0956348466427212	0.42675949830445	0.85653562002783\\
0.54	0.102624350004627	0.441267427752584	0.859607954619617\\
0.54	0.109866204559871	0.455699271686218	0.86148476026662\\
0.54	0.117359475365564	0.470048619052377	0.86209253425154\\
0.54	0.125103014888515	0.484309059839721	0.861387758068367\\
0.54	0.133095461900593	0.498474193737904	0.85935910089082\\
0.54	0.14133524079124	0.512537638903683	0.856028283765738\\
0.54	0.149820561306021	0.526493040810599	0.851449530870184\\
0.54	0.158549418718625	0.540334081158348	0.845707614739508\\
0.54	0.167519594442214	0.554054486817265	0.838914584343153\\
0.54	0.176728657084455	0.567648038782867	0.831205342370835\\
0.54	0.186173963948925	0.581108581114919	0.822732305417527\\
0.54	0.195852662983903	0.594430029835237	0.813659432944306\\
0.54	0.205761695177907	0.607606381758231	0.804155944104871\\
0.54	0.215897797399558	0.620631723228144	0.794390053406969\\
0.54	0.226257505677663	0.633500238737009	0.784523046062799\\
0.54	0.236837158915675	0.646206219397571	0.77470398286159\\
0.54	0.247632903032949	0.658744071245709	0.765065275211018\\
0.54	0.258640695523527	0.671108323347402	0.755719307772959\\
0.54	0.269856310421543	0.683293635685828	0.746756214005909\\
0.54	0.28127534366066	0.695294806804924	0.738242834604294\\
0.54	0.292893218813452	0.707106781186547	0.730222816007837\\
0.54	0.304705193195075	0.71872465633934	0.722717741087768\\
0.54	0.316706364314172	0.730143689578457	0.715729131143268\\
0.54	0.328891676652598	0.741359304476473	0.709241120561856\\
0.54	0.341255928754291	0.752367096967051	0.703223584571263\\
0.54	0.353793780602429	0.763162841084325	0.69763549659742\\
0.54	0.366499761262991	0.773742494322337	0.692428303594619\\
0.54	0.379368276771856	0.784102202600442	0.687549132887068\\
0.54	0.392393618241769	0.794238304822092	0.682943679235329\\
0.54	0.405569970164763	0.804147337016097	0.678558662193044\\
0.54	0.418891418885081	0.813826036051075	0.67434378741287\\
0.54	0.432351961217133	0.823271342915544	0.670253187715551\\
0.54	0.445945513182735	0.832480405557786	0.666246357336159\\
0.54	0.459665918841652	0.841450581281375	0.662288623474187\\
0.54	0.473506959189401	0.850179438693979	0.65835122167319\\
0.54	0.487462361096317	0.85866475920876	0.654411055137886\\
0.54	0.501525806262096	0.866904538099407	0.650450223206469\\
0.54	0.515690940160279	0.874896985111485	0.646455401874959\\
0.54	0.529951380947623	0.882640524634436	0.642417151055156\\
0.54	0.544300728313782	0.890133795440129	0.638329210947587\\
0.54	0.558732572247415	0.897375649995373	0.634187835387497\\
0.54	0.57324050169555	0.904365153357279	0.629991194998375\\
0.54	0.587818113093873	0.911101581661729	0.625738868900924\\
0.54	0.602459018746568	0.917584420216504	0.62143143163314\\
0.54	0.617156855035314	0.923813361211857	0.617070132480553\\
0.54	0.631905290438127	0.92978830106243	0.612656657833838\\
0.54	0.6466980333399	0.935509337395467	0.608192963377088\\
0.54	0.661528839617671	0.940976765701173	0.603681161495104\\
0.54	0.676391519984903	0.94619107566195	0.599123449738607\\
0.54	0.691279947080357	0.951152947177946	0.594522067901588\\
0.54	0.706188062288412	0.955863246106974	0.58987927366239\\
0.54	0.721109882279076	0.960323019737426	0.585197329319939\\
0.54	0.736039505257226	0.964533492013186	0.580478494543974\\
0.54	0.75097111691199	0.968496058529893	0.575725022018313\\
0.54	0.765898996058536	0.972212281322139	0.570939154289559\\
0.54	0.780817519965825	0.975683883461272	0.566123121053172\\
0.54	0.795721169365275	0.978912743483559	0.561279136605561\\
0.54	0.810604533136476	0.981900889668376	0.556409397399199\\
0.54	0.825462312667457	0.984650494185936	0.551516079695736\\
0.54	0.840289325888138	0.987163867133889	0.546601337319643\\
0.54	0.855080510976838	0.989443450481768	0.541667299515137\\
0.54	0.86983092974082	0.991491811941914	0.536716068908598\\
0.54	0.884535770672926	0.993311638785082	0.531749719578143\\
0.54	0.899190351687411	0.994905731618374	0.526770295231792\\
0.54	0.913790122539012	0.996276998142666	0.521779807495102\\
0.54	0.928330666930241	0.997428446906011	0.516780234308686\\
0.54	0.942807704312699	0.998363181068902	0.511773518435872\\
0.54	0.95721709138901	0.999084392196567	0.506761566080169\\
0.54	0.971554823322703	0.999595354092744	0.501746245611876\\
0.54	0.985817034663989	0.999899416688638	0.496729386403002\\
0.54	1	1	0.491669087346898\\
0.555	0	0	0.710195569525497\\
0.555	0.000100583311362513	0.0141829653360114	0.71240460418299\\
0.555	0.000404645907256436	0.0284451766772965	0.714630690536905\\
0.555	0.000915607803432999	0.0427829086109896	0.716916312064106\\
0.555	0.00163681893109844	0.057192295687301	0.719289019360234\\
0.555	0.00257155309398959	0.0716693330697584	0.721780367061016\\
0.555	0.00372300185733413	0.0862098774609879	0.724425655492835\\
0.555	0.00509426838162598	0.100809648312589	0.727263413850304\\
0.555	0.00668836121491816	0.115464229327074	0.73033458997498\\
0.555	0.00850818805808554	0.13016907025918	0.733681422514709\\
0.555	0.0105565495182326	0.144919489023162	0.737345986077549\\
0.555	0.0128361328661109	0.159710674111862	0.741368418705642\\
0.555	0.0153495058140643	0.174537687332543	0.745784862958202\\
0.555	0.0180991103316243	0.189395466863524	0.750625176092311\\
0.555	0.0210872565164405	0.204278830634725	0.755910489868999\\
0.555	0.0243161165387281	0.219182480034174	0.761650724664453\\
0.555	0.0277877186778607	0.234101003941464	0.767842183867616\\
0.555	0.0315039414701067	0.24902888308801	0.77446537092041\\
0.555	0.0354665079868145	0.263960494742775	0.781483180780606\\
0.555	0.0396769802625737	0.278890117720924	0.788839618268073\\
0.555	0.0441367538930258	0.293811937711588	0.796459186325545\\
0.555	0.0488470528220538	0.308720052919643	0.804247066923454\\
0.555	0.0538089243380495	0.323608480015096	0.812090186161949\\
0.555	0.0590232342988274	0.338471160382329	0.819859213932694\\
0.555	0.064490662604533	0.3533019666601	0.82741149906364\\
0.555	0.0702116989375697	0.368094709561873	0.834594885813505\\
0.555	0.0761866387881432	0.382843144964686	0.841252300287178\\
0.555	0.0824155797834956	0.397540981253432	0.847226939729079\\
0.555	0.0888984183382709	0.412181886906127	0.85236784790504\\
0.555	0.0956348466427212	0.42675949830445	0.856535620027828\\
0.555	0.102624350004627	0.441267427752585	0.859607954619618\\
0.555	0.109866204559871	0.455699271686218	0.861484760266622\\
0.555	0.117359475365564	0.470048619052377	0.862092534251539\\
0.555	0.125103014888515	0.484309059839721	0.861387758068367\\
0.555	0.133095461900593	0.498474193737904	0.859359100890819\\
0.555	0.14133524079124	0.512537638903683	0.856028283765736\\
0.555	0.149820561306021	0.526493040810599	0.851449530870181\\
0.555	0.158549418718625	0.540334081158348	0.845707614739507\\
0.555	0.167519594442214	0.554054486817266	0.83891458434315\\
0.555	0.176728657084455	0.567648038782867	0.831205342370836\\
0.555	0.186173963948925	0.581108581114919	0.822732305417526\\
0.555	0.195852662983903	0.594430029835236	0.813659432944304\\
0.555	0.205761695177907	0.607606381758231	0.804155944104873\\
0.555	0.215897797399558	0.620631723228143	0.79439005340697\\
0.555	0.226257505677663	0.633500238737009	0.784523046062803\\
0.555	0.236837158915675	0.646206219397571	0.774703982861591\\
0.555	0.247632903032949	0.658744071245709	0.765065275211014\\
0.555	0.258640695523527	0.671108323347402	0.75571930777296\\
0.555	0.269856310421543	0.683293635685828	0.746756214005912\\
0.555	0.28127534366066	0.695294806804924	0.738242834604299\\
0.555	0.292893218813452	0.707106781186547	0.730222816007835\\
0.555	0.304705193195075	0.71872465633934	0.722717741087766\\
0.555	0.316706364314172	0.730143689578457	0.715729131143267\\
0.555	0.328891676652598	0.741359304476472	0.709241120561855\\
0.555	0.341255928754291	0.752367096967051	0.703223584571265\\
0.555	0.353793780602429	0.763162841084324	0.697635496597421\\
0.555	0.366499761262991	0.773742494322337	0.692428303594619\\
0.555	0.379368276771857	0.784102202600442	0.687549132887068\\
0.555	0.392393618241769	0.794238304822092	0.682943679235328\\
0.555	0.405569970164763	0.804147337016096	0.678558662193045\\
0.555	0.418891418885081	0.813826036051075	0.674343787412871\\
0.555	0.432351961217133	0.823271342915544	0.67025318771555\\
0.555	0.445945513182735	0.832480405557786	0.666246357336159\\
0.555	0.459665918841652	0.841450581281375	0.662288623474186\\
0.555	0.473506959189401	0.850179438693979	0.65835122167319\\
0.555	0.487462361096317	0.85866475920876	0.654411055137885\\
0.555	0.501525806262096	0.866904538099407	0.650450223206468\\
0.555	0.515690940160279	0.874896985111485	0.646455401874957\\
0.555	0.529951380947623	0.882640524634436	0.642417151055155\\
0.555	0.544300728313782	0.890133795440129	0.638329210947587\\
0.555	0.558732572247415	0.897375649995373	0.634187835387498\\
0.555	0.57324050169555	0.904365153357279	0.629991194998377\\
0.555	0.587818113093873	0.911101581661729	0.625738868900926\\
0.555	0.602459018746568	0.917584420216505	0.621431431633139\\
0.555	0.617156855035314	0.923813361211857	0.617070132480552\\
0.555	0.631905290438127	0.92978830106243	0.612656657833837\\
0.555	0.6466980333399	0.935509337395467	0.60819296337709\\
0.555	0.661528839617671	0.940976765701173	0.603681161495103\\
0.555	0.676391519984904	0.94619107566195	0.599123449738605\\
0.555	0.691279947080357	0.951152947177946	0.594522067901592\\
0.555	0.706188062288412	0.955863246106974	0.589879273662389\\
0.555	0.721109882279076	0.960323019737426	0.585197329319938\\
0.555	0.736039505257225	0.964533492013186	0.580478494543978\\
0.555	0.75097111691199	0.968496058529893	0.575725022018314\\
0.555	0.765898996058536	0.972212281322139	0.570939154289558\\
0.555	0.780817519965826	0.975683883461272	0.566123121053172\\
0.555	0.795721169365275	0.978912743483559	0.56127913660556\\
0.555	0.810604533136476	0.981900889668376	0.556409397399201\\
0.555	0.825462312667457	0.984650494185936	0.551516079695732\\
0.555	0.840289325888138	0.987163867133889	0.546601337319641\\
0.555	0.855080510976839	0.989443450481768	0.541667299515141\\
0.555	0.86983092974082	0.991491811941914	0.536716068908599\\
0.555	0.884535770672926	0.993311638785082	0.531749719578143\\
0.555	0.899190351687411	0.994905731618374	0.526770295231794\\
0.555	0.913790122539012	0.996276998142666	0.521779807495102\\
0.555	0.928330666930242	0.997428446906011	0.516780234308682\\
0.555	0.942807704312699	0.998363181068901	0.511773518435873\\
0.555	0.95721709138901	0.999084392196567	0.50676156608017\\
0.555	0.971554823322703	0.999595354092743	0.501746245611875\\
0.555	0.985817034663989	0.999899416688637	0.496729386403004\\
0.555	1	1	0.491669087346906\\
0.57	0	0	0.710195569525497\\
0.57	0.000100583311362513	0.0141829653360114	0.71240460418299\\
0.57	0.000404645907256436	0.0284451766772965	0.714630690536905\\
0.57	0.000915607803432999	0.0427829086109896	0.716916312064106\\
0.57	0.00163681893109844	0.057192295687301	0.719289019360234\\
0.57	0.00257155309398959	0.0716693330697584	0.721780367061016\\
0.57	0.00372300185733413	0.0862098774609879	0.724425655492835\\
0.57	0.00509426838162598	0.100809648312589	0.727263413850304\\
0.57	0.00668836121491815	0.115464229327074	0.73033458997498\\
0.57	0.00850818805808555	0.13016907025918	0.733681422514709\\
0.57	0.0105565495182326	0.144919489023162	0.737345986077549\\
0.57	0.0128361328661109	0.159710674111862	0.741368418705642\\
0.57	0.0153495058140643	0.174537687332543	0.745784862958201\\
0.57	0.0180991103316243	0.189395466863524	0.750625176092311\\
0.57	0.0210872565164405	0.204278830634725	0.755910489868999\\
0.57	0.0243161165387281	0.219182480034174	0.761650724664453\\
0.57	0.0277877186778607	0.234101003941464	0.767842183867616\\
0.57	0.0315039414701067	0.24902888308801	0.77446537092041\\
0.57	0.0354665079868146	0.263960494742775	0.781483180780606\\
0.57	0.0396769802625738	0.278890117720924	0.788839618268073\\
0.57	0.0441367538930258	0.293811937711588	0.796459186325545\\
0.57	0.0488470528220538	0.308720052919643	0.804247066923455\\
0.57	0.0538089243380495	0.323608480015096	0.812090186161949\\
0.57	0.0590232342988274	0.338471160382329	0.819859213932694\\
0.57	0.064490662604533	0.3533019666601	0.827411499063639\\
0.57	0.0702116989375697	0.368094709561873	0.834594885813504\\
0.57	0.0761866387881432	0.382843144964686	0.84125230028718\\
0.57	0.0824155797834956	0.397540981253432	0.847226939729081\\
0.57	0.0888984183382709	0.412181886906127	0.852367847905041\\
0.57	0.0956348466427212	0.42675949830445	0.856535620027827\\
0.57	0.102624350004627	0.441267427752584	0.859607954619618\\
0.57	0.109866204559871	0.455699271686218	0.86148476026662\\
0.57	0.117359475365564	0.470048619052377	0.862092534251542\\
0.57	0.125103014888515	0.484309059839721	0.861387758068368\\
0.57	0.133095461900593	0.498474193737904	0.859359100890818\\
0.57	0.14133524079124	0.512537638903683	0.856028283765736\\
0.57	0.149820561306021	0.526493040810599	0.851449530870182\\
0.57	0.158549418718625	0.540334081158348	0.845707614739505\\
0.57	0.167519594442213	0.554054486817265	0.83891458434315\\
0.57	0.176728657084455	0.567648038782867	0.831205342370836\\
0.57	0.186173963948925	0.581108581114919	0.822732305417525\\
0.57	0.195852662983903	0.594430029835236	0.81365943294431\\
0.57	0.205761695177907	0.607606381758231	0.804155944104874\\
0.57	0.215897797399558	0.620631723228144	0.79439005340697\\
0.57	0.226257505677663	0.633500238737009	0.7845230460628\\
0.57	0.236837158915675	0.646206219397571	0.77470398286159\\
0.57	0.247632903032949	0.658744071245709	0.765065275211013\\
0.57	0.258640695523528	0.671108323347402	0.755719307772963\\
0.57	0.269856310421543	0.683293635685828	0.746756214005914\\
0.57	0.28127534366066	0.695294806804925	0.738242834604297\\
0.57	0.292893218813453	0.707106781186548	0.730222816007835\\
0.57	0.304705193195075	0.71872465633934	0.722717741087765\\
0.57	0.316706364314172	0.730143689578457	0.715729131143267\\
0.57	0.328891676652598	0.741359304476472	0.709241120561857\\
0.57	0.341255928754291	0.752367096967051	0.703223584571265\\
0.57	0.353793780602429	0.763162841084325	0.697635496597418\\
0.57	0.366499761262991	0.773742494322337	0.692428303594618\\
0.57	0.379368276771857	0.784102202600442	0.68754913288707\\
0.57	0.392393618241769	0.794238304822092	0.682943679235328\\
0.57	0.405569970164763	0.804147337016096	0.678558662193044\\
0.57	0.418891418885081	0.813826036051075	0.674343787412874\\
0.57	0.432351961217133	0.823271342915545	0.670253187715551\\
0.57	0.445945513182735	0.832480405557786	0.666246357336158\\
0.57	0.459665918841652	0.841450581281375	0.662288623474188\\
0.57	0.473506959189401	0.850179438693979	0.658351221673191\\
0.57	0.487462361096317	0.85866475920876	0.654411055137885\\
0.57	0.501525806262096	0.866904538099407	0.650450223206468\\
0.57	0.515690940160279	0.874896985111485	0.646455401874958\\
0.57	0.529951380947623	0.882640524634436	0.642417151055155\\
0.57	0.544300728313782	0.890133795440129	0.638329210947586\\
0.57	0.558732572247415	0.897375649995373	0.634187835387496\\
0.57	0.57324050169555	0.904365153357279	0.629991194998375\\
0.57	0.587818113093873	0.911101581661729	0.625738868900925\\
0.57	0.602459018746568	0.917584420216504	0.621431431633141\\
0.57	0.617156855035314	0.923813361211857	0.617070132480554\\
0.57	0.631905290438127	0.92978830106243	0.612656657833837\\
0.57	0.6466980333399	0.935509337395467	0.608192963377087\\
0.57	0.661528839617671	0.940976765701173	0.603681161495106\\
0.57	0.676391519984904	0.946191075661951	0.599123449738606\\
0.57	0.691279947080357	0.951152947177946	0.594522067901589\\
0.57	0.706188062288412	0.955863246106974	0.58987927366239\\
0.57	0.721109882279076	0.960323019737426	0.585197329319937\\
0.57	0.736039505257226	0.964533492013186	0.580478494543975\\
0.57	0.75097111691199	0.968496058529893	0.575725022018312\\
0.57	0.765898996058536	0.972212281322139	0.570939154289563\\
0.57	0.780817519965826	0.975683883461272	0.566123121053174\\
0.57	0.795721169365275	0.978912743483559	0.561279136605556\\
0.57	0.810604533136476	0.981900889668376	0.556409397399201\\
0.57	0.825462312667457	0.984650494185936	0.551516079695736\\
0.57	0.840289325888138	0.987163867133889	0.546601337319641\\
0.57	0.855080510976838	0.989443450481768	0.541667299515137\\
0.57	0.86983092974082	0.991491811941914	0.536716068908597\\
0.57	0.884535770672926	0.993311638785082	0.531749719578144\\
0.57	0.899190351687411	0.994905731618374	0.526770295231794\\
0.57	0.913790122539012	0.996276998142666	0.521779807495101\\
0.57	0.928330666930242	0.997428446906011	0.516780234308684\\
0.57	0.942807704312699	0.998363181068902	0.511773518435872\\
0.57	0.95721709138901	0.999084392196567	0.50676156608017\\
0.57	0.971554823322704	0.999595354092743	0.501746245611874\\
0.57	0.985817034663989	0.999899416688637	0.496729386403003\\
0.57	1	1	0.491669087346906\\
0.585	0	0	0.710195569525497\\
0.585	0.000100583311362513	0.0141829653360114	0.71240460418299\\
0.585	0.000404645907256436	0.0284451766772965	0.714630690536905\\
0.585	0.000915607803432999	0.0427829086109896	0.716916312064106\\
0.585	0.00163681893109844	0.057192295687301	0.719289019360234\\
0.585	0.00257155309398959	0.0716693330697584	0.721780367061016\\
0.585	0.00372300185733413	0.0862098774609879	0.724425655492835\\
0.585	0.00509426838162598	0.100809648312589	0.727263413850304\\
0.585	0.00668836121491816	0.115464229327074	0.73033458997498\\
0.585	0.00850818805808554	0.13016907025918	0.733681422514709\\
0.585	0.0105565495182326	0.144919489023162	0.737345986077549\\
0.585	0.0128361328661109	0.159710674111862	0.741368418705642\\
0.585	0.0153495058140643	0.174537687332543	0.745784862958201\\
0.585	0.0180991103316243	0.189395466863524	0.750625176092311\\
0.585	0.0210872565164405	0.204278830634725	0.755910489868999\\
0.585	0.0243161165387281	0.219182480034174	0.761650724664453\\
0.585	0.0277877186778607	0.234101003941464	0.767842183867616\\
0.585	0.0315039414701067	0.24902888308801	0.77446537092041\\
0.585	0.0354665079868145	0.263960494742775	0.781483180780606\\
0.585	0.0396769802625738	0.278890117720924	0.788839618268073\\
0.585	0.0441367538930258	0.293811937711588	0.796459186325544\\
0.585	0.0488470528220538	0.308720052919643	0.804247066923454\\
0.585	0.0538089243380495	0.323608480015096	0.81209018616195\\
0.585	0.0590232342988274	0.338471160382329	0.819859213932692\\
0.585	0.064490662604533	0.3533019666601	0.827411499063639\\
0.585	0.0702116989375697	0.368094709561873	0.834594885813505\\
0.585	0.0761866387881432	0.382843144964686	0.841252300287179\\
0.585	0.0824155797834956	0.397540981253432	0.847226939729082\\
0.585	0.0888984183382709	0.412181886906127	0.852367847905043\\
0.585	0.0956348466427212	0.42675949830445	0.856535620027828\\
0.585	0.102624350004627	0.441267427752584	0.859607954619617\\
0.585	0.109866204559871	0.455699271686218	0.861484760266623\\
0.585	0.117359475365564	0.470048619052377	0.862092534251542\\
0.585	0.125103014888515	0.484309059839721	0.86138775806837\\
0.585	0.133095461900593	0.498474193737904	0.859359100890822\\
0.585	0.14133524079124	0.512537638903683	0.856028283765738\\
0.585	0.149820561306021	0.526493040810599	0.851449530870183\\
0.585	0.158549418718625	0.540334081158348	0.845707614739505\\
0.585	0.167519594442214	0.554054486817266	0.83891458434315\\
0.585	0.176728657084455	0.567648038782867	0.831205342370836\\
0.585	0.186173963948925	0.581108581114919	0.82273230541753\\
0.585	0.195852662983903	0.594430029835237	0.81365943294431\\
0.585	0.205761695177907	0.607606381758231	0.804155944104866\\
0.585	0.215897797399558	0.620631723228143	0.794390053406967\\
0.585	0.226257505677663	0.633500238737009	0.784523046062799\\
0.585	0.236837158915675	0.646206219397571	0.774703982861585\\
0.585	0.247632903032949	0.658744071245709	0.765065275211015\\
0.585	0.258640695523527	0.671108323347402	0.755719307772963\\
0.585	0.269856310421543	0.683293635685828	0.74675621400591\\
0.585	0.28127534366066	0.695294806804924	0.738242834604297\\
0.585	0.292893218813452	0.707106781186548	0.730222816007836\\
0.585	0.304705193195075	0.71872465633934	0.722717741087765\\
0.585	0.316706364314172	0.730143689578457	0.715729131143267\\
0.585	0.328891676652598	0.741359304476472	0.709241120561856\\
0.585	0.341255928754291	0.752367096967051	0.703223584571267\\
0.585	0.353793780602429	0.763162841084325	0.697635496597418\\
0.585	0.366499761262991	0.773742494322337	0.692428303594618\\
0.585	0.379368276771857	0.784102202600443	0.687549132887069\\
0.585	0.392393618241769	0.794238304822092	0.682943679235329\\
0.585	0.405569970164763	0.804147337016097	0.678558662193044\\
0.585	0.418891418885081	0.813826036051075	0.674343787412871\\
0.585	0.432351961217133	0.823271342915545	0.670253187715554\\
0.585	0.445945513182735	0.832480405557786	0.666246357336157\\
0.585	0.459665918841652	0.841450581281375	0.662288623474188\\
0.585	0.473506959189401	0.850179438693979	0.658351221673191\\
0.585	0.487462361096317	0.85866475920876	0.654411055137885\\
0.585	0.501525806262096	0.866904538099407	0.650450223206469\\
0.585	0.515690940160279	0.874896985111485	0.646455401874957\\
0.585	0.529951380947623	0.882640524634436	0.642417151055155\\
0.585	0.544300728313781	0.890133795440129	0.63832921094759\\
0.585	0.558732572247415	0.897375649995373	0.634187835387498\\
0.585	0.57324050169555	0.904365153357279	0.629991194998374\\
0.585	0.587818113093873	0.911101581661729	0.625738868900924\\
0.585	0.602459018746568	0.917584420216505	0.621431431633139\\
0.585	0.617156855035314	0.923813361211857	0.617070132480553\\
0.585	0.631905290438127	0.92978830106243	0.612656657833837\\
0.585	0.6466980333399	0.935509337395467	0.608192963377088\\
0.585	0.661528839617671	0.940976765701173	0.603681161495105\\
0.585	0.676391519984904	0.946191075661951	0.599123449738606\\
0.585	0.691279947080357	0.951152947177946	0.594522067901588\\
0.585	0.706188062288412	0.955863246106974	0.589879273662391\\
0.585	0.721109882279076	0.960323019737426	0.585197329319936\\
0.585	0.736039505257225	0.964533492013185	0.580478494543977\\
0.585	0.75097111691199	0.968496058529893	0.575725022018314\\
0.585	0.765898996058536	0.972212281322139	0.570939154289557\\
0.585	0.780817519965825	0.975683883461272	0.566123121053176\\
0.585	0.795721169365275	0.97891274348356	0.561279136605561\\
0.585	0.810604533136476	0.981900889668376	0.556409397399198\\
0.585	0.825462312667457	0.984650494185936	0.551516079695735\\
0.585	0.840289325888138	0.987163867133889	0.546601337319641\\
0.585	0.855080510976839	0.989443450481767	0.541667299515139\\
0.585	0.86983092974082	0.991491811941914	0.536716068908599\\
0.585	0.884535770672926	0.993311638785082	0.531749719578142\\
0.585	0.899190351687411	0.994905731618374	0.526770295231793\\
0.585	0.913790122539012	0.996276998142666	0.521779807495101\\
0.585	0.928330666930241	0.99742844690601	0.516780234308686\\
0.585	0.942807704312699	0.998363181068902	0.511773518435872\\
0.585	0.95721709138901	0.999084392196567	0.506761566080168\\
0.585	0.971554823322704	0.999595354092743	0.501746245611878\\
0.585	0.985817034663989	0.999899416688638	0.496729386403004\\
0.585	1	1	0.491669087346897\\
0.6	0	0	0.710195569525497\\
0.6	0.000100583311362513	0.0141829653360114	0.71240460418299\\
0.6	0.000404645907256436	0.0284451766772965	0.714630690536905\\
0.6	0.000915607803432999	0.0427829086109896	0.716916312064106\\
0.6	0.00163681893109844	0.057192295687301	0.719289019360234\\
0.6	0.00257155309398959	0.0716693330697584	0.721780367061016\\
0.6	0.00372300185733413	0.0862098774609879	0.724425655492835\\
0.6	0.00509426838162598	0.100809648312589	0.727263413850304\\
0.6	0.00668836121491816	0.115464229327074	0.73033458997498\\
0.6	0.00850818805808555	0.13016907025918	0.733681422514709\\
0.6	0.0105565495182326	0.144919489023162	0.737345986077549\\
0.6	0.0128361328661109	0.159710674111862	0.741368418705642\\
0.6	0.0153495058140643	0.174537687332543	0.745784862958201\\
0.6	0.0180991103316243	0.189395466863524	0.750625176092311\\
0.6	0.0210872565164405	0.204278830634725	0.755910489868999\\
0.6	0.0243161165387281	0.219182480034174	0.761650724664453\\
0.6	0.0277877186778607	0.234101003941464	0.767842183867616\\
0.6	0.0315039414701067	0.24902888308801	0.77446537092041\\
0.6	0.0354665079868145	0.263960494742775	0.781483180780606\\
0.6	0.0396769802625738	0.278890117720924	0.788839618268073\\
0.6	0.0441367538930258	0.293811937711588	0.796459186325544\\
0.6	0.0488470528220538	0.308720052919643	0.804247066923455\\
0.6	0.0538089243380495	0.323608480015096	0.81209018616195\\
0.6	0.0590232342988274	0.338471160382329	0.819859213932693\\
0.6	0.064490662604533	0.3533019666601	0.827411499063638\\
0.6	0.0702116989375697	0.368094709561873	0.834594885813504\\
0.6	0.0761866387881432	0.382843144964686	0.841252300287179\\
0.6	0.0824155797834956	0.397540981253432	0.84722693972908\\
0.6	0.0888984183382709	0.412181886906127	0.852367847905042\\
0.6	0.0956348466427212	0.42675949830445	0.856535620027832\\
0.6	0.102624350004627	0.441267427752584	0.859607954619617\\
0.6	0.109866204559871	0.455699271686218	0.861484760266622\\
0.6	0.117359475365564	0.470048619052377	0.862092534251541\\
0.6	0.125103014888515	0.484309059839721	0.86138775806837\\
0.6	0.133095461900593	0.498474193737904	0.859359100890822\\
0.6	0.14133524079124	0.512537638903683	0.856028283765735\\
0.6	0.149820561306021	0.526493040810599	0.851449530870183\\
0.6	0.158549418718625	0.540334081158348	0.845707614739505\\
0.6	0.167519594442213	0.554054486817265	0.838914584343149\\
0.6	0.176728657084455	0.567648038782867	0.831205342370836\\
0.6	0.186173963948925	0.581108581114919	0.822732305417527\\
0.6	0.195852662983903	0.594430029835236	0.813659432944305\\
0.6	0.205761695177907	0.60760638175823	0.804155944104865\\
0.6	0.215897797399558	0.620631723228143	0.794390053406969\\
0.6	0.226257505677663	0.633500238737009	0.7845230460628\\
0.6	0.236837158915675	0.646206219397571	0.774703982861588\\
0.6	0.247632903032949	0.658744071245709	0.765065275211017\\
0.6	0.258640695523527	0.671108323347402	0.755719307772962\\
0.6	0.269856310421543	0.683293635685828	0.746756214005911\\
0.6	0.28127534366066	0.695294806804924	0.738242834604298\\
0.6	0.292893218813452	0.707106781186548	0.730222816007834\\
0.6	0.304705193195075	0.71872465633934	0.722717741087765\\
0.6	0.316706364314172	0.730143689578457	0.715729131143269\\
0.6	0.328891676652598	0.741359304476472	0.709241120561858\\
0.6	0.341255928754291	0.752367096967051	0.703223584571264\\
0.6	0.353793780602429	0.763162841084325	0.697635496597419\\
0.6	0.366499761262991	0.773742494322337	0.692428303594619\\
0.6	0.379368276771856	0.784102202600443	0.687549132887068\\
0.6	0.392393618241769	0.794238304822092	0.682943679235327\\
0.6	0.405569970164763	0.804147337016096	0.678558662193045\\
0.6	0.418891418885081	0.813826036051075	0.674343787412871\\
0.6	0.432351961217132	0.823271342915544	0.670253187715552\\
0.6	0.445945513182735	0.832480405557786	0.666246357336159\\
0.6	0.459665918841652	0.841450581281375	0.662288623474187\\
0.6	0.473506959189401	0.850179438693979	0.658351221673191\\
0.6	0.487462361096317	0.85866475920876	0.654411055137885\\
0.6	0.501525806262096	0.866904538099407	0.65045022320647\\
0.6	0.515690940160279	0.874896985111485	0.646455401874959\\
0.6	0.529951380947623	0.882640524634436	0.642417151055154\\
0.6	0.544300728313782	0.890133795440129	0.638329210947587\\
0.6	0.558732572247415	0.897375649995373	0.634187835387498\\
0.6	0.57324050169555	0.904365153357279	0.629991194998376\\
0.6	0.587818113093873	0.911101581661729	0.625738868900925\\
0.6	0.602459018746568	0.917584420216504	0.621431431633139\\
0.6	0.617156855035314	0.923813361211857	0.617070132480555\\
0.6	0.631905290438127	0.92978830106243	0.612656657833838\\
0.6	0.6466980333399	0.935509337395467	0.608192963377087\\
0.6	0.661528839617671	0.940976765701173	0.603681161495105\\
0.6	0.676391519984903	0.94619107566195	0.599123449738607\\
0.6	0.691279947080357	0.951152947177946	0.594522067901588\\
0.6	0.706188062288412	0.955863246106974	0.589879273662391\\
0.6	0.721109882279076	0.960323019737426	0.585197329319939\\
0.6	0.736039505257226	0.964533492013186	0.580478494543975\\
0.6	0.75097111691199	0.968496058529893	0.575725022018313\\
0.6	0.765898996058536	0.972212281322139	0.57093915428956\\
0.6	0.780817519965826	0.975683883461272	0.566123121053173\\
0.6	0.795721169365275	0.978912743483559	0.561279136605561\\
0.6	0.810604533136476	0.981900889668376	0.556409397399199\\
0.6	0.825462312667457	0.984650494185936	0.551516079695735\\
0.6	0.840289325888138	0.987163867133889	0.546601337319642\\
0.6	0.855080510976838	0.989443450481767	0.541667299515139\\
0.6	0.86983092974082	0.991491811941914	0.536716068908598\\
0.6	0.884535770672926	0.993311638785082	0.531749719578143\\
0.6	0.899190351687411	0.994905731618374	0.526770295231792\\
0.6	0.913790122539012	0.996276998142666	0.521779807495101\\
0.6	0.928330666930242	0.99742844690601	0.516780234308683\\
0.6	0.942807704312699	0.998363181068902	0.511773518435872\\
0.6	0.95721709138901	0.999084392196567	0.506761566080169\\
0.6	0.971554823322703	0.999595354092743	0.501746245611876\\
0.6	0.985817034663988	0.999899416688637	0.496729386403004\\
0.6	1	1	0.491669087346906\\
0.615	0	0	0.710195569525497\\
0.615	0.000100583311362513	0.0141829653360114	0.71240460418299\\
0.615	0.000404645907256436	0.0284451766772965	0.714630690536905\\
0.615	0.000915607803432999	0.0427829086109896	0.716916312064106\\
0.615	0.00163681893109844	0.057192295687301	0.719289019360234\\
0.615	0.00257155309398959	0.0716693330697584	0.721780367061016\\
0.615	0.00372300185733414	0.0862098774609879	0.724425655492835\\
0.615	0.00509426838162598	0.100809648312589	0.727263413850304\\
0.615	0.00668836121491816	0.115464229327074	0.73033458997498\\
0.615	0.00850818805808555	0.13016907025918	0.733681422514709\\
0.615	0.0105565495182326	0.144919489023162	0.737345986077549\\
0.615	0.0128361328661109	0.159710674111862	0.741368418705642\\
0.615	0.0153495058140643	0.174537687332543	0.745784862958202\\
0.615	0.0180991103316243	0.189395466863524	0.750625176092311\\
0.615	0.0210872565164405	0.204278830634725	0.755910489868998\\
0.615	0.0243161165387281	0.219182480034174	0.761650724664453\\
0.615	0.0277877186778607	0.234101003941464	0.767842183867616\\
0.615	0.0315039414701067	0.24902888308801	0.77446537092041\\
0.615	0.0354665079868145	0.263960494742775	0.781483180780606\\
0.615	0.0396769802625738	0.278890117720924	0.788839618268073\\
0.615	0.0441367538930258	0.293811937711588	0.796459186325544\\
0.615	0.0488470528220538	0.308720052919643	0.804247066923455\\
0.615	0.0538089243380495	0.323608480015096	0.812090186161951\\
0.615	0.0590232342988274	0.338471160382329	0.819859213932694\\
0.615	0.064490662604533	0.3533019666601	0.82741149906364\\
0.615	0.0702116989375697	0.368094709561873	0.834594885813504\\
0.615	0.0761866387881432	0.382843144964686	0.841252300287181\\
0.615	0.0824155797834956	0.397540981253432	0.847226939729082\\
0.615	0.0888984183382709	0.412181886906127	0.85236784790504\\
0.615	0.0956348466427212	0.42675949830445	0.856535620027828\\
0.615	0.102624350004627	0.441267427752584	0.859607954619618\\
0.615	0.109866204559871	0.455699271686218	0.86148476026662\\
0.615	0.117359475365564	0.470048619052377	0.86209253425154\\
0.615	0.125103014888515	0.484309059839721	0.861387758068367\\
0.615	0.133095461900593	0.498474193737904	0.859359100890819\\
0.615	0.14133524079124	0.512537638903683	0.856028283765735\\
0.615	0.149820561306021	0.526493040810599	0.851449530870184\\
0.615	0.158549418718625	0.540334081158348	0.845707614739504\\
0.615	0.167519594442213	0.554054486817265	0.838914584343149\\
0.615	0.176728657084455	0.567648038782867	0.831205342370836\\
0.615	0.186173963948925	0.581108581114919	0.822732305417531\\
0.615	0.195852662983903	0.594430029835237	0.813659432944305\\
0.615	0.205761695177907	0.607606381758231	0.804155944104867\\
0.615	0.215897797399558	0.620631723228143	0.794390053406972\\
0.615	0.226257505677663	0.633500238737009	0.784523046062803\\
0.615	0.236837158915675	0.646206219397571	0.774703982861588\\
0.615	0.247632903032949	0.658744071245709	0.765065275211016\\
0.615	0.258640695523528	0.671108323347402	0.755719307772964\\
0.615	0.269856310421543	0.683293635685828	0.746756214005911\\
0.615	0.28127534366066	0.695294806804924	0.738242834604296\\
0.615	0.292893218813452	0.707106781186547	0.730222816007834\\
0.615	0.304705193195076	0.71872465633934	0.72271774108777\\
0.615	0.316706364314172	0.730143689578457	0.715729131143268\\
0.615	0.328891676652598	0.741359304476472	0.709241120561853\\
0.615	0.341255928754291	0.752367096967051	0.703223584571265\\
0.615	0.353793780602429	0.763162841084324	0.69763549659742\\
0.615	0.366499761262991	0.773742494322337	0.692428303594618\\
0.615	0.379368276771857	0.784102202600442	0.687549132887069\\
0.615	0.392393618241769	0.794238304822092	0.68294367923533\\
0.615	0.405569970164763	0.804147337016097	0.678558662193044\\
0.615	0.418891418885081	0.813826036051075	0.674343787412871\\
0.615	0.432351961217133	0.823271342915545	0.670253187715552\\
0.615	0.445945513182735	0.832480405557786	0.666246357336159\\
0.615	0.459665918841652	0.841450581281375	0.662288623474187\\
0.615	0.473506959189401	0.850179438693979	0.658351221673191\\
0.615	0.487462361096317	0.85866475920876	0.654411055137885\\
0.615	0.501525806262096	0.866904538099407	0.65045022320647\\
0.615	0.515690940160279	0.874896985111485	0.646455401874959\\
0.615	0.529951380947623	0.882640524634437	0.642417151055154\\
0.615	0.544300728313782	0.890133795440129	0.638329210947585\\
0.615	0.558732572247415	0.897375649995372	0.634187835387499\\
0.615	0.57324050169555	0.904365153357279	0.629991194998376\\
0.615	0.587818113093873	0.911101581661729	0.625738868900922\\
0.615	0.602459018746568	0.917584420216504	0.62143143163314\\
0.615	0.617156855035314	0.923813361211857	0.617070132480552\\
0.615	0.631905290438127	0.92978830106243	0.612656657833839\\
0.615	0.6466980333399	0.935509337395467	0.608192963377088\\
0.615	0.661528839617671	0.940976765701173	0.603681161495106\\
0.615	0.676391519984903	0.946191075661951	0.599123449738606\\
0.615	0.691279947080358	0.951152947177946	0.594522067901587\\
0.615	0.706188062288412	0.955863246106974	0.589879273662388\\
0.615	0.721109882279076	0.960323019737426	0.585197329319936\\
0.615	0.736039505257226	0.964533492013185	0.580478494543977\\
0.615	0.75097111691199	0.968496058529893	0.575725022018314\\
0.615	0.765898996058536	0.972212281322139	0.570939154289559\\
0.615	0.780817519965826	0.975683883461272	0.566123121053174\\
0.615	0.795721169365275	0.978912743483559	0.561279136605559\\
0.615	0.810604533136476	0.981900889668376	0.556409397399199\\
0.615	0.825462312667457	0.984650494185936	0.551516079695733\\
0.615	0.840289325888138	0.987163867133889	0.546601337319644\\
0.615	0.855080510976839	0.989443450481768	0.541667299515141\\
0.615	0.86983092974082	0.991491811941915	0.536716068908597\\
0.615	0.884535770672926	0.993311638785082	0.531749719578143\\
0.615	0.899190351687411	0.994905731618374	0.526770295231793\\
0.615	0.913790122539012	0.996276998142666	0.521779807495101\\
0.615	0.928330666930241	0.997428446906011	0.516780234308685\\
0.615	0.942807704312699	0.998363181068902	0.511773518435872\\
0.615	0.95721709138901	0.999084392196567	0.506761566080169\\
0.615	0.971554823322704	0.999595354092744	0.501746245611877\\
0.615	0.985817034663989	0.999899416688638	0.496729386403002\\
0.615	1	1	0.491669087346898\\
0.63	0	0	0.710195569525497\\
0.63	0.000100583311362513	0.0141829653360114	0.71240460418299\\
0.63	0.000404645907256436	0.0284451766772965	0.714630690536905\\
0.63	0.000915607803432999	0.0427829086109896	0.716916312064106\\
0.63	0.00163681893109844	0.057192295687301	0.719289019360234\\
0.63	0.00257155309398959	0.0716693330697584	0.721780367061016\\
0.63	0.00372300185733414	0.0862098774609879	0.724425655492835\\
0.63	0.00509426838162598	0.100809648312589	0.727263413850304\\
0.63	0.00668836121491816	0.115464229327074	0.73033458997498\\
0.63	0.00850818805808555	0.13016907025918	0.733681422514709\\
0.63	0.0105565495182326	0.144919489023162	0.737345986077549\\
0.63	0.0128361328661109	0.159710674111862	0.741368418705642\\
0.63	0.0153495058140643	0.174537687332543	0.745784862958202\\
0.63	0.0180991103316243	0.189395466863524	0.750625176092311\\
0.63	0.0210872565164405	0.204278830634725	0.755910489868999\\
0.63	0.0243161165387281	0.219182480034174	0.761650724664453\\
0.63	0.0277877186778607	0.234101003941464	0.767842183867616\\
0.63	0.0315039414701067	0.24902888308801	0.77446537092041\\
0.63	0.0354665079868145	0.263960494742775	0.781483180780606\\
0.63	0.0396769802625738	0.278890117720924	0.788839618268073\\
0.63	0.0441367538930258	0.293811937711588	0.796459186325544\\
0.63	0.0488470528220538	0.308720052919643	0.804247066923455\\
0.63	0.0538089243380495	0.323608480015096	0.81209018616195\\
0.63	0.0590232342988274	0.338471160382329	0.819859213932693\\
0.63	0.064490662604533	0.3533019666601	0.827411499063641\\
0.63	0.0702116989375697	0.368094709561873	0.834594885813505\\
0.63	0.0761866387881432	0.382843144964686	0.841252300287179\\
0.63	0.0824155797834956	0.397540981253432	0.84722693972908\\
0.63	0.0888984183382709	0.412181886906127	0.852367847905041\\
0.63	0.0956348466427212	0.42675949830445	0.856535620027829\\
0.63	0.102624350004627	0.441267427752584	0.859607954619618\\
0.63	0.109866204559871	0.455699271686218	0.861484760266622\\
0.63	0.117359475365564	0.470048619052377	0.86209253425154\\
0.63	0.125103014888515	0.484309059839721	0.861387758068367\\
0.63	0.133095461900593	0.498474193737904	0.859359100890821\\
0.63	0.14133524079124	0.512537638903683	0.85602828376574\\
0.63	0.149820561306021	0.526493040810599	0.851449530870183\\
0.63	0.158549418718625	0.540334081158348	0.845707614739501\\
0.63	0.167519594442214	0.554054486817265	0.838914584343147\\
0.63	0.176728657084455	0.567648038782867	0.831205342370838\\
0.63	0.186173963948925	0.581108581114919	0.822732305417527\\
0.63	0.195852662983903	0.594430029835236	0.813659432944304\\
0.63	0.205761695177907	0.60760638175823	0.80415594410487\\
0.63	0.215897797399558	0.620631723228143	0.794390053406975\\
0.63	0.226257505677663	0.633500238737009	0.7845230460628\\
0.63	0.236837158915675	0.646206219397571	0.774703982861591\\
0.63	0.247632903032949	0.658744071245709	0.765065275211018\\
0.63	0.258640695523527	0.671108323347402	0.755719307772962\\
0.63	0.269856310421543	0.683293635685828	0.746756214005911\\
0.63	0.28127534366066	0.695294806804924	0.738242834604299\\
0.63	0.292893218813452	0.707106781186548	0.730222816007835\\
0.63	0.304705193195075	0.71872465633934	0.722717741087767\\
0.63	0.316706364314172	0.730143689578457	0.715729131143268\\
0.63	0.328891676652598	0.741359304476472	0.709241120561856\\
0.63	0.341255928754291	0.752367096967051	0.703223584571264\\
0.63	0.353793780602429	0.763162841084325	0.697635496597418\\
0.63	0.366499761262991	0.773742494322337	0.692428303594619\\
0.63	0.379368276771856	0.784102202600442	0.687549132887069\\
0.63	0.392393618241769	0.794238304822092	0.682943679235329\\
0.63	0.405569970164763	0.804147337016097	0.678558662193044\\
0.63	0.418891418885081	0.813826036051075	0.67434378741287\\
0.63	0.432351961217132	0.823271342915544	0.670253187715552\\
0.63	0.445945513182735	0.832480405557786	0.666246357336159\\
0.63	0.459665918841652	0.841450581281375	0.662288623474186\\
0.63	0.473506959189401	0.850179438693979	0.658351221673191\\
0.63	0.487462361096317	0.85866475920876	0.654411055137884\\
0.63	0.501525806262096	0.866904538099407	0.650450223206469\\
0.63	0.515690940160279	0.874896985111485	0.64645540187496\\
0.63	0.529951380947623	0.882640524634436	0.642417151055154\\
0.63	0.544300728313782	0.890133795440129	0.638329210947587\\
0.63	0.558732572247415	0.897375649995373	0.634187835387497\\
0.63	0.57324050169555	0.904365153357279	0.629991194998378\\
0.63	0.587818113093873	0.911101581661729	0.625738868900926\\
0.63	0.602459018746568	0.917584420216505	0.621431431633139\\
0.63	0.617156855035314	0.923813361211857	0.61707013248055\\
0.63	0.631905290438127	0.92978830106243	0.612656657833837\\
0.63	0.6466980333399	0.935509337395467	0.608192963377089\\
0.63	0.661528839617671	0.940976765701173	0.603681161495104\\
0.63	0.676391519984904	0.94619107566195	0.599123449738607\\
0.63	0.691279947080357	0.951152947177946	0.594522067901589\\
0.63	0.706188062288412	0.955863246106974	0.589879273662389\\
0.63	0.721109882279076	0.960323019737426	0.585197329319937\\
0.63	0.736039505257225	0.964533492013186	0.580478494543976\\
0.63	0.75097111691199	0.968496058529893	0.575725022018312\\
0.63	0.765898996058536	0.972212281322139	0.570939154289558\\
0.63	0.780817519965826	0.975683883461272	0.566123121053176\\
0.63	0.795721169365275	0.97891274348356	0.561279136605559\\
0.63	0.810604533136476	0.981900889668376	0.556409397399199\\
0.63	0.825462312667457	0.984650494185936	0.551516079695734\\
0.63	0.840289325888138	0.987163867133889	0.54660133731964\\
0.63	0.855080510976839	0.989443450481767	0.541667299515141\\
0.63	0.86983092974082	0.991491811941915	0.5367160689086\\
0.63	0.884535770672926	0.993311638785082	0.531749719578142\\
0.63	0.899190351687411	0.994905731618374	0.526770295231793\\
0.63	0.913790122539012	0.996276998142666	0.521779807495102\\
0.63	0.928330666930242	0.997428446906011	0.516780234308683\\
0.63	0.942807704312699	0.998363181068902	0.511773518435873\\
0.63	0.95721709138901	0.999084392196567	0.506761566080168\\
0.63	0.971554823322703	0.999595354092743	0.501746245611875\\
0.63	0.985817034663989	0.999899416688637	0.496729386403006\\
0.63	1	1	0.491669087346904\\
0.645	0	0	0.710195569525497\\
0.645	0.000100583311362513	0.0141829653360114	0.71240460418299\\
0.645	0.000404645907256436	0.0284451766772965	0.714630690536905\\
0.645	0.000915607803432999	0.0427829086109896	0.716916312064106\\
0.645	0.00163681893109844	0.057192295687301	0.719289019360234\\
0.645	0.00257155309398959	0.0716693330697584	0.721780367061016\\
0.645	0.00372300185733414	0.0862098774609879	0.724425655492835\\
0.645	0.00509426838162598	0.100809648312589	0.727263413850304\\
0.645	0.00668836121491816	0.115464229327074	0.73033458997498\\
0.645	0.00850818805808555	0.13016907025918	0.733681422514709\\
0.645	0.0105565495182326	0.144919489023162	0.737345986077549\\
0.645	0.0128361328661109	0.159710674111862	0.741368418705642\\
0.645	0.0153495058140643	0.174537687332543	0.745784862958202\\
0.645	0.0180991103316243	0.189395466863524	0.750625176092311\\
0.645	0.0210872565164405	0.204278830634725	0.755910489868999\\
0.645	0.0243161165387281	0.219182480034174	0.761650724664453\\
0.645	0.0277877186778607	0.234101003941464	0.767842183867616\\
0.645	0.0315039414701067	0.24902888308801	0.77446537092041\\
0.645	0.0354665079868145	0.263960494742775	0.781483180780606\\
0.645	0.0396769802625738	0.278890117720924	0.788839618268073\\
0.645	0.0441367538930258	0.293811937711588	0.796459186325544\\
0.645	0.0488470528220538	0.308720052919643	0.804247066923454\\
0.645	0.0538089243380495	0.323608480015096	0.812090186161951\\
0.645	0.0590232342988274	0.338471160382329	0.819859213932692\\
0.645	0.064490662604533	0.3533019666601	0.827411499063638\\
0.645	0.0702116989375697	0.368094709561873	0.834594885813505\\
0.645	0.0761866387881432	0.382843144964686	0.84125230028718\\
0.645	0.0824155797834956	0.397540981253432	0.847226939729078\\
0.645	0.0888984183382709	0.412181886906127	0.85236784790504\\
0.645	0.0956348466427212	0.42675949830445	0.85653562002783\\
0.645	0.102624350004627	0.441267427752584	0.859607954619618\\
0.645	0.109866204559871	0.455699271686218	0.861484760266623\\
0.645	0.117359475365564	0.470048619052377	0.862092534251539\\
0.645	0.125103014888515	0.484309059839721	0.861387758068366\\
0.645	0.133095461900593	0.498474193737904	0.859359100890822\\
0.645	0.14133524079124	0.512537638903683	0.85602828376574\\
0.645	0.149820561306021	0.526493040810599	0.851449530870181\\
0.645	0.158549418718625	0.540334081158348	0.845707614739504\\
0.645	0.167519594442214	0.554054486817265	0.838914584343152\\
0.645	0.176728657084455	0.567648038782867	0.831205342370838\\
0.645	0.186173963948925	0.581108581114919	0.822732305417524\\
0.645	0.195852662983903	0.594430029835237	0.813659432944305\\
0.645	0.205761695177907	0.60760638175823	0.804155944104874\\
0.645	0.215897797399558	0.620631723228144	0.79439005340697\\
0.645	0.226257505677663	0.633500238737009	0.7845230460628\\
0.645	0.236837158915675	0.646206219397571	0.774703982861591\\
0.645	0.247632903032949	0.658744071245709	0.765065275211015\\
0.645	0.258640695523527	0.671108323347402	0.755719307772963\\
0.645	0.269856310421543	0.683293635685828	0.746756214005913\\
0.645	0.28127534366066	0.695294806804924	0.738242834604297\\
0.645	0.292893218813452	0.707106781186548	0.730222816007834\\
0.645	0.304705193195075	0.71872465633934	0.722717741087766\\
0.645	0.316706364314172	0.730143689578457	0.715729131143268\\
0.645	0.328891676652598	0.741359304476472	0.709241120561855\\
0.645	0.341255928754291	0.752367096967051	0.703223584571265\\
0.645	0.353793780602429	0.763162841084325	0.697635496597419\\
0.645	0.366499761262991	0.773742494322337	0.692428303594617\\
0.645	0.379368276771857	0.784102202600442	0.687549132887068\\
0.645	0.392393618241769	0.794238304822092	0.68294367923533\\
0.645	0.405569970164763	0.804147337016097	0.678558662193045\\
0.645	0.418891418885081	0.813826036051075	0.67434378741287\\
0.645	0.432351961217132	0.823271342915544	0.670253187715551\\
0.645	0.445945513182735	0.832480405557786	0.666246357336161\\
0.645	0.459665918841652	0.841450581281375	0.662288623474188\\
0.645	0.473506959189401	0.850179438693979	0.65835122167319\\
0.645	0.487462361096317	0.85866475920876	0.654411055137885\\
0.645	0.501525806262096	0.866904538099407	0.650450223206468\\
0.645	0.515690940160279	0.874896985111485	0.646455401874959\\
0.645	0.529951380947623	0.882640524634437	0.642417151055153\\
0.645	0.544300728313782	0.890133795440129	0.638329210947586\\
0.645	0.558732572247415	0.897375649995373	0.634187835387499\\
0.645	0.57324050169555	0.904365153357279	0.629991194998374\\
0.645	0.587818113093873	0.911101581661729	0.625738868900926\\
0.645	0.602459018746568	0.917584420216505	0.621431431633143\\
0.645	0.617156855035314	0.923813361211857	0.617070132480552\\
0.645	0.631905290438127	0.92978830106243	0.612656657833836\\
0.645	0.6466980333399	0.935509337395467	0.608192963377088\\
0.645	0.661528839617671	0.940976765701173	0.603681161495106\\
0.645	0.676391519984904	0.946191075661951	0.599123449738605\\
0.645	0.691279947080357	0.951152947177946	0.594522067901587\\
0.645	0.706188062288412	0.955863246106974	0.589879273662391\\
0.645	0.721109882279076	0.960323019737426	0.585197329319938\\
0.645	0.736039505257226	0.964533492013186	0.580478494543973\\
0.645	0.75097111691199	0.968496058529893	0.575725022018312\\
0.645	0.765898996058536	0.972212281322139	0.570939154289562\\
0.645	0.780817519965826	0.975683883461272	0.566123121053176\\
0.645	0.795721169365275	0.97891274348356	0.561279136605559\\
0.645	0.810604533136476	0.981900889668376	0.556409397399197\\
0.645	0.825462312667457	0.984650494185936	0.551516079695734\\
0.645	0.840289325888138	0.987163867133889	0.546601337319643\\
0.645	0.855080510976839	0.989443450481767	0.541667299515141\\
0.645	0.86983092974082	0.991491811941914	0.536716068908598\\
0.645	0.884535770672926	0.993311638785082	0.531749719578143\\
0.645	0.899190351687411	0.994905731618374	0.526770295231795\\
0.645	0.913790122539012	0.996276998142666	0.521779807495102\\
0.645	0.928330666930242	0.997428446906011	0.516780234308683\\
0.645	0.942807704312699	0.998363181068902	0.51177351843587\\
0.645	0.95721709138901	0.999084392196567	0.50676156608017\\
0.645	0.971554823322704	0.999595354092743	0.501746245611875\\
0.645	0.985817034663989	0.999899416688637	0.496729386403003\\
0.645	1	1	0.491669087346906\\
0.66	0	0	0.710195569525497\\
0.66	0.000100583311362513	0.0141829653360114	0.71240460418299\\
0.66	0.000404645907256436	0.0284451766772965	0.714630690536905\\
0.66	0.000915607803432999	0.0427829086109896	0.716916312064106\\
0.66	0.00163681893109844	0.057192295687301	0.719289019360234\\
0.66	0.00257155309398959	0.0716693330697584	0.721780367061016\\
0.66	0.00372300185733413	0.0862098774609879	0.724425655492835\\
0.66	0.00509426838162598	0.100809648312589	0.727263413850304\\
0.66	0.00668836121491816	0.115464229327074	0.73033458997498\\
0.66	0.00850818805808555	0.13016907025918	0.733681422514709\\
0.66	0.0105565495182326	0.144919489023162	0.737345986077549\\
0.66	0.0128361328661109	0.159710674111862	0.741368418705642\\
0.66	0.0153495058140643	0.174537687332543	0.745784862958201\\
0.66	0.0180991103316243	0.189395466863524	0.750625176092311\\
0.66	0.0210872565164405	0.204278830634725	0.755910489868999\\
0.66	0.0243161165387281	0.219182480034174	0.761650724664453\\
0.66	0.0277877186778607	0.234101003941464	0.767842183867616\\
0.66	0.0315039414701067	0.24902888308801	0.77446537092041\\
0.66	0.0354665079868145	0.263960494742775	0.781483180780606\\
0.66	0.0396769802625738	0.278890117720924	0.788839618268073\\
0.66	0.0441367538930258	0.293811937711588	0.796459186325544\\
0.66	0.0488470528220538	0.308720052919643	0.804247066923454\\
0.66	0.0538089243380495	0.323608480015096	0.81209018616195\\
0.66	0.0590232342988274	0.338471160382329	0.819859213932694\\
0.66	0.064490662604533	0.3533019666601	0.827411499063639\\
0.66	0.0702116989375697	0.368094709561873	0.834594885813504\\
0.66	0.0761866387881432	0.382843144964686	0.841252300287182\\
0.66	0.0824155797834956	0.397540981253432	0.847226939729082\\
0.66	0.0888984183382709	0.412181886906127	0.85236784790504\\
0.66	0.0956348466427212	0.42675949830445	0.856535620027825\\
0.66	0.102624350004627	0.441267427752584	0.859607954619617\\
0.66	0.109866204559871	0.455699271686219	0.861484760266622\\
0.66	0.117359475365564	0.470048619052377	0.86209253425154\\
0.66	0.125103014888515	0.484309059839721	0.86138775806837\\
0.66	0.133095461900593	0.498474193737904	0.859359100890821\\
0.66	0.14133524079124	0.512537638903683	0.856028283765733\\
0.66	0.149820561306021	0.526493040810599	0.851449530870182\\
0.66	0.158549418718625	0.540334081158348	0.845707614739509\\
0.66	0.167519594442214	0.554054486817266	0.838914584343153\\
0.66	0.176728657084455	0.567648038782867	0.831205342370837\\
0.66	0.186173963948925	0.581108581114919	0.822732305417528\\
0.66	0.195852662983903	0.594430029835236	0.813659432944308\\
0.66	0.205761695177907	0.607606381758231	0.804155944104871\\
0.66	0.215897797399558	0.620631723228143	0.794390053406969\\
0.66	0.226257505677663	0.633500238737009	0.784523046062801\\
0.66	0.236837158915675	0.646206219397571	0.774703982861589\\
0.66	0.247632903032949	0.658744071245709	0.765065275211015\\
0.66	0.258640695523527	0.671108323347402	0.755719307772963\\
0.66	0.269856310421543	0.683293635685828	0.746756214005911\\
0.66	0.28127534366066	0.695294806804924	0.738242834604297\\
0.66	0.292893218813452	0.707106781186548	0.730222816007833\\
0.66	0.304705193195075	0.71872465633934	0.722717741087765\\
0.66	0.316706364314172	0.730143689578457	0.715729131143268\\
0.66	0.328891676652598	0.741359304476472	0.709241120561858\\
0.66	0.341255928754291	0.752367096967051	0.703223584571265\\
0.66	0.353793780602429	0.763162841084325	0.697635496597418\\
0.66	0.366499761262991	0.773742494322337	0.692428303594618\\
0.66	0.379368276771857	0.784102202600443	0.687549132887069\\
0.66	0.392393618241769	0.794238304822092	0.682943679235328\\
0.66	0.405569970164763	0.804147337016096	0.678558662193044\\
0.66	0.418891418885081	0.813826036051075	0.674343787412871\\
0.66	0.432351961217132	0.823271342915544	0.670253187715552\\
0.66	0.445945513182735	0.832480405557786	0.666246357336158\\
0.66	0.459665918841652	0.841450581281375	0.662288623474188\\
0.66	0.473506959189401	0.850179438693979	0.658351221673192\\
0.66	0.487462361096317	0.85866475920876	0.654411055137885\\
0.66	0.501525806262096	0.866904538099407	0.650450223206468\\
0.66	0.515690940160279	0.874896985111485	0.646455401874959\\
0.66	0.529951380947623	0.882640524634436	0.642417151055155\\
0.66	0.544300728313782	0.890133795440129	0.638329210947585\\
0.66	0.558732572247415	0.897375649995372	0.634187835387498\\
0.66	0.57324050169555	0.904365153357279	0.629991194998376\\
0.66	0.587818113093873	0.911101581661729	0.625738868900923\\
0.66	0.602459018746568	0.917584420216504	0.62143143163314\\
0.66	0.617156855035314	0.923813361211857	0.617070132480554\\
0.66	0.631905290438127	0.92978830106243	0.612656657833837\\
0.66	0.6466980333399	0.935509337395467	0.608192963377087\\
0.66	0.661528839617671	0.940976765701173	0.603681161495107\\
0.66	0.676391519984903	0.946191075661951	0.599123449738608\\
0.66	0.691279947080357	0.951152947177946	0.594522067901587\\
0.66	0.706188062288412	0.955863246106974	0.589879273662389\\
0.66	0.721109882279076	0.960323019737426	0.585197329319938\\
0.66	0.736039505257226	0.964533492013186	0.580478494543977\\
0.66	0.75097111691199	0.968496058529893	0.575725022018312\\
0.66	0.765898996058536	0.972212281322139	0.570939154289557\\
0.66	0.780817519965826	0.975683883461272	0.566123121053176\\
0.66	0.795721169365275	0.97891274348356	0.561279136605561\\
0.66	0.810604533136476	0.981900889668376	0.556409397399198\\
0.66	0.825462312667457	0.984650494185936	0.551516079695732\\
0.66	0.840289325888138	0.987163867133889	0.546601337319642\\
0.66	0.855080510976839	0.989443450481768	0.541667299515141\\
0.66	0.86983092974082	0.991491811941914	0.536716068908599\\
0.66	0.884535770672926	0.993311638785082	0.531749719578143\\
0.66	0.899190351687411	0.994905731618374	0.526770295231793\\
0.66	0.913790122539012	0.996276998142666	0.521779807495102\\
0.66	0.928330666930242	0.997428446906011	0.516780234308683\\
0.66	0.942807704312699	0.998363181068902	0.51177351843587\\
0.66	0.95721709138901	0.999084392196567	0.506761566080171\\
0.66	0.971554823322703	0.999595354092744	0.501746245611878\\
0.66	0.985817034663989	0.999899416688638	0.496729386403001\\
0.66	1	1	0.491669087346898\\
0.675	0	0	0.710195569525497\\
0.675	0.000100583311362513	0.0141829653360114	0.71240460418299\\
0.675	0.000404645907256436	0.0284451766772965	0.714630690536905\\
0.675	0.000915607803432999	0.0427829086109896	0.716916312064106\\
0.675	0.00163681893109844	0.057192295687301	0.719289019360234\\
0.675	0.00257155309398959	0.0716693330697584	0.721780367061016\\
0.675	0.00372300185733413	0.0862098774609879	0.724425655492835\\
0.675	0.00509426838162598	0.100809648312589	0.727263413850304\\
0.675	0.00668836121491816	0.115464229327074	0.73033458997498\\
0.675	0.00850818805808555	0.13016907025918	0.733681422514709\\
0.675	0.0105565495182326	0.144919489023162	0.737345986077549\\
0.675	0.0128361328661109	0.159710674111862	0.741368418705642\\
0.675	0.0153495058140643	0.174537687332543	0.745784862958201\\
0.675	0.0180991103316243	0.189395466863524	0.750625176092311\\
0.675	0.0210872565164405	0.204278830634725	0.755910489868999\\
0.675	0.0243161165387281	0.219182480034174	0.761650724664453\\
0.675	0.0277877186778607	0.234101003941464	0.767842183867616\\
0.675	0.0315039414701067	0.24902888308801	0.77446537092041\\
0.675	0.0354665079868145	0.263960494742775	0.781483180780606\\
0.675	0.0396769802625738	0.278890117720924	0.788839618268073\\
0.675	0.0441367538930258	0.293811937711588	0.796459186325544\\
0.675	0.0488470528220537	0.308720052919643	0.804247066923455\\
0.675	0.0538089243380495	0.323608480015096	0.81209018616195\\
0.675	0.0590232342988274	0.338471160382329	0.819859213932694\\
0.675	0.064490662604533	0.3533019666601	0.82741149906364\\
0.675	0.0702116989375697	0.368094709561873	0.834594885813505\\
0.675	0.0761866387881432	0.382843144964686	0.841252300287179\\
0.675	0.0824155797834956	0.397540981253432	0.847226939729082\\
0.675	0.0888984183382709	0.412181886906127	0.852367847905043\\
0.675	0.0956348466427212	0.42675949830445	0.85653562002783\\
0.675	0.102624350004627	0.441267427752584	0.859607954619617\\
0.675	0.109866204559871	0.455699271686218	0.86148476026662\\
0.675	0.117359475365564	0.470048619052377	0.862092534251543\\
0.675	0.125103014888515	0.484309059839721	0.861387758068369\\
0.675	0.133095461900593	0.498474193737904	0.859359100890821\\
0.675	0.14133524079124	0.512537638903683	0.856028283765733\\
0.675	0.149820561306021	0.526493040810599	0.851449530870182\\
0.675	0.158549418718625	0.540334081158348	0.845707614739505\\
0.675	0.167519594442214	0.554054486817266	0.838914584343151\\
0.675	0.176728657084455	0.567648038782867	0.831205342370838\\
0.675	0.186173963948925	0.581108581114919	0.822732305417528\\
0.675	0.195852662983903	0.594430029835236	0.813659432944307\\
0.675	0.205761695177907	0.607606381758231	0.80415594410487\\
0.675	0.215897797399558	0.620631723228144	0.79439005340697\\
0.675	0.226257505677663	0.633500238737009	0.7845230460628\\
0.675	0.236837158915675	0.646206219397571	0.774703982861587\\
0.675	0.247632903032949	0.658744071245709	0.765065275211014\\
0.675	0.258640695523527	0.671108323347402	0.755719307772962\\
0.675	0.269856310421543	0.683293635685828	0.746756214005912\\
0.675	0.28127534366066	0.695294806804924	0.738242834604295\\
0.675	0.292893218813452	0.707106781186547	0.730222816007834\\
0.675	0.304705193195075	0.71872465633934	0.722717741087767\\
0.675	0.316706364314172	0.730143689578457	0.715729131143269\\
0.675	0.328891676652598	0.741359304476472	0.709241120561856\\
0.675	0.341255928754291	0.752367096967051	0.703223584571264\\
0.675	0.353793780602429	0.763162841084325	0.697635496597419\\
0.675	0.366499761262991	0.773742494322337	0.692428303594618\\
0.675	0.379368276771857	0.784102202600442	0.687549132887068\\
0.675	0.392393618241769	0.794238304822092	0.682943679235329\\
0.675	0.405569970164763	0.804147337016097	0.678558662193045\\
0.675	0.418891418885081	0.813826036051075	0.674343787412869\\
0.675	0.432351961217133	0.823271342915544	0.670253187715552\\
0.675	0.445945513182735	0.832480405557786	0.666246357336159\\
0.675	0.459665918841652	0.841450581281375	0.662288623474186\\
0.675	0.473506959189401	0.850179438693979	0.658351221673192\\
0.675	0.487462361096317	0.85866475920876	0.654411055137887\\
0.675	0.501525806262096	0.866904538099407	0.650450223206468\\
0.675	0.515690940160279	0.874896985111485	0.646455401874958\\
0.675	0.529951380947623	0.882640524634436	0.642417151055154\\
0.675	0.544300728313782	0.890133795440129	0.638329210947587\\
0.675	0.558732572247415	0.897375649995373	0.634187835387497\\
0.675	0.57324050169555	0.904365153357279	0.629991194998376\\
0.675	0.587818113093873	0.911101581661729	0.625738868900925\\
0.675	0.602459018746568	0.917584420216504	0.621431431633141\\
0.675	0.617156855035314	0.923813361211857	0.617070132480553\\
0.675	0.631905290438127	0.92978830106243	0.612656657833838\\
0.675	0.6466980333399	0.935509337395467	0.608192963377087\\
0.675	0.661528839617671	0.940976765701173	0.603681161495103\\
0.675	0.676391519984904	0.94619107566195	0.599123449738607\\
0.675	0.691279947080357	0.951152947177946	0.59452206790159\\
0.675	0.706188062288412	0.955863246106974	0.589879273662392\\
0.675	0.721109882279076	0.960323019737426	0.585197329319937\\
0.675	0.736039505257226	0.964533492013186	0.580478494543974\\
0.675	0.75097111691199	0.968496058529893	0.575725022018313\\
0.675	0.765898996058536	0.972212281322139	0.570939154289559\\
0.675	0.780817519965826	0.975683883461272	0.566123121053171\\
0.675	0.795721169365274	0.978912743483559	0.561279136605559\\
0.675	0.810604533136476	0.981900889668376	0.556409397399202\\
0.675	0.825462312667457	0.984650494185936	0.551516079695733\\
0.675	0.840289325888138	0.987163867133889	0.546601337319641\\
0.675	0.855080510976839	0.989443450481767	0.541667299515139\\
0.675	0.86983092974082	0.991491811941914	0.536716068908599\\
0.675	0.884535770672926	0.993311638785082	0.531749719578143\\
0.675	0.899190351687411	0.994905731618374	0.526770295231793\\
0.675	0.913790122539012	0.996276998142666	0.521779807495102\\
0.675	0.928330666930242	0.997428446906011	0.516780234308685\\
0.675	0.942807704312699	0.998363181068902	0.511773518435872\\
0.675	0.95721709138901	0.999084392196567	0.506761566080167\\
0.675	0.971554823322703	0.999595354092743	0.501746245611877\\
0.675	0.985817034663989	0.999899416688638	0.496729386403004\\
0.675	1	1	0.491669087346897\\
0.69	0	0	0.710195569525497\\
0.69	0.000100583311362513	0.0141829653360114	0.71240460418299\\
0.69	0.000404645907256436	0.0284451766772965	0.714630690536905\\
0.69	0.000915607803432999	0.0427829086109896	0.716916312064106\\
0.69	0.00163681893109844	0.057192295687301	0.719289019360234\\
0.69	0.00257155309398959	0.0716693330697584	0.721780367061016\\
0.69	0.00372300185733413	0.0862098774609879	0.724425655492835\\
0.69	0.00509426838162598	0.100809648312589	0.727263413850304\\
0.69	0.00668836121491816	0.115464229327074	0.73033458997498\\
0.69	0.00850818805808555	0.13016907025918	0.733681422514709\\
0.69	0.0105565495182326	0.144919489023162	0.737345986077549\\
0.69	0.0128361328661109	0.159710674111862	0.741368418705642\\
0.69	0.0153495058140643	0.174537687332543	0.745784862958201\\
0.69	0.0180991103316243	0.189395466863524	0.750625176092311\\
0.69	0.0210872565164405	0.204278830634725	0.755910489868999\\
0.69	0.0243161165387281	0.219182480034174	0.761650724664453\\
0.69	0.0277877186778607	0.234101003941464	0.767842183867616\\
0.69	0.0315039414701067	0.24902888308801	0.77446537092041\\
0.69	0.0354665079868145	0.263960494742775	0.781483180780606\\
0.69	0.0396769802625738	0.278890117720924	0.788839618268073\\
0.69	0.0441367538930258	0.293811937711588	0.796459186325544\\
0.69	0.0488470528220538	0.308720052919643	0.804247066923456\\
0.69	0.0538089243380495	0.323608480015096	0.812090186161951\\
0.69	0.0590232342988274	0.338471160382329	0.819859213932693\\
0.69	0.064490662604533	0.3533019666601	0.827411499063639\\
0.69	0.0702116989375697	0.368094709561873	0.834594885813506\\
0.69	0.0761866387881432	0.382843144964686	0.841252300287179\\
0.69	0.0824155797834956	0.397540981253432	0.847226939729079\\
0.69	0.0888984183382709	0.412181886906127	0.852367847905041\\
0.69	0.0956348466427212	0.42675949830445	0.85653562002783\\
0.69	0.102624350004627	0.441267427752584	0.859607954619618\\
0.69	0.109866204559871	0.455699271686218	0.86148476026662\\
0.69	0.117359475365564	0.470048619052377	0.862092534251539\\
0.69	0.125103014888515	0.484309059839721	0.861387758068369\\
0.69	0.133095461900593	0.498474193737904	0.85935910089082\\
0.69	0.14133524079124	0.512537638903683	0.856028283765736\\
0.69	0.149820561306021	0.526493040810599	0.851449530870182\\
0.69	0.158549418718625	0.540334081158348	0.845707614739506\\
0.69	0.167519594442213	0.554054486817266	0.838914584343147\\
0.69	0.176728657084455	0.567648038782867	0.831205342370835\\
0.69	0.186173963948925	0.581108581114919	0.822732305417526\\
0.69	0.195852662983903	0.594430029835236	0.813659432944307\\
0.69	0.205761695177907	0.607606381758231	0.804155944104871\\
0.69	0.215897797399558	0.620631723228143	0.794390053406968\\
0.69	0.226257505677663	0.633500238737009	0.784523046062801\\
0.69	0.236837158915675	0.646206219397571	0.774703982861589\\
0.69	0.247632903032949	0.658744071245709	0.765065275211016\\
0.69	0.258640695523528	0.671108323347402	0.755719307772962\\
0.69	0.269856310421543	0.683293635685828	0.74675621400591\\
0.69	0.28127534366066	0.695294806804924	0.738242834604296\\
0.69	0.292893218813452	0.707106781186548	0.730222816007836\\
0.69	0.304705193195075	0.71872465633934	0.722717741087766\\
0.69	0.316706364314172	0.730143689578457	0.715729131143268\\
0.69	0.328891676652598	0.741359304476472	0.709241120561856\\
0.69	0.341255928754291	0.752367096967051	0.703223584571264\\
0.69	0.353793780602429	0.763162841084324	0.697635496597419\\
0.69	0.366499761262991	0.773742494322337	0.692428303594621\\
0.69	0.379368276771856	0.784102202600443	0.687549132887069\\
0.69	0.392393618241769	0.794238304822092	0.682943679235328\\
0.69	0.405569970164763	0.804147337016097	0.678558662193044\\
0.69	0.418891418885081	0.813826036051075	0.674343787412869\\
0.69	0.432351961217132	0.823271342915544	0.670253187715551\\
0.69	0.445945513182735	0.832480405557786	0.666246357336159\\
0.69	0.459665918841652	0.841450581281375	0.662288623474186\\
0.69	0.473506959189401	0.850179438693979	0.658351221673191\\
0.69	0.487462361096317	0.85866475920876	0.654411055137887\\
0.69	0.501525806262096	0.866904538099407	0.650450223206467\\
0.69	0.515690940160279	0.874896985111485	0.646455401874959\\
0.69	0.529951380947623	0.882640524634437	0.642417151055156\\
0.69	0.544300728313782	0.890133795440129	0.638329210947586\\
0.69	0.558732572247415	0.897375649995373	0.634187835387498\\
0.69	0.57324050169555	0.904365153357279	0.629991194998375\\
0.69	0.587818113093873	0.911101581661729	0.625738868900923\\
0.69	0.602459018746568	0.917584420216504	0.621431431633141\\
0.69	0.617156855035314	0.923813361211857	0.617070132480554\\
0.69	0.631905290438127	0.92978830106243	0.612656657833839\\
0.69	0.6466980333399	0.935509337395467	0.608192963377088\\
0.69	0.661528839617671	0.940976765701173	0.603681161495102\\
0.69	0.676391519984904	0.94619107566195	0.599123449738607\\
0.69	0.691279947080357	0.951152947177946	0.594522067901591\\
0.69	0.706188062288412	0.955863246106974	0.589879273662389\\
0.69	0.721109882279076	0.960323019737426	0.585197329319938\\
0.69	0.736039505257226	0.964533492013186	0.580478494543978\\
0.69	0.75097111691199	0.968496058529893	0.575725022018314\\
0.69	0.765898996058536	0.972212281322139	0.57093915428956\\
0.69	0.780817519965826	0.975683883461272	0.566123121053175\\
0.69	0.795721169365275	0.97891274348356	0.561279136605558\\
0.69	0.810604533136476	0.981900889668376	0.556409397399198\\
0.69	0.825462312667457	0.984650494185936	0.551516079695733\\
0.69	0.840289325888138	0.987163867133889	0.546601337319641\\
0.69	0.855080510976838	0.989443450481767	0.541667299515141\\
0.69	0.86983092974082	0.991491811941914	0.536716068908597\\
0.69	0.884535770672926	0.993311638785082	0.531749719578141\\
0.69	0.899190351687411	0.994905731618374	0.526770295231795\\
0.69	0.913790122539012	0.996276998142666	0.521779807495104\\
0.69	0.928330666930242	0.997428446906011	0.516780234308683\\
0.69	0.942807704312699	0.998363181068902	0.511773518435873\\
0.69	0.95721709138901	0.999084392196567	0.50676156608017\\
0.69	0.971554823322704	0.999595354092743	0.501746245611876\\
0.69	0.985817034663989	0.999899416688638	0.496729386403004\\
0.69	1	1	0.491669087346897\\
0.705	0	0	0.710195569525497\\
0.705	0.000100583311362513	0.0141829653360114	0.71240460418299\\
0.705	0.000404645907256436	0.0284451766772965	0.714630690536905\\
0.705	0.000915607803432999	0.0427829086109896	0.716916312064106\\
0.705	0.00163681893109844	0.057192295687301	0.719289019360234\\
0.705	0.00257155309398959	0.0716693330697584	0.721780367061016\\
0.705	0.00372300185733413	0.0862098774609879	0.724425655492835\\
0.705	0.00509426838162598	0.100809648312589	0.727263413850304\\
0.705	0.00668836121491816	0.115464229327074	0.73033458997498\\
0.705	0.00850818805808554	0.13016907025918	0.733681422514709\\
0.705	0.0105565495182326	0.144919489023162	0.737345986077549\\
0.705	0.0128361328661109	0.159710674111862	0.741368418705642\\
0.705	0.0153495058140643	0.174537687332543	0.745784862958202\\
0.705	0.0180991103316243	0.189395466863524	0.750625176092311\\
0.705	0.0210872565164405	0.204278830634725	0.755910489868999\\
0.705	0.0243161165387281	0.219182480034174	0.761650724664453\\
0.705	0.0277877186778607	0.234101003941464	0.767842183867616\\
0.705	0.0315039414701067	0.24902888308801	0.77446537092041\\
0.705	0.0354665079868145	0.263960494742775	0.781483180780606\\
0.705	0.0396769802625738	0.278890117720924	0.788839618268073\\
0.705	0.0441367538930258	0.293811937711588	0.796459186325544\\
0.705	0.0488470528220538	0.308720052919643	0.804247066923454\\
0.705	0.0538089243380495	0.323608480015097	0.812090186161951\\
0.705	0.0590232342988274	0.338471160382329	0.819859213932694\\
0.705	0.064490662604533	0.3533019666601	0.827411499063637\\
0.705	0.0702116989375697	0.368094709561873	0.834594885813505\\
0.705	0.0761866387881432	0.382843144964686	0.84125230028718\\
0.705	0.0824155797834956	0.397540981253432	0.847226939729082\\
0.705	0.0888984183382709	0.412181886906127	0.852367847905041\\
0.705	0.0956348466427212	0.42675949830445	0.856535620027828\\
0.705	0.102624350004627	0.441267427752584	0.859607954619616\\
0.705	0.109866204559871	0.455699271686218	0.86148476026662\\
0.705	0.117359475365564	0.470048619052377	0.862092534251539\\
0.705	0.125103014888515	0.484309059839721	0.86138775806837\\
0.705	0.133095461900593	0.498474193737904	0.859359100890819\\
0.705	0.14133524079124	0.512537638903683	0.856028283765736\\
0.705	0.149820561306021	0.526493040810599	0.851449530870181\\
0.705	0.158549418718625	0.540334081158348	0.845707614739502\\
0.705	0.167519594442214	0.554054486817265	0.838914584343146\\
0.705	0.176728657084455	0.567648038782867	0.831205342370835\\
0.705	0.186173963948925	0.581108581114919	0.822732305417528\\
0.705	0.195852662983903	0.594430029835236	0.813659432944306\\
0.705	0.205761695177907	0.607606381758231	0.804155944104868\\
0.705	0.215897797399558	0.620631723228143	0.794390053406971\\
0.705	0.226257505677663	0.633500238737009	0.784523046062802\\
0.705	0.236837158915675	0.646206219397571	0.77470398286159\\
0.705	0.247632903032949	0.658744071245709	0.765065275211016\\
0.705	0.258640695523528	0.671108323347402	0.755719307772965\\
0.705	0.269856310421543	0.683293635685828	0.746756214005911\\
0.705	0.28127534366066	0.695294806804924	0.738242834604298\\
0.705	0.292893218813452	0.707106781186547	0.730222816007835\\
0.705	0.304705193195075	0.71872465633934	0.722717741087767\\
0.705	0.316706364314172	0.730143689578457	0.715729131143267\\
0.705	0.328891676652598	0.741359304476472	0.709241120561855\\
0.705	0.341255928754291	0.752367096967051	0.703223584571266\\
0.705	0.353793780602429	0.763162841084325	0.69763549659742\\
0.705	0.366499761262991	0.773742494322337	0.692428303594619\\
0.705	0.379368276771857	0.784102202600443	0.687549132887069\\
0.705	0.392393618241769	0.794238304822092	0.682943679235329\\
0.705	0.405569970164763	0.804147337016097	0.678558662193045\\
0.705	0.418891418885081	0.813826036051075	0.674343787412871\\
0.705	0.432351961217132	0.823271342915544	0.670253187715552\\
0.705	0.445945513182735	0.832480405557786	0.666246357336158\\
0.705	0.459665918841652	0.841450581281375	0.662288623474186\\
0.705	0.473506959189401	0.850179438693979	0.65835122167319\\
0.705	0.487462361096317	0.85866475920876	0.654411055137885\\
0.705	0.501525806262096	0.866904538099407	0.650450223206469\\
0.705	0.515690940160279	0.874896985111485	0.646455401874957\\
0.705	0.529951380947623	0.882640524634436	0.642417151055156\\
0.705	0.544300728313781	0.890133795440129	0.638329210947587\\
0.705	0.558732572247415	0.897375649995373	0.634187835387498\\
0.705	0.57324050169555	0.904365153357279	0.629991194998377\\
0.705	0.587818113093873	0.911101581661729	0.625738868900924\\
0.705	0.602459018746568	0.917584420216504	0.62143143163314\\
0.705	0.617156855035314	0.923813361211857	0.617070132480553\\
0.705	0.631905290438127	0.92978830106243	0.612656657833838\\
0.705	0.6466980333399	0.935509337395467	0.608192963377089\\
0.705	0.661528839617671	0.940976765701173	0.603681161495105\\
0.705	0.676391519984903	0.94619107566195	0.599123449738607\\
0.705	0.691279947080357	0.951152947177946	0.594522067901589\\
0.705	0.706188062288412	0.955863246106974	0.589879273662389\\
0.705	0.721109882279076	0.960323019737426	0.585197329319937\\
0.705	0.736039505257226	0.964533492013186	0.580478494543975\\
0.705	0.75097111691199	0.968496058529893	0.575725022018313\\
0.705	0.765898996058536	0.972212281322139	0.570939154289561\\
0.705	0.780817519965826	0.975683883461272	0.566123121053174\\
0.705	0.795721169365275	0.978912743483559	0.56127913660556\\
0.705	0.810604533136476	0.981900889668376	0.556409397399202\\
0.705	0.825462312667457	0.984650494185936	0.551516079695735\\
0.705	0.840289325888138	0.987163867133889	0.546601337319639\\
0.705	0.855080510976839	0.989443450481767	0.541667299515139\\
0.705	0.86983092974082	0.991491811941914	0.5367160689086\\
0.705	0.884535770672926	0.993311638785082	0.531749719578143\\
0.705	0.899190351687411	0.994905731618374	0.526770295231792\\
0.705	0.913790122539012	0.996276998142666	0.521779807495102\\
0.705	0.928330666930242	0.997428446906011	0.516780234308683\\
0.705	0.942807704312699	0.998363181068902	0.511773518435872\\
0.705	0.95721709138901	0.999084392196567	0.50676156608017\\
0.705	0.971554823322703	0.999595354092743	0.501746245611875\\
0.705	0.985817034663989	0.999899416688638	0.496729386403004\\
0.705	1	1	0.491669087346897\\
0.72	0	0	0.710195569525497\\
0.72	0.000100583311362513	0.0141829653360114	0.71240460418299\\
0.72	0.000404645907256436	0.0284451766772965	0.714630690536905\\
0.72	0.000915607803432999	0.0427829086109896	0.716916312064106\\
0.72	0.00163681893109844	0.057192295687301	0.719289019360234\\
0.72	0.00257155309398959	0.0716693330697584	0.721780367061016\\
0.72	0.00372300185733413	0.0862098774609879	0.724425655492835\\
0.72	0.00509426838162598	0.100809648312589	0.727263413850304\\
0.72	0.00668836121491816	0.115464229327074	0.73033458997498\\
0.72	0.00850818805808555	0.13016907025918	0.733681422514709\\
0.72	0.0105565495182326	0.144919489023162	0.737345986077549\\
0.72	0.0128361328661109	0.159710674111862	0.741368418705642\\
0.72	0.0153495058140643	0.174537687332543	0.745784862958201\\
0.72	0.0180991103316243	0.189395466863524	0.750625176092311\\
0.72	0.0210872565164405	0.204278830634725	0.755910489868998\\
0.72	0.0243161165387281	0.219182480034174	0.761650724664453\\
0.72	0.0277877186778607	0.234101003941464	0.767842183867616\\
0.72	0.0315039414701067	0.24902888308801	0.77446537092041\\
0.72	0.0354665079868145	0.263960494742775	0.781483180780606\\
0.72	0.0396769802625738	0.278890117720924	0.788839618268073\\
0.72	0.0441367538930258	0.293811937711588	0.796459186325544\\
0.72	0.0488470528220537	0.308720052919643	0.804247066923455\\
0.72	0.0538089243380495	0.323608480015097	0.812090186161949\\
0.72	0.0590232342988274	0.338471160382329	0.819859213932694\\
0.72	0.064490662604533	0.3533019666601	0.82741149906364\\
0.72	0.0702116989375697	0.368094709561873	0.834594885813503\\
0.72	0.0761866387881432	0.382843144964686	0.84125230028718\\
0.72	0.0824155797834956	0.397540981253432	0.847226939729081\\
0.72	0.0888984183382709	0.412181886906127	0.852367847905041\\
0.72	0.0956348466427212	0.42675949830445	0.85653562002783\\
0.72	0.102624350004627	0.441267427752584	0.859607954619618\\
0.72	0.109866204559871	0.455699271686218	0.861484760266621\\
0.72	0.117359475365564	0.470048619052377	0.862092534251542\\
0.72	0.125103014888515	0.484309059839721	0.861387758068367\\
0.72	0.133095461900593	0.498474193737904	0.859359100890819\\
0.72	0.14133524079124	0.512537638903683	0.856028283765738\\
0.72	0.149820561306021	0.526493040810599	0.851449530870183\\
0.72	0.158549418718625	0.540334081158348	0.845707614739504\\
0.72	0.167519594442214	0.554054486817265	0.838914584343153\\
0.72	0.176728657084455	0.567648038782867	0.831205342370837\\
0.72	0.186173963948925	0.581108581114919	0.822732305417527\\
0.72	0.195852662983903	0.594430029835236	0.813659432944309\\
0.72	0.205761695177907	0.607606381758231	0.80415594410487\\
0.72	0.215897797399558	0.620631723228144	0.79439005340697\\
0.72	0.226257505677663	0.633500238737009	0.784523046062799\\
0.72	0.236837158915675	0.646206219397571	0.77470398286159\\
0.72	0.247632903032949	0.658744071245709	0.765065275211018\\
0.72	0.258640695523527	0.671108323347402	0.755719307772962\\
0.72	0.269856310421543	0.683293635685828	0.746756214005911\\
0.72	0.28127534366066	0.695294806804925	0.738242834604296\\
0.72	0.292893218813452	0.707106781186547	0.730222816007832\\
0.72	0.304705193195076	0.71872465633934	0.722717741087767\\
0.72	0.316706364314172	0.730143689578457	0.715729131143269\\
0.72	0.328891676652598	0.741359304476472	0.709241120561857\\
0.72	0.341255928754291	0.752367096967051	0.703223584571265\\
0.72	0.353793780602429	0.763162841084325	0.697635496597418\\
0.72	0.366499761262991	0.773742494322337	0.692428303594619\\
0.72	0.379368276771856	0.784102202600442	0.68754913288707\\
0.72	0.392393618241769	0.794238304822092	0.682943679235328\\
0.72	0.405569970164763	0.804147337016096	0.678558662193044\\
0.72	0.418891418885081	0.813826036051075	0.67434378741287\\
0.72	0.432351961217132	0.823271342915544	0.670253187715552\\
0.72	0.445945513182735	0.832480405557786	0.66624635733616\\
0.72	0.459665918841652	0.841450581281375	0.662288623474186\\
0.72	0.473506959189401	0.850179438693979	0.658351221673193\\
0.72	0.487462361096317	0.85866475920876	0.654411055137886\\
0.72	0.501525806262096	0.866904538099407	0.650450223206467\\
0.72	0.515690940160279	0.874896985111485	0.646455401874959\\
0.72	0.529951380947623	0.882640524634437	0.642417151055155\\
0.72	0.544300728313782	0.890133795440129	0.638329210947586\\
0.72	0.558732572247415	0.897375649995373	0.634187835387497\\
0.72	0.57324050169555	0.904365153357279	0.629991194998374\\
0.72	0.587818113093873	0.911101581661729	0.625738868900925\\
0.72	0.602459018746568	0.917584420216504	0.62143143163314\\
0.72	0.617156855035314	0.923813361211857	0.617070132480552\\
0.72	0.631905290438127	0.92978830106243	0.612656657833838\\
0.72	0.6466980333399	0.935509337395467	0.608192963377089\\
0.72	0.661528839617671	0.940976765701173	0.603681161495103\\
0.72	0.676391519984904	0.94619107566195	0.599123449738606\\
0.72	0.691279947080357	0.951152947177946	0.594522067901591\\
0.72	0.706188062288412	0.955863246106974	0.589879273662391\\
0.72	0.721109882279076	0.960323019737426	0.585197329319937\\
0.72	0.736039505257226	0.964533492013186	0.580478494543976\\
0.72	0.75097111691199	0.968496058529893	0.575725022018313\\
0.72	0.765898996058536	0.972212281322139	0.57093915428956\\
0.72	0.780817519965826	0.975683883461272	0.566123121053175\\
0.72	0.795721169365275	0.97891274348356	0.561279136605555\\
0.72	0.810604533136476	0.981900889668376	0.556409397399196\\
0.72	0.825462312667457	0.984650494185936	0.551516079695737\\
0.72	0.840289325888138	0.987163867133889	0.546601337319642\\
0.72	0.855080510976839	0.989443450481768	0.541667299515141\\
0.72	0.86983092974082	0.991491811941915	0.536716068908598\\
0.72	0.884535770672926	0.993311638785082	0.531749719578141\\
0.72	0.899190351687411	0.994905731618374	0.526770295231797\\
0.72	0.913790122539012	0.996276998142666	0.521779807495103\\
0.72	0.928330666930242	0.997428446906011	0.516780234308682\\
0.72	0.942807704312699	0.998363181068902	0.511773518435872\\
0.72	0.95721709138901	0.999084392196567	0.506761566080173\\
0.72	0.971554823322704	0.999595354092744	0.501746245611874\\
0.72	0.985817034663988	0.999899416688637	0.496729386403\\
0.72	1	1	0.49166908734691\\
0.735	0	0	0.710195569525497\\
0.735	0.000100583311362513	0.0141829653360114	0.71240460418299\\
0.735	0.000404645907256436	0.0284451766772965	0.714630690536905\\
0.735	0.000915607803432999	0.0427829086109896	0.716916312064106\\
0.735	0.00163681893109844	0.057192295687301	0.719289019360234\\
0.735	0.00257155309398959	0.0716693330697584	0.721780367061016\\
0.735	0.00372300185733413	0.0862098774609879	0.724425655492835\\
0.735	0.00509426838162598	0.100809648312589	0.727263413850304\\
0.735	0.00668836121491816	0.115464229327074	0.73033458997498\\
0.735	0.00850818805808554	0.13016907025918	0.733681422514709\\
0.735	0.0105565495182326	0.144919489023162	0.737345986077549\\
0.735	0.0128361328661109	0.159710674111862	0.741368418705642\\
0.735	0.0153495058140643	0.174537687332543	0.745784862958201\\
0.735	0.0180991103316243	0.189395466863524	0.750625176092311\\
0.735	0.0210872565164405	0.204278830634725	0.755910489868999\\
0.735	0.0243161165387281	0.219182480034174	0.761650724664453\\
0.735	0.0277877186778607	0.234101003941464	0.767842183867616\\
0.735	0.0315039414701067	0.24902888308801	0.77446537092041\\
0.735	0.0354665079868145	0.263960494742775	0.781483180780606\\
0.735	0.0396769802625738	0.278890117720924	0.788839618268073\\
0.735	0.0441367538930258	0.293811937711588	0.796459186325543\\
0.735	0.0488470528220538	0.308720052919643	0.804247066923455\\
0.735	0.0538089243380495	0.323608480015096	0.81209018616195\\
0.735	0.0590232342988274	0.338471160382329	0.819859213932692\\
0.735	0.064490662604533	0.3533019666601	0.827411499063641\\
0.735	0.0702116989375697	0.368094709561873	0.834594885813506\\
0.735	0.0761866387881432	0.382843144964686	0.841252300287178\\
0.735	0.0824155797834956	0.397540981253432	0.847226939729081\\
0.735	0.0888984183382709	0.412181886906127	0.852367847905041\\
0.735	0.0956348466427212	0.42675949830445	0.85653562002783\\
0.735	0.102624350004627	0.441267427752584	0.859607954619618\\
0.735	0.109866204559871	0.455699271686218	0.861484760266621\\
0.735	0.117359475365564	0.470048619052377	0.86209253425154\\
0.735	0.125103014888515	0.484309059839721	0.861387758068368\\
0.735	0.133095461900593	0.498474193737904	0.859359100890822\\
0.735	0.14133524079124	0.512537638903683	0.856028283765737\\
0.735	0.149820561306021	0.526493040810599	0.851449530870182\\
0.735	0.158549418718625	0.540334081158348	0.845707614739509\\
0.735	0.167519594442214	0.554054486817265	0.83891458434315\\
0.735	0.176728657084455	0.567648038782867	0.831205342370839\\
0.735	0.186173963948925	0.581108581114919	0.822732305417528\\
0.735	0.195852662983903	0.594430029835237	0.813659432944305\\
0.735	0.205761695177907	0.607606381758231	0.804155944104868\\
0.735	0.215897797399558	0.620631723228143	0.794390053406967\\
0.735	0.226257505677663	0.633500238737009	0.784523046062804\\
0.735	0.236837158915675	0.646206219397571	0.774703982861591\\
0.735	0.247632903032949	0.658744071245709	0.765065275211014\\
0.735	0.258640695523527	0.671108323347402	0.755719307772962\\
0.735	0.269856310421543	0.683293635685828	0.746756214005911\\
0.735	0.28127534366066	0.695294806804924	0.738242834604294\\
0.735	0.292893218813452	0.707106781186547	0.730222816007836\\
0.735	0.304705193195075	0.71872465633934	0.722717741087768\\
0.735	0.316706364314172	0.730143689578457	0.715729131143268\\
0.735	0.328891676652598	0.741359304476472	0.709241120561856\\
0.735	0.341255928754291	0.752367096967051	0.703223584571265\\
0.735	0.353793780602429	0.763162841084325	0.697635496597416\\
0.735	0.366499761262991	0.773742494322337	0.692428303594619\\
0.735	0.379368276771856	0.784102202600443	0.687549132887071\\
0.735	0.392393618241769	0.794238304822092	0.682943679235328\\
0.735	0.405569970164763	0.804147337016097	0.678558662193044\\
0.735	0.418891418885081	0.813826036051075	0.67434378741287\\
0.735	0.432351961217132	0.823271342915544	0.670253187715552\\
0.735	0.445945513182735	0.832480405557786	0.666246357336158\\
0.735	0.459665918841652	0.841450581281375	0.662288623474186\\
0.735	0.473506959189401	0.850179438693979	0.658351221673191\\
0.735	0.487462361096317	0.85866475920876	0.654411055137886\\
0.735	0.501525806262096	0.866904538099407	0.650450223206468\\
0.735	0.515690940160279	0.874896985111485	0.646455401874959\\
0.735	0.529951380947623	0.882640524634437	0.642417151055155\\
0.735	0.544300728313781	0.890133795440129	0.638329210947586\\
0.735	0.558732572247415	0.897375649995373	0.6341878353875\\
0.735	0.57324050169555	0.904365153357279	0.629991194998375\\
0.735	0.587818113093873	0.911101581661729	0.625738868900923\\
0.735	0.602459018746568	0.917584420216505	0.62143143163314\\
0.735	0.617156855035314	0.923813361211857	0.617070132480552\\
0.735	0.631905290438127	0.92978830106243	0.612656657833836\\
0.735	0.6466980333399	0.935509337395467	0.608192963377088\\
0.735	0.661528839617671	0.940976765701173	0.603681161495106\\
0.735	0.676391519984903	0.94619107566195	0.599123449738606\\
0.735	0.691279947080357	0.951152947177946	0.59452206790159\\
0.735	0.706188062288412	0.955863246106974	0.589879273662389\\
0.735	0.721109882279076	0.960323019737426	0.585197329319934\\
0.735	0.736039505257226	0.964533492013185	0.580478494543976\\
0.735	0.75097111691199	0.968496058529893	0.575725022018316\\
0.735	0.765898996058536	0.972212281322139	0.570939154289561\\
0.735	0.780817519965826	0.975683883461272	0.566123121053174\\
0.735	0.795721169365275	0.978912743483559	0.561279136605559\\
0.735	0.810604533136476	0.981900889668376	0.5564093973992\\
0.735	0.825462312667457	0.984650494185936	0.551516079695736\\
0.735	0.840289325888138	0.987163867133889	0.546601337319639\\
0.735	0.855080510976838	0.989443450481767	0.541667299515138\\
0.735	0.86983092974082	0.991491811941914	0.536716068908601\\
0.735	0.884535770672926	0.993311638785082	0.531749719578141\\
0.735	0.899190351687411	0.994905731618374	0.526770295231793\\
0.735	0.913790122539012	0.996276998142666	0.521779807495103\\
0.735	0.928330666930242	0.99742844690601	0.516780234308683\\
0.735	0.942807704312699	0.998363181068902	0.511773518435874\\
0.735	0.95721709138901	0.999084392196567	0.50676156608017\\
0.735	0.971554823322703	0.999595354092743	0.501746245611877\\
0.735	0.985817034663989	0.999899416688638	0.496729386403003\\
0.735	1	1	0.491669087346897\\
0.75	0	0	0.710195569525497\\
0.75	0.000100583311362513	0.0141829653360114	0.71240460418299\\
0.75	0.000404645907256436	0.0284451766772965	0.714630690536905\\
0.75	0.000915607803432999	0.0427829086109896	0.716916312064106\\
0.75	0.00163681893109844	0.057192295687301	0.719289019360234\\
0.75	0.00257155309398959	0.0716693330697584	0.721780367061016\\
0.75	0.00372300185733414	0.0862098774609879	0.724425655492835\\
0.75	0.00509426838162598	0.100809648312589	0.727263413850304\\
0.75	0.00668836121491816	0.115464229327074	0.73033458997498\\
0.75	0.00850818805808554	0.13016907025918	0.733681422514709\\
0.75	0.0105565495182326	0.144919489023162	0.737345986077549\\
0.75	0.0128361328661109	0.159710674111862	0.741368418705642\\
0.75	0.0153495058140643	0.174537687332543	0.745784862958201\\
0.75	0.0180991103316243	0.189395466863524	0.750625176092311\\
0.75	0.0210872565164405	0.204278830634725	0.755910489868999\\
0.75	0.0243161165387281	0.219182480034174	0.761650724664453\\
0.75	0.0277877186778607	0.234101003941464	0.767842183867616\\
0.75	0.0315039414701067	0.24902888308801	0.77446537092041\\
0.75	0.0354665079868145	0.263960494742775	0.781483180780606\\
0.75	0.0396769802625738	0.278890117720924	0.788839618268073\\
0.75	0.0441367538930258	0.293811937711588	0.796459186325544\\
0.75	0.0488470528220538	0.308720052919643	0.804247066923455\\
0.75	0.0538089243380495	0.323608480015096	0.81209018616195\\
0.75	0.0590232342988274	0.338471160382329	0.819859213932694\\
0.75	0.064490662604533	0.3533019666601	0.827411499063638\\
0.75	0.0702116989375697	0.368094709561873	0.834594885813505\\
0.75	0.0761866387881432	0.382843144964686	0.841252300287178\\
0.75	0.0824155797834956	0.397540981253432	0.847226939729081\\
0.75	0.0888984183382709	0.412181886906127	0.852367847905041\\
0.75	0.0956348466427212	0.42675949830445	0.856535620027829\\
0.75	0.102624350004627	0.441267427752585	0.859607954619615\\
0.75	0.109866204559871	0.455699271686218	0.861484760266622\\
0.75	0.117359475365564	0.470048619052377	0.86209253425154\\
0.75	0.125103014888515	0.484309059839721	0.861387758068368\\
0.75	0.133095461900593	0.498474193737904	0.859359100890821\\
0.75	0.14133524079124	0.512537638903683	0.856028283765735\\
0.75	0.149820561306021	0.526493040810599	0.85144953087018\\
0.75	0.158549418718625	0.540334081158348	0.845707614739507\\
0.75	0.167519594442213	0.554054486817265	0.83891458434315\\
0.75	0.176728657084455	0.567648038782867	0.831205342370836\\
0.75	0.186173963948925	0.581108581114919	0.822732305417526\\
0.75	0.195852662983903	0.594430029835236	0.813659432944305\\
0.75	0.205761695177907	0.607606381758231	0.80415594410487\\
0.75	0.215897797399558	0.620631723228143	0.794390053406974\\
0.75	0.226257505677663	0.633500238737009	0.784523046062803\\
0.75	0.236837158915675	0.646206219397571	0.774703982861587\\
0.75	0.247632903032949	0.658744071245709	0.765065275211016\\
0.75	0.258640695523527	0.671108323347402	0.755719307772963\\
0.75	0.269856310421543	0.683293635685828	0.746756214005909\\
0.75	0.28127534366066	0.695294806804924	0.738242834604295\\
0.75	0.292893218813452	0.707106781186547	0.730222816007835\\
0.75	0.304705193195075	0.71872465633934	0.722717741087768\\
0.75	0.316706364314172	0.730143689578457	0.715729131143269\\
0.75	0.328891676652598	0.741359304476472	0.709241120561857\\
0.75	0.341255928754291	0.752367096967051	0.703223584571265\\
0.75	0.353793780602429	0.763162841084325	0.697635496597417\\
0.75	0.366499761262991	0.773742494322337	0.692428303594618\\
0.75	0.379368276771856	0.784102202600442	0.687549132887069\\
0.75	0.392393618241769	0.794238304822092	0.682943679235329\\
0.75	0.405569970164763	0.804147337016097	0.678558662193044\\
0.75	0.418891418885081	0.813826036051075	0.67434378741287\\
0.75	0.432351961217132	0.823271342915544	0.670253187715552\\
0.75	0.445945513182735	0.832480405557786	0.666246357336159\\
0.75	0.459665918841652	0.841450581281375	0.662288623474187\\
0.75	0.473506959189401	0.850179438693979	0.658351221673192\\
0.75	0.487462361096317	0.85866475920876	0.654411055137886\\
0.75	0.501525806262096	0.866904538099407	0.650450223206467\\
0.75	0.515690940160279	0.874896985111485	0.646455401874959\\
0.75	0.529951380947623	0.882640524634436	0.642417151055155\\
0.75	0.544300728313782	0.890133795440129	0.638329210947586\\
0.75	0.558732572247415	0.897375649995373	0.634187835387499\\
0.75	0.57324050169555	0.904365153357279	0.629991194998377\\
0.75	0.587818113093873	0.911101581661729	0.625738868900924\\
0.75	0.602459018746568	0.917584420216505	0.62143143163314\\
0.75	0.617156855035314	0.923813361211857	0.617070132480552\\
0.75	0.631905290438127	0.92978830106243	0.612656657833837\\
0.75	0.6466980333399	0.935509337395467	0.608192963377087\\
0.75	0.661528839617671	0.940976765701173	0.603681161495107\\
0.75	0.676391519984903	0.94619107566195	0.599123449738607\\
0.75	0.691279947080357	0.951152947177946	0.594522067901589\\
0.75	0.706188062288412	0.955863246106974	0.58987927366239\\
0.75	0.721109882279076	0.960323019737426	0.585197329319937\\
0.75	0.736039505257226	0.964533492013186	0.580478494543974\\
0.75	0.75097111691199	0.968496058529893	0.575725022018314\\
0.75	0.765898996058536	0.972212281322139	0.57093915428956\\
0.75	0.780817519965826	0.975683883461272	0.566123121053172\\
0.75	0.795721169365275	0.978912743483559	0.561279136605559\\
0.75	0.810604533136476	0.981900889668376	0.556409397399199\\
0.75	0.825462312667457	0.984650494185936	0.551516079695735\\
0.75	0.840289325888138	0.987163867133889	0.546601337319643\\
0.75	0.855080510976838	0.989443450481768	0.541667299515137\\
0.75	0.86983092974082	0.991491811941914	0.536716068908599\\
0.75	0.884535770672926	0.993311638785082	0.531749719578141\\
0.75	0.899190351687411	0.994905731618374	0.526770295231793\\
0.75	0.913790122539012	0.996276998142666	0.521779807495104\\
0.75	0.928330666930242	0.997428446906011	0.516780234308683\\
0.75	0.942807704312699	0.998363181068901	0.511773518435871\\
0.75	0.95721709138901	0.999084392196567	0.506761566080172\\
0.75	0.971554823322704	0.999595354092744	0.501746245611877\\
0.75	0.985817034663989	0.999899416688638	0.496729386403002\\
0.75	1	1	0.491669087346898\\
0.765	0	0	0.710195569525497\\
0.765	0.000100583311362513	0.0141829653360114	0.71240460418299\\
0.765	0.000404645907256436	0.0284451766772965	0.714630690536905\\
0.765	0.000915607803432999	0.0427829086109896	0.716916312064106\\
0.765	0.00163681893109844	0.057192295687301	0.719289019360234\\
0.765	0.00257155309398959	0.0716693330697584	0.721780367061016\\
0.765	0.00372300185733413	0.0862098774609879	0.724425655492835\\
0.765	0.00509426838162598	0.100809648312589	0.727263413850304\\
0.765	0.00668836121491815	0.115464229327074	0.73033458997498\\
0.765	0.00850818805808554	0.13016907025918	0.733681422514709\\
0.765	0.0105565495182326	0.144919489023162	0.737345986077549\\
0.765	0.0128361328661109	0.159710674111862	0.741368418705642\\
0.765	0.0153495058140643	0.174537687332543	0.745784862958201\\
0.765	0.0180991103316243	0.189395466863524	0.750625176092311\\
0.765	0.0210872565164405	0.204278830634725	0.755910489868999\\
0.765	0.0243161165387281	0.219182480034174	0.761650724664453\\
0.765	0.0277877186778607	0.234101003941464	0.767842183867616\\
0.765	0.0315039414701067	0.24902888308801	0.77446537092041\\
0.765	0.0354665079868145	0.263960494742775	0.781483180780606\\
0.765	0.0396769802625738	0.278890117720924	0.788839618268073\\
0.765	0.0441367538930258	0.293811937711588	0.796459186325544\\
0.765	0.0488470528220538	0.308720052919643	0.804247066923455\\
0.765	0.0538089243380495	0.323608480015096	0.81209018616195\\
0.765	0.0590232342988274	0.338471160382329	0.819859213932693\\
0.765	0.064490662604533	0.3533019666601	0.827411499063639\\
0.765	0.0702116989375697	0.368094709561873	0.834594885813504\\
0.765	0.0761866387881432	0.382843144964686	0.841252300287179\\
0.765	0.0824155797834956	0.397540981253432	0.847226939729079\\
0.765	0.0888984183382709	0.412181886906127	0.852367847905042\\
0.765	0.0956348466427212	0.42675949830445	0.856535620027829\\
0.765	0.102624350004627	0.441267427752584	0.859607954619616\\
0.765	0.109866204559871	0.455699271686218	0.861484760266623\\
0.765	0.117359475365564	0.470048619052377	0.862092534251539\\
0.765	0.125103014888515	0.484309059839721	0.86138775806837\\
0.765	0.133095461900593	0.498474193737904	0.85935910089082\\
0.765	0.14133524079124	0.512537638903683	0.856028283765736\\
0.765	0.149820561306021	0.526493040810599	0.851449530870183\\
0.765	0.158549418718625	0.540334081158348	0.845707614739503\\
0.765	0.167519594442213	0.554054486817265	0.838914584343152\\
0.765	0.176728657084455	0.567648038782867	0.831205342370836\\
0.765	0.186173963948925	0.581108581114919	0.822732305417527\\
0.765	0.195852662983903	0.594430029835237	0.813659432944305\\
0.765	0.205761695177907	0.607606381758231	0.804155944104869\\
0.765	0.215897797399558	0.620631723228143	0.794390053406972\\
0.765	0.226257505677663	0.633500238737009	0.784523046062799\\
0.765	0.236837158915675	0.646206219397571	0.774703982861589\\
0.765	0.247632903032949	0.658744071245709	0.765065275211016\\
0.765	0.258640695523527	0.671108323347402	0.755719307772962\\
0.765	0.269856310421543	0.683293635685828	0.74675621400591\\
0.765	0.28127534366066	0.695294806804924	0.738242834604297\\
0.765	0.292893218813452	0.707106781186547	0.730222816007835\\
0.765	0.304705193195075	0.71872465633934	0.722717741087769\\
0.765	0.316706364314172	0.730143689578457	0.715729131143269\\
0.765	0.328891676652598	0.741359304476472	0.709241120561857\\
0.765	0.341255928754291	0.752367096967051	0.703223584571264\\
0.765	0.353793780602429	0.763162841084325	0.697635496597417\\
0.765	0.366499761262991	0.773742494322337	0.692428303594618\\
0.765	0.379368276771856	0.784102202600442	0.687549132887069\\
0.765	0.392393618241769	0.794238304822092	0.682943679235329\\
0.765	0.405569970164763	0.804147337016097	0.678558662193044\\
0.765	0.418891418885081	0.813826036051075	0.674343787412869\\
0.765	0.432351961217133	0.823271342915544	0.670253187715552\\
0.765	0.445945513182735	0.832480405557786	0.66624635733616\\
0.765	0.459665918841652	0.841450581281375	0.662288623474186\\
0.765	0.473506959189401	0.850179438693979	0.658351221673191\\
0.765	0.487462361096317	0.85866475920876	0.654411055137886\\
0.765	0.501525806262096	0.866904538099407	0.650450223206468\\
0.765	0.515690940160279	0.874896985111485	0.646455401874959\\
0.765	0.529951380947623	0.882640524634437	0.642417151055155\\
0.765	0.544300728313782	0.890133795440129	0.638329210947585\\
0.765	0.558732572247415	0.897375649995373	0.634187835387498\\
0.765	0.57324050169555	0.904365153357279	0.629991194998378\\
0.765	0.587818113093873	0.911101581661729	0.625738868900926\\
0.765	0.602459018746568	0.917584420216505	0.62143143163314\\
0.765	0.617156855035314	0.923813361211857	0.617070132480551\\
0.765	0.631905290438127	0.92978830106243	0.612656657833837\\
0.765	0.6466980333399	0.935509337395467	0.608192963377086\\
0.765	0.661528839617671	0.940976765701173	0.603681161495103\\
0.765	0.676391519984903	0.94619107566195	0.599123449738608\\
0.765	0.691279947080357	0.951152947177946	0.594522067901592\\
0.765	0.706188062288412	0.955863246106974	0.589879273662389\\
0.765	0.721109882279076	0.960323019737426	0.585197329319937\\
0.765	0.736039505257226	0.964533492013186	0.580478494543975\\
0.765	0.75097111691199	0.968496058529893	0.575725022018311\\
0.765	0.765898996058536	0.972212281322139	0.570939154289563\\
0.765	0.780817519965826	0.975683883461272	0.566123121053178\\
0.765	0.795721169365275	0.97891274348356	0.561279136605557\\
0.765	0.810604533136476	0.981900889668376	0.556409397399199\\
0.765	0.825462312667457	0.984650494185936	0.551516079695733\\
0.765	0.840289325888138	0.987163867133889	0.54660133731964\\
0.765	0.855080510976839	0.989443450481768	0.54166729951514\\
0.765	0.86983092974082	0.991491811941914	0.536716068908599\\
0.765	0.884535770672926	0.993311638785082	0.531749719578141\\
0.765	0.899190351687411	0.994905731618374	0.526770295231792\\
0.765	0.913790122539012	0.996276998142666	0.521779807495104\\
0.765	0.928330666930242	0.997428446906011	0.516780234308687\\
0.765	0.942807704312699	0.998363181068902	0.51177351843587\\
0.765	0.95721709138901	0.999084392196567	0.506761566080164\\
0.765	0.971554823322703	0.999595354092743	0.501746245611877\\
0.765	0.985817034663989	0.999899416688637	0.496729386403005\\
0.765	1	1	0.491669087346904\\
0.78	0	0	0.710195569525497\\
0.78	0.000100583311362513	0.0141829653360114	0.71240460418299\\
0.78	0.000404645907256436	0.0284451766772965	0.714630690536905\\
0.78	0.000915607803432999	0.0427829086109896	0.716916312064106\\
0.78	0.00163681893109844	0.057192295687301	0.719289019360234\\
0.78	0.00257155309398959	0.0716693330697584	0.721780367061016\\
0.78	0.00372300185733413	0.0862098774609879	0.724425655492835\\
0.78	0.00509426838162598	0.100809648312589	0.727263413850304\\
0.78	0.00668836121491816	0.115464229327074	0.73033458997498\\
0.78	0.00850818805808555	0.13016907025918	0.733681422514709\\
0.78	0.0105565495182326	0.144919489023162	0.737345986077549\\
0.78	0.0128361328661109	0.159710674111862	0.741368418705642\\
0.78	0.0153495058140643	0.174537687332543	0.745784862958201\\
0.78	0.0180991103316243	0.189395466863524	0.750625176092311\\
0.78	0.0210872565164405	0.204278830634725	0.755910489868999\\
0.78	0.0243161165387281	0.219182480034174	0.761650724664453\\
0.78	0.0277877186778607	0.234101003941464	0.767842183867616\\
0.78	0.0315039414701067	0.24902888308801	0.77446537092041\\
0.78	0.0354665079868145	0.263960494742775	0.781483180780606\\
0.78	0.0396769802625738	0.278890117720924	0.788839618268073\\
0.78	0.0441367538930258	0.293811937711588	0.796459186325544\\
0.78	0.0488470528220537	0.308720052919643	0.804247066923454\\
0.78	0.0538089243380495	0.323608480015096	0.81209018616195\\
0.78	0.0590232342988274	0.338471160382329	0.819859213932692\\
0.78	0.064490662604533	0.3533019666601	0.827411499063638\\
0.78	0.0702116989375697	0.368094709561873	0.834594885813505\\
0.78	0.0761866387881432	0.382843144964686	0.84125230028718\\
0.78	0.0824155797834956	0.397540981253432	0.84722693972908\\
0.78	0.0888984183382709	0.412181886906127	0.852367847905041\\
0.78	0.0956348466427212	0.42675949830445	0.856535620027829\\
0.78	0.102624350004627	0.441267427752584	0.859607954619619\\
0.78	0.109866204559871	0.455699271686218	0.861484760266623\\
0.78	0.117359475365564	0.470048619052377	0.862092534251541\\
0.78	0.125103014888515	0.484309059839721	0.861387758068371\\
0.78	0.133095461900593	0.498474193737904	0.85935910089082\\
0.78	0.14133524079124	0.512537638903683	0.856028283765737\\
0.78	0.149820561306021	0.526493040810599	0.851449530870186\\
0.78	0.158549418718625	0.540334081158348	0.845707614739503\\
0.78	0.167519594442213	0.554054486817265	0.838914584343148\\
0.78	0.176728657084455	0.567648038782867	0.831205342370836\\
0.78	0.186173963948925	0.581108581114919	0.822732305417526\\
0.78	0.195852662983903	0.594430029835236	0.813659432944303\\
0.78	0.205761695177907	0.607606381758231	0.804155944104871\\
0.78	0.215897797399558	0.620631723228143	0.794390053406971\\
0.78	0.226257505677663	0.633500238737009	0.7845230460628\\
0.78	0.236837158915675	0.646206219397571	0.774703982861591\\
0.78	0.247632903032949	0.658744071245709	0.765065275211016\\
0.78	0.258640695523527	0.671108323347402	0.755719307772963\\
0.78	0.269856310421543	0.683293635685828	0.746756214005912\\
0.78	0.28127534366066	0.695294806804924	0.738242834604294\\
0.78	0.292893218813452	0.707106781186547	0.730222816007835\\
0.78	0.304705193195075	0.71872465633934	0.72271774108777\\
0.78	0.316706364314172	0.730143689578457	0.715729131143266\\
0.78	0.328891676652598	0.741359304476472	0.709241120561856\\
0.78	0.341255928754291	0.752367096967051	0.703223584571266\\
0.78	0.353793780602429	0.763162841084325	0.697635496597418\\
0.78	0.366499761262991	0.773742494322337	0.692428303594617\\
0.78	0.379368276771857	0.784102202600442	0.68754913288707\\
0.78	0.392393618241769	0.794238304822092	0.68294367923533\\
0.78	0.405569970164763	0.804147337016097	0.678558662193044\\
0.78	0.418891418885081	0.813826036051075	0.674343787412869\\
0.78	0.432351961217132	0.823271342915544	0.670253187715551\\
0.78	0.445945513182735	0.832480405557786	0.66624635733616\\
0.78	0.459665918841652	0.841450581281375	0.662288623474187\\
0.78	0.473506959189401	0.850179438693979	0.658351221673192\\
0.78	0.487462361096317	0.85866475920876	0.654411055137886\\
0.78	0.501525806262096	0.866904538099407	0.650450223206467\\
0.78	0.515690940160279	0.874896985111485	0.646455401874959\\
0.78	0.529951380947623	0.882640524634437	0.642417151055156\\
0.78	0.544300728313782	0.890133795440129	0.638329210947585\\
0.78	0.558732572247415	0.897375649995372	0.634187835387497\\
0.78	0.57324050169555	0.904365153357279	0.629991194998377\\
0.78	0.587818113093873	0.911101581661729	0.625738868900925\\
0.78	0.602459018746568	0.917584420216504	0.621431431633141\\
0.78	0.617156855035314	0.923813361211857	0.617070132480554\\
0.78	0.631905290438127	0.92978830106243	0.612656657833838\\
0.78	0.6466980333399	0.935509337395467	0.608192963377089\\
0.78	0.661528839617671	0.940976765701173	0.603681161495103\\
0.78	0.676391519984904	0.94619107566195	0.599123449738606\\
0.78	0.691279947080357	0.951152947177946	0.59452206790159\\
0.78	0.706188062288412	0.955863246106974	0.589879273662389\\
0.78	0.721109882279076	0.960323019737426	0.585197329319937\\
0.78	0.736039505257226	0.964533492013186	0.580478494543976\\
0.78	0.75097111691199	0.968496058529893	0.575725022018312\\
0.78	0.765898996058536	0.972212281322139	0.570939154289559\\
0.78	0.780817519965826	0.975683883461272	0.566123121053177\\
0.78	0.795721169365275	0.97891274348356	0.561279136605561\\
0.78	0.810604533136476	0.981900889668376	0.556409397399199\\
0.78	0.825462312667457	0.984650494185936	0.551516079695734\\
0.78	0.840289325888138	0.987163867133889	0.546601337319638\\
0.78	0.855080510976838	0.989443450481767	0.54166729951514\\
0.78	0.86983092974082	0.991491811941914	0.5367160689086\\
0.78	0.884535770672926	0.993311638785082	0.531749719578144\\
0.78	0.899190351687411	0.994905731618374	0.526770295231792\\
0.78	0.913790122539012	0.996276998142666	0.5217798074951\\
0.78	0.928330666930242	0.997428446906011	0.516780234308683\\
0.78	0.942807704312699	0.998363181068902	0.511773518435873\\
0.78	0.95721709138901	0.999084392196567	0.506761566080169\\
0.78	0.971554823322703	0.999595354092743	0.501746245611875\\
0.78	0.985817034663989	0.999899416688638	0.496729386403004\\
0.78	1	1	0.491669087346897\\
0.795	0	0	0.710195569525497\\
0.795	0.000100583311362513	0.0141829653360114	0.71240460418299\\
0.795	0.000404645907256436	0.0284451766772965	0.714630690536905\\
0.795	0.000915607803432999	0.0427829086109896	0.716916312064106\\
0.795	0.00163681893109844	0.057192295687301	0.719289019360234\\
0.795	0.00257155309398959	0.0716693330697584	0.721780367061016\\
0.795	0.00372300185733413	0.0862098774609879	0.724425655492835\\
0.795	0.00509426838162598	0.100809648312589	0.727263413850304\\
0.795	0.00668836121491816	0.115464229327074	0.73033458997498\\
0.795	0.00850818805808554	0.13016907025918	0.733681422514709\\
0.795	0.0105565495182326	0.144919489023162	0.737345986077549\\
0.795	0.0128361328661109	0.159710674111862	0.741368418705642\\
0.795	0.0153495058140643	0.174537687332543	0.745784862958201\\
0.795	0.0180991103316243	0.189395466863524	0.750625176092311\\
0.795	0.0210872565164405	0.204278830634725	0.755910489868999\\
0.795	0.0243161165387281	0.219182480034174	0.761650724664453\\
0.795	0.0277877186778607	0.234101003941464	0.767842183867616\\
0.795	0.0315039414701067	0.24902888308801	0.77446537092041\\
0.795	0.0354665079868145	0.263960494742775	0.781483180780606\\
0.795	0.0396769802625738	0.278890117720924	0.788839618268073\\
0.795	0.0441367538930258	0.293811937711588	0.796459186325544\\
0.795	0.0488470528220538	0.308720052919643	0.804247066923454\\
0.795	0.0538089243380495	0.323608480015096	0.81209018616195\\
0.795	0.0590232342988274	0.338471160382329	0.819859213932693\\
0.795	0.064490662604533	0.3533019666601	0.827411499063639\\
0.795	0.0702116989375697	0.368094709561873	0.834594885813504\\
0.795	0.0761866387881432	0.382843144964686	0.84125230028718\\
0.795	0.0824155797834956	0.397540981253432	0.847226939729082\\
0.795	0.0888984183382709	0.412181886906127	0.852367847905041\\
0.795	0.0956348466427212	0.42675949830445	0.85653562002783\\
0.795	0.102624350004627	0.441267427752585	0.859607954619618\\
0.795	0.109866204559871	0.455699271686218	0.861484760266623\\
0.795	0.117359475365564	0.470048619052377	0.862092534251542\\
0.795	0.125103014888515	0.484309059839721	0.861387758068367\\
0.795	0.133095461900593	0.498474193737904	0.859359100890819\\
0.795	0.14133524079124	0.512537638903683	0.856028283765738\\
0.795	0.149820561306021	0.526493040810599	0.851449530870184\\
0.795	0.158549418718625	0.540334081158348	0.845707614739507\\
0.795	0.167519594442213	0.554054486817265	0.83891458434315\\
0.795	0.176728657084455	0.567648038782867	0.831205342370836\\
0.795	0.186173963948925	0.581108581114919	0.822732305417528\\
0.795	0.195852662983903	0.594430029835236	0.813659432944307\\
0.795	0.205761695177907	0.607606381758231	0.804155944104871\\
0.795	0.215897797399558	0.620631723228143	0.79439005340697\\
0.795	0.226257505677663	0.633500238737009	0.784523046062803\\
0.795	0.236837158915675	0.646206219397571	0.774703982861591\\
0.795	0.247632903032949	0.658744071245709	0.765065275211013\\
0.795	0.258640695523527	0.671108323347402	0.755719307772963\\
0.795	0.269856310421543	0.683293635685828	0.74675621400591\\
0.795	0.28127534366066	0.695294806804924	0.738242834604295\\
0.795	0.292893218813452	0.707106781186547	0.730222816007836\\
0.795	0.304705193195075	0.71872465633934	0.722717741087766\\
0.795	0.316706364314172	0.730143689578457	0.715729131143268\\
0.795	0.328891676652598	0.741359304476472	0.709241120561856\\
0.795	0.341255928754291	0.752367096967051	0.703223584571265\\
0.795	0.353793780602429	0.763162841084325	0.697635496597419\\
0.795	0.366499761262991	0.773742494322337	0.692428303594618\\
0.795	0.379368276771857	0.784102202600443	0.687549132887069\\
0.795	0.392393618241769	0.794238304822092	0.682943679235329\\
0.795	0.405569970164763	0.804147337016097	0.678558662193045\\
0.795	0.418891418885081	0.813826036051075	0.674343787412871\\
0.795	0.432351961217133	0.823271342915545	0.670253187715549\\
0.795	0.445945513182734	0.832480405557786	0.666246357336157\\
0.795	0.459665918841652	0.841450581281375	0.662288623474186\\
0.795	0.473506959189401	0.850179438693979	0.658351221673191\\
0.795	0.487462361096317	0.85866475920876	0.654411055137887\\
0.795	0.501525806262096	0.866904538099407	0.650450223206468\\
0.795	0.515690940160279	0.874896985111485	0.646455401874957\\
0.795	0.529951380947623	0.882640524634436	0.642417151055155\\
0.795	0.544300728313781	0.890133795440129	0.638329210947588\\
0.795	0.558732572247415	0.897375649995373	0.634187835387497\\
0.795	0.57324050169555	0.904365153357279	0.629991194998376\\
0.795	0.587818113093873	0.911101581661729	0.625738868900926\\
0.795	0.602459018746568	0.917584420216505	0.621431431633139\\
0.795	0.617156855035314	0.923813361211857	0.617070132480553\\
0.795	0.631905290438127	0.92978830106243	0.612656657833836\\
0.795	0.6466980333399	0.935509337395467	0.608192963377088\\
0.795	0.661528839617671	0.940976765701173	0.603681161495107\\
0.795	0.676391519984903	0.94619107566195	0.599123449738607\\
0.795	0.691279947080357	0.951152947177946	0.59452206790159\\
0.795	0.706188062288412	0.955863246106974	0.589879273662389\\
0.795	0.721109882279076	0.960323019737426	0.585197329319937\\
0.795	0.736039505257226	0.964533492013186	0.580478494543977\\
0.795	0.75097111691199	0.968496058529893	0.575725022018312\\
0.795	0.765898996058536	0.972212281322139	0.570939154289559\\
0.795	0.780817519965826	0.975683883461272	0.566123121053174\\
0.795	0.795721169365275	0.978912743483559	0.561279136605561\\
0.795	0.810604533136476	0.981900889668376	0.556409397399201\\
0.795	0.825462312667457	0.984650494185936	0.551516079695735\\
0.795	0.840289325888138	0.987163867133889	0.54660133731964\\
0.795	0.855080510976839	0.989443450481767	0.541667299515139\\
0.795	0.86983092974082	0.991491811941914	0.536716068908601\\
0.795	0.884535770672926	0.993311638785082	0.531749719578144\\
0.795	0.899190351687411	0.994905731618374	0.526770295231795\\
0.795	0.913790122539012	0.996276998142666	0.521779807495101\\
0.795	0.928330666930242	0.997428446906011	0.516780234308682\\
0.795	0.942807704312699	0.998363181068902	0.511773518435872\\
0.795	0.95721709138901	0.999084392196567	0.506761566080171\\
0.795	0.971554823322704	0.999595354092744	0.501746245611874\\
0.795	0.985817034663988	0.999899416688637	0.496729386403002\\
0.795	1	1	0.491669087346908\\
0.81	0	0	0.710195569525497\\
0.81	0.000100583311362513	0.0141829653360114	0.71240460418299\\
0.81	0.000404645907256436	0.0284451766772965	0.714630690536905\\
0.81	0.000915607803432999	0.0427829086109896	0.716916312064106\\
0.81	0.00163681893109843	0.057192295687301	0.719289019360234\\
0.81	0.00257155309398959	0.0716693330697584	0.721780367061016\\
0.81	0.00372300185733413	0.0862098774609879	0.724425655492835\\
0.81	0.00509426838162598	0.100809648312589	0.727263413850304\\
0.81	0.00668836121491815	0.115464229327074	0.73033458997498\\
0.81	0.00850818805808555	0.13016907025918	0.733681422514709\\
0.81	0.0105565495182326	0.144919489023162	0.737345986077549\\
0.81	0.0128361328661109	0.159710674111862	0.741368418705642\\
0.81	0.0153495058140643	0.174537687332543	0.745784862958201\\
0.81	0.0180991103316243	0.189395466863524	0.750625176092311\\
0.81	0.0210872565164405	0.204278830634725	0.755910489868999\\
0.81	0.0243161165387281	0.219182480034174	0.761650724664453\\
0.81	0.0277877186778607	0.234101003941464	0.767842183867616\\
0.81	0.0315039414701067	0.24902888308801	0.77446537092041\\
0.81	0.0354665079868145	0.263960494742775	0.781483180780606\\
0.81	0.0396769802625738	0.278890117720924	0.788839618268073\\
0.81	0.0441367538930258	0.293811937711588	0.796459186325544\\
0.81	0.0488470528220538	0.308720052919643	0.804247066923455\\
0.81	0.0538089243380495	0.323608480015096	0.81209018616195\\
0.81	0.0590232342988274	0.338471160382329	0.819859213932694\\
0.81	0.064490662604533	0.3533019666601	0.827411499063638\\
0.81	0.0702116989375697	0.368094709561873	0.834594885813505\\
0.81	0.0761866387881432	0.382843144964686	0.841252300287179\\
0.81	0.0824155797834956	0.397540981253432	0.847226939729082\\
0.81	0.0888984183382709	0.412181886906127	0.85236784790504\\
0.81	0.0956348466427212	0.42675949830445	0.856535620027829\\
0.81	0.102624350004627	0.441267427752584	0.859607954619618\\
0.81	0.109866204559871	0.455699271686218	0.861484760266622\\
0.81	0.117359475365564	0.470048619052377	0.862092534251539\\
0.81	0.125103014888515	0.484309059839721	0.861387758068368\\
0.81	0.133095461900593	0.498474193737904	0.85935910089082\\
0.81	0.14133524079124	0.512537638903683	0.856028283765736\\
0.81	0.149820561306021	0.526493040810599	0.851449530870183\\
0.81	0.158549418718625	0.540334081158348	0.845707614739503\\
0.81	0.167519594442213	0.554054486817265	0.83891458434315\\
0.81	0.176728657084455	0.567648038782867	0.831205342370836\\
0.81	0.186173963948925	0.581108581114919	0.822732305417527\\
0.81	0.195852662983903	0.594430029835236	0.813659432944307\\
0.81	0.205761695177907	0.607606381758231	0.804155944104871\\
0.81	0.215897797399558	0.620631723228143	0.79439005340697\\
0.81	0.226257505677663	0.633500238737009	0.784523046062803\\
0.81	0.236837158915675	0.646206219397571	0.774703982861587\\
0.81	0.247632903032949	0.658744071245709	0.765065275211016\\
0.81	0.258640695523527	0.671108323347402	0.755719307772963\\
0.81	0.269856310421543	0.683293635685828	0.74675621400591\\
0.81	0.28127534366066	0.695294806804924	0.738242834604296\\
0.81	0.292893218813452	0.707106781186547	0.730222816007836\\
0.81	0.304705193195075	0.71872465633934	0.722717741087768\\
0.81	0.316706364314172	0.730143689578457	0.715729131143269\\
0.81	0.328891676652598	0.741359304476472	0.709241120561856\\
0.81	0.341255928754291	0.752367096967051	0.703223584571265\\
0.81	0.353793780602429	0.763162841084325	0.697635496597418\\
0.81	0.366499761262991	0.773742494322337	0.692428303594619\\
0.81	0.379368276771856	0.784102202600442	0.68754913288707\\
0.81	0.392393618241769	0.794238304822092	0.682943679235329\\
0.81	0.405569970164763	0.804147337016097	0.678558662193044\\
0.81	0.418891418885081	0.813826036051075	0.674343787412872\\
0.81	0.432351961217133	0.823271342915545	0.670253187715552\\
0.81	0.445945513182735	0.832480405557786	0.666246357336158\\
0.81	0.459665918841652	0.841450581281375	0.662288623474188\\
0.81	0.473506959189401	0.850179438693979	0.658351221673189\\
0.81	0.487462361096317	0.85866475920876	0.654411055137885\\
0.81	0.501525806262096	0.866904538099407	0.65045022320647\\
0.81	0.515690940160279	0.874896985111485	0.646455401874958\\
0.81	0.529951380947623	0.882640524634436	0.642417151055154\\
0.81	0.544300728313782	0.890133795440129	0.638329210947586\\
0.81	0.558732572247415	0.897375649995373	0.634187835387498\\
0.81	0.57324050169555	0.904365153357279	0.629991194998375\\
0.81	0.587818113093873	0.911101581661729	0.625738868900925\\
0.81	0.602459018746568	0.917584420216504	0.62143143163314\\
0.81	0.617156855035314	0.923813361211857	0.617070132480553\\
0.81	0.631905290438127	0.92978830106243	0.612656657833837\\
0.81	0.6466980333399	0.935509337395467	0.608192963377087\\
0.81	0.661528839617671	0.940976765701173	0.603681161495106\\
0.81	0.676391519984904	0.94619107566195	0.599123449738607\\
0.81	0.691279947080357	0.951152947177946	0.594522067901591\\
0.81	0.706188062288412	0.955863246106974	0.58987927366239\\
0.81	0.721109882279076	0.960323019737426	0.585197329319934\\
0.81	0.736039505257226	0.964533492013185	0.580478494543978\\
0.81	0.75097111691199	0.968496058529893	0.575725022018317\\
0.81	0.765898996058536	0.972212281322139	0.570939154289556\\
0.81	0.780817519965826	0.975683883461272	0.566123121053172\\
0.81	0.795721169365275	0.978912743483559	0.56127913660556\\
0.81	0.810604533136476	0.981900889668376	0.556409397399203\\
0.81	0.825462312667457	0.984650494185936	0.551516079695735\\
0.81	0.840289325888138	0.987163867133889	0.546601337319639\\
0.81	0.855080510976839	0.989443450481767	0.54166729951514\\
0.81	0.86983092974082	0.991491811941914	0.536716068908599\\
0.81	0.884535770672926	0.993311638785082	0.531749719578143\\
0.81	0.899190351687411	0.994905731618374	0.526770295231793\\
0.81	0.913790122539012	0.996276998142666	0.521779807495103\\
0.81	0.928330666930242	0.997428446906011	0.516780234308683\\
0.81	0.942807704312699	0.998363181068902	0.511773518435872\\
0.81	0.95721709138901	0.999084392196567	0.50676156608017\\
0.81	0.971554823322704	0.999595354092743	0.501746245611877\\
0.81	0.985817034663989	0.999899416688638	0.496729386403004\\
0.81	1	1	0.491669087346897\\
0.825	0	0	0.710195569525497\\
0.825	0.000100583311362513	0.0141829653360114	0.71240460418299\\
0.825	0.000404645907256436	0.0284451766772965	0.714630690536905\\
0.825	0.000915607803432999	0.0427829086109896	0.716916312064106\\
0.825	0.00163681893109843	0.057192295687301	0.719289019360234\\
0.825	0.00257155309398959	0.0716693330697584	0.721780367061016\\
0.825	0.00372300185733413	0.0862098774609879	0.724425655492835\\
0.825	0.00509426838162598	0.100809648312589	0.727263413850304\\
0.825	0.00668836121491815	0.115464229327074	0.73033458997498\\
0.825	0.00850818805808555	0.13016907025918	0.733681422514709\\
0.825	0.0105565495182326	0.144919489023162	0.737345986077549\\
0.825	0.0128361328661109	0.159710674111862	0.741368418705642\\
0.825	0.0153495058140643	0.174537687332543	0.745784862958202\\
0.825	0.0180991103316243	0.189395466863524	0.750625176092311\\
0.825	0.0210872565164405	0.204278830634725	0.755910489868998\\
0.825	0.0243161165387281	0.219182480034174	0.761650724664453\\
0.825	0.0277877186778607	0.234101003941464	0.767842183867616\\
0.825	0.0315039414701067	0.24902888308801	0.77446537092041\\
0.825	0.0354665079868145	0.263960494742775	0.781483180780606\\
0.825	0.0396769802625738	0.278890117720924	0.788839618268073\\
0.825	0.0441367538930258	0.293811937711588	0.796459186325544\\
0.825	0.0488470528220538	0.308720052919643	0.804247066923455\\
0.825	0.0538089243380495	0.323608480015096	0.812090186161949\\
0.825	0.0590232342988274	0.338471160382329	0.819859213932695\\
0.825	0.064490662604533	0.3533019666601	0.82741149906364\\
0.825	0.0702116989375697	0.368094709561873	0.834594885813506\\
0.825	0.0761866387881432	0.382843144964686	0.841252300287179\\
0.825	0.0824155797834956	0.397540981253432	0.847226939729081\\
0.825	0.0888984183382709	0.412181886906127	0.85236784790504\\
0.825	0.0956348466427212	0.42675949830445	0.856535620027829\\
0.825	0.102624350004627	0.441267427752584	0.859607954619619\\
0.825	0.109866204559871	0.455699271686218	0.861484760266621\\
0.825	0.117359475365564	0.470048619052377	0.862092534251539\\
0.825	0.125103014888515	0.484309059839721	0.861387758068369\\
0.825	0.133095461900593	0.498474193737904	0.85935910089082\\
0.825	0.14133524079124	0.512537638903683	0.856028283765735\\
0.825	0.149820561306021	0.526493040810599	0.851449530870178\\
0.825	0.158549418718625	0.540334081158348	0.845707614739503\\
0.825	0.167519594442213	0.554054486817265	0.838914584343148\\
0.825	0.176728657084455	0.567648038782867	0.831205342370837\\
0.825	0.186173963948925	0.581108581114919	0.822732305417527\\
0.825	0.195852662983903	0.594430029835236	0.813659432944307\\
0.825	0.205761695177907	0.607606381758231	0.804155944104873\\
0.825	0.215897797399558	0.620631723228143	0.794390053406971\\
0.825	0.226257505677663	0.633500238737009	0.784523046062802\\
0.825	0.236837158915675	0.646206219397571	0.77470398286159\\
0.825	0.247632903032949	0.658744071245709	0.765065275211017\\
0.825	0.258640695523527	0.671108323347402	0.755719307772962\\
0.825	0.269856310421543	0.683293635685828	0.74675621400591\\
0.825	0.28127534366066	0.695294806804924	0.738242834604298\\
0.825	0.292893218813452	0.707106781186548	0.730222816007836\\
0.825	0.304705193195075	0.71872465633934	0.722717741087767\\
0.825	0.316706364314172	0.730143689578457	0.715729131143268\\
0.825	0.328891676652598	0.741359304476472	0.709241120561857\\
0.825	0.341255928754291	0.752367096967051	0.703223584571265\\
0.825	0.353793780602429	0.763162841084325	0.697635496597416\\
0.825	0.366499761262991	0.773742494322337	0.692428303594618\\
0.825	0.379368276771857	0.784102202600442	0.687549132887071\\
0.825	0.392393618241769	0.794238304822092	0.682943679235329\\
0.825	0.405569970164763	0.804147337016097	0.678558662193045\\
0.825	0.418891418885081	0.813826036051075	0.67434378741287\\
0.825	0.432351961217132	0.823271342915544	0.670253187715551\\
0.825	0.445945513182735	0.832480405557786	0.666246357336158\\
0.825	0.459665918841652	0.841450581281375	0.662288623474187\\
0.825	0.473506959189401	0.850179438693979	0.658351221673191\\
0.825	0.487462361096317	0.85866475920876	0.654411055137885\\
0.825	0.501525806262096	0.866904538099407	0.650450223206469\\
0.825	0.515690940160279	0.874896985111485	0.646455401874958\\
0.825	0.529951380947623	0.882640524634436	0.642417151055155\\
0.825	0.544300728313782	0.890133795440129	0.638329210947585\\
0.825	0.558732572247415	0.897375649995372	0.634187835387497\\
0.825	0.57324050169555	0.904365153357279	0.629991194998378\\
0.825	0.587818113093873	0.911101581661729	0.625738868900926\\
0.825	0.602459018746568	0.917584420216505	0.621431431633139\\
0.825	0.617156855035314	0.923813361211857	0.617070132480552\\
0.825	0.631905290438127	0.92978830106243	0.612656657833837\\
0.825	0.6466980333399	0.935509337395467	0.608192963377088\\
0.825	0.661528839617671	0.940976765701173	0.603681161495106\\
0.825	0.676391519984904	0.94619107566195	0.599123449738607\\
0.825	0.691279947080357	0.951152947177946	0.594522067901589\\
0.825	0.706188062288412	0.955863246106974	0.589879273662389\\
0.825	0.721109882279076	0.960323019737426	0.585197329319938\\
0.825	0.736039505257226	0.964533492013186	0.580478494543975\\
0.825	0.75097111691199	0.968496058529893	0.575725022018313\\
0.825	0.765898996058536	0.972212281322139	0.57093915428956\\
0.825	0.780817519965826	0.975683883461272	0.566123121053173\\
0.825	0.795721169365275	0.978912743483559	0.561279136605558\\
0.825	0.810604533136476	0.981900889668376	0.556409397399198\\
0.825	0.825462312667457	0.984650494185936	0.551516079695737\\
0.825	0.840289325888138	0.987163867133889	0.546601337319642\\
0.825	0.855080510976839	0.989443450481767	0.54166729951514\\
0.825	0.86983092974082	0.991491811941915	0.536716068908598\\
0.825	0.884535770672926	0.993311638785082	0.531749719578141\\
0.825	0.899190351687411	0.994905731618374	0.526770295231796\\
0.825	0.913790122539012	0.996276998142666	0.521779807495102\\
0.825	0.928330666930242	0.997428446906011	0.516780234308683\\
0.825	0.942807704312699	0.998363181068902	0.511773518435872\\
0.825	0.95721709138901	0.999084392196567	0.506761566080168\\
0.825	0.971554823322703	0.999595354092743	0.501746245611874\\
0.825	0.985817034663989	0.999899416688637	0.496729386403004\\
0.825	1	1	0.491669087346906\\
0.84	0	0	0.710195569525497\\
0.84	0.000100583311362513	0.0141829653360114	0.71240460418299\\
0.84	0.000404645907256436	0.0284451766772965	0.714630690536905\\
0.84	0.000915607803432999	0.0427829086109896	0.716916312064106\\
0.84	0.00163681893109844	0.057192295687301	0.719289019360234\\
0.84	0.00257155309398959	0.0716693330697584	0.721780367061016\\
0.84	0.00372300185733413	0.0862098774609879	0.724425655492835\\
0.84	0.00509426838162598	0.100809648312589	0.727263413850304\\
0.84	0.00668836121491815	0.115464229327074	0.73033458997498\\
0.84	0.00850818805808555	0.13016907025918	0.733681422514709\\
0.84	0.0105565495182326	0.144919489023162	0.737345986077549\\
0.84	0.0128361328661109	0.159710674111862	0.741368418705642\\
0.84	0.0153495058140643	0.174537687332543	0.745784862958201\\
0.84	0.0180991103316243	0.189395466863524	0.750625176092311\\
0.84	0.0210872565164405	0.204278830634725	0.755910489868999\\
0.84	0.0243161165387281	0.219182480034174	0.761650724664453\\
0.84	0.0277877186778607	0.234101003941464	0.767842183867616\\
0.84	0.0315039414701067	0.24902888308801	0.77446537092041\\
0.84	0.0354665079868145	0.263960494742775	0.781483180780606\\
0.84	0.0396769802625738	0.278890117720924	0.788839618268073\\
0.84	0.0441367538930258	0.293811937711588	0.796459186325544\\
0.84	0.0488470528220538	0.308720052919643	0.804247066923455\\
0.84	0.0538089243380495	0.323608480015096	0.81209018616195\\
0.84	0.0590232342988274	0.338471160382329	0.819859213932693\\
0.84	0.0644906626045329	0.3533019666601	0.82741149906364\\
0.84	0.0702116989375697	0.368094709561873	0.834594885813505\\
0.84	0.0761866387881432	0.382843144964686	0.841252300287181\\
0.84	0.0824155797834956	0.397540981253432	0.84722693972908\\
0.84	0.0888984183382709	0.412181886906127	0.852367847905041\\
0.84	0.0956348466427212	0.42675949830445	0.856535620027829\\
0.84	0.102624350004627	0.441267427752584	0.859607954619617\\
0.84	0.109866204559871	0.455699271686218	0.861484760266622\\
0.84	0.117359475365564	0.470048619052377	0.862092534251541\\
0.84	0.125103014888515	0.484309059839721	0.861387758068367\\
0.84	0.133095461900593	0.498474193737904	0.859359100890821\\
0.84	0.14133524079124	0.512537638903683	0.856028283765736\\
0.84	0.149820561306021	0.526493040810599	0.851449530870183\\
0.84	0.158549418718625	0.540334081158348	0.845707614739508\\
0.84	0.167519594442213	0.554054486817265	0.838914584343152\\
0.84	0.176728657084455	0.567648038782867	0.831205342370837\\
0.84	0.186173963948925	0.581108581114919	0.822732305417527\\
0.84	0.195852662983903	0.594430029835236	0.813659432944307\\
0.84	0.205761695177907	0.607606381758231	0.804155944104869\\
0.84	0.215897797399558	0.620631723228143	0.794390053406972\\
0.84	0.226257505677663	0.633500238737009	0.784523046062804\\
0.84	0.236837158915675	0.646206219397571	0.774703982861589\\
0.84	0.247632903032949	0.658744071245709	0.765065275211014\\
0.84	0.258640695523527	0.671108323347402	0.755719307772962\\
0.84	0.269856310421543	0.683293635685828	0.746756214005911\\
0.84	0.28127534366066	0.695294806804924	0.738242834604297\\
0.84	0.292893218813452	0.707106781186548	0.730222816007835\\
0.84	0.304705193195075	0.71872465633934	0.722717741087766\\
0.84	0.316706364314172	0.730143689578457	0.715729131143267\\
0.84	0.328891676652598	0.741359304476472	0.709241120561857\\
0.84	0.341255928754291	0.752367096967051	0.703223584571265\\
0.84	0.353793780602429	0.763162841084325	0.697635496597416\\
0.84	0.366499761262991	0.773742494322337	0.692428303594618\\
0.84	0.379368276771856	0.784102202600442	0.68754913288707\\
0.84	0.392393618241769	0.794238304822092	0.682943679235329\\
0.84	0.405569970164763	0.804147337016097	0.678558662193045\\
0.84	0.418891418885081	0.813826036051075	0.674343787412873\\
0.84	0.432351961217133	0.823271342915545	0.670253187715551\\
0.84	0.445945513182735	0.832480405557786	0.666246357336157\\
0.84	0.459665918841652	0.841450581281375	0.662288623474186\\
0.84	0.473506959189401	0.850179438693979	0.658351221673192\\
0.84	0.487462361096317	0.85866475920876	0.654411055137886\\
0.84	0.501525806262096	0.866904538099407	0.650450223206468\\
0.84	0.515690940160279	0.874896985111485	0.646455401874958\\
0.84	0.529951380947623	0.882640524634436	0.642417151055154\\
0.84	0.544300728313782	0.890133795440129	0.638329210947588\\
0.84	0.558732572247415	0.897375649995373	0.634187835387497\\
0.84	0.57324050169555	0.904365153357279	0.629991194998375\\
0.84	0.587818113093873	0.911101581661729	0.625738868900925\\
0.84	0.602459018746568	0.917584420216504	0.62143143163314\\
0.84	0.617156855035314	0.923813361211857	0.617070132480553\\
0.84	0.631905290438127	0.92978830106243	0.612656657833838\\
0.84	0.6466980333399	0.935509337395467	0.608192963377087\\
0.84	0.661528839617671	0.940976765701173	0.603681161495106\\
0.84	0.676391519984904	0.94619107566195	0.599123449738606\\
0.84	0.691279947080357	0.951152947177946	0.594522067901591\\
0.84	0.706188062288412	0.955863246106974	0.58987927366239\\
0.84	0.721109882279076	0.960323019737426	0.585197329319937\\
0.84	0.736039505257225	0.964533492013186	0.580478494543976\\
0.84	0.75097111691199	0.968496058529893	0.575725022018312\\
0.84	0.765898996058536	0.972212281322139	0.570939154289561\\
0.84	0.780817519965826	0.975683883461272	0.566123121053173\\
0.84	0.795721169365275	0.978912743483559	0.561279136605557\\
0.84	0.810604533136476	0.981900889668376	0.556409397399199\\
0.84	0.825462312667457	0.984650494185936	0.551516079695734\\
0.84	0.840289325888138	0.987163867133889	0.546601337319643\\
0.84	0.855080510976839	0.989443450481767	0.541667299515139\\
0.84	0.86983092974082	0.991491811941914	0.5367160689086\\
0.84	0.884535770672926	0.993311638785082	0.531749719578143\\
0.84	0.899190351687411	0.994905731618374	0.526770295231793\\
0.84	0.913790122539012	0.996276998142666	0.521779807495103\\
0.84	0.928330666930242	0.997428446906011	0.516780234308684\\
0.84	0.942807704312699	0.998363181068902	0.511773518435871\\
0.84	0.95721709138901	0.999084392196567	0.50676156608017\\
0.84	0.971554823322703	0.999595354092743	0.501746245611878\\
0.84	0.985817034663989	0.999899416688638	0.496729386403004\\
0.84	1	1	0.491669087346897\\
0.855	0	0	0.710195569525497\\
0.855	0.000100583311362513	0.0141829653360114	0.71240460418299\\
0.855	0.000404645907256436	0.0284451766772965	0.714630690536905\\
0.855	0.000915607803432999	0.0427829086109896	0.716916312064106\\
0.855	0.00163681893109844	0.057192295687301	0.719289019360234\\
0.855	0.00257155309398959	0.0716693330697584	0.721780367061016\\
0.855	0.00372300185733414	0.0862098774609879	0.724425655492835\\
0.855	0.00509426838162598	0.100809648312589	0.727263413850304\\
0.855	0.00668836121491816	0.115464229327074	0.73033458997498\\
0.855	0.00850818805808554	0.13016907025918	0.733681422514709\\
0.855	0.0105565495182326	0.144919489023162	0.737345986077549\\
0.855	0.0128361328661109	0.159710674111862	0.741368418705642\\
0.855	0.0153495058140643	0.174537687332543	0.745784862958201\\
0.855	0.0180991103316243	0.189395466863524	0.750625176092311\\
0.855	0.0210872565164405	0.204278830634725	0.755910489868999\\
0.855	0.0243161165387281	0.219182480034174	0.761650724664453\\
0.855	0.0277877186778607	0.234101003941464	0.767842183867616\\
0.855	0.0315039414701067	0.24902888308801	0.77446537092041\\
0.855	0.0354665079868145	0.263960494742775	0.781483180780606\\
0.855	0.0396769802625738	0.278890117720924	0.788839618268073\\
0.855	0.0441367538930258	0.293811937711588	0.796459186325544\\
0.855	0.0488470528220538	0.308720052919643	0.804247066923455\\
0.855	0.0538089243380495	0.323608480015096	0.81209018616195\\
0.855	0.0590232342988274	0.338471160382329	0.819859213932693\\
0.855	0.064490662604533	0.3533019666601	0.827411499063639\\
0.855	0.0702116989375697	0.368094709561873	0.834594885813504\\
0.855	0.0761866387881432	0.382843144964686	0.841252300287181\\
0.855	0.0824155797834956	0.397540981253432	0.847226939729082\\
0.855	0.0888984183382709	0.412181886906127	0.852367847905042\\
0.855	0.0956348466427212	0.42675949830445	0.856535620027828\\
0.855	0.102624350004627	0.441267427752585	0.859607954619617\\
0.855	0.109866204559871	0.455699271686218	0.86148476026662\\
0.855	0.117359475365564	0.470048619052377	0.862092534251543\\
0.855	0.125103014888515	0.484309059839721	0.861387758068369\\
0.855	0.133095461900593	0.498474193737904	0.859359100890821\\
0.855	0.14133524079124	0.512537638903683	0.856028283765737\\
0.855	0.149820561306021	0.526493040810599	0.851449530870186\\
0.855	0.158549418718625	0.540334081158348	0.845707614739504\\
0.855	0.167519594442213	0.554054486817265	0.838914584343153\\
0.855	0.176728657084455	0.567648038782867	0.831205342370838\\
0.855	0.186173963948925	0.581108581114919	0.822732305417526\\
0.855	0.195852662983903	0.594430029835236	0.813659432944306\\
0.855	0.205761695177907	0.607606381758231	0.804155944104871\\
0.855	0.215897797399558	0.620631723228144	0.794390053406973\\
0.855	0.226257505677663	0.633500238737009	0.784523046062799\\
0.855	0.236837158915675	0.646206219397571	0.774703982861587\\
0.855	0.247632903032949	0.658744071245709	0.765065275211015\\
0.855	0.258640695523527	0.671108323347402	0.755719307772963\\
0.855	0.269856310421543	0.683293635685828	0.746756214005909\\
0.855	0.28127534366066	0.695294806804924	0.738242834604296\\
0.855	0.292893218813452	0.707106781186547	0.730222816007835\\
0.855	0.304705193195075	0.71872465633934	0.722717741087768\\
0.855	0.316706364314172	0.730143689578457	0.715729131143269\\
0.855	0.328891676652598	0.741359304476472	0.709241120561856\\
0.855	0.341255928754291	0.752367096967051	0.703223584571264\\
0.855	0.353793780602429	0.763162841084325	0.697635496597418\\
0.855	0.366499761262991	0.773742494322337	0.692428303594619\\
0.855	0.379368276771856	0.784102202600442	0.687549132887067\\
0.855	0.392393618241769	0.794238304822092	0.682943679235329\\
0.855	0.405569970164763	0.804147337016097	0.678558662193047\\
0.855	0.418891418885081	0.813826036051075	0.674343787412871\\
0.855	0.432351961217132	0.823271342915544	0.670253187715551\\
0.855	0.445945513182735	0.832480405557786	0.66624635733616\\
0.855	0.459665918841652	0.841450581281375	0.662288623474187\\
0.855	0.473506959189401	0.850179438693979	0.65835122167319\\
0.855	0.487462361096317	0.85866475920876	0.654411055137886\\
0.855	0.501525806262096	0.866904538099407	0.650450223206469\\
0.855	0.515690940160279	0.874896985111485	0.646455401874959\\
0.855	0.529951380947623	0.882640524634437	0.642417151055154\\
0.855	0.544300728313782	0.890133795440129	0.638329210947585\\
0.855	0.558732572247415	0.897375649995373	0.634187835387499\\
0.855	0.57324050169555	0.904365153357279	0.629991194998376\\
0.855	0.587818113093873	0.911101581661729	0.625738868900924\\
0.855	0.602459018746568	0.917584420216504	0.621431431633141\\
0.855	0.617156855035314	0.923813361211857	0.617070132480553\\
0.855	0.631905290438127	0.92978830106243	0.612656657833836\\
0.855	0.6466980333399	0.935509337395467	0.608192963377088\\
0.855	0.661528839617671	0.940976765701173	0.603681161495107\\
0.855	0.676391519984903	0.946191075661951	0.599123449738605\\
0.855	0.691279947080357	0.951152947177946	0.594522067901588\\
0.855	0.706188062288412	0.955863246106974	0.58987927366239\\
0.855	0.721109882279076	0.960323019737426	0.585197329319937\\
0.855	0.736039505257226	0.964533492013186	0.580478494543975\\
0.855	0.75097111691199	0.968496058529893	0.575725022018312\\
0.855	0.765898996058536	0.972212281322139	0.570939154289563\\
0.855	0.780817519965826	0.975683883461272	0.566123121053177\\
0.855	0.795721169365275	0.97891274348356	0.561279136605556\\
0.855	0.810604533136476	0.981900889668376	0.556409397399198\\
0.855	0.825462312667457	0.984650494185936	0.551516079695732\\
0.855	0.840289325888138	0.987163867133889	0.546601337319641\\
0.855	0.855080510976839	0.989443450481767	0.541667299515141\\
0.855	0.86983092974082	0.991491811941914	0.536716068908598\\
0.855	0.884535770672926	0.993311638785082	0.531749719578141\\
0.855	0.899190351687411	0.994905731618374	0.526770295231794\\
0.855	0.913790122539012	0.996276998142666	0.521779807495102\\
0.855	0.928330666930242	0.99742844690601	0.516780234308683\\
0.855	0.942807704312699	0.998363181068902	0.511773518435872\\
0.855	0.95721709138901	0.999084392196567	0.506761566080169\\
0.855	0.971554823322703	0.999595354092743	0.501746245611876\\
0.855	0.985817034663989	0.999899416688637	0.496729386403004\\
0.855	1	1	0.491669087346906\\
0.87	0	0	0.710195569525497\\
0.87	0.000100583311362513	0.0141829653360114	0.71240460418299\\
0.87	0.000404645907256436	0.0284451766772965	0.714630690536905\\
0.87	0.000915607803432999	0.0427829086109896	0.716916312064106\\
0.87	0.00163681893109844	0.057192295687301	0.719289019360234\\
0.87	0.00257155309398959	0.0716693330697584	0.721780367061016\\
0.87	0.00372300185733413	0.0862098774609879	0.724425655492835\\
0.87	0.00509426838162598	0.100809648312589	0.727263413850304\\
0.87	0.00668836121491816	0.115464229327074	0.73033458997498\\
0.87	0.00850818805808555	0.13016907025918	0.733681422514709\\
0.87	0.0105565495182326	0.144919489023162	0.737345986077549\\
0.87	0.0128361328661109	0.159710674111862	0.741368418705642\\
0.87	0.0153495058140643	0.174537687332543	0.745784862958202\\
0.87	0.0180991103316243	0.189395466863524	0.750625176092311\\
0.87	0.0210872565164405	0.204278830634725	0.755910489868999\\
0.87	0.0243161165387281	0.219182480034174	0.761650724664453\\
0.87	0.0277877186778607	0.234101003941464	0.767842183867616\\
0.87	0.0315039414701067	0.24902888308801	0.77446537092041\\
0.87	0.0354665079868145	0.263960494742775	0.781483180780606\\
0.87	0.0396769802625738	0.278890117720924	0.788839618268073\\
0.87	0.0441367538930258	0.293811937711588	0.796459186325544\\
0.87	0.0488470528220538	0.308720052919643	0.804247066923454\\
0.87	0.0538089243380495	0.323608480015097	0.81209018616195\\
0.87	0.0590232342988274	0.338471160382329	0.819859213932693\\
0.87	0.064490662604533	0.3533019666601	0.82741149906364\\
0.87	0.0702116989375697	0.368094709561873	0.834594885813505\\
0.87	0.0761866387881432	0.382843144964686	0.841252300287178\\
0.87	0.0824155797834956	0.397540981253432	0.847226939729081\\
0.87	0.0888984183382709	0.412181886906127	0.852367847905043\\
0.87	0.0956348466427212	0.42675949830445	0.856535620027829\\
0.87	0.102624350004627	0.441267427752585	0.859607954619617\\
0.87	0.109866204559871	0.455699271686218	0.861484760266618\\
0.87	0.117359475365564	0.470048619052377	0.862092534251543\\
0.87	0.125103014888515	0.484309059839721	0.86138775806837\\
0.87	0.133095461900593	0.498474193737904	0.85935910089082\\
0.87	0.14133524079124	0.512537638903683	0.856028283765736\\
0.87	0.149820561306021	0.526493040810599	0.851449530870179\\
0.87	0.158549418718625	0.540334081158348	0.845707614739507\\
0.87	0.167519594442214	0.554054486817265	0.83891458434315\\
0.87	0.176728657084455	0.567648038782867	0.831205342370836\\
0.87	0.186173963948925	0.581108581114919	0.822732305417527\\
0.87	0.195852662983903	0.594430029835236	0.813659432944308\\
0.87	0.205761695177907	0.607606381758231	0.804155944104873\\
0.87	0.215897797399558	0.620631723228144	0.79439005340697\\
0.87	0.226257505677663	0.633500238737009	0.7845230460628\\
0.87	0.236837158915675	0.646206219397571	0.77470398286159\\
0.87	0.247632903032949	0.658744071245709	0.765065275211015\\
0.87	0.258640695523527	0.671108323347402	0.755719307772964\\
0.87	0.269856310421543	0.683293635685828	0.746756214005912\\
0.87	0.28127534366066	0.695294806804925	0.738242834604298\\
0.87	0.292893218813452	0.707106781186548	0.730222816007834\\
0.87	0.304705193195075	0.71872465633934	0.722717741087766\\
0.87	0.316706364314172	0.730143689578457	0.715729131143267\\
0.87	0.328891676652598	0.741359304476472	0.709241120561857\\
0.87	0.341255928754291	0.752367096967051	0.703223584571265\\
0.87	0.353793780602429	0.763162841084325	0.697635496597419\\
0.87	0.366499761262991	0.773742494322337	0.692428303594618\\
0.87	0.379368276771857	0.784102202600443	0.687549132887069\\
0.87	0.392393618241769	0.794238304822092	0.682943679235328\\
0.87	0.405569970164763	0.804147337016097	0.678558662193044\\
0.87	0.418891418885081	0.813826036051075	0.674343787412871\\
0.87	0.432351961217132	0.823271342915544	0.670253187715552\\
0.87	0.445945513182735	0.832480405557786	0.666246357336159\\
0.87	0.459665918841652	0.841450581281375	0.662288623474188\\
0.87	0.473506959189401	0.850179438693979	0.658351221673191\\
0.87	0.487462361096317	0.85866475920876	0.654411055137884\\
0.87	0.501525806262096	0.866904538099407	0.65045022320647\\
0.87	0.515690940160279	0.874896985111485	0.64645540187496\\
0.87	0.529951380947623	0.882640524634436	0.642417151055154\\
0.87	0.544300728313782	0.890133795440129	0.638329210947585\\
0.87	0.558732572247415	0.897375649995372	0.634187835387497\\
0.87	0.57324050169555	0.904365153357279	0.629991194998377\\
0.87	0.587818113093873	0.911101581661729	0.625738868900924\\
0.87	0.602459018746568	0.917584420216504	0.621431431633138\\
0.87	0.617156855035314	0.923813361211857	0.617070132480553\\
0.87	0.631905290438127	0.92978830106243	0.61265665783384\\
0.87	0.6466980333399	0.935509337395467	0.608192963377089\\
0.87	0.661528839617671	0.940976765701173	0.603681161495103\\
0.87	0.676391519984904	0.94619107566195	0.599123449738607\\
0.87	0.691279947080357	0.951152947177946	0.59452206790159\\
0.87	0.706188062288412	0.955863246106974	0.58987927366239\\
0.87	0.721109882279076	0.960323019737426	0.585197329319937\\
0.87	0.736039505257226	0.964533492013186	0.580478494543979\\
0.87	0.75097111691199	0.968496058529893	0.575725022018313\\
0.87	0.765898996058536	0.972212281322139	0.570939154289558\\
0.87	0.780817519965826	0.975683883461272	0.566123121053175\\
0.87	0.795721169365275	0.97891274348356	0.561279136605561\\
0.87	0.810604533136476	0.981900889668376	0.556409397399201\\
0.87	0.825462312667457	0.984650494185936	0.551516079695735\\
0.87	0.840289325888138	0.987163867133889	0.546601337319639\\
0.87	0.855080510976838	0.989443450481768	0.541667299515139\\
0.87	0.86983092974082	0.991491811941914	0.536716068908599\\
0.87	0.884535770672926	0.993311638785082	0.531749719578143\\
0.87	0.899190351687411	0.994905731618374	0.526770295231793\\
0.87	0.913790122539012	0.996276998142666	0.521779807495102\\
0.87	0.928330666930242	0.997428446906011	0.516780234308683\\
0.87	0.942807704312699	0.998363181068902	0.511773518435872\\
0.87	0.95721709138901	0.999084392196567	0.506761566080167\\
0.87	0.971554823322703	0.999595354092743	0.501746245611876\\
0.87	0.985817034663989	0.999899416688638	0.496729386403003\\
0.87	1	1	0.491669087346897\\
0.885	0	0	0.710195569525497\\
0.885	0.000100583311362513	0.0141829653360114	0.71240460418299\\
0.885	0.000404645907256436	0.0284451766772965	0.714630690536905\\
0.885	0.000915607803432999	0.0427829086109896	0.716916312064106\\
0.885	0.00163681893109844	0.057192295687301	0.719289019360234\\
0.885	0.00257155309398959	0.0716693330697584	0.721780367061016\\
0.885	0.00372300185733413	0.0862098774609879	0.724425655492835\\
0.885	0.00509426838162598	0.100809648312589	0.727263413850304\\
0.885	0.00668836121491816	0.115464229327074	0.73033458997498\\
0.885	0.00850818805808555	0.13016907025918	0.733681422514709\\
0.885	0.0105565495182326	0.144919489023162	0.737345986077549\\
0.885	0.0128361328661109	0.159710674111862	0.741368418705642\\
0.885	0.0153495058140643	0.174537687332543	0.745784862958201\\
0.885	0.0180991103316243	0.189395466863524	0.750625176092311\\
0.885	0.0210872565164405	0.204278830634725	0.755910489868999\\
0.885	0.0243161165387281	0.219182480034174	0.761650724664453\\
0.885	0.0277877186778607	0.234101003941464	0.767842183867616\\
0.885	0.0315039414701067	0.24902888308801	0.77446537092041\\
0.885	0.0354665079868145	0.263960494742775	0.781483180780606\\
0.885	0.0396769802625737	0.278890117720924	0.788839618268073\\
0.885	0.0441367538930258	0.293811937711588	0.796459186325544\\
0.885	0.0488470528220538	0.308720052919643	0.804247066923455\\
0.885	0.0538089243380495	0.323608480015096	0.81209018616195\\
0.885	0.0590232342988274	0.338471160382329	0.819859213932692\\
0.885	0.064490662604533	0.3533019666601	0.827411499063639\\
0.885	0.0702116989375697	0.368094709561873	0.834594885813505\\
0.885	0.0761866387881432	0.382843144964686	0.84125230028718\\
0.885	0.0824155797834956	0.397540981253432	0.847226939729081\\
0.885	0.0888984183382709	0.412181886906127	0.852367847905041\\
0.885	0.0956348466427212	0.42675949830445	0.85653562002783\\
0.885	0.102624350004627	0.441267427752584	0.859607954619617\\
0.885	0.109866204559871	0.455699271686218	0.86148476026662\\
0.885	0.117359475365564	0.470048619052377	0.862092534251541\\
0.885	0.125103014888515	0.484309059839721	0.861387758068369\\
0.885	0.133095461900593	0.498474193737904	0.85935910089082\\
0.885	0.14133524079124	0.512537638903683	0.856028283765737\\
0.885	0.149820561306021	0.526493040810599	0.851449530870183\\
0.885	0.158549418718625	0.540334081158348	0.845707614739506\\
0.885	0.167519594442214	0.554054486817265	0.83891458434315\\
0.885	0.176728657084455	0.567648038782867	0.831205342370836\\
0.885	0.186173963948925	0.581108581114919	0.822732305417529\\
0.885	0.195852662983903	0.594430029835236	0.813659432944309\\
0.885	0.205761695177907	0.607606381758231	0.804155944104873\\
0.885	0.215897797399558	0.620631723228144	0.79439005340697\\
0.885	0.226257505677663	0.633500238737009	0.784523046062801\\
0.885	0.236837158915675	0.646206219397571	0.774703982861588\\
0.885	0.247632903032949	0.658744071245709	0.765065275211017\\
0.885	0.258640695523528	0.671108323347402	0.755719307772962\\
0.885	0.269856310421543	0.683293635685828	0.746756214005909\\
0.885	0.28127534366066	0.695294806804924	0.738242834604297\\
0.885	0.292893218813452	0.707106781186547	0.730222816007834\\
0.885	0.304705193195075	0.71872465633934	0.722717741087768\\
0.885	0.316706364314172	0.730143689578457	0.715729131143269\\
0.885	0.328891676652598	0.741359304476472	0.709241120561855\\
0.885	0.341255928754291	0.752367096967051	0.703223584571265\\
0.885	0.353793780602429	0.763162841084325	0.697635496597419\\
0.885	0.366499761262991	0.773742494322337	0.692428303594618\\
0.885	0.379368276771856	0.784102202600442	0.687549132887069\\
0.885	0.392393618241769	0.794238304822092	0.682943679235328\\
0.885	0.405569970164763	0.804147337016097	0.678558662193044\\
0.885	0.418891418885081	0.813826036051075	0.67434378741287\\
0.885	0.432351961217133	0.823271342915544	0.670253187715552\\
0.885	0.445945513182735	0.832480405557786	0.666246357336159\\
0.885	0.459665918841652	0.841450581281375	0.662288623474186\\
0.885	0.473506959189401	0.850179438693979	0.658351221673191\\
0.885	0.487462361096317	0.85866475920876	0.654411055137886\\
0.885	0.501525806262096	0.866904538099407	0.650450223206467\\
0.885	0.515690940160279	0.874896985111485	0.64645540187496\\
0.885	0.529951380947623	0.882640524634437	0.642417151055155\\
0.885	0.544300728313782	0.890133795440129	0.638329210947586\\
0.885	0.558732572247415	0.897375649995373	0.634187835387499\\
0.885	0.57324050169555	0.904365153357279	0.629991194998375\\
0.885	0.587818113093873	0.911101581661729	0.625738868900923\\
0.885	0.602459018746568	0.917584420216504	0.62143143163314\\
0.885	0.617156855035314	0.923813361211857	0.617070132480552\\
0.885	0.631905290438127	0.92978830106243	0.612656657833836\\
0.885	0.6466980333399	0.935509337395467	0.608192963377089\\
0.885	0.661528839617671	0.940976765701173	0.603681161495108\\
0.885	0.676391519984903	0.946191075661951	0.599123449738606\\
0.885	0.691279947080357	0.951152947177946	0.594522067901588\\
0.885	0.706188062288412	0.955863246106974	0.589879273662389\\
0.885	0.721109882279076	0.960323019737426	0.585197329319935\\
0.885	0.736039505257226	0.964533492013185	0.580478494543975\\
0.885	0.75097111691199	0.968496058529893	0.575725022018315\\
0.885	0.765898996058536	0.972212281322139	0.570939154289561\\
0.885	0.780817519965826	0.975683883461272	0.566123121053174\\
0.885	0.795721169365275	0.978912743483559	0.561279136605559\\
0.885	0.810604533136476	0.981900889668376	0.556409397399199\\
0.885	0.825462312667457	0.984650494185936	0.551516079695736\\
0.885	0.840289325888138	0.987163867133889	0.546601337319643\\
0.885	0.855080510976838	0.989443450481768	0.541667299515141\\
0.885	0.86983092974082	0.991491811941915	0.536716068908598\\
0.885	0.884535770672926	0.993311638785082	0.531749719578143\\
0.885	0.899190351687411	0.994905731618374	0.526770295231793\\
0.885	0.913790122539012	0.996276998142666	0.521779807495102\\
0.885	0.928330666930242	0.997428446906011	0.516780234308686\\
0.885	0.942807704312699	0.998363181068902	0.511773518435873\\
0.885	0.95721709138901	0.999084392196567	0.506761566080168\\
0.885	0.971554823322703	0.999595354092743	0.501746245611875\\
0.885	0.985817034663989	0.999899416688638	0.496729386403004\\
0.885	1	1	0.491669087346897\\
0.9	0	0	0.710195569525497\\
0.9	0.000100583311362513	0.0141829653360114	0.71240460418299\\
0.9	0.000404645907256436	0.0284451766772965	0.714630690536905\\
0.9	0.000915607803432999	0.0427829086109896	0.716916312064106\\
0.9	0.00163681893109844	0.057192295687301	0.719289019360234\\
0.9	0.00257155309398959	0.0716693330697584	0.721780367061016\\
0.9	0.00372300185733414	0.0862098774609879	0.724425655492835\\
0.9	0.00509426838162598	0.100809648312589	0.727263413850304\\
0.9	0.00668836121491815	0.115464229327074	0.73033458997498\\
0.9	0.00850818805808555	0.13016907025918	0.733681422514709\\
0.9	0.0105565495182326	0.144919489023162	0.737345986077549\\
0.9	0.0128361328661109	0.159710674111862	0.741368418705642\\
0.9	0.0153495058140643	0.174537687332543	0.745784862958201\\
0.9	0.0180991103316243	0.189395466863524	0.750625176092311\\
0.9	0.0210872565164405	0.204278830634725	0.755910489868999\\
0.9	0.0243161165387281	0.219182480034174	0.761650724664453\\
0.9	0.0277877186778607	0.234101003941464	0.767842183867616\\
0.9	0.0315039414701067	0.24902888308801	0.77446537092041\\
0.9	0.0354665079868145	0.263960494742775	0.781483180780606\\
0.9	0.0396769802625738	0.278890117720924	0.788839618268073\\
0.9	0.0441367538930258	0.293811937711588	0.796459186325544\\
0.9	0.0488470528220538	0.308720052919643	0.804247066923456\\
0.9	0.0538089243380495	0.323608480015097	0.81209018616195\\
0.9	0.0590232342988274	0.338471160382329	0.819859213932694\\
0.9	0.064490662604533	0.3533019666601	0.827411499063639\\
0.9	0.0702116989375697	0.368094709561873	0.834594885813505\\
0.9	0.0761866387881432	0.382843144964686	0.84125230028718\\
0.9	0.0824155797834956	0.397540981253432	0.847226939729081\\
0.9	0.0888984183382709	0.412181886906127	0.852367847905039\\
0.9	0.0956348466427212	0.42675949830445	0.856535620027829\\
0.9	0.102624350004627	0.441267427752584	0.859607954619618\\
0.9	0.109866204559871	0.455699271686218	0.86148476026662\\
0.9	0.117359475365564	0.470048619052377	0.862092534251538\\
0.9	0.125103014888515	0.484309059839721	0.861387758068367\\
0.9	0.133095461900593	0.498474193737904	0.85935910089082\\
0.9	0.14133524079124	0.512537638903683	0.856028283765734\\
0.9	0.149820561306021	0.526493040810599	0.851449530870186\\
0.9	0.158549418718625	0.540334081158348	0.845707614739506\\
0.9	0.167519594442214	0.554054486817266	0.838914584343152\\
0.9	0.176728657084455	0.567648038782867	0.831205342370837\\
0.9	0.186173963948925	0.581108581114919	0.822732305417528\\
0.9	0.195852662983903	0.594430029835237	0.813659432944309\\
0.9	0.205761695177907	0.607606381758231	0.804155944104867\\
0.9	0.215897797399558	0.620631723228143	0.794390053406967\\
0.9	0.226257505677663	0.633500238737009	0.784523046062801\\
0.9	0.236837158915675	0.646206219397571	0.774703982861588\\
0.9	0.247632903032949	0.658744071245709	0.765065275211016\\
0.9	0.258640695523527	0.671108323347402	0.755719307772962\\
0.9	0.269856310421543	0.683293635685828	0.746756214005913\\
0.9	0.28127534366066	0.695294806804925	0.738242834604297\\
0.9	0.292893218813452	0.707106781186547	0.730222816007835\\
0.9	0.304705193195075	0.71872465633934	0.72271774108777\\
0.9	0.316706364314172	0.730143689578457	0.715729131143267\\
0.9	0.328891676652598	0.741359304476472	0.709241120561854\\
0.9	0.341255928754291	0.752367096967051	0.703223584571264\\
0.9	0.353793780602429	0.763162841084325	0.69763549659742\\
0.9	0.366499761262991	0.773742494322337	0.692428303594619\\
0.9	0.379368276771857	0.784102202600443	0.687549132887069\\
0.9	0.392393618241769	0.794238304822092	0.682943679235329\\
0.9	0.405569970164763	0.804147337016097	0.678558662193043\\
0.9	0.418891418885081	0.813826036051075	0.674343787412871\\
0.9	0.432351961217132	0.823271342915545	0.670253187715552\\
0.9	0.445945513182735	0.832480405557786	0.666246357336158\\
0.9	0.459665918841652	0.841450581281375	0.662288623474187\\
0.9	0.473506959189401	0.850179438693979	0.658351221673191\\
0.9	0.487462361096317	0.85866475920876	0.654411055137885\\
0.9	0.501525806262096	0.866904538099407	0.650450223206468\\
0.9	0.515690940160279	0.874896985111485	0.646455401874958\\
0.9	0.529951380947623	0.882640524634436	0.642417151055155\\
0.9	0.544300728313782	0.890133795440129	0.638329210947586\\
0.9	0.558732572247415	0.897375649995373	0.634187835387501\\
0.9	0.57324050169555	0.904365153357279	0.629991194998376\\
0.9	0.587818113093873	0.911101581661729	0.625738868900922\\
0.9	0.602459018746568	0.917584420216504	0.62143143163314\\
0.9	0.617156855035314	0.923813361211857	0.617070132480553\\
0.9	0.631905290438127	0.92978830106243	0.612656657833838\\
0.9	0.6466980333399	0.935509337395467	0.608192963377087\\
0.9	0.661528839617671	0.940976765701173	0.603681161495105\\
0.9	0.676391519984903	0.946191075661951	0.599123449738609\\
0.9	0.691279947080357	0.951152947177946	0.594522067901589\\
0.9	0.706188062288412	0.955863246106974	0.589879273662389\\
0.9	0.721109882279076	0.960323019737426	0.585197329319937\\
0.9	0.736039505257226	0.964533492013186	0.580478494543976\\
0.9	0.75097111691199	0.968496058529893	0.575725022018311\\
0.9	0.765898996058536	0.972212281322139	0.570939154289557\\
0.9	0.780817519965825	0.975683883461272	0.566123121053177\\
0.9	0.795721169365275	0.97891274348356	0.561279136605559\\
0.9	0.810604533136476	0.981900889668376	0.556409397399197\\
0.9	0.825462312667457	0.984650494185936	0.551516079695733\\
0.9	0.840289325888138	0.987163867133889	0.546601337319642\\
0.9	0.855080510976839	0.989443450481767	0.541667299515141\\
0.9	0.86983092974082	0.991491811941915	0.536716068908599\\
0.9	0.884535770672926	0.993311638785082	0.531749719578141\\
0.9	0.899190351687411	0.994905731618374	0.526770295231793\\
0.9	0.913790122539012	0.996276998142666	0.521779807495101\\
0.9	0.928330666930241	0.99742844690601	0.516780234308683\\
0.9	0.942807704312699	0.998363181068901	0.511773518435874\\
0.9	0.95721709138901	0.999084392196567	0.506761566080172\\
0.9	0.971554823322704	0.999595354092744	0.501746245611874\\
0.9	0.985817034663989	0.999899416688637	0.496729386403002\\
0.9	1	1	0.491669087346908\\
0.915	0	0	0.710195569525497\\
0.915	0.000100583311362513	0.0141829653360114	0.71240460418299\\
0.915	0.000404645907256436	0.0284451766772965	0.714630690536905\\
0.915	0.000915607803432999	0.0427829086109896	0.716916312064106\\
0.915	0.00163681893109844	0.057192295687301	0.719289019360234\\
0.915	0.00257155309398959	0.0716693330697584	0.721780367061016\\
0.915	0.00372300185733413	0.0862098774609879	0.724425655492835\\
0.915	0.00509426838162598	0.100809648312589	0.727263413850304\\
0.915	0.00668836121491816	0.115464229327074	0.73033458997498\\
0.915	0.00850818805808555	0.13016907025918	0.733681422514709\\
0.915	0.0105565495182326	0.144919489023162	0.737345986077549\\
0.915	0.0128361328661109	0.159710674111862	0.741368418705642\\
0.915	0.0153495058140643	0.174537687332543	0.745784862958201\\
0.915	0.0180991103316243	0.189395466863524	0.750625176092311\\
0.915	0.0210872565164405	0.204278830634725	0.755910489868999\\
0.915	0.0243161165387281	0.219182480034174	0.761650724664453\\
0.915	0.0277877186778607	0.234101003941464	0.767842183867616\\
0.915	0.0315039414701067	0.24902888308801	0.77446537092041\\
0.915	0.0354665079868145	0.263960494742775	0.781483180780606\\
0.915	0.0396769802625738	0.278890117720924	0.788839618268073\\
0.915	0.0441367538930258	0.293811937711588	0.796459186325543\\
0.915	0.0488470528220538	0.308720052919643	0.804247066923455\\
0.915	0.0538089243380495	0.323608480015096	0.812090186161951\\
0.915	0.0590232342988274	0.338471160382329	0.819859213932694\\
0.915	0.064490662604533	0.3533019666601	0.827411499063639\\
0.915	0.0702116989375697	0.368094709561873	0.834594885813504\\
0.915	0.0761866387881432	0.382843144964686	0.84125230028718\\
0.915	0.0824155797834956	0.397540981253432	0.847226939729081\\
0.915	0.0888984183382709	0.412181886906127	0.852367847905041\\
0.915	0.0956348466427212	0.42675949830445	0.856535620027826\\
0.915	0.102624350004627	0.441267427752585	0.859607954619618\\
0.915	0.109866204559871	0.455699271686218	0.861484760266623\\
0.915	0.117359475365564	0.470048619052377	0.862092534251539\\
0.915	0.125103014888515	0.484309059839721	0.861387758068368\\
0.915	0.133095461900593	0.498474193737904	0.859359100890818\\
0.915	0.14133524079124	0.512537638903683	0.856028283765736\\
0.915	0.149820561306021	0.526493040810599	0.851449530870181\\
0.915	0.158549418718625	0.540334081158348	0.845707614739503\\
0.915	0.167519594442214	0.554054486817266	0.83891458434315\\
0.915	0.176728657084455	0.567648038782867	0.831205342370836\\
0.915	0.186173963948925	0.581108581114919	0.822732305417525\\
0.915	0.195852662983903	0.594430029835237	0.813659432944304\\
0.915	0.205761695177907	0.607606381758231	0.804155944104867\\
0.915	0.215897797399558	0.620631723228143	0.79439005340697\\
0.915	0.226257505677663	0.633500238737009	0.784523046062801\\
0.915	0.236837158915675	0.646206219397571	0.774703982861589\\
0.915	0.247632903032949	0.658744071245709	0.765065275211016\\
0.915	0.258640695523528	0.671108323347402	0.755719307772963\\
0.915	0.269856310421543	0.683293635685828	0.746756214005912\\
0.915	0.28127534366066	0.695294806804924	0.738242834604296\\
0.915	0.292893218813452	0.707106781186547	0.730222816007837\\
0.915	0.304705193195075	0.71872465633934	0.722717741087766\\
0.915	0.316706364314172	0.730143689578457	0.715729131143268\\
0.915	0.328891676652598	0.741359304476472	0.709241120561856\\
0.915	0.341255928754291	0.752367096967051	0.703223584571264\\
0.915	0.353793780602429	0.763162841084325	0.697635496597419\\
0.915	0.366499761262991	0.773742494322337	0.692428303594618\\
0.915	0.379368276771857	0.784102202600442	0.687549132887069\\
0.915	0.392393618241769	0.794238304822092	0.682943679235329\\
0.915	0.405569970164763	0.804147337016097	0.678558662193044\\
0.915	0.418891418885081	0.813826036051075	0.67434378741287\\
0.915	0.432351961217132	0.823271342915544	0.670253187715551\\
0.915	0.445945513182734	0.832480405557786	0.666246357336159\\
0.915	0.459665918841652	0.841450581281375	0.662288623474187\\
0.915	0.473506959189401	0.850179438693979	0.658351221673191\\
0.915	0.487462361096317	0.85866475920876	0.654411055137886\\
0.915	0.501525806262096	0.866904538099407	0.650450223206469\\
0.915	0.515690940160279	0.874896985111485	0.646455401874957\\
0.915	0.529951380947623	0.882640524634436	0.642417151055154\\
0.915	0.544300728313782	0.890133795440129	0.638329210947586\\
0.915	0.558732572247415	0.897375649995372	0.634187835387497\\
0.915	0.57324050169555	0.904365153357279	0.629991194998379\\
0.915	0.587818113093873	0.911101581661729	0.625738868900924\\
0.915	0.602459018746568	0.917584420216504	0.621431431633139\\
0.915	0.617156855035314	0.923813361211857	0.617070132480553\\
0.915	0.631905290438127	0.92978830106243	0.612656657833838\\
0.915	0.6466980333399	0.935509337395467	0.608192963377088\\
0.915	0.661528839617671	0.940976765701173	0.603681161495104\\
0.915	0.676391519984904	0.94619107566195	0.599123449738605\\
0.915	0.691279947080357	0.951152947177946	0.59452206790159\\
0.915	0.706188062288412	0.955863246106974	0.589879273662392\\
0.915	0.721109882279076	0.960323019737426	0.585197329319937\\
0.915	0.736039505257226	0.964533492013186	0.580478494543978\\
0.915	0.75097111691199	0.968496058529893	0.575725022018312\\
0.915	0.765898996058536	0.972212281322139	0.570939154289556\\
0.915	0.780817519965826	0.975683883461272	0.566123121053178\\
0.915	0.795721169365275	0.97891274348356	0.56127913660556\\
0.915	0.810604533136476	0.981900889668376	0.556409397399197\\
0.915	0.825462312667457	0.984650494185936	0.551516079695735\\
0.915	0.840289325888138	0.987163867133889	0.546601337319641\\
0.915	0.855080510976839	0.989443450481767	0.54166729951514\\
0.915	0.86983092974082	0.991491811941915	0.5367160689086\\
0.915	0.884535770672926	0.993311638785082	0.531749719578141\\
0.915	0.899190351687411	0.994905731618374	0.526770295231793\\
0.915	0.913790122539012	0.996276998142666	0.521779807495103\\
0.915	0.928330666930242	0.997428446906011	0.516780234308683\\
0.915	0.942807704312699	0.998363181068902	0.511773518435872\\
0.915	0.95721709138901	0.999084392196567	0.50676156608017\\
0.915	0.971554823322703	0.999595354092743	0.501746245611875\\
0.915	0.985817034663988	0.999899416688637	0.496729386403004\\
0.915	1	1	0.491669087346906\\
0.93	0	0	0.710195569525497\\
0.93	0.000100583311362513	0.0141829653360114	0.71240460418299\\
0.93	0.000404645907256436	0.0284451766772965	0.714630690536905\\
0.93	0.000915607803432999	0.0427829086109896	0.716916312064106\\
0.93	0.00163681893109844	0.057192295687301	0.719289019360234\\
0.93	0.00257155309398959	0.0716693330697584	0.721780367061016\\
0.93	0.00372300185733413	0.0862098774609879	0.724425655492835\\
0.93	0.00509426838162598	0.100809648312589	0.727263413850304\\
0.93	0.00668836121491816	0.115464229327074	0.73033458997498\\
0.93	0.00850818805808555	0.13016907025918	0.733681422514709\\
0.93	0.0105565495182326	0.144919489023162	0.737345986077549\\
0.93	0.0128361328661109	0.159710674111862	0.741368418705642\\
0.93	0.0153495058140643	0.174537687332543	0.745784862958201\\
0.93	0.0180991103316243	0.189395466863524	0.750625176092311\\
0.93	0.0210872565164405	0.204278830634725	0.755910489868999\\
0.93	0.0243161165387281	0.219182480034174	0.761650724664453\\
0.93	0.0277877186778607	0.234101003941464	0.767842183867616\\
0.93	0.0315039414701067	0.24902888308801	0.77446537092041\\
0.93	0.0354665079868145	0.263960494742775	0.781483180780606\\
0.93	0.0396769802625738	0.278890117720924	0.788839618268073\\
0.93	0.0441367538930258	0.293811937711588	0.796459186325544\\
0.93	0.0488470528220538	0.308720052919643	0.804247066923455\\
0.93	0.0538089243380495	0.323608480015097	0.81209018616195\\
0.93	0.0590232342988274	0.338471160382329	0.819859213932694\\
0.93	0.064490662604533	0.3533019666601	0.827411499063638\\
0.93	0.0702116989375697	0.368094709561873	0.834594885813505\\
0.93	0.0761866387881432	0.382843144964686	0.841252300287179\\
0.93	0.0824155797834956	0.397540981253432	0.847226939729081\\
0.93	0.0888984183382709	0.412181886906127	0.852367847905042\\
0.93	0.0956348466427212	0.42675949830445	0.856535620027829\\
0.93	0.102624350004627	0.441267427752585	0.859607954619617\\
0.93	0.109866204559871	0.455699271686218	0.861484760266624\\
0.93	0.117359475365564	0.470048619052377	0.86209253425154\\
0.93	0.125103014888515	0.484309059839721	0.861387758068368\\
0.93	0.133095461900593	0.498474193737904	0.859359100890821\\
0.93	0.14133524079124	0.512537638903683	0.856028283765738\\
0.93	0.149820561306021	0.526493040810599	0.851449530870182\\
0.93	0.158549418718625	0.540334081158348	0.845707614739506\\
0.93	0.167519594442213	0.554054486817266	0.838914584343151\\
0.93	0.176728657084455	0.567648038782867	0.831205342370836\\
0.93	0.186173963948925	0.581108581114919	0.822732305417526\\
0.93	0.195852662983903	0.594430029835236	0.813659432944307\\
0.93	0.205761695177907	0.607606381758231	0.80415594410487\\
0.93	0.215897797399558	0.620631723228143	0.794390053406971\\
0.93	0.226257505677663	0.633500238737009	0.784523046062802\\
0.93	0.236837158915675	0.646206219397571	0.774703982861589\\
0.93	0.247632903032949	0.658744071245709	0.765065275211015\\
0.93	0.258640695523527	0.671108323347402	0.755719307772963\\
0.93	0.269856310421543	0.683293635685828	0.746756214005913\\
0.93	0.28127534366066	0.695294806804925	0.738242834604296\\
0.93	0.292893218813452	0.707106781186548	0.730222816007832\\
0.93	0.304705193195076	0.71872465633934	0.722717741087766\\
0.93	0.316706364314172	0.730143689578457	0.715729131143269\\
0.93	0.328891676652598	0.741359304476472	0.709241120561855\\
0.93	0.341255928754291	0.752367096967051	0.703223584571265\\
0.93	0.353793780602429	0.763162841084325	0.69763549659742\\
0.93	0.366499761262991	0.773742494322337	0.692428303594618\\
0.93	0.379368276771857	0.784102202600443	0.687549132887068\\
0.93	0.392393618241769	0.794238304822092	0.682943679235329\\
0.93	0.405569970164763	0.804147337016097	0.678558662193045\\
0.93	0.418891418885081	0.813826036051075	0.674343787412871\\
0.93	0.432351961217133	0.823271342915545	0.670253187715552\\
0.93	0.445945513182735	0.832480405557786	0.666246357336158\\
0.93	0.459665918841652	0.841450581281375	0.662288623474186\\
0.93	0.473506959189401	0.850179438693979	0.658351221673191\\
0.93	0.487462361096317	0.85866475920876	0.654411055137886\\
0.93	0.501525806262096	0.866904538099407	0.650450223206468\\
0.93	0.515690940160279	0.874896985111485	0.646455401874959\\
0.93	0.529951380947623	0.882640524634436	0.642417151055155\\
0.93	0.544300728313782	0.890133795440129	0.638329210947586\\
0.93	0.558732572247415	0.897375649995373	0.634187835387497\\
0.93	0.57324050169555	0.904365153357279	0.629991194998377\\
0.93	0.587818113093873	0.911101581661729	0.625738868900925\\
0.93	0.602459018746568	0.917584420216504	0.62143143163314\\
0.93	0.617156855035314	0.923813361211857	0.617070132480554\\
0.93	0.631905290438127	0.92978830106243	0.612656657833837\\
0.93	0.6466980333399	0.935509337395467	0.608192963377087\\
0.93	0.661528839617671	0.940976765701173	0.603681161495103\\
0.93	0.676391519984904	0.94619107566195	0.599123449738607\\
0.93	0.691279947080357	0.951152947177946	0.59452206790159\\
0.93	0.706188062288412	0.955863246106974	0.589879273662389\\
0.93	0.721109882279076	0.960323019737426	0.585197329319937\\
0.93	0.736039505257226	0.964533492013186	0.580478494543978\\
0.93	0.75097111691199	0.968496058529893	0.575725022018312\\
0.93	0.765898996058536	0.972212281322139	0.570939154289557\\
0.93	0.780817519965826	0.975683883461272	0.566123121053175\\
0.93	0.795721169365275	0.97891274348356	0.561279136605559\\
0.93	0.810604533136476	0.981900889668376	0.5564093973992\\
0.93	0.825462312667457	0.984650494185936	0.551516079695735\\
0.93	0.840289325888138	0.987163867133889	0.546601337319639\\
0.93	0.855080510976839	0.989443450481767	0.541667299515139\\
0.93	0.86983092974082	0.991491811941914	0.536716068908599\\
0.93	0.884535770672926	0.993311638785082	0.531749719578142\\
0.93	0.899190351687411	0.994905731618374	0.526770295231793\\
0.93	0.913790122539012	0.996276998142666	0.521779807495102\\
0.93	0.928330666930242	0.997428446906011	0.516780234308683\\
0.93	0.942807704312699	0.998363181068902	0.51177351843587\\
0.93	0.95721709138901	0.999084392196567	0.50676156608017\\
0.93	0.971554823322703	0.999595354092743	0.501746245611876\\
0.93	0.985817034663989	0.999899416688637	0.496729386403004\\
0.93	1	1	0.491669087346906\\
0.945	0	0	0.710195569525497\\
0.945	0.000100583311362513	0.0141829653360114	0.71240460418299\\
0.945	0.000404645907256436	0.0284451766772965	0.714630690536905\\
0.945	0.000915607803432999	0.0427829086109896	0.716916312064106\\
0.945	0.00163681893109844	0.057192295687301	0.719289019360234\\
0.945	0.00257155309398959	0.0716693330697584	0.721780367061016\\
0.945	0.00372300185733413	0.0862098774609879	0.724425655492835\\
0.945	0.00509426838162598	0.100809648312589	0.727263413850304\\
0.945	0.00668836121491816	0.115464229327074	0.73033458997498\\
0.945	0.00850818805808555	0.13016907025918	0.733681422514709\\
0.945	0.0105565495182326	0.144919489023162	0.737345986077549\\
0.945	0.0128361328661109	0.159710674111862	0.741368418705642\\
0.945	0.0153495058140643	0.174537687332543	0.745784862958201\\
0.945	0.0180991103316243	0.189395466863524	0.750625176092311\\
0.945	0.0210872565164405	0.204278830634725	0.755910489868999\\
0.945	0.0243161165387281	0.219182480034174	0.761650724664453\\
0.945	0.0277877186778607	0.234101003941464	0.767842183867616\\
0.945	0.0315039414701067	0.24902888308801	0.77446537092041\\
0.945	0.0354665079868145	0.263960494742775	0.781483180780606\\
0.945	0.0396769802625737	0.278890117720924	0.788839618268073\\
0.945	0.0441367538930258	0.293811937711588	0.796459186325544\\
0.945	0.0488470528220538	0.308720052919643	0.804247066923455\\
0.945	0.0538089243380495	0.323608480015096	0.812090186161951\\
0.945	0.0590232342988274	0.338471160382329	0.819859213932693\\
0.945	0.064490662604533	0.3533019666601	0.827411499063639\\
0.945	0.0702116989375697	0.368094709561873	0.834594885813506\\
0.945	0.0761866387881432	0.382843144964686	0.84125230028718\\
0.945	0.0824155797834956	0.397540981253432	0.847226939729079\\
0.945	0.0888984183382709	0.412181886906127	0.852367847905042\\
0.945	0.0956348466427212	0.42675949830445	0.856535620027831\\
0.945	0.102624350004627	0.441267427752584	0.859607954619618\\
0.945	0.109866204559871	0.455699271686218	0.861484760266621\\
0.945	0.117359475365564	0.470048619052377	0.862092534251541\\
0.945	0.125103014888515	0.484309059839721	0.861387758068369\\
0.945	0.133095461900593	0.498474193737904	0.85935910089082\\
0.945	0.14133524079124	0.512537638903683	0.856028283765735\\
0.945	0.149820561306021	0.526493040810599	0.851449530870183\\
0.945	0.158549418718625	0.540334081158348	0.845707614739505\\
0.945	0.167519594442214	0.554054486817266	0.838914584343148\\
0.945	0.176728657084455	0.567648038782867	0.831205342370836\\
0.945	0.186173963948925	0.581108581114919	0.82273230541753\\
0.945	0.195852662983903	0.594430029835236	0.813659432944306\\
0.945	0.205761695177907	0.607606381758231	0.804155944104866\\
0.945	0.215897797399558	0.620631723228143	0.794390053406971\\
0.945	0.226257505677663	0.633500238737009	0.784523046062802\\
0.945	0.236837158915675	0.646206219397571	0.774703982861588\\
0.945	0.247632903032949	0.658744071245709	0.765065275211015\\
0.945	0.258640695523528	0.671108323347402	0.755719307772962\\
0.945	0.269856310421543	0.683293635685828	0.746756214005909\\
0.945	0.28127534366066	0.695294806804924	0.738242834604295\\
0.945	0.292893218813452	0.707106781186547	0.730222816007835\\
0.945	0.304705193195075	0.71872465633934	0.722717741087767\\
0.945	0.316706364314172	0.730143689578457	0.715729131143268\\
0.945	0.328891676652598	0.741359304476472	0.709241120561856\\
0.945	0.341255928754291	0.752367096967051	0.703223584571265\\
0.945	0.353793780602429	0.763162841084325	0.697635496597419\\
0.945	0.366499761262991	0.773742494322337	0.692428303594619\\
0.945	0.379368276771856	0.784102202600442	0.687549132887069\\
0.945	0.392393618241769	0.794238304822092	0.682943679235329\\
0.945	0.405569970164763	0.804147337016097	0.678558662193046\\
0.945	0.418891418885081	0.813826036051075	0.67434378741287\\
0.945	0.432351961217133	0.823271342915544	0.67025318771555\\
0.945	0.445945513182735	0.832480405557786	0.666246357336159\\
0.945	0.459665918841652	0.841450581281375	0.662288623474187\\
0.945	0.473506959189401	0.850179438693979	0.658351221673192\\
0.945	0.487462361096317	0.85866475920876	0.654411055137886\\
0.945	0.501525806262096	0.866904538099407	0.650450223206468\\
0.945	0.515690940160279	0.874896985111485	0.646455401874958\\
0.945	0.529951380947623	0.882640524634436	0.642417151055155\\
0.945	0.544300728313782	0.890133795440129	0.638329210947587\\
0.945	0.558732572247415	0.897375649995373	0.634187835387497\\
0.945	0.57324050169555	0.904365153357279	0.629991194998375\\
0.945	0.587818113093873	0.911101581661729	0.625738868900926\\
0.945	0.602459018746568	0.917584420216504	0.62143143163314\\
0.945	0.617156855035314	0.923813361211857	0.617070132480554\\
0.945	0.631905290438127	0.92978830106243	0.612656657833838\\
0.945	0.6466980333399	0.935509337395467	0.608192963377087\\
0.945	0.661528839617671	0.940976765701173	0.603681161495104\\
0.945	0.676391519984904	0.94619107566195	0.599123449738607\\
0.945	0.691279947080357	0.951152947177946	0.59452206790159\\
0.945	0.706188062288412	0.955863246106974	0.589879273662389\\
0.945	0.721109882279076	0.960323019737426	0.585197329319936\\
0.945	0.736039505257226	0.964533492013186	0.580478494543978\\
0.945	0.75097111691199	0.968496058529893	0.575725022018314\\
0.945	0.765898996058536	0.972212281322139	0.570939154289559\\
0.945	0.780817519965826	0.975683883461272	0.566123121053175\\
0.945	0.795721169365275	0.97891274348356	0.561279136605558\\
0.945	0.810604533136476	0.981900889668376	0.556409397399199\\
0.945	0.825462312667457	0.984650494185936	0.551516079695735\\
0.945	0.840289325888138	0.987163867133889	0.546601337319639\\
0.945	0.855080510976838	0.989443450481768	0.541667299515141\\
0.945	0.86983092974082	0.991491811941915	0.5367160689086\\
0.945	0.884535770672926	0.993311638785082	0.531749719578141\\
0.945	0.899190351687411	0.994905731618374	0.526770295231793\\
0.945	0.913790122539012	0.996276998142666	0.521779807495103\\
0.945	0.928330666930242	0.997428446906011	0.516780234308682\\
0.945	0.942807704312699	0.998363181068901	0.511773518435872\\
0.945	0.95721709138901	0.999084392196567	0.506761566080169\\
0.945	0.971554823322703	0.999595354092743	0.501746245611874\\
0.945	0.985817034663988	0.999899416688637	0.496729386403004\\
0.945	1	1	0.491669087346906\\
0.96	0	0	0.710195569525497\\
0.96	0.000100583311362513	0.0141829653360114	0.71240460418299\\
0.96	0.000404645907256436	0.0284451766772965	0.714630690536905\\
0.96	0.000915607803432999	0.0427829086109896	0.716916312064106\\
0.96	0.00163681893109844	0.057192295687301	0.719289019360234\\
0.96	0.00257155309398959	0.0716693330697584	0.721780367061016\\
0.96	0.00372300185733414	0.0862098774609879	0.724425655492835\\
0.96	0.00509426838162598	0.100809648312589	0.727263413850304\\
0.96	0.00668836121491816	0.115464229327074	0.73033458997498\\
0.96	0.00850818805808555	0.13016907025918	0.733681422514709\\
0.96	0.0105565495182326	0.144919489023162	0.737345986077549\\
0.96	0.0128361328661109	0.159710674111862	0.741368418705642\\
0.96	0.0153495058140643	0.174537687332543	0.745784862958202\\
0.96	0.0180991103316243	0.189395466863524	0.750625176092311\\
0.96	0.0210872565164405	0.204278830634725	0.755910489868999\\
0.96	0.0243161165387281	0.219182480034174	0.761650724664453\\
0.96	0.0277877186778607	0.234101003941464	0.767842183867616\\
0.96	0.0315039414701067	0.24902888308801	0.77446537092041\\
0.96	0.0354665079868145	0.263960494742775	0.781483180780606\\
0.96	0.0396769802625738	0.278890117720924	0.788839618268073\\
0.96	0.0441367538930258	0.293811937711588	0.796459186325544\\
0.96	0.0488470528220537	0.308720052919643	0.804247066923455\\
0.96	0.0538089243380495	0.323608480015096	0.81209018616195\\
0.96	0.0590232342988274	0.338471160382329	0.819859213932694\\
0.96	0.064490662604533	0.3533019666601	0.827411499063639\\
0.96	0.0702116989375697	0.368094709561873	0.834594885813504\\
0.96	0.0761866387881432	0.382843144964686	0.841252300287181\\
0.96	0.0824155797834956	0.397540981253432	0.847226939729081\\
0.96	0.0888984183382709	0.412181886906127	0.85236784790504\\
0.96	0.0956348466427212	0.42675949830445	0.856535620027829\\
0.96	0.102624350004627	0.441267427752584	0.859607954619616\\
0.96	0.109866204559871	0.455699271686218	0.861484760266621\\
0.96	0.117359475365564	0.470048619052377	0.86209253425154\\
0.96	0.125103014888515	0.484309059839721	0.861387758068371\\
0.96	0.133095461900593	0.498474193737904	0.85935910089082\\
0.96	0.14133524079124	0.512537638903683	0.856028283765736\\
0.96	0.149820561306021	0.526493040810599	0.851449530870181\\
0.96	0.158549418718625	0.540334081158348	0.845707614739503\\
0.96	0.167519594442214	0.554054486817265	0.838914584343147\\
0.96	0.176728657084455	0.567648038782867	0.831205342370837\\
0.96	0.186173963948925	0.581108581114919	0.822732305417527\\
0.96	0.195852662983903	0.594430029835236	0.813659432944306\\
0.96	0.205761695177907	0.60760638175823	0.80415594410487\\
0.96	0.215897797399558	0.620631723228143	0.794390053406972\\
0.96	0.226257505677663	0.633500238737009	0.784523046062801\\
0.96	0.236837158915675	0.646206219397571	0.774703982861586\\
0.96	0.247632903032949	0.658744071245709	0.765065275211014\\
0.96	0.258640695523527	0.671108323347402	0.755719307772963\\
0.96	0.269856310421543	0.683293635685828	0.746756214005913\\
0.96	0.28127534366066	0.695294806804924	0.738242834604297\\
0.96	0.292893218813452	0.707106781186548	0.730222816007834\\
0.96	0.304705193195075	0.71872465633934	0.722717741087765\\
0.96	0.316706364314172	0.730143689578457	0.715729131143269\\
0.96	0.328891676652598	0.741359304476472	0.709241120561856\\
0.96	0.341255928754291	0.752367096967051	0.703223584571266\\
0.96	0.353793780602429	0.763162841084325	0.697635496597419\\
0.96	0.366499761262991	0.773742494322337	0.692428303594618\\
0.96	0.379368276771857	0.784102202600443	0.687549132887069\\
0.96	0.392393618241769	0.794238304822092	0.682943679235329\\
0.96	0.405569970164763	0.804147337016097	0.678558662193045\\
0.96	0.418891418885081	0.813826036051075	0.674343787412871\\
0.96	0.432351961217133	0.823271342915545	0.670253187715551\\
0.96	0.445945513182735	0.832480405557786	0.666246357336157\\
0.96	0.459665918841652	0.841450581281375	0.662288623474186\\
0.96	0.473506959189401	0.850179438693979	0.658351221673191\\
0.96	0.487462361096317	0.85866475920876	0.654411055137887\\
0.96	0.501525806262096	0.866904538099407	0.650450223206467\\
0.96	0.515690940160279	0.874896985111485	0.646455401874958\\
0.96	0.529951380947623	0.882640524634436	0.642417151055155\\
0.96	0.544300728313782	0.890133795440129	0.638329210947586\\
0.96	0.558732572247415	0.897375649995373	0.634187835387499\\
0.96	0.57324050169555	0.904365153357279	0.629991194998376\\
0.96	0.587818113093873	0.911101581661729	0.625738868900922\\
0.96	0.602459018746568	0.917584420216504	0.62143143163314\\
0.96	0.617156855035314	0.923813361211857	0.617070132480553\\
0.96	0.631905290438127	0.92978830106243	0.612656657833838\\
0.96	0.6466980333399	0.935509337395467	0.608192963377088\\
0.96	0.661528839617671	0.940976765701173	0.603681161495105\\
0.96	0.676391519984903	0.94619107566195	0.599123449738607\\
0.96	0.691279947080357	0.951152947177946	0.594522067901588\\
0.96	0.706188062288412	0.955863246106974	0.58987927366239\\
0.96	0.721109882279076	0.960323019737426	0.585197329319938\\
0.96	0.736039505257226	0.964533492013186	0.580478494543976\\
0.96	0.75097111691199	0.968496058529893	0.575725022018313\\
0.96	0.765898996058536	0.972212281322139	0.570939154289558\\
0.96	0.780817519965826	0.975683883461272	0.566123121053175\\
0.96	0.795721169365275	0.97891274348356	0.561279136605561\\
0.96	0.810604533136476	0.981900889668376	0.556409397399198\\
0.96	0.825462312667457	0.984650494185936	0.551516079695736\\
0.96	0.840289325888138	0.987163867133889	0.546601337319641\\
0.96	0.855080510976839	0.989443450481767	0.541667299515138\\
0.96	0.86983092974082	0.991491811941914	0.536716068908602\\
0.96	0.884535770672926	0.993311638785082	0.531749719578143\\
0.96	0.899190351687411	0.994905731618374	0.526770295231793\\
0.96	0.913790122539012	0.996276998142666	0.521779807495103\\
0.96	0.928330666930241	0.997428446906011	0.516780234308683\\
0.96	0.942807704312699	0.998363181068902	0.511773518435872\\
0.96	0.95721709138901	0.999084392196567	0.50676156608017\\
0.96	0.971554823322703	0.999595354092743	0.501746245611875\\
0.96	0.985817034663989	0.999899416688637	0.496729386403004\\
0.96	1	1	0.491669087346906\\
0.975	0	0	0.710195569525497\\
0.975	0.000100583311362513	0.0141829653360114	0.71240460418299\\
0.975	0.000404645907256436	0.0284451766772965	0.714630690536905\\
0.975	0.000915607803432999	0.0427829086109896	0.716916312064106\\
0.975	0.00163681893109844	0.057192295687301	0.719289019360234\\
0.975	0.00257155309398959	0.0716693330697584	0.721780367061016\\
0.975	0.00372300185733413	0.0862098774609879	0.724425655492835\\
0.975	0.00509426838162598	0.100809648312589	0.727263413850304\\
0.975	0.00668836121491815	0.115464229327074	0.73033458997498\\
0.975	0.00850818805808554	0.13016907025918	0.733681422514709\\
0.975	0.0105565495182326	0.144919489023162	0.737345986077549\\
0.975	0.0128361328661109	0.159710674111862	0.741368418705642\\
0.975	0.0153495058140643	0.174537687332543	0.745784862958201\\
0.975	0.0180991103316243	0.189395466863524	0.750625176092311\\
0.975	0.0210872565164405	0.204278830634725	0.755910489868999\\
0.975	0.0243161165387281	0.219182480034174	0.761650724664453\\
0.975	0.0277877186778607	0.234101003941464	0.767842183867615\\
0.975	0.0315039414701067	0.24902888308801	0.77446537092041\\
0.975	0.0354665079868145	0.263960494742775	0.781483180780606\\
0.975	0.0396769802625738	0.278890117720924	0.788839618268073\\
0.975	0.0441367538930258	0.293811937711588	0.796459186325544\\
0.975	0.0488470528220538	0.308720052919643	0.804247066923454\\
0.975	0.0538089243380495	0.323608480015096	0.81209018616195\\
0.975	0.0590232342988274	0.338471160382329	0.819859213932693\\
0.975	0.064490662604533	0.3533019666601	0.827411499063639\\
0.975	0.0702116989375697	0.368094709561873	0.834594885813504\\
0.975	0.0761866387881432	0.382843144964686	0.84125230028718\\
0.975	0.0824155797834956	0.397540981253432	0.847226939729083\\
0.975	0.0888984183382709	0.412181886906127	0.85236784790504\\
0.975	0.0956348466427212	0.42675949830445	0.856535620027828\\
0.975	0.102624350004627	0.441267427752584	0.859607954619618\\
0.975	0.109866204559871	0.455699271686218	0.861484760266622\\
0.975	0.117359475365564	0.470048619052377	0.862092534251541\\
0.975	0.125103014888515	0.484309059839721	0.861387758068369\\
0.975	0.133095461900593	0.498474193737904	0.85935910089082\\
0.975	0.14133524079124	0.512537638903683	0.856028283765736\\
0.975	0.149820561306021	0.526493040810599	0.851449530870182\\
0.975	0.158549418718625	0.540334081158348	0.845707614739505\\
0.975	0.167519594442213	0.554054486817265	0.83891458434315\\
0.975	0.176728657084455	0.567648038782867	0.831205342370837\\
0.975	0.186173963948925	0.581108581114919	0.822732305417527\\
0.975	0.195852662983903	0.594430029835237	0.813659432944307\\
0.975	0.205761695177907	0.607606381758231	0.80415594410487\\
0.975	0.215897797399558	0.620631723228143	0.794390053406971\\
0.975	0.226257505677663	0.633500238737009	0.7845230460628\\
0.975	0.236837158915675	0.646206219397571	0.77470398286159\\
0.975	0.247632903032949	0.658744071245709	0.765065275211018\\
0.975	0.258640695523527	0.671108323347402	0.755719307772964\\
0.975	0.269856310421543	0.683293635685828	0.74675621400591\\
0.975	0.28127534366066	0.695294806804924	0.738242834604296\\
0.975	0.292893218813452	0.707106781186547	0.730222816007834\\
0.975	0.304705193195076	0.71872465633934	0.722717741087768\\
0.975	0.316706364314172	0.730143689578457	0.715729131143269\\
0.975	0.328891676652598	0.741359304476472	0.709241120561856\\
0.975	0.341255928754291	0.752367096967051	0.703223584571265\\
0.975	0.353793780602429	0.763162841084325	0.697635496597419\\
0.975	0.366499761262991	0.773742494322337	0.692428303594619\\
0.975	0.379368276771856	0.784102202600443	0.687549132887067\\
0.975	0.392393618241769	0.794238304822092	0.682943679235328\\
0.975	0.405569970164763	0.804147337016097	0.678558662193046\\
0.975	0.418891418885081	0.813826036051075	0.674343787412871\\
0.975	0.432351961217132	0.823271342915544	0.670253187715553\\
0.975	0.445945513182735	0.832480405557786	0.666246357336158\\
0.975	0.459665918841652	0.841450581281375	0.662288623474186\\
0.975	0.473506959189401	0.850179438693979	0.658351221673191\\
0.975	0.487462361096317	0.85866475920876	0.654411055137885\\
0.975	0.501525806262096	0.866904538099407	0.65045022320647\\
0.975	0.515690940160279	0.874896985111485	0.646455401874958\\
0.975	0.529951380947623	0.882640524634437	0.642417151055155\\
0.975	0.544300728313782	0.890133795440129	0.638329210947586\\
0.975	0.558732572247415	0.897375649995373	0.634187835387497\\
0.975	0.57324050169555	0.904365153357279	0.629991194998375\\
0.975	0.587818113093873	0.911101581661729	0.625738868900924\\
0.975	0.602459018746568	0.917584420216504	0.621431431633141\\
0.975	0.617156855035314	0.923813361211857	0.617070132480553\\
0.975	0.631905290438127	0.92978830106243	0.612656657833836\\
0.975	0.6466980333399	0.935509337395467	0.608192963377087\\
0.975	0.661528839617671	0.940976765701173	0.603681161495105\\
0.975	0.676391519984903	0.94619107566195	0.599123449738607\\
0.975	0.691279947080357	0.951152947177946	0.594522067901591\\
0.975	0.706188062288412	0.955863246106974	0.589879273662389\\
0.975	0.721109882279076	0.960323019737426	0.585197329319937\\
0.975	0.736039505257226	0.964533492013186	0.580478494543975\\
0.975	0.75097111691199	0.968496058529893	0.575725022018312\\
0.975	0.765898996058536	0.972212281322139	0.570939154289562\\
0.975	0.780817519965826	0.975683883461272	0.566123121053174\\
0.975	0.795721169365275	0.978912743483559	0.561279136605556\\
0.975	0.810604533136476	0.981900889668376	0.556409397399201\\
0.975	0.825462312667457	0.984650494185936	0.551516079695737\\
0.975	0.840289325888138	0.987163867133889	0.54660133731964\\
0.975	0.855080510976838	0.989443450481767	0.541667299515137\\
0.975	0.86983092974082	0.991491811941914	0.536716068908601\\
0.975	0.884535770672926	0.993311638785082	0.531749719578142\\
0.975	0.899190351687411	0.994905731618374	0.526770295231793\\
0.975	0.913790122539012	0.996276998142666	0.521779807495104\\
0.975	0.928330666930242	0.997428446906011	0.516780234308685\\
0.975	0.942807704312699	0.998363181068902	0.511773518435872\\
0.975	0.95721709138901	0.999084392196567	0.506761566080166\\
0.975	0.971554823322703	0.999595354092743	0.501746245611875\\
0.975	0.985817034663989	0.999899416688637	0.496729386403005\\
0.975	1	1	0.491669087346904\\
0.99	0	0	0.710195569525497\\
0.99	0.000100583311362513	0.0141829653360114	0.71240460418299\\
0.99	0.000404645907256436	0.0284451766772965	0.714630690536905\\
0.99	0.000915607803432999	0.0427829086109896	0.716916312064106\\
0.99	0.00163681893109844	0.057192295687301	0.719289019360234\\
0.99	0.00257155309398959	0.0716693330697584	0.721780367061016\\
0.99	0.00372300185733414	0.0862098774609879	0.724425655492835\\
0.99	0.00509426838162598	0.100809648312589	0.727263413850304\\
0.99	0.00668836121491816	0.115464229327074	0.73033458997498\\
0.99	0.00850818805808554	0.13016907025918	0.733681422514709\\
0.99	0.0105565495182326	0.144919489023162	0.737345986077549\\
0.99	0.0128361328661109	0.159710674111862	0.741368418705642\\
0.99	0.0153495058140643	0.174537687332543	0.745784862958201\\
0.99	0.0180991103316243	0.189395466863524	0.750625176092311\\
0.99	0.0210872565164405	0.204278830634725	0.755910489868999\\
0.99	0.0243161165387281	0.219182480034174	0.761650724664453\\
0.99	0.0277877186778607	0.234101003941464	0.767842183867616\\
0.99	0.0315039414701067	0.24902888308801	0.77446537092041\\
0.99	0.0354665079868145	0.263960494742775	0.781483180780606\\
0.99	0.0396769802625738	0.278890117720924	0.788839618268073\\
0.99	0.0441367538930258	0.293811937711588	0.796459186325544\\
0.99	0.0488470528220537	0.308720052919643	0.804247066923455\\
0.99	0.0538089243380495	0.323608480015096	0.81209018616195\\
0.99	0.0590232342988274	0.338471160382329	0.819859213932693\\
0.99	0.064490662604533	0.3533019666601	0.82741149906364\\
0.99	0.0702116989375697	0.368094709561873	0.834594885813504\\
0.99	0.0761866387881432	0.382843144964686	0.841252300287179\\
0.99	0.0824155797834956	0.397540981253432	0.847226939729081\\
0.99	0.0888984183382709	0.412181886906127	0.852367847905042\\
0.99	0.0956348466427212	0.42675949830445	0.856535620027829\\
0.99	0.102624350004627	0.441267427752585	0.859607954619619\\
0.99	0.109866204559871	0.455699271686218	0.86148476026662\\
0.99	0.117359475365564	0.470048619052377	0.862092534251543\\
0.99	0.125103014888515	0.484309059839721	0.861387758068364\\
0.99	0.133095461900593	0.498474193737904	0.859359100890821\\
0.99	0.14133524079124	0.512537638903683	0.856028283765735\\
0.99	0.149820561306021	0.526493040810599	0.851449530870183\\
0.99	0.158549418718625	0.540334081158348	0.845707614739507\\
0.99	0.167519594442213	0.554054486817265	0.838914584343156\\
0.99	0.176728657084455	0.567648038782867	0.831205342370837\\
0.99	0.186173963948925	0.581108581114919	0.82273230541753\\
0.99	0.195852662983903	0.594430029835236	0.813659432944303\\
0.99	0.205761695177907	0.607606381758231	0.804155944104868\\
0.99	0.215897797399558	0.620631723228143	0.79439005340697\\
0.99	0.226257505677663	0.633500238737009	0.784523046062803\\
0.99	0.236837158915675	0.646206219397571	0.77470398286159\\
0.99	0.247632903032949	0.658744071245709	0.765065275211015\\
0.99	0.258640695523527	0.671108323347402	0.755719307772963\\
0.99	0.269856310421543	0.683293635685828	0.746756214005912\\
0.99	0.28127534366066	0.695294806804924	0.738242834604298\\
0.99	0.292893218813452	0.707106781186547	0.730222816007835\\
0.99	0.304705193195075	0.71872465633934	0.722717741087768\\
0.99	0.316706364314172	0.730143689578457	0.715729131143268\\
0.99	0.328891676652598	0.741359304476472	0.709241120561856\\
0.99	0.341255928754291	0.752367096967051	0.703223584571265\\
0.99	0.353793780602429	0.763162841084325	0.697635496597419\\
0.99	0.366499761262991	0.773742494322337	0.692428303594618\\
0.99	0.379368276771857	0.784102202600442	0.687549132887069\\
0.99	0.392393618241769	0.794238304822092	0.682943679235329\\
0.99	0.405569970164763	0.804147337016097	0.678558662193045\\
0.99	0.418891418885081	0.813826036051075	0.67434378741287\\
0.99	0.432351961217132	0.823271342915544	0.670253187715552\\
0.99	0.445945513182735	0.832480405557786	0.666246357336158\\
0.99	0.459665918841652	0.841450581281375	0.662288623474186\\
0.99	0.473506959189401	0.850179438693979	0.658351221673191\\
0.99	0.487462361096317	0.85866475920876	0.654411055137886\\
0.99	0.501525806262096	0.866904538099407	0.650450223206469\\
0.99	0.515690940160279	0.874896985111485	0.646455401874959\\
0.99	0.529951380947623	0.882640524634436	0.642417151055154\\
0.99	0.544300728313782	0.890133795440129	0.638329210947586\\
0.99	0.558732572247415	0.897375649995373	0.6341878353875\\
0.99	0.57324050169555	0.904365153357279	0.629991194998375\\
0.99	0.587818113093873	0.911101581661729	0.625738868900923\\
0.99	0.602459018746568	0.917584420216504	0.621431431633141\\
0.99	0.617156855035314	0.923813361211857	0.617070132480554\\
0.99	0.631905290438127	0.92978830106243	0.612656657833838\\
0.99	0.6466980333399	0.935509337395467	0.608192963377087\\
0.99	0.661528839617671	0.940976765701173	0.603681161495104\\
0.99	0.676391519984903	0.94619107566195	0.599123449738606\\
0.99	0.691279947080357	0.951152947177946	0.594522067901589\\
0.99	0.706188062288412	0.955863246106974	0.58987927366239\\
0.99	0.721109882279076	0.960323019737426	0.585197329319937\\
0.99	0.736039505257226	0.964533492013186	0.580478494543977\\
0.99	0.75097111691199	0.968496058529893	0.575725022018311\\
0.99	0.765898996058536	0.972212281322139	0.570939154289557\\
0.99	0.780817519965826	0.975683883461272	0.566123121053176\\
0.99	0.795721169365275	0.97891274348356	0.56127913660556\\
0.99	0.810604533136476	0.981900889668376	0.556409397399197\\
0.99	0.825462312667457	0.984650494185936	0.551516079695734\\
0.99	0.840289325888138	0.987163867133889	0.546601337319642\\
0.99	0.855080510976839	0.989443450481767	0.541667299515139\\
0.99	0.86983092974082	0.991491811941914	0.536716068908599\\
0.99	0.884535770672926	0.993311638785082	0.531749719578143\\
0.99	0.899190351687411	0.994905731618374	0.526770295231794\\
0.99	0.913790122539012	0.996276998142666	0.521779807495103\\
0.99	0.928330666930241	0.997428446906011	0.516780234308683\\
0.99	0.942807704312699	0.998363181068902	0.511773518435873\\
0.99	0.95721709138901	0.999084392196567	0.506761566080166\\
0.99	0.971554823322703	0.999595354092743	0.501746245611874\\
0.99	0.985817034663989	0.999899416688637	0.496729386403005\\
0.99	1	1	0.491669087346904\\
1.005	0	0	0.710195569525497\\
1.005	0.000100583311362513	0.0141829653360114	0.71240460418299\\
1.005	0.000404645907256436	0.0284451766772965	0.714630690536905\\
1.005	0.000915607803432999	0.0427829086109896	0.716916312064106\\
1.005	0.00163681893109844	0.057192295687301	0.719289019360234\\
1.005	0.00257155309398959	0.0716693330697584	0.721780367061016\\
1.005	0.00372300185733413	0.0862098774609879	0.724425655492835\\
1.005	0.00509426838162598	0.100809648312589	0.727263413850304\\
1.005	0.00668836121491815	0.115464229327074	0.73033458997498\\
1.005	0.00850818805808555	0.13016907025918	0.733681422514709\\
1.005	0.0105565495182326	0.144919489023162	0.737345986077549\\
1.005	0.0128361328661109	0.159710674111862	0.741368418705642\\
1.005	0.0153495058140643	0.174537687332543	0.745784862958202\\
1.005	0.0180991103316243	0.189395466863524	0.750625176092311\\
1.005	0.0210872565164405	0.204278830634725	0.755910489868999\\
1.005	0.0243161165387281	0.219182480034174	0.761650724664453\\
1.005	0.0277877186778607	0.234101003941464	0.767842183867616\\
1.005	0.0315039414701067	0.24902888308801	0.77446537092041\\
1.005	0.0354665079868145	0.263960494742775	0.781483180780606\\
1.005	0.0396769802625738	0.278890117720924	0.788839618268073\\
1.005	0.0441367538930258	0.293811937711588	0.796459186325543\\
1.005	0.0488470528220538	0.308720052919643	0.804247066923455\\
1.005	0.0538089243380495	0.323608480015096	0.812090186161949\\
1.005	0.0590232342988274	0.338471160382329	0.819859213932693\\
1.005	0.064490662604533	0.3533019666601	0.827411499063639\\
1.005	0.0702116989375697	0.368094709561873	0.834594885813504\\
1.005	0.0761866387881432	0.382843144964686	0.84125230028718\\
1.005	0.0824155797834956	0.397540981253432	0.847226939729081\\
1.005	0.0888984183382709	0.412181886906127	0.852367847905041\\
1.005	0.0956348466427212	0.42675949830445	0.85653562002783\\
1.005	0.102624350004627	0.441267427752585	0.859607954619617\\
1.005	0.109866204559871	0.455699271686218	0.861484760266621\\
1.005	0.117359475365564	0.470048619052377	0.862092534251538\\
1.005	0.125103014888515	0.484309059839721	0.861387758068366\\
1.005	0.133095461900593	0.498474193737904	0.859359100890822\\
1.005	0.14133524079124	0.512537638903683	0.856028283765737\\
1.005	0.149820561306021	0.526493040810599	0.851449530870186\\
1.005	0.158549418718625	0.540334081158348	0.845707614739509\\
1.005	0.167519594442213	0.554054486817266	0.83891458434315\\
1.005	0.176728657084455	0.567648038782867	0.831205342370836\\
1.005	0.186173963948925	0.581108581114919	0.822732305417528\\
1.005	0.195852662983903	0.594430029835236	0.813659432944303\\
1.005	0.205761695177907	0.60760638175823	0.804155944104866\\
1.005	0.215897797399558	0.620631723228143	0.794390053406973\\
1.005	0.226257505677663	0.633500238737009	0.784523046062803\\
1.005	0.236837158915675	0.646206219397571	0.77470398286159\\
1.005	0.247632903032949	0.658744071245709	0.765065275211015\\
1.005	0.258640695523527	0.671108323347402	0.755719307772963\\
1.005	0.269856310421543	0.683293635685828	0.746756214005914\\
1.005	0.28127534366066	0.695294806804925	0.738242834604297\\
1.005	0.292893218813452	0.707106781186548	0.730222816007833\\
1.005	0.304705193195076	0.71872465633934	0.722717741087766\\
1.005	0.316706364314172	0.730143689578457	0.715729131143268\\
1.005	0.328891676652598	0.741359304476472	0.709241120561854\\
1.005	0.341255928754291	0.752367096967051	0.703223584571264\\
1.005	0.353793780602429	0.763162841084325	0.69763549659742\\
1.005	0.366499761262991	0.773742494322337	0.69242830359462\\
1.005	0.379368276771856	0.784102202600443	0.687549132887068\\
1.005	0.392393618241769	0.794238304822092	0.682943679235327\\
1.005	0.405569970164763	0.804147337016096	0.678558662193046\\
1.005	0.418891418885081	0.813826036051075	0.674343787412871\\
1.005	0.432351961217133	0.823271342915545	0.670253187715553\\
1.005	0.445945513182735	0.832480405557786	0.666246357336159\\
1.005	0.459665918841652	0.841450581281375	0.662288623474186\\
1.005	0.473506959189401	0.850179438693979	0.658351221673191\\
1.005	0.487462361096317	0.85866475920876	0.654411055137886\\
1.005	0.501525806262096	0.866904538099407	0.650450223206467\\
1.005	0.515690940160279	0.874896985111485	0.64645540187496\\
1.005	0.529951380947623	0.882640524634437	0.642417151055157\\
1.005	0.544300728313781	0.890133795440129	0.638329210947585\\
1.005	0.558732572247415	0.897375649995373	0.634187835387497\\
1.005	0.57324050169555	0.904365153357279	0.629991194998376\\
1.005	0.587818113093873	0.911101581661729	0.625738868900924\\
1.005	0.602459018746568	0.917584420216504	0.62143143163314\\
1.005	0.617156855035314	0.923813361211857	0.617070132480554\\
1.005	0.631905290438127	0.92978830106243	0.612656657833839\\
1.005	0.6466980333399	0.935509337395467	0.608192963377088\\
1.005	0.661528839617671	0.940976765701173	0.603681161495105\\
1.005	0.676391519984903	0.946191075661951	0.599123449738607\\
1.005	0.691279947080357	0.951152947177946	0.594522067901589\\
1.005	0.706188062288412	0.955863246106974	0.58987927366239\\
1.005	0.721109882279076	0.960323019737426	0.585197329319937\\
1.005	0.736039505257226	0.964533492013186	0.580478494543978\\
1.005	0.75097111691199	0.968496058529893	0.575725022018314\\
1.005	0.765898996058536	0.972212281322139	0.570939154289557\\
1.005	0.780817519965825	0.975683883461272	0.566123121053173\\
1.005	0.795721169365275	0.978912743483559	0.561279136605561\\
1.005	0.810604533136476	0.981900889668376	0.556409397399203\\
1.005	0.825462312667457	0.984650494185936	0.551516079695733\\
1.005	0.840289325888138	0.987163867133889	0.546601337319639\\
1.005	0.855080510976838	0.989443450481768	0.541667299515139\\
1.005	0.86983092974082	0.991491811941914	0.536716068908598\\
1.005	0.884535770672926	0.993311638785082	0.531749719578141\\
1.005	0.899190351687411	0.994905731618374	0.526770295231795\\
1.005	0.913790122539012	0.996276998142666	0.521779807495105\\
1.005	0.928330666930242	0.997428446906011	0.516780234308683\\
1.005	0.942807704312699	0.998363181068902	0.511773518435873\\
1.005	0.95721709138901	0.999084392196567	0.506761566080168\\
1.005	0.971554823322703	0.999595354092743	0.501746245611876\\
1.005	0.985817034663989	0.999899416688638	0.496729386403005\\
1.005	1	1	0.491669087346895\\
1.02	0	0	0.710195569525497\\
1.02	0.000100583311362513	0.0141829653360114	0.71240460418299\\
1.02	0.000404645907256436	0.0284451766772965	0.714630690536905\\
1.02	0.000915607803432999	0.0427829086109896	0.716916312064106\\
1.02	0.00163681893109844	0.057192295687301	0.719289019360234\\
1.02	0.00257155309398959	0.0716693330697584	0.721780367061016\\
1.02	0.00372300185733413	0.0862098774609879	0.724425655492835\\
1.02	0.00509426838162598	0.100809648312589	0.727263413850304\\
1.02	0.00668836121491816	0.115464229327074	0.73033458997498\\
1.02	0.00850818805808554	0.13016907025918	0.733681422514709\\
1.02	0.0105565495182326	0.144919489023162	0.737345986077549\\
1.02	0.0128361328661109	0.159710674111862	0.741368418705642\\
1.02	0.0153495058140643	0.174537687332543	0.745784862958202\\
1.02	0.0180991103316243	0.189395466863524	0.750625176092311\\
1.02	0.0210872565164405	0.204278830634725	0.755910489868999\\
1.02	0.0243161165387281	0.219182480034174	0.761650724664453\\
1.02	0.0277877186778607	0.234101003941464	0.767842183867616\\
1.02	0.0315039414701067	0.24902888308801	0.77446537092041\\
1.02	0.0354665079868145	0.263960494742775	0.781483180780606\\
1.02	0.0396769802625738	0.278890117720924	0.788839618268073\\
1.02	0.0441367538930258	0.293811937711588	0.796459186325543\\
1.02	0.0488470528220538	0.308720052919643	0.804247066923454\\
1.02	0.0538089243380495	0.323608480015096	0.812090186161949\\
1.02	0.0590232342988274	0.338471160382329	0.819859213932693\\
1.02	0.064490662604533	0.3533019666601	0.827411499063639\\
1.02	0.0702116989375697	0.368094709561873	0.834594885813504\\
1.02	0.0761866387881432	0.382843144964686	0.841252300287177\\
1.02	0.0824155797834956	0.397540981253432	0.847226939729083\\
1.02	0.0888984183382709	0.412181886906127	0.852367847905043\\
1.02	0.0956348466427212	0.42675949830445	0.856535620027829\\
1.02	0.102624350004627	0.441267427752584	0.859607954619618\\
1.02	0.109866204559871	0.455699271686218	0.861484760266622\\
1.02	0.117359475365564	0.470048619052377	0.862092534251538\\
1.02	0.125103014888515	0.484309059839721	0.861387758068371\\
1.02	0.133095461900593	0.498474193737904	0.85935910089082\\
1.02	0.14133524079124	0.512537638903683	0.856028283765739\\
1.02	0.149820561306021	0.526493040810599	0.851449530870182\\
1.02	0.158549418718625	0.540334081158348	0.845707614739504\\
1.02	0.167519594442214	0.554054486817265	0.838914584343148\\
1.02	0.176728657084455	0.567648038782867	0.831205342370836\\
1.02	0.186173963948925	0.581108581114919	0.822732305417524\\
1.02	0.195852662983903	0.594430029835236	0.813659432944307\\
1.02	0.205761695177907	0.60760638175823	0.804155944104871\\
1.02	0.215897797399558	0.620631723228143	0.794390053406973\\
1.02	0.226257505677663	0.633500238737009	0.784523046062803\\
1.02	0.236837158915675	0.646206219397572	0.774703982861589\\
1.02	0.247632903032949	0.658744071245709	0.765065275211013\\
1.02	0.258640695523527	0.671108323347402	0.755719307772965\\
1.02	0.269856310421543	0.683293635685828	0.74675621400591\\
1.02	0.28127534366066	0.695294806804924	0.738242834604293\\
1.02	0.292893218813452	0.707106781186547	0.730222816007833\\
1.02	0.304705193195075	0.71872465633934	0.722717741087768\\
1.02	0.316706364314172	0.730143689578457	0.715729131143268\\
1.02	0.328891676652598	0.741359304476472	0.709241120561854\\
1.02	0.341255928754291	0.752367096967051	0.703223584571265\\
1.02	0.353793780602429	0.763162841084325	0.697635496597419\\
1.02	0.366499761262991	0.773742494322337	0.692428303594619\\
1.02	0.379368276771856	0.784102202600442	0.687549132887069\\
1.02	0.392393618241769	0.794238304822092	0.682943679235328\\
1.02	0.405569970164763	0.804147337016096	0.678558662193046\\
1.02	0.418891418885081	0.813826036051075	0.67434378741287\\
1.02	0.432351961217132	0.823271342915544	0.670253187715551\\
1.02	0.445945513182735	0.832480405557786	0.66624635733616\\
1.02	0.459665918841652	0.841450581281375	0.662288623474186\\
1.02	0.473506959189401	0.850179438693979	0.658351221673192\\
1.02	0.487462361096317	0.85866475920876	0.654411055137886\\
1.02	0.501525806262096	0.866904538099407	0.650450223206468\\
1.02	0.515690940160279	0.874896985111485	0.646455401874957\\
1.02	0.529951380947623	0.882640524634436	0.642417151055156\\
1.02	0.544300728313781	0.890133795440129	0.638329210947589\\
1.02	0.558732572247415	0.897375649995373	0.634187835387498\\
1.02	0.57324050169555	0.904365153357279	0.629991194998375\\
1.02	0.587818113093873	0.911101581661729	0.625738868900923\\
1.02	0.602459018746568	0.917584420216504	0.621431431633141\\
1.02	0.617156855035314	0.923813361211857	0.617070132480553\\
1.02	0.631905290438127	0.92978830106243	0.612656657833836\\
1.02	0.6466980333399	0.935509337395467	0.608192963377087\\
1.02	0.661528839617671	0.940976765701173	0.603681161495107\\
1.02	0.676391519984904	0.946191075661951	0.599123449738608\\
1.02	0.691279947080357	0.951152947177946	0.594522067901588\\
1.02	0.706188062288412	0.955863246106974	0.589879273662391\\
1.02	0.721109882279076	0.960323019737426	0.585197329319937\\
1.02	0.736039505257226	0.964533492013185	0.580478494543976\\
1.02	0.75097111691199	0.968496058529893	0.575725022018313\\
1.02	0.765898996058536	0.972212281322139	0.570939154289556\\
1.02	0.780817519965826	0.975683883461272	0.566123121053173\\
1.02	0.795721169365275	0.978912743483559	0.56127913660556\\
1.02	0.810604533136476	0.981900889668376	0.556409397399201\\
1.02	0.825462312667457	0.984650494185936	0.551516079695736\\
1.02	0.840289325888138	0.987163867133889	0.546601337319641\\
1.02	0.855080510976839	0.989443450481767	0.541667299515138\\
1.02	0.86983092974082	0.991491811941914	0.536716068908598\\
1.02	0.884535770672926	0.993311638785082	0.531749719578141\\
1.02	0.899190351687411	0.994905731618374	0.526770295231792\\
1.02	0.913790122539012	0.996276998142666	0.521779807495104\\
1.02	0.928330666930242	0.997428446906011	0.516780234308684\\
1.02	0.942807704312699	0.998363181068901	0.511773518435872\\
1.02	0.95721709138901	0.999084392196567	0.50676156608017\\
1.02	0.971554823322703	0.999595354092743	0.501746245611874\\
1.02	0.985817034663989	0.999899416688637	0.496729386403004\\
1.02	1	1	0.491669087346906\\
1.035	0	0	0.710195569525497\\
1.035	0.000100583311362513	0.0141829653360114	0.71240460418299\\
1.035	0.000404645907256436	0.0284451766772965	0.714630690536905\\
1.035	0.000915607803432999	0.0427829086109896	0.716916312064106\\
1.035	0.00163681893109844	0.057192295687301	0.719289019360234\\
1.035	0.00257155309398959	0.0716693330697584	0.721780367061016\\
1.035	0.00372300185733413	0.0862098774609879	0.724425655492835\\
1.035	0.00509426838162598	0.100809648312589	0.727263413850304\\
1.035	0.00668836121491816	0.115464229327074	0.73033458997498\\
1.035	0.00850818805808555	0.13016907025918	0.733681422514709\\
1.035	0.0105565495182326	0.144919489023162	0.737345986077549\\
1.035	0.0128361328661109	0.159710674111862	0.741368418705642\\
1.035	0.0153495058140643	0.174537687332543	0.745784862958201\\
1.035	0.0180991103316243	0.189395466863524	0.750625176092311\\
1.035	0.0210872565164405	0.204278830634725	0.755910489868999\\
1.035	0.0243161165387281	0.219182480034174	0.761650724664453\\
1.035	0.0277877186778607	0.234101003941464	0.767842183867616\\
1.035	0.0315039414701067	0.24902888308801	0.77446537092041\\
1.035	0.0354665079868145	0.263960494742775	0.781483180780606\\
1.035	0.0396769802625738	0.278890117720924	0.788839618268073\\
1.035	0.0441367538930258	0.293811937711588	0.796459186325545\\
1.035	0.0488470528220537	0.308720052919643	0.804247066923455\\
1.035	0.0538089243380495	0.323608480015096	0.812090186161951\\
1.035	0.0590232342988274	0.338471160382329	0.819859213932693\\
1.035	0.064490662604533	0.3533019666601	0.827411499063638\\
1.035	0.0702116989375697	0.368094709561873	0.834594885813505\\
1.035	0.0761866387881432	0.382843144964686	0.841252300287178\\
1.035	0.0824155797834956	0.397540981253432	0.847226939729081\\
1.035	0.0888984183382709	0.412181886906127	0.852367847905042\\
1.035	0.0956348466427212	0.42675949830445	0.856535620027828\\
1.035	0.102624350004627	0.441267427752585	0.859607954619617\\
1.035	0.109866204559871	0.455699271686218	0.861484760266622\\
1.035	0.117359475365564	0.470048619052377	0.862092534251542\\
1.035	0.125103014888515	0.484309059839721	0.861387758068371\\
1.035	0.133095461900593	0.498474193737904	0.859359100890818\\
1.035	0.14133524079124	0.512537638903683	0.856028283765736\\
1.035	0.149820561306021	0.526493040810599	0.851449530870179\\
1.035	0.158549418718625	0.540334081158348	0.845707614739505\\
1.035	0.167519594442214	0.554054486817265	0.83891458434315\\
1.035	0.176728657084455	0.567648038782867	0.831205342370835\\
1.035	0.186173963948925	0.581108581114919	0.822732305417529\\
1.035	0.195852662983903	0.594430029835236	0.813659432944303\\
1.035	0.205761695177907	0.607606381758231	0.804155944104873\\
1.035	0.215897797399558	0.620631723228143	0.794390053406974\\
1.035	0.226257505677663	0.633500238737009	0.784523046062803\\
1.035	0.236837158915675	0.646206219397571	0.774703982861585\\
1.035	0.247632903032949	0.658744071245709	0.765065275211016\\
1.035	0.258640695523527	0.671108323347402	0.755719307772961\\
1.035	0.269856310421543	0.683293635685828	0.74675621400591\\
1.035	0.28127534366066	0.695294806804924	0.738242834604298\\
1.035	0.292893218813452	0.707106781186547	0.730222816007834\\
1.035	0.304705193195075	0.71872465633934	0.722717741087766\\
1.035	0.316706364314172	0.730143689578457	0.715729131143268\\
1.035	0.328891676652598	0.741359304476472	0.709241120561856\\
1.035	0.341255928754291	0.752367096967051	0.703223584571265\\
1.035	0.353793780602429	0.763162841084324	0.69763549659742\\
1.035	0.366499761262991	0.773742494322337	0.69242830359462\\
1.035	0.379368276771856	0.784102202600443	0.687549132887069\\
1.035	0.392393618241769	0.794238304822092	0.682943679235328\\
1.035	0.405569970164763	0.804147337016097	0.678558662193046\\
1.035	0.418891418885081	0.813826036051075	0.674343787412871\\
1.035	0.432351961217132	0.823271342915544	0.67025318771555\\
1.035	0.445945513182735	0.832480405557786	0.666246357336158\\
1.035	0.459665918841652	0.841450581281375	0.662288623474187\\
1.035	0.473506959189401	0.850179438693979	0.658351221673192\\
1.035	0.487462361096317	0.85866475920876	0.654411055137886\\
1.035	0.501525806262096	0.866904538099407	0.650450223206468\\
1.035	0.515690940160279	0.874896985111485	0.646455401874959\\
1.035	0.529951380947623	0.882640524634437	0.642417151055155\\
1.035	0.544300728313781	0.890133795440129	0.638329210947586\\
1.035	0.558732572247415	0.897375649995372	0.634187835387498\\
1.035	0.57324050169555	0.904365153357279	0.629991194998378\\
1.035	0.587818113093873	0.911101581661729	0.625738868900924\\
1.035	0.602459018746568	0.917584420216504	0.62143143163314\\
1.035	0.617156855035314	0.923813361211857	0.617070132480553\\
1.035	0.631905290438127	0.92978830106243	0.612656657833837\\
1.035	0.6466980333399	0.935509337395467	0.608192963377087\\
1.035	0.661528839617671	0.940976765701173	0.603681161495105\\
1.035	0.676391519984904	0.946191075661951	0.599123449738606\\
1.035	0.691279947080357	0.951152947177946	0.594522067901588\\
1.035	0.706188062288412	0.955863246106974	0.58987927366239\\
1.035	0.721109882279076	0.960323019737426	0.585197329319938\\
1.035	0.736039505257226	0.964533492013186	0.580478494543978\\
1.035	0.75097111691199	0.968496058529893	0.575725022018314\\
1.035	0.765898996058536	0.972212281322139	0.570939154289559\\
1.035	0.780817519965826	0.975683883461272	0.566123121053175\\
1.035	0.795721169365275	0.97891274348356	0.561279136605558\\
1.035	0.810604533136476	0.981900889668376	0.556409397399198\\
1.035	0.825462312667457	0.984650494185936	0.551516079695734\\
1.035	0.840289325888138	0.987163867133889	0.546601337319642\\
1.035	0.855080510976838	0.989443450481768	0.541667299515143\\
1.035	0.86983092974082	0.991491811941915	0.536716068908599\\
1.035	0.884535770672926	0.993311638785082	0.531749719578139\\
1.035	0.899190351687411	0.994905731618374	0.526770295231792\\
1.035	0.913790122539012	0.996276998142666	0.5217798074951\\
1.035	0.928330666930241	0.99742844690601	0.516780234308683\\
1.035	0.942807704312699	0.998363181068902	0.511773518435872\\
1.035	0.95721709138901	0.999084392196567	0.50676156608017\\
1.035	0.971554823322703	0.999595354092743	0.501746245611875\\
1.035	0.985817034663989	0.999899416688637	0.496729386403004\\
1.035	1	1	0.491669087346906\\
1.05	0	0	0.710195569525497\\
1.05	0.000100583311362513	0.0141829653360114	0.71240460418299\\
1.05	0.000404645907256436	0.0284451766772965	0.714630690536905\\
1.05	0.000915607803432999	0.0427829086109896	0.716916312064106\\
1.05	0.00163681893109844	0.057192295687301	0.719289019360234\\
1.05	0.00257155309398959	0.0716693330697584	0.721780367061016\\
1.05	0.00372300185733413	0.0862098774609879	0.724425655492835\\
1.05	0.00509426838162598	0.100809648312589	0.727263413850304\\
1.05	0.00668836121491816	0.115464229327074	0.73033458997498\\
1.05	0.00850818805808555	0.13016907025918	0.733681422514709\\
1.05	0.0105565495182326	0.144919489023162	0.737345986077549\\
1.05	0.0128361328661109	0.159710674111862	0.741368418705642\\
1.05	0.0153495058140643	0.174537687332543	0.745784862958202\\
1.05	0.0180991103316243	0.189395466863524	0.750625176092311\\
1.05	0.0210872565164405	0.204278830634725	0.755910489868999\\
1.05	0.0243161165387281	0.219182480034174	0.761650724664453\\
1.05	0.0277877186778607	0.234101003941464	0.767842183867616\\
1.05	0.0315039414701067	0.24902888308801	0.77446537092041\\
1.05	0.0354665079868145	0.263960494742775	0.781483180780606\\
1.05	0.0396769802625738	0.278890117720924	0.788839618268074\\
1.05	0.0441367538930258	0.293811937711588	0.796459186325544\\
1.05	0.0488470528220537	0.308720052919643	0.804247066923455\\
1.05	0.0538089243380495	0.323608480015096	0.81209018616195\\
1.05	0.0590232342988274	0.338471160382329	0.819859213932694\\
1.05	0.064490662604533	0.3533019666601	0.827411499063639\\
1.05	0.0702116989375697	0.368094709561873	0.834594885813504\\
1.05	0.0761866387881432	0.382843144964686	0.841252300287179\\
1.05	0.0824155797834956	0.397540981253432	0.847226939729079\\
1.05	0.0888984183382709	0.412181886906127	0.852367847905041\\
1.05	0.0956348466427212	0.42675949830445	0.856535620027829\\
1.05	0.102624350004627	0.441267427752584	0.859607954619618\\
1.05	0.109866204559871	0.455699271686218	0.861484760266622\\
1.05	0.117359475365564	0.470048619052377	0.862092534251541\\
1.05	0.125103014888515	0.484309059839721	0.86138775806837\\
1.05	0.133095461900593	0.498474193737904	0.85935910089082\\
1.05	0.14133524079124	0.512537638903683	0.856028283765736\\
1.05	0.149820561306021	0.526493040810599	0.851449530870181\\
1.05	0.158549418718625	0.540334081158348	0.845707614739509\\
1.05	0.167519594442214	0.554054486817265	0.838914584343147\\
1.05	0.176728657084455	0.567648038782867	0.831205342370839\\
1.05	0.186173963948925	0.581108581114919	0.822732305417526\\
1.05	0.195852662983903	0.594430029835236	0.813659432944308\\
1.05	0.205761695177907	0.607606381758231	0.804155944104875\\
1.05	0.215897797399558	0.620631723228144	0.79439005340697\\
1.05	0.226257505677663	0.633500238737009	0.784523046062799\\
1.05	0.236837158915675	0.646206219397571	0.774703982861588\\
1.05	0.247632903032949	0.658744071245709	0.765065275211018\\
1.05	0.258640695523527	0.671108323347402	0.755719307772962\\
1.05	0.269856310421543	0.683293635685828	0.746756214005912\\
1.05	0.28127534366066	0.695294806804924	0.738242834604297\\
1.05	0.292893218813452	0.707106781186547	0.730222816007834\\
1.05	0.304705193195075	0.71872465633934	0.722717741087767\\
1.05	0.316706364314172	0.730143689578457	0.715729131143268\\
1.05	0.328891676652598	0.741359304476472	0.709241120561858\\
1.05	0.341255928754291	0.752367096967051	0.703223584571265\\
1.05	0.353793780602429	0.763162841084325	0.697635496597417\\
1.05	0.366499761262991	0.773742494322337	0.692428303594618\\
1.05	0.379368276771857	0.784102202600442	0.68754913288707\\
1.05	0.392393618241769	0.794238304822092	0.682943679235328\\
1.05	0.405569970164763	0.804147337016096	0.678558662193043\\
1.05	0.418891418885081	0.813826036051075	0.674343787412872\\
1.05	0.432351961217132	0.823271342915544	0.670253187715553\\
1.05	0.445945513182735	0.832480405557786	0.666246357336158\\
1.05	0.459665918841652	0.841450581281375	0.662288623474186\\
1.05	0.473506959189401	0.850179438693979	0.658351221673191\\
1.05	0.487462361096317	0.85866475920876	0.654411055137887\\
1.05	0.501525806262096	0.866904538099407	0.650450223206469\\
1.05	0.51569094016028	0.874896985111485	0.646455401874958\\
1.05	0.529951380947623	0.882640524634436	0.642417151055154\\
1.05	0.544300728313782	0.890133795440129	0.638329210947585\\
1.05	0.558732572247415	0.897375649995372	0.634187835387498\\
1.05	0.57324050169555	0.904365153357279	0.629991194998377\\
1.05	0.587818113093873	0.911101581661729	0.625738868900924\\
1.05	0.602459018746568	0.917584420216504	0.62143143163314\\
1.05	0.617156855035314	0.923813361211857	0.617070132480554\\
1.05	0.631905290438127	0.92978830106243	0.612656657833837\\
1.05	0.6466980333399	0.935509337395467	0.608192963377088\\
1.05	0.661528839617671	0.940976765701173	0.603681161495105\\
1.05	0.676391519984904	0.94619107566195	0.599123449738607\\
1.05	0.691279947080357	0.951152947177946	0.594522067901589\\
1.05	0.706188062288412	0.955863246106974	0.589879273662389\\
1.05	0.721109882279076	0.960323019737426	0.585197329319936\\
1.05	0.736039505257226	0.964533492013185	0.580478494543975\\
1.05	0.75097111691199	0.968496058529893	0.575725022018314\\
1.05	0.765898996058536	0.972212281322139	0.570939154289561\\
1.05	0.780817519965826	0.975683883461272	0.566123121053175\\
1.05	0.795721169365275	0.978912743483559	0.561279136605559\\
1.05	0.810604533136476	0.981900889668376	0.556409397399201\\
1.05	0.825462312667457	0.984650494185936	0.551516079695734\\
1.05	0.840289325888138	0.987163867133889	0.54660133731964\\
1.05	0.855080510976839	0.989443450481767	0.54166729951514\\
1.05	0.86983092974082	0.991491811941914	0.536716068908602\\
1.05	0.884535770672926	0.993311638785082	0.531749719578143\\
1.05	0.899190351687411	0.994905731618374	0.526770295231795\\
1.05	0.913790122539012	0.996276998142666	0.521779807495104\\
1.05	0.928330666930242	0.997428446906011	0.516780234308681\\
1.05	0.942807704312699	0.998363181068901	0.511773518435871\\
1.05	0.95721709138901	0.999084392196567	0.506761566080173\\
1.05	0.971554823322704	0.999595354092744	0.501746245611874\\
1.05	0.985817034663989	0.999899416688637	0.496729386403\\
1.05	1	1	0.49166908734691\\
1.065	0	0	0.710195569525497\\
1.065	0.000100583311362513	0.0141829653360114	0.71240460418299\\
1.065	0.000404645907256436	0.0284451766772965	0.714630690536905\\
1.065	0.000915607803432999	0.0427829086109896	0.716916312064106\\
1.065	0.00163681893109844	0.057192295687301	0.719289019360234\\
1.065	0.00257155309398959	0.0716693330697584	0.721780367061016\\
1.065	0.00372300185733413	0.0862098774609879	0.724425655492835\\
1.065	0.00509426838162598	0.100809648312589	0.727263413850304\\
1.065	0.00668836121491815	0.115464229327074	0.73033458997498\\
1.065	0.00850818805808555	0.13016907025918	0.733681422514709\\
1.065	0.0105565495182326	0.144919489023162	0.737345986077549\\
1.065	0.0128361328661109	0.159710674111862	0.741368418705642\\
1.065	0.0153495058140643	0.174537687332543	0.745784862958201\\
1.065	0.0180991103316243	0.189395466863524	0.750625176092311\\
1.065	0.0210872565164405	0.204278830634725	0.755910489868999\\
1.065	0.0243161165387281	0.219182480034174	0.761650724664453\\
1.065	0.0277877186778607	0.234101003941464	0.767842183867616\\
1.065	0.0315039414701067	0.24902888308801	0.77446537092041\\
1.065	0.0354665079868145	0.263960494742775	0.781483180780606\\
1.065	0.0396769802625738	0.278890117720924	0.788839618268073\\
1.065	0.0441367538930258	0.293811937711588	0.796459186325544\\
1.065	0.0488470528220538	0.308720052919643	0.804247066923454\\
1.065	0.0538089243380495	0.323608480015096	0.81209018616195\\
1.065	0.0590232342988274	0.338471160382329	0.819859213932694\\
1.065	0.064490662604533	0.3533019666601	0.827411499063641\\
1.065	0.0702116989375697	0.368094709561873	0.834594885813506\\
1.065	0.0761866387881432	0.382843144964686	0.84125230028718\\
1.065	0.0824155797834956	0.397540981253432	0.847226939729079\\
1.065	0.0888984183382709	0.412181886906127	0.85236784790504\\
1.065	0.0956348466427212	0.42675949830445	0.856535620027829\\
1.065	0.102624350004627	0.441267427752584	0.859607954619617\\
1.065	0.109866204559871	0.455699271686218	0.861484760266621\\
1.065	0.117359475365564	0.470048619052377	0.862092534251539\\
1.065	0.125103014888515	0.484309059839721	0.861387758068367\\
1.065	0.133095461900593	0.498474193737904	0.859359100890821\\
1.065	0.14133524079124	0.512537638903683	0.856028283765739\\
1.065	0.149820561306021	0.526493040810599	0.851449530870184\\
1.065	0.158549418718625	0.540334081158348	0.845707614739502\\
1.065	0.167519594442214	0.554054486817265	0.838914584343149\\
1.065	0.176728657084455	0.567648038782868	0.831205342370839\\
1.065	0.186173963948925	0.581108581114919	0.822732305417525\\
1.065	0.195852662983903	0.594430029835237	0.813659432944308\\
1.065	0.205761695177907	0.607606381758231	0.804155944104864\\
1.065	0.215897797399558	0.620631723228143	0.79439005340697\\
1.065	0.226257505677663	0.633500238737009	0.784523046062802\\
1.065	0.236837158915675	0.646206219397571	0.774703982861591\\
1.065	0.247632903032949	0.658744071245709	0.765065275211014\\
1.065	0.258640695523528	0.671108323347402	0.755719307772959\\
1.065	0.269856310421543	0.683293635685828	0.746756214005914\\
1.065	0.28127534366066	0.695294806804925	0.738242834604296\\
1.065	0.292893218813452	0.707106781186547	0.730222816007833\\
1.065	0.304705193195076	0.71872465633934	0.722717741087767\\
1.065	0.316706364314172	0.730143689578457	0.715729131143267\\
1.065	0.328891676652598	0.741359304476472	0.709241120561857\\
1.065	0.341255928754291	0.752367096967051	0.703223584571265\\
1.065	0.353793780602429	0.763162841084325	0.697635496597418\\
1.065	0.366499761262991	0.773742494322337	0.692428303594617\\
1.065	0.379368276771857	0.784102202600442	0.687549132887069\\
1.065	0.392393618241769	0.794238304822092	0.68294367923533\\
1.065	0.405569970164763	0.804147337016097	0.678558662193044\\
1.065	0.418891418885081	0.813826036051075	0.67434378741287\\
1.065	0.432351961217132	0.823271342915544	0.670253187715552\\
1.065	0.445945513182735	0.832480405557786	0.66624635733616\\
1.065	0.459665918841652	0.841450581281375	0.662288623474186\\
1.065	0.473506959189401	0.850179438693979	0.65835122167319\\
1.065	0.487462361096317	0.85866475920876	0.654411055137886\\
1.065	0.501525806262096	0.866904538099407	0.65045022320647\\
1.065	0.515690940160279	0.874896985111485	0.646455401874958\\
1.065	0.529951380947623	0.882640524634436	0.642417151055154\\
1.065	0.544300728313782	0.890133795440129	0.638329210947588\\
1.065	0.558732572247415	0.897375649995373	0.634187835387498\\
1.065	0.57324050169555	0.904365153357279	0.629991194998375\\
1.065	0.587818113093873	0.911101581661729	0.625738868900922\\
1.065	0.602459018746568	0.917584420216504	0.621431431633141\\
1.065	0.617156855035314	0.923813361211857	0.617070132480554\\
1.065	0.631905290438127	0.92978830106243	0.612656657833838\\
1.065	0.6466980333399	0.935509337395467	0.608192963377088\\
1.065	0.661528839617671	0.940976765701173	0.603681161495104\\
1.065	0.676391519984904	0.94619107566195	0.599123449738607\\
1.065	0.691279947080357	0.951152947177946	0.594522067901589\\
1.065	0.706188062288412	0.955863246106974	0.58987927366239\\
1.065	0.721109882279076	0.960323019737426	0.585197329319938\\
1.065	0.736039505257226	0.964533492013186	0.580478494543972\\
1.065	0.75097111691199	0.968496058529893	0.575725022018312\\
1.065	0.765898996058536	0.972212281322139	0.570939154289561\\
1.065	0.780817519965826	0.975683883461272	0.566123121053177\\
1.065	0.795721169365275	0.97891274348356	0.56127913660556\\
1.065	0.810604533136476	0.981900889668376	0.556409397399198\\
1.065	0.825462312667457	0.984650494185936	0.551516079695733\\
1.065	0.840289325888138	0.987163867133889	0.546601337319637\\
1.065	0.855080510976838	0.989443450481767	0.54166729951514\\
1.065	0.86983092974082	0.991491811941914	0.536716068908599\\
1.065	0.884535770672926	0.993311638785082	0.531749719578143\\
1.065	0.899190351687411	0.994905731618374	0.526770295231793\\
1.065	0.913790122539012	0.996276998142666	0.521779807495102\\
1.065	0.928330666930242	0.997428446906011	0.516780234308684\\
1.065	0.942807704312699	0.998363181068902	0.511773518435873\\
1.065	0.95721709138901	0.999084392196567	0.50676156608017\\
1.065	0.971554823322703	0.999595354092743	0.501746245611876\\
1.065	0.985817034663989	0.999899416688637	0.496729386403004\\
1.065	1	1	0.491669087346906\\
1.08	0	0	0.710195569525497\\
1.08	0.000100583311362513	0.0141829653360114	0.71240460418299\\
1.08	0.000404645907256436	0.0284451766772965	0.714630690536905\\
1.08	0.000915607803432999	0.0427829086109896	0.716916312064106\\
1.08	0.00163681893109844	0.057192295687301	0.719289019360234\\
1.08	0.00257155309398959	0.0716693330697584	0.721780367061016\\
1.08	0.00372300185733413	0.0862098774609879	0.724425655492835\\
1.08	0.00509426838162598	0.100809648312589	0.727263413850304\\
1.08	0.00668836121491816	0.115464229327074	0.73033458997498\\
1.08	0.00850818805808554	0.13016907025918	0.733681422514709\\
1.08	0.0105565495182326	0.144919489023162	0.737345986077549\\
1.08	0.0128361328661109	0.159710674111862	0.741368418705642\\
1.08	0.0153495058140643	0.174537687332543	0.745784862958201\\
1.08	0.0180991103316243	0.189395466863524	0.750625176092311\\
1.08	0.0210872565164405	0.204278830634725	0.755910489868998\\
1.08	0.0243161165387281	0.219182480034174	0.761650724664453\\
1.08	0.0277877186778607	0.234101003941464	0.767842183867616\\
1.08	0.0315039414701067	0.24902888308801	0.77446537092041\\
1.08	0.0354665079868145	0.263960494742775	0.781483180780606\\
1.08	0.0396769802625738	0.278890117720924	0.788839618268073\\
1.08	0.0441367538930258	0.293811937711588	0.796459186325544\\
1.08	0.0488470528220538	0.308720052919643	0.804247066923454\\
1.08	0.0538089243380495	0.323608480015097	0.81209018616195\\
1.08	0.0590232342988274	0.338471160382329	0.819859213932692\\
1.08	0.064490662604533	0.3533019666601	0.827411499063639\\
1.08	0.0702116989375697	0.368094709561873	0.834594885813505\\
1.08	0.0761866387881432	0.382843144964686	0.84125230028718\\
1.08	0.0824155797834956	0.397540981253432	0.847226939729082\\
1.08	0.0888984183382709	0.412181886906127	0.85236784790504\\
1.08	0.0956348466427212	0.42675949830445	0.85653562002783\\
1.08	0.102624350004627	0.441267427752584	0.859607954619617\\
1.08	0.109866204559871	0.455699271686218	0.861484760266618\\
1.08	0.117359475365564	0.470048619052377	0.862092534251539\\
1.08	0.125103014888515	0.484309059839721	0.861387758068367\\
1.08	0.133095461900593	0.498474193737904	0.85935910089082\\
1.08	0.14133524079124	0.512537638903683	0.856028283765739\\
1.08	0.149820561306021	0.526493040810599	0.851449530870185\\
1.08	0.158549418718625	0.540334081158348	0.845707614739501\\
1.08	0.167519594442214	0.554054486817265	0.838914584343151\\
1.08	0.176728657084455	0.567648038782867	0.831205342370834\\
1.08	0.186173963948925	0.581108581114919	0.822732305417528\\
1.08	0.195852662983903	0.594430029835236	0.813659432944303\\
1.08	0.205761695177907	0.60760638175823	0.804155944104874\\
1.08	0.215897797399558	0.620631723228144	0.794390053406972\\
1.08	0.226257505677663	0.633500238737009	0.784523046062799\\
1.08	0.236837158915675	0.646206219397571	0.774703982861587\\
1.08	0.247632903032949	0.658744071245709	0.765065275211012\\
1.08	0.258640695523528	0.671108323347402	0.755719307772965\\
1.08	0.269856310421543	0.683293635685828	0.746756214005913\\
1.08	0.28127534366066	0.695294806804924	0.738242834604293\\
1.08	0.292893218813452	0.707106781186547	0.730222816007837\\
1.08	0.304705193195075	0.71872465633934	0.722717741087768\\
1.08	0.316706364314172	0.730143689578457	0.715729131143269\\
1.08	0.328891676652598	0.741359304476472	0.709241120561857\\
1.08	0.341255928754291	0.752367096967051	0.703223584571265\\
1.08	0.353793780602429	0.763162841084325	0.697635496597418\\
1.08	0.366499761262991	0.773742494322337	0.692428303594618\\
1.08	0.379368276771857	0.784102202600443	0.687549132887068\\
1.08	0.392393618241769	0.794238304822092	0.682943679235329\\
1.08	0.405569970164763	0.804147337016097	0.678558662193046\\
1.08	0.418891418885081	0.813826036051075	0.67434378741287\\
1.08	0.432351961217132	0.823271342915544	0.67025318771555\\
1.08	0.445945513182735	0.832480405557786	0.666246357336159\\
1.08	0.459665918841652	0.841450581281375	0.662288623474187\\
1.08	0.473506959189401	0.850179438693979	0.658351221673191\\
1.08	0.487462361096317	0.85866475920876	0.654411055137885\\
1.08	0.501525806262096	0.866904538099407	0.650450223206468\\
1.08	0.515690940160279	0.874896985111485	0.64645540187496\\
1.08	0.529951380947623	0.882640524634437	0.642417151055155\\
1.08	0.544300728313782	0.890133795440129	0.638329210947585\\
1.08	0.558732572247415	0.897375649995373	0.634187835387498\\
1.08	0.57324050169555	0.904365153357279	0.629991194998377\\
1.08	0.587818113093873	0.911101581661729	0.625738868900924\\
1.08	0.602459018746568	0.917584420216504	0.62143143163314\\
1.08	0.617156855035314	0.923813361211857	0.617070132480554\\
1.08	0.631905290438127	0.92978830106243	0.612656657833836\\
1.08	0.6466980333399	0.935509337395467	0.608192963377088\\
1.08	0.661528839617671	0.940976765701173	0.603681161495106\\
1.08	0.676391519984903	0.94619107566195	0.599123449738606\\
1.08	0.691279947080357	0.951152947177946	0.59452206790159\\
1.08	0.706188062288412	0.955863246106974	0.589879273662389\\
1.08	0.721109882279076	0.960323019737426	0.585197329319939\\
1.08	0.736039505257226	0.964533492013186	0.580478494543976\\
1.08	0.75097111691199	0.968496058529893	0.57572502201831\\
1.08	0.765898996058536	0.972212281322139	0.570939154289562\\
1.08	0.780817519965826	0.975683883461272	0.566123121053174\\
1.08	0.795721169365275	0.978912743483559	0.561279136605556\\
1.08	0.810604533136476	0.981900889668376	0.5564093973992\\
1.08	0.825462312667457	0.984650494185936	0.55151607969574\\
1.08	0.840289325888138	0.987163867133889	0.546601337319642\\
1.08	0.855080510976838	0.989443450481767	0.541667299515136\\
1.08	0.86983092974082	0.991491811941914	0.536716068908598\\
1.08	0.884535770672926	0.993311638785082	0.531749719578145\\
1.08	0.899190351687411	0.994905731618374	0.526770295231795\\
1.08	0.913790122539012	0.996276998142666	0.521779807495099\\
1.08	0.928330666930242	0.997428446906011	0.51678023430868\\
1.08	0.942807704312699	0.998363181068901	0.511773518435872\\
1.08	0.95721709138901	0.999084392196567	0.50676156608017\\
1.08	0.971554823322703	0.999595354092743	0.501746245611874\\
1.08	0.985817034663989	0.999899416688637	0.496729386403005\\
1.08	1	1	0.491669087346904\\
1.095	0	0	0.710195569525497\\
1.095	0.000100583311362513	0.0141829653360114	0.71240460418299\\
1.095	0.000404645907256436	0.0284451766772965	0.714630690536905\\
1.095	0.000915607803432999	0.0427829086109896	0.716916312064106\\
1.095	0.00163681893109844	0.057192295687301	0.719289019360234\\
1.095	0.00257155309398959	0.0716693330697584	0.721780367061016\\
1.095	0.00372300185733414	0.0862098774609879	0.724425655492835\\
1.095	0.00509426838162598	0.100809648312589	0.727263413850304\\
1.095	0.00668836121491815	0.115464229327074	0.73033458997498\\
1.095	0.00850818805808554	0.13016907025918	0.733681422514709\\
1.095	0.0105565495182326	0.144919489023162	0.737345986077549\\
1.095	0.0128361328661109	0.159710674111862	0.741368418705642\\
1.095	0.0153495058140643	0.174537687332543	0.745784862958201\\
1.095	0.0180991103316243	0.189395466863524	0.750625176092311\\
1.095	0.0210872565164405	0.204278830634725	0.755910489868999\\
1.095	0.0243161165387281	0.219182480034174	0.761650724664453\\
1.095	0.0277877186778607	0.234101003941464	0.767842183867615\\
1.095	0.0315039414701067	0.24902888308801	0.77446537092041\\
1.095	0.0354665079868145	0.263960494742775	0.781483180780607\\
1.095	0.0396769802625738	0.278890117720924	0.788839618268073\\
1.095	0.0441367538930258	0.293811937711588	0.796459186325544\\
1.095	0.0488470528220537	0.308720052919643	0.804247066923455\\
1.095	0.0538089243380495	0.323608480015096	0.81209018616195\\
1.095	0.0590232342988274	0.338471160382329	0.819859213932692\\
1.095	0.064490662604533	0.3533019666601	0.827411499063638\\
1.095	0.0702116989375697	0.368094709561873	0.834594885813504\\
1.095	0.0761866387881432	0.382843144964686	0.841252300287179\\
1.095	0.0824155797834956	0.397540981253432	0.847226939729081\\
1.095	0.0888984183382709	0.412181886906127	0.852367847905042\\
1.095	0.0956348466427212	0.42675949830445	0.85653562002783\\
1.095	0.102624350004627	0.441267427752584	0.859607954619616\\
1.095	0.109866204559871	0.455699271686218	0.861484760266624\\
1.095	0.117359475365564	0.470048619052377	0.862092534251541\\
1.095	0.125103014888515	0.484309059839721	0.861387758068367\\
1.095	0.133095461900593	0.498474193737904	0.859359100890822\\
1.095	0.14133524079124	0.512537638903683	0.856028283765735\\
1.095	0.149820561306021	0.526493040810599	0.851449530870181\\
1.095	0.158549418718625	0.540334081158348	0.845707614739507\\
1.095	0.167519594442213	0.554054486817265	0.838914584343152\\
1.095	0.176728657084455	0.567648038782867	0.83120534237084\\
1.095	0.186173963948925	0.581108581114919	0.822732305417531\\
1.095	0.195852662983903	0.594430029835236	0.81365943294431\\
1.095	0.205761695177907	0.607606381758231	0.804155944104872\\
1.095	0.215897797399558	0.620631723228143	0.794390053406965\\
1.095	0.226257505677663	0.633500238737009	0.7845230460628\\
1.095	0.236837158915675	0.646206219397571	0.774703982861589\\
1.095	0.247632903032949	0.658744071245709	0.765065275211019\\
1.095	0.258640695523528	0.671108323347402	0.755719307772966\\
1.095	0.269856310421543	0.683293635685828	0.746756214005909\\
1.095	0.28127534366066	0.695294806804924	0.738242834604296\\
1.095	0.292893218813452	0.707106781186548	0.730222816007836\\
1.095	0.304705193195075	0.71872465633934	0.722717741087766\\
1.095	0.316706364314172	0.730143689578457	0.715729131143268\\
1.095	0.328891676652598	0.741359304476472	0.709241120561854\\
1.095	0.341255928754291	0.752367096967051	0.703223584571265\\
1.095	0.353793780602429	0.763162841084325	0.697635496597419\\
1.095	0.366499761262991	0.773742494322337	0.692428303594617\\
1.095	0.379368276771857	0.784102202600442	0.68754913288707\\
1.095	0.392393618241769	0.794238304822092	0.682943679235329\\
1.095	0.405569970164763	0.804147337016097	0.678558662193045\\
1.095	0.418891418885081	0.813826036051075	0.674343787412872\\
1.095	0.432351961217132	0.823271342915545	0.670253187715551\\
1.095	0.445945513182735	0.832480405557786	0.666246357336158\\
1.095	0.459665918841652	0.841450581281375	0.662288623474187\\
1.095	0.473506959189401	0.850179438693979	0.658351221673189\\
1.095	0.487462361096317	0.85866475920876	0.654411055137886\\
1.095	0.501525806262096	0.866904538099407	0.650450223206469\\
1.095	0.515690940160279	0.874896985111485	0.646455401874958\\
1.095	0.529951380947623	0.882640524634436	0.642417151055156\\
1.095	0.544300728313781	0.890133795440129	0.638329210947586\\
1.095	0.558732572247415	0.897375649995372	0.634187835387497\\
1.095	0.57324050169555	0.904365153357279	0.629991194998376\\
1.095	0.587818113093873	0.911101581661729	0.625738868900925\\
1.095	0.602459018746568	0.917584420216504	0.621431431633141\\
1.095	0.617156855035314	0.923813361211857	0.617070132480554\\
1.095	0.631905290438127	0.92978830106243	0.612656657833837\\
1.095	0.6466980333399	0.935509337395467	0.608192963377087\\
1.095	0.661528839617671	0.940976765701173	0.603681161495106\\
1.095	0.676391519984904	0.94619107566195	0.599123449738607\\
1.095	0.691279947080357	0.951152947177946	0.59452206790159\\
1.095	0.706188062288412	0.955863246106974	0.589879273662389\\
1.095	0.721109882279076	0.960323019737426	0.585197329319937\\
1.095	0.736039505257226	0.964533492013186	0.580478494543978\\
1.095	0.75097111691199	0.968496058529893	0.575725022018312\\
1.095	0.765898996058536	0.972212281322139	0.570939154289557\\
1.095	0.780817519965825	0.975683883461272	0.566123121053175\\
1.095	0.795721169365275	0.978912743483559	0.561279136605561\\
1.095	0.810604533136476	0.981900889668376	0.556409397399199\\
1.095	0.825462312667457	0.984650494185936	0.551516079695735\\
1.095	0.840289325888138	0.987163867133889	0.546601337319643\\
1.095	0.855080510976839	0.989443450481768	0.541667299515141\\
1.095	0.86983092974082	0.991491811941915	0.536716068908599\\
1.095	0.884535770672926	0.993311638785082	0.531749719578142\\
1.095	0.899190351687411	0.994905731618374	0.526770295231793\\
1.095	0.913790122539012	0.996276998142666	0.521779807495101\\
1.095	0.928330666930241	0.99742844690601	0.516780234308683\\
1.095	0.942807704312699	0.998363181068902	0.511773518435873\\
1.095	0.95721709138901	0.999084392196567	0.506761566080171\\
1.095	0.971554823322703	0.999595354092743	0.501746245611874\\
1.095	0.985817034663989	0.999899416688637	0.496729386403003\\
1.095	1	1	0.491669087346906\\
1.11	0	0	0.710195569525497\\
1.11	0.000100583311362513	0.0141829653360114	0.71240460418299\\
1.11	0.000404645907256436	0.0284451766772965	0.714630690536905\\
1.11	0.000915607803432999	0.0427829086109896	0.716916312064106\\
1.11	0.00163681893109844	0.057192295687301	0.719289019360234\\
1.11	0.00257155309398959	0.0716693330697584	0.721780367061016\\
1.11	0.00372300185733413	0.0862098774609879	0.724425655492835\\
1.11	0.00509426838162598	0.100809648312589	0.727263413850304\\
1.11	0.00668836121491816	0.115464229327074	0.73033458997498\\
1.11	0.00850818805808554	0.13016907025918	0.733681422514709\\
1.11	0.0105565495182326	0.144919489023162	0.737345986077549\\
1.11	0.0128361328661109	0.159710674111862	0.741368418705642\\
1.11	0.0153495058140643	0.174537687332543	0.745784862958202\\
1.11	0.0180991103316243	0.189395466863524	0.750625176092311\\
1.11	0.0210872565164405	0.204278830634725	0.755910489868999\\
1.11	0.0243161165387281	0.219182480034174	0.761650724664453\\
1.11	0.0277877186778607	0.234101003941464	0.767842183867616\\
1.11	0.0315039414701067	0.24902888308801	0.77446537092041\\
1.11	0.0354665079868145	0.263960494742775	0.781483180780606\\
1.11	0.0396769802625738	0.278890117720924	0.788839618268073\\
1.11	0.0441367538930258	0.293811937711588	0.796459186325544\\
1.11	0.0488470528220538	0.308720052919643	0.804247066923455\\
1.11	0.0538089243380495	0.323608480015097	0.812090186161951\\
1.11	0.0590232342988274	0.338471160382329	0.819859213932694\\
1.11	0.064490662604533	0.3533019666601	0.82741149906364\\
1.11	0.0702116989375697	0.368094709561873	0.834594885813505\\
1.11	0.0761866387881432	0.382843144964686	0.84125230028718\\
1.11	0.0824155797834956	0.397540981253432	0.847226939729082\\
1.11	0.0888984183382709	0.412181886906127	0.85236784790504\\
1.11	0.0956348466427212	0.42675949830445	0.856535620027829\\
1.11	0.102624350004627	0.441267427752584	0.859607954619618\\
1.11	0.109866204559871	0.455699271686218	0.861484760266624\\
1.11	0.117359475365564	0.470048619052377	0.862092534251543\\
1.11	0.125103014888515	0.484309059839721	0.861387758068371\\
1.11	0.133095461900593	0.498474193737904	0.85935910089082\\
1.11	0.14133524079124	0.512537638903683	0.856028283765735\\
1.11	0.149820561306021	0.526493040810599	0.851449530870183\\
1.11	0.158549418718625	0.540334081158348	0.845707614739508\\
1.11	0.167519594442213	0.554054486817265	0.838914584343154\\
1.11	0.176728657084455	0.567648038782867	0.831205342370843\\
1.11	0.186173963948925	0.581108581114919	0.822732305417524\\
1.11	0.195852662983903	0.594430029835237	0.813659432944307\\
1.11	0.205761695177907	0.607606381758231	0.804155944104868\\
1.11	0.215897797399558	0.620631723228143	0.794390053406971\\
1.11	0.226257505677663	0.633500238737009	0.784523046062805\\
1.11	0.236837158915675	0.646206219397571	0.774703982861591\\
1.11	0.247632903032949	0.658744071245709	0.765065275211016\\
1.11	0.258640695523527	0.671108323347402	0.755719307772961\\
1.11	0.269856310421543	0.683293635685828	0.746756214005909\\
1.11	0.28127534366066	0.695294806804924	0.738242834604295\\
1.11	0.292893218813452	0.707106781186547	0.730222816007835\\
1.11	0.304705193195075	0.71872465633934	0.722717741087769\\
1.11	0.316706364314172	0.730143689578457	0.715729131143268\\
1.11	0.328891676652598	0.741359304476472	0.709241120561855\\
1.11	0.341255928754291	0.752367096967051	0.703223584571264\\
1.11	0.353793780602429	0.763162841084324	0.697635496597419\\
1.11	0.366499761262991	0.773742494322338	0.692428303594619\\
1.11	0.379368276771857	0.784102202600443	0.687549132887068\\
1.11	0.392393618241769	0.794238304822092	0.682943679235329\\
1.11	0.405569970164763	0.804147337016097	0.678558662193045\\
1.11	0.418891418885081	0.813826036051075	0.67434378741287\\
1.11	0.432351961217132	0.823271342915544	0.670253187715552\\
1.11	0.445945513182735	0.832480405557786	0.66624635733616\\
1.11	0.459665918841652	0.841450581281375	0.662288623474188\\
1.11	0.473506959189401	0.850179438693979	0.65835122167319\\
1.11	0.487462361096317	0.85866475920876	0.654411055137886\\
1.11	0.501525806262096	0.866904538099407	0.650450223206469\\
1.11	0.515690940160279	0.874896985111485	0.646455401874958\\
1.11	0.529951380947623	0.882640524634436	0.642417151055155\\
1.11	0.544300728313782	0.890133795440129	0.638329210947587\\
1.11	0.558732572247415	0.897375649995373	0.634187835387497\\
1.11	0.57324050169555	0.904365153357279	0.629991194998375\\
1.11	0.587818113093873	0.911101581661729	0.625738868900925\\
1.11	0.602459018746568	0.917584420216504	0.62143143163314\\
1.11	0.617156855035314	0.923813361211857	0.617070132480552\\
1.11	0.631905290438127	0.92978830106243	0.612656657833838\\
1.11	0.6466980333399	0.935509337395467	0.608192963377088\\
1.11	0.661528839617671	0.940976765701173	0.603681161495105\\
1.11	0.676391519984903	0.94619107566195	0.599123449738607\\
1.11	0.691279947080357	0.951152947177946	0.594522067901589\\
1.11	0.706188062288412	0.955863246106974	0.58987927366239\\
1.11	0.721109882279076	0.960323019737426	0.585197329319938\\
1.11	0.736039505257226	0.964533492013186	0.580478494543978\\
1.11	0.75097111691199	0.968496058529893	0.575725022018313\\
1.11	0.765898996058536	0.972212281322139	0.570939154289558\\
1.11	0.780817519965826	0.975683883461272	0.566123121053173\\
1.11	0.795721169365275	0.978912743483559	0.56127913660556\\
1.11	0.810604533136476	0.981900889668376	0.556409397399199\\
1.11	0.825462312667457	0.984650494185936	0.551516079695735\\
1.11	0.840289325888138	0.987163867133889	0.546601337319642\\
1.11	0.855080510976839	0.989443450481768	0.541667299515138\\
1.11	0.86983092974082	0.991491811941914	0.536716068908598\\
1.11	0.884535770672926	0.993311638785082	0.531749719578144\\
1.11	0.899190351687411	0.994905731618374	0.526770295231796\\
1.11	0.913790122539012	0.996276998142666	0.521779807495101\\
1.11	0.928330666930242	0.997428446906011	0.516780234308683\\
1.11	0.942807704312699	0.998363181068902	0.511773518435868\\
1.11	0.95721709138901	0.999084392196567	0.506761566080168\\
1.11	0.971554823322703	0.999595354092743	0.50174624561188\\
1.11	0.985817034663989	0.999899416688638	0.496729386403004\\
1.11	1	1	0.491669087346897\\
1.125	0	0	0.710195569525497\\
1.125	0.000100583311362513	0.0141829653360114	0.71240460418299\\
1.125	0.000404645907256436	0.0284451766772965	0.714630690536905\\
1.125	0.000915607803432999	0.0427829086109896	0.716916312064106\\
1.125	0.00163681893109844	0.057192295687301	0.719289019360234\\
1.125	0.00257155309398959	0.0716693330697584	0.721780367061016\\
1.125	0.00372300185733413	0.0862098774609879	0.724425655492835\\
1.125	0.00509426838162598	0.100809648312589	0.727263413850304\\
1.125	0.00668836121491816	0.115464229327074	0.73033458997498\\
1.125	0.00850818805808555	0.13016907025918	0.733681422514709\\
1.125	0.0105565495182326	0.144919489023162	0.737345986077549\\
1.125	0.0128361328661109	0.159710674111862	0.741368418705642\\
1.125	0.0153495058140643	0.174537687332543	0.745784862958202\\
1.125	0.0180991103316243	0.189395466863524	0.750625176092311\\
1.125	0.0210872565164405	0.204278830634725	0.755910489868999\\
1.125	0.0243161165387281	0.219182480034174	0.761650724664453\\
1.125	0.0277877186778607	0.234101003941464	0.767842183867616\\
1.125	0.0315039414701067	0.24902888308801	0.77446537092041\\
1.125	0.0354665079868145	0.263960494742775	0.781483180780606\\
1.125	0.0396769802625738	0.278890117720924	0.788839618268073\\
1.125	0.0441367538930258	0.293811937711588	0.796459186325543\\
1.125	0.0488470528220538	0.308720052919643	0.804247066923454\\
1.125	0.0538089243380495	0.323608480015097	0.812090186161949\\
1.125	0.0590232342988274	0.338471160382329	0.819859213932693\\
1.125	0.064490662604533	0.3533019666601	0.827411499063639\\
1.125	0.0702116989375697	0.368094709561873	0.834594885813505\\
1.125	0.0761866387881432	0.382843144964686	0.841252300287177\\
1.125	0.0824155797834956	0.397540981253432	0.847226939729081\\
1.125	0.0888984183382709	0.412181886906127	0.852367847905042\\
1.125	0.0956348466427212	0.42675949830445	0.856535620027828\\
1.125	0.102624350004627	0.441267427752584	0.859607954619618\\
1.125	0.109866204559871	0.455699271686218	0.861484760266618\\
1.125	0.117359475365564	0.470048619052377	0.862092534251542\\
1.125	0.125103014888515	0.484309059839721	0.861387758068371\\
1.125	0.133095461900593	0.498474193737904	0.85935910089082\\
1.125	0.14133524079124	0.512537638903683	0.856028283765736\\
1.125	0.149820561306021	0.526493040810599	0.851449530870183\\
1.125	0.158549418718625	0.540334081158348	0.845707614739507\\
1.125	0.167519594442214	0.554054486817265	0.838914584343152\\
1.125	0.176728657084455	0.567648038782868	0.831205342370834\\
1.125	0.186173963948925	0.581108581114919	0.822732305417523\\
1.125	0.195852662983903	0.594430029835236	0.813659432944304\\
1.125	0.205761695177907	0.607606381758231	0.804155944104871\\
1.125	0.215897797399558	0.620631723228143	0.794390053406972\\
1.125	0.226257505677663	0.633500238737009	0.784523046062799\\
1.125	0.236837158915675	0.646206219397571	0.774703982861588\\
1.125	0.247632903032949	0.658744071245709	0.765065275211015\\
1.125	0.258640695523527	0.671108323347402	0.755719307772963\\
1.125	0.269856310421543	0.683293635685828	0.746756214005911\\
1.125	0.28127534366066	0.695294806804924	0.738242834604296\\
1.125	0.292893218813452	0.707106781186547	0.730222816007835\\
1.125	0.304705193195075	0.71872465633934	0.722717741087767\\
1.125	0.316706364314172	0.730143689578457	0.715729131143268\\
1.125	0.328891676652598	0.741359304476472	0.709241120561856\\
1.125	0.341255928754291	0.752367096967051	0.703223584571264\\
1.125	0.353793780602429	0.763162841084324	0.697635496597419\\
1.125	0.366499761262991	0.773742494322337	0.69242830359462\\
1.125	0.379368276771857	0.784102202600443	0.687549132887069\\
1.125	0.392393618241769	0.794238304822092	0.682943679235329\\
1.125	0.405569970164763	0.804147337016097	0.678558662193045\\
1.125	0.418891418885081	0.813826036051075	0.67434378741287\\
1.125	0.432351961217132	0.823271342915544	0.670253187715552\\
1.125	0.445945513182735	0.832480405557786	0.666246357336158\\
1.125	0.459665918841652	0.841450581281375	0.662288623474187\\
1.125	0.473506959189401	0.850179438693979	0.658351221673191\\
1.125	0.487462361096317	0.85866475920876	0.654411055137886\\
1.125	0.501525806262096	0.866904538099407	0.650450223206469\\
1.125	0.515690940160279	0.874896985111485	0.646455401874958\\
1.125	0.529951380947623	0.882640524634436	0.642417151055155\\
1.125	0.544300728313782	0.890133795440129	0.638329210947586\\
1.125	0.558732572247415	0.897375649995373	0.634187835387498\\
1.125	0.57324050169555	0.904365153357279	0.629991194998376\\
1.125	0.587818113093873	0.911101581661729	0.625738868900924\\
1.125	0.602459018746568	0.917584420216504	0.621431431633139\\
1.125	0.617156855035314	0.923813361211857	0.617070132480553\\
1.125	0.631905290438127	0.92978830106243	0.612656657833839\\
1.125	0.6466980333399	0.935509337395467	0.608192963377087\\
1.125	0.661528839617671	0.940976765701173	0.603681161495104\\
1.125	0.676391519984903	0.94619107566195	0.599123449738607\\
1.125	0.691279947080357	0.951152947177946	0.594522067901588\\
1.125	0.706188062288412	0.955863246106974	0.58987927366239\\
1.125	0.721109882279076	0.960323019737426	0.585197329319937\\
1.125	0.736039505257226	0.964533492013186	0.580478494543973\\
1.125	0.75097111691199	0.968496058529893	0.575725022018312\\
1.125	0.765898996058536	0.972212281322139	0.570939154289562\\
1.125	0.780817519965826	0.975683883461272	0.566123121053173\\
1.125	0.795721169365275	0.978912743483559	0.561279136605556\\
1.125	0.810604533136476	0.981900889668376	0.556409397399202\\
1.125	0.825462312667457	0.984650494185936	0.551516079695737\\
1.125	0.840289325888138	0.987163867133889	0.546601337319641\\
1.125	0.855080510976839	0.989443450481767	0.541667299515138\\
1.125	0.86983092974082	0.991491811941914	0.536716068908599\\
1.125	0.884535770672926	0.993311638785082	0.531749719578142\\
1.125	0.899190351687411	0.994905731618374	0.526770295231793\\
1.125	0.913790122539012	0.996276998142666	0.521779807495103\\
1.125	0.928330666930242	0.997428446906011	0.516780234308685\\
1.125	0.942807704312699	0.998363181068902	0.511773518435873\\
1.125	0.95721709138901	0.999084392196567	0.506761566080167\\
1.125	0.971554823322703	0.999595354092743	0.501746245611875\\
1.125	0.985817034663989	0.999899416688637	0.496729386403004\\
1.125	1	1	0.491669087346906\\
1.14	0	0	0.710195569525497\\
1.14	0.000100583311362513	0.0141829653360114	0.71240460418299\\
1.14	0.000404645907256436	0.0284451766772965	0.714630690536905\\
1.14	0.000915607803432999	0.0427829086109896	0.716916312064106\\
1.14	0.00163681893109844	0.057192295687301	0.719289019360234\\
1.14	0.00257155309398959	0.0716693330697584	0.721780367061016\\
1.14	0.00372300185733414	0.0862098774609879	0.724425655492835\\
1.14	0.00509426838162598	0.100809648312589	0.727263413850304\\
1.14	0.00668836121491816	0.115464229327074	0.73033458997498\\
1.14	0.00850818805808554	0.13016907025918	0.733681422514709\\
1.14	0.0105565495182326	0.144919489023162	0.737345986077549\\
1.14	0.0128361328661109	0.159710674111862	0.741368418705642\\
1.14	0.0153495058140643	0.174537687332543	0.745784862958202\\
1.14	0.0180991103316243	0.189395466863524	0.750625176092311\\
1.14	0.0210872565164405	0.204278830634725	0.755910489868999\\
1.14	0.0243161165387281	0.219182480034174	0.761650724664453\\
1.14	0.0277877186778607	0.234101003941464	0.767842183867616\\
1.14	0.0315039414701067	0.24902888308801	0.77446537092041\\
1.14	0.0354665079868145	0.263960494742775	0.781483180780606\\
1.14	0.0396769802625737	0.278890117720924	0.788839618268074\\
1.14	0.0441367538930258	0.293811937711588	0.796459186325545\\
1.14	0.0488470528220537	0.308720052919643	0.804247066923455\\
1.14	0.0538089243380495	0.323608480015096	0.81209018616195\\
1.14	0.0590232342988274	0.338471160382329	0.819859213932693\\
1.14	0.064490662604533	0.3533019666601	0.827411499063638\\
1.14	0.0702116989375697	0.368094709561873	0.834594885813505\\
1.14	0.0761866387881432	0.382843144964686	0.84125230028718\\
1.14	0.0824155797834956	0.397540981253432	0.847226939729079\\
1.14	0.0888984183382709	0.412181886906127	0.852367847905041\\
1.14	0.0956348466427212	0.42675949830445	0.856535620027831\\
1.14	0.102624350004627	0.441267427752584	0.859607954619618\\
1.14	0.109866204559871	0.455699271686218	0.861484760266623\\
1.14	0.117359475365564	0.470048619052377	0.862092534251538\\
1.14	0.125103014888515	0.484309059839721	0.861387758068369\\
1.14	0.133095461900593	0.498474193737904	0.85935910089082\\
1.14	0.14133524079124	0.512537638903683	0.856028283765735\\
1.14	0.149820561306021	0.526493040810599	0.851449530870182\\
1.14	0.158549418718625	0.540334081158348	0.845707614739505\\
1.14	0.167519594442213	0.554054486817265	0.838914584343146\\
1.14	0.176728657084455	0.567648038782867	0.831205342370832\\
1.14	0.186173963948925	0.581108581114919	0.822732305417527\\
1.14	0.195852662983903	0.594430029835236	0.813659432944309\\
1.14	0.205761695177907	0.607606381758231	0.804155944104872\\
1.14	0.215897797399558	0.620631723228143	0.794390053406971\\
1.14	0.226257505677663	0.633500238737009	0.784523046062802\\
1.14	0.236837158915675	0.646206219397571	0.774703982861589\\
1.14	0.247632903032949	0.658744071245709	0.765065275211017\\
1.14	0.258640695523527	0.671108323347402	0.755719307772963\\
1.14	0.269856310421543	0.683293635685828	0.74675621400591\\
1.14	0.28127534366066	0.695294806804924	0.738242834604295\\
1.14	0.292893218813452	0.707106781186547	0.730222816007835\\
1.14	0.304705193195075	0.71872465633934	0.722717741087767\\
1.14	0.316706364314172	0.730143689578457	0.715729131143268\\
1.14	0.328891676652598	0.741359304476472	0.709241120561856\\
1.14	0.341255928754291	0.752367096967051	0.703223584571265\\
1.14	0.353793780602429	0.763162841084325	0.697635496597417\\
1.14	0.366499761262991	0.773742494322337	0.692428303594619\\
1.14	0.379368276771856	0.784102202600442	0.68754913288707\\
1.14	0.392393618241769	0.794238304822092	0.682943679235327\\
1.14	0.405569970164763	0.804147337016096	0.678558662193046\\
1.14	0.418891418885081	0.813826036051075	0.674343787412871\\
1.14	0.432351961217132	0.823271342915544	0.670253187715551\\
1.14	0.445945513182735	0.832480405557786	0.666246357336159\\
1.14	0.459665918841652	0.841450581281375	0.662288623474186\\
1.14	0.473506959189401	0.850179438693979	0.65835122167319\\
1.14	0.487462361096317	0.85866475920876	0.654411055137885\\
1.14	0.501525806262096	0.866904538099407	0.65045022320647\\
1.14	0.515690940160279	0.874896985111485	0.646455401874958\\
1.14	0.529951380947623	0.882640524634436	0.642417151055154\\
1.14	0.544300728313782	0.890133795440129	0.638329210947587\\
1.14	0.558732572247415	0.897375649995373	0.634187835387497\\
1.14	0.57324050169555	0.904365153357279	0.629991194998376\\
1.14	0.587818113093873	0.911101581661729	0.625738868900925\\
1.14	0.602459018746568	0.917584420216504	0.621431431633141\\
1.14	0.617156855035314	0.923813361211857	0.617070132480552\\
1.14	0.631905290438127	0.92978830106243	0.612656657833838\\
1.14	0.6466980333399	0.935509337395467	0.60819296337709\\
1.14	0.661528839617671	0.940976765701173	0.603681161495103\\
1.14	0.676391519984903	0.94619107566195	0.599123449738608\\
1.14	0.691279947080357	0.951152947177946	0.594522067901589\\
1.14	0.706188062288412	0.955863246106974	0.589879273662389\\
1.14	0.721109882279076	0.960323019737426	0.585197329319938\\
1.14	0.736039505257226	0.964533492013186	0.580478494543976\\
1.14	0.75097111691199	0.968496058529893	0.575725022018313\\
1.14	0.765898996058536	0.972212281322139	0.570939154289558\\
1.14	0.780817519965826	0.975683883461272	0.566123121053175\\
1.14	0.795721169365275	0.97891274348356	0.561279136605558\\
1.14	0.810604533136476	0.981900889668376	0.556409397399197\\
1.14	0.825462312667457	0.984650494185936	0.551516079695735\\
1.14	0.840289325888138	0.987163867133889	0.546601337319641\\
1.14	0.855080510976838	0.989443450481767	0.541667299515139\\
1.14	0.86983092974082	0.991491811941914	0.5367160689086\\
1.14	0.884535770672926	0.993311638785082	0.531749719578143\\
1.14	0.899190351687411	0.994905731618374	0.526770295231793\\
1.14	0.913790122539012	0.996276998142666	0.5217798074951\\
1.14	0.928330666930242	0.99742844690601	0.516780234308683\\
1.14	0.942807704312699	0.998363181068902	0.511773518435874\\
1.14	0.95721709138901	0.999084392196567	0.506761566080168\\
1.14	0.971554823322703	0.999595354092743	0.501746245611874\\
1.14	0.985817034663988	0.999899416688637	0.496729386403005\\
1.14	1	1	0.491669087346904\\
1.155	0	0	0.710195569525497\\
1.155	0.000100583311362513	0.0141829653360114	0.71240460418299\\
1.155	0.000404645907256436	0.0284451766772965	0.714630690536905\\
1.155	0.000915607803432999	0.0427829086109896	0.716916312064106\\
1.155	0.00163681893109844	0.057192295687301	0.719289019360234\\
1.155	0.00257155309398959	0.0716693330697584	0.721780367061016\\
1.155	0.00372300185733413	0.0862098774609879	0.724425655492835\\
1.155	0.00509426838162598	0.100809648312589	0.727263413850304\\
1.155	0.00668836121491816	0.115464229327074	0.73033458997498\\
1.155	0.00850818805808555	0.13016907025918	0.733681422514709\\
1.155	0.0105565495182326	0.144919489023162	0.737345986077549\\
1.155	0.0128361328661109	0.159710674111862	0.741368418705642\\
1.155	0.0153495058140643	0.174537687332543	0.745784862958202\\
1.155	0.0180991103316243	0.189395466863524	0.750625176092311\\
1.155	0.0210872565164405	0.204278830634725	0.755910489868998\\
1.155	0.0243161165387281	0.219182480034174	0.761650724664453\\
1.155	0.0277877186778607	0.234101003941464	0.767842183867616\\
1.155	0.0315039414701067	0.24902888308801	0.77446537092041\\
1.155	0.0354665079868145	0.263960494742775	0.781483180780606\\
1.155	0.0396769802625738	0.278890117720924	0.788839618268073\\
1.155	0.0441367538930258	0.293811937711588	0.796459186325544\\
1.155	0.0488470528220538	0.308720052919643	0.804247066923455\\
1.155	0.0538089243380495	0.323608480015097	0.812090186161951\\
1.155	0.0590232342988274	0.338471160382329	0.819859213932694\\
1.155	0.064490662604533	0.3533019666601	0.827411499063639\\
1.155	0.0702116989375697	0.368094709561873	0.834594885813504\\
1.155	0.0761866387881432	0.382843144964686	0.841252300287179\\
1.155	0.0824155797834956	0.397540981253432	0.847226939729081\\
1.155	0.0888984183382709	0.412181886906127	0.852367847905039\\
1.155	0.0956348466427212	0.42675949830445	0.856535620027829\\
1.155	0.102624350004627	0.441267427752585	0.859607954619618\\
1.155	0.109866204559871	0.455699271686218	0.861484760266623\\
1.155	0.117359475365564	0.470048619052377	0.862092534251536\\
1.155	0.125103014888515	0.484309059839721	0.861387758068369\\
1.155	0.133095461900593	0.498474193737904	0.85935910089082\\
1.155	0.14133524079124	0.512537638903683	0.856028283765735\\
1.155	0.149820561306021	0.526493040810599	0.851449530870183\\
1.155	0.158549418718625	0.540334081158348	0.845707614739502\\
1.155	0.167519594442213	0.554054486817265	0.838914584343149\\
1.155	0.176728657084455	0.567648038782867	0.831205342370841\\
1.155	0.186173963948925	0.581108581114919	0.822732305417528\\
1.155	0.195852662983903	0.594430029835237	0.81365943294431\\
1.155	0.205761695177907	0.607606381758231	0.804155944104869\\
1.155	0.215897797399558	0.620631723228144	0.794390053406969\\
1.155	0.226257505677663	0.633500238737009	0.784523046062798\\
1.155	0.236837158915675	0.646206219397571	0.774703982861588\\
1.155	0.247632903032949	0.658744071245709	0.765065275211017\\
1.155	0.258640695523527	0.671108323347402	0.755719307772961\\
1.155	0.269856310421543	0.683293635685828	0.74675621400591\\
1.155	0.28127534366066	0.695294806804924	0.738242834604294\\
1.155	0.292893218813452	0.707106781186547	0.730222816007835\\
1.155	0.304705193195075	0.71872465633934	0.722717741087768\\
1.155	0.316706364314172	0.730143689578457	0.715729131143268\\
1.155	0.328891676652598	0.741359304476472	0.709241120561856\\
1.155	0.341255928754291	0.752367096967051	0.703223584571264\\
1.155	0.353793780602429	0.763162841084324	0.697635496597418\\
1.155	0.366499761262991	0.773742494322337	0.692428303594618\\
1.155	0.379368276771857	0.784102202600442	0.687549132887069\\
1.155	0.392393618241769	0.794238304822092	0.68294367923533\\
1.155	0.405569970164763	0.804147337016097	0.678558662193044\\
1.155	0.418891418885081	0.813826036051075	0.67434378741287\\
1.155	0.432351961217132	0.823271342915544	0.670253187715552\\
1.155	0.445945513182735	0.832480405557786	0.66624635733616\\
1.155	0.459665918841652	0.841450581281375	0.662288623474186\\
1.155	0.473506959189401	0.850179438693979	0.65835122167319\\
1.155	0.487462361096317	0.85866475920876	0.654411055137885\\
1.155	0.501525806262096	0.866904538099407	0.650450223206469\\
1.155	0.515690940160279	0.874896985111485	0.646455401874959\\
1.155	0.529951380947623	0.882640524634436	0.642417151055154\\
1.155	0.544300728313782	0.890133795440129	0.638329210947586\\
1.155	0.558732572247415	0.897375649995373	0.634187835387497\\
1.155	0.57324050169555	0.904365153357279	0.629991194998375\\
1.155	0.587818113093873	0.911101581661729	0.625738868900925\\
1.155	0.602459018746568	0.917584420216504	0.621431431633141\\
1.155	0.617156855035314	0.923813361211857	0.617070132480554\\
1.155	0.631905290438127	0.92978830106243	0.612656657833834\\
1.155	0.6466980333399	0.935509337395467	0.608192963377087\\
1.155	0.661528839617671	0.940976765701173	0.603681161495107\\
1.155	0.676391519984904	0.94619107566195	0.599123449738607\\
1.155	0.691279947080357	0.951152947177946	0.594522067901592\\
1.155	0.706188062288412	0.955863246106974	0.589879273662389\\
1.155	0.721109882279076	0.960323019737426	0.585197329319935\\
1.155	0.736039505257226	0.964533492013186	0.580478494543976\\
1.155	0.75097111691199	0.968496058529893	0.575725022018313\\
1.155	0.765898996058536	0.972212281322139	0.570939154289561\\
1.155	0.780817519965826	0.975683883461272	0.566123121053174\\
1.155	0.795721169365275	0.978912743483559	0.561279136605558\\
1.155	0.810604533136476	0.981900889668376	0.556409397399202\\
1.155	0.825462312667457	0.984650494185936	0.551516079695737\\
1.155	0.840289325888138	0.987163867133889	0.54660133731964\\
1.155	0.855080510976838	0.989443450481767	0.541667299515138\\
1.155	0.86983092974082	0.991491811941914	0.536716068908599\\
1.155	0.884535770672926	0.993311638785082	0.531749719578143\\
1.155	0.899190351687411	0.994905731618374	0.526770295231796\\
1.155	0.913790122539012	0.996276998142666	0.521779807495103\\
1.155	0.928330666930241	0.997428446906011	0.516780234308679\\
1.155	0.942807704312699	0.998363181068901	0.51177351843587\\
1.155	0.95721709138901	0.999084392196567	0.50676156608017\\
1.155	0.971554823322704	0.999595354092743	0.501746245611876\\
1.155	0.985817034663989	0.999899416688637	0.496729386403004\\
1.155	1	1	0.491669087346906\\
1.17	0	0	0.710195569525497\\
1.17	0.000100583311362513	0.0141829653360114	0.71240460418299\\
1.17	0.000404645907256436	0.0284451766772965	0.714630690536905\\
1.17	0.000915607803432999	0.0427829086109896	0.716916312064106\\
1.17	0.00163681893109844	0.057192295687301	0.719289019360234\\
1.17	0.00257155309398959	0.0716693330697584	0.721780367061016\\
1.17	0.00372300185733413	0.0862098774609879	0.724425655492835\\
1.17	0.00509426838162598	0.100809648312589	0.727263413850304\\
1.17	0.00668836121491816	0.115464229327074	0.73033458997498\\
1.17	0.00850818805808554	0.13016907025918	0.733681422514709\\
1.17	0.0105565495182326	0.144919489023162	0.737345986077549\\
1.17	0.0128361328661109	0.159710674111862	0.741368418705642\\
1.17	0.0153495058140643	0.174537687332543	0.745784862958201\\
1.17	0.0180991103316243	0.189395466863524	0.750625176092311\\
1.17	0.0210872565164405	0.204278830634725	0.755910489868999\\
1.17	0.0243161165387281	0.219182480034174	0.761650724664453\\
1.17	0.0277877186778607	0.234101003941464	0.767842183867616\\
1.17	0.0315039414701067	0.24902888308801	0.77446537092041\\
1.17	0.0354665079868145	0.263960494742775	0.781483180780606\\
1.17	0.0396769802625738	0.278890117720924	0.788839618268073\\
1.17	0.0441367538930258	0.293811937711588	0.796459186325544\\
1.17	0.0488470528220538	0.308720052919643	0.804247066923454\\
1.17	0.0538089243380495	0.323608480015096	0.812090186161949\\
1.17	0.0590232342988274	0.338471160382329	0.819859213932693\\
1.17	0.064490662604533	0.3533019666601	0.827411499063639\\
1.17	0.0702116989375697	0.368094709561873	0.834594885813505\\
1.17	0.0761866387881432	0.382843144964686	0.841252300287179\\
1.17	0.0824155797834956	0.397540981253432	0.847226939729082\\
1.17	0.0888984183382709	0.412181886906127	0.852367847905039\\
1.17	0.0956348466427212	0.42675949830445	0.856535620027827\\
1.17	0.102624350004627	0.441267427752584	0.859607954619617\\
1.17	0.109866204559871	0.455699271686218	0.861484760266621\\
1.17	0.117359475365564	0.470048619052377	0.86209253425154\\
1.17	0.125103014888515	0.484309059839721	0.861387758068368\\
1.17	0.133095461900593	0.498474193737904	0.859359100890822\\
1.17	0.14133524079124	0.512537638903683	0.856028283765737\\
1.17	0.149820561306021	0.526493040810599	0.851449530870184\\
1.17	0.158549418718625	0.540334081158348	0.845707614739502\\
1.17	0.167519594442214	0.554054486817265	0.838914584343155\\
1.17	0.176728657084455	0.567648038782868	0.831205342370836\\
1.17	0.186173963948925	0.581108581114919	0.822732305417527\\
1.17	0.195852662983903	0.594430029835236	0.813659432944304\\
1.17	0.205761695177907	0.607606381758231	0.804155944104868\\
1.17	0.215897797399558	0.620631723228143	0.794390053406966\\
1.17	0.226257505677663	0.633500238737009	0.784523046062799\\
1.17	0.236837158915675	0.646206219397571	0.774703982861592\\
1.17	0.247632903032949	0.658744071245709	0.765065275211015\\
1.17	0.258640695523527	0.671108323347402	0.755719307772963\\
1.17	0.269856310421543	0.683293635685828	0.746756214005909\\
1.17	0.28127534366066	0.695294806804924	0.738242834604296\\
1.17	0.292893218813452	0.707106781186547	0.730222816007836\\
1.17	0.304705193195075	0.71872465633934	0.722717741087768\\
1.17	0.316706364314172	0.730143689578457	0.715729131143268\\
1.17	0.328891676652598	0.741359304476472	0.709241120561856\\
1.17	0.341255928754291	0.752367096967051	0.703223584571265\\
1.17	0.353793780602429	0.763162841084325	0.697635496597419\\
1.17	0.366499761262991	0.773742494322337	0.692428303594617\\
1.17	0.379368276771857	0.784102202600442	0.687549132887068\\
1.17	0.392393618241769	0.794238304822092	0.682943679235329\\
1.17	0.405569970164763	0.804147337016096	0.678558662193045\\
1.17	0.418891418885081	0.813826036051075	0.674343787412871\\
1.17	0.432351961217133	0.823271342915545	0.670253187715552\\
1.17	0.445945513182735	0.832480405557786	0.666246357336158\\
1.17	0.459665918841652	0.841450581281375	0.662288623474186\\
1.17	0.473506959189401	0.850179438693979	0.658351221673192\\
1.17	0.487462361096317	0.85866475920876	0.654411055137886\\
1.17	0.501525806262096	0.866904538099407	0.650450223206468\\
1.17	0.515690940160279	0.874896985111485	0.646455401874959\\
1.17	0.529951380947623	0.882640524634437	0.642417151055155\\
1.17	0.544300728313782	0.890133795440129	0.638329210947586\\
1.17	0.558732572247415	0.897375649995373	0.634187835387497\\
1.17	0.57324050169555	0.904365153357279	0.629991194998374\\
1.17	0.587818113093873	0.911101581661729	0.625738868900924\\
1.17	0.602459018746568	0.917584420216504	0.621431431633141\\
1.17	0.617156855035314	0.923813361211857	0.617070132480554\\
1.17	0.631905290438127	0.92978830106243	0.612656657833839\\
1.17	0.6466980333399	0.935509337395467	0.608192963377087\\
1.17	0.661528839617671	0.940976765701173	0.603681161495103\\
1.17	0.676391519984903	0.94619107566195	0.599123449738606\\
1.17	0.691279947080357	0.951152947177946	0.59452206790159\\
1.17	0.706188062288412	0.955863246106974	0.58987927366239\\
1.17	0.721109882279076	0.960323019737426	0.585197329319937\\
1.17	0.736039505257226	0.964533492013186	0.580478494543975\\
1.17	0.75097111691199	0.968496058529893	0.575725022018313\\
1.17	0.765898996058536	0.972212281322139	0.570939154289561\\
1.17	0.780817519965826	0.975683883461272	0.566123121053174\\
1.17	0.795721169365275	0.978912743483559	0.561279136605559\\
1.17	0.810604533136476	0.981900889668376	0.5564093973992\\
1.17	0.825462312667457	0.984650494185936	0.551516079695735\\
1.17	0.840289325888138	0.987163867133889	0.54660133731964\\
1.17	0.855080510976838	0.989443450481767	0.54166729951514\\
1.17	0.86983092974082	0.991491811941914	0.536716068908601\\
1.17	0.884535770672926	0.993311638785082	0.531749719578144\\
1.17	0.899190351687411	0.994905731618374	0.526770295231793\\
1.17	0.913790122539012	0.996276998142666	0.521779807495102\\
1.17	0.928330666930242	0.997428446906011	0.516780234308683\\
1.17	0.942807704312699	0.998363181068902	0.511773518435872\\
1.17	0.95721709138901	0.999084392196567	0.506761566080167\\
1.17	0.971554823322703	0.999595354092743	0.501746245611874\\
1.17	0.985817034663989	0.999899416688637	0.496729386403005\\
1.17	1	1	0.491669087346904\\
1.185	0	0	0.710195569525497\\
1.185	0.000100583311362513	0.0141829653360114	0.71240460418299\\
1.185	0.000404645907256436	0.0284451766772965	0.714630690536905\\
1.185	0.000915607803432999	0.0427829086109896	0.716916312064106\\
1.185	0.00163681893109844	0.057192295687301	0.719289019360234\\
1.185	0.00257155309398959	0.0716693330697584	0.721780367061016\\
1.185	0.00372300185733413	0.0862098774609879	0.724425655492835\\
1.185	0.00509426838162598	0.100809648312589	0.727263413850304\\
1.185	0.00668836121491816	0.115464229327074	0.73033458997498\\
1.185	0.00850818805808554	0.13016907025918	0.733681422514709\\
1.185	0.0105565495182326	0.144919489023162	0.737345986077549\\
1.185	0.0128361328661109	0.159710674111862	0.741368418705642\\
1.185	0.0153495058140643	0.174537687332543	0.745784862958202\\
1.185	0.0180991103316243	0.189395466863524	0.750625176092311\\
1.185	0.0210872565164405	0.204278830634725	0.755910489868999\\
1.185	0.0243161165387281	0.219182480034174	0.761650724664453\\
1.185	0.0277877186778607	0.234101003941464	0.767842183867616\\
1.185	0.0315039414701067	0.24902888308801	0.77446537092041\\
1.185	0.0354665079868145	0.263960494742775	0.781483180780606\\
1.185	0.0396769802625738	0.278890117720924	0.788839618268073\\
1.185	0.0441367538930258	0.293811937711588	0.796459186325545\\
1.185	0.0488470528220538	0.308720052919643	0.804247066923456\\
1.185	0.0538089243380495	0.323608480015096	0.81209018616195\\
1.185	0.0590232342988274	0.338471160382329	0.819859213932692\\
1.185	0.064490662604533	0.3533019666601	0.827411499063638\\
1.185	0.0702116989375697	0.368094709561873	0.834594885813505\\
1.185	0.0761866387881432	0.382843144964686	0.84125230028718\\
1.185	0.0824155797834956	0.397540981253432	0.847226939729082\\
1.185	0.0888984183382709	0.412181886906127	0.852367847905041\\
1.185	0.0956348466427212	0.42675949830445	0.856535620027828\\
1.185	0.102624350004627	0.441267427752585	0.859607954619617\\
1.185	0.109866204559871	0.455699271686218	0.861484760266622\\
1.185	0.117359475365564	0.470048619052377	0.862092534251542\\
1.185	0.125103014888515	0.484309059839721	0.861387758068369\\
1.185	0.133095461900593	0.498474193737904	0.859359100890821\\
1.185	0.14133524079124	0.512537638903683	0.856028283765739\\
1.185	0.149820561306021	0.526493040810599	0.851449530870181\\
1.185	0.158549418718625	0.540334081158348	0.845707614739508\\
1.185	0.167519594442214	0.554054486817265	0.838914584343152\\
1.185	0.176728657084455	0.567648038782867	0.831205342370835\\
1.185	0.186173963948925	0.581108581114919	0.822732305417527\\
1.185	0.195852662983903	0.594430029835236	0.813659432944308\\
1.185	0.205761695177907	0.607606381758231	0.804155944104871\\
1.185	0.215897797399558	0.620631723228143	0.794390053406971\\
1.185	0.226257505677663	0.633500238737009	0.784523046062803\\
1.185	0.236837158915675	0.646206219397571	0.774703982861591\\
1.185	0.247632903032949	0.658744071245709	0.765065275211016\\
1.185	0.258640695523527	0.671108323347402	0.755719307772963\\
1.185	0.269856310421543	0.683293635685828	0.74675621400591\\
1.185	0.28127534366066	0.695294806804924	0.738242834604296\\
1.185	0.292893218813452	0.707106781186547	0.730222816007835\\
1.185	0.304705193195075	0.71872465633934	0.722717741087767\\
1.185	0.316706364314172	0.730143689578457	0.715729131143268\\
1.185	0.328891676652598	0.741359304476472	0.709241120561855\\
1.185	0.341255928754291	0.752367096967051	0.703223584571264\\
1.185	0.353793780602429	0.763162841084325	0.697635496597419\\
1.185	0.366499761262991	0.773742494322337	0.692428303594619\\
1.185	0.379368276771857	0.784102202600443	0.687549132887069\\
1.185	0.392393618241769	0.794238304822092	0.682943679235328\\
1.185	0.405569970164763	0.804147337016096	0.678558662193044\\
1.185	0.418891418885081	0.813826036051075	0.674343787412871\\
1.185	0.432351961217133	0.823271342915544	0.670253187715552\\
1.185	0.445945513182735	0.832480405557786	0.666246357336159\\
1.185	0.459665918841652	0.841450581281375	0.662288623474186\\
1.185	0.473506959189401	0.850179438693979	0.658351221673191\\
1.185	0.487462361096317	0.85866475920876	0.654411055137886\\
1.185	0.501525806262096	0.866904538099407	0.650450223206468\\
1.185	0.515690940160279	0.874896985111485	0.646455401874959\\
1.185	0.529951380947623	0.882640524634436	0.642417151055156\\
1.185	0.544300728313782	0.890133795440129	0.638329210947586\\
1.185	0.558732572247415	0.897375649995373	0.634187835387499\\
1.185	0.57324050169555	0.904365153357279	0.629991194998376\\
1.185	0.587818113093873	0.911101581661729	0.625738868900923\\
1.185	0.602459018746568	0.917584420216504	0.621431431633139\\
1.185	0.617156855035314	0.923813361211857	0.617070132480553\\
1.185	0.631905290438127	0.92978830106243	0.612656657833838\\
1.185	0.6466980333399	0.935509337395467	0.608192963377089\\
1.185	0.661528839617671	0.940976765701173	0.603681161495106\\
1.185	0.676391519984904	0.946191075661951	0.599123449738607\\
1.185	0.691279947080357	0.951152947177946	0.594522067901588\\
1.185	0.706188062288412	0.955863246106974	0.58987927366239\\
1.185	0.721109882279076	0.960323019737426	0.585197329319937\\
1.185	0.736039505257226	0.964533492013186	0.580478494543975\\
1.185	0.75097111691199	0.968496058529893	0.575725022018311\\
1.185	0.765898996058536	0.972212281322139	0.57093915428956\\
1.185	0.780817519965826	0.975683883461272	0.566123121053176\\
1.185	0.795721169365275	0.97891274348356	0.561279136605559\\
1.185	0.810604533136476	0.981900889668376	0.5564093973992\\
1.185	0.825462312667457	0.984650494185936	0.551516079695734\\
1.185	0.840289325888138	0.987163867133889	0.546601337319639\\
1.185	0.855080510976838	0.989443450481767	0.541667299515137\\
1.185	0.86983092974082	0.991491811941914	0.536716068908601\\
1.185	0.884535770672926	0.993311638785082	0.531749719578144\\
1.185	0.899190351687411	0.994905731618374	0.526770295231793\\
1.185	0.913790122539012	0.996276998142666	0.521779807495103\\
1.185	0.928330666930242	0.997428446906011	0.516780234308684\\
1.185	0.942807704312699	0.998363181068901	0.511773518435872\\
1.185	0.95721709138901	0.999084392196567	0.506761566080171\\
1.185	0.971554823322704	0.999595354092744	0.501746245611876\\
1.185	0.985817034663989	0.999899416688638	0.496729386403002\\
1.185	1	1	0.491669087346898\\
1.2	0	0	0.710195569525497\\
1.2	0.000100583311362513	0.0141829653360114	0.71240460418299\\
1.2	0.000404645907256436	0.0284451766772965	0.714630690536905\\
1.2	0.000915607803432999	0.0427829086109896	0.716916312064106\\
1.2	0.00163681893109844	0.057192295687301	0.719289019360234\\
1.2	0.00257155309398959	0.0716693330697584	0.721780367061016\\
1.2	0.00372300185733413	0.0862098774609879	0.724425655492835\\
1.2	0.00509426838162598	0.100809648312589	0.727263413850304\\
1.2	0.00668836121491815	0.115464229327074	0.73033458997498\\
1.2	0.00850818805808555	0.13016907025918	0.733681422514709\\
1.2	0.0105565495182326	0.144919489023162	0.737345986077549\\
1.2	0.0128361328661109	0.159710674111862	0.741368418705642\\
1.2	0.0153495058140643	0.174537687332543	0.745784862958201\\
1.2	0.0180991103316243	0.189395466863524	0.750625176092311\\
1.2	0.0210872565164405	0.204278830634725	0.755910489868999\\
1.2	0.0243161165387281	0.219182480034174	0.761650724664453\\
1.2	0.0277877186778607	0.234101003941464	0.767842183867616\\
1.2	0.0315039414701067	0.24902888308801	0.77446537092041\\
1.2	0.0354665079868145	0.263960494742775	0.781483180780606\\
1.2	0.0396769802625738	0.278890117720924	0.788839618268073\\
1.2	0.0441367538930258	0.293811937711588	0.796459186325544\\
1.2	0.0488470528220538	0.308720052919643	0.804247066923456\\
1.2	0.0538089243380495	0.323608480015097	0.812090186161951\\
1.2	0.0590232342988274	0.338471160382329	0.819859213932694\\
1.2	0.064490662604533	0.3533019666601	0.827411499063639\\
1.2	0.0702116989375697	0.368094709561873	0.834594885813505\\
1.2	0.0761866387881432	0.382843144964686	0.841252300287179\\
1.2	0.0824155797834956	0.397540981253432	0.847226939729079\\
1.2	0.0888984183382709	0.412181886906127	0.852367847905042\\
1.2	0.0956348466427212	0.42675949830445	0.85653562002783\\
1.2	0.102624350004627	0.441267427752584	0.859607954619618\\
1.2	0.109866204559871	0.455699271686218	0.861484760266622\\
1.2	0.117359475365564	0.470048619052377	0.862092534251541\\
1.2	0.125103014888515	0.484309059839721	0.861387758068368\\
1.2	0.133095461900593	0.498474193737904	0.85935910089082\\
1.2	0.14133524079124	0.512537638903683	0.856028283765738\\
1.2	0.149820561306021	0.526493040810599	0.85144953087018\\
1.2	0.158549418718625	0.540334081158348	0.845707614739505\\
1.2	0.167519594442214	0.554054486817265	0.83891458434315\\
1.2	0.176728657084455	0.567648038782867	0.831205342370838\\
1.2	0.186173963948925	0.581108581114919	0.822732305417526\\
1.2	0.195852662983903	0.594430029835237	0.813659432944308\\
1.2	0.205761695177907	0.607606381758231	0.80415594410487\\
1.2	0.215897797399558	0.620631723228143	0.794390053406971\\
1.2	0.226257505677663	0.633500238737009	0.784523046062803\\
1.2	0.236837158915675	0.646206219397571	0.774703982861587\\
1.2	0.247632903032949	0.658744071245709	0.765065275211016\\
1.2	0.258640695523527	0.671108323347402	0.755719307772963\\
1.2	0.269856310421543	0.683293635685828	0.74675621400591\\
1.2	0.28127534366066	0.695294806804924	0.738242834604297\\
1.2	0.292893218813452	0.707106781186547	0.730222816007834\\
1.2	0.304705193195076	0.71872465633934	0.722717741087768\\
1.2	0.316706364314172	0.730143689578457	0.715729131143269\\
1.2	0.328891676652598	0.741359304476472	0.709241120561854\\
1.2	0.341255928754291	0.752367096967051	0.703223584571265\\
1.2	0.353793780602429	0.763162841084325	0.69763549659742\\
1.2	0.366499761262991	0.773742494322337	0.692428303594619\\
1.2	0.379368276771857	0.784102202600443	0.687549132887069\\
1.2	0.392393618241769	0.794238304822092	0.682943679235329\\
1.2	0.405569970164763	0.804147337016097	0.678558662193044\\
1.2	0.418891418885081	0.813826036051075	0.67434378741287\\
1.2	0.432351961217133	0.823271342915544	0.670253187715552\\
1.2	0.445945513182735	0.832480405557786	0.666246357336159\\
1.2	0.459665918841652	0.841450581281375	0.662288623474187\\
1.2	0.473506959189401	0.850179438693979	0.658351221673191\\
1.2	0.487462361096317	0.85866475920876	0.654411055137885\\
1.2	0.501525806262096	0.866904538099407	0.650450223206468\\
1.2	0.515690940160279	0.874896985111485	0.646455401874958\\
1.2	0.529951380947623	0.882640524634436	0.642417151055155\\
1.2	0.544300728313782	0.890133795440129	0.638329210947589\\
1.2	0.558732572247415	0.897375649995373	0.634187835387498\\
1.2	0.57324050169555	0.904365153357279	0.629991194998373\\
1.2	0.587818113093873	0.911101581661729	0.625738868900923\\
1.2	0.602459018746568	0.917584420216504	0.621431431633141\\
1.2	0.617156855035314	0.923813361211857	0.617070132480553\\
1.2	0.631905290438127	0.92978830106243	0.612656657833838\\
1.2	0.6466980333399	0.935509337395467	0.608192963377086\\
1.2	0.661528839617671	0.940976765701173	0.603681161495103\\
1.2	0.676391519984904	0.94619107566195	0.599123449738608\\
1.2	0.691279947080357	0.951152947177946	0.594522067901591\\
1.2	0.706188062288412	0.955863246106974	0.589879273662389\\
1.2	0.721109882279076	0.960323019737426	0.585197329319937\\
1.2	0.736039505257226	0.964533492013186	0.580478494543978\\
1.2	0.75097111691199	0.968496058529893	0.575725022018312\\
1.2	0.765898996058536	0.972212281322139	0.570939154289556\\
1.2	0.780817519965826	0.975683883461272	0.566123121053175\\
1.2	0.795721169365275	0.97891274348356	0.56127913660556\\
1.2	0.810604533136476	0.981900889668376	0.5564093973992\\
1.2	0.825462312667457	0.984650494185936	0.551516079695735\\
1.2	0.840289325888138	0.987163867133889	0.54660133731964\\
1.2	0.855080510976839	0.989443450481767	0.541667299515139\\
1.2	0.86983092974082	0.991491811941914	0.536716068908598\\
1.2	0.884535770672926	0.993311638785082	0.531749719578143\\
1.2	0.899190351687411	0.994905731618374	0.526770295231792\\
1.2	0.913790122539012	0.996276998142666	0.521779807495101\\
1.2	0.928330666930241	0.99742844690601	0.516780234308685\\
1.2	0.942807704312699	0.998363181068902	0.511773518435873\\
1.2	0.95721709138901	0.999084392196567	0.506761566080168\\
1.2	0.971554823322703	0.999595354092743	0.501746245611875\\
1.2	0.985817034663989	0.999899416688637	0.496729386403004\\
1.2	1	1	0.491669087346906\\
1.215	0	0	0.710195569525497\\
1.215	0.000100583311362513	0.0141829653360114	0.71240460418299\\
1.215	0.000404645907256436	0.0284451766772965	0.714630690536905\\
1.215	0.000915607803432999	0.0427829086109896	0.716916312064106\\
1.215	0.00163681893109844	0.057192295687301	0.719289019360234\\
1.215	0.00257155309398959	0.0716693330697584	0.721780367061016\\
1.215	0.00372300185733413	0.0862098774609879	0.724425655492835\\
1.215	0.00509426838162598	0.100809648312589	0.727263413850304\\
1.215	0.00668836121491816	0.115464229327074	0.73033458997498\\
1.215	0.00850818805808554	0.13016907025918	0.733681422514709\\
1.215	0.0105565495182326	0.144919489023162	0.737345986077549\\
1.215	0.0128361328661109	0.159710674111862	0.741368418705642\\
1.215	0.0153495058140643	0.174537687332543	0.745784862958201\\
1.215	0.0180991103316243	0.189395466863524	0.750625176092311\\
1.215	0.0210872565164405	0.204278830634725	0.755910489868998\\
1.215	0.0243161165387281	0.219182480034174	0.761650724664453\\
1.215	0.0277877186778607	0.234101003941464	0.767842183867616\\
1.215	0.0315039414701067	0.24902888308801	0.77446537092041\\
1.215	0.0354665079868145	0.263960494742775	0.781483180780606\\
1.215	0.0396769802625737	0.278890117720924	0.788839618268073\\
1.215	0.0441367538930258	0.293811937711588	0.796459186325543\\
1.215	0.0488470528220538	0.308720052919643	0.804247066923454\\
1.215	0.0538089243380495	0.323608480015096	0.812090186161951\\
1.215	0.0590232342988274	0.338471160382329	0.819859213932694\\
1.215	0.064490662604533	0.3533019666601	0.827411499063639\\
1.215	0.0702116989375697	0.368094709561873	0.834594885813505\\
1.215	0.0761866387881432	0.382843144964686	0.841252300287181\\
1.215	0.0824155797834956	0.397540981253432	0.847226939729081\\
1.215	0.0888984183382709	0.412181886906127	0.852367847905041\\
1.215	0.0956348466427212	0.42675949830445	0.85653562002783\\
1.215	0.102624350004627	0.441267427752585	0.859607954619618\\
1.215	0.109866204559871	0.455699271686218	0.861484760266622\\
1.215	0.117359475365564	0.470048619052377	0.86209253425154\\
1.215	0.125103014888515	0.484309059839721	0.861387758068368\\
1.215	0.133095461900593	0.498474193737904	0.859359100890819\\
1.215	0.14133524079124	0.512537638903683	0.856028283765735\\
1.215	0.149820561306021	0.526493040810599	0.851449530870182\\
1.215	0.158549418718625	0.540334081158348	0.845707614739506\\
1.215	0.167519594442213	0.554054486817265	0.838914584343148\\
1.215	0.176728657084455	0.567648038782867	0.831205342370835\\
1.215	0.186173963948925	0.581108581114919	0.822732305417525\\
1.215	0.195852662983903	0.594430029835236	0.813659432944304\\
1.215	0.205761695177907	0.607606381758231	0.804155944104872\\
1.215	0.215897797399558	0.620631723228144	0.79439005340697\\
1.215	0.226257505677663	0.633500238737009	0.784523046062799\\
1.215	0.236837158915675	0.646206219397571	0.774703982861587\\
1.215	0.247632903032949	0.658744071245709	0.765065275211017\\
1.215	0.258640695523527	0.671108323347402	0.755719307772961\\
1.215	0.269856310421543	0.683293635685828	0.746756214005913\\
1.215	0.28127534366066	0.695294806804925	0.738242834604296\\
1.215	0.292893218813452	0.707106781186547	0.730222816007833\\
1.215	0.304705193195076	0.71872465633934	0.722717741087771\\
1.215	0.316706364314172	0.730143689578457	0.715729131143268\\
1.215	0.328891676652598	0.741359304476472	0.709241120561856\\
1.215	0.341255928754291	0.752367096967051	0.703223584571265\\
1.215	0.353793780602429	0.763162841084325	0.697635496597417\\
1.215	0.366499761262991	0.773742494322337	0.692428303594618\\
1.215	0.379368276771856	0.784102202600442	0.687549132887069\\
1.215	0.392393618241769	0.794238304822092	0.682943679235328\\
1.215	0.405569970164763	0.804147337016096	0.678558662193045\\
1.215	0.418891418885081	0.813826036051075	0.674343787412871\\
1.215	0.432351961217133	0.823271342915544	0.670253187715551\\
1.215	0.445945513182735	0.832480405557786	0.666246357336159\\
1.215	0.459665918841652	0.841450581281375	0.662288623474187\\
1.215	0.473506959189401	0.850179438693979	0.658351221673192\\
1.215	0.487462361096317	0.85866475920876	0.654411055137884\\
1.215	0.501525806262096	0.866904538099407	0.650450223206467\\
1.215	0.515690940160279	0.874896985111485	0.646455401874958\\
1.215	0.529951380947623	0.882640524634436	0.642417151055155\\
1.215	0.544300728313782	0.890133795440129	0.638329210947586\\
1.215	0.558732572247415	0.897375649995373	0.6341878353875\\
1.215	0.57324050169555	0.904365153357279	0.629991194998376\\
1.215	0.587818113093873	0.911101581661729	0.625738868900922\\
1.215	0.602459018746568	0.917584420216504	0.621431431633139\\
1.215	0.617156855035314	0.923813361211857	0.617070132480552\\
1.215	0.631905290438127	0.92978830106243	0.612656657833837\\
1.215	0.6466980333399	0.935509337395467	0.608192963377089\\
1.215	0.661528839617671	0.940976765701173	0.603681161495106\\
1.215	0.676391519984903	0.946191075661951	0.599123449738607\\
1.215	0.691279947080357	0.951152947177946	0.594522067901589\\
1.215	0.706188062288412	0.955863246106974	0.589879273662389\\
1.215	0.721109882279076	0.960323019737426	0.585197329319937\\
1.215	0.736039505257226	0.964533492013186	0.580478494543975\\
1.215	0.75097111691199	0.968496058529893	0.575725022018312\\
1.215	0.765898996058536	0.972212281322139	0.57093915428956\\
1.215	0.780817519965826	0.975683883461272	0.566123121053174\\
1.215	0.795721169365275	0.978912743483559	0.56127913660556\\
1.215	0.810604533136476	0.981900889668376	0.556409397399202\\
1.215	0.825462312667457	0.984650494185936	0.551516079695734\\
1.215	0.840289325888138	0.987163867133889	0.546601337319639\\
1.215	0.855080510976838	0.989443450481767	0.54166729951514\\
1.215	0.86983092974082	0.991491811941914	0.536716068908599\\
1.215	0.884535770672926	0.993311638785082	0.531749719578144\\
1.215	0.899190351687411	0.994905731618374	0.526770295231792\\
1.215	0.913790122539012	0.996276998142666	0.521779807495102\\
1.215	0.928330666930242	0.997428446906011	0.516780234308685\\
1.215	0.942807704312699	0.998363181068902	0.511773518435871\\
1.215	0.95721709138901	0.999084392196567	0.506761566080168\\
1.215	0.971554823322703	0.999595354092743	0.501746245611874\\
1.215	0.985817034663989	0.999899416688637	0.496729386403004\\
1.215	1	1	0.491669087346906\\
1.23	0	0	0.710195569525497\\
1.23	0.000100583311362513	0.0141829653360114	0.71240460418299\\
1.23	0.000404645907256436	0.0284451766772965	0.714630690536905\\
1.23	0.000915607803432999	0.0427829086109896	0.716916312064106\\
1.23	0.00163681893109844	0.057192295687301	0.719289019360234\\
1.23	0.00257155309398959	0.0716693330697584	0.721780367061016\\
1.23	0.00372300185733413	0.0862098774609879	0.724425655492835\\
1.23	0.00509426838162598	0.100809648312589	0.727263413850304\\
1.23	0.00668836121491816	0.115464229327074	0.73033458997498\\
1.23	0.00850818805808555	0.13016907025918	0.733681422514709\\
1.23	0.0105565495182326	0.144919489023162	0.737345986077549\\
1.23	0.0128361328661109	0.159710674111862	0.741368418705642\\
1.23	0.0153495058140643	0.174537687332543	0.745784862958202\\
1.23	0.0180991103316243	0.189395466863524	0.750625176092311\\
1.23	0.0210872565164405	0.204278830634725	0.755910489868999\\
1.23	0.0243161165387281	0.219182480034174	0.761650724664453\\
1.23	0.0277877186778607	0.234101003941464	0.767842183867616\\
1.23	0.0315039414701067	0.24902888308801	0.77446537092041\\
1.23	0.0354665079868145	0.263960494742775	0.781483180780606\\
1.23	0.0396769802625738	0.278890117720924	0.788839618268073\\
1.23	0.0441367538930258	0.293811937711588	0.796459186325544\\
1.23	0.0488470528220538	0.308720052919643	0.804247066923454\\
1.23	0.0538089243380495	0.323608480015097	0.81209018616195\\
1.23	0.0590232342988274	0.338471160382329	0.819859213932694\\
1.23	0.064490662604533	0.3533019666601	0.827411499063639\\
1.23	0.0702116989375697	0.368094709561873	0.834594885813504\\
1.23	0.0761866387881432	0.382843144964686	0.841252300287181\\
1.23	0.0824155797834956	0.397540981253432	0.847226939729082\\
1.23	0.0888984183382709	0.412181886906127	0.852367847905041\\
1.23	0.0956348466427212	0.42675949830445	0.856535620027828\\
1.23	0.102624350004627	0.441267427752584	0.859607954619617\\
1.23	0.109866204559871	0.455699271686218	0.86148476026662\\
1.23	0.117359475365564	0.470048619052377	0.862092534251539\\
1.23	0.125103014888515	0.484309059839721	0.86138775806837\\
1.23	0.133095461900593	0.498474193737904	0.859359100890821\\
1.23	0.14133524079124	0.512537638903683	0.856028283765736\\
1.23	0.149820561306021	0.526493040810599	0.851449530870185\\
1.23	0.158549418718625	0.540334081158348	0.845707614739508\\
1.23	0.167519594442214	0.554054486817265	0.838914584343149\\
1.23	0.176728657084455	0.567648038782867	0.831205342370837\\
1.23	0.186173963948925	0.581108581114919	0.822732305417527\\
1.23	0.195852662983903	0.594430029835236	0.813659432944309\\
1.23	0.205761695177907	0.607606381758231	0.804155944104871\\
1.23	0.215897797399558	0.620631723228143	0.794390053406968\\
1.23	0.226257505677663	0.633500238737009	0.784523046062798\\
1.23	0.236837158915675	0.646206219397571	0.77470398286159\\
1.23	0.247632903032949	0.658744071245709	0.765065275211014\\
1.23	0.258640695523528	0.671108323347402	0.755719307772962\\
1.23	0.269856310421543	0.683293635685828	0.746756214005913\\
1.23	0.28127534366066	0.695294806804924	0.738242834604295\\
1.23	0.292893218813452	0.707106781186547	0.730222816007834\\
1.23	0.304705193195075	0.71872465633934	0.722717741087767\\
1.23	0.316706364314172	0.730143689578457	0.715729131143267\\
1.23	0.328891676652598	0.741359304476472	0.709241120561855\\
1.23	0.341255928754291	0.752367096967051	0.703223584571266\\
1.23	0.353793780602429	0.763162841084325	0.697635496597419\\
1.23	0.366499761262991	0.773742494322337	0.692428303594617\\
1.23	0.379368276771857	0.784102202600442	0.687549132887069\\
1.23	0.392393618241769	0.794238304822092	0.682943679235329\\
1.23	0.405569970164763	0.804147337016096	0.678558662193044\\
1.23	0.418891418885081	0.813826036051075	0.674343787412872\\
1.23	0.432351961217133	0.823271342915545	0.670253187715553\\
1.23	0.445945513182735	0.832480405557786	0.666246357336159\\
1.23	0.459665918841652	0.841450581281375	0.662288623474186\\
1.23	0.473506959189401	0.850179438693979	0.658351221673192\\
1.23	0.487462361096317	0.85866475920876	0.654411055137886\\
1.23	0.501525806262096	0.866904538099407	0.650450223206468\\
1.23	0.515690940160279	0.874896985111485	0.646455401874959\\
1.23	0.529951380947623	0.882640524634436	0.642417151055154\\
1.23	0.544300728313782	0.890133795440129	0.638329210947586\\
1.23	0.558732572247415	0.897375649995372	0.6341878353875\\
1.23	0.57324050169555	0.904365153357279	0.629991194998377\\
1.23	0.587818113093873	0.911101581661729	0.625738868900924\\
1.23	0.602459018746568	0.917584420216505	0.62143143163314\\
1.23	0.617156855035314	0.923813361211857	0.617070132480552\\
1.23	0.631905290438127	0.92978830106243	0.612656657833836\\
1.23	0.6466980333399	0.935509337395467	0.608192963377087\\
1.23	0.661528839617671	0.940976765701173	0.603681161495104\\
1.23	0.676391519984904	0.94619107566195	0.599123449738609\\
1.23	0.691279947080357	0.951152947177946	0.594522067901591\\
1.23	0.706188062288412	0.955863246106974	0.58987927366239\\
1.23	0.721109882279076	0.960323019737426	0.585197329319936\\
1.23	0.736039505257226	0.964533492013185	0.580478494543978\\
1.23	0.75097111691199	0.968496058529893	0.575725022018314\\
1.23	0.765898996058536	0.972212281322139	0.570939154289556\\
1.23	0.780817519965826	0.975683883461272	0.566123121053173\\
1.23	0.795721169365275	0.978912743483559	0.561279136605559\\
1.23	0.810604533136476	0.981900889668376	0.556409397399202\\
1.23	0.825462312667457	0.984650494185936	0.551516079695737\\
1.23	0.840289325888138	0.987163867133889	0.546601337319638\\
1.23	0.855080510976839	0.989443450481767	0.54166729951514\\
1.23	0.86983092974082	0.991491811941914	0.536716068908601\\
1.23	0.884535770672926	0.993311638785082	0.531749719578144\\
1.23	0.899190351687411	0.994905731618374	0.526770295231793\\
1.23	0.913790122539012	0.996276998142666	0.521779807495101\\
1.23	0.928330666930242	0.997428446906011	0.516780234308684\\
1.23	0.942807704312699	0.998363181068902	0.511773518435873\\
1.23	0.95721709138901	0.999084392196567	0.506761566080171\\
1.23	0.971554823322704	0.999595354092744	0.501746245611874\\
1.23	0.985817034663988	0.999899416688637	0.496729386403002\\
1.23	1	1	0.491669087346908\\
1.245	0	0	0.710195569525497\\
1.245	0.000100583311362513	0.0141829653360114	0.71240460418299\\
1.245	0.000404645907256436	0.0284451766772965	0.714630690536905\\
1.245	0.000915607803432999	0.0427829086109896	0.716916312064106\\
1.245	0.00163681893109844	0.057192295687301	0.719289019360234\\
1.245	0.00257155309398959	0.0716693330697584	0.721780367061016\\
1.245	0.00372300185733413	0.0862098774609879	0.724425655492835\\
1.245	0.00509426838162598	0.100809648312589	0.727263413850304\\
1.245	0.00668836121491816	0.115464229327074	0.73033458997498\\
1.245	0.00850818805808554	0.13016907025918	0.733681422514709\\
1.245	0.0105565495182326	0.144919489023162	0.737345986077549\\
1.245	0.0128361328661109	0.159710674111862	0.741368418705642\\
1.245	0.0153495058140643	0.174537687332543	0.745784862958202\\
1.245	0.0180991103316243	0.189395466863524	0.750625176092311\\
1.245	0.0210872565164405	0.204278830634725	0.755910489868999\\
1.245	0.0243161165387281	0.219182480034174	0.761650724664453\\
1.245	0.0277877186778607	0.234101003941464	0.767842183867616\\
1.245	0.0315039414701067	0.24902888308801	0.77446537092041\\
1.245	0.0354665079868145	0.263960494742775	0.781483180780606\\
1.245	0.0396769802625738	0.278890117720924	0.788839618268073\\
1.245	0.0441367538930258	0.293811937711588	0.796459186325544\\
1.245	0.0488470528220537	0.308720052919643	0.804247066923455\\
1.245	0.0538089243380495	0.323608480015096	0.81209018616195\\
1.245	0.0590232342988274	0.338471160382329	0.819859213932693\\
1.245	0.064490662604533	0.3533019666601	0.827411499063639\\
1.245	0.0702116989375697	0.368094709561873	0.834594885813504\\
1.245	0.0761866387881432	0.382843144964686	0.841252300287179\\
1.245	0.0824155797834956	0.397540981253432	0.847226939729082\\
1.245	0.0888984183382709	0.412181886906127	0.852367847905042\\
1.245	0.0956348466427212	0.42675949830445	0.856535620027828\\
1.245	0.102624350004627	0.441267427752584	0.859607954619617\\
1.245	0.109866204559871	0.455699271686218	0.861484760266622\\
1.245	0.117359475365564	0.470048619052377	0.86209253425154\\
1.245	0.125103014888515	0.484309059839721	0.86138775806837\\
1.245	0.133095461900593	0.498474193737904	0.859359100890821\\
1.245	0.14133524079124	0.512537638903683	0.856028283765736\\
1.245	0.149820561306021	0.526493040810599	0.851449530870184\\
1.245	0.158549418718625	0.540334081158348	0.845707614739501\\
1.245	0.167519594442214	0.554054486817265	0.838914584343155\\
1.245	0.176728657084455	0.567648038782867	0.831205342370835\\
1.245	0.186173963948925	0.581108581114919	0.822732305417528\\
1.245	0.195852662983903	0.594430029835236	0.813659432944309\\
1.245	0.205761695177907	0.607606381758231	0.80415594410487\\
1.245	0.215897797399558	0.620631723228143	0.794390053406969\\
1.245	0.226257505677663	0.633500238737009	0.784523046062801\\
1.245	0.236837158915675	0.646206219397571	0.774703982861591\\
1.245	0.247632903032949	0.658744071245709	0.765065275211014\\
1.245	0.258640695523528	0.671108323347402	0.755719307772964\\
1.245	0.269856310421543	0.683293635685828	0.746756214005912\\
1.245	0.28127534366066	0.695294806804924	0.738242834604296\\
1.245	0.292893218813452	0.707106781186547	0.730222816007833\\
1.245	0.304705193195075	0.71872465633934	0.722717741087768\\
1.245	0.316706364314172	0.730143689578457	0.715729131143268\\
1.245	0.328891676652598	0.741359304476472	0.709241120561856\\
1.245	0.341255928754291	0.752367096967051	0.703223584571265\\
1.245	0.353793780602429	0.763162841084325	0.69763549659742\\
1.245	0.366499761262991	0.773742494322337	0.692428303594618\\
1.245	0.379368276771857	0.784102202600442	0.687549132887068\\
1.245	0.392393618241769	0.794238304822092	0.682943679235329\\
1.245	0.405569970164763	0.804147337016097	0.678558662193044\\
1.245	0.418891418885081	0.813826036051075	0.674343787412871\\
1.245	0.432351961217132	0.823271342915544	0.670253187715552\\
1.245	0.445945513182735	0.832480405557786	0.666246357336159\\
1.245	0.459665918841652	0.841450581281375	0.662288623474186\\
1.245	0.473506959189401	0.850179438693979	0.658351221673191\\
1.245	0.487462361096317	0.85866475920876	0.654411055137887\\
1.245	0.501525806262096	0.866904538099407	0.650450223206468\\
1.245	0.515690940160279	0.874896985111485	0.646455401874959\\
1.245	0.529951380947623	0.882640524634436	0.642417151055156\\
1.245	0.544300728313782	0.890133795440129	0.638329210947585\\
1.245	0.558732572247415	0.897375649995372	0.634187835387498\\
1.245	0.57324050169555	0.904365153357279	0.629991194998377\\
1.245	0.587818113093873	0.911101581661729	0.625738868900923\\
1.245	0.602459018746568	0.917584420216505	0.621431431633142\\
1.245	0.617156855035314	0.923813361211857	0.617070132480552\\
1.245	0.631905290438127	0.92978830106243	0.612656657833837\\
1.245	0.6466980333399	0.935509337395467	0.608192963377088\\
1.245	0.661528839617671	0.940976765701173	0.603681161495104\\
1.245	0.676391519984904	0.94619107566195	0.599123449738606\\
1.245	0.691279947080357	0.951152947177946	0.594522067901589\\
1.245	0.706188062288412	0.955863246106974	0.589879273662389\\
1.245	0.721109882279076	0.960323019737426	0.585197329319938\\
1.245	0.736039505257226	0.964533492013186	0.580478494543977\\
1.245	0.75097111691199	0.968496058529893	0.575725022018313\\
1.245	0.765898996058536	0.972212281322139	0.570939154289557\\
1.245	0.780817519965826	0.975683883461272	0.566123121053173\\
1.245	0.795721169365275	0.97891274348356	0.561279136605561\\
1.245	0.810604533136476	0.981900889668376	0.556409397399197\\
1.245	0.825462312667457	0.984650494185936	0.551516079695735\\
1.245	0.840289325888138	0.987163867133889	0.546601337319641\\
1.245	0.855080510976839	0.989443450481767	0.541667299515139\\
1.245	0.86983092974082	0.991491811941914	0.536716068908601\\
1.245	0.884535770672926	0.993311638785082	0.531749719578143\\
1.245	0.899190351687411	0.994905731618374	0.526770295231795\\
1.245	0.913790122539012	0.996276998142666	0.521779807495101\\
1.245	0.928330666930241	0.99742844690601	0.516780234308682\\
1.245	0.942807704312699	0.998363181068902	0.511773518435876\\
1.245	0.95721709138901	0.999084392196567	0.50676156608017\\
1.245	0.971554823322704	0.999595354092743	0.501746245611874\\
1.245	0.985817034663989	0.999899416688637	0.496729386403004\\
1.245	1	1	0.491669087346906\\
1.26	0	0	0.710195569525497\\
1.26	0.000100583311362513	0.0141829653360114	0.71240460418299\\
1.26	0.000404645907256436	0.0284451766772965	0.714630690536905\\
1.26	0.000915607803432999	0.0427829086109896	0.716916312064106\\
1.26	0.00163681893109844	0.057192295687301	0.719289019360234\\
1.26	0.00257155309398959	0.0716693330697584	0.721780367061016\\
1.26	0.00372300185733413	0.0862098774609879	0.724425655492835\\
1.26	0.00509426838162598	0.100809648312589	0.727263413850304\\
1.26	0.00668836121491815	0.115464229327074	0.73033458997498\\
1.26	0.00850818805808555	0.13016907025918	0.733681422514709\\
1.26	0.0105565495182326	0.144919489023162	0.737345986077549\\
1.26	0.0128361328661109	0.159710674111862	0.741368418705642\\
1.26	0.0153495058140643	0.174537687332543	0.745784862958201\\
1.26	0.0180991103316243	0.189395466863524	0.750625176092311\\
1.26	0.0210872565164405	0.204278830634725	0.755910489868999\\
1.26	0.0243161165387281	0.219182480034174	0.761650724664453\\
1.26	0.0277877186778607	0.234101003941464	0.767842183867616\\
1.26	0.0315039414701067	0.24902888308801	0.77446537092041\\
1.26	0.0354665079868145	0.263960494742775	0.781483180780606\\
1.26	0.0396769802625738	0.278890117720924	0.788839618268073\\
1.26	0.0441367538930258	0.293811937711588	0.796459186325544\\
1.26	0.0488470528220537	0.308720052919643	0.804247066923455\\
1.26	0.0538089243380495	0.323608480015096	0.81209018616195\\
1.26	0.0590232342988274	0.338471160382329	0.819859213932694\\
1.26	0.064490662604533	0.3533019666601	0.82741149906364\\
1.26	0.0702116989375697	0.368094709561873	0.834594885813504\\
1.26	0.0761866387881432	0.382843144964686	0.841252300287179\\
1.26	0.0824155797834956	0.397540981253432	0.847226939729081\\
1.26	0.0888984183382709	0.412181886906127	0.852367847905041\\
1.26	0.0956348466427212	0.42675949830445	0.85653562002783\\
1.26	0.102624350004627	0.441267427752585	0.859607954619618\\
1.26	0.109866204559871	0.455699271686218	0.861484760266621\\
1.26	0.117359475365564	0.470048619052377	0.862092534251541\\
1.26	0.125103014888515	0.484309059839721	0.861387758068368\\
1.26	0.133095461900593	0.498474193737904	0.859359100890819\\
1.26	0.14133524079124	0.512537638903683	0.856028283765735\\
1.26	0.149820561306021	0.526493040810599	0.851449530870179\\
1.26	0.158549418718625	0.540334081158348	0.845707614739502\\
1.26	0.167519594442213	0.554054486817266	0.838914584343147\\
1.26	0.176728657084455	0.567648038782867	0.831205342370835\\
1.26	0.186173963948925	0.581108581114919	0.822732305417527\\
1.26	0.195852662983903	0.594430029835236	0.813659432944308\\
1.26	0.205761695177907	0.607606381758231	0.804155944104869\\
1.26	0.215897797399558	0.620631723228144	0.79439005340697\\
1.26	0.226257505677663	0.633500238737009	0.784523046062804\\
1.26	0.236837158915675	0.646206219397571	0.774703982861588\\
1.26	0.247632903032949	0.658744071245709	0.765065275211016\\
1.26	0.258640695523527	0.671108323347402	0.755719307772964\\
1.26	0.269856310421543	0.683293635685828	0.746756214005913\\
1.26	0.28127534366066	0.695294806804925	0.738242834604294\\
1.26	0.292893218813452	0.707106781186547	0.730222816007832\\
1.26	0.304705193195075	0.71872465633934	0.722717741087769\\
1.26	0.316706364314172	0.730143689578457	0.715729131143268\\
1.26	0.328891676652598	0.741359304476472	0.709241120561854\\
1.26	0.341255928754291	0.752367096967051	0.703223584571266\\
1.26	0.353793780602429	0.763162841084325	0.69763549659742\\
1.26	0.366499761262991	0.773742494322337	0.692428303594617\\
1.26	0.379368276771857	0.784102202600442	0.687549132887069\\
1.26	0.392393618241769	0.794238304822092	0.682943679235329\\
1.26	0.405569970164763	0.804147337016097	0.678558662193043\\
1.26	0.418891418885081	0.813826036051075	0.674343787412869\\
1.26	0.432351961217132	0.823271342915544	0.670253187715553\\
1.26	0.445945513182735	0.832480405557786	0.666246357336159\\
1.26	0.459665918841652	0.841450581281375	0.662288623474187\\
1.26	0.473506959189401	0.850179438693979	0.658351221673191\\
1.26	0.487462361096317	0.85866475920876	0.654411055137885\\
1.26	0.501525806262096	0.866904538099407	0.650450223206469\\
1.26	0.515690940160279	0.874896985111485	0.646455401874959\\
1.26	0.529951380947623	0.882640524634436	0.642417151055154\\
1.26	0.544300728313782	0.890133795440129	0.638329210947586\\
1.26	0.558732572247415	0.897375649995373	0.634187835387498\\
1.26	0.57324050169555	0.904365153357279	0.629991194998377\\
1.26	0.587818113093873	0.911101581661729	0.625738868900923\\
1.26	0.602459018746568	0.917584420216504	0.621431431633139\\
1.26	0.617156855035314	0.923813361211857	0.617070132480554\\
1.26	0.631905290438127	0.92978830106243	0.612656657833837\\
1.26	0.6466980333399	0.935509337395467	0.608192963377086\\
1.26	0.661528839617671	0.940976765701173	0.603681161495106\\
1.26	0.676391519984904	0.946191075661951	0.599123449738608\\
1.26	0.691279947080357	0.951152947177946	0.594522067901588\\
1.26	0.706188062288412	0.955863246106974	0.589879273662389\\
1.26	0.721109882279076	0.960323019737426	0.585197329319936\\
1.26	0.736039505257226	0.964533492013185	0.580478494543978\\
1.26	0.75097111691199	0.968496058529893	0.575725022018314\\
1.26	0.765898996058536	0.972212281322139	0.570939154289558\\
1.26	0.780817519965826	0.975683883461272	0.566123121053174\\
1.26	0.795721169365275	0.97891274348356	0.56127913660556\\
1.26	0.810604533136476	0.981900889668376	0.556409397399199\\
1.26	0.825462312667457	0.984650494185936	0.551516079695732\\
1.26	0.840289325888138	0.987163867133889	0.546601337319641\\
1.26	0.855080510976839	0.989443450481767	0.54166729951514\\
1.26	0.86983092974082	0.991491811941914	0.536716068908599\\
1.26	0.884535770672926	0.993311638785082	0.531749719578143\\
1.26	0.899190351687411	0.994905731618374	0.526770295231793\\
1.26	0.913790122539012	0.996276998142666	0.521779807495103\\
1.26	0.928330666930242	0.997428446906011	0.516780234308683\\
1.26	0.942807704312699	0.998363181068901	0.511773518435872\\
1.26	0.95721709138901	0.999084392196567	0.506761566080169\\
1.26	0.971554823322703	0.999595354092743	0.501746245611874\\
1.26	0.985817034663989	0.999899416688637	0.496729386403005\\
1.26	1	1	0.491669087346904\\
1.275	0	0	0.710195569525497\\
1.275	0.000100583311362513	0.0141829653360114	0.71240460418299\\
1.275	0.000404645907256436	0.0284451766772965	0.714630690536905\\
1.275	0.000915607803432999	0.0427829086109896	0.716916312064106\\
1.275	0.00163681893109844	0.057192295687301	0.719289019360234\\
1.275	0.00257155309398959	0.0716693330697584	0.721780367061016\\
1.275	0.00372300185733413	0.0862098774609879	0.724425655492835\\
1.275	0.00509426838162598	0.100809648312589	0.727263413850304\\
1.275	0.00668836121491816	0.115464229327074	0.73033458997498\\
1.275	0.00850818805808554	0.13016907025918	0.733681422514709\\
1.275	0.0105565495182326	0.144919489023162	0.737345986077549\\
1.275	0.0128361328661109	0.159710674111862	0.741368418705642\\
1.275	0.0153495058140643	0.174537687332543	0.745784862958201\\
1.275	0.0180991103316243	0.189395466863524	0.750625176092311\\
1.275	0.0210872565164405	0.204278830634725	0.755910489868999\\
1.275	0.0243161165387281	0.219182480034174	0.761650724664453\\
1.275	0.0277877186778607	0.234101003941464	0.767842183867616\\
1.275	0.0315039414701067	0.24902888308801	0.774465370920409\\
1.275	0.0354665079868145	0.263960494742775	0.781483180780606\\
1.275	0.0396769802625738	0.278890117720924	0.788839618268073\\
1.275	0.0441367538930258	0.293811937711588	0.796459186325544\\
1.275	0.0488470528220538	0.308720052919643	0.804247066923455\\
1.275	0.0538089243380495	0.323608480015097	0.81209018616195\\
1.275	0.0590232342988274	0.338471160382329	0.819859213932693\\
1.275	0.064490662604533	0.3533019666601	0.82741149906364\\
1.275	0.0702116989375697	0.368094709561873	0.834594885813505\\
1.275	0.0761866387881432	0.382843144964686	0.84125230028718\\
1.275	0.0824155797834956	0.397540981253432	0.847226939729081\\
1.275	0.0888984183382709	0.412181886906127	0.852367847905041\\
1.275	0.0956348466427212	0.42675949830445	0.856535620027829\\
1.275	0.102624350004627	0.441267427752584	0.859607954619618\\
1.275	0.109866204559871	0.455699271686218	0.861484760266623\\
1.275	0.117359475365564	0.470048619052377	0.862092534251541\\
1.275	0.125103014888515	0.484309059839721	0.861387758068368\\
1.275	0.133095461900593	0.498474193737904	0.859359100890822\\
1.275	0.14133524079124	0.512537638903683	0.856028283765737\\
1.275	0.149820561306021	0.526493040810599	0.851449530870179\\
1.275	0.158549418718625	0.540334081158348	0.845707614739505\\
1.275	0.167519594442213	0.554054486817265	0.838914584343149\\
1.275	0.176728657084455	0.567648038782867	0.831205342370839\\
1.275	0.186173963948925	0.581108581114919	0.822732305417527\\
1.275	0.195852662983903	0.594430029835236	0.813659432944306\\
1.275	0.205761695177907	0.607606381758231	0.804155944104871\\
1.275	0.215897797399558	0.620631723228143	0.79439005340697\\
1.275	0.226257505677663	0.633500238737009	0.7845230460628\\
1.275	0.236837158915675	0.646206219397571	0.774703982861589\\
1.275	0.247632903032949	0.658744071245709	0.765065275211017\\
1.275	0.258640695523527	0.671108323347402	0.755719307772964\\
1.275	0.269856310421543	0.683293635685828	0.746756214005911\\
1.275	0.28127534366066	0.695294806804924	0.738242834604295\\
1.275	0.292893218813452	0.707106781186547	0.730222816007835\\
1.275	0.304705193195075	0.71872465633934	0.722717741087768\\
1.275	0.316706364314172	0.730143689578457	0.715729131143267\\
1.275	0.328891676652598	0.741359304476472	0.709241120561855\\
1.275	0.341255928754291	0.752367096967051	0.703223584571266\\
1.275	0.353793780602429	0.763162841084325	0.697635496597419\\
1.275	0.366499761262991	0.773742494322337	0.692428303594619\\
1.275	0.379368276771857	0.784102202600443	0.687549132887069\\
1.275	0.392393618241769	0.794238304822092	0.682943679235329\\
1.275	0.405569970164763	0.804147337016097	0.678558662193045\\
1.275	0.418891418885081	0.813826036051075	0.674343787412869\\
1.275	0.432351961217132	0.823271342915544	0.670253187715551\\
1.275	0.445945513182735	0.832480405557786	0.66624635733616\\
1.275	0.459665918841652	0.841450581281375	0.662288623474186\\
1.275	0.473506959189401	0.850179438693979	0.658351221673191\\
1.275	0.487462361096317	0.85866475920876	0.654411055137886\\
1.275	0.501525806262096	0.866904538099407	0.650450223206467\\
1.275	0.515690940160279	0.874896985111485	0.646455401874959\\
1.275	0.529951380947623	0.882640524634436	0.642417151055156\\
1.275	0.544300728313782	0.890133795440129	0.638329210947586\\
1.275	0.558732572247415	0.897375649995373	0.6341878353875\\
1.275	0.57324050169555	0.904365153357279	0.629991194998377\\
1.275	0.587818113093873	0.911101581661729	0.625738868900923\\
1.275	0.602459018746568	0.917584420216504	0.621431431633139\\
1.275	0.617156855035314	0.923813361211857	0.617070132480553\\
1.275	0.631905290438127	0.92978830106243	0.612656657833837\\
1.275	0.6466980333399	0.935509337395467	0.608192963377088\\
1.275	0.661528839617671	0.940976765701173	0.603681161495104\\
1.275	0.676391519984904	0.94619107566195	0.599123449738607\\
1.275	0.691279947080357	0.951152947177946	0.59452206790159\\
1.275	0.706188062288412	0.955863246106974	0.589879273662391\\
1.275	0.721109882279076	0.960323019737426	0.585197329319939\\
1.275	0.736039505257226	0.964533492013186	0.580478494543976\\
1.275	0.75097111691199	0.968496058529893	0.575725022018311\\
1.275	0.765898996058536	0.972212281322139	0.570939154289559\\
1.275	0.780817519965826	0.975683883461272	0.566123121053176\\
1.275	0.795721169365275	0.97891274348356	0.561279136605558\\
1.275	0.810604533136476	0.981900889668376	0.556409397399201\\
1.275	0.825462312667457	0.984650494185936	0.551516079695734\\
1.275	0.840289325888138	0.987163867133889	0.546601337319639\\
1.275	0.855080510976839	0.989443450481768	0.54166729951514\\
1.275	0.86983092974082	0.991491811941914	0.536716068908599\\
1.275	0.884535770672926	0.993311638785082	0.531749719578144\\
1.275	0.899190351687411	0.994905731618374	0.526770295231793\\
1.275	0.913790122539012	0.996276998142666	0.521779807495102\\
1.275	0.928330666930242	0.997428446906011	0.516780234308683\\
1.275	0.942807704312699	0.998363181068902	0.51177351843587\\
1.275	0.95721709138901	0.999084392196567	0.50676156608017\\
1.275	0.971554823322704	0.999595354092743	0.501746245611876\\
1.275	0.985817034663988	0.999899416688637	0.496729386403004\\
1.275	1	1	0.491669087346906\\
1.29	0	0	0.710195569525497\\
1.29	0.000100583311362513	0.0141829653360114	0.71240460418299\\
1.29	0.000404645907256436	0.0284451766772965	0.714630690536905\\
1.29	0.000915607803432999	0.0427829086109896	0.716916312064106\\
1.29	0.00163681893109844	0.057192295687301	0.719289019360234\\
1.29	0.00257155309398959	0.0716693330697584	0.721780367061016\\
1.29	0.00372300185733413	0.0862098774609879	0.724425655492835\\
1.29	0.00509426838162598	0.100809648312589	0.727263413850304\\
1.29	0.00668836121491815	0.115464229327074	0.73033458997498\\
1.29	0.00850818805808555	0.13016907025918	0.733681422514709\\
1.29	0.0105565495182326	0.144919489023162	0.737345986077549\\
1.29	0.0128361328661109	0.159710674111862	0.741368418705642\\
1.29	0.0153495058140643	0.174537687332543	0.745784862958201\\
1.29	0.0180991103316243	0.189395466863524	0.750625176092311\\
1.29	0.0210872565164405	0.204278830634725	0.755910489868999\\
1.29	0.0243161165387281	0.219182480034174	0.761650724664453\\
1.29	0.0277877186778607	0.234101003941464	0.767842183867616\\
1.29	0.0315039414701067	0.24902888308801	0.77446537092041\\
1.29	0.0354665079868146	0.263960494742775	0.781483180780605\\
1.29	0.0396769802625737	0.278890117720924	0.788839618268073\\
1.29	0.0441367538930258	0.293811937711588	0.796459186325544\\
1.29	0.0488470528220538	0.308720052919643	0.804247066923455\\
1.29	0.0538089243380495	0.323608480015096	0.81209018616195\\
1.29	0.0590232342988274	0.338471160382329	0.819859213932693\\
1.29	0.064490662604533	0.3533019666601	0.827411499063638\\
1.29	0.0702116989375697	0.368094709561873	0.834594885813504\\
1.29	0.0761866387881432	0.382843144964686	0.84125230028718\\
1.29	0.0824155797834956	0.397540981253432	0.847226939729081\\
1.29	0.0888984183382709	0.412181886906127	0.852367847905042\\
1.29	0.0956348466427212	0.42675949830445	0.856535620027829\\
1.29	0.102624350004627	0.441267427752585	0.859607954619617\\
1.29	0.109866204559871	0.455699271686218	0.861484760266622\\
1.29	0.117359475365564	0.470048619052377	0.862092534251539\\
1.29	0.125103014888515	0.484309059839721	0.861387758068368\\
1.29	0.133095461900593	0.498474193737904	0.85935910089082\\
1.29	0.14133524079124	0.512537638903683	0.856028283765737\\
1.29	0.149820561306021	0.526493040810599	0.851449530870185\\
1.29	0.158549418718625	0.540334081158348	0.845707614739506\\
1.29	0.167519594442213	0.554054486817265	0.838914584343154\\
1.29	0.176728657084455	0.567648038782867	0.831205342370838\\
1.29	0.186173963948925	0.581108581114919	0.822732305417526\\
1.29	0.195852662983903	0.594430029835236	0.813659432944308\\
1.29	0.205761695177907	0.607606381758231	0.80415594410487\\
1.29	0.215897797399558	0.620631723228143	0.794390053406968\\
1.29	0.226257505677663	0.633500238737009	0.784523046062801\\
1.29	0.236837158915675	0.646206219397571	0.774703982861591\\
1.29	0.247632903032949	0.658744071245709	0.765065275211015\\
1.29	0.258640695523528	0.671108323347402	0.755719307772963\\
1.29	0.269856310421543	0.683293635685828	0.746756214005911\\
1.29	0.28127534366066	0.695294806804924	0.738242834604296\\
1.29	0.292893218813452	0.707106781186547	0.730222816007833\\
1.29	0.304705193195075	0.71872465633934	0.722717741087768\\
1.29	0.316706364314172	0.730143689578457	0.715729131143269\\
1.29	0.328891676652598	0.741359304476472	0.709241120561855\\
1.29	0.341255928754291	0.752367096967051	0.703223584571263\\
1.29	0.353793780602429	0.763162841084324	0.69763549659742\\
1.29	0.366499761262991	0.773742494322337	0.69242830359462\\
1.29	0.379368276771857	0.784102202600443	0.687549132887068\\
1.29	0.392393618241769	0.794238304822092	0.682943679235329\\
1.29	0.405569970164763	0.804147337016097	0.678558662193044\\
1.29	0.418891418885081	0.813826036051075	0.674343787412871\\
1.29	0.432351961217132	0.823271342915545	0.670253187715551\\
1.29	0.445945513182735	0.832480405557786	0.666246357336158\\
1.29	0.459665918841652	0.841450581281375	0.662288623474187\\
1.29	0.473506959189401	0.850179438693979	0.658351221673192\\
1.29	0.487462361096317	0.85866475920876	0.654411055137886\\
1.29	0.501525806262096	0.866904538099407	0.650450223206467\\
1.29	0.515690940160279	0.874896985111485	0.646455401874958\\
1.29	0.529951380947623	0.882640524634436	0.642417151055155\\
1.29	0.544300728313782	0.890133795440129	0.638329210947585\\
1.29	0.558732572247415	0.897375649995372	0.634187835387497\\
1.29	0.57324050169555	0.904365153357279	0.629991194998378\\
1.29	0.587818113093873	0.911101581661729	0.625738868900927\\
1.29	0.602459018746568	0.917584420216505	0.62143143163314\\
1.29	0.617156855035314	0.923813361211857	0.617070132480553\\
1.29	0.631905290438127	0.92978830106243	0.612656657833837\\
1.29	0.6466980333399	0.935509337395467	0.608192963377087\\
1.29	0.661528839617671	0.940976765701173	0.603681161495103\\
1.29	0.676391519984903	0.94619107566195	0.599123449738607\\
1.29	0.691279947080357	0.951152947177946	0.594522067901592\\
1.29	0.706188062288412	0.955863246106974	0.589879273662389\\
1.29	0.721109882279076	0.960323019737426	0.585197329319937\\
1.29	0.736039505257226	0.964533492013186	0.580478494543978\\
1.29	0.75097111691199	0.968496058529893	0.575725022018312\\
1.29	0.765898996058536	0.972212281322139	0.570939154289558\\
1.29	0.780817519965826	0.975683883461272	0.566123121053176\\
1.29	0.795721169365275	0.97891274348356	0.56127913660556\\
1.29	0.810604533136476	0.981900889668376	0.556409397399199\\
1.29	0.825462312667457	0.984650494185936	0.551516079695735\\
1.29	0.840289325888138	0.987163867133889	0.546601337319638\\
1.29	0.855080510976839	0.989443450481767	0.541667299515141\\
1.29	0.86983092974082	0.991491811941915	0.536716068908601\\
1.29	0.884535770672926	0.993311638785082	0.531749719578141\\
1.29	0.899190351687411	0.994905731618374	0.526770295231795\\
1.29	0.913790122539012	0.996276998142666	0.521779807495104\\
1.29	0.928330666930242	0.997428446906011	0.516780234308683\\
1.29	0.942807704312699	0.998363181068902	0.511773518435869\\
1.29	0.95721709138901	0.999084392196567	0.506761566080169\\
1.29	0.971554823322704	0.999595354092743	0.501746245611876\\
1.29	0.985817034663988	0.999899416688637	0.496729386403004\\
1.29	1	1	0.491669087346906\\
1.305	0	0	0.710195569525497\\
1.305	0.000100583311362513	0.0141829653360114	0.71240460418299\\
1.305	0.000404645907256436	0.0284451766772965	0.714630690536905\\
1.305	0.000915607803432999	0.0427829086109896	0.716916312064106\\
1.305	0.00163681893109844	0.057192295687301	0.719289019360234\\
1.305	0.00257155309398959	0.0716693330697584	0.721780367061016\\
1.305	0.00372300185733414	0.0862098774609879	0.724425655492835\\
1.305	0.00509426838162598	0.100809648312589	0.727263413850304\\
1.305	0.00668836121491816	0.115464229327074	0.73033458997498\\
1.305	0.00850818805808555	0.13016907025918	0.733681422514709\\
1.305	0.0105565495182326	0.144919489023162	0.737345986077549\\
1.305	0.0128361328661109	0.159710674111862	0.741368418705642\\
1.305	0.0153495058140643	0.174537687332543	0.745784862958201\\
1.305	0.0180991103316243	0.189395466863524	0.750625176092311\\
1.305	0.0210872565164405	0.204278830634725	0.755910489868998\\
1.305	0.0243161165387281	0.219182480034174	0.761650724664453\\
1.305	0.0277877186778607	0.234101003941464	0.767842183867616\\
1.305	0.0315039414701067	0.24902888308801	0.77446537092041\\
1.305	0.0354665079868145	0.263960494742775	0.781483180780606\\
1.305	0.0396769802625738	0.278890117720924	0.788839618268072\\
1.305	0.0441367538930258	0.293811937711588	0.796459186325543\\
1.305	0.0488470528220537	0.308720052919643	0.804247066923454\\
1.305	0.0538089243380495	0.323608480015096	0.81209018616195\\
1.305	0.0590232342988274	0.338471160382329	0.819859213932694\\
1.305	0.064490662604533	0.3533019666601	0.827411499063638\\
1.305	0.0702116989375697	0.368094709561873	0.834594885813503\\
1.305	0.0761866387881432	0.382843144964686	0.841252300287178\\
1.305	0.0824155797834956	0.397540981253432	0.847226939729081\\
1.305	0.0888984183382709	0.412181886906127	0.852367847905041\\
1.305	0.0956348466427212	0.42675949830445	0.85653562002783\\
1.305	0.102624350004627	0.441267427752585	0.859607954619618\\
1.305	0.109866204559871	0.455699271686218	0.86148476026662\\
1.305	0.117359475365564	0.470048619052377	0.862092534251539\\
1.305	0.125103014888515	0.484309059839721	0.861387758068367\\
1.305	0.133095461900593	0.498474193737904	0.859359100890818\\
1.305	0.14133524079124	0.512537638903683	0.856028283765736\\
1.305	0.149820561306021	0.526493040810599	0.851449530870186\\
1.305	0.158549418718625	0.540334081158348	0.845707614739505\\
1.305	0.167519594442214	0.554054486817265	0.838914584343151\\
1.305	0.176728657084455	0.567648038782867	0.831205342370834\\
1.305	0.186173963948925	0.581108581114919	0.822732305417527\\
1.305	0.195852662983903	0.594430029835236	0.813659432944308\\
1.305	0.205761695177907	0.607606381758231	0.804155944104867\\
1.305	0.215897797399558	0.620631723228143	0.794390053406969\\
1.305	0.226257505677663	0.633500238737009	0.784523046062802\\
1.305	0.236837158915675	0.646206219397571	0.774703982861587\\
1.305	0.247632903032949	0.658744071245709	0.765065275211016\\
1.305	0.258640695523527	0.671108323347402	0.755719307772963\\
1.305	0.269856310421543	0.683293635685828	0.746756214005911\\
1.305	0.28127534366066	0.695294806804924	0.738242834604296\\
1.305	0.292893218813452	0.707106781186547	0.730222816007834\\
1.305	0.304705193195075	0.71872465633934	0.722717741087769\\
1.305	0.316706364314172	0.730143689578457	0.715729131143268\\
1.305	0.328891676652598	0.741359304476472	0.709241120561855\\
1.305	0.341255928754291	0.752367096967051	0.703223584571265\\
1.305	0.353793780602429	0.763162841084325	0.697635496597419\\
1.305	0.366499761262991	0.773742494322337	0.692428303594619\\
1.305	0.379368276771856	0.784102202600442	0.687549132887069\\
1.305	0.392393618241769	0.794238304822092	0.682943679235329\\
1.305	0.405569970164763	0.804147337016097	0.678558662193044\\
1.305	0.418891418885081	0.813826036051075	0.674343787412871\\
1.305	0.432351961217133	0.823271342915545	0.670253187715552\\
1.305	0.445945513182735	0.832480405557786	0.666246357336157\\
1.305	0.459665918841652	0.841450581281375	0.662288623474186\\
1.305	0.473506959189401	0.850179438693979	0.658351221673191\\
1.305	0.487462361096317	0.85866475920876	0.654411055137887\\
1.305	0.501525806262096	0.866904538099407	0.650450223206469\\
1.305	0.515690940160279	0.874896985111485	0.646455401874958\\
1.305	0.529951380947623	0.882640524634436	0.642417151055154\\
1.305	0.544300728313782	0.890133795440129	0.638329210947586\\
1.305	0.558732572247415	0.897375649995373	0.634187835387499\\
1.305	0.57324050169555	0.904365153357279	0.629991194998375\\
1.305	0.587818113093873	0.911101581661729	0.625738868900923\\
1.305	0.602459018746568	0.917584420216504	0.621431431633141\\
1.305	0.617156855035314	0.923813361211857	0.617070132480554\\
1.305	0.631905290438127	0.92978830106243	0.612656657833838\\
1.305	0.6466980333399	0.935509337395467	0.608192963377088\\
1.305	0.661528839617671	0.940976765701173	0.603681161495104\\
1.305	0.676391519984904	0.94619107566195	0.599123449738606\\
1.305	0.691279947080357	0.951152947177946	0.59452206790159\\
1.305	0.706188062288412	0.955863246106974	0.589879273662391\\
1.305	0.721109882279076	0.960323019737426	0.585197329319937\\
1.305	0.736039505257226	0.964533492013186	0.580478494543976\\
1.305	0.75097111691199	0.968496058529893	0.575725022018312\\
1.305	0.765898996058536	0.972212281322139	0.570939154289558\\
1.305	0.780817519965826	0.975683883461272	0.566123121053174\\
1.305	0.795721169365275	0.978912743483559	0.56127913660556\\
1.305	0.810604533136476	0.981900889668376	0.556409397399201\\
1.305	0.825462312667457	0.984650494185936	0.551516079695733\\
1.305	0.840289325888138	0.987163867133889	0.546601337319641\\
1.305	0.855080510976838	0.989443450481767	0.54166729951514\\
1.305	0.86983092974082	0.991491811941914	0.5367160689086\\
1.305	0.884535770672926	0.993311638785082	0.531749719578142\\
1.305	0.899190351687411	0.994905731618374	0.526770295231793\\
1.305	0.913790122539012	0.996276998142666	0.521779807495103\\
1.305	0.928330666930242	0.997428446906011	0.516780234308684\\
1.305	0.942807704312699	0.998363181068902	0.511773518435872\\
1.305	0.95721709138901	0.999084392196567	0.50676156608017\\
1.305	0.971554823322703	0.999595354092743	0.501746245611874\\
1.305	0.985817034663989	0.999899416688637	0.496729386403004\\
1.305	1	1	0.491669087346906\\
1.32	0	0	0.710195569525497\\
1.32	0.000100583311362513	0.0141829653360114	0.71240460418299\\
1.32	0.000404645907256436	0.0284451766772965	0.714630690536905\\
1.32	0.000915607803432999	0.0427829086109896	0.716916312064106\\
1.32	0.00163681893109844	0.057192295687301	0.719289019360234\\
1.32	0.00257155309398959	0.0716693330697584	0.721780367061016\\
1.32	0.00372300185733413	0.0862098774609879	0.724425655492835\\
1.32	0.00509426838162598	0.100809648312589	0.727263413850304\\
1.32	0.00668836121491816	0.115464229327074	0.73033458997498\\
1.32	0.00850818805808554	0.13016907025918	0.733681422514709\\
1.32	0.0105565495182326	0.144919489023162	0.737345986077549\\
1.32	0.0128361328661109	0.159710674111862	0.741368418705642\\
1.32	0.0153495058140643	0.174537687332543	0.745784862958201\\
1.32	0.0180991103316243	0.189395466863524	0.750625176092311\\
1.32	0.0210872565164405	0.204278830634725	0.755910489868999\\
1.32	0.0243161165387281	0.219182480034174	0.761650724664453\\
1.32	0.0277877186778607	0.234101003941464	0.767842183867616\\
1.32	0.0315039414701067	0.24902888308801	0.77446537092041\\
1.32	0.0354665079868145	0.263960494742775	0.781483180780606\\
1.32	0.0396769802625738	0.278890117720924	0.788839618268073\\
1.32	0.0441367538930258	0.293811937711588	0.796459186325544\\
1.32	0.0488470528220538	0.308720052919643	0.804247066923455\\
1.32	0.0538089243380495	0.323608480015097	0.81209018616195\\
1.32	0.0590232342988274	0.338471160382329	0.819859213932694\\
1.32	0.064490662604533	0.3533019666601	0.82741149906364\\
1.32	0.0702116989375697	0.368094709561873	0.834594885813505\\
1.32	0.0761866387881432	0.382843144964686	0.84125230028718\\
1.32	0.0824155797834956	0.397540981253432	0.847226939729081\\
1.32	0.0888984183382709	0.412181886906127	0.852367847905041\\
1.32	0.0956348466427212	0.42675949830445	0.856535620027829\\
1.32	0.102624350004627	0.441267427752584	0.859607954619617\\
1.32	0.109866204559871	0.455699271686218	0.861484760266622\\
1.32	0.117359475365564	0.470048619052377	0.862092534251542\\
1.32	0.125103014888515	0.484309059839721	0.861387758068369\\
1.32	0.133095461900593	0.498474193737904	0.85935910089082\\
1.32	0.14133524079124	0.512537638903683	0.856028283765735\\
1.32	0.149820561306021	0.526493040810599	0.851449530870179\\
1.32	0.158549418718625	0.540334081158348	0.845707614739506\\
1.32	0.167519594442214	0.554054486817265	0.838914584343148\\
1.32	0.176728657084455	0.567648038782867	0.831205342370836\\
1.32	0.186173963948925	0.581108581114919	0.822732305417527\\
1.32	0.195852662983903	0.594430029835236	0.813659432944306\\
1.32	0.205761695177907	0.607606381758231	0.804155944104871\\
1.32	0.215897797399558	0.620631723228143	0.794390053406973\\
1.32	0.226257505677663	0.633500238737009	0.7845230460628\\
1.32	0.236837158915675	0.646206219397571	0.774703982861591\\
1.32	0.247632903032949	0.658744071245709	0.765065275211016\\
1.32	0.258640695523527	0.671108323347402	0.755719307772962\\
1.32	0.269856310421543	0.683293635685828	0.746756214005912\\
1.32	0.28127534366066	0.695294806804925	0.738242834604296\\
1.32	0.292893218813452	0.707106781186547	0.730222816007834\\
1.32	0.304705193195075	0.71872465633934	0.722717741087768\\
1.32	0.316706364314172	0.730143689578457	0.715729131143267\\
1.32	0.328891676652598	0.741359304476472	0.709241120561855\\
1.32	0.341255928754291	0.752367096967051	0.703223584571265\\
1.32	0.353793780602429	0.763162841084325	0.697635496597419\\
1.32	0.366499761262991	0.773742494322337	0.692428303594618\\
1.32	0.379368276771856	0.784102202600442	0.687549132887069\\
1.32	0.392393618241769	0.794238304822092	0.682943679235329\\
1.32	0.405569970164763	0.804147337016097	0.678558662193045\\
1.32	0.418891418885081	0.813826036051075	0.674343787412871\\
1.32	0.432351961217132	0.823271342915544	0.670253187715552\\
1.32	0.445945513182735	0.832480405557786	0.666246357336158\\
1.32	0.459665918841652	0.841450581281375	0.662288623474186\\
1.32	0.473506959189401	0.850179438693979	0.658351221673191\\
1.32	0.487462361096317	0.85866475920876	0.654411055137886\\
1.32	0.501525806262096	0.866904538099407	0.650450223206469\\
1.32	0.515690940160279	0.874896985111485	0.646455401874959\\
1.32	0.529951380947623	0.882640524634436	0.642417151055154\\
1.32	0.544300728313782	0.890133795440129	0.638329210947586\\
1.32	0.558732572247415	0.897375649995373	0.634187835387497\\
1.32	0.57324050169555	0.904365153357279	0.629991194998376\\
1.32	0.587818113093873	0.911101581661729	0.625738868900924\\
1.32	0.602459018746568	0.917584420216504	0.621431431633141\\
1.32	0.617156855035314	0.923813361211857	0.617070132480554\\
1.32	0.631905290438127	0.92978830106243	0.612656657833837\\
1.32	0.6466980333399	0.935509337395467	0.608192963377087\\
1.32	0.661528839617671	0.940976765701173	0.603681161495104\\
1.32	0.676391519984904	0.94619107566195	0.599123449738606\\
1.32	0.691279947080357	0.951152947177946	0.59452206790159\\
1.32	0.706188062288412	0.955863246106974	0.589879273662389\\
1.32	0.721109882279076	0.960323019737426	0.58519732931994\\
1.32	0.736039505257226	0.964533492013186	0.580478494543978\\
1.32	0.75097111691199	0.968496058529893	0.575725022018311\\
1.32	0.765898996058536	0.972212281322139	0.570939154289558\\
1.32	0.780817519965826	0.975683883461272	0.566123121053175\\
1.32	0.795721169365275	0.97891274348356	0.561279136605559\\
1.32	0.810604533136476	0.981900889668376	0.556409397399198\\
1.32	0.825462312667457	0.984650494185936	0.551516079695734\\
1.32	0.840289325888138	0.987163867133889	0.546601337319642\\
1.32	0.855080510976839	0.989443450481767	0.54166729951514\\
1.32	0.86983092974082	0.991491811941914	0.536716068908599\\
1.32	0.884535770672926	0.993311638785082	0.531749719578144\\
1.32	0.899190351687411	0.994905731618374	0.526770295231793\\
1.32	0.913790122539012	0.996276998142666	0.521779807495102\\
1.32	0.928330666930242	0.997428446906011	0.516780234308684\\
1.32	0.942807704312699	0.998363181068902	0.51177351843587\\
1.32	0.95721709138901	0.999084392196567	0.50676156608017\\
1.32	0.971554823322703	0.999595354092743	0.501746245611876\\
1.32	0.985817034663989	0.999899416688637	0.496729386403004\\
1.32	1	1	0.491669087346906\\
1.335	0	0	0.710195569525497\\
1.335	0.000100583311362513	0.0141829653360114	0.71240460418299\\
1.335	0.000404645907256436	0.0284451766772965	0.714630690536905\\
1.335	0.000915607803432999	0.0427829086109896	0.716916312064106\\
1.335	0.00163681893109844	0.057192295687301	0.719289019360234\\
1.335	0.00257155309398959	0.0716693330697584	0.721780367061016\\
1.335	0.00372300185733413	0.0862098774609879	0.724425655492835\\
1.335	0.00509426838162598	0.100809648312589	0.727263413850304\\
1.335	0.00668836121491816	0.115464229327074	0.73033458997498\\
1.335	0.00850818805808555	0.13016907025918	0.733681422514709\\
1.335	0.0105565495182326	0.144919489023162	0.737345986077549\\
1.335	0.0128361328661109	0.159710674111862	0.741368418705642\\
1.335	0.0153495058140643	0.174537687332543	0.745784862958201\\
1.335	0.0180991103316243	0.189395466863524	0.750625176092311\\
1.335	0.0210872565164405	0.204278830634725	0.755910489868999\\
1.335	0.0243161165387281	0.219182480034174	0.761650724664453\\
1.335	0.0277877186778607	0.234101003941464	0.767842183867616\\
1.335	0.0315039414701067	0.24902888308801	0.77446537092041\\
1.335	0.0354665079868145	0.263960494742775	0.781483180780606\\
1.335	0.0396769802625738	0.278890117720924	0.788839618268073\\
1.335	0.0441367538930258	0.293811937711588	0.796459186325544\\
1.335	0.0488470528220537	0.308720052919643	0.804247066923455\\
1.335	0.0538089243380495	0.323608480015097	0.812090186161949\\
1.335	0.0590232342988274	0.338471160382329	0.819859213932693\\
1.335	0.064490662604533	0.3533019666601	0.827411499063639\\
1.335	0.0702116989375697	0.368094709561873	0.834594885813505\\
1.335	0.0761866387881432	0.382843144964686	0.84125230028718\\
1.335	0.0824155797834956	0.397540981253432	0.847226939729082\\
1.335	0.0888984183382709	0.412181886906127	0.852367847905042\\
1.335	0.0956348466427212	0.42675949830445	0.856535620027829\\
1.335	0.102624350004627	0.441267427752585	0.859607954619618\\
1.335	0.109866204559871	0.455699271686218	0.861484760266622\\
1.335	0.117359475365564	0.470048619052377	0.86209253425154\\
1.335	0.125103014888515	0.484309059839721	0.861387758068369\\
1.335	0.133095461900593	0.498474193737904	0.859359100890821\\
1.335	0.14133524079124	0.512537638903683	0.856028283765734\\
1.335	0.149820561306021	0.526493040810599	0.851449530870182\\
1.335	0.158549418718625	0.540334081158348	0.845707614739509\\
1.335	0.167519594442213	0.554054486817265	0.838914584343151\\
1.335	0.176728657084455	0.567648038782867	0.83120534237084\\
1.335	0.186173963948925	0.581108581114919	0.822732305417527\\
1.335	0.195852662983903	0.594430029835236	0.81365943294431\\
1.335	0.205761695177907	0.607606381758231	0.804155944104872\\
1.335	0.215897797399558	0.620631723228144	0.79439005340697\\
1.335	0.226257505677663	0.633500238737009	0.784523046062799\\
1.335	0.236837158915675	0.646206219397571	0.774703982861591\\
1.335	0.247632903032949	0.658744071245709	0.765065275211015\\
1.335	0.258640695523527	0.671108323347402	0.755719307772962\\
1.335	0.269856310421543	0.683293635685828	0.746756214005912\\
1.335	0.28127534366066	0.695294806804924	0.738242834604296\\
1.335	0.292893218813452	0.707106781186547	0.730222816007833\\
1.335	0.304705193195075	0.71872465633934	0.722717741087769\\
1.335	0.316706364314172	0.730143689578457	0.715729131143268\\
1.335	0.328891676652598	0.741359304476472	0.709241120561856\\
1.335	0.341255928754291	0.752367096967051	0.703223584571265\\
1.335	0.353793780602429	0.763162841084325	0.69763549659742\\
1.335	0.366499761262991	0.773742494322337	0.692428303594617\\
1.335	0.379368276771857	0.784102202600442	0.687549132887068\\
1.335	0.392393618241769	0.794238304822092	0.682943679235329\\
1.335	0.405569970164763	0.804147337016097	0.678558662193044\\
1.335	0.418891418885081	0.813826036051075	0.67434378741287\\
1.335	0.432351961217132	0.823271342915544	0.670253187715553\\
1.335	0.445945513182735	0.832480405557786	0.666246357336158\\
1.335	0.459665918841652	0.841450581281375	0.662288623474186\\
1.335	0.473506959189401	0.850179438693979	0.658351221673192\\
1.335	0.487462361096317	0.85866475920876	0.654411055137886\\
1.335	0.501525806262096	0.866904538099407	0.650450223206468\\
1.335	0.515690940160279	0.874896985111485	0.646455401874959\\
1.335	0.529951380947623	0.882640524634436	0.642417151055155\\
1.335	0.544300728313782	0.890133795440129	0.638329210947586\\
1.335	0.558732572247415	0.897375649995373	0.634187835387499\\
1.335	0.57324050169555	0.904365153357279	0.629991194998375\\
1.335	0.587818113093873	0.911101581661729	0.625738868900922\\
1.335	0.602459018746568	0.917584420216504	0.62143143163314\\
1.335	0.617156855035314	0.923813361211857	0.617070132480554\\
1.335	0.631905290438127	0.92978830106243	0.612656657833839\\
1.335	0.6466980333399	0.935509337395467	0.608192963377087\\
1.335	0.661528839617671	0.940976765701173	0.603681161495105\\
1.335	0.676391519984903	0.946191075661951	0.599123449738605\\
1.335	0.691279947080357	0.951152947177946	0.594522067901588\\
1.335	0.706188062288412	0.955863246106974	0.58987927366239\\
1.335	0.721109882279076	0.960323019737426	0.585197329319937\\
1.335	0.736039505257226	0.964533492013186	0.580478494543978\\
1.335	0.75097111691199	0.968496058529893	0.575725022018312\\
1.335	0.765898996058536	0.972212281322139	0.570939154289558\\
1.335	0.780817519965826	0.975683883461272	0.566123121053176\\
1.335	0.795721169365275	0.97891274348356	0.56127913660556\\
1.335	0.810604533136476	0.981900889668376	0.556409397399199\\
1.335	0.825462312667457	0.984650494185936	0.551516079695733\\
1.335	0.840289325888138	0.987163867133889	0.546601337319639\\
1.335	0.855080510976839	0.989443450481767	0.541667299515139\\
1.335	0.86983092974082	0.991491811941914	0.536716068908598\\
1.335	0.884535770672926	0.993311638785082	0.531749719578143\\
1.335	0.899190351687411	0.994905731618374	0.526770295231798\\
1.335	0.913790122539012	0.996276998142666	0.521779807495101\\
1.335	0.928330666930242	0.99742844690601	0.516780234308683\\
1.335	0.942807704312699	0.998363181068902	0.511773518435872\\
1.335	0.95721709138901	0.999084392196567	0.506761566080169\\
1.335	0.971554823322704	0.999595354092744	0.501746245611877\\
1.335	0.985817034663989	0.999899416688637	0.496729386403002\\
1.335	1	1	0.491669087346908\\
1.35	0	0	0.710195569525497\\
1.35	0.000100583311362513	0.0141829653360114	0.71240460418299\\
1.35	0.000404645907256436	0.0284451766772965	0.714630690536905\\
1.35	0.000915607803432999	0.0427829086109896	0.716916312064106\\
1.35	0.00163681893109844	0.057192295687301	0.719289019360234\\
1.35	0.00257155309398959	0.0716693330697584	0.721780367061016\\
1.35	0.00372300185733414	0.0862098774609879	0.724425655492835\\
1.35	0.00509426838162598	0.100809648312589	0.727263413850304\\
1.35	0.00668836121491816	0.115464229327074	0.73033458997498\\
1.35	0.00850818805808555	0.13016907025918	0.733681422514709\\
1.35	0.0105565495182326	0.144919489023162	0.737345986077549\\
1.35	0.0128361328661109	0.159710674111862	0.741368418705642\\
1.35	0.0153495058140643	0.174537687332543	0.745784862958202\\
1.35	0.0180991103316243	0.189395466863524	0.750625176092311\\
1.35	0.0210872565164405	0.204278830634725	0.755910489868999\\
1.35	0.0243161165387281	0.219182480034174	0.761650724664453\\
1.35	0.0277877186778607	0.234101003941464	0.767842183867616\\
1.35	0.0315039414701067	0.24902888308801	0.77446537092041\\
1.35	0.0354665079868145	0.263960494742775	0.781483180780606\\
1.35	0.0396769802625738	0.278890117720924	0.788839618268073\\
1.35	0.0441367538930258	0.293811937711588	0.796459186325544\\
1.35	0.0488470528220538	0.308720052919643	0.804247066923455\\
1.35	0.0538089243380495	0.323608480015096	0.81209018616195\\
1.35	0.0590232342988274	0.338471160382329	0.819859213932692\\
1.35	0.064490662604533	0.3533019666601	0.827411499063639\\
1.35	0.0702116989375697	0.368094709561873	0.834594885813505\\
1.35	0.0761866387881432	0.382843144964686	0.841252300287177\\
1.35	0.0824155797834956	0.397540981253432	0.847226939729082\\
1.35	0.0888984183382709	0.412181886906127	0.852367847905041\\
1.35	0.0956348466427212	0.42675949830445	0.85653562002783\\
1.35	0.102624350004627	0.441267427752584	0.859607954619617\\
1.35	0.109866204559871	0.455699271686218	0.861484760266622\\
1.35	0.117359475365564	0.470048619052377	0.862092534251541\\
1.35	0.125103014888515	0.484309059839721	0.861387758068369\\
1.35	0.133095461900593	0.498474193737904	0.859359100890821\\
1.35	0.14133524079124	0.512537638903683	0.856028283765738\\
1.35	0.149820561306021	0.526493040810599	0.851449530870183\\
1.35	0.158549418718625	0.540334081158348	0.845707614739504\\
1.35	0.167519594442214	0.554054486817265	0.838914584343153\\
1.35	0.176728657084455	0.567648038782867	0.831205342370838\\
1.35	0.186173963948925	0.581108581114919	0.822732305417528\\
1.35	0.195852662983903	0.594430029835236	0.813659432944307\\
1.35	0.205761695177907	0.607606381758231	0.80415594410487\\
1.35	0.215897797399558	0.620631723228143	0.794390053406969\\
1.35	0.226257505677663	0.633500238737009	0.784523046062804\\
1.35	0.236837158915675	0.646206219397571	0.774703982861592\\
1.35	0.247632903032949	0.658744071245709	0.765065275211017\\
1.35	0.258640695523527	0.671108323347402	0.755719307772961\\
1.35	0.269856310421543	0.683293635685828	0.746756214005914\\
1.35	0.28127534366066	0.695294806804925	0.738242834604294\\
1.35	0.292893218813452	0.707106781186547	0.730222816007834\\
1.35	0.304705193195075	0.71872465633934	0.722717741087769\\
1.35	0.316706364314172	0.730143689578457	0.715729131143267\\
1.35	0.328891676652598	0.741359304476472	0.709241120561855\\
1.35	0.341255928754291	0.752367096967051	0.703223584571264\\
1.35	0.353793780602429	0.763162841084325	0.697635496597419\\
1.35	0.366499761262991	0.773742494322337	0.692428303594619\\
1.35	0.379368276771856	0.784102202600442	0.687549132887069\\
1.35	0.392393618241769	0.794238304822092	0.682943679235329\\
1.35	0.405569970164763	0.804147337016097	0.678558662193044\\
1.35	0.418891418885081	0.813826036051075	0.674343787412869\\
1.35	0.432351961217132	0.823271342915544	0.670253187715552\\
1.35	0.445945513182735	0.832480405557786	0.666246357336159\\
1.35	0.459665918841652	0.841450581281375	0.662288623474186\\
1.35	0.473506959189401	0.850179438693979	0.658351221673191\\
1.35	0.487462361096317	0.85866475920876	0.654411055137887\\
1.35	0.501525806262096	0.866904538099407	0.650450223206469\\
1.35	0.515690940160279	0.874896985111485	0.646455401874959\\
1.35	0.529951380947623	0.882640524634436	0.642417151055154\\
1.35	0.544300728313782	0.890133795440129	0.638329210947586\\
1.35	0.558732572247415	0.897375649995373	0.634187835387499\\
1.35	0.57324050169555	0.904365153357279	0.629991194998376\\
1.35	0.587818113093873	0.911101581661729	0.625738868900923\\
1.35	0.602459018746568	0.917584420216504	0.62143143163314\\
1.35	0.617156855035314	0.923813361211857	0.617070132480554\\
1.35	0.631905290438127	0.92978830106243	0.612656657833837\\
1.35	0.6466980333399	0.935509337395467	0.608192963377088\\
1.35	0.661528839617671	0.940976765701173	0.603681161495106\\
1.35	0.676391519984903	0.94619107566195	0.599123449738607\\
1.35	0.691279947080357	0.951152947177946	0.594522067901588\\
1.35	0.706188062288412	0.955863246106974	0.589879273662389\\
1.35	0.721109882279076	0.960323019737426	0.585197329319937\\
1.35	0.736039505257226	0.964533492013186	0.580478494543978\\
1.35	0.75097111691199	0.968496058529893	0.575725022018313\\
1.35	0.765898996058536	0.972212281322139	0.570939154289558\\
1.35	0.780817519965826	0.975683883461272	0.566123121053176\\
1.35	0.795721169365275	0.97891274348356	0.561279136605559\\
1.35	0.810604533136476	0.981900889668376	0.556409397399199\\
1.35	0.825462312667457	0.984650494185936	0.551516079695734\\
1.35	0.840289325888138	0.987163867133889	0.546601337319641\\
1.35	0.855080510976839	0.989443450481767	0.541667299515139\\
1.35	0.86983092974082	0.991491811941914	0.536716068908599\\
1.35	0.884535770672926	0.993311638785082	0.531749719578142\\
1.35	0.899190351687411	0.994905731618374	0.526770295231792\\
1.35	0.913790122539012	0.996276998142666	0.521779807495102\\
1.35	0.928330666930242	0.997428446906011	0.516780234308683\\
1.35	0.942807704312699	0.998363181068902	0.511773518435873\\
1.35	0.95721709138901	0.999084392196567	0.506761566080167\\
1.35	0.971554823322703	0.999595354092743	0.501746245611874\\
1.35	0.985817034663989	0.999899416688637	0.496729386403005\\
1.35	1	1	0.491669087346904\\
1.365	0	0	0.710195569525497\\
1.365	0.000100583311362513	0.0141829653360114	0.71240460418299\\
1.365	0.000404645907256436	0.0284451766772965	0.714630690536905\\
1.365	0.000915607803432999	0.0427829086109896	0.716916312064106\\
1.365	0.00163681893109844	0.057192295687301	0.719289019360234\\
1.365	0.00257155309398959	0.0716693330697584	0.721780367061016\\
1.365	0.00372300185733413	0.0862098774609879	0.724425655492835\\
1.365	0.00509426838162598	0.100809648312589	0.727263413850304\\
1.365	0.00668836121491816	0.115464229327074	0.73033458997498\\
1.365	0.00850818805808554	0.13016907025918	0.733681422514709\\
1.365	0.0105565495182326	0.144919489023162	0.737345986077549\\
1.365	0.0128361328661109	0.159710674111862	0.741368418705642\\
1.365	0.0153495058140643	0.174537687332543	0.745784862958201\\
1.365	0.0180991103316243	0.189395466863524	0.750625176092311\\
1.365	0.0210872565164405	0.204278830634725	0.755910489868999\\
1.365	0.0243161165387281	0.219182480034174	0.761650724664453\\
1.365	0.0277877186778607	0.234101003941464	0.767842183867616\\
1.365	0.0315039414701067	0.24902888308801	0.77446537092041\\
1.365	0.0354665079868145	0.263960494742775	0.781483180780606\\
1.365	0.0396769802625738	0.278890117720924	0.788839618268073\\
1.365	0.0441367538930258	0.293811937711588	0.796459186325543\\
1.365	0.0488470528220538	0.308720052919643	0.804247066923455\\
1.365	0.0538089243380495	0.323608480015097	0.812090186161951\\
1.365	0.0590232342988274	0.338471160382329	0.819859213932694\\
1.365	0.064490662604533	0.3533019666601	0.827411499063639\\
1.365	0.0702116989375697	0.368094709561873	0.834594885813505\\
1.365	0.0761866387881432	0.382843144964686	0.841252300287177\\
1.365	0.0824155797834956	0.397540981253432	0.84722693972908\\
1.365	0.0888984183382709	0.412181886906127	0.852367847905041\\
1.365	0.0956348466427212	0.42675949830445	0.856535620027828\\
1.365	0.102624350004627	0.441267427752585	0.859607954619616\\
1.365	0.109866204559871	0.455699271686218	0.861484760266621\\
1.365	0.117359475365564	0.470048619052377	0.86209253425154\\
1.365	0.125103014888515	0.484309059839721	0.861387758068368\\
1.365	0.133095461900593	0.498474193737904	0.85935910089082\\
1.365	0.14133524079124	0.512537638903683	0.856028283765736\\
1.365	0.149820561306021	0.526493040810599	0.851449530870182\\
1.365	0.158549418718625	0.540334081158348	0.845707614739506\\
1.365	0.167519594442213	0.554054486817265	0.838914584343153\\
1.365	0.176728657084455	0.567648038782867	0.831205342370835\\
1.365	0.186173963948925	0.581108581114919	0.822732305417527\\
1.365	0.195852662983903	0.594430029835236	0.813659432944307\\
1.365	0.205761695177907	0.607606381758231	0.804155944104871\\
1.365	0.215897797399558	0.620631723228144	0.79439005340697\\
1.365	0.226257505677663	0.633500238737009	0.784523046062799\\
1.365	0.236837158915675	0.646206219397571	0.77470398286159\\
1.365	0.247632903032949	0.658744071245709	0.765065275211014\\
1.365	0.258640695523528	0.671108323347402	0.755719307772961\\
1.365	0.269856310421543	0.683293635685828	0.746756214005912\\
1.365	0.28127534366066	0.695294806804924	0.738242834604296\\
1.365	0.292893218813452	0.707106781186547	0.730222816007835\\
1.365	0.304705193195075	0.71872465633934	0.722717741087768\\
1.365	0.316706364314172	0.730143689578457	0.715729131143268\\
1.365	0.328891676652598	0.741359304476472	0.709241120561854\\
1.365	0.341255928754291	0.752367096967051	0.703223584571265\\
1.365	0.353793780602429	0.763162841084325	0.69763549659742\\
1.365	0.366499761262991	0.773742494322337	0.692428303594619\\
1.365	0.379368276771857	0.784102202600443	0.687549132887069\\
1.365	0.392393618241769	0.794238304822092	0.682943679235329\\
1.365	0.405569970164763	0.804147337016097	0.678558662193045\\
1.365	0.418891418885081	0.813826036051075	0.674343787412871\\
1.365	0.432351961217133	0.823271342915545	0.670253187715552\\
1.365	0.445945513182735	0.832480405557786	0.666246357336158\\
1.365	0.459665918841652	0.841450581281375	0.662288623474185\\
1.365	0.473506959189401	0.850179438693979	0.658351221673191\\
1.365	0.487462361096317	0.85866475920876	0.654411055137886\\
1.365	0.501525806262096	0.866904538099407	0.650450223206469\\
1.365	0.515690940160279	0.874896985111485	0.64645540187496\\
1.365	0.529951380947623	0.882640524634437	0.642417151055155\\
1.365	0.544300728313782	0.890133795440129	0.638329210947585\\
1.365	0.558732572247415	0.897375649995373	0.634187835387499\\
1.365	0.57324050169555	0.904365153357279	0.629991194998376\\
1.365	0.587818113093873	0.911101581661729	0.625738868900924\\
1.365	0.602459018746568	0.917584420216504	0.62143143163314\\
1.365	0.617156855035314	0.923813361211857	0.617070132480553\\
1.365	0.631905290438127	0.92978830106243	0.612656657833838\\
1.365	0.6466980333399	0.935509337395467	0.608192963377086\\
1.365	0.661528839617671	0.940976765701173	0.603681161495105\\
1.365	0.676391519984903	0.94619107566195	0.599123449738609\\
1.365	0.691279947080357	0.951152947177946	0.59452206790159\\
1.365	0.706188062288412	0.955863246106974	0.589879273662389\\
1.365	0.721109882279076	0.960323019737426	0.585197329319937\\
1.365	0.736039505257226	0.964533492013186	0.580478494543976\\
1.365	0.75097111691199	0.968496058529893	0.575725022018312\\
1.365	0.765898996058536	0.972212281322139	0.570939154289558\\
1.365	0.780817519965826	0.975683883461272	0.566123121053177\\
1.365	0.795721169365275	0.97891274348356	0.561279136605561\\
1.365	0.810604533136476	0.981900889668376	0.556409397399198\\
1.365	0.825462312667457	0.984650494185936	0.551516079695734\\
1.365	0.840289325888138	0.987163867133889	0.546601337319641\\
1.365	0.855080510976839	0.989443450481768	0.541667299515142\\
1.365	0.86983092974082	0.991491811941915	0.536716068908598\\
1.365	0.884535770672926	0.993311638785082	0.531749719578145\\
1.365	0.899190351687411	0.994905731618374	0.526770295231796\\
1.365	0.913790122539012	0.996276998142666	0.521779807495099\\
1.365	0.928330666930242	0.997428446906011	0.516780234308682\\
1.365	0.942807704312699	0.998363181068902	0.511773518435872\\
1.365	0.95721709138901	0.999084392196567	0.506761566080171\\
1.365	0.971554823322703	0.999595354092744	0.501746245611876\\
1.365	0.985817034663989	0.999899416688638	0.496729386403001\\
1.365	1	1	0.491669087346898\\
1.38	0	0	0.710195569525497\\
1.38	0.000100583311362513	0.0141829653360114	0.71240460418299\\
1.38	0.000404645907256436	0.0284451766772965	0.714630690536905\\
1.38	0.000915607803432999	0.0427829086109896	0.716916312064106\\
1.38	0.00163681893109844	0.057192295687301	0.719289019360234\\
1.38	0.00257155309398959	0.0716693330697584	0.721780367061016\\
1.38	0.00372300185733413	0.0862098774609879	0.724425655492835\\
1.38	0.00509426838162598	0.100809648312589	0.727263413850304\\
1.38	0.00668836121491815	0.115464229327074	0.73033458997498\\
1.38	0.00850818805808555	0.13016907025918	0.733681422514709\\
1.38	0.0105565495182326	0.144919489023162	0.737345986077549\\
1.38	0.0128361328661109	0.159710674111862	0.741368418705642\\
1.38	0.0153495058140643	0.174537687332543	0.745784862958201\\
1.38	0.0180991103316243	0.189395466863524	0.750625176092311\\
1.38	0.0210872565164405	0.204278830634725	0.755910489868998\\
1.38	0.0243161165387281	0.219182480034174	0.761650724664453\\
1.38	0.0277877186778607	0.234101003941464	0.767842183867616\\
1.38	0.0315039414701067	0.24902888308801	0.77446537092041\\
1.38	0.0354665079868146	0.263960494742775	0.781483180780606\\
1.38	0.0396769802625738	0.278890117720924	0.788839618268073\\
1.38	0.0441367538930258	0.293811937711588	0.796459186325544\\
1.38	0.0488470528220538	0.308720052919643	0.804247066923455\\
1.38	0.0538089243380495	0.323608480015096	0.81209018616195\\
1.38	0.0590232342988274	0.338471160382329	0.819859213932694\\
1.38	0.064490662604533	0.3533019666601	0.827411499063639\\
1.38	0.0702116989375697	0.368094709561873	0.834594885813505\\
1.38	0.0761866387881432	0.382843144964686	0.841252300287179\\
1.38	0.0824155797834956	0.397540981253432	0.847226939729081\\
1.38	0.0888984183382709	0.412181886906127	0.852367847905042\\
1.38	0.0956348466427212	0.42675949830445	0.856535620027828\\
1.38	0.102624350004627	0.441267427752584	0.859607954619617\\
1.38	0.109866204559871	0.455699271686218	0.861484760266623\\
1.38	0.117359475365564	0.470048619052377	0.862092534251538\\
1.38	0.125103014888515	0.484309059839721	0.861387758068368\\
1.38	0.133095461900593	0.498474193737904	0.859359100890821\\
1.38	0.14133524079124	0.512537638903683	0.856028283765735\\
1.38	0.149820561306021	0.526493040810599	0.851449530870183\\
1.38	0.158549418718625	0.540334081158348	0.845707614739505\\
1.38	0.167519594442214	0.554054486817266	0.83891458434315\\
1.38	0.176728657084455	0.567648038782867	0.831205342370835\\
1.38	0.186173963948925	0.581108581114919	0.822732305417526\\
1.38	0.195852662983903	0.594430029835236	0.813659432944306\\
1.38	0.205761695177907	0.607606381758231	0.804155944104868\\
1.38	0.215897797399558	0.620631723228143	0.794390053406966\\
1.38	0.226257505677663	0.633500238737009	0.784523046062802\\
1.38	0.236837158915675	0.646206219397571	0.774703982861592\\
1.38	0.247632903032949	0.658744071245709	0.765065275211014\\
1.38	0.258640695523527	0.671108323347402	0.755719307772962\\
1.38	0.269856310421543	0.683293635685828	0.746756214005913\\
1.38	0.28127534366066	0.695294806804925	0.738242834604297\\
1.38	0.292893218813452	0.707106781186547	0.730222816007833\\
1.38	0.304705193195075	0.71872465633934	0.722717741087767\\
1.38	0.316706364314172	0.730143689578457	0.715729131143268\\
1.38	0.328891676652598	0.741359304476472	0.709241120561856\\
1.38	0.341255928754291	0.752367096967051	0.703223584571265\\
1.38	0.353793780602429	0.763162841084325	0.697635496597419\\
1.38	0.366499761262991	0.773742494322337	0.692428303594618\\
1.38	0.379368276771857	0.784102202600442	0.687549132887069\\
1.38	0.392393618241769	0.794238304822092	0.682943679235329\\
1.38	0.405569970164763	0.804147337016097	0.678558662193045\\
1.38	0.418891418885081	0.813826036051075	0.67434378741287\\
1.38	0.432351961217132	0.823271342915544	0.670253187715552\\
1.38	0.445945513182735	0.832480405557786	0.666246357336159\\
1.38	0.459665918841652	0.841450581281375	0.662288623474186\\
1.38	0.473506959189401	0.850179438693979	0.658351221673191\\
1.38	0.487462361096317	0.85866475920876	0.654411055137885\\
1.38	0.501525806262096	0.866904538099407	0.650450223206469\\
1.38	0.515690940160279	0.874896985111485	0.646455401874959\\
1.38	0.529951380947623	0.882640524634436	0.642417151055155\\
1.38	0.544300728313782	0.890133795440129	0.638329210947586\\
1.38	0.558732572247415	0.897375649995373	0.634187835387497\\
1.38	0.57324050169555	0.904365153357279	0.629991194998374\\
1.38	0.587818113093873	0.911101581661729	0.625738868900924\\
1.38	0.602459018746568	0.917584420216504	0.621431431633142\\
1.38	0.617156855035314	0.923813361211857	0.617070132480553\\
1.38	0.631905290438127	0.92978830106243	0.612656657833837\\
1.38	0.6466980333399	0.935509337395467	0.608192963377087\\
1.38	0.661528839617671	0.940976765701173	0.603681161495104\\
1.38	0.676391519984904	0.94619107566195	0.599123449738606\\
1.38	0.691279947080357	0.951152947177946	0.59452206790159\\
1.38	0.706188062288412	0.955863246106974	0.589879273662391\\
1.38	0.721109882279076	0.960323019737426	0.585197329319937\\
1.38	0.736039505257226	0.964533492013186	0.580478494543977\\
1.38	0.75097111691199	0.968496058529893	0.575725022018312\\
1.38	0.765898996058536	0.972212281322139	0.570939154289558\\
1.38	0.780817519965826	0.975683883461272	0.566123121053177\\
1.38	0.795721169365275	0.97891274348356	0.56127913660556\\
1.38	0.810604533136476	0.981900889668376	0.556409397399195\\
1.38	0.825462312667457	0.984650494185936	0.551516079695732\\
1.38	0.840289325888138	0.987163867133889	0.546601337319643\\
1.38	0.855080510976839	0.989443450481767	0.541667299515141\\
1.38	0.86983092974082	0.991491811941914	0.536716068908598\\
1.38	0.884535770672926	0.993311638785082	0.531749719578143\\
1.38	0.899190351687411	0.994905731618374	0.526770295231795\\
1.38	0.913790122539012	0.996276998142666	0.521779807495103\\
1.38	0.928330666930242	0.997428446906011	0.516780234308684\\
1.38	0.942807704312699	0.998363181068902	0.511773518435872\\
1.38	0.95721709138901	0.999084392196567	0.50676156608017\\
1.38	0.971554823322703	0.999595354092743	0.501746245611877\\
1.38	0.985817034663989	0.999899416688638	0.496729386403004\\
1.38	1	1	0.491669087346897\\
1.395	0	0	0.710195569525497\\
1.395	0.000100583311362513	0.0141829653360114	0.71240460418299\\
1.395	0.000404645907256436	0.0284451766772965	0.714630690536905\\
1.395	0.000915607803432999	0.0427829086109896	0.716916312064106\\
1.395	0.00163681893109844	0.057192295687301	0.719289019360234\\
1.395	0.00257155309398959	0.0716693330697584	0.721780367061016\\
1.395	0.00372300185733414	0.0862098774609879	0.724425655492835\\
1.395	0.00509426838162598	0.100809648312589	0.727263413850304\\
1.395	0.00668836121491816	0.115464229327074	0.73033458997498\\
1.395	0.00850818805808554	0.13016907025918	0.733681422514709\\
1.395	0.0105565495182326	0.144919489023162	0.737345986077549\\
1.395	0.0128361328661109	0.159710674111862	0.741368418705642\\
1.395	0.0153495058140643	0.174537687332543	0.745784862958202\\
1.395	0.0180991103316243	0.189395466863524	0.750625176092311\\
1.395	0.0210872565164405	0.204278830634725	0.755910489868999\\
1.395	0.0243161165387281	0.219182480034174	0.761650724664453\\
1.395	0.0277877186778607	0.234101003941464	0.767842183867616\\
1.395	0.0315039414701067	0.24902888308801	0.77446537092041\\
1.395	0.0354665079868145	0.263960494742775	0.781483180780606\\
1.395	0.0396769802625738	0.278890117720924	0.788839618268074\\
1.395	0.0441367538930258	0.293811937711588	0.796459186325544\\
1.395	0.0488470528220538	0.308720052919643	0.804247066923454\\
1.395	0.0538089243380495	0.323608480015097	0.812090186161951\\
1.395	0.0590232342988274	0.338471160382329	0.819859213932693\\
1.395	0.064490662604533	0.3533019666601	0.82741149906364\\
1.395	0.0702116989375697	0.368094709561873	0.834594885813505\\
1.395	0.0761866387881432	0.382843144964686	0.841252300287179\\
1.395	0.0824155797834956	0.397540981253432	0.847226939729081\\
1.395	0.0888984183382709	0.412181886906127	0.852367847905042\\
1.395	0.0956348466427212	0.42675949830445	0.85653562002783\\
1.395	0.102624350004627	0.441267427752585	0.859607954619619\\
1.395	0.109866204559871	0.455699271686218	0.86148476026662\\
1.395	0.117359475365564	0.470048619052377	0.862092534251543\\
1.395	0.125103014888515	0.484309059839721	0.86138775806837\\
1.395	0.133095461900593	0.498474193737904	0.859359100890822\\
1.395	0.14133524079124	0.512537638903683	0.856028283765736\\
1.395	0.149820561306021	0.526493040810599	0.851449530870181\\
1.395	0.158549418718625	0.540334081158348	0.845707614739506\\
1.395	0.167519594442214	0.554054486817265	0.838914584343147\\
1.395	0.176728657084455	0.567648038782867	0.831205342370836\\
1.395	0.186173963948925	0.581108581114919	0.822732305417527\\
1.395	0.195852662983903	0.594430029835236	0.813659432944308\\
1.395	0.205761695177907	0.607606381758231	0.804155944104869\\
1.395	0.215897797399558	0.620631723228143	0.79439005340697\\
1.395	0.226257505677663	0.633500238737009	0.784523046062805\\
1.395	0.236837158915675	0.646206219397571	0.774703982861589\\
1.395	0.247632903032949	0.658744071245709	0.765065275211015\\
1.395	0.258640695523527	0.671108323347402	0.755719307772962\\
1.395	0.269856310421543	0.683293635685828	0.746756214005912\\
1.395	0.28127534366066	0.695294806804925	0.738242834604295\\
1.395	0.292893218813452	0.707106781186547	0.730222816007833\\
1.395	0.304705193195075	0.71872465633934	0.722717741087767\\
1.395	0.316706364314172	0.730143689578457	0.715729131143268\\
1.395	0.328891676652598	0.741359304476472	0.709241120561856\\
1.395	0.341255928754291	0.752367096967051	0.703223584571265\\
1.395	0.353793780602429	0.763162841084325	0.697635496597419\\
1.395	0.366499761262991	0.773742494322337	0.692428303594618\\
1.395	0.379368276771857	0.784102202600443	0.687549132887069\\
1.395	0.392393618241769	0.794238304822092	0.682943679235329\\
1.395	0.405569970164763	0.804147337016097	0.678558662193045\\
1.395	0.418891418885081	0.813826036051075	0.674343787412871\\
1.395	0.432351961217133	0.823271342915545	0.670253187715551\\
1.395	0.445945513182735	0.832480405557786	0.666246357336158\\
1.395	0.459665918841652	0.841450581281375	0.662288623474186\\
1.395	0.473506959189401	0.850179438693979	0.658351221673191\\
1.395	0.487462361096317	0.85866475920876	0.654411055137886\\
1.395	0.501525806262096	0.866904538099407	0.650450223206468\\
1.395	0.515690940160279	0.874896985111485	0.646455401874959\\
1.395	0.529951380947623	0.882640524634436	0.642417151055156\\
1.395	0.544300728313782	0.890133795440129	0.638329210947587\\
1.395	0.558732572247415	0.897375649995373	0.634187835387499\\
1.395	0.57324050169555	0.904365153357279	0.629991194998377\\
1.395	0.587818113093873	0.911101581661729	0.625738868900922\\
1.395	0.602459018746568	0.917584420216504	0.621431431633138\\
1.395	0.617156855035314	0.923813361211857	0.617070132480554\\
1.395	0.631905290438127	0.92978830106243	0.612656657833839\\
1.395	0.6466980333399	0.935509337395467	0.608192963377087\\
1.395	0.661528839617671	0.940976765701173	0.603681161495104\\
1.395	0.676391519984904	0.946191075661951	0.599123449738606\\
1.395	0.691279947080357	0.951152947177946	0.594522067901588\\
1.395	0.706188062288412	0.955863246106974	0.589879273662389\\
1.395	0.721109882279076	0.960323019737426	0.585197329319938\\
1.395	0.736039505257226	0.964533492013186	0.580478494543978\\
1.395	0.75097111691199	0.968496058529893	0.575725022018313\\
1.395	0.765898996058536	0.972212281322139	0.570939154289556\\
1.395	0.780817519965826	0.975683883461272	0.566123121053176\\
1.395	0.795721169365275	0.97891274348356	0.561279136605562\\
1.395	0.810604533136476	0.981900889668376	0.556409397399201\\
1.395	0.825462312667457	0.984650494185936	0.551516079695733\\
1.395	0.840289325888138	0.987163867133889	0.546601337319639\\
1.395	0.855080510976839	0.989443450481767	0.541667299515139\\
1.395	0.86983092974082	0.991491811941914	0.536716068908598\\
1.395	0.884535770672926	0.993311638785082	0.531749719578142\\
1.395	0.899190351687411	0.994905731618374	0.526770295231795\\
1.395	0.913790122539012	0.996276998142666	0.521779807495101\\
1.395	0.928330666930242	0.99742844690601	0.516780234308685\\
1.395	0.942807704312699	0.998363181068902	0.511773518435872\\
1.395	0.95721709138901	0.999084392196567	0.506761566080168\\
1.395	0.971554823322704	0.999595354092743	0.501746245611876\\
1.395	0.985817034663989	0.999899416688637	0.496729386403004\\
1.395	1	1	0.491669087346906\\
1.41	0	0	0.710195569525497\\
1.41	0.000100583311362513	0.0141829653360114	0.71240460418299\\
1.41	0.000404645907256436	0.0284451766772965	0.714630690536905\\
1.41	0.000915607803432999	0.0427829086109896	0.716916312064106\\
1.41	0.00163681893109844	0.057192295687301	0.719289019360234\\
1.41	0.00257155309398959	0.0716693330697584	0.721780367061016\\
1.41	0.00372300185733413	0.0862098774609879	0.724425655492835\\
1.41	0.00509426838162598	0.100809648312589	0.727263413850304\\
1.41	0.00668836121491815	0.115464229327074	0.73033458997498\\
1.41	0.00850818805808554	0.13016907025918	0.733681422514709\\
1.41	0.0105565495182326	0.144919489023162	0.737345986077549\\
1.41	0.0128361328661109	0.159710674111862	0.741368418705642\\
1.41	0.0153495058140643	0.174537687332543	0.745784862958201\\
1.41	0.0180991103316243	0.189395466863524	0.750625176092311\\
1.41	0.0210872565164405	0.204278830634725	0.755910489868999\\
1.41	0.0243161165387281	0.219182480034174	0.761650724664453\\
1.41	0.0277877186778607	0.234101003941464	0.767842183867616\\
1.41	0.0315039414701067	0.24902888308801	0.77446537092041\\
1.41	0.0354665079868145	0.263960494742775	0.781483180780606\\
1.41	0.0396769802625737	0.278890117720924	0.788839618268073\\
1.41	0.0441367538930258	0.293811937711588	0.796459186325544\\
1.41	0.0488470528220538	0.308720052919643	0.804247066923454\\
1.41	0.0538089243380495	0.323608480015097	0.81209018616195\\
1.41	0.0590232342988274	0.338471160382329	0.819859213932693\\
1.41	0.064490662604533	0.3533019666601	0.82741149906364\\
1.41	0.0702116989375697	0.368094709561873	0.834594885813504\\
1.41	0.0761866387881432	0.382843144964686	0.841252300287179\\
1.41	0.0824155797834956	0.397540981253432	0.84722693972908\\
1.41	0.0888984183382709	0.412181886906127	0.852367847905042\\
1.41	0.0956348466427212	0.42675949830445	0.856535620027827\\
1.41	0.102624350004627	0.441267427752584	0.859607954619617\\
1.41	0.109866204559871	0.455699271686218	0.861484760266619\\
1.41	0.117359475365564	0.470048619052377	0.86209253425154\\
1.41	0.125103014888515	0.484309059839721	0.861387758068369\\
1.41	0.133095461900593	0.498474193737904	0.85935910089082\\
1.41	0.14133524079124	0.512537638903683	0.856028283765736\\
1.41	0.149820561306021	0.526493040810599	0.851449530870186\\
1.41	0.158549418718625	0.540334081158348	0.845707614739508\\
1.41	0.167519594442213	0.554054486817265	0.838914584343151\\
1.41	0.176728657084455	0.567648038782867	0.831205342370837\\
1.41	0.186173963948925	0.581108581114919	0.822732305417528\\
1.41	0.195852662983903	0.594430029835236	0.813659432944306\\
1.41	0.205761695177907	0.60760638175823	0.804155944104869\\
1.41	0.215897797399558	0.620631723228143	0.794390053406971\\
1.41	0.226257505677663	0.633500238737009	0.784523046062802\\
1.41	0.236837158915675	0.646206219397571	0.774703982861588\\
1.41	0.247632903032949	0.658744071245709	0.765065275211014\\
1.41	0.258640695523527	0.671108323347402	0.755719307772959\\
1.41	0.269856310421543	0.683293635685828	0.746756214005914\\
1.41	0.28127534366066	0.695294806804925	0.738242834604297\\
1.41	0.292893218813452	0.707106781186547	0.730222816007832\\
1.41	0.304705193195075	0.71872465633934	0.722717741087768\\
1.41	0.316706364314172	0.730143689578457	0.715729131143269\\
1.41	0.328891676652598	0.741359304476472	0.709241120561857\\
1.41	0.341255928754291	0.752367096967051	0.703223584571264\\
1.41	0.353793780602429	0.763162841084324	0.697635496597417\\
1.41	0.366499761262991	0.773742494322337	0.692428303594619\\
1.41	0.379368276771856	0.784102202600442	0.687549132887069\\
1.41	0.392393618241769	0.794238304822092	0.682943679235328\\
1.41	0.405569970164763	0.804147337016096	0.678558662193046\\
1.41	0.418891418885081	0.813826036051075	0.674343787412871\\
1.41	0.432351961217132	0.823271342915544	0.67025318771555\\
1.41	0.445945513182735	0.832480405557786	0.666246357336159\\
1.41	0.459665918841652	0.841450581281375	0.662288623474188\\
1.41	0.473506959189401	0.850179438693979	0.658351221673191\\
1.41	0.487462361096317	0.85866475920876	0.654411055137885\\
1.41	0.501525806262096	0.866904538099407	0.650450223206469\\
1.41	0.515690940160279	0.874896985111485	0.646455401874957\\
1.41	0.529951380947623	0.882640524634436	0.642417151055155\\
1.41	0.544300728313782	0.890133795440129	0.63832921094759\\
1.41	0.558732572247415	0.897375649995373	0.634187835387498\\
1.41	0.57324050169555	0.904365153357279	0.629991194998376\\
1.41	0.587818113093873	0.911101581661729	0.625738868900926\\
1.41	0.602459018746568	0.917584420216505	0.62143143163314\\
1.41	0.617156855035314	0.923813361211857	0.617070132480551\\
1.41	0.631905290438127	0.92978830106243	0.612656657833838\\
1.41	0.6466980333399	0.935509337395467	0.608192963377089\\
1.41	0.661528839617671	0.940976765701173	0.603681161495105\\
1.41	0.676391519984904	0.946191075661951	0.599123449738608\\
1.41	0.691279947080357	0.951152947177946	0.594522067901589\\
1.41	0.706188062288412	0.955863246106974	0.589879273662389\\
1.41	0.721109882279076	0.960323019737426	0.585197329319937\\
1.41	0.736039505257226	0.964533492013186	0.580478494543976\\
1.41	0.75097111691199	0.968496058529893	0.575725022018315\\
1.41	0.765898996058536	0.972212281322139	0.570939154289558\\
1.41	0.780817519965826	0.975683883461272	0.566123121053172\\
1.41	0.795721169365275	0.978912743483559	0.561279136605561\\
1.41	0.810604533136476	0.981900889668376	0.556409397399201\\
1.41	0.825462312667457	0.984650494185936	0.551516079695735\\
1.41	0.840289325888138	0.987163867133889	0.54660133731964\\
1.41	0.855080510976839	0.989443450481768	0.541667299515138\\
1.41	0.86983092974082	0.991491811941914	0.536716068908599\\
1.41	0.884535770672926	0.993311638785082	0.531749719578144\\
1.41	0.899190351687411	0.994905731618374	0.526770295231793\\
1.41	0.913790122539012	0.996276998142666	0.521779807495102\\
1.41	0.928330666930242	0.997428446906011	0.516780234308683\\
1.41	0.942807704312699	0.998363181068901	0.511773518435873\\
1.41	0.95721709138901	0.999084392196567	0.506761566080169\\
1.41	0.971554823322703	0.999595354092743	0.501746245611874\\
1.41	0.985817034663989	0.999899416688637	0.496729386403004\\
1.41	1	1	0.491669087346906\\
1.425	0	0	0.710195569525497\\
1.425	0.000100583311362513	0.0141829653360114	0.71240460418299\\
1.425	0.000404645907256436	0.0284451766772965	0.714630690536905\\
1.425	0.000915607803432999	0.0427829086109896	0.716916312064106\\
1.425	0.00163681893109844	0.057192295687301	0.719289019360234\\
1.425	0.00257155309398959	0.0716693330697584	0.721780367061016\\
1.425	0.00372300185733414	0.0862098774609879	0.724425655492835\\
1.425	0.00509426838162598	0.100809648312589	0.727263413850304\\
1.425	0.00668836121491815	0.115464229327074	0.73033458997498\\
1.425	0.00850818805808554	0.13016907025918	0.733681422514709\\
1.425	0.0105565495182326	0.144919489023162	0.737345986077549\\
1.425	0.0128361328661109	0.159710674111862	0.741368418705642\\
1.425	0.0153495058140643	0.174537687332543	0.745784862958201\\
1.425	0.0180991103316243	0.189395466863524	0.750625176092311\\
1.425	0.0210872565164405	0.204278830634725	0.755910489868999\\
1.425	0.0243161165387281	0.219182480034174	0.761650724664453\\
1.425	0.0277877186778607	0.234101003941464	0.767842183867616\\
1.425	0.0315039414701067	0.24902888308801	0.77446537092041\\
1.425	0.0354665079868145	0.263960494742775	0.781483180780607\\
1.425	0.0396769802625738	0.278890117720924	0.788839618268073\\
1.425	0.0441367538930258	0.293811937711588	0.796459186325543\\
1.425	0.0488470528220538	0.308720052919643	0.804247066923455\\
1.425	0.0538089243380495	0.323608480015096	0.81209018616195\\
1.425	0.0590232342988274	0.338471160382329	0.819859213932693\\
1.425	0.064490662604533	0.3533019666601	0.827411499063638\\
1.425	0.0702116989375697	0.368094709561873	0.834594885813505\\
1.425	0.0761866387881432	0.382843144964686	0.84125230028718\\
1.425	0.0824155797834956	0.397540981253432	0.84722693972908\\
1.425	0.0888984183382709	0.412181886906127	0.852367847905042\\
1.425	0.0956348466427212	0.42675949830445	0.856535620027828\\
1.425	0.102624350004627	0.441267427752584	0.859607954619618\\
1.425	0.109866204559871	0.455699271686218	0.861484760266623\\
1.425	0.117359475365564	0.470048619052377	0.862092534251537\\
1.425	0.125103014888515	0.484309059839721	0.861387758068369\\
1.425	0.133095461900593	0.498474193737904	0.859359100890821\\
1.425	0.14133524079124	0.512537638903683	0.856028283765738\\
1.425	0.149820561306021	0.526493040810599	0.851449530870183\\
1.425	0.158549418718625	0.540334081158348	0.845707614739507\\
1.425	0.167519594442213	0.554054486817265	0.838914584343153\\
1.425	0.176728657084455	0.567648038782867	0.831205342370837\\
1.425	0.186173963948925	0.581108581114919	0.822732305417526\\
1.425	0.195852662983903	0.594430029835236	0.813659432944306\\
1.425	0.205761695177907	0.607606381758231	0.804155944104871\\
1.425	0.215897797399558	0.620631723228143	0.794390053406973\\
1.425	0.226257505677663	0.633500238737009	0.7845230460628\\
1.425	0.236837158915675	0.646206219397571	0.77470398286159\\
1.425	0.247632903032949	0.658744071245709	0.765065275211017\\
1.425	0.258640695523528	0.671108323347402	0.755719307772962\\
1.425	0.269856310421543	0.683293635685828	0.746756214005914\\
1.425	0.28127534366066	0.695294806804925	0.738242834604294\\
1.425	0.292893218813452	0.707106781186547	0.730222816007832\\
1.425	0.304705193195075	0.71872465633934	0.722717741087769\\
1.425	0.316706364314172	0.730143689578457	0.715729131143268\\
1.425	0.328891676652598	0.741359304476472	0.709241120561855\\
1.425	0.341255928754291	0.752367096967051	0.703223584571265\\
1.425	0.353793780602429	0.763162841084325	0.697635496597419\\
1.425	0.366499761262991	0.773742494322337	0.692428303594618\\
1.425	0.379368276771857	0.784102202600442	0.687549132887068\\
1.425	0.392393618241769	0.794238304822092	0.682943679235328\\
1.425	0.405569970164763	0.804147337016096	0.678558662193044\\
1.425	0.418891418885081	0.813826036051075	0.674343787412871\\
1.425	0.432351961217132	0.823271342915544	0.670253187715552\\
1.425	0.445945513182735	0.832480405557786	0.666246357336158\\
1.425	0.459665918841652	0.841450581281375	0.662288623474186\\
1.425	0.473506959189401	0.850179438693979	0.658351221673193\\
1.425	0.487462361096317	0.85866475920876	0.654411055137886\\
1.425	0.501525806262096	0.866904538099407	0.650450223206467\\
1.425	0.515690940160279	0.874896985111485	0.646455401874958\\
1.425	0.529951380947623	0.882640524634436	0.642417151055154\\
1.425	0.544300728313782	0.890133795440129	0.638329210947586\\
1.425	0.558732572247415	0.897375649995373	0.634187835387498\\
1.425	0.57324050169555	0.904365153357279	0.629991194998377\\
1.425	0.587818113093873	0.911101581661729	0.625738868900925\\
1.425	0.602459018746568	0.917584420216504	0.62143143163314\\
1.425	0.617156855035314	0.923813361211857	0.617070132480553\\
1.425	0.631905290438127	0.92978830106243	0.612656657833837\\
1.425	0.6466980333399	0.935509337395467	0.608192963377087\\
1.425	0.661528839617671	0.940976765701173	0.603681161495104\\
1.425	0.676391519984904	0.94619107566195	0.599123449738607\\
1.425	0.691279947080357	0.951152947177946	0.594522067901589\\
1.425	0.706188062288412	0.955863246106974	0.58987927366239\\
1.425	0.721109882279076	0.960323019737426	0.585197329319937\\
1.425	0.736039505257225	0.964533492013185	0.580478494543975\\
1.425	0.75097111691199	0.968496058529893	0.575725022018313\\
1.425	0.765898996058536	0.972212281322139	0.570939154289561\\
1.425	0.780817519965826	0.975683883461272	0.566123121053175\\
1.425	0.795721169365275	0.97891274348356	0.56127913660556\\
1.425	0.810604533136476	0.981900889668376	0.556409397399199\\
1.425	0.825462312667457	0.984650494185936	0.551516079695735\\
1.425	0.840289325888138	0.987163867133889	0.546601337319641\\
1.425	0.855080510976839	0.989443450481767	0.541667299515137\\
1.425	0.86983092974082	0.991491811941914	0.536716068908598\\
1.425	0.884535770672926	0.993311638785082	0.531749719578143\\
1.425	0.899190351687411	0.994905731618374	0.526770295231793\\
1.425	0.913790122539012	0.996276998142666	0.521779807495103\\
1.425	0.928330666930242	0.997428446906011	0.516780234308684\\
1.425	0.942807704312699	0.998363181068902	0.511773518435872\\
1.425	0.95721709138901	0.999084392196567	0.50676156608017\\
1.425	0.971554823322703	0.999595354092743	0.501746245611874\\
1.425	0.985817034663989	0.999899416688637	0.496729386403004\\
1.425	1	1	0.491669087346906\\
1.44	0	0	0.710195569525497\\
1.44	0.000100583311362513	0.0141829653360114	0.71240460418299\\
1.44	0.000404645907256436	0.0284451766772965	0.714630690536905\\
1.44	0.000915607803432999	0.0427829086109896	0.716916312064106\\
1.44	0.00163681893109844	0.057192295687301	0.719289019360234\\
1.44	0.00257155309398959	0.0716693330697584	0.721780367061016\\
1.44	0.00372300185733413	0.0862098774609879	0.724425655492835\\
1.44	0.00509426838162598	0.100809648312589	0.727263413850304\\
1.44	0.00668836121491816	0.115464229327074	0.73033458997498\\
1.44	0.00850818805808554	0.13016907025918	0.733681422514709\\
1.44	0.0105565495182326	0.144919489023162	0.737345986077549\\
1.44	0.0128361328661109	0.159710674111862	0.741368418705642\\
1.44	0.0153495058140643	0.174537687332543	0.745784862958201\\
1.44	0.0180991103316243	0.189395466863524	0.750625176092311\\
1.44	0.0210872565164405	0.204278830634725	0.755910489868999\\
1.44	0.0243161165387281	0.219182480034174	0.761650724664453\\
1.44	0.0277877186778607	0.234101003941464	0.767842183867616\\
1.44	0.0315039414701067	0.24902888308801	0.774465370920409\\
1.44	0.0354665079868145	0.263960494742775	0.781483180780606\\
1.44	0.0396769802625738	0.278890117720924	0.788839618268073\\
1.44	0.0441367538930258	0.293811937711588	0.796459186325544\\
1.44	0.0488470528220538	0.308720052919643	0.804247066923454\\
1.44	0.0538089243380495	0.323608480015097	0.812090186161951\\
1.44	0.0590232342988274	0.338471160382329	0.819859213932694\\
1.44	0.064490662604533	0.3533019666601	0.827411499063638\\
1.44	0.0702116989375697	0.368094709561873	0.834594885813505\\
1.44	0.0761866387881432	0.382843144964686	0.841252300287179\\
1.44	0.0824155797834956	0.397540981253432	0.847226939729081\\
1.44	0.0888984183382709	0.412181886906127	0.85236784790504\\
1.44	0.0956348466427212	0.42675949830445	0.856535620027831\\
1.44	0.102624350004627	0.441267427752584	0.859607954619617\\
1.44	0.109866204559871	0.455699271686218	0.861484760266622\\
1.44	0.117359475365564	0.470048619052377	0.862092534251542\\
1.44	0.125103014888515	0.484309059839721	0.861387758068369\\
1.44	0.133095461900593	0.498474193737904	0.859359100890819\\
1.44	0.14133524079124	0.512537638903683	0.856028283765738\\
1.44	0.149820561306021	0.526493040810599	0.851449530870183\\
1.44	0.158549418718625	0.540334081158348	0.845707614739502\\
1.44	0.167519594442214	0.554054486817265	0.83891458434315\\
1.44	0.176728657084455	0.567648038782867	0.831205342370837\\
1.44	0.186173963948925	0.581108581114919	0.822732305417525\\
1.44	0.195852662983903	0.594430029835236	0.813659432944308\\
1.44	0.205761695177907	0.607606381758231	0.804155944104873\\
1.44	0.215897797399558	0.620631723228144	0.794390053406974\\
1.44	0.226257505677663	0.633500238737009	0.784523046062802\\
1.44	0.236837158915675	0.646206219397571	0.774703982861593\\
1.44	0.247632903032949	0.658744071245709	0.765065275211014\\
1.44	0.258640695523528	0.671108323347402	0.755719307772959\\
1.44	0.269856310421543	0.683293635685828	0.746756214005912\\
1.44	0.28127534366066	0.695294806804924	0.738242834604296\\
1.44	0.292893218813452	0.707106781186547	0.730222816007835\\
1.44	0.304705193195075	0.71872465633934	0.722717741087768\\
1.44	0.316706364314172	0.730143689578457	0.715729131143268\\
1.44	0.328891676652598	0.741359304476472	0.709241120561857\\
1.44	0.341255928754291	0.752367096967051	0.703223584571266\\
1.44	0.353793780602429	0.763162841084325	0.697635496597417\\
1.44	0.366499761262991	0.773742494322337	0.692428303594618\\
1.44	0.379368276771857	0.784102202600442	0.68754913288707\\
1.44	0.392393618241769	0.794238304822092	0.682943679235329\\
1.44	0.405569970164763	0.804147337016097	0.678558662193045\\
1.44	0.418891418885081	0.813826036051075	0.67434378741287\\
1.44	0.432351961217133	0.823271342915544	0.670253187715552\\
1.44	0.445945513182735	0.832480405557786	0.666246357336159\\
1.44	0.459665918841652	0.841450581281375	0.662288623474185\\
1.44	0.473506959189401	0.850179438693979	0.658351221673192\\
1.44	0.487462361096317	0.85866475920876	0.654411055137887\\
1.44	0.501525806262096	0.866904538099407	0.650450223206467\\
1.44	0.515690940160279	0.874896985111485	0.646455401874959\\
1.44	0.529951380947623	0.882640524634436	0.642417151055155\\
1.44	0.544300728313782	0.890133795440129	0.638329210947585\\
1.44	0.558732572247415	0.897375649995372	0.634187835387497\\
1.44	0.57324050169555	0.904365153357279	0.629991194998376\\
1.44	0.587818113093873	0.911101581661729	0.625738868900926\\
1.44	0.602459018746568	0.917584420216504	0.62143143163314\\
1.44	0.617156855035314	0.923813361211857	0.617070132480553\\
1.44	0.631905290438127	0.92978830106243	0.612656657833839\\
1.44	0.6466980333399	0.935509337395467	0.608192963377089\\
1.44	0.661528839617671	0.940976765701173	0.603681161495105\\
1.44	0.676391519984903	0.946191075661951	0.599123449738607\\
1.44	0.691279947080357	0.951152947177946	0.594522067901587\\
1.44	0.706188062288412	0.955863246106974	0.589879273662389\\
1.44	0.721109882279076	0.960323019737426	0.585197329319939\\
1.44	0.736039505257225	0.964533492013186	0.580478494543977\\
1.44	0.75097111691199	0.968496058529893	0.575725022018311\\
1.44	0.765898996058536	0.972212281322139	0.570939154289558\\
1.44	0.780817519965826	0.975683883461272	0.566123121053173\\
1.44	0.795721169365275	0.978912743483559	0.56127913660556\\
1.44	0.810604533136476	0.981900889668376	0.556409397399202\\
1.44	0.825462312667457	0.984650494185936	0.551516079695735\\
1.44	0.840289325888138	0.987163867133889	0.546601337319641\\
1.44	0.855080510976838	0.989443450481768	0.541667299515141\\
1.44	0.86983092974082	0.991491811941915	0.536716068908598\\
1.44	0.884535770672926	0.993311638785082	0.53174971957814\\
1.44	0.899190351687411	0.994905731618374	0.526770295231793\\
1.44	0.913790122539012	0.996276998142666	0.521779807495102\\
1.44	0.928330666930241	0.99742844690601	0.516780234308683\\
1.44	0.942807704312699	0.998363181068901	0.511773518435874\\
1.44	0.95721709138901	0.999084392196567	0.50676156608017\\
1.44	0.971554823322703	0.999595354092743	0.501746245611874\\
1.44	0.985817034663989	0.999899416688637	0.496729386403004\\
1.44	1	1	0.491669087346906\\
1.455	0	0	0.710195569525497\\
1.455	0.000100583311362513	0.0141829653360114	0.71240460418299\\
1.455	0.000404645907256436	0.0284451766772965	0.714630690536905\\
1.455	0.000915607803432999	0.0427829086109896	0.716916312064106\\
1.455	0.00163681893109843	0.057192295687301	0.719289019360234\\
1.455	0.00257155309398959	0.0716693330697585	0.721780367061016\\
1.455	0.00372300185733413	0.0862098774609879	0.724425655492835\\
1.455	0.00509426838162598	0.100809648312589	0.727263413850304\\
1.455	0.00668836121491815	0.115464229327074	0.73033458997498\\
1.455	0.00850818805808555	0.13016907025918	0.733681422514709\\
1.455	0.0105565495182326	0.144919489023162	0.737345986077549\\
1.455	0.0128361328661109	0.159710674111862	0.741368418705642\\
1.455	0.0153495058140643	0.174537687332543	0.745784862958201\\
1.455	0.0180991103316243	0.189395466863524	0.750625176092311\\
1.455	0.0210872565164405	0.204278830634725	0.755910489868999\\
1.455	0.0243161165387281	0.219182480034174	0.761650724664453\\
1.455	0.0277877186778607	0.234101003941464	0.767842183867616\\
1.455	0.0315039414701067	0.24902888308801	0.77446537092041\\
1.455	0.0354665079868145	0.263960494742775	0.781483180780606\\
1.455	0.0396769802625738	0.278890117720924	0.788839618268073\\
1.455	0.0441367538930258	0.293811937711588	0.796459186325544\\
1.455	0.0488470528220538	0.308720052919643	0.804247066923454\\
1.455	0.0538089243380495	0.323608480015096	0.81209018616195\\
1.455	0.0590232342988274	0.338471160382329	0.819859213932693\\
1.455	0.064490662604533	0.3533019666601	0.82741149906364\\
1.455	0.0702116989375697	0.368094709561873	0.834594885813505\\
1.455	0.0761866387881432	0.382843144964686	0.841252300287179\\
1.455	0.0824155797834956	0.397540981253432	0.847226939729081\\
1.455	0.0888984183382709	0.412181886906127	0.852367847905041\\
1.455	0.0956348466427212	0.42675949830445	0.85653562002783\\
1.455	0.102624350004627	0.441267427752584	0.859607954619618\\
1.455	0.109866204559871	0.455699271686218	0.861484760266622\\
1.455	0.117359475365564	0.470048619052377	0.862092534251541\\
1.455	0.125103014888515	0.484309059839721	0.861387758068367\\
1.455	0.133095461900593	0.498474193737904	0.85935910089082\\
1.455	0.14133524079124	0.512537638903683	0.856028283765736\\
1.455	0.149820561306021	0.526493040810599	0.85144953087018\\
1.455	0.158549418718625	0.540334081158348	0.845707614739504\\
1.455	0.167519594442214	0.554054486817265	0.838914584343149\\
1.455	0.176728657084455	0.567648038782867	0.831205342370835\\
1.455	0.186173963948925	0.581108581114919	0.822732305417528\\
1.455	0.195852662983903	0.594430029835236	0.813659432944312\\
1.455	0.205761695177907	0.607606381758231	0.804155944104871\\
1.455	0.215897797399558	0.620631723228144	0.794390053406968\\
1.455	0.226257505677663	0.633500238737009	0.7845230460628\\
1.455	0.236837158915675	0.646206219397571	0.774703982861589\\
1.455	0.247632903032949	0.658744071245709	0.765065275211013\\
1.455	0.258640695523528	0.671108323347402	0.755719307772962\\
1.455	0.269856310421543	0.683293635685828	0.746756214005915\\
1.455	0.28127534366066	0.695294806804925	0.738242834604297\\
1.455	0.292893218813452	0.707106781186547	0.730222816007832\\
1.455	0.304705193195075	0.71872465633934	0.722717741087768\\
1.455	0.316706364314172	0.730143689578457	0.715729131143268\\
1.455	0.328891676652598	0.741359304476472	0.709241120561855\\
1.455	0.341255928754291	0.752367096967051	0.703223584571265\\
1.455	0.353793780602429	0.763162841084325	0.697635496597418\\
1.455	0.366499761262991	0.773742494322337	0.692428303594618\\
1.455	0.379368276771857	0.784102202600443	0.68754913288707\\
1.455	0.392393618241769	0.794238304822092	0.682943679235329\\
1.455	0.405569970164763	0.804147337016097	0.678558662193044\\
1.455	0.418891418885081	0.813826036051075	0.67434378741287\\
1.455	0.432351961217132	0.823271342915544	0.670253187715552\\
1.455	0.445945513182735	0.832480405557786	0.666246357336158\\
1.455	0.459665918841652	0.841450581281375	0.662288623474186\\
1.455	0.473506959189401	0.850179438693979	0.658351221673191\\
1.455	0.487462361096317	0.85866475920876	0.654411055137887\\
1.455	0.501525806262096	0.866904538099407	0.650450223206467\\
1.455	0.515690940160279	0.874896985111485	0.646455401874959\\
1.455	0.529951380947623	0.882640524634436	0.642417151055157\\
1.455	0.544300728313781	0.890133795440129	0.638329210947586\\
1.455	0.558732572247415	0.897375649995373	0.634187835387498\\
1.455	0.57324050169555	0.904365153357279	0.629991194998375\\
1.455	0.587818113093873	0.911101581661729	0.625738868900923\\
1.455	0.602459018746568	0.917584420216504	0.621431431633141\\
1.455	0.617156855035314	0.923813361211857	0.617070132480552\\
1.455	0.631905290438127	0.92978830106243	0.612656657833836\\
1.455	0.6466980333399	0.935509337395467	0.608192963377089\\
1.455	0.661528839617671	0.940976765701173	0.603681161495106\\
1.455	0.676391519984903	0.94619107566195	0.599123449738607\\
1.455	0.691279947080357	0.951152947177946	0.594522067901588\\
1.455	0.706188062288412	0.955863246106974	0.589879273662389\\
1.455	0.721109882279076	0.960323019737426	0.585197329319938\\
1.455	0.736039505257226	0.964533492013186	0.580478494543977\\
1.455	0.75097111691199	0.968496058529893	0.575725022018312\\
1.455	0.765898996058536	0.972212281322139	0.570939154289559\\
1.455	0.780817519965826	0.975683883461272	0.566123121053176\\
1.455	0.795721169365275	0.97891274348356	0.561279136605557\\
1.455	0.810604533136476	0.981900889668376	0.556409397399197\\
1.455	0.825462312667457	0.984650494185936	0.551516079695733\\
1.455	0.840289325888138	0.987163867133889	0.546601337319642\\
1.455	0.855080510976839	0.989443450481768	0.541667299515142\\
1.455	0.86983092974082	0.991491811941915	0.536716068908598\\
1.455	0.884535770672926	0.993311638785082	0.531749719578141\\
1.455	0.899190351687411	0.994905731618374	0.526770295231794\\
1.455	0.913790122539012	0.996276998142666	0.521779807495103\\
1.455	0.928330666930242	0.997428446906011	0.516780234308683\\
1.455	0.942807704312699	0.998363181068902	0.511773518435872\\
1.455	0.95721709138901	0.999084392196567	0.50676156608017\\
1.455	0.971554823322703	0.999595354092743	0.501746245611874\\
1.455	0.985817034663989	0.999899416688637	0.496729386403004\\
1.455	1	1	0.491669087346906\\
1.47	0	0	0.710195569525497\\
1.47	0.000100583311362513	0.0141829653360114	0.71240460418299\\
1.47	0.000404645907256436	0.0284451766772965	0.714630690536905\\
1.47	0.000915607803432999	0.0427829086109896	0.716916312064106\\
1.47	0.00163681893109843	0.057192295687301	0.719289019360234\\
1.47	0.00257155309398959	0.0716693330697584	0.721780367061016\\
1.47	0.00372300185733413	0.0862098774609879	0.724425655492835\\
1.47	0.00509426838162598	0.100809648312589	0.727263413850304\\
1.47	0.00668836121491815	0.115464229327074	0.73033458997498\\
1.47	0.00850818805808555	0.13016907025918	0.733681422514709\\
1.47	0.0105565495182326	0.144919489023162	0.737345986077549\\
1.47	0.0128361328661109	0.159710674111862	0.741368418705642\\
1.47	0.0153495058140643	0.174537687332543	0.745784862958202\\
1.47	0.0180991103316243	0.189395466863524	0.750625176092311\\
1.47	0.0210872565164405	0.204278830634725	0.755910489868998\\
1.47	0.0243161165387281	0.219182480034174	0.761650724664453\\
1.47	0.0277877186778607	0.234101003941464	0.767842183867616\\
1.47	0.0315039414701067	0.24902888308801	0.77446537092041\\
1.47	0.0354665079868145	0.263960494742775	0.781483180780606\\
1.47	0.0396769802625738	0.278890117720924	0.788839618268074\\
1.47	0.0441367538930258	0.293811937711588	0.796459186325544\\
1.47	0.0488470528220538	0.308720052919643	0.804247066923455\\
1.47	0.0538089243380495	0.323608480015096	0.81209018616195\\
1.47	0.0590232342988274	0.338471160382329	0.819859213932694\\
1.47	0.064490662604533	0.3533019666601	0.82741149906364\\
1.47	0.0702116989375697	0.368094709561873	0.834594885813505\\
1.47	0.0761866387881432	0.382843144964686	0.841252300287179\\
1.47	0.0824155797834956	0.397540981253432	0.847226939729081\\
1.47	0.0888984183382709	0.412181886906127	0.852367847905041\\
1.47	0.0956348466427212	0.42675949830445	0.856535620027828\\
1.47	0.102624350004627	0.441267427752585	0.859607954619617\\
1.47	0.109866204559871	0.455699271686218	0.861484760266622\\
1.47	0.117359475365563	0.470048619052377	0.862092534251539\\
1.47	0.125103014888515	0.484309059839721	0.861387758068369\\
1.47	0.133095461900593	0.498474193737904	0.85935910089082\\
1.47	0.14133524079124	0.512537638903683	0.856028283765736\\
1.47	0.149820561306021	0.526493040810599	0.851449530870183\\
1.47	0.158549418718625	0.540334081158348	0.845707614739509\\
1.47	0.167519594442214	0.554054486817265	0.838914584343148\\
1.47	0.176728657084455	0.567648038782867	0.831205342370837\\
1.47	0.186173963948925	0.581108581114919	0.822732305417528\\
1.47	0.195852662983903	0.594430029835237	0.813659432944305\\
1.47	0.205761695177907	0.607606381758231	0.804155944104867\\
1.47	0.215897797399558	0.620631723228143	0.794390053406969\\
1.47	0.226257505677663	0.633500238737009	0.7845230460628\\
1.47	0.236837158915675	0.646206219397571	0.774703982861588\\
1.47	0.247632903032949	0.658744071245709	0.765065275211016\\
1.47	0.258640695523527	0.671108323347402	0.755719307772962\\
1.47	0.269856310421543	0.683293635685828	0.746756214005913\\
1.47	0.28127534366066	0.695294806804925	0.738242834604295\\
1.47	0.292893218813452	0.707106781186547	0.730222816007832\\
1.47	0.304705193195075	0.71872465633934	0.722717741087768\\
1.47	0.316706364314172	0.730143689578457	0.715729131143269\\
1.47	0.328891676652598	0.741359304476472	0.709241120561857\\
1.47	0.341255928754291	0.752367096967051	0.703223584571265\\
1.47	0.353793780602429	0.763162841084325	0.697635496597419\\
1.47	0.366499761262991	0.773742494322337	0.692428303594617\\
1.47	0.379368276771856	0.784102202600442	0.687549132887068\\
1.47	0.392393618241769	0.794238304822092	0.68294367923533\\
1.47	0.405569970164763	0.804147337016097	0.678558662193045\\
1.47	0.418891418885081	0.813826036051075	0.67434378741287\\
1.47	0.432351961217132	0.823271342915544	0.670253187715552\\
1.47	0.445945513182735	0.832480405557786	0.666246357336159\\
1.47	0.459665918841652	0.841450581281375	0.662288623474185\\
1.47	0.473506959189401	0.850179438693979	0.658351221673191\\
1.47	0.487462361096317	0.85866475920876	0.654411055137886\\
1.47	0.501525806262096	0.866904538099407	0.650450223206468\\
1.47	0.515690940160279	0.874896985111485	0.646455401874957\\
1.47	0.529951380947623	0.882640524634436	0.642417151055156\\
1.47	0.544300728313782	0.890133795440129	0.638329210947588\\
1.47	0.558732572247415	0.897375649995373	0.634187835387497\\
1.47	0.57324050169555	0.904365153357279	0.629991194998375\\
1.47	0.587818113093873	0.911101581661729	0.625738868900925\\
1.47	0.602459018746568	0.917584420216504	0.621431431633141\\
1.47	0.617156855035314	0.923813361211857	0.617070132480553\\
1.47	0.631905290438127	0.92978830106243	0.612656657833837\\
1.47	0.6466980333399	0.935509337395467	0.608192963377088\\
1.47	0.661528839617671	0.940976765701173	0.603681161495106\\
1.47	0.676391519984904	0.946191075661951	0.599123449738607\\
1.47	0.691279947080357	0.951152947177946	0.594522067901587\\
1.47	0.706188062288412	0.955863246106974	0.589879273662388\\
1.47	0.721109882279076	0.960323019737426	0.585197329319939\\
1.47	0.736039505257225	0.964533492013186	0.580478494543978\\
1.47	0.75097111691199	0.968496058529893	0.575725022018313\\
1.47	0.765898996058536	0.972212281322139	0.570939154289558\\
1.47	0.780817519965826	0.975683883461272	0.566123121053176\\
1.47	0.795721169365275	0.97891274348356	0.56127913660556\\
1.47	0.810604533136476	0.981900889668376	0.556409397399198\\
1.47	0.825462312667457	0.984650494185936	0.551516079695735\\
1.47	0.840289325888138	0.987163867133889	0.546601337319641\\
1.47	0.855080510976838	0.989443450481767	0.541667299515138\\
1.47	0.86983092974082	0.991491811941914	0.536716068908599\\
1.47	0.884535770672926	0.993311638785082	0.531749719578142\\
1.47	0.899190351687411	0.994905731618374	0.526770295231793\\
1.47	0.913790122539012	0.996276998142666	0.521779807495102\\
1.47	0.928330666930242	0.997428446906011	0.516780234308683\\
1.47	0.942807704312699	0.998363181068902	0.511773518435872\\
1.47	0.95721709138901	0.999084392196567	0.50676156608017\\
1.47	0.971554823322703	0.999595354092743	0.501746245611874\\
1.47	0.985817034663989	0.999899416688637	0.496729386403004\\
1.47	1	1	0.491669087346906\\
1.485	0	0	0.710195569525497\\
1.485	0.000100583311362513	0.0141829653360114	0.71240460418299\\
1.485	0.000404645907256436	0.0284451766772965	0.714630690536905\\
1.485	0.000915607803432999	0.0427829086109896	0.716916312064106\\
1.485	0.00163681893109844	0.057192295687301	0.719289019360234\\
1.485	0.00257155309398959	0.0716693330697585	0.721780367061016\\
1.485	0.00372300185733413	0.0862098774609879	0.724425655492835\\
1.485	0.00509426838162598	0.100809648312589	0.727263413850304\\
1.485	0.00668836121491815	0.115464229327074	0.73033458997498\\
1.485	0.00850818805808554	0.13016907025918	0.733681422514709\\
1.485	0.0105565495182326	0.144919489023162	0.737345986077549\\
1.485	0.0128361328661109	0.159710674111862	0.741368418705642\\
1.485	0.0153495058140643	0.174537687332543	0.745784862958201\\
1.485	0.0180991103316243	0.189395466863524	0.750625176092311\\
1.485	0.0210872565164405	0.204278830634725	0.755910489868999\\
1.485	0.0243161165387281	0.219182480034174	0.761650724664453\\
1.485	0.0277877186778607	0.234101003941464	0.767842183867616\\
1.485	0.0315039414701067	0.24902888308801	0.77446537092041\\
1.485	0.0354665079868145	0.263960494742775	0.781483180780606\\
1.485	0.0396769802625738	0.278890117720924	0.788839618268073\\
1.485	0.0441367538930258	0.293811937711588	0.796459186325544\\
1.485	0.0488470528220538	0.308720052919643	0.804247066923455\\
1.485	0.0538089243380495	0.323608480015096	0.81209018616195\\
1.485	0.0590232342988274	0.338471160382329	0.819859213932694\\
1.485	0.064490662604533	0.3533019666601	0.82741149906364\\
1.485	0.0702116989375697	0.368094709561873	0.834594885813504\\
1.485	0.0761866387881432	0.382843144964686	0.841252300287178\\
1.485	0.0824155797834956	0.397540981253432	0.847226939729081\\
1.485	0.0888984183382709	0.412181886906127	0.852367847905041\\
1.485	0.0956348466427212	0.42675949830445	0.856535620027829\\
1.485	0.102624350004627	0.441267427752584	0.859607954619617\\
1.485	0.109866204559871	0.455699271686218	0.861484760266622\\
1.485	0.117359475365564	0.470048619052377	0.86209253425154\\
1.485	0.125103014888515	0.484309059839721	0.861387758068369\\
1.485	0.133095461900593	0.498474193737904	0.85935910089082\\
1.485	0.14133524079124	0.512537638903683	0.856028283765737\\
1.485	0.149820561306021	0.526493040810599	0.851449530870186\\
1.485	0.158549418718625	0.540334081158348	0.845707614739505\\
1.485	0.167519594442213	0.554054486817265	0.83891458434315\\
1.485	0.176728657084455	0.567648038782867	0.831205342370837\\
1.485	0.186173963948925	0.581108581114919	0.822732305417526\\
1.485	0.195852662983903	0.594430029835236	0.813659432944304\\
1.485	0.205761695177907	0.607606381758231	0.80415594410487\\
1.485	0.215897797399558	0.620631723228143	0.79439005340697\\
1.485	0.226257505677663	0.633500238737009	0.7845230460628\\
1.485	0.236837158915675	0.646206219397571	0.77470398286159\\
1.485	0.247632903032949	0.658744071245709	0.765065275211016\\
1.485	0.258640695523527	0.671108323347402	0.755719307772961\\
1.485	0.269856310421543	0.683293635685828	0.746756214005913\\
1.485	0.28127534366066	0.695294806804925	0.738242834604295\\
1.485	0.292893218813452	0.707106781186547	0.730222816007833\\
1.485	0.304705193195075	0.71872465633934	0.722717741087769\\
1.485	0.316706364314172	0.730143689578457	0.715729131143269\\
1.485	0.328891676652598	0.741359304476472	0.709241120561855\\
1.485	0.341255928754291	0.752367096967051	0.703223584571265\\
1.485	0.353793780602429	0.763162841084325	0.697635496597419\\
1.485	0.366499761262991	0.773742494322337	0.692428303594618\\
1.485	0.379368276771856	0.784102202600442	0.687549132887069\\
1.485	0.392393618241769	0.794238304822092	0.682943679235329\\
1.485	0.405569970164763	0.804147337016097	0.678558662193044\\
1.485	0.418891418885081	0.813826036051075	0.674343787412871\\
1.485	0.432351961217132	0.823271342915544	0.670253187715552\\
1.485	0.445945513182735	0.832480405557786	0.666246357336159\\
1.485	0.459665918841652	0.841450581281375	0.662288623474186\\
1.485	0.473506959189401	0.850179438693979	0.658351221673191\\
1.485	0.487462361096317	0.85866475920876	0.654411055137886\\
1.485	0.501525806262096	0.866904538099407	0.650450223206468\\
1.485	0.515690940160279	0.874896985111485	0.646455401874958\\
1.485	0.529951380947623	0.882640524634436	0.642417151055155\\
1.485	0.544300728313782	0.890133795440129	0.638329210947586\\
1.485	0.558732572247415	0.897375649995373	0.634187835387498\\
1.485	0.57324050169555	0.904365153357279	0.629991194998374\\
1.485	0.587818113093873	0.911101581661729	0.625738868900923\\
1.485	0.602459018746568	0.917584420216504	0.621431431633142\\
1.485	0.617156855035314	0.923813361211857	0.617070132480553\\
1.485	0.631905290438127	0.92978830106243	0.612656657833837\\
1.485	0.6466980333399	0.935509337395467	0.608192963377087\\
1.485	0.661528839617671	0.940976765701173	0.603681161495105\\
1.485	0.676391519984904	0.94619107566195	0.599123449738608\\
1.485	0.691279947080357	0.951152947177946	0.59452206790159\\
1.485	0.706188062288412	0.955863246106974	0.589879273662388\\
1.485	0.721109882279076	0.960323019737426	0.585197329319937\\
1.485	0.736039505257226	0.964533492013186	0.580478494543978\\
1.485	0.75097111691199	0.968496058529893	0.575725022018312\\
1.485	0.765898996058536	0.972212281322139	0.570939154289558\\
1.485	0.780817519965826	0.975683883461272	0.566123121053175\\
1.485	0.795721169365275	0.97891274348356	0.56127913660556\\
1.485	0.810604533136476	0.981900889668376	0.556409397399199\\
1.485	0.825462312667457	0.984650494185936	0.551516079695733\\
1.485	0.840289325888138	0.987163867133889	0.546601337319641\\
1.485	0.855080510976839	0.989443450481767	0.541667299515141\\
1.485	0.86983092974082	0.991491811941914	0.536716068908599\\
1.485	0.884535770672926	0.993311638785082	0.531749719578143\\
1.485	0.899190351687411	0.994905731618374	0.526770295231793\\
1.485	0.913790122539012	0.996276998142666	0.521779807495103\\
1.485	0.928330666930242	0.997428446906011	0.516780234308683\\
1.485	0.942807704312699	0.998363181068902	0.511773518435872\\
1.485	0.95721709138901	0.999084392196567	0.50676156608017\\
1.485	0.971554823322703	0.999595354092743	0.501746245611874\\
1.485	0.985817034663988	0.999899416688637	0.496729386403004\\
1.485	1	1	0.491669087346906\\
1.5	0	0	0.710195569525497\\
1.5	0.000100583311362513	0.0141829653360114	0.71240460418299\\
1.5	0.000404645907256436	0.0284451766772965	0.714630690536905\\
1.5	0.000915607803432999	0.0427829086109896	0.716916312064106\\
1.5	0.00163681893109844	0.057192295687301	0.719289019360234\\
1.5	0.00257155309398959	0.0716693330697585	0.721780367061016\\
1.5	0.00372300185733413	0.0862098774609879	0.724425655492835\\
1.5	0.00509426838162598	0.100809648312589	0.727263413850304\\
1.5	0.00668836121491815	0.115464229327074	0.73033458997498\\
1.5	0.00850818805808554	0.13016907025918	0.733681422514709\\
1.5	0.0105565495182326	0.144919489023162	0.737345986077549\\
1.5	0.0128361328661109	0.159710674111862	0.741368418705642\\
1.5	0.0153495058140643	0.174537687332543	0.745784862958201\\
1.5	0.0180991103316243	0.189395466863524	0.750625176092311\\
1.5	0.0210872565164405	0.204278830634725	0.755910489868999\\
1.5	0.0243161165387281	0.219182480034174	0.761650724664453\\
1.5	0.0277877186778607	0.234101003941464	0.767842183867616\\
1.5	0.0315039414701067	0.24902888308801	0.77446537092041\\
1.5	0.0354665079868145	0.263960494742775	0.781483180780606\\
1.5	0.0396769802625738	0.278890117720924	0.788839618268073\\
1.5	0.0441367538930258	0.293811937711588	0.796459186325544\\
1.5	0.0488470528220538	0.308720052919643	0.804247066923454\\
1.5	0.0538089243380495	0.323608480015096	0.81209018616195\\
1.5	0.0590232342988274	0.338471160382329	0.819859213932689\\
1.5	0.064490662604533	0.3533019666601	0.827411499063639\\
1.5	0.0702116989375697	0.368094709561873	0.834594885813506\\
1.5	0.0761866387881432	0.382843144964686	0.841252300287179\\
1.5	0.0824155797834956	0.397540981253432	0.847226939729081\\
1.5	0.0888984183382709	0.412181886906127	0.852367847905041\\
1.5	0.0956348466427212	0.42675949830445	0.85653562002783\\
1.5	0.102624350004627	0.441267427752584	0.859607954619617\\
1.5	0.109866204559871	0.455699271686218	0.861484760266622\\
1.5	0.117359475365564	0.470048619052377	0.862092534251541\\
1.5	0.125103014888515	0.484309059839721	0.861387758068369\\
1.5	0.133095461900593	0.498474193737904	0.85935910089082\\
1.5	0.14133524079124	0.512537638903683	0.856028283765736\\
1.5	0.149820561306021	0.526493040810599	0.851449530870179\\
1.5	0.158549418718625	0.540334081158348	0.845707614739503\\
1.5	0.167519594442213	0.554054486817265	0.83891458434315\\
1.5	0.176728657084455	0.567648038782867	0.831205342370837\\
1.5	0.186173963948925	0.581108581114919	0.822732305417533\\
1.5	0.195852662983903	0.594430029835236	0.813659432944309\\
1.5	0.205761695177907	0.607606381758231	0.80415594410487\\
1.5	0.215897797399558	0.620631723228143	0.79439005340697\\
1.5	0.226257505677663	0.633500238737009	0.784523046062802\\
1.5	0.236837158915675	0.646206219397571	0.774703982861591\\
1.5	0.247632903032949	0.658744071245709	0.765065275211016\\
1.5	0.258640695523527	0.671108323347402	0.755719307772962\\
1.5	0.269856310421543	0.683293635685828	0.746756214005913\\
1.5	0.28127534366066	0.695294806804925	0.738242834604294\\
1.5	0.292893218813452	0.707106781186547	0.730222816007833\\
1.5	0.304705193195075	0.71872465633934	0.722717741087769\\
1.5	0.316706364314172	0.730143689578457	0.715729131143268\\
1.5	0.328891676652598	0.741359304476472	0.709241120561856\\
1.5	0.341255928754291	0.752367096967051	0.703223584571265\\
1.5	0.353793780602429	0.763162841084325	0.697635496597419\\
1.5	0.366499761262991	0.773742494322337	0.692428303594618\\
1.5	0.379368276771857	0.784102202600442	0.687549132887069\\
1.5	0.392393618241769	0.794238304822092	0.682943679235329\\
1.5	0.405569970164763	0.804147337016097	0.678558662193044\\
1.5	0.418891418885081	0.813826036051075	0.67434378741287\\
1.5	0.432351961217132	0.823271342915544	0.670253187715552\\
1.5	0.445945513182735	0.832480405557786	0.666246357336158\\
1.5	0.459665918841652	0.841450581281375	0.662288623474186\\
1.5	0.473506959189401	0.850179438693979	0.658351221673191\\
1.5	0.487462361096317	0.85866475920876	0.654411055137886\\
1.5	0.501525806262096	0.866904538099407	0.650450223206468\\
1.5	0.515690940160279	0.874896985111485	0.646455401874959\\
1.5	0.529951380947623	0.882640524634436	0.642417151055155\\
1.5	0.544300728313782	0.890133795440129	0.638329210947586\\
1.5	0.558732572247415	0.897375649995373	0.634187835387498\\
1.5	0.57324050169555	0.904365153357279	0.629991194998376\\
1.5	0.587818113093873	0.911101581661729	0.625738868900922\\
1.5	0.602459018746568	0.917584420216504	0.621431431633141\\
1.5	0.617156855035314	0.923813361211857	0.617070132480554\\
1.5	0.631905290438127	0.92978830106243	0.612656657833837\\
1.5	0.6466980333399	0.935509337395467	0.608192963377087\\
1.5	0.661528839617671	0.940976765701173	0.603681161495104\\
1.5	0.676391519984904	0.94619107566195	0.599123449738606\\
1.5	0.691279947080357	0.951152947177946	0.59452206790159\\
1.5	0.706188062288412	0.955863246106974	0.589879273662389\\
1.5	0.721109882279076	0.960323019737426	0.585197329319937\\
1.5	0.736039505257226	0.964533492013186	0.580478494543977\\
1.5	0.75097111691199	0.968496058529893	0.575725022018312\\
1.5	0.765898996058536	0.972212281322139	0.570939154289557\\
1.5	0.780817519965826	0.975683883461272	0.566123121053175\\
1.5	0.795721169365275	0.97891274348356	0.561279136605559\\
1.5	0.810604533136476	0.981900889668376	0.556409397399199\\
1.5	0.825462312667457	0.984650494185936	0.551516079695734\\
1.5	0.840289325888138	0.987163867133889	0.54660133731964\\
1.5	0.855080510976839	0.989443450481767	0.541667299515139\\
1.5	0.86983092974082	0.991491811941914	0.536716068908599\\
1.5	0.884535770672926	0.993311638785082	0.531749719578143\\
1.5	0.899190351687411	0.994905731618374	0.526770295231793\\
1.5	0.913790122539012	0.996276998142666	0.521779807495103\\
1.5	0.928330666930242	0.997428446906011	0.516780234308683\\
1.5	0.942807704312699	0.998363181068902	0.511773518435872\\
1.5	0.95721709138901	0.999084392196567	0.50676156608017\\
1.5	0.971554823322703	0.999595354092743	0.501746245611874\\
1.5	0.985817034663989	0.999899416688637	0.496729386403004\\
1.5	1	1	0.491669087346906\\
};
\addplot3[%
surf,
shader=interp,colormap={mymap}{[1pt] rgb(0pt)=(1,0.4,0); rgb(2pt)=(1,0.4,0)},mesh/rows=101]
table[row sep=crcr, point meta=\thisrow{c}] {%
%
x	y	z	c\\
0.5	0	0	0.710202759607373\\
0.5	1.11327767495586e-05	0.00471862581271275	0.713003157837853\\
0.5	4.46175251031912e-05	0.009446325184051	0.715846486660142\\
0.5	0.000100583311362513	0.0141829653360114	0.718788838379794\\
0.5	0.000179158434668431	0.018928411755669	0.721886092228356\\
0.5	0.000280470402701511	0.02368252819604	0.725200636220892\\
0.5	0.000404645907256436	0.0284451766772965	0.728799085274723\\
0.5	0.000551810799695644	0.0332162174883388	0.732749123983336\\
0.5	0.000722090066287311	0.037995509188729	0.737115575253328\\
0.5	0.000915607803432999	0.0427829086109896	0.741955893907011\\
0.5	0.0011324871927904	0.0475782708632729	0.747315373652289\\
0.5	0.00137285047629673	0.0523814493324038	0.753222424045811\\
0.5	0.00163681893109844	0.057192295687301	0.759684310525652\\
0.5	0.00192451284439304	0.0620106598827802	0.766683748102665\\
0.5	0.00223605148818898	0.0668363901637434	0.774176695511256\\
0.5	0.00257155309398959	0.0716693330697584	0.782091614494494\\
0.5	0.00293113482740716	0.0765093334400312	0.790330346421089\\
0.5	0.00331491276271365	0.0813562344187764	0.79877062752665\\
0.5	0.00372300185733413	0.0862098774609879	0.807270128930359\\
0.5	0.00415551592628969	0.0910701023386145	0.815671782700195\\
0.5	0.00461256761659621	0.0959367471471425	0.82381005350733\\
0.5	0.00509426838162598	0.100809648312589	0.831517746503351\\
0.5	0.00560072845543873	0.105688640598912	0.838632911419452\\
0.5	0.00613205682708918	0.11057355711583	0.845005411376292\\
0.5	0.00668836121491816	0.115464229327074	0.850502769049297\\
0.5	0.00726974804083431	0.120360487059049	0.855014975717658\\
0.5	0.00787632240459385	0.125262158509929	0.858458041096608\\
0.5	0.00850818805808555	0.13016907025918	0.860776163494182\\
0.5	0.00916544737962849	0.135081047277508	0.861942500814233\\
0.5	0.00984820134829017	0.139997912937243	0.861958614648411\\
0.5	0.0105565495182326	0.144919489023162	0.860852735617087\\
0.5	0.0112905899930939	0.149845595743738	0.858677054109646\\
0.5	0.012050419400414	0.154776051742839	0.855504274967457\\
0.5	0.0128361328661109	0.159710674111862	0.851423687967759\\
0.5	0.0136478239890177	0.164649278402306	0.84653700048434\\
0.5	0.0144855848154856	0.169591678638796	0.840954157904247\\
0.5	0.0153495058140643	0.174537687332543	0.834789345403929\\
0.5	0.0162396758502651	0.179487115495259	0.828157325806199\\
0.5	0.0171561821614173	0.184439772653511	0.821170226448696\\
0.5	0.0180991103316243	0.189395466863524	0.813934846725693\\
0.5	0.0190685442668299	0.194354004726437	0.806550519904208\\
0.5	0.0200645661700016	0.199315191403999	0.799107529846312\\
0.5	0.0210872565164405	0.204278830634725	0.791686056496189\\
0.5	0.0221366940292259	0.20924472475049	0.784355603830603\\
0.5	0.023212955654804	0.214212674693577	0.777174850281121\\
0.5	0.0243161165387281	0.219182480034174	0.770191853862643\\
0.5	0.025446250001561	0.224153938988321	0.763444541554457\\
0.5	0.0266034275149466	0.229126848436298	0.756961413910482\\
0.5	0.0277877186778607	0.234101003941464	0.750762400416764\\
0.5	0.0289991911930496	0.239076199769547	0.744859807805785\\
0.5	0.0302379108436654	0.244052228908366	0.73925931151808\\
0.5	0.0315039414701067	0.24902888308801	0.733960949044504\\
0.5	0.0327973449470747	0.254005952801449	0.728960082407259\\
0.5	0.0341181811608519	0.258983227325592	0.724248305111179\\
0.5	0.0354665079868145	0.263960494742775	0.719814276220681\\
0.5	0.0368423812671862	0.268937541962694	0.715644470612632\\
0.5	0.0382458547890421	0.273914154744766	0.711723839838041\\
0.5	0.0396769802625738	0.278890117720924	0.708036382389731\\
0.5	0.0411358072996216	0.283865214418836	0.704565625567164\\
0.5	0.0426223833924861	0.288839227285554	0.701295023640316\\
0.5	0.0441367538930258	0.293811937711588	0.698208278749354\\
0.5	0.0456789619920503	0.298783126055387	0.695289592055349\\
0.5	0.0472490486990192	0.303752571668251	0.692523853197512\\
0.5	0.0488470528220538	0.308720052919643	0.689896776228118\\
0.5	0.0504730109482718	0.313685347222913	0.687394989990281\\
0.5	0.0521269574244539	0.318648231061428	0.685006090465981\\
0.5	0.0538089243380495	0.323608480015096	0.682718662029128\\
0.5	0.0555189414985328	0.328565868787292	0.680522273853469\\
0.5	0.0572570364191156	0.333520171232163	0.67840745699785\\
0.5	0.0590232342988274	0.338471160382329	0.676365666960044\\
0.5	0.0608175580049697	0.34341860847696	0.674389235783157\\
0.5	0.0626400280559541	0.348362286990222	0.672471317134862\\
0.5	0.064490662604533	0.3533019666601	0.670605827172182\\
0.5	0.0663694774214291	0.358237417517579	0.668787383459802\\
0.5	0.0682764858793752	0.363168408916186	0.667011243730727\\
0.5	0.0702116989375697	0.368094709561873	0.665273245863648\\
0.5	0.0721751251265572	0.373016087543259	0.663569750098933\\
0.5	0.0741667705335422	0.377932310362199	0.66189758422006\\
0.5	0.0761866387881432	0.382843144964686	0.660253992184443\\
0.5	0.0782347310485954	0.38774835777208	0.658636586490997\\
0.5	0.0803110459884098	0.392647714712651	0.657043304415871\\
0.5	0.0824155797834956	0.397540981253432	0.65547236812651\\
0.5	0.084548326099754	0.402427922432375	0.653922248592746\\
0.5	0.0867092760811502	0.4073083028908	0.652391633146571\\
0.5	0.0888984183382709	0.412181886906127	0.65087939649558\\
0.5	0.0911157389373742	0.417048438424896	0.649384574964844\\
0.5	0.0933612213899392	0.421907721096044	0.647906343724683\\
0.5	0.0956348466427212	0.42675949830445	0.646443996754803\\
0.5	0.0979365930683194	0.431603533204733	0.644996929295963\\
0.5	0.100266436456264	0.436439588755293	0.643564622546847\\
0.5	0.102624350004627	0.441267427752584	0.642146630374431\\
0.5	0.105010304312169	0.446086812865616	0.640742567819433\\
0.5	0.107424267371012	0.450897506670668	0.639352101193731\\
0.5	0.109866204559871	0.455699271686218	0.637974939582442\\
0.5	0.112336078637817	0.460491870408058	0.636610827579862\\
0.5	0.114833849738607	0.465275065344603	0.635259539104384\\
0.5	0.117359475365564	0.470048619052377	0.633920872153155\\
0.5	0.119912910387022	0.474812294171664	0.632594644371965\\
0.5	0.122494107032347	0.479565853462319	0.631280689329641\\
0.5	0.125103014888515	0.484309059839721	0.629978853399059\\
0.5	0.127739580897283	0.489041676410868	0.628690460824559\\
0.505	0	0	0.710202759607373\\
0.505	1.11327767495586e-05	0.00471862581271274	0.713003157837853\\
0.505	4.46175251031912e-05	0.009446325184051	0.715846486660142\\
0.505	0.000100583311362513	0.0141829653360114	0.718788838379794\\
0.505	0.000179158434668431	0.018928411755669	0.721886092228356\\
0.505	0.000280470402701511	0.02368252819604	0.725200636220892\\
0.505	0.000404645907256436	0.0284451766772965	0.728799085274723\\
0.505	0.000551810799695644	0.0332162174883388	0.732749123983336\\
0.505	0.000722090066287311	0.037995509188729	0.737115575253328\\
0.505	0.000915607803432999	0.0427829086109896	0.741955893907011\\
0.505	0.0011324871927904	0.0475782708632729	0.747315373652289\\
0.505	0.00137285047629673	0.0523814493324038	0.75322242404581\\
0.505	0.00163681893109844	0.057192295687301	0.759684310525652\\
0.505	0.00192451284439304	0.0620106598827802	0.766683748102665\\
0.505	0.00223605148818898	0.0668363901637434	0.774176695511256\\
0.505	0.00257155309398959	0.0716693330697584	0.782091614494494\\
0.505	0.00293113482740716	0.0765093334400312	0.79033034642109\\
0.505	0.00331491276271365	0.0813562344187764	0.79877062752665\\
0.505	0.00372300185733414	0.0862098774609879	0.807270128930359\\
0.505	0.00415551592628969	0.0910701023386145	0.815671782700194\\
0.505	0.00461256761659621	0.0959367471471425	0.823810053507331\\
0.505	0.00509426838162598	0.100809648312589	0.831517746503354\\
0.505	0.00560072845543873	0.105688640598912	0.838632911419452\\
0.505	0.00613205682708918	0.11057355711583	0.845005411376294\\
0.505	0.00668836121491815	0.115464229327074	0.850502769049302\\
0.505	0.00726974804083431	0.120360487059049	0.855014975717657\\
0.505	0.00787632240459385	0.125262158509929	0.858458041096606\\
0.505	0.00850818805808555	0.13016907025918	0.860776163494182\\
0.505	0.00916544737962849	0.135081047277508	0.861942500814225\\
0.505	0.00984820134829017	0.139997912937243	0.861958614648421\\
0.505	0.0105565495182326	0.144919489023162	0.860852735617091\\
0.505	0.0112905899930939	0.149845595743738	0.858677054109652\\
0.505	0.012050419400414	0.154776051742839	0.855504274967457\\
0.505	0.0128361328661109	0.159710674111862	0.851423687967754\\
0.505	0.0136478239890177	0.164649278402306	0.84653700048434\\
0.505	0.0144855848154856	0.169591678638796	0.840954157904249\\
0.505	0.0153495058140643	0.174537687332543	0.834789345403935\\
0.505	0.0162396758502651	0.179487115495259	0.828157325806202\\
0.505	0.0171561821614173	0.184439772653511	0.821170226448694\\
0.505	0.0180991103316243	0.189395466863524	0.81393484672569\\
0.505	0.0190685442668299	0.194354004726437	0.806550519904212\\
0.505	0.0200645661700016	0.199315191403999	0.799107529846305\\
0.505	0.0210872565164405	0.204278830634725	0.791686056496193\\
0.505	0.0221366940292259	0.20924472475049	0.784355603830604\\
0.505	0.023212955654804	0.214212674693577	0.777174850281118\\
0.505	0.0243161165387281	0.219182480034174	0.770191853862644\\
0.505	0.025446250001561	0.224153938988321	0.763444541554458\\
0.505	0.0266034275149466	0.229126848436298	0.756961413910483\\
0.505	0.0277877186778607	0.234101003941464	0.750762400416764\\
0.505	0.0289991911930496	0.239076199769547	0.744859807805784\\
0.505	0.0302379108436654	0.244052228908366	0.73925931151808\\
0.505	0.0315039414701067	0.24902888308801	0.733960949044503\\
0.505	0.0327973449470747	0.254005952801449	0.728960082407262\\
0.505	0.0341181811608519	0.258983227325592	0.724248305111177\\
0.505	0.0354665079868145	0.263960494742775	0.719814276220681\\
0.505	0.0368423812671862	0.268937541962694	0.715644470612631\\
0.505	0.0382458547890421	0.273914154744766	0.711723839838042\\
0.505	0.0396769802625738	0.278890117720924	0.70803638238973\\
0.505	0.0411358072996216	0.283865214418836	0.704565625567163\\
0.505	0.0426223833924861	0.288839227285554	0.701295023640317\\
0.505	0.0441367538930258	0.293811937711588	0.698208278749353\\
0.505	0.0456789619920503	0.298783126055387	0.695289592055349\\
0.505	0.0472490486990192	0.303752571668251	0.692523853197513\\
0.505	0.0488470528220538	0.308720052919643	0.68989677622812\\
0.505	0.0504730109482718	0.313685347222913	0.687394989990281\\
0.505	0.0521269574244539	0.318648231061428	0.68500609046598\\
0.505	0.0538089243380495	0.323608480015096	0.682718662029128\\
0.505	0.0555189414985328	0.328565868787292	0.68052227385347\\
0.505	0.0572570364191156	0.333520171232163	0.678407456997849\\
0.505	0.0590232342988274	0.338471160382329	0.676365666960043\\
0.505	0.0608175580049697	0.34341860847696	0.674389235783156\\
0.505	0.0626400280559541	0.348362286990222	0.672471317134862\\
0.505	0.064490662604533	0.3533019666601	0.670605827172183\\
0.505	0.0663694774214291	0.358237417517579	0.668787383459804\\
0.505	0.0682764858793752	0.363168408916186	0.667011243730728\\
0.505	0.0702116989375697	0.368094709561873	0.665273245863648\\
0.505	0.0721751251265572	0.373016087543259	0.663569750098932\\
0.505	0.0741667705335423	0.377932310362199	0.661897584220061\\
0.505	0.0761866387881432	0.382843144964686	0.660253992184443\\
0.505	0.0782347310485954	0.38774835777208	0.658636586490997\\
0.505	0.0803110459884098	0.392647714712651	0.657043304415871\\
0.505	0.0824155797834956	0.397540981253432	0.655472368126509\\
0.505	0.084548326099754	0.402427922432375	0.653922248592745\\
0.505	0.0867092760811502	0.407308302890799	0.65239163314657\\
0.505	0.0888984183382709	0.412181886906127	0.650879396495579\\
0.505	0.0911157389373742	0.417048438424896	0.649384574964845\\
0.505	0.0933612213899392	0.421907721096044	0.647906343724686\\
0.505	0.0956348466427212	0.42675949830445	0.646443996754804\\
0.505	0.0979365930683194	0.431603533204733	0.644996929295961\\
0.505	0.100266436456264	0.436439588755293	0.643564622546846\\
0.505	0.102624350004627	0.441267427752584	0.64214663037443\\
0.505	0.105010304312169	0.446086812865616	0.640742567819433\\
0.505	0.107424267371012	0.450897506670668	0.639352101193731\\
0.505	0.109866204559871	0.455699271686218	0.637974939582444\\
0.505	0.112336078637817	0.460491870408058	0.636610827579861\\
0.505	0.114833849738607	0.465275065344603	0.635259539104383\\
0.505	0.117359475365564	0.470048619052377	0.633920872153155\\
0.505	0.119912910387022	0.474812294171664	0.632594644371966\\
0.505	0.122494107032347	0.479565853462319	0.631280689329641\\
0.505	0.125103014888515	0.484309059839721	0.629978853399059\\
0.505	0.127739580897283	0.489041676410868	0.62869046082456\\
0.51	0	0	0.710202759607373\\
0.51	1.11327767495586e-05	0.00471862581271274	0.713003157837853\\
0.51	4.46175251031912e-05	0.009446325184051	0.715846486660142\\
0.51	0.000100583311362513	0.0141829653360114	0.718788838379794\\
0.51	0.000179158434668431	0.018928411755669	0.721886092228356\\
0.51	0.000280470402701511	0.02368252819604	0.725200636220892\\
0.51	0.000404645907256436	0.0284451766772965	0.728799085274723\\
0.51	0.000551810799695644	0.0332162174883388	0.732749123983336\\
0.51	0.000722090066287311	0.037995509188729	0.737115575253327\\
0.51	0.000915607803432999	0.0427829086109896	0.741955893907011\\
0.51	0.0011324871927904	0.0475782708632729	0.747315373652289\\
0.51	0.00137285047629673	0.0523814493324038	0.75322242404581\\
0.51	0.00163681893109844	0.057192295687301	0.759684310525652\\
0.51	0.00192451284439304	0.0620106598827802	0.766683748102665\\
0.51	0.00223605148818898	0.0668363901637434	0.774176695511256\\
0.51	0.00257155309398959	0.0716693330697584	0.782091614494493\\
0.51	0.00293113482740716	0.0765093334400312	0.79033034642109\\
0.51	0.00331491276271365	0.0813562344187764	0.79877062752665\\
0.51	0.00372300185733413	0.0862098774609879	0.807270128930359\\
0.51	0.00415551592628969	0.0910701023386145	0.815671782700194\\
0.51	0.00461256761659621	0.0959367471471425	0.82381005350733\\
0.51	0.00509426838162598	0.100809648312589	0.831517746503354\\
0.51	0.00560072845543873	0.105688640598912	0.838632911419452\\
0.51	0.00613205682708918	0.11057355711583	0.845005411376292\\
0.51	0.00668836121491816	0.115464229327074	0.850502769049303\\
0.51	0.00726974804083431	0.120360487059049	0.855014975717657\\
0.51	0.00787632240459385	0.125262158509929	0.858458041096606\\
0.51	0.00850818805808555	0.13016907025918	0.860776163494181\\
0.51	0.00916544737962849	0.135081047277508	0.861942500814222\\
0.51	0.00984820134829017	0.139997912937243	0.861958614648422\\
0.51	0.0105565495182326	0.144919489023162	0.860852735617092\\
0.51	0.0112905899930939	0.149845595743738	0.858677054109651\\
0.51	0.012050419400414	0.154776051742839	0.855504274967456\\
0.51	0.0128361328661109	0.159710674111862	0.851423687967753\\
0.51	0.0136478239890177	0.164649278402306	0.846537000484341\\
0.51	0.0144855848154856	0.169591678638796	0.840954157904249\\
0.51	0.0153495058140643	0.174537687332543	0.834789345403937\\
0.51	0.0162396758502651	0.179487115495259	0.828157325806201\\
0.51	0.0171561821614173	0.184439772653511	0.821170226448692\\
0.51	0.0180991103316243	0.189395466863524	0.81393484672569\\
0.51	0.0190685442668299	0.194354004726437	0.806550519904214\\
0.51	0.0200645661700016	0.199315191403999	0.799107529846304\\
0.51	0.0210872565164405	0.204278830634725	0.791686056496195\\
0.51	0.0221366940292259	0.20924472475049	0.784355603830602\\
0.51	0.023212955654804	0.214212674693577	0.777174850281118\\
0.51	0.0243161165387281	0.219182480034174	0.770191853862643\\
0.51	0.025446250001561	0.224153938988321	0.763444541554457\\
0.51	0.0266034275149466	0.229126848436298	0.756961413910483\\
0.51	0.0277877186778607	0.234101003941464	0.750762400416763\\
0.51	0.0289991911930496	0.239076199769547	0.744859807805784\\
0.51	0.0302379108436654	0.244052228908366	0.73925931151808\\
0.51	0.0315039414701067	0.24902888308801	0.733960949044504\\
0.51	0.0327973449470747	0.254005952801449	0.728960082407259\\
0.51	0.0341181811608519	0.258983227325591	0.724248305111178\\
0.51	0.0354665079868145	0.263960494742775	0.719814276220682\\
0.51	0.0368423812671862	0.268937541962694	0.715644470612629\\
0.51	0.0382458547890421	0.273914154744766	0.711723839838042\\
0.51	0.0396769802625738	0.278890117720924	0.70803638238973\\
0.51	0.0411358072996216	0.283865214418836	0.704565625567166\\
0.51	0.0426223833924861	0.288839227285554	0.701295023640315\\
0.51	0.0441367538930258	0.293811937711588	0.698208278749353\\
0.51	0.0456789619920503	0.298783126055387	0.69528959205535\\
0.51	0.0472490486990192	0.303752571668251	0.69252385319751\\
0.51	0.0488470528220538	0.308720052919643	0.689896776228119\\
0.51	0.0504730109482718	0.313685347222913	0.687394989990282\\
0.51	0.0521269574244539	0.318648231061428	0.685006090465981\\
0.51	0.0538089243380495	0.323608480015096	0.682718662029128\\
0.51	0.0555189414985328	0.328565868787292	0.680522273853469\\
0.51	0.0572570364191156	0.333520171232163	0.67840745699785\\
0.51	0.0590232342988274	0.338471160382329	0.676365666960045\\
0.51	0.0608175580049697	0.34341860847696	0.674389235783156\\
0.51	0.0626400280559541	0.348362286990222	0.672471317134861\\
0.51	0.064490662604533	0.3533019666601	0.670605827172182\\
0.51	0.0663694774214291	0.358237417517579	0.668787383459805\\
0.51	0.0682764858793752	0.363168408916186	0.667011243730728\\
0.51	0.0702116989375697	0.368094709561873	0.665273245863648\\
0.51	0.0721751251265572	0.373016087543259	0.663569750098932\\
0.51	0.0741667705335422	0.377932310362199	0.66189758422006\\
0.51	0.0761866387881432	0.382843144964686	0.660253992184443\\
0.51	0.0782347310485954	0.38774835777208	0.658636586490998\\
0.51	0.0803110459884098	0.392647714712651	0.657043304415871\\
0.51	0.0824155797834956	0.397540981253432	0.65547236812651\\
0.51	0.084548326099754	0.402427922432375	0.653922248592745\\
0.51	0.0867092760811502	0.407308302890799	0.652391633146569\\
0.51	0.0888984183382709	0.412181886906127	0.650879396495579\\
0.51	0.0911157389373742	0.417048438424897	0.649384574964844\\
0.51	0.0933612213899392	0.421907721096044	0.647906343724685\\
0.51	0.0956348466427212	0.42675949830445	0.646443996754807\\
0.51	0.0979365930683194	0.431603533204733	0.644996929295962\\
0.51	0.100266436456264	0.436439588755293	0.643564622546847\\
0.51	0.102624350004627	0.441267427752584	0.642146630374428\\
0.51	0.105010304312169	0.446086812865616	0.640742567819433\\
0.51	0.107424267371012	0.450897506670668	0.63935210119373\\
0.51	0.109866204559871	0.455699271686218	0.637974939582443\\
0.51	0.112336078637817	0.460491870408058	0.636610827579862\\
0.51	0.114833849738607	0.465275065344603	0.635259539104382\\
0.51	0.117359475365564	0.470048619052377	0.633920872153154\\
0.51	0.119912910387023	0.474812294171664	0.632594644371965\\
0.51	0.122494107032347	0.479565853462319	0.631280689329641\\
0.51	0.125103014888515	0.484309059839721	0.629978853399059\\
0.51	0.127739580897283	0.489041676410868	0.628690460824559\\
0.515	0	0	0.710202759607373\\
0.515	1.11327767495586e-05	0.00471862581271275	0.713003157837853\\
0.515	4.46175251031912e-05	0.009446325184051	0.715846486660142\\
0.515	0.000100583311362513	0.0141829653360114	0.718788838379794\\
0.515	0.000179158434668431	0.018928411755669	0.721886092228356\\
0.515	0.000280470402701511	0.02368252819604	0.725200636220892\\
0.515	0.000404645907256436	0.0284451766772965	0.728799085274723\\
0.515	0.000551810799695644	0.0332162174883389	0.732749123983336\\
0.515	0.000722090066287311	0.037995509188729	0.737115575253327\\
0.515	0.000915607803432999	0.0427829086109896	0.741955893907011\\
0.515	0.0011324871927904	0.0475782708632729	0.747315373652289\\
0.515	0.00137285047629673	0.0523814493324038	0.753222424045811\\
0.515	0.00163681893109844	0.057192295687301	0.759684310525652\\
0.515	0.00192451284439304	0.0620106598827802	0.766683748102665\\
0.515	0.00223605148818898	0.0668363901637434	0.774176695511256\\
0.515	0.00257155309398959	0.0716693330697584	0.782091614494494\\
0.515	0.00293113482740716	0.0765093334400312	0.79033034642109\\
0.515	0.00331491276271365	0.0813562344187764	0.79877062752665\\
0.515	0.00372300185733413	0.0862098774609879	0.807270128930359\\
0.515	0.00415551592628969	0.0910701023386145	0.815671782700193\\
0.515	0.00461256761659621	0.0959367471471425	0.823810053507331\\
0.515	0.00509426838162598	0.100809648312589	0.831517746503354\\
0.515	0.00560072845543873	0.105688640598912	0.838632911419452\\
0.515	0.00613205682708918	0.11057355711583	0.845005411376293\\
0.515	0.00668836121491816	0.115464229327074	0.850502769049302\\
0.515	0.00726974804083431	0.120360487059049	0.855014975717656\\
0.515	0.00787632240459385	0.125262158509929	0.858458041096607\\
0.515	0.00850818805808555	0.13016907025918	0.860776163494181\\
0.515	0.00916544737962849	0.135081047277508	0.861942500814224\\
0.515	0.00984820134829017	0.139997912937243	0.861958614648418\\
0.515	0.0105565495182326	0.144919489023162	0.860852735617091\\
0.515	0.0112905899930939	0.149845595743738	0.85867705410965\\
0.515	0.012050419400414	0.154776051742839	0.855504274967457\\
0.515	0.0128361328661109	0.159710674111862	0.851423687967756\\
0.515	0.0136478239890177	0.164649278402306	0.84653700048434\\
0.515	0.0144855848154856	0.169591678638796	0.840954157904249\\
0.515	0.0153495058140643	0.174537687332543	0.834789345403935\\
0.515	0.0162396758502651	0.179487115495259	0.828157325806199\\
0.515	0.0171561821614173	0.184439772653511	0.821170226448693\\
0.515	0.0180991103316243	0.189395466863524	0.813934846725693\\
0.515	0.0190685442668299	0.194354004726437	0.806550519904216\\
0.515	0.0200645661700016	0.199315191403999	0.799107529846304\\
0.515	0.0210872565164405	0.204278830634725	0.791686056496193\\
0.515	0.0221366940292259	0.20924472475049	0.784355603830602\\
0.515	0.023212955654804	0.214212674693577	0.777174850281118\\
0.515	0.0243161165387281	0.219182480034174	0.770191853862642\\
0.515	0.025446250001561	0.224153938988321	0.76344454155446\\
0.515	0.0266034275149466	0.229126848436298	0.756961413910481\\
0.515	0.0277877186778607	0.234101003941464	0.750762400416761\\
0.515	0.0289991911930496	0.239076199769547	0.744859807805786\\
0.515	0.0302379108436654	0.244052228908366	0.739259311518081\\
0.515	0.0315039414701067	0.24902888308801	0.733960949044503\\
0.515	0.0327973449470747	0.254005952801449	0.72896008240726\\
0.515	0.0341181811608519	0.258983227325592	0.724248305111177\\
0.515	0.0354665079868145	0.263960494742775	0.719814276220681\\
0.515	0.0368423812671862	0.268937541962694	0.71564447061263\\
0.515	0.0382458547890421	0.273914154744766	0.711723839838041\\
0.515	0.0396769802625738	0.278890117720924	0.70803638238973\\
0.515	0.0411358072996216	0.283865214418836	0.704565625567165\\
0.515	0.0426223833924861	0.288839227285554	0.701295023640316\\
0.515	0.0441367538930258	0.293811937711588	0.698208278749354\\
0.515	0.0456789619920503	0.298783126055387	0.695289592055349\\
0.515	0.0472490486990192	0.303752571668251	0.692523853197511\\
0.515	0.0488470528220538	0.308720052919643	0.68989677622812\\
0.515	0.0504730109482718	0.313685347222913	0.68739498999028\\
0.515	0.0521269574244539	0.318648231061427	0.68500609046598\\
0.515	0.0538089243380495	0.323608480015097	0.682718662029128\\
0.515	0.0555189414985328	0.328565868787292	0.680522273853469\\
0.515	0.0572570364191156	0.333520171232163	0.67840745699785\\
0.515	0.0590232342988274	0.338471160382329	0.676365666960044\\
0.515	0.0608175580049697	0.34341860847696	0.674389235783156\\
0.515	0.0626400280559541	0.348362286990222	0.672471317134861\\
0.515	0.064490662604533	0.3533019666601	0.670605827172182\\
0.515	0.0663694774214291	0.358237417517579	0.668787383459804\\
0.515	0.0682764858793752	0.363168408916186	0.667011243730729\\
0.515	0.0702116989375697	0.368094709561873	0.665273245863649\\
0.515	0.0721751251265572	0.373016087543259	0.663569750098931\\
0.515	0.0741667705335422	0.377932310362199	0.66189758422006\\
0.515	0.0761866387881432	0.382843144964686	0.660253992184442\\
0.515	0.0782347310485954	0.38774835777208	0.658636586490997\\
0.515	0.0803110459884098	0.392647714712651	0.657043304415871\\
0.515	0.0824155797834956	0.397540981253432	0.655472368126509\\
0.515	0.084548326099754	0.402427922432375	0.653922248592747\\
0.515	0.0867092760811501	0.407308302890799	0.652391633146571\\
0.515	0.0888984183382709	0.412181886906127	0.650879396495578\\
0.515	0.0911157389373742	0.417048438424896	0.649384574964843\\
0.515	0.0933612213899392	0.421907721096044	0.647906343724685\\
0.515	0.0956348466427212	0.42675949830445	0.646443996754805\\
0.515	0.0979365930683194	0.431603533204733	0.644996929295962\\
0.515	0.100266436456264	0.436439588755293	0.643564622546847\\
0.515	0.102624350004627	0.441267427752585	0.64214663037443\\
0.515	0.105010304312169	0.446086812865616	0.640742567819432\\
0.515	0.107424267371012	0.450897506670668	0.639352101193731\\
0.515	0.109866204559871	0.455699271686218	0.637974939582443\\
0.515	0.112336078637817	0.460491870408058	0.636610827579862\\
0.515	0.114833849738607	0.465275065344603	0.635259539104383\\
0.515	0.117359475365564	0.470048619052377	0.633920872153155\\
0.515	0.119912910387022	0.474812294171664	0.632594644371964\\
0.515	0.122494107032347	0.479565853462319	0.63128068932964\\
0.515	0.125103014888515	0.484309059839721	0.629978853399057\\
0.515	0.127739580897283	0.489041676410868	0.628690460824557\\
0.52	0	0	0.710202759607373\\
0.52	1.11327767495586e-05	0.00471862581271275	0.713003157837853\\
0.52	4.46175251031912e-05	0.009446325184051	0.715846486660142\\
0.52	0.000100583311362513	0.0141829653360114	0.718788838379794\\
0.52	0.000179158434668431	0.018928411755669	0.721886092228356\\
0.52	0.000280470402701511	0.02368252819604	0.725200636220892\\
0.52	0.000404645907256436	0.0284451766772965	0.728799085274723\\
0.52	0.000551810799695644	0.0332162174883389	0.732749123983336\\
0.52	0.000722090066287311	0.037995509188729	0.737115575253328\\
0.52	0.000915607803432999	0.0427829086109896	0.741955893907011\\
0.52	0.0011324871927904	0.0475782708632729	0.747315373652289\\
0.52	0.00137285047629673	0.0523814493324038	0.753222424045811\\
0.52	0.00163681893109844	0.057192295687301	0.759684310525652\\
0.52	0.00192451284439304	0.0620106598827802	0.766683748102665\\
0.52	0.00223605148818898	0.0668363901637434	0.774176695511256\\
0.52	0.00257155309398959	0.0716693330697584	0.782091614494493\\
0.52	0.00293113482740716	0.0765093334400312	0.79033034642109\\
0.52	0.00331491276271365	0.0813562344187764	0.79877062752665\\
0.52	0.00372300185733413	0.0862098774609879	0.807270128930359\\
0.52	0.00415551592628969	0.0910701023386145	0.815671782700193\\
0.52	0.00461256761659621	0.0959367471471425	0.82381005350733\\
0.52	0.00509426838162598	0.100809648312589	0.831517746503354\\
0.52	0.00560072845543873	0.105688640598912	0.838632911419452\\
0.52	0.00613205682708918	0.11057355711583	0.845005411376293\\
0.52	0.00668836121491816	0.115464229327074	0.850502769049301\\
0.52	0.00726974804083431	0.120360487059049	0.855014975717657\\
0.52	0.00787632240459385	0.125262158509929	0.858458041096605\\
0.52	0.00850818805808554	0.13016907025918	0.860776163494183\\
0.52	0.00916544737962849	0.135081047277508	0.861942500814225\\
0.52	0.00984820134829017	0.139997912937243	0.86195861464842\\
0.52	0.0105565495182326	0.144919489023162	0.86085273561709\\
0.52	0.0112905899930939	0.149845595743738	0.858677054109651\\
0.52	0.012050419400414	0.154776051742839	0.855504274967457\\
0.52	0.0128361328661109	0.159710674111862	0.851423687967756\\
0.52	0.0136478239890177	0.164649278402306	0.846537000484339\\
0.52	0.0144855848154856	0.169591678638796	0.840954157904249\\
0.52	0.0153495058140643	0.174537687332543	0.834789345403935\\
0.52	0.0162396758502651	0.179487115495259	0.828157325806199\\
0.52	0.0171561821614173	0.184439772653511	0.821170226448693\\
0.52	0.0180991103316243	0.189395466863524	0.813934846725693\\
0.52	0.0190685442668299	0.194354004726437	0.806550519904212\\
0.52	0.0200645661700016	0.199315191403999	0.799107529846303\\
0.52	0.0210872565164405	0.204278830634725	0.791686056496195\\
0.52	0.0221366940292259	0.20924472475049	0.784355603830603\\
0.52	0.023212955654804	0.214212674693577	0.777174850281116\\
0.52	0.0243161165387281	0.219182480034174	0.770191853862644\\
0.52	0.025446250001561	0.224153938988321	0.763444541554457\\
0.52	0.0266034275149466	0.229126848436298	0.756961413910481\\
0.52	0.0277877186778607	0.234101003941464	0.750762400416763\\
0.52	0.0289991911930496	0.239076199769547	0.744859807805786\\
0.52	0.0302379108436654	0.244052228908366	0.73925931151808\\
0.52	0.0315039414701067	0.24902888308801	0.733960949044504\\
0.52	0.0327973449470747	0.254005952801449	0.72896008240726\\
0.52	0.0341181811608519	0.258983227325591	0.724248305111177\\
0.52	0.0354665079868145	0.263960494742775	0.719814276220682\\
0.52	0.0368423812671862	0.268937541962694	0.715644470612629\\
0.52	0.0382458547890421	0.273914154744766	0.711723839838041\\
0.52	0.0396769802625738	0.278890117720924	0.708036382389732\\
0.52	0.0411358072996216	0.283865214418836	0.704565625567165\\
0.52	0.0426223833924861	0.288839227285554	0.701295023640315\\
0.52	0.0441367538930258	0.293811937711588	0.698208278749353\\
0.52	0.0456789619920503	0.298783126055387	0.695289592055351\\
0.52	0.0472490486990192	0.303752571668251	0.692523853197512\\
0.52	0.0488470528220538	0.308720052919643	0.689896776228118\\
0.52	0.0504730109482718	0.313685347222913	0.687394989990282\\
0.52	0.0521269574244539	0.318648231061428	0.68500609046598\\
0.52	0.0538089243380495	0.323608480015096	0.682718662029128\\
0.52	0.0555189414985328	0.328565868787292	0.68052227385347\\
0.52	0.0572570364191156	0.333520171232163	0.67840745699785\\
0.52	0.0590232342988274	0.338471160382329	0.676365666960045\\
0.52	0.0608175580049697	0.34341860847696	0.674389235783157\\
0.52	0.0626400280559541	0.348362286990222	0.672471317134861\\
0.52	0.064490662604533	0.3533019666601	0.670605827172182\\
0.52	0.0663694774214291	0.358237417517579	0.668787383459804\\
0.52	0.0682764858793752	0.363168408916186	0.667011243730729\\
0.52	0.0702116989375697	0.368094709561873	0.66527324586365\\
0.52	0.0721751251265572	0.373016087543259	0.663569750098931\\
0.52	0.0741667705335422	0.377932310362199	0.66189758422006\\
0.52	0.0761866387881432	0.382843144964686	0.660253992184443\\
0.52	0.0782347310485954	0.38774835777208	0.658636586490997\\
0.52	0.0803110459884098	0.392647714712651	0.657043304415871\\
0.52	0.0824155797834956	0.397540981253432	0.655472368126508\\
0.52	0.084548326099754	0.402427922432375	0.653922248592746\\
0.52	0.0867092760811502	0.407308302890799	0.652391633146571\\
0.52	0.0888984183382709	0.412181886906127	0.65087939649558\\
0.52	0.0911157389373742	0.417048438424896	0.649384574964843\\
0.52	0.0933612213899392	0.421907721096044	0.647906343724684\\
0.52	0.0956348466427212	0.42675949830445	0.646443996754806\\
0.52	0.0979365930683194	0.431603533204733	0.644996929295963\\
0.52	0.100266436456264	0.436439588755293	0.643564622546847\\
0.52	0.102624350004627	0.441267427752584	0.64214663037443\\
0.52	0.105010304312169	0.446086812865616	0.640742567819433\\
0.52	0.107424267371012	0.450897506670668	0.63935210119373\\
0.52	0.109866204559871	0.455699271686218	0.637974939582443\\
0.52	0.112336078637817	0.460491870408058	0.636610827579862\\
0.52	0.114833849738607	0.465275065344603	0.635259539104382\\
0.52	0.117359475365564	0.470048619052377	0.633920872153155\\
0.52	0.119912910387022	0.474812294171664	0.632594644371965\\
0.52	0.122494107032347	0.479565853462319	0.631280689329641\\
0.52	0.125103014888515	0.484309059839721	0.629978853399058\\
0.52	0.127739580897283	0.489041676410868	0.628690460824553\\
0.525	0	0	0.710202759607373\\
0.525	1.11327767495586e-05	0.00471862581271275	0.713003157837853\\
0.525	4.46175251031912e-05	0.009446325184051	0.715846486660142\\
0.525	0.000100583311362513	0.0141829653360114	0.718788838379794\\
0.525	0.000179158434668431	0.018928411755669	0.721886092228356\\
0.525	0.000280470402701511	0.02368252819604	0.725200636220892\\
0.525	0.000404645907256436	0.0284451766772965	0.728799085274723\\
0.525	0.000551810799695644	0.0332162174883388	0.732749123983336\\
0.525	0.000722090066287311	0.037995509188729	0.737115575253327\\
0.525	0.000915607803432999	0.0427829086109896	0.741955893907011\\
0.525	0.0011324871927904	0.0475782708632729	0.747315373652289\\
0.525	0.00137285047629673	0.0523814493324038	0.753222424045811\\
0.525	0.00163681893109844	0.057192295687301	0.759684310525652\\
0.525	0.00192451284439304	0.0620106598827802	0.766683748102665\\
0.525	0.00223605148818898	0.0668363901637434	0.774176695511256\\
0.525	0.00257155309398959	0.0716693330697584	0.782091614494494\\
0.525	0.00293113482740716	0.0765093334400312	0.79033034642109\\
0.525	0.00331491276271365	0.0813562344187764	0.79877062752665\\
0.525	0.00372300185733414	0.0862098774609879	0.80727012893036\\
0.525	0.00415551592628969	0.0910701023386145	0.815671782700193\\
0.525	0.00461256761659621	0.0959367471471425	0.82381005350733\\
0.525	0.00509426838162598	0.100809648312589	0.831517746503354\\
0.525	0.00560072845543873	0.105688640598912	0.838632911419452\\
0.525	0.00613205682708918	0.11057355711583	0.845005411376293\\
0.525	0.00668836121491816	0.115464229327074	0.850502769049302\\
0.525	0.00726974804083431	0.120360487059049	0.855014975717656\\
0.525	0.00787632240459385	0.125262158509929	0.858458041096605\\
0.525	0.00850818805808555	0.13016907025918	0.860776163494182\\
0.525	0.00916544737962849	0.135081047277508	0.861942500814223\\
0.525	0.00984820134829017	0.139997912937243	0.861958614648422\\
0.525	0.0105565495182326	0.144919489023162	0.86085273561709\\
0.525	0.0112905899930939	0.149845595743738	0.85867705410965\\
0.525	0.012050419400414	0.154776051742839	0.855504274967459\\
0.525	0.0128361328661109	0.159710674111862	0.851423687967754\\
0.525	0.0136478239890177	0.164649278402306	0.84653700048434\\
0.525	0.0144855848154856	0.169591678638796	0.840954157904247\\
0.525	0.0153495058140643	0.174537687332543	0.834789345403933\\
0.525	0.0162396758502651	0.179487115495259	0.828157325806198\\
0.525	0.0171561821614173	0.184439772653511	0.821170226448693\\
0.525	0.0180991103316243	0.189395466863524	0.813934846725692\\
0.525	0.0190685442668299	0.194354004726437	0.806550519904211\\
0.525	0.0200645661700016	0.199315191403999	0.799107529846304\\
0.525	0.0210872565164405	0.204278830634725	0.791686056496195\\
0.525	0.0221366940292259	0.20924472475049	0.7843556038306\\
0.525	0.023212955654804	0.214212674693577	0.777174850281118\\
0.525	0.0243161165387281	0.219182480034174	0.770191853862643\\
0.525	0.025446250001561	0.224153938988321	0.763444541554458\\
0.525	0.0266034275149466	0.229126848436298	0.756961413910482\\
0.525	0.0277877186778607	0.234101003941464	0.750762400416764\\
0.525	0.0289991911930496	0.239076199769547	0.744859807805788\\
0.525	0.0302379108436654	0.244052228908366	0.739259311518081\\
0.525	0.0315039414701067	0.24902888308801	0.733960949044501\\
0.525	0.0327973449470747	0.254005952801449	0.728960082407259\\
0.525	0.0341181811608519	0.258983227325591	0.724248305111178\\
0.525	0.0354665079868145	0.263960494742775	0.719814276220682\\
0.525	0.0368423812671862	0.268937541962694	0.715644470612631\\
0.525	0.0382458547890421	0.273914154744766	0.711723839838043\\
0.525	0.0396769802625737	0.278890117720924	0.70803638238973\\
0.525	0.0411358072996216	0.283865214418836	0.704565625567163\\
0.525	0.0426223833924861	0.288839227285554	0.701295023640317\\
0.525	0.0441367538930258	0.293811937711588	0.698208278749354\\
0.525	0.0456789619920503	0.298783126055387	0.69528959205535\\
0.525	0.0472490486990192	0.303752571668251	0.69252385319751\\
0.525	0.0488470528220538	0.308720052919643	0.689896776228117\\
0.525	0.0504730109482718	0.313685347222913	0.687394989990282\\
0.525	0.0521269574244539	0.318648231061428	0.685006090465982\\
0.525	0.0538089243380495	0.323608480015097	0.682718662029128\\
0.525	0.0555189414985328	0.328565868787292	0.680522273853468\\
0.525	0.0572570364191156	0.333520171232163	0.678407456997851\\
0.525	0.0590232342988274	0.338471160382329	0.676365666960044\\
0.525	0.0608175580049697	0.34341860847696	0.674389235783156\\
0.525	0.0626400280559541	0.348362286990222	0.672471317134862\\
0.525	0.064490662604533	0.3533019666601	0.670605827172181\\
0.525	0.0663694774214291	0.358237417517579	0.668787383459803\\
0.525	0.0682764858793752	0.363168408916186	0.667011243730729\\
0.525	0.0702116989375697	0.368094709561873	0.66527324586365\\
0.525	0.0721751251265572	0.373016087543259	0.663569750098932\\
0.525	0.0741667705335423	0.377932310362199	0.661897584220059\\
0.525	0.0761866387881432	0.382843144964686	0.660253992184442\\
0.525	0.0782347310485954	0.38774835777208	0.658636586490997\\
0.525	0.0803110459884098	0.392647714712651	0.657043304415871\\
0.525	0.0824155797834956	0.397540981253432	0.655472368126508\\
0.525	0.084548326099754	0.402427922432375	0.653922248592745\\
0.525	0.0867092760811502	0.4073083028908	0.652391633146569\\
0.525	0.0888984183382709	0.412181886906127	0.650879396495579\\
0.525	0.0911157389373742	0.417048438424896	0.649384574964845\\
0.525	0.0933612213899392	0.421907721096044	0.647906343724685\\
0.525	0.0956348466427212	0.42675949830445	0.646443996754804\\
0.525	0.0979365930683194	0.431603533204733	0.644996929295962\\
0.525	0.100266436456264	0.436439588755293	0.643564622546848\\
0.525	0.102624350004627	0.441267427752584	0.64214663037443\\
0.525	0.105010304312169	0.446086812865616	0.640742567819433\\
0.525	0.107424267371012	0.450897506670668	0.639352101193731\\
0.525	0.109866204559871	0.455699271686218	0.637974939582442\\
0.525	0.112336078637817	0.460491870408058	0.636610827579862\\
0.525	0.114833849738607	0.465275065344603	0.635259539104382\\
0.525	0.117359475365564	0.470048619052377	0.633920872153155\\
0.525	0.119912910387022	0.474812294171664	0.632594644371966\\
0.525	0.122494107032347	0.479565853462319	0.63128068932964\\
0.525	0.125103014888515	0.484309059839721	0.629978853399059\\
0.525	0.127739580897283	0.489041676410868	0.628690460824559\\
0.53	0	0	0.710202759607373\\
0.53	1.11327767495586e-05	0.00471862581271274	0.713003157837853\\
0.53	4.46175251031912e-05	0.009446325184051	0.715846486660142\\
0.53	0.000100583311362513	0.0141829653360114	0.718788838379794\\
0.53	0.000179158434668431	0.018928411755669	0.721886092228356\\
0.53	0.000280470402701511	0.02368252819604	0.725200636220892\\
0.53	0.000404645907256436	0.0284451766772965	0.728799085274723\\
0.53	0.000551810799695644	0.0332162174883388	0.732749123983336\\
0.53	0.000722090066287311	0.037995509188729	0.737115575253328\\
0.53	0.000915607803432999	0.0427829086109896	0.741955893907011\\
0.53	0.0011324871927904	0.0475782708632729	0.747315373652289\\
0.53	0.00137285047629673	0.0523814493324038	0.753222424045811\\
0.53	0.00163681893109844	0.057192295687301	0.759684310525652\\
0.53	0.00192451284439304	0.0620106598827802	0.766683748102665\\
0.53	0.00223605148818898	0.0668363901637434	0.774176695511256\\
0.53	0.00257155309398959	0.0716693330697584	0.782091614494493\\
0.53	0.00293113482740716	0.0765093334400312	0.79033034642109\\
0.53	0.00331491276271365	0.0813562344187764	0.79877062752665\\
0.53	0.00372300185733413	0.0862098774609879	0.807270128930359\\
0.53	0.00415551592628969	0.0910701023386145	0.815671782700194\\
0.53	0.00461256761659621	0.0959367471471425	0.82381005350733\\
0.53	0.00509426838162598	0.100809648312589	0.831517746503354\\
0.53	0.00560072845543873	0.105688640598912	0.838632911419452\\
0.53	0.00613205682708918	0.11057355711583	0.845005411376292\\
0.53	0.00668836121491815	0.115464229327074	0.850502769049303\\
0.53	0.00726974804083431	0.120360487059049	0.855014975717658\\
0.53	0.00787632240459385	0.125262158509929	0.858458041096606\\
0.53	0.00850818805808555	0.13016907025918	0.860776163494181\\
0.53	0.00916544737962849	0.135081047277508	0.861942500814225\\
0.53	0.00984820134829017	0.139997912937243	0.861958614648418\\
0.53	0.0105565495182326	0.144919489023162	0.86085273561709\\
0.53	0.0112905899930939	0.149845595743738	0.85867705410965\\
0.53	0.012050419400414	0.154776051742839	0.855504274967458\\
0.53	0.0128361328661109	0.159710674111862	0.851423687967754\\
0.53	0.0136478239890177	0.164649278402306	0.846537000484342\\
0.53	0.0144855848154856	0.169591678638796	0.840954157904249\\
0.53	0.0153495058140643	0.174537687332543	0.834789345403934\\
0.53	0.0162396758502651	0.179487115495259	0.828157325806201\\
0.53	0.0171561821614173	0.184439772653511	0.821170226448696\\
0.53	0.0180991103316243	0.189395466863524	0.813934846725694\\
0.53	0.0190685442668299	0.194354004726437	0.806550519904213\\
0.53	0.0200645661700016	0.199315191403999	0.799107529846304\\
0.53	0.0210872565164405	0.204278830634725	0.791686056496196\\
0.53	0.0221366940292259	0.20924472475049	0.784355603830602\\
0.53	0.023212955654804	0.214212674693577	0.77717485028112\\
0.53	0.0243161165387281	0.219182480034174	0.770191853862643\\
0.53	0.025446250001561	0.224153938988321	0.763444541554458\\
0.53	0.0266034275149466	0.229126848436298	0.756961413910482\\
0.53	0.0277877186778607	0.234101003941464	0.750762400416766\\
0.53	0.0289991911930496	0.239076199769547	0.744859807805786\\
0.53	0.0302379108436654	0.244052228908366	0.739259311518078\\
0.53	0.0315039414701067	0.24902888308801	0.733960949044502\\
0.53	0.0327973449470747	0.254005952801449	0.72896008240726\\
0.53	0.0341181811608519	0.258983227325591	0.724248305111177\\
0.53	0.0354665079868145	0.263960494742775	0.719814276220683\\
0.53	0.0368423812671862	0.268937541962694	0.71564447061263\\
0.53	0.0382458547890421	0.273914154744766	0.711723839838042\\
0.53	0.0396769802625738	0.278890117720924	0.70803638238973\\
0.53	0.0411358072996216	0.283865214418836	0.704565625567165\\
0.53	0.0426223833924861	0.288839227285554	0.701295023640315\\
0.53	0.0441367538930258	0.293811937711588	0.698208278749353\\
0.53	0.0456789619920503	0.298783126055387	0.695289592055351\\
0.53	0.0472490486990192	0.303752571668251	0.69252385319751\\
0.53	0.0488470528220538	0.308720052919643	0.689896776228118\\
0.53	0.0504730109482718	0.313685347222913	0.687394989990282\\
0.53	0.0521269574244539	0.318648231061428	0.68500609046598\\
0.53	0.0538089243380495	0.323608480015096	0.682718662029128\\
0.53	0.0555189414985328	0.328565868787292	0.680522273853469\\
0.53	0.0572570364191156	0.333520171232163	0.678407456997849\\
0.53	0.0590232342988274	0.338471160382329	0.676365666960044\\
0.53	0.0608175580049697	0.34341860847696	0.674389235783156\\
0.53	0.0626400280559541	0.348362286990222	0.672471317134862\\
0.53	0.064490662604533	0.3533019666601	0.670605827172182\\
0.53	0.0663694774214291	0.358237417517579	0.668787383459803\\
0.53	0.0682764858793752	0.363168408916186	0.667011243730728\\
0.53	0.0702116989375697	0.368094709561873	0.66527324586365\\
0.53	0.0721751251265572	0.373016087543259	0.663569750098933\\
0.53	0.0741667705335423	0.377932310362199	0.66189758422006\\
0.53	0.0761866387881432	0.382843144964686	0.660253992184442\\
0.53	0.0782347310485954	0.38774835777208	0.658636586490997\\
0.53	0.0803110459884098	0.392647714712651	0.657043304415872\\
0.53	0.0824155797834956	0.397540981253432	0.655472368126509\\
0.53	0.084548326099754	0.402427922432375	0.653922248592744\\
0.53	0.0867092760811502	0.407308302890799	0.652391633146571\\
0.53	0.0888984183382709	0.412181886906127	0.650879396495579\\
0.53	0.0911157389373742	0.417048438424897	0.649384574964844\\
0.53	0.0933612213899392	0.421907721096044	0.647906343724685\\
0.53	0.0956348466427212	0.42675949830445	0.646443996754805\\
0.53	0.0979365930683194	0.431603533204733	0.644996929295961\\
0.53	0.100266436456264	0.436439588755293	0.643564622546847\\
0.53	0.102624350004627	0.441267427752584	0.64214663037443\\
0.53	0.105010304312169	0.446086812865616	0.640742567819433\\
0.53	0.107424267371012	0.450897506670668	0.639352101193731\\
0.53	0.109866204559871	0.455699271686218	0.637974939582444\\
0.53	0.112336078637817	0.460491870408058	0.636610827579861\\
0.53	0.114833849738607	0.465275065344603	0.635259539104382\\
0.53	0.117359475365564	0.470048619052377	0.633920872153154\\
0.53	0.119912910387022	0.474812294171664	0.632594644371966\\
0.53	0.122494107032347	0.479565853462319	0.63128068932964\\
0.53	0.125103014888515	0.484309059839721	0.629978853399057\\
0.53	0.127739580897283	0.489041676410868	0.628690460824558\\
0.535	0	0	0.710202759607373\\
0.535	1.11327767495586e-05	0.00471862581271275	0.713003157837853\\
0.535	4.46175251031912e-05	0.009446325184051	0.715846486660142\\
0.535	0.000100583311362513	0.0141829653360114	0.718788838379794\\
0.535	0.000179158434668431	0.018928411755669	0.721886092228356\\
0.535	0.000280470402701511	0.02368252819604	0.725200636220892\\
0.535	0.000404645907256436	0.0284451766772965	0.728799085274723\\
0.535	0.000551810799695644	0.0332162174883388	0.732749123983336\\
0.535	0.000722090066287311	0.037995509188729	0.737115575253328\\
0.535	0.000915607803432999	0.0427829086109896	0.741955893907011\\
0.535	0.0011324871927904	0.0475782708632729	0.747315373652289\\
0.535	0.00137285047629673	0.0523814493324038	0.753222424045811\\
0.535	0.00163681893109844	0.057192295687301	0.759684310525652\\
0.535	0.00192451284439304	0.0620106598827802	0.766683748102665\\
0.535	0.00223605148818899	0.0668363901637434	0.774176695511257\\
0.535	0.00257155309398959	0.0716693330697584	0.782091614494493\\
0.535	0.00293113482740716	0.0765093334400312	0.79033034642109\\
0.535	0.00331491276271365	0.0813562344187764	0.79877062752665\\
0.535	0.00372300185733414	0.0862098774609879	0.807270128930359\\
0.535	0.00415551592628969	0.0910701023386145	0.815671782700193\\
0.535	0.00461256761659621	0.0959367471471425	0.823810053507331\\
0.535	0.00509426838162598	0.100809648312589	0.831517746503354\\
0.535	0.00560072845543873	0.105688640598912	0.838632911419452\\
0.535	0.00613205682708918	0.11057355711583	0.845005411376292\\
0.535	0.00668836121491816	0.115464229327074	0.850502769049302\\
0.535	0.00726974804083431	0.120360487059049	0.855014975717658\\
0.535	0.00787632240459385	0.125262158509929	0.858458041096606\\
0.535	0.00850818805808555	0.13016907025918	0.860776163494182\\
0.535	0.00916544737962849	0.135081047277508	0.861942500814227\\
0.535	0.00984820134829017	0.139997912937243	0.861958614648419\\
0.535	0.0105565495182326	0.144919489023162	0.86085273561709\\
0.535	0.0112905899930939	0.149845595743738	0.85867705410965\\
0.535	0.012050419400414	0.154776051742839	0.855504274967458\\
0.535	0.0128361328661109	0.159710674111862	0.851423687967754\\
0.535	0.0136478239890177	0.164649278402306	0.84653700048434\\
0.535	0.0144855848154856	0.169591678638796	0.840954157904248\\
0.535	0.0153495058140643	0.174537687332543	0.834789345403935\\
0.535	0.0162396758502651	0.179487115495259	0.8281573258062\\
0.535	0.0171561821614173	0.184439772653511	0.821170226448695\\
0.535	0.0180991103316243	0.189395466863524	0.813934846725692\\
0.535	0.0190685442668299	0.194354004726437	0.806550519904213\\
0.535	0.0200645661700016	0.199315191403999	0.799107529846306\\
0.535	0.0210872565164405	0.204278830634725	0.791686056496195\\
0.535	0.0221366940292259	0.20924472475049	0.784355603830601\\
0.535	0.023212955654804	0.214212674693577	0.777174850281118\\
0.535	0.0243161165387281	0.219182480034174	0.770191853862644\\
0.535	0.025446250001561	0.224153938988321	0.763444541554458\\
0.535	0.0266034275149466	0.229126848436298	0.756961413910483\\
0.535	0.0277877186778607	0.234101003941464	0.750762400416764\\
0.535	0.0289991911930496	0.239076199769547	0.744859807805785\\
0.535	0.0302379108436654	0.244052228908366	0.73925931151808\\
0.535	0.0315039414701067	0.24902888308801	0.733960949044503\\
0.535	0.0327973449470747	0.254005952801449	0.728960082407259\\
0.535	0.0341181811608519	0.258983227325591	0.724248305111178\\
0.535	0.0354665079868145	0.263960494742775	0.719814276220682\\
0.535	0.0368423812671862	0.268937541962694	0.71564447061263\\
0.535	0.0382458547890421	0.273914154744766	0.711723839838043\\
0.535	0.0396769802625738	0.278890117720924	0.708036382389731\\
0.535	0.0411358072996216	0.283865214418836	0.704565625567163\\
0.535	0.0426223833924861	0.288839227285554	0.701295023640315\\
0.535	0.0441367538930258	0.293811937711588	0.698208278749354\\
0.535	0.0456789619920503	0.298783126055387	0.695289592055349\\
0.535	0.0472490486990192	0.303752571668251	0.692523853197511\\
0.535	0.0488470528220538	0.308720052919643	0.689896776228119\\
0.535	0.0504730109482718	0.313685347222913	0.687394989990281\\
0.535	0.0521269574244539	0.318648231061428	0.685006090465979\\
0.535	0.0538089243380495	0.323608480015096	0.682718662029128\\
0.535	0.0555189414985328	0.328565868787292	0.68052227385347\\
0.535	0.0572570364191156	0.333520171232163	0.67840745699785\\
0.535	0.0590232342988274	0.338471160382329	0.676365666960044\\
0.535	0.0608175580049697	0.34341860847696	0.674389235783156\\
0.535	0.0626400280559541	0.348362286990222	0.672471317134861\\
0.535	0.064490662604533	0.3533019666601	0.670605827172182\\
0.535	0.066369477421429	0.358237417517579	0.668787383459804\\
0.535	0.0682764858793752	0.363168408916186	0.667011243730728\\
0.535	0.0702116989375697	0.368094709561873	0.665273245863649\\
0.535	0.0721751251265572	0.373016087543259	0.663569750098932\\
0.535	0.0741667705335423	0.377932310362199	0.66189758422006\\
0.535	0.0761866387881432	0.382843144964686	0.660253992184442\\
0.535	0.0782347310485954	0.38774835777208	0.658636586490998\\
0.535	0.0803110459884098	0.392647714712651	0.657043304415871\\
0.535	0.0824155797834956	0.397540981253432	0.65547236812651\\
0.535	0.084548326099754	0.402427922432375	0.653922248592745\\
0.535	0.0867092760811502	0.407308302890799	0.65239163314657\\
0.535	0.0888984183382709	0.412181886906127	0.650879396495579\\
0.535	0.0911157389373742	0.417048438424896	0.649384574964845\\
0.535	0.0933612213899392	0.421907721096044	0.647906343724685\\
0.535	0.0956348466427212	0.42675949830445	0.646443996754804\\
0.535	0.0979365930683194	0.431603533204733	0.644996929295961\\
0.535	0.100266436456264	0.436439588755293	0.643564622546847\\
0.535	0.102624350004627	0.441267427752584	0.64214663037443\\
0.535	0.105010304312169	0.446086812865616	0.640742567819433\\
0.535	0.107424267371012	0.450897506670668	0.639352101193729\\
0.535	0.109866204559871	0.455699271686218	0.637974939582444\\
0.535	0.112336078637817	0.460491870408058	0.636610827579863\\
0.535	0.114833849738607	0.465275065344603	0.635259539104383\\
0.535	0.117359475365564	0.470048619052377	0.633920872153155\\
0.535	0.119912910387022	0.474812294171664	0.632594644371965\\
0.535	0.122494107032347	0.479565853462319	0.631280689329641\\
0.535	0.125103014888515	0.484309059839721	0.629978853399058\\
0.535	0.127739580897283	0.489041676410868	0.628690460824557\\
0.54	0	0	0.710202759607373\\
0.54	1.11327767495586e-05	0.00471862581271275	0.713003157837853\\
0.54	4.46175251031912e-05	0.009446325184051	0.715846486660142\\
0.54	0.000100583311362513	0.0141829653360114	0.718788838379794\\
0.54	0.000179158434668431	0.018928411755669	0.721886092228356\\
0.54	0.000280470402701511	0.02368252819604	0.725200636220892\\
0.54	0.000404645907256436	0.0284451766772965	0.728799085274723\\
0.54	0.000551810799695644	0.0332162174883388	0.732749123983336\\
0.54	0.000722090066287311	0.037995509188729	0.737115575253327\\
0.54	0.000915607803432999	0.0427829086109896	0.741955893907011\\
0.54	0.0011324871927904	0.0475782708632729	0.747315373652289\\
0.54	0.00137285047629673	0.0523814493324038	0.753222424045811\\
0.54	0.00163681893109844	0.057192295687301	0.759684310525652\\
0.54	0.00192451284439304	0.0620106598827802	0.766683748102665\\
0.54	0.00223605148818898	0.0668363901637434	0.774176695511256\\
0.54	0.00257155309398959	0.0716693330697584	0.782091614494494\\
0.54	0.00293113482740716	0.0765093334400312	0.79033034642109\\
0.54	0.00331491276271365	0.0813562344187764	0.79877062752665\\
0.54	0.00372300185733413	0.0862098774609879	0.807270128930359\\
0.54	0.00415551592628969	0.0910701023386145	0.815671782700193\\
0.54	0.00461256761659621	0.0959367471471425	0.82381005350733\\
0.54	0.00509426838162598	0.100809648312589	0.831517746503355\\
0.54	0.00560072845543873	0.105688640598912	0.838632911419452\\
0.54	0.00613205682708918	0.11057355711583	0.845005411376294\\
0.54	0.00668836121491815	0.115464229327074	0.850502769049301\\
0.54	0.00726974804083431	0.120360487059049	0.855014975717656\\
0.54	0.00787632240459385	0.125262158509929	0.858458041096605\\
0.54	0.00850818805808555	0.13016907025918	0.860776163494183\\
0.54	0.00916544737962849	0.135081047277508	0.861942500814223\\
0.54	0.00984820134829017	0.139997912937243	0.86195861464842\\
0.54	0.0105565495182326	0.144919489023162	0.860852735617092\\
0.54	0.0112905899930939	0.149845595743738	0.858677054109649\\
0.54	0.012050419400414	0.154776051742839	0.855504274967457\\
0.54	0.0128361328661109	0.159710674111862	0.851423687967753\\
0.54	0.0136478239890177	0.164649278402306	0.846537000484341\\
0.54	0.0144855848154856	0.169591678638796	0.840954157904249\\
0.54	0.0153495058140643	0.174537687332543	0.834789345403934\\
0.54	0.0162396758502651	0.179487115495259	0.828157325806204\\
0.54	0.0171561821614173	0.184439772653511	0.821170226448693\\
0.54	0.0180991103316243	0.189395466863524	0.813934846725693\\
0.54	0.0190685442668299	0.194354004726437	0.806550519904211\\
0.54	0.0200645661700016	0.199315191403999	0.799107529846303\\
0.54	0.0210872565164405	0.204278830634725	0.791686056496192\\
0.54	0.0221366940292259	0.20924472475049	0.784355603830602\\
0.54	0.023212955654804	0.214212674693577	0.777174850281119\\
0.54	0.0243161165387281	0.219182480034174	0.770191853862643\\
0.54	0.025446250001561	0.224153938988321	0.763444541554457\\
0.54	0.0266034275149466	0.229126848436298	0.756961413910483\\
0.54	0.0277877186778607	0.234101003941464	0.750762400416766\\
0.54	0.0289991911930496	0.239076199769547	0.744859807805786\\
0.54	0.0302379108436654	0.244052228908366	0.739259311518077\\
0.54	0.0315039414701067	0.24902888308801	0.733960949044503\\
0.54	0.0327973449470747	0.254005952801449	0.72896008240726\\
0.54	0.0341181811608519	0.258983227325591	0.724248305111178\\
0.54	0.0354665079868145	0.263960494742775	0.719814276220681\\
0.54	0.0368423812671862	0.268937541962694	0.71564447061263\\
0.54	0.0382458547890421	0.273914154744766	0.711723839838042\\
0.54	0.0396769802625738	0.278890117720924	0.708036382389731\\
0.54	0.0411358072996216	0.283865214418836	0.704565625567164\\
0.54	0.0426223833924861	0.288839227285554	0.701295023640315\\
0.54	0.0441367538930258	0.293811937711588	0.698208278749354\\
0.54	0.0456789619920503	0.298783126055387	0.695289592055349\\
0.54	0.0472490486990192	0.303752571668251	0.692523853197512\\
0.54	0.0488470528220538	0.308720052919643	0.68989677622812\\
0.54	0.0504730109482718	0.313685347222913	0.687394989990281\\
0.54	0.0521269574244539	0.318648231061428	0.68500609046598\\
0.54	0.0538089243380495	0.323608480015096	0.682718662029128\\
0.54	0.0555189414985328	0.328565868787292	0.68052227385347\\
0.54	0.0572570364191156	0.333520171232163	0.67840745699785\\
0.54	0.0590232342988274	0.338471160382329	0.676365666960044\\
0.54	0.0608175580049697	0.34341860847696	0.674389235783157\\
0.54	0.0626400280559541	0.348362286990222	0.672471317134861\\
0.54	0.064490662604533	0.3533019666601	0.670605827172182\\
0.54	0.0663694774214291	0.358237417517579	0.668787383459805\\
0.54	0.0682764858793752	0.363168408916186	0.667011243730728\\
0.54	0.0702116989375697	0.368094709561873	0.665273245863649\\
0.54	0.0721751251265572	0.373016087543259	0.663569750098932\\
0.54	0.0741667705335423	0.377932310362199	0.661897584220061\\
0.54	0.0761866387881432	0.382843144964686	0.660253992184441\\
0.54	0.0782347310485954	0.38774835777208	0.658636586490997\\
0.54	0.0803110459884098	0.392647714712651	0.657043304415873\\
0.54	0.0824155797834956	0.397540981253432	0.655472368126509\\
0.54	0.084548326099754	0.402427922432375	0.653922248592744\\
0.54	0.0867092760811502	0.407308302890799	0.652391633146571\\
0.54	0.0888984183382709	0.412181886906127	0.65087939649558\\
0.54	0.0911157389373742	0.417048438424896	0.649384574964842\\
0.54	0.0933612213899392	0.421907721096044	0.647906343724686\\
0.54	0.0956348466427212	0.42675949830445	0.646443996754806\\
0.54	0.0979365930683194	0.431603533204733	0.64499692929596\\
0.54	0.100266436456264	0.436439588755293	0.643564622546846\\
0.54	0.102624350004627	0.441267427752584	0.64214663037443\\
0.54	0.105010304312169	0.446086812865616	0.640742567819433\\
0.54	0.107424267371012	0.450897506670668	0.639352101193731\\
0.54	0.109866204559871	0.455699271686218	0.637974939582443\\
0.54	0.112336078637817	0.460491870408058	0.636610827579862\\
0.54	0.114833849738607	0.465275065344603	0.635259539104382\\
0.54	0.117359475365564	0.470048619052377	0.633920872153155\\
0.54	0.119912910387022	0.474812294171664	0.632594644371965\\
0.54	0.122494107032347	0.479565853462319	0.631280689329642\\
0.54	0.125103014888515	0.484309059839721	0.629978853399061\\
0.54	0.127739580897283	0.489041676410868	0.628690460824557\\
0.545	0	0	0.710202759607373\\
0.545	1.11327767495586e-05	0.00471862581271275	0.713003157837853\\
0.545	4.46175251031912e-05	0.009446325184051	0.715846486660142\\
0.545	0.000100583311362513	0.0141829653360114	0.718788838379794\\
0.545	0.000179158434668431	0.018928411755669	0.721886092228356\\
0.545	0.000280470402701511	0.02368252819604	0.725200636220892\\
0.545	0.000404645907256436	0.0284451766772965	0.728799085274723\\
0.545	0.000551810799695644	0.0332162174883389	0.732749123983336\\
0.545	0.000722090066287311	0.037995509188729	0.737115575253328\\
0.545	0.000915607803432999	0.0427829086109896	0.741955893907011\\
0.545	0.0011324871927904	0.0475782708632729	0.747315373652289\\
0.545	0.00137285047629673	0.0523814493324038	0.75322242404581\\
0.545	0.00163681893109844	0.057192295687301	0.759684310525652\\
0.545	0.00192451284439304	0.0620106598827802	0.766683748102665\\
0.545	0.00223605148818898	0.0668363901637434	0.774176695511257\\
0.545	0.00257155309398959	0.0716693330697584	0.782091614494494\\
0.545	0.00293113482740716	0.0765093334400312	0.79033034642109\\
0.545	0.00331491276271365	0.0813562344187764	0.798770627526651\\
0.545	0.00372300185733413	0.0862098774609879	0.807270128930359\\
0.545	0.00415551592628969	0.0910701023386145	0.815671782700193\\
0.545	0.00461256761659621	0.0959367471471425	0.823810053507331\\
0.545	0.00509426838162598	0.100809648312589	0.831517746503353\\
0.545	0.00560072845543873	0.105688640598912	0.838632911419452\\
0.545	0.00613205682708918	0.11057355711583	0.845005411376293\\
0.545	0.00668836121491815	0.115464229327074	0.850502769049302\\
0.545	0.00726974804083431	0.120360487059049	0.855014975717656\\
0.545	0.00787632240459385	0.125262158509929	0.858458041096606\\
0.545	0.00850818805808555	0.13016907025918	0.860776163494181\\
0.545	0.00916544737962849	0.135081047277508	0.861942500814224\\
0.545	0.00984820134829017	0.139997912937243	0.861958614648422\\
0.545	0.0105565495182326	0.144919489023162	0.860852735617091\\
0.545	0.0112905899930939	0.149845595743738	0.858677054109651\\
0.545	0.012050419400414	0.154776051742839	0.855504274967457\\
0.545	0.0128361328661109	0.159710674111862	0.851423687967756\\
0.545	0.0136478239890177	0.164649278402306	0.846537000484343\\
0.545	0.0144855848154856	0.169591678638796	0.840954157904249\\
0.545	0.0153495058140643	0.174537687332543	0.834789345403937\\
0.545	0.0162396758502651	0.179487115495259	0.828157325806203\\
0.545	0.0171561821614173	0.184439772653511	0.821170226448691\\
0.545	0.0180991103316243	0.189395466863524	0.813934846725692\\
0.545	0.0190685442668299	0.194354004726437	0.806550519904211\\
0.545	0.0200645661700016	0.199315191403999	0.799107529846304\\
0.545	0.0210872565164405	0.204278830634725	0.791686056496194\\
0.545	0.0221366940292259	0.20924472475049	0.784355603830604\\
0.545	0.023212955654804	0.214212674693577	0.777174850281118\\
0.545	0.0243161165387281	0.219182480034174	0.770191853862641\\
0.545	0.025446250001561	0.224153938988321	0.763444541554459\\
0.545	0.0266034275149466	0.229126848436298	0.756961413910484\\
0.545	0.0277877186778607	0.234101003941464	0.750762400416766\\
0.545	0.0289991911930496	0.239076199769547	0.744859807805783\\
0.545	0.0302379108436654	0.244052228908366	0.739259311518077\\
0.545	0.0315039414701067	0.24902888308801	0.733960949044503\\
0.545	0.0327973449470747	0.254005952801449	0.728960082407263\\
0.545	0.0341181811608519	0.258983227325592	0.724248305111178\\
0.545	0.0354665079868145	0.263960494742775	0.71981427622068\\
0.545	0.0368423812671862	0.268937541962694	0.715644470612631\\
0.545	0.0382458547890421	0.273914154744766	0.711723839838043\\
0.545	0.0396769802625738	0.278890117720924	0.708036382389731\\
0.545	0.0411358072996216	0.283865214418836	0.704565625567163\\
0.545	0.0426223833924861	0.288839227285554	0.701295023640316\\
0.545	0.0441367538930258	0.293811937711588	0.698208278749354\\
0.545	0.0456789619920503	0.298783126055387	0.695289592055348\\
0.545	0.0472490486990192	0.303752571668251	0.692523853197512\\
0.545	0.0488470528220538	0.308720052919643	0.689896776228121\\
0.545	0.0504730109482718	0.313685347222913	0.687394989990281\\
0.545	0.0521269574244539	0.318648231061428	0.685006090465979\\
0.545	0.0538089243380495	0.323608480015096	0.682718662029128\\
0.545	0.0555189414985328	0.328565868787292	0.68052227385347\\
0.545	0.0572570364191156	0.333520171232163	0.67840745699785\\
0.545	0.0590232342988274	0.338471160382329	0.676365666960043\\
0.545	0.0608175580049697	0.34341860847696	0.674389235783156\\
0.545	0.0626400280559541	0.348362286990222	0.672471317134862\\
0.545	0.064490662604533	0.3533019666601	0.670605827172182\\
0.545	0.0663694774214291	0.358237417517579	0.668787383459805\\
0.545	0.0682764858793752	0.363168408916186	0.667011243730728\\
0.545	0.0702116989375697	0.368094709561873	0.665273245863649\\
0.545	0.0721751251265572	0.373016087543259	0.663569750098933\\
0.545	0.0741667705335422	0.377932310362199	0.66189758422006\\
0.545	0.0761866387881432	0.382843144964686	0.660253992184442\\
0.545	0.0782347310485954	0.38774835777208	0.658636586490997\\
0.545	0.0803110459884098	0.392647714712651	0.657043304415872\\
0.545	0.0824155797834956	0.397540981253432	0.65547236812651\\
0.545	0.084548326099754	0.402427922432375	0.653922248592745\\
0.545	0.0867092760811502	0.407308302890799	0.652391633146569\\
0.545	0.0888984183382709	0.412181886906127	0.65087939649558\\
0.545	0.0911157389373742	0.417048438424896	0.649384574964844\\
0.545	0.0933612213899392	0.421907721096044	0.647906343724685\\
0.545	0.0956348466427212	0.42675949830445	0.646443996754807\\
0.545	0.0979365930683194	0.431603533204733	0.644996929295961\\
0.545	0.100266436456264	0.436439588755293	0.643564622546846\\
0.545	0.102624350004627	0.441267427752584	0.64214663037443\\
0.545	0.105010304312169	0.446086812865616	0.640742567819433\\
0.545	0.107424267371012	0.450897506670668	0.63935210119373\\
0.545	0.109866204559871	0.455699271686218	0.637974939582443\\
0.545	0.112336078637817	0.460491870408058	0.636610827579862\\
0.545	0.114833849738607	0.465275065344603	0.635259539104384\\
0.545	0.117359475365564	0.470048619052377	0.633920872153155\\
0.545	0.119912910387022	0.474812294171664	0.632594644371964\\
0.545	0.122494107032347	0.479565853462319	0.63128068932964\\
0.545	0.125103014888515	0.484309059839721	0.629978853399061\\
0.545	0.127739580897283	0.489041676410868	0.628690460824564\\
0.55	0	0	0.710202759607373\\
0.55	1.11327767495586e-05	0.00471862581271274	0.713003157837853\\
0.55	4.46175251031912e-05	0.009446325184051	0.715846486660142\\
0.55	0.000100583311362513	0.0141829653360114	0.718788838379794\\
0.55	0.000179158434668431	0.018928411755669	0.721886092228356\\
0.55	0.000280470402701511	0.02368252819604	0.725200636220892\\
0.55	0.000404645907256436	0.0284451766772965	0.728799085274723\\
0.55	0.000551810799695644	0.0332162174883388	0.732749123983336\\
0.55	0.000722090066287311	0.037995509188729	0.737115575253328\\
0.55	0.000915607803432999	0.0427829086109896	0.741955893907011\\
0.55	0.0011324871927904	0.0475782708632729	0.747315373652289\\
0.55	0.00137285047629673	0.0523814493324038	0.75322242404581\\
0.55	0.00163681893109844	0.057192295687301	0.759684310525652\\
0.55	0.00192451284439304	0.0620106598827802	0.766683748102665\\
0.55	0.00223605148818899	0.0668363901637434	0.774176695511256\\
0.55	0.00257155309398959	0.0716693330697584	0.782091614494493\\
0.55	0.00293113482740716	0.0765093334400313	0.79033034642109\\
0.55	0.00331491276271365	0.0813562344187764	0.798770627526651\\
0.55	0.00372300185733413	0.0862098774609879	0.807270128930359\\
0.55	0.00415551592628969	0.0910701023386145	0.815671782700193\\
0.55	0.00461256761659621	0.0959367471471425	0.82381005350733\\
0.55	0.00509426838162598	0.100809648312589	0.831517746503354\\
0.55	0.00560072845543873	0.105688640598912	0.838632911419451\\
0.55	0.00613205682708918	0.11057355711583	0.845005411376292\\
0.55	0.00668836121491816	0.115464229327074	0.850502769049302\\
0.55	0.00726974804083431	0.120360487059049	0.855014975717658\\
0.55	0.00787632240459385	0.125262158509929	0.858458041096606\\
0.55	0.00850818805808555	0.13016907025918	0.860776163494181\\
0.55	0.00916544737962849	0.135081047277508	0.861942500814225\\
0.55	0.00984820134829017	0.139997912937243	0.86195861464842\\
0.55	0.0105565495182326	0.144919489023162	0.860852735617091\\
0.55	0.0112905899930939	0.149845595743738	0.858677054109652\\
0.55	0.012050419400414	0.154776051742839	0.855504274967456\\
0.55	0.0128361328661109	0.159710674111862	0.851423687967757\\
0.55	0.0136478239890177	0.164649278402306	0.846537000484341\\
0.55	0.0144855848154856	0.169591678638796	0.840954157904247\\
0.55	0.0153495058140643	0.174537687332543	0.834789345403936\\
0.55	0.0162396758502651	0.179487115495259	0.828157325806196\\
0.55	0.0171561821614173	0.184439772653511	0.821170226448693\\
0.55	0.0180991103316243	0.189395466863524	0.813934846725693\\
0.55	0.0190685442668299	0.194354004726437	0.806550519904212\\
0.55	0.0200645661700016	0.199315191403999	0.799107529846304\\
0.55	0.0210872565164405	0.204278830634725	0.791686056496197\\
0.55	0.0221366940292259	0.20924472475049	0.784355603830603\\
0.55	0.023212955654804	0.214212674693577	0.777174850281116\\
0.55	0.0243161165387281	0.219182480034174	0.770191853862644\\
0.55	0.025446250001561	0.224153938988321	0.76344454155446\\
0.55	0.0266034275149466	0.229126848436298	0.756961413910482\\
0.55	0.0277877186778607	0.234101003941464	0.750762400416764\\
0.55	0.0289991911930496	0.239076199769547	0.744859807805783\\
0.55	0.0302379108436654	0.244052228908366	0.739259311518079\\
0.55	0.0315039414701067	0.24902888308801	0.733960949044505\\
0.55	0.0327973449470747	0.254005952801449	0.728960082407261\\
0.55	0.0341181811608519	0.258983227325591	0.724248305111176\\
0.55	0.0354665079868145	0.263960494742775	0.719814276220683\\
0.55	0.0368423812671862	0.268937541962694	0.715644470612631\\
0.55	0.0382458547890421	0.273914154744766	0.711723839838041\\
0.55	0.0396769802625738	0.278890117720924	0.70803638238973\\
0.55	0.0411358072996216	0.283865214418836	0.704565625567166\\
0.55	0.0426223833924861	0.288839227285554	0.701295023640316\\
0.55	0.0441367538930258	0.293811937711588	0.698208278749352\\
0.55	0.0456789619920503	0.298783126055387	0.695289592055351\\
0.55	0.0472490486990192	0.303752571668251	0.692523853197511\\
0.55	0.0488470528220538	0.308720052919643	0.689896776228118\\
0.55	0.0504730109482718	0.313685347222913	0.687394989990282\\
0.55	0.0521269574244539	0.318648231061428	0.685006090465981\\
0.55	0.0538089243380495	0.323608480015096	0.682718662029127\\
0.55	0.0555189414985328	0.328565868787292	0.68052227385347\\
0.55	0.0572570364191156	0.333520171232163	0.67840745699785\\
0.55	0.0590232342988274	0.338471160382329	0.676365666960043\\
0.55	0.0608175580049697	0.34341860847696	0.674389235783157\\
0.55	0.0626400280559541	0.348362286990222	0.672471317134862\\
0.55	0.064490662604533	0.3533019666601	0.670605827172181\\
0.55	0.0663694774214291	0.358237417517579	0.668787383459803\\
0.55	0.0682764858793752	0.363168408916186	0.667011243730728\\
0.55	0.0702116989375697	0.368094709561873	0.665273245863649\\
0.55	0.0721751251265572	0.373016087543259	0.663569750098932\\
0.55	0.0741667705335423	0.377932310362199	0.661897584220061\\
0.55	0.0761866387881432	0.382843144964686	0.660253992184442\\
0.55	0.0782347310485954	0.38774835777208	0.658636586490996\\
0.55	0.0803110459884098	0.392647714712651	0.657043304415871\\
0.55	0.0824155797834956	0.397540981253432	0.655472368126509\\
0.55	0.084548326099754	0.402427922432375	0.653922248592746\\
0.55	0.0867092760811502	0.4073083028908	0.65239163314657\\
0.55	0.0888984183382709	0.412181886906127	0.65087939649558\\
0.55	0.0911157389373742	0.417048438424897	0.649384574964844\\
0.55	0.0933612213899392	0.421907721096044	0.647906343724683\\
0.55	0.0956348466427212	0.42675949830445	0.646443996754806\\
0.55	0.0979365930683194	0.431603533204733	0.644996929295963\\
0.55	0.100266436456264	0.436439588755293	0.643564622546847\\
0.55	0.102624350004627	0.441267427752585	0.642146630374429\\
0.55	0.105010304312169	0.446086812865616	0.640742567819434\\
0.55	0.107424267371012	0.450897506670668	0.63935210119373\\
0.55	0.109866204559871	0.455699271686218	0.637974939582443\\
0.55	0.112336078637817	0.460491870408058	0.636610827579862\\
0.55	0.114833849738607	0.465275065344603	0.635259539104382\\
0.55	0.117359475365564	0.470048619052377	0.633920872153156\\
0.55	0.119912910387022	0.474812294171664	0.632594644371966\\
0.55	0.122494107032347	0.479565853462319	0.63128068932964\\
0.55	0.125103014888515	0.484309059839721	0.629978853399058\\
0.55	0.127739580897283	0.489041676410868	0.628690460824562\\
0.555	0	0	0.710202759607373\\
0.555	1.11327767495586e-05	0.00471862581271275	0.713003157837853\\
0.555	4.46175251031912e-05	0.009446325184051	0.715846486660142\\
0.555	0.000100583311362513	0.0141829653360114	0.718788838379794\\
0.555	0.000179158434668431	0.018928411755669	0.721886092228356\\
0.555	0.000280470402701511	0.02368252819604	0.725200636220892\\
0.555	0.000404645907256436	0.0284451766772965	0.728799085274723\\
0.555	0.000551810799695644	0.0332162174883388	0.732749123983336\\
0.555	0.000722090066287311	0.037995509188729	0.737115575253327\\
0.555	0.000915607803432999	0.0427829086109896	0.741955893907011\\
0.555	0.0011324871927904	0.0475782708632729	0.747315373652289\\
0.555	0.00137285047629673	0.0523814493324038	0.753222424045811\\
0.555	0.00163681893109844	0.057192295687301	0.759684310525652\\
0.555	0.00192451284439304	0.0620106598827802	0.766683748102665\\
0.555	0.00223605148818898	0.0668363901637434	0.774176695511256\\
0.555	0.00257155309398959	0.0716693330697584	0.782091614494493\\
0.555	0.00293113482740716	0.0765093334400312	0.79033034642109\\
0.555	0.00331491276271365	0.0813562344187764	0.79877062752665\\
0.555	0.00372300185733413	0.0862098774609879	0.807270128930359\\
0.555	0.00415551592628969	0.0910701023386145	0.815671782700194\\
0.555	0.00461256761659621	0.0959367471471425	0.823810053507329\\
0.555	0.00509426838162598	0.100809648312589	0.831517746503354\\
0.555	0.00560072845543873	0.105688640598912	0.838632911419452\\
0.555	0.00613205682708918	0.11057355711583	0.845005411376292\\
0.555	0.00668836121491816	0.115464229327074	0.850502769049301\\
0.555	0.00726974804083431	0.120360487059049	0.855014975717657\\
0.555	0.00787632240459385	0.125262158509929	0.858458041096606\\
0.555	0.00850818805808554	0.13016907025918	0.860776163494183\\
0.555	0.00916544737962849	0.135081047277508	0.861942500814223\\
0.555	0.00984820134829017	0.139997912937243	0.86195861464842\\
0.555	0.0105565495182326	0.144919489023162	0.860852735617092\\
0.555	0.0112905899930939	0.149845595743738	0.85867705410965\\
0.555	0.012050419400414	0.154776051742839	0.855504274967458\\
0.555	0.0128361328661109	0.159710674111862	0.851423687967757\\
0.555	0.0136478239890177	0.164649278402306	0.846537000484339\\
0.555	0.0144855848154856	0.169591678638796	0.840954157904248\\
0.555	0.0153495058140643	0.174537687332543	0.834789345403934\\
0.555	0.0162396758502651	0.179487115495259	0.828157325806199\\
0.555	0.0171561821614173	0.184439772653511	0.821170226448694\\
0.555	0.0180991103316243	0.189395466863524	0.813934846725692\\
0.555	0.0190685442668299	0.194354004726437	0.806550519904211\\
0.555	0.0200645661700016	0.199315191403999	0.799107529846306\\
0.555	0.0210872565164405	0.204278830634725	0.791686056496193\\
0.555	0.0221366940292259	0.20924472475049	0.7843556038306\\
0.555	0.023212955654804	0.214212674693577	0.777174850281117\\
0.555	0.0243161165387281	0.219182480034174	0.770191853862645\\
0.555	0.025446250001561	0.224153938988321	0.763444541554458\\
0.555	0.0266034275149466	0.229126848436298	0.756961413910483\\
0.555	0.0277877186778607	0.234101003941464	0.750762400416763\\
0.555	0.0289991911930496	0.239076199769547	0.744859807805785\\
0.555	0.0302379108436654	0.244052228908366	0.739259311518081\\
0.555	0.0315039414701067	0.24902888308801	0.733960949044503\\
0.555	0.0327973449470747	0.254005952801449	0.72896008240726\\
0.555	0.0341181811608519	0.258983227325592	0.72424830511118\\
0.555	0.0354665079868145	0.263960494742775	0.719814276220682\\
0.555	0.0368423812671862	0.268937541962694	0.715644470612628\\
0.555	0.0382458547890421	0.273914154744766	0.711723839838042\\
0.555	0.0396769802625737	0.278890117720924	0.708036382389731\\
0.555	0.0411358072996216	0.283865214418836	0.704565625567162\\
0.555	0.0426223833924861	0.288839227285554	0.701295023640315\\
0.555	0.0441367538930258	0.293811937711588	0.698208278749354\\
0.555	0.0456789619920503	0.298783126055387	0.69528959205535\\
0.555	0.0472490486990192	0.303752571668251	0.692523853197511\\
0.555	0.0488470528220538	0.308720052919643	0.689896776228119\\
0.555	0.0504730109482718	0.313685347222913	0.687394989990282\\
0.555	0.0521269574244539	0.318648231061428	0.68500609046598\\
0.555	0.0538089243380495	0.323608480015096	0.682718662029127\\
0.555	0.0555189414985328	0.328565868787292	0.680522273853469\\
0.555	0.0572570364191156	0.333520171232163	0.678407456997851\\
0.555	0.0590232342988274	0.338471160382329	0.676365666960043\\
0.555	0.0608175580049697	0.34341860847696	0.674389235783155\\
0.555	0.0626400280559541	0.348362286990222	0.672471317134863\\
0.555	0.064490662604533	0.3533019666601	0.670605827172182\\
0.555	0.0663694774214291	0.358237417517579	0.668787383459804\\
0.555	0.0682764858793752	0.363168408916186	0.667011243730728\\
0.555	0.0702116989375697	0.368094709561873	0.665273245863649\\
0.555	0.0721751251265572	0.373016087543259	0.663569750098932\\
0.555	0.0741667705335423	0.377932310362199	0.661897584220061\\
0.555	0.0761866387881432	0.382843144964686	0.660253992184442\\
0.555	0.0782347310485954	0.38774835777208	0.658636586490997\\
0.555	0.0803110459884098	0.392647714712651	0.657043304415871\\
0.555	0.0824155797834956	0.397540981253432	0.655472368126509\\
0.555	0.084548326099754	0.402427922432375	0.653922248592744\\
0.555	0.0867092760811502	0.407308302890799	0.652391633146571\\
0.555	0.0888984183382709	0.412181886906127	0.65087939649558\\
0.555	0.0911157389373742	0.417048438424896	0.649384574964845\\
0.555	0.0933612213899392	0.421907721096044	0.647906343724685\\
0.555	0.0956348466427212	0.42675949830445	0.646443996754805\\
0.555	0.0979365930683194	0.431603533204733	0.644996929295962\\
0.555	0.100266436456264	0.436439588755293	0.643564622546847\\
0.555	0.102624350004627	0.441267427752585	0.64214663037443\\
0.555	0.105010304312169	0.446086812865616	0.640742567819434\\
0.555	0.107424267371012	0.450897506670668	0.639352101193731\\
0.555	0.109866204559871	0.455699271686218	0.637974939582443\\
0.555	0.112336078637817	0.460491870408058	0.636610827579862\\
0.555	0.114833849738607	0.465275065344603	0.635259539104382\\
0.555	0.117359475365564	0.470048619052377	0.633920872153155\\
0.555	0.119912910387022	0.474812294171664	0.632594644371966\\
0.555	0.122494107032347	0.479565853462319	0.63128068932964\\
0.555	0.125103014888515	0.484309059839721	0.629978853399059\\
0.555	0.127739580897283	0.489041676410868	0.628690460824559\\
0.56	0	0	0.710202759607373\\
0.56	1.11327767495586e-05	0.00471862581271275	0.713003157837853\\
0.56	4.46175251031912e-05	0.009446325184051	0.715846486660142\\
0.56	0.000100583311362513	0.0141829653360114	0.718788838379794\\
0.56	0.000179158434668431	0.018928411755669	0.721886092228356\\
0.56	0.000280470402701511	0.02368252819604	0.725200636220892\\
0.56	0.000404645907256436	0.0284451766772965	0.728799085274723\\
0.56	0.000551810799695644	0.0332162174883388	0.732749123983336\\
0.56	0.000722090066287311	0.037995509188729	0.737115575253328\\
0.56	0.000915607803432999	0.0427829086109896	0.741955893907011\\
0.56	0.0011324871927904	0.047578270863273	0.747315373652289\\
0.56	0.00137285047629673	0.0523814493324038	0.753222424045811\\
0.56	0.00163681893109844	0.057192295687301	0.759684310525652\\
0.56	0.00192451284439304	0.0620106598827802	0.766683748102665\\
0.56	0.00223605148818898	0.0668363901637434	0.774176695511256\\
0.56	0.00257155309398959	0.0716693330697584	0.782091614494493\\
0.56	0.00293113482740716	0.0765093334400312	0.79033034642109\\
0.56	0.00331491276271365	0.0813562344187764	0.798770627526651\\
0.56	0.00372300185733413	0.0862098774609879	0.807270128930359\\
0.56	0.00415551592628969	0.0910701023386145	0.815671782700193\\
0.56	0.00461256761659621	0.0959367471471424	0.82381005350733\\
0.56	0.00509426838162598	0.100809648312589	0.831517746503353\\
0.56	0.00560072845543873	0.105688640598912	0.838632911419452\\
0.56	0.00613205682708918	0.11057355711583	0.845005411376293\\
0.56	0.00668836121491815	0.115464229327074	0.850502769049302\\
0.56	0.00726974804083431	0.120360487059049	0.855014975717657\\
0.56	0.00787632240459385	0.125262158509929	0.858458041096605\\
0.56	0.00850818805808555	0.13016907025918	0.860776163494183\\
0.56	0.00916544737962849	0.135081047277508	0.861942500814224\\
0.56	0.00984820134829017	0.139997912937243	0.861958614648421\\
0.56	0.0105565495182326	0.144919489023162	0.86085273561709\\
0.56	0.0112905899930939	0.149845595743738	0.858677054109652\\
0.56	0.012050419400414	0.154776051742839	0.855504274967458\\
0.56	0.0128361328661109	0.159710674111862	0.851423687967753\\
0.56	0.0136478239890177	0.164649278402306	0.84653700048434\\
0.56	0.0144855848154856	0.169591678638796	0.840954157904248\\
0.56	0.0153495058140643	0.174537687332543	0.834789345403933\\
0.56	0.0162396758502651	0.179487115495259	0.8281573258062\\
0.56	0.0171561821614173	0.184439772653511	0.821170226448694\\
0.56	0.0180991103316243	0.189395466863524	0.813934846725692\\
0.56	0.0190685442668299	0.194354004726437	0.806550519904211\\
0.56	0.0200645661700016	0.199315191403999	0.799107529846303\\
0.56	0.0210872565164405	0.204278830634725	0.791686056496193\\
0.56	0.0221366940292259	0.20924472475049	0.784355603830602\\
0.56	0.023212955654804	0.214212674693577	0.777174850281118\\
0.56	0.0243161165387281	0.219182480034174	0.770191853862643\\
0.56	0.025446250001561	0.224153938988321	0.763444541554459\\
0.56	0.0266034275149466	0.229126848436298	0.756961413910484\\
0.56	0.0277877186778607	0.234101003941464	0.750762400416763\\
0.56	0.0289991911930496	0.239076199769547	0.744859807805785\\
0.56	0.0302379108436654	0.244052228908366	0.739259311518078\\
0.56	0.0315039414701067	0.24902888308801	0.733960949044504\\
0.56	0.0327973449470747	0.254005952801449	0.728960082407263\\
0.56	0.0341181811608519	0.258983227325592	0.724248305111176\\
0.56	0.0354665079868145	0.263960494742775	0.719814276220679\\
0.56	0.0368423812671862	0.268937541962694	0.715644470612631\\
0.56	0.0382458547890421	0.273914154744766	0.711723839838043\\
0.56	0.0396769802625737	0.278890117720924	0.70803638238973\\
0.56	0.0411358072996216	0.283865214418836	0.704565625567164\\
0.56	0.0426223833924861	0.288839227285554	0.701295023640317\\
0.56	0.0441367538930258	0.293811937711588	0.698208278749354\\
0.56	0.0456789619920503	0.298783126055387	0.695289592055349\\
0.56	0.0472490486990192	0.303752571668251	0.692523853197512\\
0.56	0.0488470528220538	0.308720052919643	0.689896776228118\\
0.56	0.0504730109482718	0.313685347222913	0.687394989990282\\
0.56	0.0521269574244539	0.318648231061428	0.685006090465982\\
0.56	0.0538089243380495	0.323608480015097	0.682718662029128\\
0.56	0.0555189414985328	0.328565868787292	0.680522273853469\\
0.56	0.0572570364191156	0.333520171232163	0.678407456997849\\
0.56	0.0590232342988274	0.338471160382329	0.676365666960043\\
0.56	0.0608175580049696	0.34341860847696	0.674389235783157\\
0.56	0.0626400280559541	0.348362286990222	0.672471317134862\\
0.56	0.064490662604533	0.3533019666601	0.670605827172182\\
0.56	0.0663694774214291	0.358237417517579	0.668787383459804\\
0.56	0.0682764858793752	0.363168408916186	0.667011243730728\\
0.56	0.0702116989375697	0.368094709561873	0.66527324586365\\
0.56	0.0721751251265572	0.373016087543259	0.663569750098933\\
0.56	0.0741667705335423	0.377932310362199	0.66189758422006\\
0.56	0.0761866387881432	0.382843144964686	0.660253992184442\\
0.56	0.0782347310485954	0.38774835777208	0.658636586490997\\
0.56	0.0803110459884098	0.392647714712651	0.657043304415871\\
0.56	0.0824155797834956	0.397540981253432	0.655472368126508\\
0.56	0.084548326099754	0.402427922432375	0.653922248592744\\
0.56	0.0867092760811502	0.407308302890799	0.652391633146569\\
0.56	0.0888984183382709	0.412181886906127	0.65087939649558\\
0.56	0.0911157389373742	0.417048438424896	0.649384574964844\\
0.56	0.0933612213899392	0.421907721096044	0.647906343724686\\
0.56	0.0956348466427212	0.42675949830445	0.646443996754805\\
0.56	0.0979365930683194	0.431603533204733	0.644996929295961\\
0.56	0.100266436456264	0.436439588755293	0.643564622546848\\
0.56	0.102624350004627	0.441267427752585	0.642146630374429\\
0.56	0.105010304312169	0.446086812865616	0.640742567819432\\
0.56	0.107424267371012	0.450897506670668	0.639352101193731\\
0.56	0.109866204559871	0.455699271686218	0.637974939582444\\
0.56	0.112336078637817	0.460491870408058	0.636610827579862\\
0.56	0.114833849738607	0.465275065344603	0.635259539104382\\
0.56	0.117359475365564	0.470048619052377	0.633920872153154\\
0.56	0.119912910387022	0.474812294171664	0.632594644371966\\
0.56	0.122494107032347	0.479565853462319	0.631280689329641\\
0.56	0.125103014888515	0.484309059839721	0.629978853399057\\
0.56	0.127739580897283	0.489041676410868	0.628690460824561\\
0.565	0	0	0.710202759607373\\
0.565	1.11327767495586e-05	0.00471862581271275	0.713003157837853\\
0.565	4.46175251031912e-05	0.009446325184051	0.715846486660142\\
0.565	0.000100583311362513	0.0141829653360114	0.718788838379794\\
0.565	0.000179158434668431	0.018928411755669	0.721886092228356\\
0.565	0.000280470402701511	0.02368252819604	0.725200636220892\\
0.565	0.000404645907256436	0.0284451766772965	0.728799085274723\\
0.565	0.000551810799695644	0.0332162174883388	0.732749123983336\\
0.565	0.000722090066287311	0.037995509188729	0.737115575253327\\
0.565	0.000915607803432999	0.0427829086109896	0.741955893907011\\
0.565	0.0011324871927904	0.0475782708632729	0.747315373652289\\
0.565	0.00137285047629673	0.0523814493324038	0.753222424045811\\
0.565	0.00163681893109844	0.057192295687301	0.759684310525652\\
0.565	0.00192451284439304	0.0620106598827802	0.766683748102665\\
0.565	0.00223605148818898	0.0668363901637434	0.774176695511256\\
0.565	0.00257155309398959	0.0716693330697584	0.782091614494494\\
0.565	0.00293113482740716	0.0765093334400312	0.79033034642109\\
0.565	0.00331491276271365	0.0813562344187764	0.79877062752665\\
0.565	0.00372300185733413	0.0862098774609879	0.80727012893036\\
0.565	0.00415551592628969	0.0910701023386145	0.815671782700194\\
0.565	0.00461256761659621	0.0959367471471424	0.823810053507331\\
0.565	0.00509426838162598	0.100809648312589	0.831517746503353\\
0.565	0.00560072845543873	0.105688640598912	0.838632911419452\\
0.565	0.00613205682708918	0.11057355711583	0.845005411376293\\
0.565	0.00668836121491816	0.115464229327074	0.850502769049302\\
0.565	0.00726974804083431	0.120360487059049	0.855014975717657\\
0.565	0.00787632240459385	0.125262158509929	0.858458041096606\\
0.565	0.00850818805808555	0.13016907025918	0.860776163494182\\
0.565	0.00916544737962849	0.135081047277508	0.861942500814225\\
0.565	0.00984820134829017	0.139997912937243	0.86195861464842\\
0.565	0.0105565495182326	0.144919489023162	0.860852735617089\\
0.565	0.0112905899930939	0.149845595743738	0.858677054109649\\
0.565	0.012050419400414	0.154776051742839	0.855504274967455\\
0.565	0.0128361328661109	0.159710674111862	0.851423687967753\\
0.565	0.0136478239890177	0.164649278402306	0.84653700048434\\
0.565	0.0144855848154856	0.169591678638796	0.840954157904248\\
0.565	0.0153495058140643	0.174537687332543	0.834789345403936\\
0.565	0.0162396758502651	0.179487115495259	0.828157325806201\\
0.565	0.0171561821614173	0.184439772653511	0.821170226448695\\
0.565	0.0180991103316243	0.189395466863524	0.813934846725692\\
0.565	0.0190685442668299	0.194354004726437	0.806550519904212\\
0.565	0.0200645661700016	0.199315191403999	0.799107529846305\\
0.565	0.0210872565164405	0.204278830634725	0.791686056496197\\
0.565	0.0221366940292259	0.20924472475049	0.784355603830602\\
0.565	0.023212955654804	0.214212674693577	0.777174850281118\\
0.565	0.0243161165387281	0.219182480034174	0.770191853862644\\
0.565	0.025446250001561	0.224153938988321	0.763444541554461\\
0.565	0.0266034275149466	0.229126848436298	0.756961413910482\\
0.565	0.0277877186778607	0.234101003941464	0.750762400416763\\
0.565	0.0289991911930496	0.239076199769547	0.744859807805785\\
0.565	0.0302379108436654	0.244052228908366	0.73925931151808\\
0.565	0.0315039414701067	0.24902888308801	0.733960949044504\\
0.565	0.0327973449470747	0.254005952801449	0.728960082407259\\
0.565	0.0341181811608519	0.258983227325591	0.724248305111179\\
0.565	0.0354665079868146	0.263960494742775	0.719814276220683\\
0.565	0.0368423812671862	0.268937541962694	0.715644470612629\\
0.565	0.0382458547890421	0.273914154744766	0.711723839838041\\
0.565	0.0396769802625737	0.278890117720924	0.70803638238973\\
0.565	0.0411358072996216	0.283865214418836	0.704565625567165\\
0.565	0.0426223833924861	0.288839227285554	0.701295023640316\\
0.565	0.0441367538930258	0.293811937711588	0.698208278749353\\
0.565	0.0456789619920503	0.298783126055387	0.69528959205535\\
0.565	0.0472490486990192	0.303752571668251	0.69252385319751\\
0.565	0.0488470528220538	0.308720052919643	0.689896776228119\\
0.565	0.0504730109482718	0.313685347222913	0.687394989990282\\
0.565	0.0521269574244539	0.318648231061428	0.68500609046598\\
0.565	0.0538089243380495	0.323608480015097	0.682718662029127\\
0.565	0.0555189414985328	0.328565868787292	0.680522273853469\\
0.565	0.0572570364191156	0.333520171232163	0.678407456997851\\
0.565	0.0590232342988274	0.338471160382329	0.676365666960042\\
0.565	0.0608175580049696	0.34341860847696	0.674389235783156\\
0.565	0.0626400280559541	0.348362286990222	0.672471317134862\\
0.565	0.064490662604533	0.3533019666601	0.670605827172181\\
0.565	0.0663694774214291	0.358237417517579	0.668787383459804\\
0.565	0.0682764858793752	0.363168408916186	0.667011243730728\\
0.565	0.0702116989375697	0.368094709561873	0.665273245863648\\
0.565	0.0721751251265572	0.373016087543259	0.663569750098933\\
0.565	0.0741667705335423	0.377932310362199	0.661897584220061\\
0.565	0.0761866387881432	0.382843144964686	0.660253992184442\\
0.565	0.0782347310485954	0.38774835777208	0.658636586490997\\
0.565	0.0803110459884098	0.392647714712651	0.657043304415873\\
0.565	0.0824155797834956	0.397540981253432	0.65547236812651\\
0.565	0.084548326099754	0.402427922432375	0.653922248592745\\
0.565	0.0867092760811502	0.4073083028908	0.652391633146569\\
0.565	0.0888984183382709	0.412181886906127	0.650879396495579\\
0.565	0.0911157389373742	0.417048438424897	0.649384574964843\\
0.565	0.0933612213899392	0.421907721096044	0.647906343724684\\
0.565	0.0956348466427212	0.42675949830445	0.646443996754806\\
0.565	0.0979365930683194	0.431603533204733	0.644996929295961\\
0.565	0.100266436456264	0.436439588755293	0.643564622546846\\
0.565	0.102624350004627	0.441267427752584	0.642146630374431\\
0.565	0.105010304312169	0.446086812865616	0.640742567819433\\
0.565	0.107424267371012	0.450897506670668	0.63935210119373\\
0.565	0.109866204559871	0.455699271686218	0.637974939582443\\
0.565	0.112336078637817	0.460491870408058	0.636610827579862\\
0.565	0.114833849738607	0.465275065344603	0.635259539104381\\
0.565	0.117359475365564	0.470048619052377	0.633920872153154\\
0.565	0.119912910387023	0.474812294171664	0.632594644371965\\
0.565	0.122494107032347	0.479565853462319	0.631280689329642\\
0.565	0.125103014888515	0.484309059839721	0.629978853399059\\
0.565	0.127739580897283	0.489041676410868	0.628690460824556\\
0.57	0	0	0.710202759607373\\
0.57	1.11327767495586e-05	0.00471862581271275	0.713003157837853\\
0.57	4.46175251031912e-05	0.009446325184051	0.715846486660142\\
0.57	0.000100583311362513	0.0141829653360114	0.718788838379794\\
0.57	0.000179158434668431	0.018928411755669	0.721886092228356\\
0.57	0.000280470402701511	0.02368252819604	0.725200636220892\\
0.57	0.000404645907256436	0.0284451766772965	0.728799085274723\\
0.57	0.000551810799695644	0.0332162174883389	0.732749123983336\\
0.57	0.000722090066287311	0.037995509188729	0.737115575253328\\
0.57	0.000915607803432999	0.0427829086109896	0.741955893907011\\
0.57	0.0011324871927904	0.0475782708632729	0.747315373652289\\
0.57	0.00137285047629673	0.0523814493324038	0.753222424045811\\
0.57	0.00163681893109844	0.057192295687301	0.759684310525653\\
0.57	0.00192451284439304	0.0620106598827801	0.766683748102665\\
0.57	0.00223605148818898	0.0668363901637434	0.774176695511256\\
0.57	0.00257155309398959	0.0716693330697584	0.782091614494493\\
0.57	0.00293113482740716	0.0765093334400313	0.79033034642109\\
0.57	0.00331491276271365	0.0813562344187764	0.79877062752665\\
0.57	0.00372300185733413	0.0862098774609879	0.807270128930359\\
0.57	0.00415551592628969	0.0910701023386145	0.815671782700194\\
0.57	0.00461256761659621	0.0959367471471425	0.823810053507331\\
0.57	0.00509426838162598	0.100809648312589	0.831517746503355\\
0.57	0.00560072845543873	0.105688640598912	0.838632911419451\\
0.57	0.00613205682708918	0.11057355711583	0.845005411376293\\
0.57	0.00668836121491815	0.115464229327074	0.850502769049302\\
0.57	0.00726974804083431	0.120360487059049	0.855014975717657\\
0.57	0.00787632240459385	0.125262158509929	0.858458041096607\\
0.57	0.00850818805808555	0.13016907025918	0.860776163494183\\
0.57	0.00916544737962849	0.135081047277508	0.861942500814225\\
0.57	0.00984820134829017	0.139997912937243	0.86195861464842\\
0.57	0.0105565495182326	0.144919489023162	0.860852735617091\\
0.57	0.0112905899930939	0.149845595743738	0.858677054109649\\
0.57	0.012050419400414	0.154776051742839	0.855504274967456\\
0.57	0.0128361328661109	0.159710674111862	0.851423687967755\\
0.57	0.0136478239890177	0.164649278402306	0.846537000484338\\
0.57	0.0144855848154856	0.169591678638796	0.840954157904247\\
0.57	0.0153495058140643	0.174537687332543	0.834789345403936\\
0.57	0.0162396758502651	0.179487115495259	0.8281573258062\\
0.57	0.0171561821614173	0.184439772653511	0.821170226448695\\
0.57	0.0180991103316243	0.189395466863524	0.813934846725692\\
0.57	0.0190685442668299	0.194354004726437	0.806550519904212\\
0.57	0.0200645661700016	0.199315191403999	0.799107529846306\\
0.57	0.0210872565164405	0.204278830634725	0.791686056496194\\
0.57	0.0221366940292259	0.20924472475049	0.784355603830602\\
0.57	0.023212955654804	0.214212674693577	0.777174850281119\\
0.57	0.0243161165387281	0.219182480034174	0.770191853862644\\
0.57	0.025446250001561	0.224153938988321	0.763444541554458\\
0.57	0.0266034275149466	0.229126848436298	0.756961413910482\\
0.57	0.0277877186778607	0.234101003941464	0.750762400416764\\
0.57	0.0289991911930496	0.239076199769547	0.744859807805785\\
0.57	0.0302379108436654	0.244052228908366	0.739259311518081\\
0.57	0.0315039414701067	0.24902888308801	0.733960949044503\\
0.57	0.0327973449470747	0.254005952801449	0.728960082407261\\
0.57	0.0341181811608519	0.258983227325592	0.724248305111179\\
0.57	0.0354665079868146	0.263960494742775	0.719814276220682\\
0.57	0.0368423812671862	0.268937541962694	0.71564447061263\\
0.57	0.0382458547890421	0.273914154744766	0.711723839838041\\
0.57	0.0396769802625738	0.278890117720924	0.70803638238973\\
0.57	0.0411358072996216	0.283865214418836	0.704565625567163\\
0.57	0.0426223833924861	0.288839227285554	0.701295023640317\\
0.57	0.0441367538930258	0.293811937711588	0.698208278749354\\
0.57	0.0456789619920503	0.298783126055387	0.695289592055349\\
0.57	0.0472490486990192	0.303752571668251	0.692523853197512\\
0.57	0.0488470528220538	0.308720052919643	0.689896776228117\\
0.57	0.0504730109482718	0.313685347222913	0.687394989990282\\
0.57	0.0521269574244539	0.318648231061428	0.685006090465981\\
0.57	0.0538089243380495	0.323608480015096	0.682718662029128\\
0.57	0.0555189414985328	0.328565868787292	0.68052227385347\\
0.57	0.0572570364191156	0.333520171232163	0.678407456997851\\
0.57	0.0590232342988274	0.338471160382329	0.676365666960043\\
0.57	0.0608175580049697	0.34341860847696	0.674389235783155\\
0.57	0.0626400280559541	0.348362286990222	0.672471317134863\\
0.57	0.064490662604533	0.3533019666601	0.670605827172182\\
0.57	0.0663694774214291	0.358237417517579	0.668787383459804\\
0.57	0.0682764858793752	0.363168408916186	0.667011243730728\\
0.57	0.0702116989375697	0.368094709561873	0.66527324586365\\
0.57	0.0721751251265572	0.373016087543259	0.663569750098933\\
0.57	0.0741667705335422	0.377932310362199	0.661897584220059\\
0.57	0.0761866387881432	0.382843144964686	0.660253992184443\\
0.57	0.0782347310485954	0.38774835777208	0.658636586490997\\
0.57	0.0803110459884098	0.392647714712651	0.657043304415871\\
0.57	0.0824155797834956	0.397540981253432	0.655472368126511\\
0.57	0.084548326099754	0.402427922432375	0.653922248592747\\
0.57	0.0867092760811502	0.407308302890799	0.652391633146571\\
0.57	0.0888984183382709	0.412181886906127	0.650879396495579\\
0.57	0.0911157389373742	0.417048438424896	0.649384574964844\\
0.57	0.0933612213899392	0.421907721096044	0.647906343724684\\
0.57	0.0956348466427212	0.42675949830445	0.646443996754806\\
0.57	0.0979365930683194	0.431603533204733	0.644996929295962\\
0.57	0.100266436456264	0.436439588755293	0.643564622546847\\
0.57	0.102624350004627	0.441267427752584	0.64214663037443\\
0.57	0.105010304312169	0.446086812865616	0.640742567819434\\
0.57	0.107424267371012	0.450897506670668	0.63935210119373\\
0.57	0.109866204559871	0.455699271686218	0.637974939582443\\
0.57	0.112336078637817	0.460491870408058	0.636610827579862\\
0.57	0.114833849738607	0.465275065344603	0.635259539104382\\
0.57	0.117359475365564	0.470048619052377	0.633920872153153\\
0.57	0.119912910387023	0.474812294171664	0.632594644371965\\
0.57	0.122494107032347	0.479565853462319	0.631280689329641\\
0.57	0.125103014888515	0.484309059839721	0.629978853399059\\
0.57	0.127739580897283	0.489041676410868	0.628690460824559\\
0.575	0	0	0.710202759607373\\
0.575	1.11327767495586e-05	0.00471862581271274	0.713003157837853\\
0.575	4.46175251031912e-05	0.009446325184051	0.715846486660142\\
0.575	0.000100583311362513	0.0141829653360114	0.718788838379794\\
0.575	0.000179158434668431	0.018928411755669	0.721886092228356\\
0.575	0.000280470402701511	0.02368252819604	0.725200636220892\\
0.575	0.000404645907256436	0.0284451766772965	0.728799085274723\\
0.575	0.000551810799695644	0.0332162174883389	0.732749123983336\\
0.575	0.000722090066287311	0.037995509188729	0.737115575253328\\
0.575	0.000915607803432999	0.0427829086109896	0.741955893907011\\
0.575	0.0011324871927904	0.047578270863273	0.747315373652289\\
0.575	0.00137285047629673	0.0523814493324038	0.75322242404581\\
0.575	0.00163681893109844	0.057192295687301	0.759684310525652\\
0.575	0.00192451284439304	0.0620106598827802	0.766683748102665\\
0.575	0.00223605148818898	0.0668363901637434	0.774176695511257\\
0.575	0.00257155309398959	0.0716693330697584	0.782091614494493\\
0.575	0.00293113482740716	0.0765093334400312	0.79033034642109\\
0.575	0.00331491276271365	0.0813562344187764	0.79877062752665\\
0.575	0.00372300185733413	0.0862098774609879	0.80727012893036\\
0.575	0.00415551592628969	0.0910701023386145	0.815671782700193\\
0.575	0.00461256761659621	0.0959367471471425	0.82381005350733\\
0.575	0.00509426838162598	0.100809648312589	0.831517746503354\\
0.575	0.00560072845543873	0.105688640598912	0.838632911419452\\
0.575	0.00613205682708918	0.11057355711583	0.845005411376292\\
0.575	0.00668836121491815	0.115464229327074	0.850502769049301\\
0.575	0.00726974804083431	0.120360487059049	0.855014975717657\\
0.575	0.00787632240459385	0.125262158509929	0.858458041096606\\
0.575	0.00850818805808554	0.13016907025918	0.860776163494183\\
0.575	0.00916544737962849	0.135081047277508	0.861942500814225\\
0.575	0.00984820134829017	0.139997912937243	0.861958614648419\\
0.575	0.0105565495182326	0.144919489023162	0.860852735617091\\
0.575	0.0112905899930939	0.149845595743738	0.858677054109652\\
0.575	0.012050419400414	0.154776051742839	0.855504274967458\\
0.575	0.0128361328661109	0.159710674111862	0.851423687967753\\
0.575	0.0136478239890177	0.164649278402306	0.846537000484339\\
0.575	0.0144855848154856	0.169591678638796	0.840954157904251\\
0.575	0.0153495058140643	0.174537687332543	0.834789345403935\\
0.575	0.0162396758502651	0.179487115495259	0.8281573258062\\
0.575	0.0171561821614173	0.184439772653511	0.821170226448693\\
0.575	0.0180991103316243	0.189395466863524	0.813934846725691\\
0.575	0.0190685442668299	0.194354004726437	0.806550519904213\\
0.575	0.0200645661700016	0.199315191403999	0.799107529846305\\
0.575	0.0210872565164405	0.204278830634725	0.791686056496194\\
0.575	0.0221366940292259	0.20924472475049	0.784355603830603\\
0.575	0.023212955654804	0.214212674693577	0.777174850281118\\
0.575	0.0243161165387281	0.219182480034174	0.770191853862641\\
0.575	0.025446250001561	0.224153938988321	0.763444541554457\\
0.575	0.0266034275149466	0.229126848436298	0.756961413910483\\
0.575	0.0277877186778607	0.234101003941464	0.750762400416765\\
0.575	0.0289991911930496	0.239076199769547	0.744859807805786\\
0.575	0.0302379108436654	0.244052228908366	0.739259311518077\\
0.575	0.0315039414701067	0.24902888308801	0.733960949044504\\
0.575	0.0327973449470747	0.254005952801449	0.728960082407263\\
0.575	0.0341181811608519	0.258983227325592	0.724248305111177\\
0.575	0.0354665079868145	0.263960494742775	0.71981427622068\\
0.575	0.0368423812671862	0.268937541962694	0.71564447061263\\
0.575	0.0382458547890421	0.273914154744766	0.711723839838041\\
0.575	0.0396769802625738	0.278890117720924	0.70803638238973\\
0.575	0.0411358072996216	0.283865214418836	0.704565625567163\\
0.575	0.0426223833924861	0.288839227285554	0.701295023640316\\
0.575	0.0441367538930258	0.293811937711588	0.698208278749354\\
0.575	0.0456789619920503	0.298783126055387	0.695289592055349\\
0.575	0.0472490486990192	0.303752571668251	0.692523853197513\\
0.575	0.0488470528220538	0.308720052919643	0.689896776228119\\
0.575	0.0504730109482718	0.313685347222913	0.687394989990281\\
0.575	0.0521269574244539	0.318648231061428	0.68500609046598\\
0.575	0.0538089243380495	0.323608480015096	0.682718662029127\\
0.575	0.0555189414985328	0.328565868787292	0.680522273853469\\
0.575	0.0572570364191156	0.333520171232163	0.678407456997851\\
0.575	0.0590232342988274	0.338471160382329	0.676365666960044\\
0.575	0.0608175580049697	0.34341860847696	0.674389235783156\\
0.575	0.0626400280559541	0.348362286990222	0.672471317134862\\
0.575	0.064490662604533	0.3533019666601	0.670605827172182\\
0.575	0.0663694774214291	0.358237417517579	0.668787383459804\\
0.575	0.0682764858793752	0.363168408916186	0.667011243730727\\
0.575	0.0702116989375697	0.368094709561873	0.665273245863648\\
0.575	0.0721751251265572	0.373016087543259	0.663569750098933\\
0.575	0.0741667705335422	0.377932310362199	0.66189758422006\\
0.575	0.0761866387881432	0.382843144964686	0.660253992184441\\
0.575	0.0782347310485954	0.38774835777208	0.658636586490997\\
0.575	0.0803110459884098	0.392647714712651	0.657043304415871\\
0.575	0.0824155797834956	0.397540981253432	0.655472368126508\\
0.575	0.084548326099754	0.402427922432375	0.653922248592745\\
0.575	0.0867092760811502	0.407308302890799	0.652391633146571\\
0.575	0.0888984183382709	0.412181886906127	0.65087939649558\\
0.575	0.0911157389373742	0.417048438424897	0.649384574964845\\
0.575	0.0933612213899392	0.421907721096044	0.647906343724684\\
0.575	0.0956348466427212	0.42675949830445	0.646443996754805\\
0.575	0.0979365930683194	0.431603533204733	0.644996929295962\\
0.575	0.100266436456264	0.436439588755293	0.643564622546847\\
0.575	0.102624350004627	0.441267427752585	0.64214663037443\\
0.575	0.105010304312169	0.446086812865616	0.640742567819433\\
0.575	0.107424267371012	0.450897506670668	0.639352101193731\\
0.575	0.109866204559871	0.455699271686218	0.637974939582443\\
0.575	0.112336078637817	0.460491870408058	0.636610827579862\\
0.575	0.114833849738607	0.465275065344603	0.635259539104383\\
0.575	0.117359475365564	0.470048619052377	0.633920872153154\\
0.575	0.119912910387022	0.474812294171664	0.632594644371964\\
0.575	0.122494107032347	0.479565853462319	0.631280689329641\\
0.575	0.125103014888515	0.484309059839721	0.62997885339906\\
0.575	0.127739580897283	0.489041676410868	0.628690460824557\\
0.58	0	0	0.710202759607373\\
0.58	1.11327767495586e-05	0.00471862581271274	0.713003157837853\\
0.58	4.46175251031912e-05	0.009446325184051	0.715846486660142\\
0.58	0.000100583311362513	0.0141829653360114	0.718788838379794\\
0.58	0.000179158434668431	0.018928411755669	0.721886092228356\\
0.58	0.000280470402701511	0.02368252819604	0.725200636220892\\
0.58	0.000404645907256436	0.0284451766772965	0.728799085274723\\
0.58	0.000551810799695644	0.0332162174883388	0.732749123983336\\
0.58	0.000722090066287311	0.037995509188729	0.737115575253327\\
0.58	0.000915607803432999	0.0427829086109896	0.741955893907011\\
0.58	0.0011324871927904	0.0475782708632729	0.747315373652289\\
0.58	0.00137285047629673	0.0523814493324038	0.75322242404581\\
0.58	0.00163681893109844	0.057192295687301	0.759684310525652\\
0.58	0.00192451284439304	0.0620106598827802	0.766683748102665\\
0.58	0.00223605148818899	0.0668363901637434	0.774176695511256\\
0.58	0.00257155309398959	0.0716693330697584	0.782091614494494\\
0.58	0.00293113482740716	0.0765093334400312	0.79033034642109\\
0.58	0.00331491276271365	0.0813562344187764	0.79877062752665\\
0.58	0.00372300185733413	0.0862098774609879	0.807270128930359\\
0.58	0.00415551592628969	0.0910701023386145	0.815671782700194\\
0.58	0.00461256761659621	0.0959367471471425	0.82381005350733\\
0.58	0.00509426838162598	0.100809648312589	0.831517746503353\\
0.58	0.00560072845543873	0.105688640598912	0.838632911419452\\
0.58	0.00613205682708918	0.11057355711583	0.845005411376293\\
0.58	0.00668836121491816	0.115464229327074	0.850502769049301\\
0.58	0.00726974804083431	0.120360487059049	0.855014975717657\\
0.58	0.00787632240459385	0.125262158509929	0.858458041096606\\
0.58	0.00850818805808554	0.13016907025918	0.860776163494182\\
0.58	0.00916544737962849	0.135081047277508	0.861942500814223\\
0.58	0.00984820134829017	0.139997912937243	0.86195861464842\\
0.58	0.0105565495182326	0.144919489023162	0.86085273561709\\
0.58	0.0112905899930939	0.149845595743738	0.858677054109649\\
0.58	0.012050419400414	0.154776051742839	0.855504274967459\\
0.58	0.0128361328661109	0.159710674111862	0.851423687967753\\
0.58	0.0136478239890177	0.164649278402306	0.846537000484344\\
0.58	0.0144855848154856	0.169591678638796	0.840954157904251\\
0.58	0.0153495058140643	0.174537687332543	0.834789345403934\\
0.58	0.0162396758502651	0.179487115495259	0.8281573258062\\
0.58	0.0171561821614173	0.184439772653511	0.821170226448694\\
0.58	0.0180991103316243	0.189395466863524	0.813934846725693\\
0.58	0.0190685442668299	0.194354004726437	0.806550519904213\\
0.58	0.0200645661700016	0.199315191403999	0.799107529846305\\
0.58	0.0210872565164405	0.204278830634725	0.791686056496195\\
0.58	0.0221366940292259	0.20924472475049	0.784355603830601\\
0.58	0.023212955654804	0.214212674693577	0.777174850281116\\
0.58	0.0243161165387281	0.219182480034174	0.770191853862643\\
0.58	0.025446250001561	0.224153938988321	0.76344454155446\\
0.58	0.0266034275149466	0.229126848436298	0.756961413910483\\
0.58	0.0277877186778607	0.234101003941464	0.750762400416766\\
0.58	0.0289991911930496	0.239076199769547	0.744859807805783\\
0.58	0.0302379108436654	0.244052228908366	0.73925931151808\\
0.58	0.0315039414701067	0.24902888308801	0.733960949044505\\
0.58	0.0327973449470747	0.254005952801449	0.72896008240726\\
0.58	0.0341181811608519	0.258983227325592	0.724248305111176\\
0.58	0.0354665079868145	0.263960494742775	0.719814276220681\\
0.58	0.0368423812671862	0.268937541962694	0.715644470612629\\
0.58	0.0382458547890421	0.273914154744766	0.711723839838041\\
0.58	0.0396769802625738	0.278890117720924	0.708036382389731\\
0.58	0.0411358072996216	0.283865214418836	0.704565625567165\\
0.58	0.0426223833924861	0.288839227285554	0.701295023640317\\
0.58	0.0441367538930258	0.293811937711588	0.698208278749353\\
0.58	0.0456789619920503	0.298783126055387	0.695289592055349\\
0.58	0.0472490486990192	0.303752571668251	0.692523853197512\\
0.58	0.0488470528220538	0.308720052919643	0.689896776228118\\
0.58	0.0504730109482718	0.313685347222913	0.687394989990282\\
0.58	0.0521269574244539	0.318648231061428	0.685006090465981\\
0.58	0.0538089243380495	0.323608480015096	0.682718662029127\\
0.58	0.0555189414985328	0.328565868787292	0.680522273853468\\
0.58	0.0572570364191156	0.333520171232163	0.678407456997849\\
0.58	0.0590232342988274	0.338471160382329	0.676365666960044\\
0.58	0.0608175580049697	0.34341860847696	0.674389235783157\\
0.58	0.0626400280559541	0.348362286990222	0.672471317134862\\
0.58	0.064490662604533	0.3533019666601	0.670605827172181\\
0.58	0.0663694774214291	0.358237417517579	0.668787383459804\\
0.58	0.0682764858793752	0.363168408916186	0.667011243730727\\
0.58	0.0702116989375697	0.368094709561873	0.665273245863648\\
0.58	0.0721751251265572	0.373016087543259	0.663569750098932\\
0.58	0.0741667705335422	0.377932310362199	0.661897584220061\\
0.58	0.0761866387881432	0.382843144964686	0.660253992184443\\
0.58	0.0782347310485954	0.38774835777208	0.658636586490998\\
0.58	0.0803110459884098	0.392647714712651	0.65704330441587\\
0.58	0.0824155797834956	0.397540981253432	0.655472368126508\\
0.58	0.084548326099754	0.402427922432375	0.653922248592745\\
0.58	0.0867092760811502	0.407308302890799	0.65239163314657\\
0.58	0.0888984183382709	0.412181886906127	0.65087939649558\\
0.58	0.0911157389373742	0.417048438424897	0.649384574964844\\
0.58	0.0933612213899392	0.421907721096044	0.647906343724685\\
0.58	0.0956348466427212	0.42675949830445	0.646443996754806\\
0.58	0.0979365930683194	0.431603533204733	0.644996929295961\\
0.58	0.100266436456264	0.436439588755293	0.643564622546846\\
0.58	0.102624350004627	0.441267427752584	0.64214663037443\\
0.58	0.105010304312169	0.446086812865616	0.640742567819434\\
0.58	0.107424267371012	0.450897506670668	0.639352101193731\\
0.58	0.109866204559871	0.455699271686218	0.637974939582443\\
0.58	0.112336078637817	0.460491870408058	0.636610827579862\\
0.58	0.114833849738607	0.465275065344603	0.635259539104383\\
0.58	0.117359475365564	0.470048619052377	0.633920872153155\\
0.58	0.119912910387022	0.474812294171664	0.632594644371965\\
0.58	0.122494107032347	0.479565853462319	0.631280689329639\\
0.58	0.125103014888515	0.484309059839721	0.629978853399059\\
0.58	0.127739580897283	0.489041676410868	0.628690460824563\\
0.585	0	0	0.710202759607373\\
0.585	1.11327767495586e-05	0.00471862581271275	0.713003157837853\\
0.585	4.46175251031912e-05	0.009446325184051	0.715846486660142\\
0.585	0.000100583311362513	0.0141829653360114	0.718788838379794\\
0.585	0.000179158434668431	0.018928411755669	0.721886092228357\\
0.585	0.000280470402701511	0.02368252819604	0.725200636220892\\
0.585	0.000404645907256436	0.0284451766772965	0.728799085274723\\
0.585	0.000551810799695644	0.0332162174883389	0.732749123983336\\
0.585	0.000722090066287311	0.037995509188729	0.737115575253327\\
0.585	0.000915607803432999	0.0427829086109896	0.741955893907011\\
0.585	0.0011324871927904	0.0475782708632729	0.747315373652289\\
0.585	0.00137285047629673	0.0523814493324038	0.753222424045811\\
0.585	0.00163681893109844	0.057192295687301	0.759684310525652\\
0.585	0.00192451284439304	0.0620106598827802	0.766683748102665\\
0.585	0.00223605148818898	0.0668363901637434	0.774176695511256\\
0.585	0.00257155309398959	0.0716693330697584	0.782091614494493\\
0.585	0.00293113482740716	0.0765093334400312	0.79033034642109\\
0.585	0.00331491276271365	0.0813562344187764	0.79877062752665\\
0.585	0.00372300185733413	0.0862098774609879	0.807270128930359\\
0.585	0.00415551592628969	0.0910701023386145	0.815671782700194\\
0.585	0.00461256761659621	0.0959367471471425	0.82381005350733\\
0.585	0.00509426838162598	0.100809648312589	0.831517746503353\\
0.585	0.00560072845543873	0.105688640598912	0.838632911419452\\
0.585	0.00613205682708918	0.11057355711583	0.845005411376293\\
0.585	0.00668836121491816	0.115464229327074	0.850502769049302\\
0.585	0.00726974804083431	0.120360487059049	0.855014975717657\\
0.585	0.00787632240459385	0.125262158509929	0.858458041096605\\
0.585	0.00850818805808554	0.13016907025918	0.860776163494182\\
0.585	0.00916544737962849	0.135081047277508	0.861942500814224\\
0.585	0.00984820134829017	0.139997912937243	0.861958614648422\\
0.585	0.0105565495182326	0.144919489023162	0.860852735617092\\
0.585	0.0112905899930939	0.149845595743738	0.858677054109649\\
0.585	0.012050419400414	0.154776051742839	0.855504274967458\\
0.585	0.0128361328661109	0.159710674111862	0.851423687967755\\
0.585	0.0136478239890177	0.164649278402306	0.846537000484341\\
0.585	0.0144855848154856	0.169591678638796	0.840954157904248\\
0.585	0.0153495058140643	0.174537687332543	0.834789345403933\\
0.585	0.0162396758502651	0.179487115495259	0.828157325806201\\
0.585	0.0171561821614173	0.184439772653511	0.821170226448696\\
0.585	0.0180991103316243	0.189395466863524	0.813934846725692\\
0.585	0.0190685442668299	0.194354004726437	0.806550519904213\\
0.585	0.0200645661700016	0.199315191403999	0.799107529846306\\
0.585	0.0210872565164405	0.204278830634725	0.791686056496195\\
0.585	0.0221366940292259	0.20924472475049	0.784355603830601\\
0.585	0.023212955654804	0.214212674693577	0.777174850281119\\
0.585	0.0243161165387281	0.219182480034174	0.770191853862645\\
0.585	0.025446250001561	0.224153938988321	0.763444541554457\\
0.585	0.0266034275149466	0.229126848436298	0.756961413910484\\
0.585	0.0277877186778607	0.234101003941464	0.750762400416764\\
0.585	0.0289991911930496	0.239076199769547	0.744859807805783\\
0.585	0.0302379108436654	0.244052228908366	0.73925931151808\\
0.585	0.0315039414701067	0.24902888308801	0.733960949044503\\
0.585	0.0327973449470747	0.254005952801449	0.72896008240726\\
0.585	0.0341181811608519	0.258983227325591	0.724248305111179\\
0.585	0.0354665079868145	0.263960494742775	0.719814276220681\\
0.585	0.0368423812671862	0.268937541962694	0.715644470612629\\
0.585	0.0382458547890421	0.273914154744766	0.711723839838043\\
0.585	0.0396769802625738	0.278890117720924	0.708036382389731\\
0.585	0.0411358072996216	0.283865214418836	0.704565625567163\\
0.585	0.0426223833924861	0.288839227285554	0.701295023640315\\
0.585	0.0441367538930258	0.293811937711588	0.698208278749355\\
0.585	0.0456789619920503	0.298783126055387	0.69528959205535\\
0.585	0.0472490486990192	0.303752571668251	0.692523853197511\\
0.585	0.0488470528220538	0.308720052919643	0.689896776228119\\
0.585	0.0504730109482718	0.313685347222913	0.687394989990281\\
0.585	0.0521269574244539	0.318648231061428	0.685006090465981\\
0.585	0.0538089243380495	0.323608480015096	0.682718662029128\\
0.585	0.0555189414985328	0.328565868787292	0.680522273853469\\
0.585	0.0572570364191156	0.333520171232163	0.678407456997849\\
0.585	0.0590232342988274	0.338471160382329	0.676365666960044\\
0.585	0.0608175580049697	0.34341860847696	0.674389235783156\\
0.585	0.0626400280559541	0.348362286990222	0.672471317134863\\
0.585	0.064490662604533	0.3533019666601	0.670605827172182\\
0.585	0.0663694774214291	0.358237417517579	0.668787383459803\\
0.585	0.0682764858793752	0.363168408916186	0.667011243730729\\
0.585	0.0702116989375697	0.368094709561873	0.665273245863648\\
0.585	0.0721751251265572	0.373016087543259	0.663569750098932\\
0.585	0.0741667705335422	0.377932310362199	0.66189758422006\\
0.585	0.0761866387881432	0.382843144964686	0.660253992184442\\
0.585	0.0782347310485954	0.38774835777208	0.658636586490998\\
0.585	0.0803110459884098	0.392647714712651	0.657043304415871\\
0.585	0.0824155797834956	0.397540981253432	0.655472368126508\\
0.585	0.084548326099754	0.402427922432375	0.653922248592745\\
0.585	0.0867092760811502	0.407308302890799	0.65239163314657\\
0.585	0.0888984183382709	0.412181886906127	0.650879396495579\\
0.585	0.0911157389373742	0.417048438424896	0.649384574964844\\
0.585	0.0933612213899392	0.421907721096044	0.647906343724685\\
0.585	0.0956348466427212	0.42675949830445	0.646443996754807\\
0.585	0.0979365930683194	0.431603533204733	0.644996929295963\\
0.585	0.100266436456264	0.436439588755293	0.643564622546846\\
0.585	0.102624350004627	0.441267427752584	0.642146630374429\\
0.585	0.105010304312169	0.446086812865616	0.640742567819433\\
0.585	0.107424267371012	0.450897506670668	0.63935210119373\\
0.585	0.109866204559871	0.455699271686218	0.637974939582443\\
0.585	0.112336078637817	0.460491870408058	0.636610827579862\\
0.585	0.114833849738607	0.465275065344603	0.635259539104383\\
0.585	0.117359475365564	0.470048619052377	0.633920872153155\\
0.585	0.119912910387023	0.474812294171664	0.632594644371966\\
0.585	0.122494107032347	0.479565853462319	0.631280689329641\\
0.585	0.125103014888515	0.484309059839721	0.629978853399058\\
0.585	0.127739580897283	0.489041676410868	0.628690460824559\\
0.59	0	0	0.710202759607373\\
0.59	1.11327767495586e-05	0.00471862581271274	0.713003157837853\\
0.59	4.46175251031912e-05	0.009446325184051	0.715846486660142\\
0.59	0.000100583311362513	0.0141829653360114	0.718788838379794\\
0.59	0.000179158434668431	0.018928411755669	0.721886092228356\\
0.59	0.000280470402701511	0.02368252819604	0.725200636220892\\
0.59	0.000404645907256436	0.0284451766772965	0.728799085274723\\
0.59	0.000551810799695644	0.0332162174883388	0.732749123983336\\
0.59	0.000722090066287311	0.037995509188729	0.737115575253327\\
0.59	0.000915607803432999	0.0427829086109896	0.741955893907011\\
0.59	0.0011324871927904	0.047578270863273	0.747315373652289\\
0.59	0.00137285047629673	0.0523814493324038	0.753222424045811\\
0.59	0.00163681893109844	0.057192295687301	0.759684310525652\\
0.59	0.00192451284439304	0.0620106598827802	0.766683748102665\\
0.59	0.00223605148818898	0.0668363901637434	0.774176695511257\\
0.59	0.00257155309398959	0.0716693330697584	0.782091614494493\\
0.59	0.00293113482740716	0.0765093334400312	0.79033034642109\\
0.59	0.00331491276271365	0.0813562344187764	0.798770627526651\\
0.59	0.00372300185733414	0.0862098774609879	0.807270128930359\\
0.59	0.00415551592628969	0.0910701023386145	0.815671782700193\\
0.59	0.00461256761659621	0.0959367471471424	0.823810053507331\\
0.59	0.00509426838162598	0.100809648312589	0.831517746503354\\
0.59	0.00560072845543873	0.105688640598912	0.838632911419451\\
0.59	0.00613205682708918	0.11057355711583	0.845005411376293\\
0.59	0.00668836121491816	0.115464229327074	0.850502769049302\\
0.59	0.00726974804083431	0.120360487059049	0.855014975717657\\
0.59	0.00787632240459385	0.125262158509929	0.858458041096607\\
0.59	0.00850818805808555	0.13016907025918	0.860776163494182\\
0.59	0.00916544737962849	0.135081047277508	0.861942500814225\\
0.59	0.00984820134829017	0.139997912937243	0.861958614648421\\
0.59	0.0105565495182326	0.144919489023162	0.860852735617092\\
0.59	0.0112905899930939	0.149845595743738	0.858677054109651\\
0.59	0.012050419400414	0.154776051742839	0.855504274967458\\
0.59	0.0128361328661109	0.159710674111862	0.851423687967755\\
0.59	0.0136478239890177	0.164649278402306	0.84653700048434\\
0.59	0.0144855848154856	0.169591678638796	0.840954157904246\\
0.59	0.0153495058140643	0.174537687332543	0.834789345403935\\
0.59	0.0162396758502651	0.179487115495259	0.828157325806202\\
0.59	0.0171561821614173	0.184439772653511	0.821170226448694\\
0.59	0.0180991103316243	0.189395466863524	0.813934846725692\\
0.59	0.0190685442668299	0.194354004726437	0.806550519904214\\
0.59	0.0200645661700016	0.199315191403999	0.799107529846306\\
0.59	0.0210872565164405	0.204278830634725	0.791686056496193\\
0.59	0.0221366940292259	0.20924472475049	0.784355603830601\\
0.59	0.023212955654804	0.214212674693577	0.777174850281118\\
0.59	0.0243161165387281	0.219182480034174	0.770191853862642\\
0.59	0.025446250001561	0.224153938988321	0.763444541554458\\
0.59	0.0266034275149466	0.229126848436298	0.756961413910485\\
0.59	0.0277877186778607	0.234101003941464	0.750762400416763\\
0.59	0.0289991911930496	0.239076199769547	0.744859807805786\\
0.59	0.0302379108436654	0.244052228908366	0.739259311518079\\
0.59	0.0315039414701067	0.24902888308801	0.733960949044504\\
0.59	0.0327973449470747	0.254005952801449	0.728960082407262\\
0.59	0.0341181811608519	0.258983227325592	0.724248305111177\\
0.59	0.0354665079868145	0.263960494742775	0.719814276220681\\
0.59	0.0368423812671862	0.268937541962694	0.715644470612631\\
0.59	0.0382458547890421	0.273914154744766	0.711723839838042\\
0.59	0.0396769802625738	0.278890117720924	0.708036382389731\\
0.59	0.0411358072996216	0.283865214418836	0.704565625567165\\
0.59	0.0426223833924861	0.288839227285554	0.701295023640316\\
0.59	0.0441367538930258	0.293811937711588	0.698208278749353\\
0.59	0.0456789619920503	0.298783126055387	0.69528959205535\\
0.59	0.0472490486990192	0.303752571668251	0.69252385319751\\
0.59	0.0488470528220538	0.308720052919643	0.689896776228118\\
0.59	0.0504730109482718	0.313685347222913	0.687394989990282\\
0.59	0.0521269574244539	0.318648231061427	0.68500609046598\\
0.59	0.0538089243380495	0.323608480015096	0.682718662029128\\
0.59	0.0555189414985328	0.328565868787292	0.68052227385347\\
0.59	0.0572570364191156	0.333520171232163	0.67840745699785\\
0.59	0.0590232342988274	0.338471160382329	0.676365666960043\\
0.59	0.0608175580049697	0.34341860847696	0.674389235783156\\
0.59	0.0626400280559541	0.348362286990222	0.672471317134862\\
0.59	0.064490662604533	0.3533019666601	0.670605827172183\\
0.59	0.0663694774214291	0.358237417517579	0.668787383459804\\
0.59	0.0682764858793752	0.363168408916186	0.667011243730728\\
0.59	0.0702116989375697	0.368094709561873	0.665273245863649\\
0.59	0.0721751251265572	0.373016087543259	0.663569750098933\\
0.59	0.0741667705335422	0.377932310362199	0.661897584220061\\
0.59	0.0761866387881432	0.382843144964686	0.660253992184442\\
0.59	0.0782347310485954	0.38774835777208	0.658636586490997\\
0.59	0.0803110459884098	0.392647714712651	0.657043304415872\\
0.59	0.0824155797834956	0.397540981253432	0.655472368126508\\
0.59	0.084548326099754	0.402427922432375	0.653922248592744\\
0.59	0.0867092760811502	0.407308302890799	0.65239163314657\\
0.59	0.0888984183382709	0.412181886906127	0.650879396495579\\
0.59	0.0911157389373742	0.417048438424896	0.649384574964843\\
0.59	0.0933612213899392	0.421907721096044	0.647906343724684\\
0.59	0.0956348466427212	0.42675949830445	0.646443996754805\\
0.59	0.0979365930683194	0.431603533204733	0.644996929295962\\
0.59	0.100266436456264	0.436439588755293	0.643564622546847\\
0.59	0.102624350004627	0.441267427752584	0.642146630374431\\
0.59	0.105010304312169	0.446086812865616	0.640742567819434\\
0.59	0.107424267371012	0.450897506670668	0.63935210119373\\
0.59	0.109866204559871	0.455699271686218	0.637974939582443\\
0.59	0.112336078637817	0.460491870408058	0.636610827579861\\
0.59	0.114833849738607	0.465275065344603	0.635259539104382\\
0.59	0.117359475365564	0.470048619052377	0.633920872153154\\
0.59	0.119912910387022	0.474812294171664	0.632594644371965\\
0.59	0.122494107032347	0.479565853462319	0.631280689329641\\
0.59	0.125103014888515	0.484309059839721	0.62997885339906\\
0.59	0.127739580897283	0.489041676410868	0.628690460824563\\
0.595	0	0	0.710202759607373\\
0.595	1.11327767495586e-05	0.00471862581271275	0.713003157837853\\
0.595	4.46175251031912e-05	0.009446325184051	0.715846486660142\\
0.595	0.000100583311362513	0.0141829653360114	0.718788838379794\\
0.595	0.000179158434668431	0.018928411755669	0.721886092228356\\
0.595	0.000280470402701511	0.02368252819604	0.725200636220892\\
0.595	0.000404645907256436	0.0284451766772965	0.728799085274723\\
0.595	0.000551810799695644	0.0332162174883388	0.732749123983336\\
0.595	0.000722090066287311	0.037995509188729	0.737115575253327\\
0.595	0.000915607803432999	0.0427829086109896	0.741955893907011\\
0.595	0.0011324871927904	0.0475782708632729	0.747315373652289\\
0.595	0.00137285047629673	0.0523814493324038	0.753222424045811\\
0.595	0.00163681893109844	0.057192295687301	0.759684310525652\\
0.595	0.00192451284439304	0.0620106598827802	0.766683748102665\\
0.595	0.00223605148818898	0.0668363901637434	0.774176695511256\\
0.595	0.00257155309398959	0.0716693330697584	0.782091614494494\\
0.595	0.00293113482740716	0.0765093334400312	0.79033034642109\\
0.595	0.00331491276271365	0.0813562344187764	0.79877062752665\\
0.595	0.00372300185733413	0.0862098774609879	0.80727012893036\\
0.595	0.00415551592628969	0.0910701023386145	0.815671782700193\\
0.595	0.00461256761659621	0.0959367471471425	0.823810053507331\\
0.595	0.00509426838162598	0.100809648312589	0.831517746503354\\
0.595	0.00560072845543873	0.105688640598912	0.838632911419452\\
0.595	0.00613205682708918	0.11057355711583	0.845005411376292\\
0.595	0.00668836121491816	0.115464229327074	0.850502769049302\\
0.595	0.00726974804083431	0.120360487059049	0.855014975717656\\
0.595	0.00787632240459385	0.125262158509929	0.858458041096608\\
0.595	0.00850818805808555	0.13016907025918	0.860776163494181\\
0.595	0.00916544737962849	0.135081047277508	0.861942500814226\\
0.595	0.00984820134829017	0.139997912937243	0.861958614648419\\
0.595	0.0105565495182326	0.144919489023162	0.860852735617091\\
0.595	0.0112905899930939	0.149845595743738	0.858677054109651\\
0.595	0.012050419400414	0.154776051742839	0.855504274967457\\
0.595	0.0128361328661109	0.159710674111862	0.851423687967756\\
0.595	0.0136478239890177	0.164649278402306	0.846537000484339\\
0.595	0.0144855848154856	0.169591678638796	0.840954157904247\\
0.595	0.0153495058140643	0.174537687332543	0.834789345403936\\
0.595	0.0162396758502651	0.179487115495259	0.8281573258062\\
0.595	0.0171561821614173	0.184439772653511	0.821170226448694\\
0.595	0.0180991103316243	0.189395466863524	0.813934846725692\\
0.595	0.0190685442668299	0.194354004726437	0.806550519904212\\
0.595	0.0200645661700016	0.199315191403999	0.799107529846304\\
0.595	0.0210872565164405	0.204278830634725	0.791686056496192\\
0.595	0.0221366940292259	0.20924472475049	0.784355603830602\\
0.595	0.023212955654804	0.214212674693577	0.777174850281117\\
0.595	0.0243161165387281	0.219182480034174	0.770191853862642\\
0.595	0.025446250001561	0.224153938988321	0.763444541554461\\
0.595	0.0266034275149466	0.229126848436298	0.756961413910484\\
0.595	0.0277877186778607	0.234101003941464	0.750762400416765\\
0.595	0.0289991911930496	0.239076199769547	0.744859807805785\\
0.595	0.0302379108436654	0.244052228908366	0.739259311518078\\
0.595	0.0315039414701067	0.24902888308801	0.733960949044504\\
0.595	0.0327973449470747	0.254005952801449	0.728960082407259\\
0.595	0.0341181811608519	0.258983227325591	0.724248305111178\\
0.595	0.0354665079868145	0.263960494742775	0.719814276220683\\
0.595	0.0368423812671862	0.268937541962694	0.715644470612629\\
0.595	0.0382458547890421	0.273914154744766	0.711723839838042\\
0.595	0.0396769802625738	0.278890117720924	0.708036382389729\\
0.595	0.0411358072996216	0.283865214418836	0.704565625567163\\
0.595	0.0426223833924861	0.288839227285554	0.701295023640317\\
0.595	0.0441367538930258	0.293811937711588	0.698208278749353\\
0.595	0.0456789619920503	0.298783126055387	0.695289592055349\\
0.595	0.0472490486990192	0.303752571668251	0.692523853197512\\
0.595	0.0488470528220537	0.308720052919643	0.689896776228118\\
0.595	0.0504730109482718	0.313685347222913	0.687394989990283\\
0.595	0.0521269574244539	0.318648231061428	0.68500609046598\\
0.595	0.0538089243380495	0.323608480015096	0.682718662029127\\
0.595	0.0555189414985328	0.328565868787292	0.680522273853471\\
0.595	0.0572570364191156	0.333520171232163	0.67840745699785\\
0.595	0.0590232342988274	0.338471160382329	0.676365666960042\\
0.595	0.0608175580049697	0.34341860847696	0.674389235783156\\
0.595	0.0626400280559541	0.348362286990222	0.672471317134863\\
0.595	0.064490662604533	0.3533019666601	0.670605827172181\\
0.595	0.0663694774214291	0.358237417517579	0.668787383459804\\
0.595	0.0682764858793752	0.363168408916186	0.667011243730729\\
0.595	0.0702116989375697	0.368094709561873	0.665273245863648\\
0.595	0.0721751251265572	0.373016087543259	0.663569750098932\\
0.595	0.0741667705335422	0.377932310362199	0.661897584220061\\
0.595	0.0761866387881432	0.382843144964686	0.660253992184442\\
0.595	0.0782347310485954	0.38774835777208	0.658636586490998\\
0.595	0.0803110459884098	0.392647714712651	0.657043304415872\\
0.595	0.0824155797834956	0.397540981253432	0.65547236812651\\
0.595	0.084548326099754	0.402427922432375	0.653922248592746\\
0.595	0.0867092760811502	0.4073083028908	0.65239163314657\\
0.595	0.0888984183382709	0.412181886906127	0.650879396495579\\
0.595	0.0911157389373742	0.417048438424897	0.649384574964844\\
0.595	0.0933612213899392	0.421907721096044	0.647906343724684\\
0.595	0.0956348466427212	0.42675949830445	0.646443996754805\\
0.595	0.0979365930683194	0.431603533204733	0.644996929295961\\
0.595	0.100266436456264	0.436439588755293	0.643564622546846\\
0.595	0.102624350004627	0.441267427752584	0.642146630374429\\
0.595	0.105010304312169	0.446086812865616	0.640742567819434\\
0.595	0.107424267371012	0.450897506670668	0.63935210119373\\
0.595	0.109866204559871	0.455699271686218	0.637974939582442\\
0.595	0.112336078637817	0.460491870408058	0.636610827579862\\
0.595	0.114833849738607	0.465275065344603	0.635259539104382\\
0.595	0.117359475365564	0.470048619052377	0.633920872153154\\
0.595	0.119912910387022	0.474812294171664	0.632594644371965\\
0.595	0.122494107032347	0.479565853462319	0.631280689329641\\
0.595	0.125103014888515	0.484309059839721	0.629978853399058\\
0.595	0.127739580897283	0.489041676410868	0.628690460824561\\
0.6	0	0	0.710202759607373\\
0.6	1.11327767495586e-05	0.00471862581271275	0.713003157837853\\
0.6	4.46175251031912e-05	0.009446325184051	0.715846486660142\\
0.6	0.000100583311362513	0.0141829653360114	0.718788838379794\\
0.6	0.000179158434668431	0.018928411755669	0.721886092228356\\
0.6	0.000280470402701511	0.02368252819604	0.725200636220892\\
0.6	0.000404645907256436	0.0284451766772965	0.728799085274723\\
0.6	0.000551810799695644	0.0332162174883389	0.732749123983336\\
0.6	0.000722090066287311	0.037995509188729	0.737115575253328\\
0.6	0.000915607803432999	0.0427829086109896	0.741955893907011\\
0.6	0.0011324871927904	0.047578270863273	0.747315373652289\\
0.6	0.00137285047629673	0.0523814493324038	0.75322242404581\\
0.6	0.00163681893109844	0.057192295687301	0.759684310525652\\
0.6	0.00192451284439304	0.0620106598827802	0.766683748102665\\
0.6	0.00223605148818898	0.0668363901637434	0.774176695511256\\
0.6	0.00257155309398959	0.0716693330697584	0.782091614494493\\
0.6	0.00293113482740716	0.0765093334400313	0.79033034642109\\
0.6	0.00331491276271365	0.0813562344187764	0.79877062752665\\
0.6	0.00372300185733413	0.0862098774609879	0.80727012893036\\
0.6	0.00415551592628969	0.0910701023386145	0.815671782700193\\
0.6	0.00461256761659621	0.0959367471471425	0.82381005350733\\
0.6	0.00509426838162598	0.100809648312589	0.831517746503354\\
0.6	0.00560072845543873	0.105688640598912	0.838632911419453\\
0.6	0.00613205682708918	0.11057355711583	0.845005411376292\\
0.6	0.00668836121491816	0.115464229327074	0.850502769049301\\
0.6	0.00726974804083431	0.120360487059049	0.855014975717657\\
0.6	0.00787632240459385	0.125262158509929	0.858458041096605\\
0.6	0.00850818805808555	0.13016907025918	0.860776163494182\\
0.6	0.00916544737962849	0.135081047277508	0.861942500814225\\
0.6	0.00984820134829017	0.139997912937243	0.861958614648418\\
0.6	0.0105565495182326	0.144919489023162	0.860852735617091\\
0.6	0.0112905899930939	0.149845595743738	0.858677054109652\\
0.6	0.012050419400414	0.154776051742839	0.855504274967456\\
0.6	0.0128361328661109	0.159710674111862	0.851423687967755\\
0.6	0.0136478239890177	0.164649278402306	0.846537000484337\\
0.6	0.0144855848154856	0.169591678638796	0.840954157904249\\
0.6	0.0153495058140643	0.174537687332543	0.834789345403935\\
0.6	0.0162396758502651	0.179487115495259	0.828157325806199\\
0.6	0.0171561821614173	0.184439772653511	0.821170226448694\\
0.6	0.0180991103316243	0.189395466863524	0.813934846725692\\
0.6	0.0190685442668299	0.194354004726437	0.806550519904212\\
0.6	0.0200645661700016	0.199315191403999	0.799107529846303\\
0.6	0.0210872565164405	0.204278830634725	0.791686056496194\\
0.6	0.0221366940292259	0.20924472475049	0.784355603830602\\
0.6	0.023212955654804	0.214212674693577	0.777174850281117\\
0.6	0.0243161165387281	0.219182480034174	0.770191853862645\\
0.6	0.025446250001561	0.224153938988321	0.76344454155446\\
0.6	0.0266034275149466	0.229126848436298	0.756961413910481\\
0.6	0.0277877186778607	0.234101003941464	0.750762400416763\\
0.6	0.0289991911930496	0.239076199769547	0.744859807805784\\
0.6	0.0302379108436654	0.244052228908366	0.739259311518081\\
0.6	0.0315039414701067	0.24902888308801	0.733960949044503\\
0.6	0.0327973449470747	0.254005952801449	0.728960082407262\\
0.6	0.0341181811608519	0.258983227325592	0.724248305111177\\
0.6	0.0354665079868145	0.263960494742775	0.71981427622068\\
0.6	0.0368423812671862	0.268937541962694	0.715644470612631\\
0.6	0.0382458547890421	0.273914154744766	0.711723839838041\\
0.6	0.0396769802625738	0.278890117720924	0.70803638238973\\
0.6	0.0411358072996216	0.283865214418836	0.704565625567166\\
0.6	0.0426223833924861	0.288839227285554	0.701295023640315\\
0.6	0.0441367538930258	0.293811937711588	0.698208278749352\\
0.6	0.0456789619920503	0.298783126055387	0.69528959205535\\
0.6	0.0472490486990192	0.303752571668251	0.692523853197511\\
0.6	0.0488470528220538	0.308720052919643	0.689896776228118\\
0.6	0.0504730109482718	0.313685347222913	0.687394989990282\\
0.6	0.0521269574244539	0.318648231061428	0.685006090465981\\
0.6	0.0538089243380495	0.323608480015096	0.682718662029127\\
0.6	0.0555189414985328	0.328565868787292	0.680522273853469\\
0.6	0.0572570364191156	0.333520171232163	0.678407456997851\\
0.6	0.0590232342988274	0.338471160382329	0.676365666960045\\
0.6	0.0608175580049697	0.34341860847696	0.674389235783155\\
0.6	0.0626400280559541	0.348362286990222	0.672471317134861\\
0.6	0.064490662604533	0.3533019666601	0.670605827172183\\
0.6	0.066369477421429	0.358237417517579	0.668787383459803\\
0.6	0.0682764858793752	0.363168408916186	0.667011243730729\\
0.6	0.0702116989375697	0.368094709561873	0.66527324586365\\
0.6	0.0721751251265572	0.373016087543259	0.663569750098931\\
0.6	0.0741667705335422	0.377932310362199	0.661897584220061\\
0.6	0.0761866387881432	0.382843144964686	0.660253992184443\\
0.6	0.0782347310485954	0.38774835777208	0.658636586490996\\
0.6	0.0803110459884098	0.392647714712651	0.657043304415871\\
0.6	0.0824155797834956	0.397540981253432	0.65547236812651\\
0.6	0.084548326099754	0.402427922432375	0.653922248592746\\
0.6	0.0867092760811502	0.4073083028908	0.65239163314657\\
0.6	0.0888984183382709	0.412181886906127	0.650879396495579\\
0.6	0.0911157389373742	0.417048438424896	0.649384574964845\\
0.6	0.0933612213899392	0.421907721096044	0.647906343724684\\
0.6	0.0956348466427212	0.42675949830445	0.646443996754805\\
0.6	0.0979365930683194	0.431603533204733	0.644996929295962\\
0.6	0.100266436456264	0.436439588755293	0.643564622546846\\
0.6	0.102624350004627	0.441267427752584	0.642146630374429\\
0.6	0.105010304312169	0.446086812865616	0.640742567819433\\
0.6	0.107424267371012	0.450897506670668	0.639352101193731\\
0.6	0.109866204559871	0.455699271686218	0.637974939582443\\
0.6	0.112336078637817	0.460491870408058	0.636610827579862\\
0.6	0.114833849738607	0.465275065344603	0.635259539104383\\
0.6	0.117359475365564	0.470048619052377	0.633920872153155\\
0.6	0.119912910387022	0.474812294171664	0.632594644371964\\
0.6	0.122494107032347	0.479565853462319	0.631280689329641\\
0.6	0.125103014888515	0.484309059839721	0.629978853399058\\
0.6	0.127739580897283	0.489041676410868	0.628690460824559\\
0.605	0	0	0.710202759607373\\
0.605	1.11327767495586e-05	0.00471862581271275	0.713003157837853\\
0.605	4.46175251031912e-05	0.009446325184051	0.715846486660142\\
0.605	0.000100583311362513	0.0141829653360114	0.718788838379794\\
0.605	0.000179158434668431	0.018928411755669	0.721886092228356\\
0.605	0.000280470402701511	0.02368252819604	0.725200636220892\\
0.605	0.000404645907256436	0.0284451766772965	0.728799085274723\\
0.605	0.000551810799695644	0.0332162174883389	0.732749123983336\\
0.605	0.000722090066287311	0.037995509188729	0.737115575253328\\
0.605	0.000915607803432999	0.0427829086109896	0.741955893907011\\
0.605	0.0011324871927904	0.0475782708632729	0.747315373652289\\
0.605	0.00137285047629673	0.0523814493324038	0.753222424045811\\
0.605	0.00163681893109844	0.057192295687301	0.759684310525652\\
0.605	0.00192451284439304	0.0620106598827802	0.766683748102665\\
0.605	0.00223605148818899	0.0668363901637434	0.774176695511256\\
0.605	0.00257155309398959	0.0716693330697584	0.782091614494494\\
0.605	0.00293113482740716	0.0765093334400312	0.79033034642109\\
0.605	0.00331491276271365	0.0813562344187764	0.798770627526651\\
0.605	0.00372300185733414	0.0862098774609879	0.807270128930359\\
0.605	0.00415551592628969	0.0910701023386145	0.815671782700194\\
0.605	0.00461256761659621	0.0959367471471425	0.82381005350733\\
0.605	0.00509426838162598	0.100809648312589	0.831517746503354\\
0.605	0.00560072845543873	0.105688640598912	0.838632911419452\\
0.605	0.00613205682708918	0.11057355711583	0.845005411376294\\
0.605	0.00668836121491816	0.115464229327074	0.850502769049301\\
0.605	0.00726974804083431	0.120360487059049	0.855014975717658\\
0.605	0.00787632240459385	0.125262158509929	0.858458041096604\\
0.605	0.00850818805808555	0.13016907025918	0.860776163494181\\
0.605	0.00916544737962849	0.135081047277508	0.861942500814223\\
0.605	0.00984820134829017	0.139997912937243	0.86195861464842\\
0.605	0.0105565495182326	0.144919489023162	0.860852735617091\\
0.605	0.0112905899930939	0.149845595743738	0.858677054109652\\
0.605	0.012050419400414	0.154776051742839	0.855504274967458\\
0.605	0.0128361328661109	0.159710674111862	0.851423687967754\\
0.605	0.0136478239890177	0.164649278402306	0.84653700048434\\
0.605	0.0144855848154856	0.169591678638796	0.84095415790425\\
0.605	0.0153495058140643	0.174537687332543	0.834789345403934\\
0.605	0.0162396758502651	0.179487115495259	0.8281573258062\\
0.605	0.0171561821614173	0.184439772653511	0.821170226448695\\
0.605	0.0180991103316243	0.189395466863524	0.813934846725692\\
0.605	0.0190685442668299	0.194354004726437	0.806550519904212\\
0.605	0.0200645661700016	0.199315191403999	0.799107529846306\\
0.605	0.0210872565164405	0.204278830634725	0.791686056496196\\
0.605	0.0221366940292259	0.20924472475049	0.7843556038306\\
0.605	0.023212955654804	0.214212674693577	0.777174850281118\\
0.605	0.0243161165387281	0.219182480034174	0.770191853862643\\
0.605	0.025446250001561	0.224153938988321	0.763444541554457\\
0.605	0.0266034275149466	0.229126848436298	0.756961413910482\\
0.605	0.0277877186778607	0.234101003941464	0.750762400416763\\
0.605	0.0289991911930496	0.239076199769547	0.744859807805787\\
0.605	0.0302379108436654	0.244052228908366	0.739259311518079\\
0.605	0.0315039414701067	0.24902888308801	0.733960949044502\\
0.605	0.0327973449470747	0.254005952801449	0.728960082407261\\
0.605	0.0341181811608519	0.258983227325592	0.724248305111178\\
0.605	0.0354665079868145	0.263960494742775	0.719814276220682\\
0.605	0.0368423812671862	0.268937541962694	0.715644470612629\\
0.605	0.0382458547890421	0.273914154744766	0.711723839838041\\
0.605	0.0396769802625738	0.278890117720924	0.70803638238973\\
0.605	0.0411358072996216	0.283865214418836	0.704565625567164\\
0.605	0.0426223833924861	0.288839227285554	0.701295023640317\\
0.605	0.0441367538930258	0.293811937711588	0.698208278749355\\
0.605	0.0456789619920503	0.298783126055387	0.695289592055347\\
0.605	0.0472490486990191	0.303752571668251	0.692523853197512\\
0.605	0.0488470528220538	0.308720052919643	0.689896776228121\\
0.605	0.0504730109482718	0.313685347222913	0.687394989990281\\
0.605	0.0521269574244539	0.318648231061428	0.68500609046598\\
0.605	0.0538089243380495	0.323608480015097	0.682718662029128\\
0.605	0.0555189414985328	0.328565868787292	0.680522273853469\\
0.605	0.0572570364191156	0.333520171232163	0.678407456997849\\
0.605	0.0590232342988274	0.338471160382329	0.676365666960045\\
0.605	0.0608175580049697	0.34341860847696	0.674389235783157\\
0.605	0.0626400280559541	0.348362286990222	0.67247131713486\\
0.605	0.064490662604533	0.3533019666601	0.670605827172182\\
0.605	0.0663694774214291	0.358237417517579	0.668787383459804\\
0.605	0.0682764858793752	0.363168408916186	0.667011243730727\\
0.605	0.0702116989375697	0.368094709561873	0.665273245863649\\
0.605	0.0721751251265572	0.373016087543259	0.663569750098932\\
0.605	0.0741667705335422	0.377932310362199	0.661897584220061\\
0.605	0.0761866387881432	0.382843144964686	0.660253992184443\\
0.605	0.0782347310485954	0.38774835777208	0.658636586490997\\
0.605	0.0803110459884098	0.392647714712651	0.65704330441587\\
0.605	0.0824155797834956	0.397540981253432	0.655472368126509\\
0.605	0.084548326099754	0.402427922432375	0.653922248592746\\
0.605	0.0867092760811502	0.4073083028908	0.652391633146569\\
0.605	0.0888984183382709	0.412181886906127	0.650879396495579\\
0.605	0.0911157389373742	0.417048438424896	0.649384574964845\\
0.605	0.0933612213899392	0.421907721096044	0.647906343724685\\
0.605	0.0956348466427212	0.42675949830445	0.646443996754804\\
0.605	0.0979365930683194	0.431603533204733	0.644996929295963\\
0.605	0.100266436456264	0.436439588755293	0.643564622546847\\
0.605	0.102624350004627	0.441267427752584	0.642146630374429\\
0.605	0.105010304312169	0.446086812865616	0.640742567819434\\
0.605	0.107424267371012	0.450897506670668	0.639352101193731\\
0.605	0.109866204559871	0.455699271686218	0.637974939582442\\
0.605	0.112336078637817	0.460491870408058	0.636610827579862\\
0.605	0.114833849738607	0.465275065344603	0.635259539104383\\
0.605	0.117359475365564	0.470048619052377	0.633920872153155\\
0.605	0.119912910387023	0.474812294171664	0.632594644371965\\
0.605	0.122494107032347	0.479565853462319	0.631280689329641\\
0.605	0.125103014888515	0.484309059839721	0.629978853399058\\
0.605	0.127739580897283	0.489041676410868	0.628690460824558\\
0.61	0	0	0.710202759607373\\
0.61	1.11327767495586e-05	0.00471862581271275	0.713003157837853\\
0.61	4.46175251031912e-05	0.009446325184051	0.715846486660142\\
0.61	0.000100583311362513	0.0141829653360114	0.718788838379794\\
0.61	0.000179158434668431	0.018928411755669	0.721886092228356\\
0.61	0.000280470402701511	0.02368252819604	0.725200636220892\\
0.61	0.000404645907256436	0.0284451766772965	0.728799085274723\\
0.61	0.000551810799695644	0.0332162174883389	0.732749123983336\\
0.61	0.000722090066287311	0.037995509188729	0.737115575253328\\
0.61	0.000915607803432999	0.0427829086109896	0.741955893907011\\
0.61	0.0011324871927904	0.0475782708632729	0.747315373652289\\
0.61	0.00137285047629673	0.0523814493324038	0.75322242404581\\
0.61	0.00163681893109844	0.057192295687301	0.759684310525652\\
0.61	0.00192451284439304	0.0620106598827802	0.766683748102665\\
0.61	0.00223605148818898	0.0668363901637434	0.774176695511256\\
0.61	0.00257155309398959	0.0716693330697584	0.782091614494493\\
0.61	0.00293113482740716	0.0765093334400312	0.79033034642109\\
0.61	0.00331491276271365	0.0813562344187764	0.79877062752665\\
0.61	0.00372300185733414	0.0862098774609879	0.807270128930359\\
0.61	0.00415551592628969	0.0910701023386145	0.815671782700194\\
0.61	0.00461256761659621	0.0959367471471425	0.82381005350733\\
0.61	0.00509426838162598	0.100809648312589	0.831517746503354\\
0.61	0.00560072845543873	0.105688640598912	0.838632911419452\\
0.61	0.00613205682708918	0.11057355711583	0.845005411376293\\
0.61	0.00668836121491816	0.115464229327074	0.850502769049303\\
0.61	0.00726974804083431	0.120360487059049	0.855014975717657\\
0.61	0.00787632240459385	0.125262158509929	0.858458041096608\\
0.61	0.00850818805808555	0.13016907025918	0.860776163494183\\
0.61	0.00916544737962849	0.135081047277508	0.861942500814224\\
0.61	0.00984820134829017	0.139997912937243	0.861958614648422\\
0.61	0.0105565495182326	0.144919489023162	0.860852735617092\\
0.61	0.0112905899930939	0.149845595743738	0.858677054109648\\
0.61	0.012050419400414	0.154776051742839	0.855504274967459\\
0.61	0.0128361328661109	0.159710674111862	0.851423687967755\\
0.61	0.0136478239890177	0.164649278402306	0.846537000484341\\
0.61	0.0144855848154856	0.169591678638796	0.840954157904248\\
0.61	0.0153495058140643	0.174537687332543	0.834789345403934\\
0.61	0.0162396758502651	0.179487115495259	0.8281573258062\\
0.61	0.0171561821614173	0.184439772653511	0.821170226448695\\
0.61	0.0180991103316243	0.189395466863524	0.813934846725693\\
0.61	0.0190685442668299	0.194354004726437	0.806550519904212\\
0.61	0.0200645661700016	0.199315191403999	0.799107529846307\\
0.61	0.0210872565164405	0.204278830634725	0.791686056496192\\
0.61	0.0221366940292259	0.20924472475049	0.784355603830601\\
0.61	0.023212955654804	0.214212674693577	0.777174850281119\\
0.61	0.0243161165387281	0.219182480034174	0.770191853862644\\
0.61	0.025446250001561	0.224153938988321	0.76344454155446\\
0.61	0.0266034275149466	0.229126848436298	0.756961413910484\\
0.61	0.0277877186778607	0.234101003941464	0.750762400416765\\
0.61	0.0289991911930496	0.239076199769547	0.744859807805784\\
0.61	0.0302379108436654	0.244052228908366	0.739259311518077\\
0.61	0.0315039414701067	0.24902888308801	0.733960949044504\\
0.61	0.0327973449470747	0.254005952801449	0.728960082407263\\
0.61	0.0341181811608519	0.258983227325592	0.724248305111178\\
0.61	0.0354665079868145	0.263960494742775	0.71981427622068\\
0.61	0.0368423812671862	0.268937541962694	0.715644470612628\\
0.61	0.0382458547890421	0.273914154744766	0.711723839838041\\
0.61	0.0396769802625738	0.278890117720924	0.708036382389731\\
0.61	0.0411358072996216	0.283865214418836	0.704565625567165\\
0.61	0.0426223833924861	0.288839227285554	0.701295023640317\\
0.61	0.0441367538930258	0.293811937711588	0.698208278749353\\
0.61	0.0456789619920503	0.298783126055387	0.695289592055349\\
0.61	0.0472490486990192	0.303752571668251	0.692523853197511\\
0.61	0.0488470528220537	0.308720052919643	0.689896776228119\\
0.61	0.0504730109482718	0.313685347222913	0.687394989990283\\
0.61	0.0521269574244539	0.318648231061428	0.68500609046598\\
0.61	0.0538089243380495	0.323608480015096	0.682718662029129\\
0.61	0.0555189414985328	0.328565868787292	0.68052227385347\\
0.61	0.0572570364191156	0.333520171232163	0.678407456997849\\
0.61	0.0590232342988274	0.338471160382329	0.676365666960043\\
0.61	0.0608175580049697	0.34341860847696	0.674389235783157\\
0.61	0.0626400280559541	0.348362286990222	0.672471317134861\\
0.61	0.064490662604533	0.3533019666601	0.670605827172181\\
0.61	0.0663694774214291	0.358237417517579	0.668787383459804\\
0.61	0.0682764858793752	0.363168408916186	0.667011243730728\\
0.61	0.0702116989375697	0.368094709561873	0.665273245863649\\
0.61	0.0721751251265572	0.373016087543259	0.663569750098931\\
0.61	0.0741667705335422	0.377932310362199	0.66189758422006\\
0.61	0.0761866387881432	0.382843144964686	0.660253992184443\\
0.61	0.0782347310485954	0.38774835777208	0.658636586490997\\
0.61	0.0803110459884098	0.392647714712651	0.657043304415872\\
0.61	0.0824155797834956	0.397540981253432	0.655472368126509\\
0.61	0.084548326099754	0.402427922432375	0.653922248592746\\
0.61	0.0867092760811502	0.4073083028908	0.65239163314657\\
0.61	0.0888984183382709	0.412181886906127	0.650879396495578\\
0.61	0.0911157389373742	0.417048438424896	0.649384574964845\\
0.61	0.0933612213899392	0.421907721096044	0.647906343724686\\
0.61	0.0956348466427212	0.42675949830445	0.646443996754804\\
0.61	0.0979365930683194	0.431603533204733	0.644996929295961\\
0.61	0.100266436456264	0.436439588755293	0.643564622546847\\
0.61	0.102624350004627	0.441267427752584	0.64214663037443\\
0.61	0.105010304312169	0.446086812865616	0.640742567819433\\
0.61	0.107424267371012	0.450897506670668	0.639352101193731\\
0.61	0.109866204559871	0.455699271686218	0.637974939582444\\
0.61	0.112336078637817	0.460491870408058	0.636610827579862\\
0.61	0.114833849738607	0.465275065344603	0.635259539104383\\
0.61	0.117359475365564	0.470048619052377	0.633920872153154\\
0.61	0.119912910387022	0.474812294171664	0.632594644371965\\
0.61	0.122494107032347	0.479565853462319	0.63128068932964\\
0.61	0.125103014888515	0.484309059839721	0.629978853399058\\
0.61	0.127739580897283	0.489041676410868	0.628690460824559\\
0.615	0	0	0.710202759607373\\
0.615	1.11327767495586e-05	0.00471862581271274	0.713003157837853\\
0.615	4.46175251031912e-05	0.009446325184051	0.715846486660142\\
0.615	0.000100583311362513	0.0141829653360114	0.718788838379794\\
0.615	0.000179158434668431	0.018928411755669	0.721886092228356\\
0.615	0.000280470402701511	0.02368252819604	0.725200636220892\\
0.615	0.000404645907256436	0.0284451766772965	0.728799085274723\\
0.615	0.000551810799695644	0.0332162174883388	0.732749123983336\\
0.615	0.000722090066287311	0.037995509188729	0.737115575253327\\
0.615	0.000915607803432999	0.0427829086109896	0.741955893907011\\
0.615	0.0011324871927904	0.0475782708632729	0.747315373652289\\
0.615	0.00137285047629673	0.0523814493324038	0.753222424045811\\
0.615	0.00163681893109844	0.057192295687301	0.759684310525652\\
0.615	0.00192451284439304	0.0620106598827802	0.766683748102665\\
0.615	0.00223605148818899	0.0668363901637434	0.774176695511257\\
0.615	0.00257155309398959	0.0716693330697584	0.782091614494493\\
0.615	0.00293113482740716	0.0765093334400312	0.79033034642109\\
0.615	0.00331491276271365	0.0813562344187764	0.79877062752665\\
0.615	0.00372300185733414	0.0862098774609879	0.807270128930358\\
0.615	0.00415551592628969	0.0910701023386145	0.815671782700193\\
0.615	0.00461256761659621	0.0959367471471425	0.823810053507331\\
0.615	0.00509426838162598	0.100809648312589	0.831517746503353\\
0.615	0.00560072845543873	0.105688640598912	0.838632911419452\\
0.615	0.00613205682708918	0.11057355711583	0.845005411376292\\
0.615	0.00668836121491816	0.115464229327074	0.850502769049302\\
0.615	0.00726974804083431	0.120360487059049	0.855014975717656\\
0.615	0.00787632240459385	0.125262158509929	0.858458041096606\\
0.615	0.00850818805808555	0.13016907025918	0.860776163494183\\
0.615	0.00916544737962849	0.135081047277508	0.861942500814225\\
0.615	0.00984820134829017	0.139997912937243	0.86195861464842\\
0.615	0.0105565495182326	0.144919489023162	0.860852735617091\\
0.615	0.0112905899930939	0.149845595743738	0.858677054109647\\
0.615	0.012050419400414	0.154776051742839	0.855504274967457\\
0.615	0.0128361328661109	0.159710674111862	0.851423687967754\\
0.615	0.0136478239890177	0.164649278402306	0.846537000484338\\
0.615	0.0144855848154856	0.169591678638796	0.840954157904249\\
0.615	0.0153495058140643	0.174537687332543	0.834789345403934\\
0.615	0.0162396758502651	0.179487115495259	0.8281573258062\\
0.615	0.0171561821614173	0.184439772653511	0.821170226448694\\
0.615	0.0180991103316243	0.189395466863524	0.81393484672569\\
0.615	0.0190685442668299	0.194354004726437	0.806550519904211\\
0.615	0.0200645661700016	0.199315191403999	0.799107529846304\\
0.615	0.0210872565164405	0.204278830634725	0.791686056496193\\
0.615	0.0221366940292259	0.20924472475049	0.784355603830602\\
0.615	0.023212955654804	0.214212674693577	0.777174850281119\\
0.615	0.0243161165387281	0.219182480034174	0.770191853862645\\
0.615	0.025446250001561	0.224153938988321	0.763444541554458\\
0.615	0.0266034275149466	0.229126848436298	0.756961413910482\\
0.615	0.0277877186778607	0.234101003941464	0.750762400416763\\
0.615	0.0289991911930496	0.239076199769547	0.744859807805784\\
0.615	0.0302379108436654	0.244052228908366	0.739259311518081\\
0.615	0.0315039414701067	0.24902888308801	0.733960949044504\\
0.615	0.0327973449470747	0.254005952801449	0.72896008240726\\
0.615	0.0341181811608519	0.258983227325592	0.724248305111177\\
0.615	0.0354665079868145	0.263960494742775	0.71981427622068\\
0.615	0.0368423812671862	0.268937541962694	0.715644470612631\\
0.615	0.0382458547890421	0.273914154744766	0.711723839838043\\
0.615	0.0396769802625738	0.278890117720924	0.708036382389729\\
0.615	0.0411358072996216	0.283865214418836	0.704565625567164\\
0.615	0.0426223833924861	0.288839227285554	0.701295023640316\\
0.615	0.0441367538930258	0.293811937711588	0.698208278749354\\
0.615	0.0456789619920503	0.298783126055387	0.69528959205535\\
0.615	0.0472490486990192	0.303752571668251	0.692523853197512\\
0.615	0.0488470528220538	0.308720052919643	0.689896776228119\\
0.615	0.0504730109482718	0.313685347222913	0.687394989990281\\
0.615	0.0521269574244539	0.318648231061428	0.68500609046598\\
0.615	0.0538089243380495	0.323608480015096	0.682718662029127\\
0.615	0.0555189414985328	0.328565868787292	0.680522273853469\\
0.615	0.0572570364191156	0.333520171232163	0.67840745699785\\
0.615	0.0590232342988274	0.338471160382329	0.676365666960043\\
0.615	0.0608175580049697	0.34341860847696	0.674389235783156\\
0.615	0.0626400280559541	0.348362286990222	0.672471317134862\\
0.615	0.064490662604533	0.3533019666601	0.670605827172182\\
0.615	0.0663694774214291	0.358237417517579	0.668787383459804\\
0.615	0.0682764858793752	0.363168408916186	0.667011243730728\\
0.615	0.0702116989375697	0.368094709561873	0.665273245863649\\
0.615	0.0721751251265572	0.373016087543259	0.663569750098933\\
0.615	0.0741667705335422	0.377932310362199	0.661897584220059\\
0.615	0.0761866387881432	0.382843144964686	0.660253992184441\\
0.615	0.0782347310485954	0.38774835777208	0.658636586490997\\
0.615	0.0803110459884098	0.392647714712651	0.657043304415872\\
0.615	0.0824155797834956	0.397540981253432	0.655472368126509\\
0.615	0.084548326099754	0.402427922432375	0.653922248592744\\
0.615	0.0867092760811502	0.407308302890799	0.65239163314657\\
0.615	0.0888984183382709	0.412181886906127	0.650879396495579\\
0.615	0.0911157389373742	0.417048438424896	0.649384574964843\\
0.615	0.0933612213899392	0.421907721096044	0.647906343724686\\
0.615	0.0956348466427212	0.42675949830445	0.646443996754806\\
0.615	0.0979365930683194	0.431603533204733	0.644996929295961\\
0.615	0.100266436456264	0.436439588755293	0.643564622546847\\
0.615	0.102624350004627	0.441267427752584	0.642146630374429\\
0.615	0.105010304312169	0.446086812865616	0.640742567819433\\
0.615	0.107424267371012	0.450897506670668	0.63935210119373\\
0.615	0.109866204559871	0.455699271686218	0.637974939582443\\
0.615	0.112336078637817	0.460491870408058	0.636610827579863\\
0.615	0.114833849738607	0.465275065344603	0.635259539104384\\
0.615	0.117359475365564	0.470048619052377	0.633920872153155\\
0.615	0.119912910387022	0.474812294171664	0.632594644371964\\
0.615	0.122494107032347	0.479565853462319	0.631280689329641\\
0.615	0.125103014888515	0.484309059839721	0.629978853399059\\
0.615	0.127739580897283	0.489041676410868	0.628690460824559\\
0.62	0	0	0.710202759607373\\
0.62	1.11327767495586e-05	0.00471862581271274	0.713003157837853\\
0.62	4.46175251031912e-05	0.009446325184051	0.715846486660142\\
0.62	0.000100583311362513	0.0141829653360114	0.718788838379794\\
0.62	0.000179158434668431	0.018928411755669	0.721886092228356\\
0.62	0.000280470402701511	0.02368252819604	0.725200636220892\\
0.62	0.000404645907256436	0.0284451766772965	0.728799085274723\\
0.62	0.000551810799695644	0.0332162174883388	0.732749123983336\\
0.62	0.000722090066287311	0.037995509188729	0.737115575253328\\
0.62	0.000915607803432999	0.0427829086109896	0.741955893907011\\
0.62	0.0011324871927904	0.0475782708632729	0.747315373652289\\
0.62	0.00137285047629673	0.0523814493324038	0.75322242404581\\
0.62	0.00163681893109844	0.057192295687301	0.759684310525652\\
0.62	0.00192451284439304	0.0620106598827802	0.766683748102665\\
0.62	0.00223605148818898	0.0668363901637434	0.774176695511256\\
0.62	0.00257155309398959	0.0716693330697584	0.782091614494494\\
0.62	0.00293113482740716	0.0765093334400312	0.79033034642109\\
0.62	0.00331491276271365	0.0813562344187764	0.798770627526651\\
0.62	0.00372300185733414	0.0862098774609879	0.80727012893036\\
0.62	0.00415551592628969	0.0910701023386145	0.815671782700193\\
0.62	0.00461256761659621	0.0959367471471425	0.823810053507331\\
0.62	0.00509426838162598	0.100809648312589	0.831517746503354\\
0.62	0.00560072845543873	0.105688640598912	0.838632911419452\\
0.62	0.00613205682708918	0.11057355711583	0.845005411376292\\
0.62	0.00668836121491816	0.115464229327074	0.850502769049302\\
0.62	0.00726974804083431	0.120360487059049	0.855014975717657\\
0.62	0.00787632240459385	0.125262158509929	0.858458041096605\\
0.62	0.00850818805808555	0.13016907025918	0.860776163494182\\
0.62	0.00916544737962849	0.135081047277508	0.861942500814224\\
0.62	0.00984820134829017	0.139997912937243	0.86195861464842\\
0.62	0.0105565495182326	0.144919489023162	0.860852735617089\\
0.62	0.0112905899930939	0.149845595743738	0.85867705410965\\
0.62	0.012050419400414	0.154776051742839	0.855504274967458\\
0.62	0.0128361328661109	0.159710674111862	0.851423687967753\\
0.62	0.0136478239890177	0.164649278402306	0.846537000484341\\
0.62	0.0144855848154856	0.169591678638796	0.84095415790425\\
0.62	0.0153495058140643	0.174537687332543	0.834789345403934\\
0.62	0.0162396758502651	0.179487115495259	0.828157325806201\\
0.62	0.0171561821614173	0.184439772653511	0.821170226448693\\
0.62	0.0180991103316243	0.189395466863524	0.813934846725691\\
0.62	0.0190685442668299	0.194354004726437	0.806550519904213\\
0.62	0.0200645661700016	0.199315191403999	0.799107529846304\\
0.62	0.0210872565164405	0.204278830634725	0.791686056496193\\
0.62	0.0221366940292259	0.20924472475049	0.784355603830602\\
0.62	0.023212955654804	0.214212674693577	0.77717485028112\\
0.62	0.0243161165387281	0.219182480034174	0.770191853862643\\
0.62	0.025446250001561	0.224153938988321	0.763444541554457\\
0.62	0.0266034275149466	0.229126848436298	0.756961413910483\\
0.62	0.0277877186778607	0.234101003941464	0.750762400416764\\
0.62	0.0289991911930496	0.239076199769547	0.744859807805785\\
0.62	0.0302379108436654	0.244052228908366	0.73925931151808\\
0.62	0.0315039414701067	0.24902888308801	0.733960949044503\\
0.62	0.0327973449470747	0.254005952801449	0.728960082407261\\
0.62	0.0341181811608519	0.258983227325592	0.724248305111177\\
0.62	0.0354665079868145	0.263960494742775	0.719814276220682\\
0.62	0.0368423812671862	0.268937541962694	0.715644470612631\\
0.62	0.0382458547890421	0.273914154744766	0.711723839838041\\
0.62	0.0396769802625738	0.278890117720924	0.70803638238973\\
0.62	0.0411358072996216	0.283865214418836	0.704565625567165\\
0.62	0.0426223833924861	0.288839227285554	0.701295023640315\\
0.62	0.0441367538930258	0.293811937711588	0.698208278749354\\
0.62	0.0456789619920503	0.298783126055387	0.695289592055351\\
0.62	0.0472490486990192	0.303752571668251	0.692523853197512\\
0.62	0.0488470528220538	0.308720052919643	0.689896776228118\\
0.62	0.0504730109482718	0.313685347222913	0.687394989990281\\
0.62	0.0521269574244539	0.318648231061428	0.685006090465981\\
0.62	0.0538089243380495	0.323608480015097	0.682718662029128\\
0.62	0.0555189414985328	0.328565868787292	0.68052227385347\\
0.62	0.0572570364191156	0.333520171232163	0.67840745699785\\
0.62	0.0590232342988274	0.338471160382329	0.676365666960043\\
0.62	0.0608175580049697	0.34341860847696	0.674389235783156\\
0.62	0.0626400280559541	0.348362286990222	0.672471317134861\\
0.62	0.064490662604533	0.3533019666601	0.670605827172182\\
0.62	0.0663694774214291	0.358237417517579	0.668787383459804\\
0.62	0.0682764858793752	0.363168408916186	0.667011243730728\\
0.62	0.0702116989375697	0.368094709561873	0.665273245863649\\
0.62	0.0721751251265572	0.373016087543259	0.663569750098932\\
0.62	0.0741667705335423	0.377932310362199	0.66189758422006\\
0.62	0.0761866387881432	0.382843144964686	0.660253992184442\\
0.62	0.0782347310485954	0.38774835777208	0.658636586490997\\
0.62	0.0803110459884098	0.392647714712651	0.657043304415871\\
0.62	0.0824155797834956	0.397540981253432	0.655472368126509\\
0.62	0.084548326099754	0.402427922432375	0.653922248592746\\
0.62	0.0867092760811502	0.4073083028908	0.65239163314657\\
0.62	0.0888984183382709	0.412181886906127	0.650879396495579\\
0.62	0.0911157389373742	0.417048438424896	0.649384574964844\\
0.62	0.0933612213899392	0.421907721096044	0.647906343724685\\
0.62	0.0956348466427212	0.42675949830445	0.646443996754805\\
0.62	0.0979365930683194	0.431603533204733	0.644996929295961\\
0.62	0.100266436456264	0.436439588755293	0.643564622546847\\
0.62	0.102624350004627	0.441267427752584	0.642146630374429\\
0.62	0.105010304312169	0.446086812865616	0.640742567819432\\
0.62	0.107424267371012	0.450897506670668	0.639352101193731\\
0.62	0.109866204559871	0.455699271686218	0.637974939582442\\
0.62	0.112336078637817	0.460491870408058	0.636610827579862\\
0.62	0.114833849738607	0.465275065344603	0.635259539104383\\
0.62	0.117359475365564	0.470048619052377	0.633920872153156\\
0.62	0.119912910387022	0.474812294171664	0.632594644371965\\
0.62	0.122494107032347	0.479565853462319	0.63128068932964\\
0.62	0.125103014888515	0.484309059839721	0.629978853399059\\
0.62	0.127739580897283	0.489041676410868	0.628690460824561\\
0.625	0	0	0.710202759607373\\
0.625	1.11327767495586e-05	0.00471862581271274	0.713003157837853\\
0.625	4.46175251031912e-05	0.009446325184051	0.715846486660142\\
0.625	0.000100583311362513	0.0141829653360114	0.718788838379794\\
0.625	0.000179158434668431	0.018928411755669	0.721886092228356\\
0.625	0.000280470402701511	0.02368252819604	0.725200636220892\\
0.625	0.000404645907256436	0.0284451766772965	0.728799085274723\\
0.625	0.000551810799695644	0.0332162174883389	0.732749123983336\\
0.625	0.000722090066287311	0.037995509188729	0.737115575253328\\
0.625	0.000915607803432999	0.0427829086109896	0.741955893907011\\
0.625	0.0011324871927904	0.047578270863273	0.747315373652289\\
0.625	0.00137285047629673	0.0523814493324038	0.753222424045811\\
0.625	0.00163681893109844	0.057192295687301	0.759684310525652\\
0.625	0.00192451284439304	0.0620106598827802	0.766683748102665\\
0.625	0.00223605148818899	0.0668363901637434	0.774176695511256\\
0.625	0.00257155309398959	0.0716693330697584	0.782091614494493\\
0.625	0.00293113482740716	0.0765093334400312	0.79033034642109\\
0.625	0.00331491276271365	0.0813562344187764	0.798770627526651\\
0.625	0.00372300185733413	0.0862098774609879	0.80727012893036\\
0.625	0.00415551592628969	0.0910701023386145	0.815671782700193\\
0.625	0.00461256761659621	0.0959367471471425	0.82381005350733\\
0.625	0.00509426838162598	0.100809648312589	0.831517746503354\\
0.625	0.00560072845543873	0.105688640598912	0.838632911419452\\
0.625	0.00613205682708918	0.11057355711583	0.845005411376292\\
0.625	0.00668836121491815	0.115464229327074	0.850502769049302\\
0.625	0.00726974804083431	0.120360487059049	0.855014975717656\\
0.625	0.00787632240459385	0.125262158509929	0.858458041096606\\
0.625	0.00850818805808554	0.13016907025918	0.860776163494182\\
0.625	0.00916544737962849	0.135081047277508	0.861942500814224\\
0.625	0.00984820134829017	0.139997912937243	0.86195861464842\\
0.625	0.0105565495182326	0.144919489023162	0.860852735617091\\
0.625	0.0112905899930939	0.149845595743738	0.858677054109652\\
0.625	0.012050419400414	0.154776051742839	0.85550427496746\\
0.625	0.0128361328661109	0.159710674111862	0.851423687967755\\
0.625	0.0136478239890177	0.164649278402306	0.846537000484343\\
0.625	0.0144855848154856	0.169591678638796	0.840954157904247\\
0.625	0.0153495058140643	0.174537687332543	0.834789345403936\\
0.625	0.0162396758502651	0.179487115495259	0.828157325806201\\
0.625	0.0171561821614173	0.184439772653511	0.821170226448695\\
0.625	0.0180991103316243	0.189395466863524	0.813934846725693\\
0.625	0.0190685442668299	0.194354004726437	0.806550519904213\\
0.625	0.0200645661700016	0.199315191403999	0.799107529846304\\
0.625	0.0210872565164405	0.204278830634725	0.791686056496194\\
0.625	0.0221366940292259	0.20924472475049	0.784355603830605\\
0.625	0.023212955654804	0.214212674693577	0.777174850281119\\
0.625	0.0243161165387281	0.219182480034174	0.770191853862642\\
0.625	0.025446250001561	0.224153938988321	0.763444541554458\\
0.625	0.0266034275149466	0.229126848436298	0.756961413910484\\
0.625	0.0277877186778607	0.234101003941464	0.750762400416764\\
0.625	0.0289991911930496	0.239076199769547	0.744859807805785\\
0.625	0.0302379108436654	0.244052228908366	0.739259311518079\\
0.625	0.0315039414701067	0.24902888308801	0.733960949044504\\
0.625	0.0327973449470747	0.254005952801449	0.728960082407262\\
0.625	0.0341181811608519	0.258983227325592	0.724248305111179\\
0.625	0.0354665079868145	0.263960494742775	0.719814276220682\\
0.625	0.0368423812671862	0.268937541962694	0.715644470612628\\
0.625	0.0382458547890421	0.273914154744766	0.711723839838041\\
0.625	0.0396769802625738	0.278890117720924	0.708036382389731\\
0.625	0.0411358072996216	0.283865214418836	0.704565625567163\\
0.625	0.0426223833924861	0.288839227285554	0.701295023640316\\
0.625	0.0441367538930258	0.293811937711588	0.698208278749354\\
0.625	0.0456789619920503	0.298783126055387	0.69528959205535\\
0.625	0.0472490486990192	0.303752571668251	0.692523853197511\\
0.625	0.0488470528220538	0.308720052919643	0.689896776228119\\
0.625	0.0504730109482718	0.313685347222913	0.687394989990283\\
0.625	0.0521269574244539	0.318648231061428	0.68500609046598\\
0.625	0.0538089243380495	0.323608480015096	0.682718662029127\\
0.625	0.0555189414985328	0.328565868787292	0.68052227385347\\
0.625	0.0572570364191156	0.333520171232163	0.678407456997851\\
0.625	0.0590232342988274	0.338471160382329	0.676365666960044\\
0.625	0.0608175580049697	0.34341860847696	0.674389235783156\\
0.625	0.0626400280559541	0.348362286990222	0.672471317134862\\
0.625	0.064490662604533	0.3533019666601	0.670605827172181\\
0.625	0.0663694774214291	0.358237417517579	0.668787383459804\\
0.625	0.0682764858793752	0.363168408916186	0.667011243730728\\
0.625	0.0702116989375697	0.368094709561873	0.66527324586365\\
0.625	0.0721751251265572	0.373016087543259	0.663569750098933\\
0.625	0.0741667705335422	0.377932310362199	0.66189758422006\\
0.625	0.0761866387881432	0.382843144964686	0.660253992184442\\
0.625	0.0782347310485954	0.38774835777208	0.658636586490998\\
0.625	0.0803110459884098	0.392647714712651	0.657043304415871\\
0.625	0.0824155797834956	0.397540981253432	0.655472368126508\\
0.625	0.084548326099754	0.402427922432375	0.653922248592745\\
0.625	0.0867092760811502	0.407308302890799	0.652391633146571\\
0.625	0.0888984183382709	0.412181886906127	0.65087939649558\\
0.625	0.0911157389373742	0.417048438424896	0.649384574964843\\
0.625	0.0933612213899392	0.421907721096044	0.647906343724685\\
0.625	0.0956348466427212	0.42675949830445	0.646443996754807\\
0.625	0.0979365930683194	0.431603533204733	0.644996929295962\\
0.625	0.100266436456264	0.436439588755293	0.643564622546847\\
0.625	0.102624350004627	0.441267427752584	0.642146630374431\\
0.625	0.105010304312169	0.446086812865616	0.640742567819433\\
0.625	0.107424267371012	0.450897506670668	0.63935210119373\\
0.625	0.109866204559871	0.455699271686218	0.637974939582443\\
0.625	0.112336078637817	0.460491870408058	0.636610827579862\\
0.625	0.114833849738607	0.465275065344603	0.635259539104383\\
0.625	0.117359475365564	0.470048619052377	0.633920872153155\\
0.625	0.119912910387022	0.474812294171664	0.632594644371966\\
0.625	0.122494107032347	0.479565853462319	0.63128068932964\\
0.625	0.125103014888515	0.484309059839721	0.629978853399058\\
0.625	0.127739580897283	0.489041676410868	0.628690460824558\\
0.63	0	0	0.710202759607373\\
0.63	1.11327767495586e-05	0.00471862581271275	0.713003157837853\\
0.63	4.46175251031912e-05	0.009446325184051	0.715846486660142\\
0.63	0.000100583311362513	0.0141829653360114	0.718788838379794\\
0.63	0.000179158434668431	0.018928411755669	0.721886092228357\\
0.63	0.000280470402701511	0.02368252819604	0.725200636220892\\
0.63	0.000404645907256436	0.0284451766772965	0.728799085274723\\
0.63	0.000551810799695644	0.0332162174883388	0.732749123983336\\
0.63	0.000722090066287311	0.037995509188729	0.737115575253328\\
0.63	0.000915607803432999	0.0427829086109896	0.741955893907011\\
0.63	0.0011324871927904	0.0475782708632729	0.747315373652289\\
0.63	0.00137285047629673	0.0523814493324038	0.75322242404581\\
0.63	0.00163681893109844	0.057192295687301	0.759684310525652\\
0.63	0.00192451284439304	0.0620106598827802	0.766683748102665\\
0.63	0.00223605148818898	0.0668363901637434	0.774176695511256\\
0.63	0.00257155309398959	0.0716693330697584	0.782091614494494\\
0.63	0.00293113482740716	0.0765093334400312	0.79033034642109\\
0.63	0.00331491276271365	0.0813562344187764	0.79877062752665\\
0.63	0.00372300185733414	0.0862098774609879	0.80727012893036\\
0.63	0.00415551592628969	0.0910701023386145	0.815671782700193\\
0.63	0.00461256761659621	0.0959367471471424	0.823810053507331\\
0.63	0.00509426838162598	0.100809648312589	0.831517746503353\\
0.63	0.00560072845543873	0.105688640598912	0.838632911419451\\
0.63	0.00613205682708918	0.11057355711583	0.845005411376293\\
0.63	0.00668836121491816	0.115464229327074	0.850502769049302\\
0.63	0.00726974804083431	0.120360487059049	0.855014975717657\\
0.63	0.00787632240459385	0.125262158509929	0.858458041096606\\
0.63	0.00850818805808555	0.13016907025918	0.860776163494182\\
0.63	0.00916544737962849	0.135081047277508	0.861942500814222\\
0.63	0.00984820134829017	0.139997912937243	0.86195861464842\\
0.63	0.0105565495182326	0.144919489023162	0.860852735617091\\
0.63	0.0112905899930939	0.149845595743738	0.858677054109652\\
0.63	0.012050419400414	0.154776051742839	0.855504274967458\\
0.63	0.0128361328661109	0.159710674111862	0.851423687967754\\
0.63	0.0136478239890177	0.164649278402306	0.846537000484341\\
0.63	0.0144855848154856	0.169591678638796	0.840954157904247\\
0.63	0.0153495058140643	0.174537687332543	0.834789345403936\\
0.63	0.0162396758502651	0.179487115495259	0.828157325806199\\
0.63	0.0171561821614173	0.184439772653511	0.821170226448695\\
0.63	0.0180991103316243	0.189395466863524	0.813934846725691\\
0.63	0.0190685442668299	0.194354004726437	0.806550519904211\\
0.63	0.0200645661700016	0.199315191403999	0.799107529846306\\
0.63	0.0210872565164405	0.204278830634725	0.791686056496196\\
0.63	0.0221366940292259	0.20924472475049	0.784355603830602\\
0.63	0.023212955654804	0.214212674693577	0.777174850281116\\
0.63	0.0243161165387281	0.219182480034174	0.770191853862643\\
0.63	0.025446250001561	0.224153938988321	0.763444541554459\\
0.63	0.0266034275149466	0.229126848436298	0.756961413910483\\
0.63	0.0277877186778607	0.234101003941464	0.750762400416764\\
0.63	0.0289991911930496	0.239076199769547	0.744859807805785\\
0.63	0.0302379108436654	0.244052228908366	0.739259311518079\\
0.63	0.0315039414701067	0.24902888308801	0.733960949044504\\
0.63	0.0327973449470747	0.254005952801449	0.72896008240726\\
0.63	0.0341181811608519	0.258983227325592	0.724248305111178\\
0.63	0.0354665079868145	0.263960494742775	0.719814276220681\\
0.63	0.0368423812671862	0.268937541962694	0.715644470612629\\
0.63	0.0382458547890421	0.273914154744766	0.711723839838041\\
0.63	0.0396769802625738	0.278890117720924	0.70803638238973\\
0.63	0.0411358072996216	0.283865214418836	0.704565625567165\\
0.63	0.0426223833924861	0.288839227285554	0.701295023640316\\
0.63	0.0441367538930258	0.293811937711588	0.698208278749354\\
0.63	0.0456789619920503	0.298783126055387	0.69528959205535\\
0.63	0.0472490486990192	0.303752571668251	0.692523853197511\\
0.63	0.0488470528220538	0.308720052919643	0.689896776228119\\
0.63	0.0504730109482718	0.313685347222913	0.687394989990281\\
0.63	0.0521269574244539	0.318648231061428	0.685006090465981\\
0.63	0.0538089243380495	0.323608480015096	0.682718662029128\\
0.63	0.0555189414985328	0.328565868787292	0.680522273853469\\
0.63	0.0572570364191156	0.333520171232163	0.67840745699785\\
0.63	0.0590232342988274	0.338471160382329	0.676365666960044\\
0.63	0.0608175580049697	0.34341860847696	0.674389235783157\\
0.63	0.0626400280559541	0.348362286990222	0.672471317134862\\
0.63	0.064490662604533	0.3533019666601	0.670605827172183\\
0.63	0.066369477421429	0.358237417517579	0.668787383459804\\
0.63	0.0682764858793752	0.363168408916186	0.667011243730727\\
0.63	0.0702116989375697	0.368094709561873	0.665273245863649\\
0.63	0.0721751251265572	0.373016087543259	0.663569750098932\\
0.63	0.0741667705335422	0.377932310362199	0.661897584220061\\
0.63	0.0761866387881432	0.382843144964686	0.660253992184442\\
0.63	0.0782347310485954	0.38774835777208	0.658636586490996\\
0.63	0.0803110459884098	0.392647714712651	0.657043304415872\\
0.63	0.0824155797834956	0.397540981253432	0.655472368126509\\
0.63	0.084548326099754	0.402427922432375	0.653922248592745\\
0.63	0.0867092760811502	0.4073083028908	0.652391633146569\\
0.63	0.0888984183382709	0.412181886906127	0.650879396495578\\
0.63	0.0911157389373742	0.417048438424896	0.649384574964844\\
0.63	0.0933612213899392	0.421907721096044	0.647906343724684\\
0.63	0.0956348466427212	0.42675949830445	0.646443996754804\\
0.63	0.0979365930683194	0.431603533204733	0.644996929295963\\
0.63	0.100266436456264	0.436439588755293	0.643564622546847\\
0.63	0.102624350004627	0.441267427752584	0.64214663037443\\
0.63	0.105010304312169	0.446086812865616	0.640742567819434\\
0.63	0.107424267371012	0.450897506670668	0.639352101193731\\
0.63	0.109866204559871	0.455699271686218	0.637974939582442\\
0.63	0.112336078637817	0.460491870408058	0.636610827579862\\
0.63	0.114833849738607	0.465275065344603	0.635259539104383\\
0.63	0.117359475365564	0.470048619052377	0.633920872153154\\
0.63	0.119912910387023	0.474812294171664	0.632594644371966\\
0.63	0.122494107032347	0.479565853462319	0.631280689329641\\
0.63	0.125103014888515	0.484309059839721	0.629978853399059\\
0.63	0.127739580897283	0.489041676410868	0.62869046082456\\
0.635	0	0	0.710202759607373\\
0.635	1.11327767495586e-05	0.00471862581271275	0.713003157837853\\
0.635	4.46175251031912e-05	0.009446325184051	0.715846486660142\\
0.635	0.000100583311362513	0.0141829653360114	0.718788838379794\\
0.635	0.000179158434668431	0.018928411755669	0.721886092228356\\
0.635	0.000280470402701511	0.02368252819604	0.725200636220892\\
0.635	0.000404645907256436	0.0284451766772965	0.728799085274723\\
0.635	0.000551810799695644	0.0332162174883388	0.732749123983336\\
0.635	0.000722090066287311	0.037995509188729	0.737115575253327\\
0.635	0.000915607803432999	0.0427829086109896	0.741955893907011\\
0.635	0.0011324871927904	0.0475782708632729	0.747315373652289\\
0.635	0.00137285047629673	0.0523814493324038	0.75322242404581\\
0.635	0.00163681893109844	0.057192295687301	0.759684310525652\\
0.635	0.00192451284439304	0.0620106598827802	0.766683748102665\\
0.635	0.00223605148818898	0.0668363901637434	0.774176695511257\\
0.635	0.00257155309398959	0.0716693330697584	0.782091614494494\\
0.635	0.00293113482740716	0.0765093334400312	0.79033034642109\\
0.635	0.00331491276271365	0.0813562344187764	0.79877062752665\\
0.635	0.00372300185733414	0.0862098774609879	0.807270128930359\\
0.635	0.00415551592628969	0.0910701023386145	0.815671782700194\\
0.635	0.00461256761659621	0.0959367471471425	0.823810053507331\\
0.635	0.00509426838162598	0.100809648312589	0.831517746503354\\
0.635	0.00560072845543873	0.105688640598912	0.838632911419452\\
0.635	0.00613205682708918	0.11057355711583	0.845005411376293\\
0.635	0.00668836121491816	0.115464229327074	0.850502769049302\\
0.635	0.00726974804083431	0.120360487059049	0.855014975717657\\
0.635	0.00787632240459385	0.125262158509929	0.858458041096605\\
0.635	0.00850818805808555	0.13016907025918	0.860776163494181\\
0.635	0.00916544737962849	0.135081047277508	0.861942500814224\\
0.635	0.00984820134829017	0.139997912937243	0.86195861464842\\
0.635	0.0105565495182326	0.144919489023162	0.860852735617091\\
0.635	0.0112905899930939	0.149845595743738	0.858677054109649\\
0.635	0.012050419400414	0.154776051742839	0.855504274967456\\
0.635	0.0128361328661109	0.159710674111862	0.851423687967754\\
0.635	0.0136478239890177	0.164649278402306	0.84653700048434\\
0.635	0.0144855848154856	0.169591678638796	0.84095415790425\\
0.635	0.0153495058140643	0.174537687332543	0.834789345403935\\
0.635	0.0162396758502651	0.179487115495259	0.8281573258062\\
0.635	0.0171561821614173	0.184439772653511	0.821170226448694\\
0.635	0.0180991103316243	0.189395466863524	0.813934846725692\\
0.635	0.0190685442668299	0.194354004726437	0.806550519904214\\
0.635	0.0200645661700016	0.199315191403999	0.799107529846306\\
0.635	0.0210872565164405	0.204278830634725	0.791686056496195\\
0.635	0.0221366940292259	0.20924472475049	0.7843556038306\\
0.635	0.023212955654804	0.214212674693577	0.777174850281117\\
0.635	0.0243161165387281	0.219182480034174	0.770191853862644\\
0.635	0.025446250001561	0.224153938988321	0.763444541554458\\
0.635	0.0266034275149466	0.229126848436298	0.756961413910482\\
0.635	0.0277877186778607	0.234101003941464	0.750762400416765\\
0.635	0.0289991911930496	0.239076199769547	0.744859807805785\\
0.635	0.0302379108436654	0.244052228908366	0.739259311518078\\
0.635	0.0315039414701067	0.24902888308801	0.733960949044503\\
0.635	0.0327973449470747	0.254005952801449	0.728960082407261\\
0.635	0.0341181811608519	0.258983227325592	0.724248305111178\\
0.635	0.0354665079868145	0.263960494742775	0.719814276220682\\
0.635	0.0368423812671862	0.268937541962694	0.71564447061263\\
0.635	0.0382458547890421	0.273914154744766	0.71172383983804\\
0.635	0.0396769802625738	0.278890117720924	0.708036382389729\\
0.635	0.0411358072996216	0.283865214418836	0.704565625567166\\
0.635	0.0426223833924861	0.288839227285554	0.701295023640317\\
0.635	0.0441367538930258	0.293811937711588	0.698208278749352\\
0.635	0.0456789619920503	0.298783126055387	0.695289592055349\\
0.635	0.0472490486990192	0.303752571668251	0.692523853197512\\
0.635	0.0488470528220538	0.308720052919643	0.689896776228119\\
0.635	0.0504730109482718	0.313685347222913	0.68739498999028\\
0.635	0.0521269574244539	0.318648231061427	0.685006090465981\\
0.635	0.0538089243380495	0.323608480015097	0.682718662029129\\
0.635	0.0555189414985328	0.328565868787292	0.680522273853469\\
0.635	0.0572570364191156	0.333520171232163	0.67840745699785\\
0.635	0.0590232342988274	0.338471160382329	0.676365666960043\\
0.635	0.0608175580049697	0.34341860847696	0.674389235783156\\
0.635	0.0626400280559541	0.348362286990222	0.672471317134862\\
0.635	0.064490662604533	0.3533019666601	0.670605827172182\\
0.635	0.0663694774214291	0.358237417517579	0.668787383459804\\
0.635	0.0682764858793752	0.363168408916186	0.667011243730728\\
0.635	0.0702116989375697	0.368094709561873	0.665273245863649\\
0.635	0.0721751251265572	0.373016087543259	0.663569750098933\\
0.635	0.0741667705335422	0.377932310362199	0.661897584220061\\
0.635	0.0761866387881432	0.382843144964686	0.660253992184443\\
0.635	0.0782347310485954	0.38774835777208	0.658636586490996\\
0.635	0.0803110459884098	0.392647714712651	0.657043304415871\\
0.635	0.0824155797834956	0.397540981253432	0.655472368126509\\
0.635	0.084548326099754	0.402427922432375	0.653922248592745\\
0.635	0.0867092760811502	0.407308302890799	0.652391633146571\\
0.635	0.0888984183382709	0.412181886906127	0.650879396495578\\
0.635	0.0911157389373742	0.417048438424896	0.649384574964843\\
0.635	0.0933612213899392	0.421907721096044	0.647906343724685\\
0.635	0.0956348466427212	0.42675949830445	0.646443996754805\\
0.635	0.0979365930683194	0.431603533204733	0.644996929295961\\
0.635	0.100266436456264	0.436439588755293	0.643564622546847\\
0.635	0.102624350004627	0.441267427752584	0.64214663037443\\
0.635	0.105010304312169	0.446086812865616	0.640742567819433\\
0.635	0.107424267371012	0.450897506670668	0.63935210119373\\
0.635	0.109866204559871	0.455699271686218	0.637974939582443\\
0.635	0.112336078637817	0.460491870408058	0.636610827579861\\
0.635	0.114833849738607	0.465275065344603	0.635259539104383\\
0.635	0.117359475365564	0.470048619052377	0.633920872153154\\
0.635	0.119912910387022	0.474812294171664	0.632594644371967\\
0.635	0.122494107032347	0.479565853462319	0.631280689329641\\
0.635	0.125103014888515	0.484309059839721	0.629978853399058\\
0.635	0.127739580897283	0.489041676410868	0.628690460824558\\
0.64	0	0	0.710202759607373\\
0.64	1.11327767495586e-05	0.00471862581271275	0.713003157837853\\
0.64	4.46175251031912e-05	0.009446325184051	0.715846486660142\\
0.64	0.000100583311362513	0.0141829653360114	0.718788838379794\\
0.64	0.000179158434668431	0.018928411755669	0.721886092228356\\
0.64	0.000280470402701511	0.02368252819604	0.725200636220892\\
0.64	0.000404645907256436	0.0284451766772965	0.728799085274723\\
0.64	0.000551810799695644	0.0332162174883389	0.732749123983336\\
0.64	0.000722090066287311	0.037995509188729	0.737115575253328\\
0.64	0.000915607803432999	0.0427829086109896	0.741955893907011\\
0.64	0.0011324871927904	0.0475782708632729	0.747315373652289\\
0.64	0.00137285047629673	0.0523814493324038	0.753222424045811\\
0.64	0.00163681893109844	0.057192295687301	0.759684310525652\\
0.64	0.00192451284439304	0.0620106598827802	0.766683748102665\\
0.64	0.00223605148818898	0.0668363901637434	0.774176695511256\\
0.64	0.00257155309398959	0.0716693330697584	0.782091614494494\\
0.64	0.00293113482740716	0.0765093334400312	0.79033034642109\\
0.64	0.00331491276271365	0.0813562344187764	0.79877062752665\\
0.64	0.00372300185733413	0.0862098774609879	0.807270128930359\\
0.64	0.00415551592628969	0.0910701023386145	0.815671782700194\\
0.64	0.00461256761659621	0.0959367471471424	0.82381005350733\\
0.64	0.00509426838162598	0.100809648312589	0.831517746503353\\
0.64	0.00560072845543873	0.105688640598912	0.838632911419452\\
0.64	0.00613205682708918	0.11057355711583	0.845005411376293\\
0.64	0.00668836121491816	0.115464229327074	0.850502769049302\\
0.64	0.00726974804083431	0.120360487059049	0.855014975717657\\
0.64	0.00787632240459385	0.125262158509929	0.858458041096606\\
0.64	0.00850818805808555	0.13016907025918	0.860776163494181\\
0.64	0.00916544737962849	0.135081047277508	0.861942500814227\\
0.64	0.00984820134829017	0.139997912937243	0.86195861464842\\
0.64	0.0105565495182326	0.144919489023162	0.860852735617091\\
0.64	0.0112905899930939	0.149845595743738	0.858677054109652\\
0.64	0.012050419400414	0.154776051742839	0.855504274967456\\
0.64	0.0128361328661109	0.159710674111862	0.851423687967755\\
0.64	0.0136478239890177	0.164649278402306	0.846537000484341\\
0.64	0.0144855848154856	0.169591678638796	0.84095415790425\\
0.64	0.0153495058140643	0.174537687332543	0.834789345403935\\
0.64	0.0162396758502651	0.179487115495259	0.828157325806199\\
0.64	0.0171561821614173	0.184439772653511	0.821170226448693\\
0.64	0.0180991103316243	0.189395466863524	0.813934846725692\\
0.64	0.0190685442668299	0.194354004726437	0.806550519904213\\
0.64	0.0200645661700016	0.199315191403999	0.799107529846306\\
0.64	0.0210872565164405	0.204278830634725	0.791686056496192\\
0.64	0.0221366940292259	0.20924472475049	0.784355603830601\\
0.64	0.023212955654804	0.214212674693577	0.777174850281119\\
0.64	0.0243161165387281	0.219182480034174	0.770191853862644\\
0.64	0.025446250001561	0.224153938988321	0.763444541554458\\
0.64	0.0266034275149466	0.229126848436298	0.756961413910483\\
0.64	0.0277877186778607	0.234101003941464	0.750762400416764\\
0.64	0.0289991911930496	0.239076199769547	0.744859807805784\\
0.64	0.0302379108436654	0.244052228908366	0.739259311518079\\
0.64	0.0315039414701067	0.24902888308801	0.733960949044504\\
0.64	0.0327973449470747	0.254005952801449	0.728960082407259\\
0.64	0.0341181811608519	0.258983227325591	0.724248305111179\\
0.64	0.0354665079868145	0.263960494742775	0.719814276220683\\
0.64	0.0368423812671862	0.268937541962694	0.715644470612629\\
0.64	0.0382458547890421	0.273914154744766	0.711723839838041\\
0.64	0.0396769802625738	0.278890117720924	0.70803638238973\\
0.64	0.0411358072996216	0.283865214418836	0.704565625567165\\
0.64	0.0426223833924861	0.288839227285554	0.701295023640316\\
0.64	0.0441367538930258	0.293811937711588	0.698208278749354\\
0.64	0.0456789619920503	0.298783126055387	0.695289592055348\\
0.64	0.0472490486990192	0.303752571668251	0.69252385319751\\
0.64	0.0488470528220538	0.308720052919643	0.68989677622812\\
0.64	0.0504730109482718	0.313685347222913	0.687394989990282\\
0.64	0.0521269574244539	0.318648231061428	0.68500609046598\\
0.64	0.0538089243380495	0.323608480015096	0.682718662029128\\
0.64	0.0555189414985328	0.328565868787292	0.68052227385347\\
0.64	0.0572570364191156	0.333520171232163	0.67840745699785\\
0.64	0.0590232342988274	0.338471160382329	0.676365666960044\\
0.64	0.0608175580049697	0.34341860847696	0.674389235783156\\
0.64	0.0626400280559541	0.348362286990222	0.672471317134862\\
0.64	0.064490662604533	0.3533019666601	0.670605827172182\\
0.64	0.066369477421429	0.358237417517579	0.668787383459803\\
0.64	0.0682764858793752	0.363168408916186	0.667011243730727\\
0.64	0.0702116989375697	0.368094709561873	0.665273245863649\\
0.64	0.0721751251265572	0.373016087543259	0.663569750098932\\
0.64	0.0741667705335422	0.377932310362199	0.66189758422006\\
0.64	0.0761866387881432	0.382843144964686	0.660253992184443\\
0.64	0.0782347310485954	0.38774835777208	0.658636586490998\\
0.64	0.0803110459884098	0.392647714712651	0.657043304415871\\
0.64	0.0824155797834956	0.397540981253432	0.65547236812651\\
0.64	0.084548326099754	0.402427922432375	0.653922248592746\\
0.64	0.0867092760811502	0.4073083028908	0.652391633146571\\
0.64	0.0888984183382709	0.412181886906127	0.650879396495578\\
0.64	0.0911157389373742	0.417048438424896	0.649384574964844\\
0.64	0.0933612213899392	0.421907721096044	0.647906343724685\\
0.64	0.0956348466427212	0.42675949830445	0.646443996754805\\
0.64	0.0979365930683194	0.431603533204733	0.644996929295962\\
0.64	0.100266436456264	0.436439588755293	0.643564622546847\\
0.64	0.102624350004627	0.441267427752585	0.64214663037443\\
0.64	0.105010304312169	0.446086812865616	0.640742567819433\\
0.64	0.107424267371012	0.450897506670668	0.639352101193729\\
0.64	0.109866204559871	0.455699271686218	0.637974939582443\\
0.64	0.112336078637817	0.460491870408058	0.636610827579862\\
0.64	0.114833849738607	0.465275065344603	0.635259539104383\\
0.64	0.117359475365564	0.470048619052377	0.633920872153154\\
0.64	0.119912910387023	0.474812294171664	0.632594644371965\\
0.64	0.122494107032347	0.479565853462319	0.631280689329642\\
0.64	0.125103014888515	0.484309059839721	0.629978853399059\\
0.64	0.127739580897283	0.489041676410868	0.628690460824559\\
0.645	0	0	0.710202759607373\\
0.645	1.11327767495586e-05	0.00471862581271275	0.713003157837853\\
0.645	4.46175251031912e-05	0.009446325184051	0.715846486660142\\
0.645	0.000100583311362513	0.0141829653360114	0.718788838379794\\
0.645	0.000179158434668431	0.018928411755669	0.721886092228356\\
0.645	0.000280470402701511	0.02368252819604	0.725200636220892\\
0.645	0.000404645907256436	0.0284451766772965	0.728799085274723\\
0.645	0.000551810799695644	0.0332162174883388	0.732749123983336\\
0.645	0.000722090066287311	0.037995509188729	0.737115575253328\\
0.645	0.000915607803432999	0.0427829086109896	0.741955893907011\\
0.645	0.0011324871927904	0.047578270863273	0.747315373652289\\
0.645	0.00137285047629673	0.0523814493324038	0.75322242404581\\
0.645	0.00163681893109844	0.057192295687301	0.759684310525652\\
0.645	0.00192451284439304	0.0620106598827802	0.766683748102665\\
0.645	0.00223605148818898	0.0668363901637434	0.774176695511256\\
0.645	0.00257155309398959	0.0716693330697584	0.782091614494493\\
0.645	0.00293113482740716	0.0765093334400312	0.79033034642109\\
0.645	0.00331491276271365	0.0813562344187764	0.79877062752665\\
0.645	0.00372300185733414	0.0862098774609879	0.807270128930359\\
0.645	0.00415551592628969	0.0910701023386145	0.815671782700193\\
0.645	0.00461256761659621	0.0959367471471424	0.823810053507331\\
0.645	0.00509426838162598	0.100809648312589	0.831517746503354\\
0.645	0.00560072845543873	0.105688640598912	0.838632911419451\\
0.645	0.00613205682708918	0.11057355711583	0.845005411376292\\
0.645	0.00668836121491816	0.115464229327074	0.850502769049302\\
0.645	0.00726974804083431	0.120360487059049	0.855014975717657\\
0.645	0.00787632240459385	0.125262158509929	0.858458041096607\\
0.645	0.00850818805808555	0.13016907025918	0.860776163494183\\
0.645	0.00916544737962849	0.135081047277508	0.861942500814225\\
0.645	0.00984820134829017	0.139997912937243	0.86195861464842\\
0.645	0.0105565495182326	0.144919489023162	0.86085273561709\\
0.645	0.0112905899930939	0.149845595743738	0.85867705410965\\
0.645	0.012050419400414	0.154776051742839	0.855504274967455\\
0.645	0.0128361328661109	0.159710674111862	0.851423687967754\\
0.645	0.0136478239890177	0.164649278402306	0.84653700048434\\
0.645	0.0144855848154856	0.169591678638796	0.840954157904248\\
0.645	0.0153495058140643	0.174537687332543	0.834789345403934\\
0.645	0.0162396758502651	0.179487115495259	0.8281573258062\\
0.645	0.0171561821614173	0.184439772653511	0.821170226448695\\
0.645	0.0180991103316243	0.189395466863524	0.813934846725693\\
0.645	0.0190685442668299	0.194354004726437	0.806550519904212\\
0.645	0.0200645661700016	0.199315191403999	0.799107529846303\\
0.645	0.0210872565164405	0.204278830634725	0.791686056496195\\
0.645	0.0221366940292259	0.20924472475049	0.784355603830605\\
0.645	0.023212955654804	0.214212674693577	0.777174850281116\\
0.645	0.0243161165387281	0.219182480034174	0.770191853862643\\
0.645	0.025446250001561	0.224153938988321	0.763444541554457\\
0.645	0.0266034275149466	0.229126848436298	0.756961413910482\\
0.645	0.0277877186778607	0.234101003941464	0.750762400416763\\
0.645	0.0289991911930496	0.239076199769547	0.744859807805784\\
0.645	0.0302379108436654	0.244052228908366	0.73925931151808\\
0.645	0.0315039414701067	0.24902888308801	0.733960949044504\\
0.645	0.0327973449470747	0.254005952801449	0.728960082407261\\
0.645	0.0341181811608519	0.258983227325592	0.724248305111178\\
0.645	0.0354665079868145	0.263960494742775	0.71981427622068\\
0.645	0.0368423812671862	0.268937541962694	0.71564447061263\\
0.645	0.0382458547890421	0.273914154744766	0.711723839838042\\
0.645	0.0396769802625738	0.278890117720924	0.708036382389731\\
0.645	0.0411358072996216	0.283865214418836	0.704565625567166\\
0.645	0.0426223833924861	0.288839227285554	0.701295023640315\\
0.645	0.0441367538930258	0.293811937711588	0.698208278749352\\
0.645	0.0456789619920503	0.298783126055387	0.69528959205535\\
0.645	0.0472490486990192	0.303752571668251	0.692523853197512\\
0.645	0.0488470528220538	0.308720052919643	0.689896776228119\\
0.645	0.0504730109482718	0.313685347222913	0.687394989990281\\
0.645	0.0521269574244539	0.318648231061428	0.685006090465981\\
0.645	0.0538089243380495	0.323608480015096	0.682718662029128\\
0.645	0.0555189414985328	0.328565868787292	0.68052227385347\\
0.645	0.0572570364191156	0.333520171232163	0.678407456997848\\
0.645	0.0590232342988274	0.338471160382329	0.676365666960044\\
0.645	0.0608175580049697	0.34341860847696	0.674389235783157\\
0.645	0.0626400280559541	0.348362286990222	0.672471317134862\\
0.645	0.064490662604533	0.3533019666601	0.670605827172183\\
0.645	0.0663694774214291	0.35823741751758	0.668787383459804\\
0.645	0.0682764858793752	0.363168408916186	0.667011243730728\\
0.645	0.0702116989375697	0.368094709561873	0.665273245863649\\
0.645	0.0721751251265572	0.373016087543259	0.663569750098932\\
0.645	0.0741667705335422	0.377932310362199	0.661897584220059\\
0.645	0.0761866387881432	0.382843144964686	0.660253992184442\\
0.645	0.0782347310485954	0.38774835777208	0.658636586490997\\
0.645	0.0803110459884098	0.392647714712651	0.657043304415871\\
0.645	0.0824155797834956	0.397540981253432	0.655472368126509\\
0.645	0.084548326099754	0.402427922432375	0.653922248592745\\
0.645	0.0867092760811502	0.407308302890799	0.65239163314657\\
0.645	0.0888984183382709	0.412181886906127	0.65087939649558\\
0.645	0.0911157389373742	0.417048438424896	0.649384574964845\\
0.645	0.0933612213899392	0.421907721096044	0.647906343724686\\
0.645	0.0956348466427212	0.42675949830445	0.646443996754804\\
0.645	0.0979365930683194	0.431603533204733	0.64499692929596\\
0.645	0.100266436456264	0.436439588755293	0.643564622546847\\
0.645	0.102624350004627	0.441267427752584	0.642146630374431\\
0.645	0.105010304312169	0.446086812865616	0.640742567819433\\
0.645	0.107424267371012	0.450897506670668	0.639352101193731\\
0.645	0.109866204559871	0.455699271686218	0.637974939582444\\
0.645	0.112336078637817	0.460491870408058	0.636610827579862\\
0.645	0.114833849738607	0.465275065344603	0.635259539104382\\
0.645	0.117359475365564	0.470048619052377	0.633920872153154\\
0.645	0.119912910387022	0.474812294171664	0.632594644371964\\
0.645	0.122494107032347	0.479565853462319	0.631280689329641\\
0.645	0.125103014888515	0.484309059839721	0.629978853399059\\
0.645	0.127739580897283	0.489041676410868	0.628690460824563\\
0.65	0	0	0.710202759607373\\
0.65	1.11327767495586e-05	0.00471862581271275	0.713003157837853\\
0.65	4.46175251031912e-05	0.009446325184051	0.715846486660142\\
0.65	0.000100583311362513	0.0141829653360114	0.718788838379794\\
0.65	0.000179158434668431	0.018928411755669	0.721886092228356\\
0.65	0.000280470402701511	0.02368252819604	0.725200636220892\\
0.65	0.000404645907256436	0.0284451766772965	0.728799085274723\\
0.65	0.000551810799695644	0.0332162174883389	0.732749123983336\\
0.65	0.000722090066287311	0.037995509188729	0.737115575253328\\
0.65	0.000915607803432999	0.0427829086109896	0.741955893907011\\
0.65	0.0011324871927904	0.047578270863273	0.747315373652289\\
0.65	0.00137285047629673	0.0523814493324038	0.753222424045811\\
0.65	0.00163681893109844	0.057192295687301	0.759684310525652\\
0.65	0.00192451284439304	0.0620106598827802	0.766683748102665\\
0.65	0.00223605148818898	0.0668363901637434	0.774176695511257\\
0.65	0.00257155309398959	0.0716693330697584	0.782091614494494\\
0.65	0.00293113482740716	0.0765093334400312	0.79033034642109\\
0.65	0.00331491276271365	0.0813562344187764	0.79877062752665\\
0.65	0.00372300185733413	0.0862098774609879	0.807270128930359\\
0.65	0.00415551592628969	0.0910701023386145	0.815671782700193\\
0.65	0.00461256761659621	0.0959367471471424	0.823810053507331\\
0.65	0.00509426838162598	0.100809648312589	0.831517746503354\\
0.65	0.00560072845543873	0.105688640598912	0.838632911419451\\
0.65	0.00613205682708918	0.11057355711583	0.845005411376293\\
0.65	0.00668836121491816	0.115464229327074	0.850502769049301\\
0.65	0.00726974804083431	0.120360487059049	0.855014975717657\\
0.65	0.00787632240459385	0.125262158509929	0.858458041096605\\
0.65	0.00850818805808555	0.13016907025918	0.860776163494184\\
0.65	0.00916544737962849	0.135081047277508	0.861942500814224\\
0.65	0.00984820134829017	0.139997912937243	0.86195861464842\\
0.65	0.0105565495182326	0.144919489023162	0.860852735617091\\
0.65	0.0112905899930939	0.149845595743738	0.858677054109648\\
0.65	0.012050419400414	0.154776051742839	0.855504274967459\\
0.65	0.0128361328661109	0.159710674111862	0.851423687967754\\
0.65	0.0136478239890177	0.164649278402306	0.846537000484341\\
0.65	0.0144855848154856	0.169591678638796	0.840954157904247\\
0.65	0.0153495058140643	0.174537687332543	0.834789345403935\\
0.65	0.0162396758502651	0.179487115495259	0.828157325806201\\
0.65	0.0171561821614173	0.184439772653511	0.821170226448695\\
0.65	0.0180991103316243	0.189395466863524	0.813934846725692\\
0.65	0.0190685442668299	0.194354004726437	0.806550519904211\\
0.65	0.0200645661700016	0.199315191403999	0.799107529846306\\
0.65	0.0210872565164405	0.204278830634725	0.791686056496197\\
0.65	0.0221366940292259	0.20924472475049	0.7843556038306\\
0.65	0.023212955654804	0.214212674693577	0.777174850281117\\
0.65	0.0243161165387281	0.219182480034174	0.770191853862643\\
0.65	0.025446250001561	0.224153938988321	0.763444541554458\\
0.65	0.0266034275149466	0.229126848436298	0.756961413910484\\
0.65	0.0277877186778607	0.234101003941464	0.750762400416764\\
0.65	0.0289991911930496	0.239076199769547	0.744859807805786\\
0.65	0.0302379108436654	0.244052228908366	0.73925931151808\\
0.65	0.0315039414701067	0.24902888308801	0.733960949044503\\
0.65	0.0327973449470747	0.254005952801449	0.728960082407261\\
0.65	0.0341181811608519	0.258983227325592	0.724248305111177\\
0.65	0.0354665079868145	0.263960494742775	0.719814276220679\\
0.65	0.0368423812671862	0.268937541962694	0.715644470612631\\
0.65	0.0382458547890421	0.273914154744766	0.711723839838044\\
0.65	0.0396769802625738	0.278890117720924	0.708036382389731\\
0.65	0.0411358072996216	0.283865214418836	0.704565625567163\\
0.65	0.0426223833924861	0.288839227285554	0.701295023640316\\
0.65	0.0441367538930258	0.293811937711588	0.698208278749354\\
0.65	0.0456789619920503	0.298783126055387	0.695289592055349\\
0.65	0.0472490486990192	0.303752571668251	0.692523853197512\\
0.65	0.0488470528220538	0.308720052919643	0.68989677622812\\
0.65	0.0504730109482718	0.313685347222913	0.68739498999028\\
0.65	0.0521269574244539	0.318648231061428	0.685006090465981\\
0.65	0.0538089243380495	0.323608480015097	0.682718662029128\\
0.65	0.0555189414985328	0.328565868787292	0.68052227385347\\
0.65	0.0572570364191156	0.333520171232163	0.678407456997851\\
0.65	0.0590232342988274	0.338471160382329	0.676365666960044\\
0.65	0.0608175580049697	0.34341860847696	0.674389235783156\\
0.65	0.0626400280559541	0.348362286990222	0.672471317134862\\
0.65	0.064490662604533	0.3533019666601	0.670605827172183\\
0.65	0.0663694774214291	0.358237417517579	0.668787383459804\\
0.65	0.0682764858793752	0.363168408916186	0.667011243730727\\
0.65	0.0702116989375697	0.368094709561873	0.665273245863649\\
0.65	0.0721751251265572	0.373016087543259	0.663569750098934\\
0.65	0.0741667705335422	0.377932310362199	0.66189758422006\\
0.65	0.0761866387881432	0.382843144964686	0.660253992184441\\
0.65	0.0782347310485954	0.38774835777208	0.658636586490998\\
0.65	0.0803110459884098	0.392647714712651	0.657043304415871\\
0.65	0.0824155797834956	0.397540981253432	0.655472368126509\\
0.65	0.084548326099754	0.402427922432375	0.653922248592745\\
0.65	0.0867092760811502	0.4073083028908	0.652391633146569\\
0.65	0.0888984183382709	0.412181886906127	0.650879396495578\\
0.65	0.0911157389373742	0.417048438424896	0.649384574964843\\
0.65	0.0933612213899392	0.421907721096044	0.647906343724686\\
0.65	0.0956348466427212	0.42675949830445	0.646443996754807\\
0.65	0.0979365930683194	0.431603533204733	0.64499692929596\\
0.65	0.100266436456264	0.436439588755293	0.643564622546845\\
0.65	0.102624350004627	0.441267427752584	0.64214663037443\\
0.65	0.105010304312169	0.446086812865616	0.640742567819433\\
0.65	0.107424267371012	0.450897506670668	0.63935210119373\\
0.65	0.109866204559871	0.455699271686218	0.637974939582443\\
0.65	0.112336078637817	0.460491870408058	0.636610827579863\\
0.65	0.114833849738607	0.465275065344603	0.635259539104383\\
0.65	0.117359475365564	0.470048619052377	0.633920872153155\\
0.65	0.119912910387023	0.474812294171664	0.632594644371964\\
0.65	0.122494107032347	0.479565853462319	0.631280689329642\\
0.65	0.125103014888515	0.484309059839721	0.629978853399058\\
0.65	0.127739580897283	0.489041676410868	0.628690460824559\\
0.655	0	0	0.710202759607373\\
0.655	1.11327767495586e-05	0.00471862581271275	0.713003157837853\\
0.655	4.46175251031912e-05	0.009446325184051	0.715846486660142\\
0.655	0.000100583311362513	0.0141829653360114	0.718788838379794\\
0.655	0.000179158434668431	0.018928411755669	0.721886092228357\\
0.655	0.000280470402701511	0.02368252819604	0.725200636220892\\
0.655	0.000404645907256436	0.0284451766772965	0.728799085274723\\
0.655	0.000551810799695644	0.0332162174883388	0.732749123983336\\
0.655	0.000722090066287311	0.037995509188729	0.737115575253327\\
0.655	0.000915607803432999	0.0427829086109896	0.741955893907011\\
0.655	0.0011324871927904	0.0475782708632729	0.747315373652289\\
0.655	0.00137285047629673	0.0523814493324038	0.753222424045811\\
0.655	0.00163681893109844	0.057192295687301	0.759684310525652\\
0.655	0.00192451284439304	0.0620106598827802	0.766683748102665\\
0.655	0.00223605148818899	0.0668363901637434	0.774176695511256\\
0.655	0.00257155309398959	0.0716693330697584	0.782091614494494\\
0.655	0.00293113482740716	0.0765093334400312	0.79033034642109\\
0.655	0.00331491276271365	0.0813562344187764	0.79877062752665\\
0.655	0.00372300185733413	0.0862098774609879	0.807270128930359\\
0.655	0.00415551592628969	0.0910701023386145	0.815671782700193\\
0.655	0.00461256761659621	0.0959367471471424	0.82381005350733\\
0.655	0.00509426838162598	0.100809648312589	0.831517746503354\\
0.655	0.00560072845543873	0.105688640598912	0.838632911419452\\
0.655	0.00613205682708918	0.11057355711583	0.845005411376293\\
0.655	0.00668836121491815	0.115464229327074	0.850502769049302\\
0.655	0.00726974804083431	0.120360487059049	0.855014975717656\\
0.655	0.00787632240459385	0.125262158509929	0.858458041096605\\
0.655	0.00850818805808554	0.13016907025918	0.860776163494181\\
0.655	0.00916544737962849	0.135081047277508	0.861942500814225\\
0.655	0.00984820134829017	0.139997912937243	0.861958614648422\\
0.655	0.0105565495182326	0.144919489023162	0.860852735617091\\
0.655	0.0112905899930939	0.149845595743738	0.858677054109652\\
0.655	0.012050419400414	0.154776051742839	0.855504274967458\\
0.655	0.0128361328661109	0.159710674111862	0.851423687967755\\
0.655	0.0136478239890177	0.164649278402306	0.846537000484338\\
0.655	0.0144855848154856	0.169591678638796	0.840954157904247\\
0.655	0.0153495058140643	0.174537687332543	0.834789345403935\\
0.655	0.0162396758502651	0.179487115495259	0.828157325806199\\
0.655	0.0171561821614173	0.184439772653511	0.821170226448695\\
0.655	0.0180991103316243	0.189395466863524	0.81393484672569\\
0.655	0.0190685442668299	0.194354004726437	0.806550519904213\\
0.655	0.0200645661700016	0.199315191403999	0.799107529846307\\
0.655	0.0210872565164405	0.204278830634725	0.791686056496195\\
0.655	0.0221366940292259	0.20924472475049	0.7843556038306\\
0.655	0.023212955654804	0.214212674693577	0.777174850281118\\
0.655	0.0243161165387281	0.219182480034174	0.770191853862643\\
0.655	0.025446250001561	0.224153938988321	0.76344454155446\\
0.655	0.0266034275149466	0.229126848436298	0.756961413910483\\
0.655	0.0277877186778607	0.234101003941464	0.750762400416765\\
0.655	0.0289991911930496	0.239076199769547	0.744859807805785\\
0.655	0.0302379108436654	0.244052228908366	0.739259311518078\\
0.655	0.0315039414701067	0.24902888308801	0.733960949044504\\
0.655	0.0327973449470747	0.254005952801449	0.728960082407262\\
0.655	0.0341181811608519	0.258983227325592	0.724248305111179\\
0.655	0.0354665079868145	0.263960494742775	0.719814276220681\\
0.655	0.0368423812671862	0.268937541962694	0.715644470612629\\
0.655	0.0382458547890421	0.273914154744766	0.711723839838041\\
0.655	0.0396769802625738	0.278890117720924	0.708036382389729\\
0.655	0.0411358072996216	0.283865214418836	0.704565625567166\\
0.655	0.0426223833924861	0.288839227285554	0.701295023640317\\
0.655	0.0441367538930258	0.293811937711588	0.698208278749353\\
0.655	0.0456789619920503	0.298783126055387	0.695289592055349\\
0.655	0.0472490486990192	0.303752571668251	0.692523853197511\\
0.655	0.0488470528220538	0.308720052919643	0.68989677622812\\
0.655	0.0504730109482718	0.313685347222913	0.68739498999028\\
0.655	0.0521269574244539	0.318648231061427	0.68500609046598\\
0.655	0.0538089243380495	0.323608480015096	0.682718662029128\\
0.655	0.0555189414985328	0.328565868787292	0.680522273853468\\
0.655	0.0572570364191156	0.333520171232163	0.67840745699785\\
0.655	0.0590232342988274	0.338471160382329	0.676365666960045\\
0.655	0.0608175580049697	0.34341860847696	0.674389235783157\\
0.655	0.0626400280559541	0.348362286990222	0.672471317134862\\
0.655	0.064490662604533	0.3533019666601	0.670605827172182\\
0.655	0.0663694774214291	0.358237417517579	0.668787383459804\\
0.655	0.0682764858793752	0.363168408916186	0.667011243730729\\
0.655	0.0702116989375697	0.368094709561873	0.665273245863649\\
0.655	0.0721751251265572	0.373016087543259	0.663569750098932\\
0.655	0.0741667705335422	0.377932310362199	0.661897584220061\\
0.655	0.0761866387881432	0.382843144964686	0.660253992184441\\
0.655	0.0782347310485954	0.38774835777208	0.658636586490997\\
0.655	0.0803110459884098	0.392647714712651	0.657043304415872\\
0.655	0.0824155797834956	0.397540981253432	0.655472368126509\\
0.655	0.084548326099754	0.402427922432375	0.653922248592746\\
0.655	0.0867092760811502	0.4073083028908	0.65239163314657\\
0.655	0.0888984183382709	0.412181886906127	0.650879396495579\\
0.655	0.0911157389373742	0.417048438424897	0.649384574964843\\
0.655	0.0933612213899392	0.421907721096044	0.647906343724684\\
0.655	0.0956348466427212	0.42675949830445	0.646443996754807\\
0.655	0.0979365930683194	0.431603533204733	0.644996929295962\\
0.655	0.100266436456264	0.436439588755293	0.643564622546846\\
0.655	0.102624350004627	0.441267427752585	0.64214663037443\\
0.655	0.105010304312169	0.446086812865616	0.640742567819433\\
0.655	0.107424267371012	0.450897506670668	0.63935210119373\\
0.655	0.109866204559871	0.455699271686218	0.637974939582442\\
0.655	0.112336078637817	0.460491870408058	0.636610827579862\\
0.655	0.114833849738607	0.465275065344603	0.635259539104383\\
0.655	0.117359475365564	0.470048619052377	0.633920872153154\\
0.655	0.119912910387022	0.474812294171664	0.632594644371964\\
0.655	0.122494107032347	0.479565853462319	0.631280689329641\\
0.655	0.125103014888515	0.484309059839721	0.629978853399059\\
0.655	0.127739580897283	0.489041676410868	0.628690460824557\\
0.66	0	0	0.710202759607373\\
0.66	1.11327767495586e-05	0.00471862581271275	0.713003157837853\\
0.66	4.46175251031912e-05	0.009446325184051	0.715846486660142\\
0.66	0.000100583311362513	0.0141829653360114	0.718788838379794\\
0.66	0.000179158434668431	0.018928411755669	0.721886092228356\\
0.66	0.000280470402701511	0.02368252819604	0.725200636220892\\
0.66	0.000404645907256436	0.0284451766772965	0.728799085274723\\
0.66	0.000551810799695644	0.0332162174883388	0.732749123983336\\
0.66	0.000722090066287311	0.037995509188729	0.737115575253328\\
0.66	0.000915607803432999	0.0427829086109896	0.741955893907011\\
0.66	0.0011324871927904	0.0475782708632729	0.747315373652289\\
0.66	0.00137285047629673	0.0523814493324038	0.75322242404581\\
0.66	0.00163681893109844	0.057192295687301	0.759684310525652\\
0.66	0.00192451284439304	0.0620106598827802	0.766683748102665\\
0.66	0.00223605148818898	0.0668363901637434	0.774176695511257\\
0.66	0.00257155309398959	0.0716693330697584	0.782091614494494\\
0.66	0.00293113482740716	0.0765093334400312	0.79033034642109\\
0.66	0.00331491276271365	0.0813562344187764	0.79877062752665\\
0.66	0.00372300185733413	0.0862098774609879	0.807270128930359\\
0.66	0.00415551592628969	0.0910701023386145	0.815671782700194\\
0.66	0.00461256761659621	0.0959367471471425	0.823810053507331\\
0.66	0.00509426838162598	0.100809648312589	0.831517746503354\\
0.66	0.00560072845543873	0.105688640598912	0.838632911419452\\
0.66	0.00613205682708918	0.11057355711583	0.845005411376293\\
0.66	0.00668836121491816	0.115464229327074	0.850502769049302\\
0.66	0.00726974804083431	0.120360487059049	0.855014975717656\\
0.66	0.00787632240459385	0.125262158509929	0.858458041096606\\
0.66	0.00850818805808555	0.13016907025918	0.860776163494181\\
0.66	0.00916544737962849	0.135081047277508	0.861942500814223\\
0.66	0.00984820134829017	0.139997912937243	0.861958614648422\\
0.66	0.0105565495182326	0.144919489023162	0.86085273561709\\
0.66	0.0112905899930939	0.149845595743738	0.858677054109649\\
0.66	0.012050419400414	0.154776051742839	0.855504274967457\\
0.66	0.0128361328661109	0.159710674111862	0.851423687967753\\
0.66	0.0136478239890177	0.164649278402306	0.84653700048434\\
0.66	0.0144855848154856	0.169591678638796	0.84095415790425\\
0.66	0.0153495058140643	0.174537687332543	0.834789345403933\\
0.66	0.0162396758502651	0.179487115495259	0.828157325806201\\
0.66	0.0171561821614173	0.184439772653511	0.821170226448694\\
0.66	0.0180991103316243	0.189395466863524	0.813934846725691\\
0.66	0.0190685442668299	0.194354004726437	0.806550519904214\\
0.66	0.0200645661700016	0.199315191403999	0.799107529846305\\
0.66	0.0210872565164405	0.204278830634725	0.791686056496194\\
0.66	0.0221366940292259	0.20924472475049	0.7843556038306\\
0.66	0.023212955654804	0.214212674693577	0.777174850281116\\
0.66	0.0243161165387281	0.219182480034174	0.770191853862643\\
0.66	0.025446250001561	0.224153938988321	0.763444541554457\\
0.66	0.0266034275149466	0.229126848436298	0.756961413910483\\
0.66	0.0277877186778607	0.234101003941464	0.750762400416764\\
0.66	0.0289991911930496	0.239076199769547	0.744859807805783\\
0.66	0.0302379108436654	0.244052228908366	0.739259311518079\\
0.66	0.0315039414701067	0.24902888308801	0.733960949044504\\
0.66	0.0327973449470747	0.254005952801449	0.72896008240726\\
0.66	0.0341181811608519	0.258983227325592	0.724248305111176\\
0.66	0.0354665079868146	0.263960494742775	0.71981427622068\\
0.66	0.0368423812671862	0.268937541962694	0.71564447061263\\
0.66	0.0382458547890421	0.273914154744766	0.711723839838042\\
0.66	0.0396769802625738	0.278890117720924	0.70803638238973\\
0.66	0.0411358072996216	0.283865214418836	0.704565625567164\\
0.66	0.0426223833924861	0.288839227285554	0.701295023640316\\
0.66	0.0441367538930258	0.293811937711588	0.698208278749354\\
0.66	0.0456789619920503	0.298783126055387	0.695289592055349\\
0.66	0.0472490486990192	0.303752571668251	0.692523853197512\\
0.66	0.0488470528220538	0.308720052919643	0.689896776228119\\
0.66	0.0504730109482718	0.313685347222913	0.687394989990281\\
0.66	0.0521269574244539	0.318648231061428	0.68500609046598\\
0.66	0.0538089243380495	0.323608480015096	0.682718662029128\\
0.66	0.0555189414985328	0.328565868787292	0.68052227385347\\
0.66	0.0572570364191156	0.333520171232163	0.678407456997849\\
0.66	0.0590232342988274	0.338471160382329	0.676365666960043\\
0.66	0.0608175580049697	0.34341860847696	0.674389235783156\\
0.66	0.0626400280559541	0.348362286990222	0.672471317134861\\
0.66	0.064490662604533	0.3533019666601	0.670605827172183\\
0.66	0.0663694774214291	0.358237417517579	0.668787383459804\\
0.66	0.0682764858793752	0.363168408916186	0.667011243730728\\
0.66	0.0702116989375697	0.368094709561873	0.665273245863649\\
0.66	0.0721751251265572	0.373016087543259	0.663569750098932\\
0.66	0.0741667705335422	0.377932310362199	0.661897584220061\\
0.66	0.0761866387881432	0.382843144964686	0.660253992184442\\
0.66	0.0782347310485954	0.38774835777208	0.658636586490996\\
0.66	0.0803110459884098	0.392647714712651	0.657043304415871\\
0.66	0.0824155797834956	0.397540981253432	0.655472368126509\\
0.66	0.084548326099754	0.402427922432375	0.653922248592745\\
0.66	0.0867092760811502	0.4073083028908	0.65239163314657\\
0.66	0.0888984183382709	0.412181886906127	0.650879396495578\\
0.66	0.0911157389373742	0.417048438424896	0.649384574964845\\
0.66	0.0933612213899392	0.421907721096044	0.647906343724685\\
0.66	0.0956348466427212	0.42675949830445	0.646443996754806\\
0.66	0.0979365930683194	0.431603533204733	0.644996929295962\\
0.66	0.100266436456264	0.436439588755293	0.643564622546847\\
0.66	0.102624350004627	0.441267427752585	0.64214663037443\\
0.66	0.105010304312169	0.446086812865616	0.640742567819433\\
0.66	0.107424267371012	0.450897506670668	0.639352101193731\\
0.66	0.109866204559871	0.455699271686218	0.637974939582442\\
0.66	0.112336078637817	0.460491870408058	0.636610827579861\\
0.66	0.114833849738607	0.465275065344603	0.635259539104383\\
0.66	0.117359475365564	0.470048619052377	0.633920872153155\\
0.66	0.119912910387022	0.474812294171664	0.632594644371966\\
0.66	0.122494107032347	0.479565853462319	0.63128068932964\\
0.66	0.125103014888515	0.484309059839721	0.62997885339906\\
0.66	0.127739580897283	0.489041676410868	0.62869046082456\\
0.665	0	0	0.710202759607373\\
0.665	1.11327767495586e-05	0.00471862581271274	0.713003157837853\\
0.665	4.46175251031912e-05	0.009446325184051	0.715846486660142\\
0.665	0.000100583311362513	0.0141829653360114	0.718788838379794\\
0.665	0.000179158434668431	0.018928411755669	0.721886092228357\\
0.665	0.000280470402701511	0.02368252819604	0.725200636220892\\
0.665	0.000404645907256436	0.0284451766772965	0.728799085274723\\
0.665	0.000551810799695644	0.0332162174883389	0.732749123983336\\
0.665	0.000722090066287311	0.037995509188729	0.737115575253328\\
0.665	0.000915607803432999	0.0427829086109896	0.741955893907011\\
0.665	0.0011324871927904	0.0475782708632729	0.747315373652289\\
0.665	0.00137285047629673	0.0523814493324038	0.753222424045811\\
0.665	0.00163681893109844	0.057192295687301	0.759684310525652\\
0.665	0.00192451284439304	0.0620106598827802	0.766683748102665\\
0.665	0.00223605148818899	0.0668363901637434	0.774176695511256\\
0.665	0.00257155309398959	0.0716693330697584	0.782091614494494\\
0.665	0.00293113482740716	0.0765093334400312	0.79033034642109\\
0.665	0.00331491276271365	0.0813562344187764	0.79877062752665\\
0.665	0.00372300185733414	0.0862098774609879	0.807270128930359\\
0.665	0.00415551592628969	0.0910701023386145	0.815671782700194\\
0.665	0.00461256761659621	0.0959367471471425	0.82381005350733\\
0.665	0.00509426838162598	0.100809648312589	0.831517746503354\\
0.665	0.00560072845543873	0.105688640598912	0.838632911419451\\
0.665	0.00613205682708918	0.11057355711583	0.845005411376293\\
0.665	0.00668836121491816	0.115464229327074	0.850502769049302\\
0.665	0.00726974804083431	0.120360487059049	0.855014975717658\\
0.665	0.00787632240459385	0.125262158509929	0.858458041096605\\
0.665	0.00850818805808554	0.13016907025918	0.860776163494182\\
0.665	0.00916544737962849	0.135081047277508	0.861942500814224\\
0.665	0.00984820134829017	0.139997912937243	0.861958614648419\\
0.665	0.0105565495182326	0.144919489023162	0.86085273561709\\
0.665	0.0112905899930939	0.149845595743738	0.85867705410965\\
0.665	0.012050419400414	0.154776051742839	0.855504274967458\\
0.665	0.0128361328661109	0.159710674111862	0.851423687967754\\
0.665	0.0136478239890177	0.164649278402306	0.84653700048434\\
0.665	0.0144855848154856	0.169591678638796	0.840954157904248\\
0.665	0.0153495058140643	0.174537687332543	0.834789345403935\\
0.665	0.0162396758502651	0.179487115495259	0.828157325806201\\
0.665	0.0171561821614173	0.184439772653511	0.821170226448694\\
0.665	0.0180991103316243	0.189395466863524	0.813934846725692\\
0.665	0.0190685442668299	0.194354004726437	0.806550519904212\\
0.665	0.0200645661700016	0.199315191403999	0.799107529846304\\
0.665	0.0210872565164405	0.204278830634725	0.791686056496194\\
0.665	0.0221366940292259	0.20924472475049	0.784355603830599\\
0.665	0.023212955654804	0.214212674693577	0.777174850281117\\
0.665	0.0243161165387281	0.219182480034174	0.770191853862644\\
0.665	0.025446250001561	0.224153938988321	0.763444541554458\\
0.665	0.0266034275149466	0.229126848436298	0.756961413910482\\
0.665	0.0277877186778607	0.234101003941464	0.750762400416764\\
0.665	0.0289991911930496	0.239076199769547	0.744859807805785\\
0.665	0.0302379108436654	0.244052228908366	0.73925931151808\\
0.665	0.0315039414701067	0.24902888308801	0.733960949044504\\
0.665	0.0327973449470747	0.254005952801449	0.728960082407259\\
0.665	0.0341181811608519	0.258983227325591	0.724248305111178\\
0.665	0.0354665079868145	0.263960494742775	0.719814276220681\\
0.665	0.0368423812671862	0.268937541962694	0.715644470612629\\
0.665	0.0382458547890421	0.273914154744766	0.711723839838042\\
0.665	0.0396769802625738	0.278890117720924	0.70803638238973\\
0.665	0.0411358072996216	0.283865214418836	0.704565625567164\\
0.665	0.0426223833924861	0.288839227285554	0.701295023640316\\
0.665	0.0441367538930258	0.293811937711588	0.698208278749353\\
0.665	0.0456789619920503	0.298783126055387	0.695289592055349\\
0.665	0.0472490486990192	0.303752571668251	0.692523853197512\\
0.665	0.0488470528220538	0.308720052919643	0.689896776228119\\
0.665	0.0504730109482718	0.313685347222913	0.687394989990282\\
0.665	0.0521269574244539	0.318648231061428	0.685006090465982\\
0.665	0.0538089243380495	0.323608480015097	0.682718662029128\\
0.665	0.0555189414985328	0.328565868787292	0.680522273853468\\
0.665	0.0572570364191156	0.333520171232163	0.67840745699785\\
0.665	0.0590232342988274	0.338471160382329	0.676365666960044\\
0.665	0.0608175580049697	0.34341860847696	0.674389235783157\\
0.665	0.0626400280559541	0.348362286990222	0.672471317134861\\
0.665	0.064490662604533	0.3533019666601	0.670605827172182\\
0.665	0.066369477421429	0.358237417517579	0.668787383459804\\
0.665	0.0682764858793752	0.363168408916186	0.667011243730727\\
0.665	0.0702116989375697	0.368094709561873	0.66527324586365\\
0.665	0.0721751251265572	0.373016087543259	0.663569750098933\\
0.665	0.0741667705335422	0.377932310362199	0.66189758422006\\
0.665	0.0761866387881432	0.382843144964686	0.660253992184442\\
0.665	0.0782347310485954	0.38774835777208	0.658636586490997\\
0.665	0.0803110459884098	0.392647714712651	0.657043304415871\\
0.665	0.0824155797834956	0.397540981253432	0.65547236812651\\
0.665	0.084548326099754	0.402427922432375	0.653922248592745\\
0.665	0.0867092760811502	0.4073083028908	0.65239163314657\\
0.665	0.0888984183382709	0.412181886906127	0.650879396495578\\
0.665	0.0911157389373742	0.417048438424897	0.649384574964843\\
0.665	0.0933612213899392	0.421907721096044	0.647906343724685\\
0.665	0.0956348466427212	0.42675949830445	0.646443996754806\\
0.665	0.0979365930683194	0.431603533204733	0.644996929295962\\
0.665	0.100266436456264	0.436439588755293	0.643564622546845\\
0.665	0.102624350004627	0.441267427752584	0.64214663037443\\
0.665	0.105010304312169	0.446086812865616	0.640742567819434\\
0.665	0.107424267371012	0.450897506670668	0.639352101193731\\
0.665	0.109866204559871	0.455699271686218	0.637974939582443\\
0.665	0.112336078637817	0.460491870408058	0.636610827579862\\
0.665	0.114833849738607	0.465275065344603	0.635259539104382\\
0.665	0.117359475365564	0.470048619052377	0.633920872153154\\
0.665	0.119912910387022	0.474812294171664	0.632594644371965\\
0.665	0.122494107032347	0.479565853462319	0.631280689329641\\
0.665	0.125103014888515	0.484309059839721	0.62997885339906\\
0.665	0.127739580897283	0.489041676410868	0.628690460824563\\
0.67	0	0	0.710202759607373\\
0.67	1.11327767495586e-05	0.00471862581271274	0.713003157837853\\
0.67	4.46175251031912e-05	0.009446325184051	0.715846486660142\\
0.67	0.000100583311362513	0.0141829653360114	0.718788838379794\\
0.67	0.000179158434668431	0.018928411755669	0.721886092228356\\
0.67	0.000280470402701511	0.02368252819604	0.725200636220892\\
0.67	0.000404645907256436	0.0284451766772965	0.728799085274723\\
0.67	0.000551810799695644	0.0332162174883388	0.732749123983336\\
0.67	0.000722090066287311	0.037995509188729	0.737115575253327\\
0.67	0.000915607803432999	0.0427829086109896	0.741955893907011\\
0.67	0.0011324871927904	0.0475782708632729	0.747315373652289\\
0.67	0.00137285047629673	0.0523814493324038	0.75322242404581\\
0.67	0.00163681893109844	0.057192295687301	0.759684310525652\\
0.67	0.00192451284439304	0.0620106598827802	0.766683748102665\\
0.67	0.00223605148818898	0.0668363901637434	0.774176695511256\\
0.67	0.00257155309398959	0.0716693330697584	0.782091614494493\\
0.67	0.00293113482740716	0.0765093334400312	0.79033034642109\\
0.67	0.00331491276271365	0.0813562344187764	0.79877062752665\\
0.67	0.00372300185733413	0.0862098774609879	0.807270128930358\\
0.67	0.00415551592628969	0.0910701023386145	0.815671782700194\\
0.67	0.00461256761659621	0.0959367471471425	0.82381005350733\\
0.67	0.00509426838162598	0.100809648312589	0.831517746503354\\
0.67	0.00560072845543873	0.105688640598912	0.838632911419452\\
0.67	0.00613205682708918	0.11057355711583	0.845005411376292\\
0.67	0.00668836121491816	0.115464229327074	0.850502769049301\\
0.67	0.00726974804083431	0.120360487059049	0.855014975717657\\
0.67	0.00787632240459385	0.125262158509929	0.858458041096607\\
0.67	0.00850818805808554	0.13016907025918	0.860776163494182\\
0.67	0.00916544737962849	0.135081047277508	0.861942500814224\\
0.67	0.00984820134829017	0.139997912937243	0.86195861464842\\
0.67	0.0105565495182326	0.144919489023162	0.86085273561709\\
0.67	0.0112905899930939	0.149845595743738	0.858677054109652\\
0.67	0.012050419400414	0.154776051742839	0.855504274967457\\
0.67	0.0128361328661109	0.159710674111862	0.851423687967754\\
0.67	0.0136478239890177	0.164649278402306	0.84653700048434\\
0.67	0.0144855848154856	0.169591678638796	0.840954157904248\\
0.67	0.0153495058140643	0.174537687332543	0.834789345403935\\
0.67	0.0162396758502651	0.179487115495259	0.828157325806201\\
0.67	0.0171561821614173	0.184439772653511	0.821170226448693\\
0.67	0.0180991103316243	0.189395466863524	0.813934846725691\\
0.67	0.0190685442668299	0.194354004726437	0.806550519904213\\
0.67	0.0200645661700016	0.199315191403999	0.799107529846304\\
0.67	0.0210872565164405	0.204278830634725	0.791686056496193\\
0.67	0.0221366940292259	0.20924472475049	0.784355603830603\\
0.67	0.023212955654804	0.214212674693577	0.77717485028112\\
0.67	0.0243161165387281	0.219182480034174	0.770191853862644\\
0.67	0.025446250001561	0.224153938988321	0.763444541554458\\
0.67	0.0266034275149466	0.229126848436298	0.756961413910483\\
0.67	0.0277877186778607	0.234101003941464	0.750762400416766\\
0.67	0.0289991911930496	0.239076199769547	0.744859807805786\\
0.67	0.0302379108436654	0.244052228908366	0.73925931151808\\
0.67	0.0315039414701067	0.24902888308801	0.733960949044502\\
0.67	0.0327973449470747	0.254005952801449	0.728960082407261\\
0.67	0.0341181811608519	0.258983227325592	0.724248305111178\\
0.67	0.0354665079868145	0.263960494742775	0.719814276220681\\
0.67	0.0368423812671862	0.268937541962694	0.71564447061263\\
0.67	0.0382458547890421	0.273914154744766	0.711723839838042\\
0.67	0.0396769802625738	0.278890117720924	0.70803638238973\\
0.67	0.0411358072996216	0.283865214418836	0.704565625567164\\
0.67	0.0426223833924861	0.288839227285554	0.701295023640316\\
0.67	0.0441367538930258	0.293811937711588	0.698208278749355\\
0.67	0.0456789619920503	0.298783126055387	0.69528959205535\\
0.67	0.0472490486990192	0.303752571668251	0.692523853197511\\
0.67	0.0488470528220538	0.308720052919643	0.689896776228118\\
0.67	0.0504730109482718	0.313685347222913	0.687394989990281\\
0.67	0.0521269574244539	0.318648231061428	0.685006090465982\\
0.67	0.0538089243380495	0.323608480015097	0.682718662029128\\
0.67	0.0555189414985328	0.328565868787292	0.680522273853468\\
0.67	0.0572570364191156	0.333520171232163	0.67840745699785\\
0.67	0.0590232342988274	0.338471160382329	0.676365666960043\\
0.67	0.0608175580049697	0.34341860847696	0.674389235783156\\
0.67	0.0626400280559541	0.348362286990222	0.672471317134863\\
0.67	0.064490662604533	0.3533019666601	0.670605827172183\\
0.67	0.0663694774214291	0.35823741751758	0.668787383459803\\
0.67	0.0682764858793752	0.363168408916186	0.667011243730728\\
0.67	0.0702116989375697	0.368094709561873	0.665273245863649\\
0.67	0.0721751251265572	0.373016087543259	0.663569750098933\\
0.67	0.0741667705335422	0.377932310362199	0.661897584220059\\
0.67	0.0761866387881432	0.382843144964686	0.660253992184441\\
0.67	0.0782347310485954	0.38774835777208	0.658636586490997\\
0.67	0.0803110459884098	0.392647714712651	0.657043304415871\\
0.67	0.0824155797834956	0.397540981253432	0.655472368126509\\
0.67	0.084548326099754	0.402427922432375	0.653922248592746\\
0.67	0.0867092760811502	0.407308302890799	0.65239163314657\\
0.67	0.0888984183382709	0.412181886906127	0.65087939649558\\
0.67	0.0911157389373742	0.417048438424896	0.649384574964843\\
0.67	0.0933612213899392	0.421907721096044	0.647906343724684\\
0.67	0.0956348466427212	0.42675949830445	0.646443996754806\\
0.67	0.0979365930683194	0.431603533204733	0.644996929295962\\
0.67	0.100266436456264	0.436439588755293	0.643564622546846\\
0.67	0.102624350004627	0.441267427752584	0.642146630374428\\
0.67	0.105010304312169	0.446086812865616	0.640742567819433\\
0.67	0.107424267371012	0.450897506670668	0.639352101193731\\
0.67	0.109866204559871	0.455699271686218	0.637974939582442\\
0.67	0.112336078637817	0.460491870408058	0.636610827579862\\
0.67	0.114833849738607	0.465275065344603	0.635259539104383\\
0.67	0.117359475365564	0.470048619052377	0.633920872153155\\
0.67	0.119912910387023	0.474812294171664	0.632594644371965\\
0.67	0.122494107032347	0.479565853462319	0.631280689329641\\
0.67	0.125103014888515	0.484309059839721	0.629978853399058\\
0.67	0.127739580897283	0.489041676410868	0.62869046082456\\
0.675	0	0	0.710202759607373\\
0.675	1.11327767495586e-05	0.00471862581271275	0.713003157837853\\
0.675	4.46175251031912e-05	0.009446325184051	0.715846486660142\\
0.675	0.000100583311362513	0.0141829653360114	0.718788838379794\\
0.675	0.000179158434668431	0.018928411755669	0.721886092228356\\
0.675	0.000280470402701511	0.02368252819604	0.725200636220892\\
0.675	0.000404645907256436	0.0284451766772965	0.728799085274723\\
0.675	0.000551810799695644	0.0332162174883389	0.732749123983336\\
0.675	0.000722090066287311	0.037995509188729	0.737115575253328\\
0.675	0.000915607803432999	0.0427829086109896	0.741955893907011\\
0.675	0.0011324871927904	0.0475782708632729	0.747315373652289\\
0.675	0.00137285047629673	0.0523814493324038	0.753222424045811\\
0.675	0.00163681893109844	0.057192295687301	0.759684310525652\\
0.675	0.00192451284439304	0.0620106598827802	0.766683748102665\\
0.675	0.00223605148818899	0.0668363901637434	0.774176695511256\\
0.675	0.00257155309398959	0.0716693330697584	0.782091614494493\\
0.675	0.00293113482740716	0.0765093334400312	0.79033034642109\\
0.675	0.00331491276271365	0.0813562344187764	0.798770627526651\\
0.675	0.00372300185733413	0.0862098774609879	0.807270128930359\\
0.675	0.00415551592628969	0.0910701023386145	0.815671782700193\\
0.675	0.00461256761659621	0.0959367471471425	0.82381005350733\\
0.675	0.00509426838162598	0.100809648312589	0.831517746503354\\
0.675	0.00560072845543873	0.105688640598912	0.838632911419452\\
0.675	0.00613205682708918	0.11057355711583	0.845005411376292\\
0.675	0.00668836121491816	0.115464229327074	0.850502769049302\\
0.675	0.00726974804083431	0.120360487059049	0.855014975717655\\
0.675	0.00787632240459385	0.125262158509929	0.858458041096607\\
0.675	0.00850818805808554	0.13016907025918	0.860776163494183\\
0.675	0.00916544737962849	0.135081047277508	0.861942500814224\\
0.675	0.00984820134829017	0.139997912937243	0.86195861464842\\
0.675	0.0105565495182326	0.144919489023162	0.860852735617091\\
0.675	0.0112905899930939	0.149845595743738	0.858677054109652\\
0.675	0.012050419400414	0.154776051742839	0.855504274967457\\
0.675	0.0128361328661109	0.159710674111862	0.851423687967756\\
0.675	0.0136478239890177	0.164649278402306	0.846537000484342\\
0.675	0.0144855848154856	0.169591678638796	0.840954157904248\\
0.675	0.0153495058140643	0.174537687332543	0.834789345403935\\
0.675	0.0162396758502651	0.179487115495259	0.8281573258062\\
0.675	0.0171561821614173	0.184439772653511	0.821170226448693\\
0.675	0.0180991103316243	0.189395466863524	0.813934846725694\\
0.675	0.0190685442668299	0.194354004726437	0.806550519904212\\
0.675	0.0200645661700016	0.199315191403999	0.799107529846304\\
0.675	0.0210872565164405	0.204278830634725	0.791686056496195\\
0.675	0.0221366940292259	0.20924472475049	0.784355603830603\\
0.675	0.023212955654804	0.214212674693577	0.777174850281117\\
0.675	0.0243161165387281	0.219182480034174	0.770191853862643\\
0.675	0.025446250001561	0.224153938988321	0.763444541554458\\
0.675	0.0266034275149466	0.229126848436298	0.756961413910482\\
0.675	0.0277877186778607	0.234101003941464	0.750762400416764\\
0.675	0.0289991911930496	0.239076199769547	0.744859807805785\\
0.675	0.0302379108436654	0.244052228908366	0.73925931151808\\
0.675	0.0315039414701067	0.24902888308801	0.733960949044503\\
0.675	0.0327973449470747	0.254005952801449	0.728960082407259\\
0.675	0.0341181811608519	0.258983227325591	0.724248305111178\\
0.675	0.0354665079868145	0.263960494742775	0.719814276220682\\
0.675	0.0368423812671862	0.268937541962694	0.715644470612629\\
0.675	0.0382458547890421	0.273914154744766	0.711723839838042\\
0.675	0.0396769802625738	0.278890117720924	0.70803638238973\\
0.675	0.0411358072996216	0.283865214418836	0.704565625567164\\
0.675	0.0426223833924861	0.288839227285554	0.701295023640316\\
0.675	0.0441367538930258	0.293811937711588	0.698208278749354\\
0.675	0.0456789619920503	0.298783126055387	0.69528959205535\\
0.675	0.0472490486990192	0.303752571668251	0.692523853197513\\
0.675	0.0488470528220538	0.308720052919643	0.689896776228119\\
0.675	0.0504730109482718	0.313685347222913	0.687394989990279\\
0.675	0.0521269574244539	0.318648231061427	0.68500609046598\\
0.675	0.0538089243380495	0.323608480015096	0.682718662029129\\
0.675	0.0555189414985328	0.328565868787292	0.680522273853469\\
0.675	0.0572570364191156	0.333520171232163	0.67840745699785\\
0.675	0.0590232342988274	0.338471160382329	0.676365666960044\\
0.675	0.0608175580049697	0.34341860847696	0.674389235783157\\
0.675	0.0626400280559541	0.348362286990222	0.672471317134862\\
0.675	0.064490662604533	0.3533019666601	0.670605827172182\\
0.675	0.0663694774214291	0.358237417517579	0.668787383459804\\
0.675	0.0682764858793752	0.363168408916186	0.667011243730728\\
0.675	0.0702116989375697	0.368094709561873	0.66527324586365\\
0.675	0.0721751251265572	0.373016087543259	0.663569750098933\\
0.675	0.0741667705335422	0.377932310362199	0.66189758422006\\
0.675	0.0761866387881432	0.382843144964686	0.660253992184441\\
0.675	0.0782347310485954	0.38774835777208	0.658636586490997\\
0.675	0.0803110459884098	0.392647714712651	0.657043304415872\\
0.675	0.0824155797834956	0.397540981253432	0.655472368126508\\
0.675	0.084548326099754	0.402427922432375	0.653922248592745\\
0.675	0.0867092760811502	0.407308302890799	0.65239163314657\\
0.675	0.0888984183382709	0.412181886906127	0.650879396495579\\
0.675	0.0911157389373742	0.417048438424896	0.649384574964844\\
0.675	0.0933612213899392	0.421907721096044	0.647906343724686\\
0.675	0.0956348466427212	0.42675949830445	0.646443996754805\\
0.675	0.0979365930683194	0.431603533204733	0.644996929295962\\
0.675	0.100266436456264	0.436439588755293	0.643564622546846\\
0.675	0.102624350004627	0.441267427752584	0.642146630374429\\
0.675	0.105010304312169	0.446086812865616	0.640742567819432\\
0.675	0.107424267371012	0.450897506670668	0.639352101193731\\
0.675	0.109866204559871	0.455699271686218	0.637974939582443\\
0.675	0.112336078637817	0.460491870408058	0.636610827579861\\
0.675	0.114833849738607	0.465275065344603	0.635259539104383\\
0.675	0.117359475365564	0.470048619052377	0.633920872153155\\
0.675	0.119912910387022	0.474812294171664	0.632594644371964\\
0.675	0.122494107032347	0.479565853462319	0.631280689329641\\
0.675	0.125103014888515	0.484309059839721	0.62997885339906\\
0.675	0.127739580897283	0.489041676410868	0.628690460824557\\
0.68	0	0	0.710202759607373\\
0.68	1.11327767495586e-05	0.00471862581271275	0.713003157837853\\
0.68	4.46175251031912e-05	0.009446325184051	0.715846486660142\\
0.68	0.000100583311362513	0.0141829653360114	0.718788838379794\\
0.68	0.000179158434668431	0.018928411755669	0.721886092228356\\
0.68	0.000280470402701511	0.02368252819604	0.725200636220892\\
0.68	0.000404645907256436	0.0284451766772965	0.728799085274723\\
0.68	0.000551810799695644	0.0332162174883388	0.732749123983336\\
0.68	0.000722090066287311	0.037995509188729	0.737115575253327\\
0.68	0.000915607803432999	0.0427829086109896	0.741955893907011\\
0.68	0.0011324871927904	0.047578270863273	0.747315373652289\\
0.68	0.00137285047629673	0.0523814493324038	0.753222424045811\\
0.68	0.00163681893109844	0.057192295687301	0.759684310525652\\
0.68	0.00192451284439304	0.0620106598827802	0.766683748102665\\
0.68	0.00223605148818898	0.0668363901637434	0.774176695511256\\
0.68	0.00257155309398959	0.0716693330697584	0.782091614494493\\
0.68	0.00293113482740716	0.0765093334400312	0.79033034642109\\
0.68	0.00331491276271365	0.0813562344187764	0.798770627526651\\
0.68	0.00372300185733413	0.0862098774609879	0.80727012893036\\
0.68	0.00415551592628969	0.0910701023386145	0.815671782700194\\
0.68	0.00461256761659621	0.0959367471471425	0.82381005350733\\
0.68	0.00509426838162598	0.100809648312589	0.831517746503354\\
0.68	0.00560072845543873	0.105688640598912	0.838632911419452\\
0.68	0.00613205682708918	0.11057355711583	0.845005411376293\\
0.68	0.00668836121491816	0.115464229327074	0.850502769049303\\
0.68	0.00726974804083431	0.120360487059049	0.855014975717656\\
0.68	0.00787632240459385	0.125262158509929	0.858458041096605\\
0.68	0.00850818805808554	0.13016907025918	0.860776163494183\\
0.68	0.00916544737962849	0.135081047277508	0.861942500814225\\
0.68	0.00984820134829017	0.139997912937243	0.86195861464842\\
0.68	0.0105565495182326	0.144919489023162	0.860852735617092\\
0.68	0.0112905899930939	0.149845595743738	0.858677054109651\\
0.68	0.012050419400414	0.154776051742839	0.855504274967457\\
0.68	0.0128361328661109	0.159710674111862	0.851423687967754\\
0.68	0.0136478239890177	0.164649278402306	0.846537000484339\\
0.68	0.0144855848154856	0.169591678638796	0.840954157904248\\
0.68	0.0153495058140643	0.174537687332543	0.834789345403935\\
0.68	0.0162396758502651	0.179487115495259	0.828157325806201\\
0.68	0.0171561821614173	0.184439772653511	0.821170226448695\\
0.68	0.0180991103316243	0.189395466863524	0.813934846725691\\
0.68	0.0190685442668299	0.194354004726437	0.806550519904212\\
0.68	0.0200645661700016	0.199315191403999	0.799107529846305\\
0.68	0.0210872565164405	0.204278830634725	0.791686056496194\\
0.68	0.0221366940292259	0.20924472475049	0.784355603830601\\
0.68	0.023212955654804	0.214212674693577	0.777174850281119\\
0.68	0.0243161165387281	0.219182480034174	0.770191853862643\\
0.68	0.025446250001561	0.224153938988321	0.763444541554458\\
0.68	0.0266034275149466	0.229126848436298	0.756961413910483\\
0.68	0.0277877186778607	0.234101003941464	0.750762400416764\\
0.68	0.0289991911930496	0.239076199769547	0.744859807805785\\
0.68	0.0302379108436654	0.244052228908366	0.739259311518081\\
0.68	0.0315039414701067	0.24902888308801	0.733960949044502\\
0.68	0.0327973449470747	0.254005952801449	0.728960082407259\\
0.68	0.0341181811608519	0.258983227325592	0.72424830511118\\
0.68	0.0354665079868145	0.263960494742775	0.71981427622068\\
0.68	0.0368423812671862	0.268937541962694	0.715644470612631\\
0.68	0.0382458547890421	0.273914154744766	0.711723839838042\\
0.68	0.0396769802625738	0.278890117720924	0.70803638238973\\
0.68	0.0411358072996216	0.283865214418836	0.704565625567164\\
0.68	0.0426223833924861	0.288839227285554	0.701295023640315\\
0.68	0.0441367538930258	0.293811937711588	0.698208278749355\\
0.68	0.0456789619920503	0.298783126055387	0.695289592055348\\
0.68	0.0472490486990192	0.303752571668251	0.692523853197512\\
0.68	0.0488470528220538	0.308720052919643	0.689896776228121\\
0.68	0.0504730109482718	0.313685347222913	0.687394989990282\\
0.68	0.0521269574244539	0.318648231061428	0.68500609046598\\
0.68	0.0538089243380495	0.323608480015096	0.682718662029127\\
0.68	0.0555189414985328	0.328565868787292	0.68052227385347\\
0.68	0.0572570364191156	0.333520171232163	0.678407456997851\\
0.68	0.0590232342988274	0.338471160382329	0.676365666960043\\
0.68	0.0608175580049697	0.34341860847696	0.674389235783155\\
0.68	0.0626400280559541	0.348362286990222	0.672471317134862\\
0.68	0.064490662604533	0.3533019666601	0.670605827172183\\
0.68	0.066369477421429	0.358237417517579	0.668787383459804\\
0.68	0.0682764858793752	0.363168408916186	0.667011243730728\\
0.68	0.0702116989375697	0.368094709561873	0.66527324586365\\
0.68	0.0721751251265572	0.373016087543259	0.663569750098932\\
0.68	0.0741667705335422	0.377932310362199	0.661897584220061\\
0.68	0.0761866387881432	0.382843144964686	0.660253992184442\\
0.68	0.0782347310485954	0.38774835777208	0.658636586490996\\
0.68	0.0803110459884098	0.392647714712651	0.657043304415872\\
0.68	0.0824155797834956	0.397540981253432	0.65547236812651\\
0.68	0.084548326099754	0.402427922432375	0.653922248592746\\
0.68	0.0867092760811502	0.4073083028908	0.652391633146569\\
0.68	0.0888984183382709	0.412181886906127	0.650879396495579\\
0.68	0.0911157389373742	0.417048438424897	0.649384574964843\\
0.68	0.0933612213899392	0.421907721096044	0.647906343724685\\
0.68	0.0956348466427212	0.42675949830445	0.646443996754806\\
0.68	0.0979365930683194	0.431603533204733	0.644996929295962\\
0.68	0.100266436456264	0.436439588755293	0.643564622546848\\
0.68	0.102624350004627	0.441267427752585	0.64214663037443\\
0.68	0.105010304312169	0.446086812865616	0.640742567819433\\
0.68	0.107424267371012	0.450897506670668	0.63935210119373\\
0.68	0.109866204559871	0.455699271686218	0.637974939582443\\
0.68	0.112336078637817	0.460491870408058	0.636610827579861\\
0.68	0.114833849738607	0.465275065344603	0.635259539104382\\
0.68	0.117359475365564	0.470048619052377	0.633920872153155\\
0.68	0.119912910387023	0.474812294171664	0.632594644371965\\
0.68	0.122494107032347	0.479565853462319	0.631280689329641\\
0.68	0.125103014888515	0.484309059839721	0.629978853399059\\
0.68	0.127739580897283	0.489041676410868	0.628690460824557\\
0.685	0	0	0.710202759607373\\
0.685	1.11327767495586e-05	0.00471862581271275	0.713003157837853\\
0.685	4.46175251031912e-05	0.009446325184051	0.715846486660142\\
0.685	0.000100583311362513	0.0141829653360114	0.718788838379794\\
0.685	0.000179158434668431	0.018928411755669	0.721886092228356\\
0.685	0.000280470402701511	0.02368252819604	0.725200636220892\\
0.685	0.000404645907256436	0.0284451766772965	0.728799085274723\\
0.685	0.000551810799695644	0.0332162174883388	0.732749123983336\\
0.685	0.000722090066287311	0.037995509188729	0.737115575253327\\
0.685	0.000915607803432999	0.0427829086109896	0.741955893907011\\
0.685	0.0011324871927904	0.0475782708632729	0.747315373652289\\
0.685	0.00137285047629673	0.0523814493324038	0.753222424045811\\
0.685	0.00163681893109844	0.057192295687301	0.759684310525652\\
0.685	0.00192451284439304	0.0620106598827802	0.766683748102665\\
0.685	0.00223605148818898	0.0668363901637434	0.774176695511257\\
0.685	0.00257155309398959	0.0716693330697584	0.782091614494493\\
0.685	0.00293113482740716	0.0765093334400312	0.790330346421089\\
0.685	0.00331491276271365	0.0813562344187764	0.79877062752665\\
0.685	0.00372300185733413	0.0862098774609879	0.807270128930359\\
0.685	0.00415551592628969	0.0910701023386146	0.815671782700193\\
0.685	0.00461256761659621	0.0959367471471424	0.82381005350733\\
0.685	0.00509426838162598	0.100809648312589	0.831517746503354\\
0.685	0.00560072845543873	0.105688640598912	0.838632911419451\\
0.685	0.00613205682708918	0.11057355711583	0.845005411376294\\
0.685	0.00668836121491815	0.115464229327074	0.850502769049302\\
0.685	0.00726974804083431	0.120360487059049	0.855014975717657\\
0.685	0.00787632240459385	0.125262158509929	0.858458041096606\\
0.685	0.00850818805808555	0.13016907025918	0.860776163494183\\
0.685	0.00916544737962849	0.135081047277508	0.861942500814224\\
0.685	0.00984820134829017	0.139997912937243	0.861958614648421\\
0.685	0.0105565495182326	0.144919489023162	0.86085273561709\\
0.685	0.0112905899930939	0.149845595743738	0.858677054109649\\
0.685	0.012050419400414	0.154776051742839	0.855504274967456\\
0.685	0.0128361328661109	0.159710674111862	0.851423687967753\\
0.685	0.0136478239890177	0.164649278402306	0.84653700048434\\
0.685	0.0144855848154856	0.169591678638796	0.840954157904248\\
0.685	0.0153495058140643	0.174537687332543	0.834789345403936\\
0.685	0.0162396758502651	0.179487115495259	0.828157325806202\\
0.685	0.0171561821614173	0.184439772653511	0.821170226448692\\
0.685	0.0180991103316243	0.189395466863524	0.813934846725691\\
0.685	0.0190685442668299	0.194354004726437	0.806550519904213\\
0.685	0.0200645661700016	0.199315191403999	0.799107529846305\\
0.685	0.0210872565164405	0.204278830634725	0.791686056496193\\
0.685	0.0221366940292259	0.20924472475049	0.784355603830604\\
0.685	0.023212955654804	0.214212674693577	0.777174850281119\\
0.685	0.0243161165387281	0.219182480034174	0.770191853862643\\
0.685	0.025446250001561	0.224153938988321	0.763444541554458\\
0.685	0.0266034275149466	0.229126848436298	0.756961413910483\\
0.685	0.0277877186778607	0.234101003941464	0.750762400416763\\
0.685	0.0289991911930496	0.239076199769547	0.744859807805784\\
0.685	0.0302379108436654	0.244052228908366	0.73925931151808\\
0.685	0.0315039414701067	0.24902888308801	0.733960949044502\\
0.685	0.0327973449470747	0.254005952801449	0.728960082407262\\
0.685	0.0341181811608519	0.258983227325592	0.724248305111179\\
0.685	0.0354665079868145	0.263960494742775	0.71981427622068\\
0.685	0.0368423812671862	0.268937541962694	0.715644470612629\\
0.685	0.0382458547890421	0.273914154744766	0.711723839838043\\
0.685	0.0396769802625738	0.278890117720924	0.708036382389731\\
0.685	0.0411358072996216	0.283865214418836	0.704565625567163\\
0.685	0.0426223833924861	0.288839227285554	0.701295023640316\\
0.685	0.0441367538930258	0.293811937711588	0.698208278749355\\
0.685	0.0456789619920503	0.298783126055387	0.69528959205535\\
0.685	0.0472490486990192	0.303752571668251	0.692523853197511\\
0.685	0.0488470528220537	0.308720052919643	0.689896776228118\\
0.685	0.0504730109482718	0.313685347222913	0.687394989990281\\
0.685	0.0521269574244539	0.318648231061428	0.68500609046598\\
0.685	0.0538089243380495	0.323608480015096	0.682718662029127\\
0.685	0.0555189414985328	0.328565868787292	0.680522273853471\\
0.685	0.0572570364191156	0.333520171232163	0.67840745699785\\
0.685	0.0590232342988274	0.338471160382329	0.676365666960044\\
0.685	0.0608175580049697	0.34341860847696	0.674389235783157\\
0.685	0.0626400280559541	0.348362286990222	0.672471317134861\\
0.685	0.064490662604533	0.3533019666601	0.670605827172181\\
0.685	0.0663694774214291	0.358237417517579	0.668787383459804\\
0.685	0.0682764858793752	0.363168408916186	0.667011243730728\\
0.685	0.0702116989375697	0.368094709561873	0.665273245863649\\
0.685	0.0721751251265572	0.373016087543259	0.663569750098933\\
0.685	0.0741667705335422	0.377932310362199	0.661897584220059\\
0.685	0.0761866387881432	0.382843144964686	0.660253992184442\\
0.685	0.0782347310485954	0.38774835777208	0.658636586490997\\
0.685	0.0803110459884098	0.392647714712651	0.657043304415872\\
0.685	0.0824155797834956	0.397540981253432	0.655472368126509\\
0.685	0.084548326099754	0.402427922432375	0.653922248592745\\
0.685	0.0867092760811502	0.4073083028908	0.652391633146571\\
0.685	0.0888984183382709	0.412181886906127	0.650879396495579\\
0.685	0.0911157389373742	0.417048438424896	0.649384574964845\\
0.685	0.0933612213899392	0.421907721096044	0.647906343724684\\
0.685	0.0956348466427212	0.42675949830445	0.646443996754805\\
0.685	0.0979365930683194	0.431603533204733	0.644996929295961\\
0.685	0.100266436456264	0.436439588755293	0.643564622546847\\
0.685	0.102624350004627	0.441267427752584	0.64214663037443\\
0.685	0.105010304312169	0.446086812865616	0.640742567819433\\
0.685	0.107424267371012	0.450897506670668	0.639352101193731\\
0.685	0.109866204559871	0.455699271686218	0.637974939582444\\
0.685	0.112336078637817	0.460491870408058	0.636610827579862\\
0.685	0.114833849738607	0.465275065344603	0.635259539104381\\
0.685	0.117359475365564	0.470048619052377	0.633920872153155\\
0.685	0.119912910387022	0.474812294171664	0.632594644371965\\
0.685	0.122494107032347	0.479565853462319	0.631280689329641\\
0.685	0.125103014888515	0.484309059839721	0.629978853399059\\
0.685	0.127739580897283	0.489041676410868	0.628690460824555\\
0.69	0	0	0.710202759607373\\
0.69	1.11327767495586e-05	0.00471862581271275	0.713003157837853\\
0.69	4.46175251031912e-05	0.009446325184051	0.715846486660142\\
0.69	0.000100583311362513	0.0141829653360114	0.718788838379794\\
0.69	0.000179158434668431	0.018928411755669	0.721886092228356\\
0.69	0.000280470402701511	0.02368252819604	0.725200636220892\\
0.69	0.000404645907256436	0.0284451766772965	0.728799085274723\\
0.69	0.000551810799695644	0.0332162174883389	0.732749123983336\\
0.69	0.000722090066287311	0.037995509188729	0.737115575253328\\
0.69	0.000915607803432999	0.0427829086109896	0.741955893907011\\
0.69	0.0011324871927904	0.0475782708632729	0.747315373652289\\
0.69	0.00137285047629673	0.0523814493324038	0.75322242404581\\
0.69	0.00163681893109844	0.057192295687301	0.759684310525653\\
0.69	0.00192451284439304	0.0620106598827802	0.766683748102665\\
0.69	0.00223605148818899	0.0668363901637434	0.774176695511256\\
0.69	0.00257155309398959	0.0716693330697584	0.782091614494494\\
0.69	0.00293113482740716	0.0765093334400312	0.79033034642109\\
0.69	0.00331491276271365	0.0813562344187764	0.79877062752665\\
0.69	0.00372300185733413	0.0862098774609879	0.80727012893036\\
0.69	0.00415551592628969	0.0910701023386145	0.815671782700193\\
0.69	0.00461256761659621	0.0959367471471425	0.82381005350733\\
0.69	0.00509426838162598	0.100809648312589	0.831517746503354\\
0.69	0.00560072845543873	0.105688640598912	0.838632911419452\\
0.69	0.00613205682708918	0.11057355711583	0.845005411376293\\
0.69	0.00668836121491816	0.115464229327074	0.850502769049302\\
0.69	0.00726974804083431	0.120360487059049	0.855014975717657\\
0.69	0.00787632240459385	0.125262158509929	0.858458041096606\\
0.69	0.00850818805808555	0.13016907025918	0.860776163494182\\
0.69	0.00916544737962849	0.135081047277508	0.861942500814225\\
0.69	0.00984820134829017	0.139997912937243	0.861958614648422\\
0.69	0.0105565495182326	0.144919489023162	0.86085273561709\\
0.69	0.0112905899930939	0.149845595743738	0.858677054109649\\
0.69	0.012050419400414	0.154776051742839	0.855504274967456\\
0.69	0.0128361328661109	0.159710674111862	0.851423687967756\\
0.69	0.0136478239890177	0.164649278402306	0.84653700048434\\
0.69	0.0144855848154856	0.169591678638796	0.840954157904248\\
0.69	0.0153495058140643	0.174537687332543	0.834789345403936\\
0.69	0.0162396758502651	0.179487115495259	0.8281573258062\\
0.69	0.0171561821614173	0.184439772653511	0.821170226448692\\
0.69	0.0180991103316243	0.189395466863524	0.813934846725693\\
0.69	0.0190685442668299	0.194354004726437	0.806550519904214\\
0.69	0.0200645661700016	0.199315191403999	0.799107529846305\\
0.69	0.0210872565164405	0.204278830634725	0.791686056496194\\
0.69	0.0221366940292259	0.20924472475049	0.784355603830603\\
0.69	0.023212955654804	0.214212674693577	0.777174850281117\\
0.69	0.0243161165387281	0.219182480034174	0.770191853862642\\
0.69	0.025446250001561	0.224153938988321	0.763444541554457\\
0.69	0.0266034275149466	0.229126848436298	0.756961413910482\\
0.69	0.0277877186778607	0.234101003941464	0.750762400416764\\
0.69	0.0289991911930496	0.239076199769547	0.744859807805786\\
0.69	0.0302379108436654	0.244052228908366	0.73925931151808\\
0.69	0.0315039414701067	0.24902888308801	0.733960949044504\\
0.69	0.0327973449470747	0.254005952801449	0.728960082407259\\
0.69	0.0341181811608519	0.258983227325591	0.724248305111177\\
0.69	0.0354665079868145	0.263960494742775	0.719814276220683\\
0.69	0.0368423812671862	0.268937541962694	0.71564447061263\\
0.69	0.0382458547890421	0.273914154744766	0.711723839838042\\
0.69	0.0396769802625738	0.278890117720924	0.70803638238973\\
0.69	0.0411358072996216	0.283865214418836	0.704565625567162\\
0.69	0.0426223833924861	0.288839227285554	0.701295023640317\\
0.69	0.0441367538930258	0.293811937711588	0.698208278749355\\
0.69	0.0456789619920503	0.298783126055387	0.695289592055349\\
0.69	0.0472490486990192	0.303752571668251	0.692523853197512\\
0.69	0.0488470528220538	0.308720052919643	0.689896776228119\\
0.69	0.0504730109482718	0.313685347222913	0.687394989990282\\
0.69	0.0521269574244539	0.318648231061428	0.685006090465981\\
0.69	0.0538089243380495	0.323608480015097	0.682718662029127\\
0.69	0.0555189414985328	0.328565868787292	0.680522273853468\\
0.69	0.0572570364191156	0.333520171232163	0.67840745699785\\
0.69	0.0590232342988274	0.338471160382329	0.676365666960044\\
0.69	0.0608175580049697	0.34341860847696	0.674389235783156\\
0.69	0.0626400280559541	0.348362286990222	0.672471317134863\\
0.69	0.064490662604533	0.3533019666601	0.670605827172183\\
0.69	0.0663694774214291	0.358237417517579	0.668787383459804\\
0.69	0.0682764858793752	0.363168408916186	0.667011243730728\\
0.69	0.0702116989375697	0.368094709561873	0.665273245863649\\
0.69	0.0721751251265572	0.373016087543259	0.663569750098932\\
0.69	0.0741667705335422	0.377932310362199	0.661897584220061\\
0.69	0.0761866387881432	0.382843144964686	0.660253992184441\\
0.69	0.0782347310485954	0.38774835777208	0.658636586490996\\
0.69	0.0803110459884098	0.392647714712651	0.657043304415871\\
0.69	0.0824155797834956	0.397540981253432	0.65547236812651\\
0.69	0.084548326099754	0.402427922432375	0.653922248592744\\
0.69	0.0867092760811502	0.407308302890799	0.65239163314657\\
0.69	0.0888984183382709	0.412181886906127	0.650879396495579\\
0.69	0.0911157389373742	0.417048438424896	0.649384574964844\\
0.69	0.0933612213899392	0.421907721096044	0.647906343724686\\
0.69	0.0956348466427212	0.42675949830445	0.646443996754806\\
0.69	0.0979365930683194	0.431603533204733	0.644996929295961\\
0.69	0.100266436456264	0.436439588755293	0.643564622546846\\
0.69	0.102624350004627	0.441267427752585	0.642146630374429\\
0.69	0.105010304312169	0.446086812865616	0.640742567819433\\
0.69	0.107424267371012	0.450897506670668	0.63935210119373\\
0.69	0.109866204559871	0.455699271686218	0.637974939582442\\
0.69	0.112336078637817	0.460491870408058	0.636610827579863\\
0.69	0.114833849738607	0.465275065344603	0.635259539104384\\
0.69	0.117359475365564	0.470048619052377	0.633920872153155\\
0.69	0.119912910387023	0.474812294171664	0.632594644371964\\
0.69	0.122494107032347	0.479565853462319	0.63128068932964\\
0.69	0.125103014888515	0.484309059839721	0.62997885339906\\
0.69	0.127739580897283	0.489041676410868	0.628690460824558\\
0.695	0	0	0.710202759607373\\
0.695	1.11327767495586e-05	0.00471862581271275	0.713003157837853\\
0.695	4.46175251031912e-05	0.009446325184051	0.715846486660142\\
0.695	0.000100583311362513	0.0141829653360114	0.718788838379794\\
0.695	0.000179158434668431	0.018928411755669	0.721886092228356\\
0.695	0.000280470402701511	0.02368252819604	0.725200636220892\\
0.695	0.000404645907256436	0.0284451766772965	0.728799085274723\\
0.695	0.000551810799695644	0.0332162174883388	0.732749123983336\\
0.695	0.000722090066287311	0.037995509188729	0.737115575253327\\
0.695	0.000915607803432999	0.0427829086109896	0.741955893907011\\
0.695	0.0011324871927904	0.0475782708632729	0.747315373652289\\
0.695	0.00137285047629673	0.0523814493324038	0.753222424045811\\
0.695	0.00163681893109844	0.057192295687301	0.759684310525652\\
0.695	0.00192451284439304	0.0620106598827802	0.766683748102665\\
0.695	0.00223605148818899	0.0668363901637434	0.774176695511257\\
0.695	0.00257155309398959	0.0716693330697584	0.782091614494493\\
0.695	0.00293113482740716	0.0765093334400312	0.79033034642109\\
0.695	0.00331491276271365	0.0813562344187764	0.79877062752665\\
0.695	0.00372300185733413	0.0862098774609879	0.807270128930359\\
0.695	0.00415551592628969	0.0910701023386145	0.815671782700194\\
0.695	0.00461256761659621	0.0959367471471425	0.82381005350733\\
0.695	0.00509426838162598	0.100809648312589	0.831517746503354\\
0.695	0.00560072845543873	0.105688640598912	0.838632911419451\\
0.695	0.00613205682708918	0.11057355711583	0.845005411376292\\
0.695	0.00668836121491815	0.115464229327074	0.850502769049302\\
0.695	0.00726974804083431	0.120360487059049	0.855014975717659\\
0.695	0.00787632240459385	0.125262158509929	0.858458041096605\\
0.695	0.00850818805808555	0.13016907025918	0.860776163494181\\
0.695	0.00916544737962849	0.135081047277508	0.861942500814226\\
0.695	0.00984820134829017	0.139997912937243	0.86195861464842\\
0.695	0.0105565495182326	0.144919489023162	0.860852735617091\\
0.695	0.0112905899930939	0.149845595743738	0.858677054109651\\
0.695	0.012050419400414	0.154776051742839	0.85550427496746\\
0.695	0.0128361328661109	0.159710674111862	0.851423687967755\\
0.695	0.0136478239890177	0.164649278402306	0.846537000484341\\
0.695	0.0144855848154856	0.169591678638796	0.84095415790425\\
0.695	0.0153495058140643	0.174537687332543	0.834789345403935\\
0.695	0.0162396758502651	0.179487115495259	0.828157325806199\\
0.695	0.0171561821614173	0.184439772653511	0.821170226448694\\
0.695	0.0180991103316243	0.189395466863524	0.813934846725695\\
0.695	0.0190685442668299	0.194354004726437	0.806550519904213\\
0.695	0.0200645661700016	0.199315191403999	0.799107529846303\\
0.695	0.0210872565164405	0.204278830634725	0.791686056496195\\
0.695	0.0221366940292259	0.20924472475049	0.784355603830601\\
0.695	0.023212955654804	0.214212674693577	0.777174850281117\\
0.695	0.0243161165387281	0.219182480034174	0.770191853862644\\
0.695	0.025446250001561	0.224153938988321	0.763444541554459\\
0.695	0.0266034275149466	0.229126848436298	0.756961413910484\\
0.695	0.0277877186778607	0.234101003941464	0.750762400416766\\
0.695	0.0289991911930496	0.239076199769547	0.744859807805785\\
0.695	0.0302379108436654	0.244052228908366	0.739259311518079\\
0.695	0.0315039414701067	0.24902888308801	0.733960949044502\\
0.695	0.0327973449470747	0.254005952801449	0.728960082407262\\
0.695	0.0341181811608519	0.258983227325592	0.724248305111179\\
0.695	0.0354665079868145	0.263960494742775	0.71981427622068\\
0.695	0.0368423812671862	0.268937541962694	0.71564447061263\\
0.695	0.0382458547890421	0.273914154744766	0.711723839838041\\
0.695	0.0396769802625738	0.278890117720924	0.70803638238973\\
0.695	0.0411358072996216	0.283865214418836	0.704565625567164\\
0.695	0.0426223833924861	0.288839227285554	0.701295023640316\\
0.695	0.0441367538930258	0.293811937711588	0.698208278749353\\
0.695	0.0456789619920503	0.298783126055387	0.69528959205535\\
0.695	0.0472490486990192	0.303752571668251	0.692523853197512\\
0.695	0.0488470528220538	0.308720052919643	0.689896776228118\\
0.695	0.0504730109482718	0.313685347222913	0.687394989990282\\
0.695	0.0521269574244539	0.318648231061428	0.685006090465981\\
0.695	0.0538089243380495	0.323608480015096	0.682718662029127\\
0.695	0.0555189414985328	0.328565868787292	0.680522273853471\\
0.695	0.0572570364191156	0.333520171232163	0.678407456997849\\
0.695	0.0590232342988274	0.338471160382329	0.676365666960042\\
0.695	0.0608175580049697	0.34341860847696	0.674389235783155\\
0.695	0.0626400280559541	0.348362286990221	0.672471317134861\\
0.695	0.064490662604533	0.3533019666601	0.670605827172184\\
0.695	0.066369477421429	0.358237417517579	0.668787383459805\\
0.695	0.0682764858793752	0.363168408916186	0.667011243730727\\
0.695	0.0702116989375697	0.368094709561873	0.66527324586365\\
0.695	0.0721751251265572	0.373016087543259	0.663569750098933\\
0.695	0.0741667705335422	0.377932310362199	0.66189758422006\\
0.695	0.0761866387881432	0.382843144964686	0.660253992184443\\
0.695	0.0782347310485954	0.38774835777208	0.658636586490998\\
0.695	0.0803110459884098	0.392647714712651	0.657043304415871\\
0.695	0.0824155797834956	0.397540981253432	0.655472368126509\\
0.695	0.084548326099754	0.402427922432375	0.653922248592746\\
0.695	0.0867092760811502	0.407308302890799	0.652391633146568\\
0.695	0.0888984183382709	0.412181886906127	0.65087939649558\\
0.695	0.0911157389373742	0.417048438424896	0.649384574964843\\
0.695	0.0933612213899392	0.421907721096044	0.647906343724684\\
0.695	0.0956348466427212	0.42675949830445	0.646443996754807\\
0.695	0.0979365930683194	0.431603533204733	0.644996929295963\\
0.695	0.100266436456264	0.436439588755293	0.643564622546847\\
0.695	0.102624350004627	0.441267427752584	0.64214663037443\\
0.695	0.105010304312169	0.446086812865616	0.640742567819435\\
0.695	0.107424267371012	0.450897506670668	0.639352101193731\\
0.695	0.109866204559871	0.455699271686218	0.637974939582441\\
0.695	0.112336078637817	0.460491870408058	0.636610827579861\\
0.695	0.114833849738607	0.465275065344603	0.635259539104383\\
0.695	0.117359475365564	0.470048619052377	0.633920872153155\\
0.695	0.119912910387022	0.474812294171664	0.632594644371966\\
0.695	0.122494107032347	0.479565853462319	0.63128068932964\\
0.695	0.125103014888515	0.484309059839721	0.629978853399059\\
0.695	0.127739580897283	0.489041676410868	0.628690460824563\\
0.7	0	0	0.710202759607373\\
0.7	1.11327767495586e-05	0.00471862581271274	0.713003157837853\\
0.7	4.46175251031912e-05	0.009446325184051	0.715846486660142\\
0.7	0.000100583311362513	0.0141829653360114	0.718788838379794\\
0.7	0.000179158434668431	0.018928411755669	0.721886092228356\\
0.7	0.000280470402701511	0.02368252819604	0.725200636220892\\
0.7	0.000404645907256436	0.0284451766772965	0.728799085274723\\
0.7	0.000551810799695644	0.0332162174883388	0.732749123983336\\
0.7	0.000722090066287311	0.037995509188729	0.737115575253327\\
0.7	0.000915607803432999	0.0427829086109896	0.741955893907011\\
0.7	0.0011324871927904	0.047578270863273	0.747315373652289\\
0.7	0.00137285047629673	0.0523814493324038	0.75322242404581\\
0.7	0.00163681893109844	0.057192295687301	0.759684310525652\\
0.7	0.00192451284439304	0.0620106598827802	0.766683748102665\\
0.7	0.00223605148818898	0.0668363901637434	0.774176695511256\\
0.7	0.00257155309398959	0.0716693330697584	0.782091614494493\\
0.7	0.00293113482740716	0.0765093334400312	0.79033034642109\\
0.7	0.00331491276271365	0.0813562344187764	0.79877062752665\\
0.7	0.00372300185733414	0.0862098774609879	0.807270128930359\\
0.7	0.00415551592628969	0.0910701023386145	0.815671782700193\\
0.7	0.00461256761659621	0.0959367471471424	0.823810053507331\\
0.7	0.00509426838162598	0.100809648312589	0.831517746503353\\
0.7	0.00560072845543873	0.105688640598912	0.838632911419451\\
0.7	0.00613205682708918	0.11057355711583	0.845005411376291\\
0.7	0.00668836121491816	0.115464229327074	0.850502769049301\\
0.7	0.00726974804083431	0.120360487059049	0.855014975717655\\
0.7	0.00787632240459385	0.125262158509929	0.858458041096607\\
0.7	0.00850818805808554	0.13016907025918	0.860776163494183\\
0.7	0.00916544737962849	0.135081047277508	0.861942500814223\\
0.7	0.00984820134829017	0.139997912937243	0.86195861464842\\
0.7	0.0105565495182326	0.144919489023162	0.860852735617091\\
0.7	0.0112905899930939	0.149845595743738	0.85867705410965\\
0.7	0.012050419400414	0.154776051742839	0.855504274967459\\
0.7	0.0128361328661109	0.159710674111862	0.851423687967755\\
0.7	0.0136478239890177	0.164649278402306	0.846537000484339\\
0.7	0.0144855848154856	0.169591678638796	0.840954157904248\\
0.7	0.0153495058140643	0.174537687332543	0.834789345403936\\
0.7	0.0162396758502651	0.179487115495259	0.828157325806199\\
0.7	0.0171561821614173	0.184439772653511	0.821170226448695\\
0.7	0.0180991103316243	0.189395466863524	0.813934846725692\\
0.7	0.0190685442668299	0.194354004726437	0.806550519904212\\
0.7	0.0200645661700016	0.199315191403999	0.799107529846304\\
0.7	0.0210872565164405	0.204278830634725	0.791686056496196\\
0.7	0.0221366940292259	0.20924472475049	0.7843556038306\\
0.7	0.023212955654804	0.214212674693577	0.777174850281119\\
0.7	0.0243161165387281	0.219182480034174	0.770191853862643\\
0.7	0.025446250001561	0.224153938988321	0.763444541554459\\
0.7	0.0266034275149466	0.229126848436298	0.756961413910484\\
0.7	0.0277877186778607	0.234101003941464	0.750762400416763\\
0.7	0.0289991911930496	0.239076199769547	0.744859807805784\\
0.7	0.0302379108436654	0.244052228908366	0.739259311518079\\
0.7	0.0315039414701067	0.24902888308801	0.733960949044503\\
0.7	0.0327973449470747	0.254005952801449	0.728960082407261\\
0.7	0.0341181811608519	0.258983227325592	0.724248305111176\\
0.7	0.0354665079868145	0.263960494742775	0.719814276220681\\
0.7	0.0368423812671862	0.268937541962694	0.715644470612629\\
0.7	0.0382458547890421	0.273914154744766	0.711723839838042\\
0.7	0.0396769802625738	0.278890117720924	0.708036382389731\\
0.7	0.0411358072996216	0.283865214418836	0.704565625567165\\
0.7	0.0426223833924861	0.288839227285554	0.701295023640316\\
0.7	0.0441367538930258	0.293811937711588	0.698208278749353\\
0.7	0.0456789619920503	0.298783126055387	0.69528959205535\\
0.7	0.0472490486990192	0.303752571668251	0.692523853197512\\
0.7	0.0488470528220538	0.308720052919643	0.689896776228119\\
0.7	0.0504730109482718	0.313685347222913	0.687394989990281\\
0.7	0.0521269574244539	0.318648231061428	0.685006090465981\\
0.7	0.0538089243380495	0.323608480015097	0.682718662029127\\
0.7	0.0555189414985328	0.328565868787292	0.680522273853469\\
0.7	0.0572570364191156	0.333520171232163	0.678407456997851\\
0.7	0.0590232342988274	0.338471160382329	0.676365666960044\\
0.7	0.0608175580049697	0.34341860847696	0.674389235783156\\
0.7	0.0626400280559541	0.348362286990222	0.67247131713486\\
0.7	0.064490662604533	0.3533019666601	0.670605827172182\\
0.7	0.066369477421429	0.358237417517579	0.668787383459805\\
0.7	0.0682764858793752	0.363168408916186	0.667011243730727\\
0.7	0.0702116989375697	0.368094709561873	0.665273245863647\\
0.7	0.0721751251265572	0.373016087543259	0.663569750098933\\
0.7	0.0741667705335422	0.377932310362199	0.66189758422006\\
0.7	0.0761866387881432	0.382843144964686	0.660253992184442\\
0.7	0.0782347310485954	0.38774835777208	0.658636586490997\\
0.7	0.0803110459884098	0.392647714712651	0.657043304415872\\
0.7	0.0824155797834956	0.397540981253432	0.655472368126509\\
0.7	0.084548326099754	0.402427922432375	0.653922248592746\\
0.7	0.0867092760811502	0.4073083028908	0.65239163314657\\
0.7	0.0888984183382709	0.412181886906127	0.650879396495578\\
0.7	0.0911157389373742	0.417048438424897	0.649384574964844\\
0.7	0.0933612213899392	0.421907721096044	0.647906343724683\\
0.7	0.0956348466427212	0.42675949830445	0.646443996754805\\
0.7	0.0979365930683194	0.431603533204733	0.644996929295961\\
0.7	0.100266436456264	0.436439588755293	0.643564622546848\\
0.7	0.102624350004627	0.441267427752584	0.64214663037443\\
0.7	0.105010304312169	0.446086812865616	0.640742567819433\\
0.7	0.107424267371012	0.450897506670668	0.639352101193732\\
0.7	0.109866204559871	0.455699271686218	0.637974939582444\\
0.7	0.112336078637817	0.460491870408058	0.636610827579861\\
0.7	0.114833849738607	0.465275065344603	0.635259539104383\\
0.7	0.117359475365564	0.470048619052377	0.633920872153156\\
0.7	0.119912910387023	0.474812294171664	0.632594644371964\\
0.7	0.122494107032347	0.479565853462319	0.631280689329641\\
0.7	0.125103014888515	0.484309059839721	0.629978853399058\\
0.7	0.127739580897283	0.489041676410868	0.628690460824557\\
0.705	0	0	0.710202759607373\\
0.705	1.11327767495586e-05	0.00471862581271275	0.713003157837853\\
0.705	4.46175251031912e-05	0.009446325184051	0.715846486660142\\
0.705	0.000100583311362513	0.0141829653360114	0.718788838379794\\
0.705	0.000179158434668431	0.018928411755669	0.721886092228356\\
0.705	0.000280470402701511	0.02368252819604	0.725200636220892\\
0.705	0.000404645907256436	0.0284451766772965	0.728799085274723\\
0.705	0.000551810799695644	0.0332162174883389	0.732749123983336\\
0.705	0.000722090066287311	0.037995509188729	0.737115575253328\\
0.705	0.000915607803432999	0.0427829086109896	0.741955893907011\\
0.705	0.0011324871927904	0.0475782708632729	0.747315373652289\\
0.705	0.00137285047629673	0.0523814493324038	0.753222424045811\\
0.705	0.00163681893109844	0.057192295687301	0.759684310525652\\
0.705	0.00192451284439304	0.0620106598827802	0.766683748102665\\
0.705	0.00223605148818898	0.0668363901637434	0.774176695511257\\
0.705	0.00257155309398959	0.0716693330697584	0.782091614494494\\
0.705	0.00293113482740716	0.0765093334400312	0.79033034642109\\
0.705	0.00331491276271365	0.0813562344187764	0.79877062752665\\
0.705	0.00372300185733413	0.0862098774609879	0.807270128930359\\
0.705	0.00415551592628969	0.0910701023386145	0.815671782700194\\
0.705	0.00461256761659621	0.0959367471471425	0.823810053507331\\
0.705	0.00509426838162598	0.100809648312589	0.831517746503353\\
0.705	0.00560072845543873	0.105688640598912	0.838632911419452\\
0.705	0.00613205682708918	0.11057355711583	0.845005411376293\\
0.705	0.00668836121491816	0.115464229327074	0.850502769049302\\
0.705	0.00726974804083431	0.120360487059049	0.855014975717657\\
0.705	0.00787632240459385	0.125262158509929	0.858458041096605\\
0.705	0.00850818805808554	0.13016907025918	0.860776163494182\\
0.705	0.00916544737962849	0.135081047277508	0.861942500814222\\
0.705	0.00984820134829017	0.139997912937243	0.861958614648418\\
0.705	0.0105565495182326	0.144919489023162	0.86085273561709\\
0.705	0.0112905899930939	0.149845595743738	0.85867705410965\\
0.705	0.012050419400414	0.154776051742839	0.855504274967455\\
0.705	0.0128361328661109	0.159710674111862	0.851423687967754\\
0.705	0.0136478239890177	0.164649278402306	0.846537000484341\\
0.705	0.0144855848154856	0.169591678638796	0.840954157904248\\
0.705	0.0153495058140643	0.174537687332543	0.834789345403936\\
0.705	0.0162396758502651	0.179487115495259	0.8281573258062\\
0.705	0.0171561821614173	0.184439772653511	0.821170226448695\\
0.705	0.0180991103316243	0.189395466863524	0.813934846725694\\
0.705	0.0190685442668299	0.194354004726437	0.806550519904211\\
0.705	0.0200645661700016	0.199315191403999	0.799107529846304\\
0.705	0.0210872565164405	0.204278830634725	0.791686056496193\\
0.705	0.0221366940292259	0.20924472475049	0.7843556038306\\
0.705	0.023212955654804	0.214212674693577	0.777174850281117\\
0.705	0.0243161165387281	0.219182480034174	0.770191853862644\\
0.705	0.025446250001561	0.224153938988321	0.76344454155446\\
0.705	0.0266034275149466	0.229126848436298	0.756961413910482\\
0.705	0.0277877186778607	0.234101003941464	0.750762400416763\\
0.705	0.0289991911930496	0.239076199769547	0.744859807805786\\
0.705	0.0302379108436654	0.244052228908366	0.739259311518079\\
0.705	0.0315039414701067	0.24902888308801	0.733960949044504\\
0.705	0.0327973449470747	0.254005952801449	0.72896008240726\\
0.705	0.0341181811608519	0.258983227325591	0.724248305111178\\
0.705	0.0354665079868145	0.263960494742775	0.719814276220683\\
0.705	0.0368423812671862	0.268937541962694	0.71564447061263\\
0.705	0.0382458547890421	0.273914154744766	0.711723839838041\\
0.705	0.0396769802625738	0.278890117720924	0.70803638238973\\
0.705	0.0411358072996216	0.283865214418836	0.704565625567166\\
0.705	0.0426223833924861	0.288839227285554	0.701295023640316\\
0.705	0.0441367538930258	0.293811937711588	0.698208278749353\\
0.705	0.0456789619920503	0.298783126055387	0.69528959205535\\
0.705	0.0472490486990192	0.303752571668251	0.692523853197512\\
0.705	0.0488470528220538	0.308720052919643	0.689896776228119\\
0.705	0.0504730109482718	0.313685347222913	0.687394989990281\\
0.705	0.0521269574244539	0.318648231061428	0.685006090465981\\
0.705	0.0538089243380495	0.323608480015097	0.682718662029128\\
0.705	0.0555189414985328	0.328565868787292	0.680522273853469\\
0.705	0.0572570364191156	0.333520171232163	0.67840745699785\\
0.705	0.0590232342988274	0.338471160382329	0.676365666960044\\
0.705	0.0608175580049697	0.34341860847696	0.674389235783156\\
0.705	0.0626400280559541	0.348362286990222	0.672471317134862\\
0.705	0.064490662604533	0.3533019666601	0.670605827172182\\
0.705	0.0663694774214291	0.358237417517579	0.668787383459803\\
0.705	0.0682764858793752	0.363168408916186	0.667011243730729\\
0.705	0.0702116989375697	0.368094709561873	0.665273245863649\\
0.705	0.0721751251265572	0.373016087543259	0.663569750098931\\
0.705	0.0741667705335422	0.377932310362199	0.661897584220061\\
0.705	0.0761866387881432	0.382843144964686	0.660253992184442\\
0.705	0.0782347310485954	0.38774835777208	0.658636586490997\\
0.705	0.0803110459884098	0.392647714712651	0.657043304415871\\
0.705	0.0824155797834956	0.397540981253432	0.655472368126509\\
0.705	0.084548326099754	0.402427922432375	0.653922248592746\\
0.705	0.0867092760811502	0.4073083028908	0.65239163314657\\
0.705	0.0888984183382709	0.412181886906127	0.650879396495579\\
0.705	0.0911157389373742	0.417048438424896	0.649384574964845\\
0.705	0.0933612213899392	0.421907721096044	0.647906343724685\\
0.705	0.0956348466427212	0.42675949830445	0.646443996754805\\
0.705	0.0979365930683194	0.431603533204733	0.644996929295962\\
0.705	0.100266436456264	0.436439588755293	0.643564622546846\\
0.705	0.102624350004627	0.441267427752584	0.64214663037443\\
0.705	0.105010304312169	0.446086812865616	0.640742567819433\\
0.705	0.107424267371012	0.450897506670668	0.63935210119373\\
0.705	0.109866204559871	0.455699271686218	0.637974939582444\\
0.705	0.112336078637817	0.460491870408058	0.636610827579863\\
0.705	0.114833849738607	0.465275065344603	0.635259539104381\\
0.705	0.117359475365564	0.470048619052377	0.633920872153154\\
0.705	0.119912910387023	0.474812294171664	0.632594644371966\\
0.705	0.122494107032347	0.479565853462319	0.631280689329641\\
0.705	0.125103014888515	0.484309059839721	0.62997885339906\\
0.705	0.127739580897283	0.489041676410868	0.628690460824556\\
0.71	0	0	0.710202759607373\\
0.71	1.11327767495586e-05	0.00471862581271274	0.713003157837853\\
0.71	4.46175251031912e-05	0.009446325184051	0.715846486660142\\
0.71	0.000100583311362513	0.0141829653360114	0.718788838379794\\
0.71	0.000179158434668431	0.018928411755669	0.721886092228356\\
0.71	0.000280470402701511	0.02368252819604	0.725200636220892\\
0.71	0.000404645907256436	0.0284451766772965	0.728799085274723\\
0.71	0.000551810799695644	0.0332162174883388	0.732749123983336\\
0.71	0.000722090066287311	0.037995509188729	0.737115575253327\\
0.71	0.000915607803432999	0.0427829086109896	0.741955893907011\\
0.71	0.0011324871927904	0.047578270863273	0.747315373652289\\
0.71	0.00137285047629673	0.0523814493324038	0.753222424045811\\
0.71	0.00163681893109844	0.057192295687301	0.759684310525652\\
0.71	0.00192451284439304	0.0620106598827802	0.766683748102665\\
0.71	0.00223605148818898	0.0668363901637434	0.774176695511256\\
0.71	0.00257155309398959	0.0716693330697584	0.782091614494494\\
0.71	0.00293113482740716	0.0765093334400312	0.79033034642109\\
0.71	0.00331491276271365	0.0813562344187764	0.79877062752665\\
0.71	0.00372300185733413	0.0862098774609879	0.80727012893036\\
0.71	0.00415551592628969	0.0910701023386145	0.815671782700194\\
0.71	0.00461256761659621	0.0959367471471425	0.82381005350733\\
0.71	0.00509426838162598	0.100809648312589	0.831517746503354\\
0.71	0.00560072845543873	0.105688640598912	0.838632911419451\\
0.71	0.00613205682708918	0.11057355711583	0.845005411376293\\
0.71	0.00668836121491816	0.115464229327074	0.850502769049302\\
0.71	0.00726974804083431	0.120360487059049	0.855014975717658\\
0.71	0.00787632240459385	0.125262158509929	0.858458041096605\\
0.71	0.00850818805808555	0.13016907025918	0.860776163494182\\
0.71	0.00916544737962849	0.135081047277508	0.861942500814224\\
0.71	0.00984820134829017	0.139997912937243	0.86195861464842\\
0.71	0.0105565495182326	0.144919489023162	0.860852735617091\\
0.71	0.0112905899930939	0.149845595743738	0.858677054109651\\
0.71	0.012050419400414	0.154776051742839	0.855504274967457\\
0.71	0.0128361328661109	0.159710674111862	0.851423687967756\\
0.71	0.0136478239890177	0.164649278402306	0.84653700048434\\
0.71	0.0144855848154856	0.169591678638796	0.84095415790425\\
0.71	0.0153495058140643	0.174537687332543	0.834789345403934\\
0.71	0.0162396758502651	0.179487115495259	0.828157325806202\\
0.71	0.0171561821614173	0.184439772653511	0.821170226448694\\
0.71	0.0180991103316243	0.189395466863524	0.813934846725691\\
0.71	0.0190685442668299	0.194354004726437	0.806550519904208\\
0.71	0.0200645661700016	0.199315191403999	0.799107529846305\\
0.71	0.0210872565164405	0.204278830634725	0.791686056496194\\
0.71	0.0221366940292259	0.20924472475049	0.784355603830602\\
0.71	0.023212955654804	0.214212674693577	0.777174850281119\\
0.71	0.0243161165387281	0.219182480034174	0.770191853862645\\
0.71	0.025446250001561	0.224153938988321	0.763444541554458\\
0.71	0.0266034275149466	0.229126848436298	0.756961413910482\\
0.71	0.0277877186778607	0.234101003941464	0.750762400416765\\
0.71	0.0289991911930496	0.239076199769547	0.744859807805784\\
0.71	0.0302379108436654	0.244052228908366	0.739259311518078\\
0.71	0.0315039414701067	0.24902888308801	0.733960949044503\\
0.71	0.0327973449470747	0.254005952801449	0.728960082407262\\
0.71	0.0341181811608519	0.258983227325592	0.724248305111178\\
0.71	0.0354665079868145	0.263960494742775	0.71981427622068\\
0.71	0.0368423812671862	0.268937541962694	0.71564447061263\\
0.71	0.0382458547890421	0.273914154744766	0.711723839838043\\
0.71	0.0396769802625738	0.278890117720924	0.70803638238973\\
0.71	0.0411358072996216	0.283865214418836	0.704565625567164\\
0.71	0.0426223833924861	0.288839227285554	0.701295023640316\\
0.71	0.0441367538930258	0.293811937711588	0.698208278749355\\
0.71	0.0456789619920503	0.298783126055387	0.69528959205535\\
0.71	0.0472490486990192	0.303752571668251	0.69252385319751\\
0.71	0.0488470528220538	0.308720052919643	0.689896776228117\\
0.71	0.0504730109482718	0.313685347222913	0.687394989990282\\
0.71	0.0521269574244539	0.318648231061428	0.685006090465982\\
0.71	0.0538089243380495	0.323608480015097	0.682718662029128\\
0.71	0.0555189414985328	0.328565868787292	0.680522273853469\\
0.71	0.0572570364191156	0.333520171232163	0.67840745699785\\
0.71	0.0590232342988274	0.338471160382329	0.676365666960045\\
0.71	0.0608175580049697	0.34341860847696	0.674389235783156\\
0.71	0.0626400280559541	0.348362286990222	0.672471317134861\\
0.71	0.064490662604533	0.3533019666601	0.670605827172183\\
0.71	0.0663694774214291	0.358237417517579	0.668787383459804\\
0.71	0.0682764858793752	0.363168408916186	0.667011243730728\\
0.71	0.0702116989375697	0.368094709561873	0.66527324586365\\
0.71	0.0721751251265572	0.373016087543259	0.663569750098932\\
0.71	0.0741667705335422	0.377932310362199	0.66189758422006\\
0.71	0.0761866387881432	0.382843144964686	0.660253992184442\\
0.71	0.0782347310485954	0.38774835777208	0.658636586490997\\
0.71	0.0803110459884098	0.392647714712651	0.657043304415871\\
0.71	0.0824155797834956	0.397540981253432	0.655472368126509\\
0.71	0.084548326099754	0.402427922432375	0.653922248592746\\
0.71	0.0867092760811502	0.407308302890799	0.65239163314657\\
0.71	0.0888984183382709	0.412181886906127	0.650879396495579\\
0.71	0.0911157389373742	0.417048438424896	0.649384574964842\\
0.71	0.0933612213899392	0.421907721096044	0.647906343724685\\
0.71	0.0956348466427212	0.42675949830445	0.646443996754806\\
0.71	0.0979365930683194	0.431603533204733	0.644996929295962\\
0.71	0.100266436456264	0.436439588755293	0.643564622546847\\
0.71	0.102624350004627	0.441267427752585	0.642146630374429\\
0.71	0.105010304312169	0.446086812865616	0.640742567819433\\
0.71	0.107424267371012	0.450897506670668	0.639352101193731\\
0.71	0.109866204559871	0.455699271686218	0.637974939582443\\
0.71	0.112336078637817	0.460491870408058	0.636610827579862\\
0.71	0.114833849738607	0.465275065344603	0.635259539104383\\
0.71	0.117359475365564	0.470048619052377	0.633920872153154\\
0.71	0.119912910387023	0.474812294171664	0.632594644371964\\
0.71	0.122494107032347	0.479565853462319	0.63128068932964\\
0.71	0.125103014888515	0.484309059839721	0.629978853399059\\
0.71	0.127739580897283	0.489041676410868	0.628690460824562\\
0.715	0	0	0.710202759607373\\
0.715	1.11327767495586e-05	0.00471862581271275	0.713003157837853\\
0.715	4.46175251031912e-05	0.009446325184051	0.715846486660142\\
0.715	0.000100583311362513	0.0141829653360114	0.718788838379794\\
0.715	0.000179158434668431	0.018928411755669	0.721886092228357\\
0.715	0.000280470402701511	0.02368252819604	0.725200636220892\\
0.715	0.000404645907256436	0.0284451766772965	0.728799085274723\\
0.715	0.000551810799695644	0.0332162174883388	0.732749123983336\\
0.715	0.000722090066287311	0.037995509188729	0.737115575253328\\
0.715	0.000915607803432999	0.0427829086109896	0.741955893907011\\
0.715	0.0011324871927904	0.047578270863273	0.747315373652289\\
0.715	0.00137285047629673	0.0523814493324038	0.75322242404581\\
0.715	0.00163681893109844	0.057192295687301	0.759684310525652\\
0.715	0.00192451284439304	0.0620106598827802	0.766683748102665\\
0.715	0.00223605148818898	0.0668363901637434	0.774176695511256\\
0.715	0.00257155309398959	0.0716693330697584	0.782091614494493\\
0.715	0.00293113482740716	0.0765093334400312	0.79033034642109\\
0.715	0.00331491276271365	0.0813562344187764	0.79877062752665\\
0.715	0.00372300185733414	0.0862098774609879	0.807270128930359\\
0.715	0.00415551592628969	0.0910701023386145	0.815671782700194\\
0.715	0.00461256761659621	0.0959367471471424	0.823810053507331\\
0.715	0.00509426838162598	0.100809648312589	0.831517746503354\\
0.715	0.00560072845543873	0.105688640598912	0.838632911419452\\
0.715	0.00613205682708918	0.11057355711583	0.845005411376292\\
0.715	0.00668836121491816	0.115464229327074	0.850502769049301\\
0.715	0.00726974804083431	0.120360487059049	0.855014975717657\\
0.715	0.00787632240459385	0.125262158509929	0.858458041096606\\
0.715	0.00850818805808555	0.13016907025918	0.860776163494182\\
0.715	0.00916544737962849	0.135081047277508	0.861942500814226\\
0.715	0.00984820134829017	0.139997912937243	0.861958614648422\\
0.715	0.0105565495182326	0.144919489023162	0.860852735617091\\
0.715	0.0112905899930939	0.149845595743738	0.858677054109652\\
0.715	0.012050419400414	0.154776051742839	0.855504274967458\\
0.715	0.0128361328661109	0.159710674111862	0.851423687967752\\
0.715	0.0136478239890177	0.164649278402306	0.846537000484339\\
0.715	0.0144855848154856	0.169591678638796	0.840954157904248\\
0.715	0.0153495058140643	0.174537687332543	0.834789345403935\\
0.715	0.0162396758502651	0.179487115495259	0.8281573258062\\
0.715	0.0171561821614173	0.184439772653511	0.821170226448694\\
0.715	0.0180991103316243	0.189395466863524	0.81393484672569\\
0.715	0.0190685442668299	0.194354004726437	0.806550519904211\\
0.715	0.0200645661700016	0.199315191403999	0.799107529846308\\
0.715	0.0210872565164405	0.204278830634725	0.791686056496194\\
0.715	0.0221366940292259	0.20924472475049	0.784355603830601\\
0.715	0.023212955654804	0.214212674693577	0.777174850281119\\
0.715	0.0243161165387281	0.219182480034174	0.770191853862644\\
0.715	0.025446250001561	0.224153938988321	0.763444541554457\\
0.715	0.0266034275149466	0.229126848436298	0.756961413910481\\
0.715	0.0277877186778607	0.234101003941464	0.750762400416763\\
0.715	0.0289991911930496	0.239076199769547	0.744859807805784\\
0.715	0.0302379108436654	0.244052228908366	0.73925931151808\\
0.715	0.0315039414701067	0.24902888308801	0.733960949044504\\
0.715	0.0327973449470747	0.254005952801449	0.728960082407261\\
0.715	0.0341181811608519	0.258983227325592	0.724248305111177\\
0.715	0.0354665079868145	0.263960494742775	0.719814276220682\\
0.715	0.0368423812671862	0.268937541962694	0.715644470612631\\
0.715	0.0382458547890421	0.273914154744766	0.711723839838041\\
0.715	0.0396769802625738	0.278890117720924	0.70803638238973\\
0.715	0.0411358072996216	0.283865214418836	0.704565625567166\\
0.715	0.0426223833924861	0.288839227285554	0.701295023640316\\
0.715	0.0441367538930258	0.293811937711588	0.698208278749353\\
0.715	0.0456789619920503	0.298783126055387	0.69528959205535\\
0.715	0.0472490486990192	0.303752571668251	0.692523853197512\\
0.715	0.0488470528220538	0.308720052919643	0.689896776228118\\
0.715	0.0504730109482718	0.313685347222913	0.687394989990281\\
0.715	0.0521269574244539	0.318648231061428	0.685006090465981\\
0.715	0.0538089243380495	0.323608480015097	0.682718662029128\\
0.715	0.0555189414985328	0.328565868787292	0.680522273853469\\
0.715	0.0572570364191156	0.333520171232163	0.678407456997849\\
0.715	0.0590232342988274	0.338471160382329	0.676365666960043\\
0.715	0.0608175580049697	0.34341860847696	0.674389235783157\\
0.715	0.0626400280559541	0.348362286990222	0.672471317134861\\
0.715	0.064490662604533	0.3533019666601	0.670605827172181\\
0.715	0.0663694774214291	0.358237417517579	0.668787383459804\\
0.715	0.0682764858793752	0.363168408916186	0.667011243730728\\
0.715	0.0702116989375697	0.368094709561873	0.665273245863649\\
0.715	0.0721751251265572	0.373016087543259	0.663569750098932\\
0.715	0.0741667705335422	0.377932310362199	0.661897584220061\\
0.715	0.0761866387881432	0.382843144964686	0.660253992184442\\
0.715	0.0782347310485954	0.38774835777208	0.658636586490997\\
0.715	0.0803110459884098	0.392647714712651	0.657043304415871\\
0.715	0.0824155797834956	0.397540981253432	0.655472368126508\\
0.715	0.084548326099754	0.402427922432375	0.653922248592745\\
0.715	0.0867092760811502	0.4073083028908	0.65239163314657\\
0.715	0.0888984183382709	0.412181886906127	0.650879396495578\\
0.715	0.0911157389373742	0.417048438424896	0.649384574964845\\
0.715	0.0933612213899392	0.421907721096044	0.647906343724685\\
0.715	0.0956348466427212	0.42675949830445	0.646443996754804\\
0.715	0.0979365930683194	0.431603533204733	0.644996929295962\\
0.715	0.100266436456264	0.436439588755293	0.643564622546847\\
0.715	0.102624350004627	0.441267427752584	0.64214663037443\\
0.715	0.105010304312169	0.446086812865616	0.640742567819433\\
0.715	0.107424267371012	0.450897506670668	0.639352101193732\\
0.715	0.109866204559871	0.455699271686218	0.637974939582443\\
0.715	0.112336078637817	0.460491870408058	0.636610827579862\\
0.715	0.114833849738607	0.465275065344603	0.635259539104382\\
0.715	0.117359475365564	0.470048619052377	0.633920872153155\\
0.715	0.119912910387023	0.474812294171664	0.632594644371966\\
0.715	0.122494107032347	0.479565853462319	0.631280689329641\\
0.715	0.125103014888515	0.484309059839721	0.629978853399058\\
0.715	0.127739580897283	0.489041676410868	0.628690460824558\\
0.72	0	0	0.710202759607373\\
0.72	1.11327767495586e-05	0.00471862581271275	0.713003157837853\\
0.72	4.46175251031912e-05	0.009446325184051	0.715846486660142\\
0.72	0.000100583311362513	0.0141829653360114	0.718788838379794\\
0.72	0.000179158434668431	0.018928411755669	0.721886092228356\\
0.72	0.000280470402701511	0.02368252819604	0.725200636220892\\
0.72	0.000404645907256436	0.0284451766772965	0.728799085274723\\
0.72	0.000551810799695644	0.0332162174883389	0.732749123983336\\
0.72	0.000722090066287311	0.037995509188729	0.737115575253327\\
0.72	0.000915607803432999	0.0427829086109896	0.741955893907011\\
0.72	0.0011324871927904	0.0475782708632729	0.747315373652289\\
0.72	0.00137285047629673	0.0523814493324038	0.753222424045811\\
0.72	0.00163681893109844	0.057192295687301	0.759684310525652\\
0.72	0.00192451284439304	0.0620106598827802	0.766683748102665\\
0.72	0.00223605148818899	0.0668363901637434	0.774176695511256\\
0.72	0.00257155309398959	0.0716693330697584	0.782091614494493\\
0.72	0.00293113482740716	0.0765093334400312	0.79033034642109\\
0.72	0.00331491276271365	0.0813562344187764	0.79877062752665\\
0.72	0.00372300185733413	0.0862098774609879	0.80727012893036\\
0.72	0.00415551592628969	0.0910701023386145	0.815671782700193\\
0.72	0.00461256761659621	0.0959367471471425	0.82381005350733\\
0.72	0.00509426838162598	0.100809648312589	0.831517746503354\\
0.72	0.00560072845543873	0.105688640598912	0.838632911419452\\
0.72	0.00613205682708918	0.11057355711583	0.845005411376292\\
0.72	0.00668836121491816	0.115464229327074	0.850502769049302\\
0.72	0.00726974804083431	0.120360487059049	0.855014975717657\\
0.72	0.00787632240459385	0.125262158509929	0.858458041096606\\
0.72	0.00850818805808555	0.13016907025918	0.860776163494182\\
0.72	0.00916544737962849	0.135081047277508	0.861942500814223\\
0.72	0.00984820134829017	0.139997912937243	0.86195861464842\\
0.72	0.0105565495182326	0.144919489023162	0.86085273561709\\
0.72	0.0112905899930939	0.149845595743738	0.858677054109649\\
0.72	0.012050419400414	0.154776051742839	0.855504274967456\\
0.72	0.0128361328661109	0.159710674111862	0.851423687967753\\
0.72	0.0136478239890177	0.164649278402306	0.846537000484341\\
0.72	0.0144855848154856	0.169591678638796	0.840954157904249\\
0.72	0.0153495058140643	0.174537687332543	0.834789345403936\\
0.72	0.0162396758502651	0.179487115495259	0.8281573258062\\
0.72	0.0171561821614173	0.184439772653511	0.821170226448695\\
0.72	0.0180991103316243	0.189395466863524	0.813934846725692\\
0.72	0.0190685442668299	0.194354004726437	0.806550519904213\\
0.72	0.0200645661700016	0.199315191403999	0.799107529846305\\
0.72	0.0210872565164405	0.204278830634725	0.791686056496193\\
0.72	0.0221366940292259	0.20924472475049	0.784355603830604\\
0.72	0.023212955654804	0.214212674693577	0.777174850281118\\
0.72	0.0243161165387281	0.219182480034174	0.770191853862641\\
0.72	0.025446250001561	0.224153938988321	0.763444541554457\\
0.72	0.0266034275149466	0.229126848436298	0.756961413910483\\
0.72	0.0277877186778607	0.234101003941464	0.750762400416765\\
0.72	0.0289991911930496	0.239076199769547	0.744859807805785\\
0.72	0.0302379108436654	0.244052228908366	0.739259311518079\\
0.72	0.0315039414701067	0.24902888308801	0.733960949044503\\
0.72	0.0327973449470747	0.254005952801449	0.728960082407262\\
0.72	0.0341181811608519	0.258983227325592	0.724248305111178\\
0.72	0.0354665079868145	0.263960494742775	0.719814276220681\\
0.72	0.0368423812671862	0.268937541962694	0.715644470612629\\
0.72	0.0382458547890421	0.273914154744766	0.711723839838041\\
0.72	0.0396769802625738	0.278890117720924	0.708036382389731\\
0.72	0.0411358072996216	0.283865214418836	0.704565625567164\\
0.72	0.0426223833924861	0.288839227285554	0.701295023640316\\
0.72	0.0441367538930258	0.293811937711588	0.698208278749355\\
0.72	0.0456789619920503	0.298783126055387	0.695289592055349\\
0.72	0.0472490486990192	0.303752571668251	0.692523853197511\\
0.72	0.0488470528220537	0.308720052919643	0.689896776228118\\
0.72	0.0504730109482718	0.313685347222913	0.687394989990281\\
0.72	0.0521269574244539	0.318648231061428	0.685006090465982\\
0.72	0.0538089243380495	0.323608480015097	0.682718662029128\\
0.72	0.0555189414985328	0.328565868787292	0.680522273853469\\
0.72	0.0572570364191156	0.333520171232163	0.678407456997851\\
0.72	0.0590232342988274	0.338471160382329	0.676365666960044\\
0.72	0.0608175580049697	0.34341860847696	0.674389235783156\\
0.72	0.0626400280559541	0.348362286990222	0.67247131713486\\
0.72	0.064490662604533	0.3533019666601	0.670605827172181\\
0.72	0.066369477421429	0.358237417517579	0.668787383459805\\
0.72	0.0682764858793752	0.363168408916186	0.667011243730728\\
0.72	0.0702116989375697	0.368094709561873	0.665273245863648\\
0.72	0.0721751251265572	0.373016087543259	0.663569750098931\\
0.72	0.0741667705335422	0.377932310362199	0.661897584220061\\
0.72	0.0761866387881432	0.382843144964686	0.660253992184443\\
0.72	0.0782347310485954	0.38774835777208	0.658636586490997\\
0.72	0.0803110459884098	0.392647714712651	0.657043304415872\\
0.72	0.0824155797834956	0.397540981253432	0.655472368126511\\
0.72	0.084548326099754	0.402427922432375	0.653922248592744\\
0.72	0.0867092760811502	0.407308302890799	0.652391633146569\\
0.72	0.0888984183382709	0.412181886906127	0.650879396495579\\
0.72	0.0911157389373742	0.417048438424896	0.649384574964843\\
0.72	0.0933612213899392	0.421907721096044	0.647906343724684\\
0.72	0.0956348466427212	0.42675949830445	0.646443996754806\\
0.72	0.0979365930683194	0.431603533204733	0.644996929295963\\
0.72	0.100266436456264	0.436439588755293	0.643564622546846\\
0.72	0.102624350004627	0.441267427752584	0.64214663037443\\
0.72	0.105010304312169	0.446086812865616	0.640742567819432\\
0.72	0.107424267371012	0.450897506670668	0.63935210119373\\
0.72	0.109866204559871	0.455699271686218	0.637974939582443\\
0.72	0.112336078637817	0.460491870408058	0.636610827579862\\
0.72	0.114833849738607	0.465275065344603	0.635259539104383\\
0.72	0.117359475365564	0.470048619052377	0.633920872153154\\
0.72	0.119912910387022	0.474812294171664	0.632594644371965\\
0.72	0.122494107032347	0.479565853462319	0.631280689329641\\
0.72	0.125103014888515	0.484309059839721	0.629978853399058\\
0.72	0.127739580897283	0.489041676410868	0.628690460824559\\
0.725	0	0	0.710202759607373\\
0.725	1.11327767495586e-05	0.00471862581271275	0.713003157837853\\
0.725	4.46175251031912e-05	0.009446325184051	0.715846486660142\\
0.725	0.000100583311362513	0.0141829653360114	0.718788838379794\\
0.725	0.000179158434668431	0.018928411755669	0.721886092228356\\
0.725	0.000280470402701511	0.02368252819604	0.725200636220892\\
0.725	0.000404645907256436	0.0284451766772965	0.728799085274723\\
0.725	0.000551810799695644	0.0332162174883389	0.732749123983336\\
0.725	0.000722090066287311	0.037995509188729	0.737115575253328\\
0.725	0.000915607803432999	0.0427829086109896	0.741955893907011\\
0.725	0.0011324871927904	0.0475782708632729	0.747315373652289\\
0.725	0.00137285047629673	0.0523814493324038	0.753222424045811\\
0.725	0.00163681893109844	0.057192295687301	0.759684310525652\\
0.725	0.00192451284439304	0.0620106598827802	0.766683748102665\\
0.725	0.00223605148818899	0.0668363901637434	0.774176695511257\\
0.725	0.00257155309398959	0.0716693330697584	0.782091614494493\\
0.725	0.00293113482740716	0.0765093334400312	0.79033034642109\\
0.725	0.00331491276271365	0.0813562344187764	0.79877062752665\\
0.725	0.00372300185733413	0.0862098774609879	0.807270128930359\\
0.725	0.00415551592628969	0.0910701023386145	0.815671782700193\\
0.725	0.00461256761659621	0.0959367471471425	0.82381005350733\\
0.725	0.00509426838162598	0.100809648312589	0.831517746503354\\
0.725	0.00560072845543873	0.105688640598912	0.838632911419452\\
0.725	0.00613205682708918	0.11057355711583	0.845005411376293\\
0.725	0.00668836121491816	0.115464229327074	0.850502769049302\\
0.725	0.00726974804083431	0.120360487059049	0.855014975717657\\
0.725	0.00787632240459385	0.125262158509929	0.858458041096607\\
0.725	0.00850818805808554	0.13016907025918	0.860776163494181\\
0.725	0.00916544737962849	0.135081047277508	0.861942500814221\\
0.725	0.00984820134829017	0.139997912937243	0.861958614648419\\
0.725	0.0105565495182326	0.144919489023162	0.860852735617091\\
0.725	0.0112905899930939	0.149845595743738	0.858677054109649\\
0.725	0.012050419400414	0.154776051742839	0.855504274967456\\
0.725	0.0128361328661109	0.159710674111862	0.851423687967755\\
0.725	0.0136478239890177	0.164649278402306	0.84653700048434\\
0.725	0.0144855848154856	0.169591678638796	0.840954157904249\\
0.725	0.0153495058140643	0.174537687332543	0.834789345403935\\
0.725	0.0162396758502651	0.179487115495259	0.8281573258062\\
0.725	0.0171561821614173	0.184439772653511	0.821170226448693\\
0.725	0.0180991103316243	0.189395466863524	0.813934846725693\\
0.725	0.0190685442668299	0.194354004726437	0.806550519904212\\
0.725	0.0200645661700016	0.199315191403999	0.799107529846305\\
0.725	0.0210872565164405	0.204278830634725	0.791686056496196\\
0.725	0.0221366940292259	0.20924472475049	0.784355603830604\\
0.725	0.023212955654804	0.214212674693577	0.777174850281116\\
0.725	0.0243161165387281	0.219182480034174	0.770191853862643\\
0.725	0.025446250001561	0.224153938988321	0.763444541554459\\
0.725	0.0266034275149466	0.229126848436298	0.756961413910485\\
0.725	0.0277877186778607	0.234101003941464	0.750762400416764\\
0.725	0.0289991911930496	0.239076199769547	0.744859807805784\\
0.725	0.0302379108436654	0.244052228908366	0.739259311518079\\
0.725	0.0315039414701067	0.24902888308801	0.733960949044505\\
0.725	0.0327973449470747	0.254005952801449	0.72896008240726\\
0.725	0.0341181811608519	0.258983227325591	0.724248305111179\\
0.725	0.0354665079868145	0.263960494742775	0.719814276220682\\
0.725	0.0368423812671862	0.268937541962694	0.715644470612628\\
0.725	0.0382458547890421	0.273914154744766	0.711723839838043\\
0.725	0.0396769802625738	0.278890117720924	0.708036382389731\\
0.725	0.0411358072996216	0.283865214418836	0.704565625567163\\
0.725	0.0426223833924861	0.288839227285554	0.701295023640316\\
0.725	0.0441367538930258	0.293811937711588	0.698208278749353\\
0.725	0.0456789619920503	0.298783126055387	0.69528959205535\\
0.725	0.0472490486990192	0.303752571668251	0.692523853197513\\
0.725	0.0488470528220538	0.308720052919643	0.689896776228118\\
0.725	0.0504730109482718	0.313685347222913	0.68739498999028\\
0.725	0.0521269574244539	0.318648231061427	0.68500609046598\\
0.725	0.0538089243380495	0.323608480015096	0.682718662029129\\
0.725	0.0555189414985328	0.328565868787292	0.680522273853469\\
0.725	0.0572570364191156	0.333520171232163	0.678407456997849\\
0.725	0.0590232342988274	0.338471160382329	0.676365666960044\\
0.725	0.0608175580049697	0.34341860847696	0.674389235783157\\
0.725	0.0626400280559541	0.348362286990222	0.672471317134863\\
0.725	0.064490662604533	0.3533019666601	0.67060582717218\\
0.725	0.0663694774214291	0.358237417517579	0.668787383459804\\
0.725	0.0682764858793752	0.363168408916186	0.667011243730729\\
0.725	0.0702116989375697	0.368094709561873	0.66527324586365\\
0.725	0.0721751251265572	0.373016087543259	0.663569750098932\\
0.725	0.0741667705335422	0.377932310362199	0.66189758422006\\
0.725	0.0761866387881432	0.382843144964686	0.660253992184442\\
0.725	0.0782347310485954	0.38774835777208	0.658636586490996\\
0.725	0.0803110459884098	0.392647714712651	0.657043304415871\\
0.725	0.0824155797834956	0.397540981253432	0.65547236812651\\
0.725	0.084548326099754	0.402427922432375	0.653922248592747\\
0.725	0.0867092760811502	0.4073083028908	0.65239163314657\\
0.725	0.0888984183382709	0.412181886906127	0.650879396495579\\
0.725	0.0911157389373742	0.417048438424897	0.649384574964845\\
0.725	0.0933612213899392	0.421907721096044	0.647906343724683\\
0.725	0.0956348466427212	0.42675949830445	0.646443996754804\\
0.725	0.0979365930683194	0.431603533204733	0.644996929295962\\
0.725	0.100266436456264	0.436439588755293	0.643564622546848\\
0.725	0.102624350004627	0.441267427752585	0.642146630374429\\
0.725	0.105010304312169	0.446086812865616	0.640742567819434\\
0.725	0.107424267371012	0.450897506670668	0.63935210119373\\
0.725	0.109866204559871	0.455699271686218	0.637974939582442\\
0.725	0.112336078637817	0.460491870408058	0.636610827579861\\
0.725	0.114833849738607	0.465275065344603	0.635259539104383\\
0.725	0.117359475365564	0.470048619052377	0.633920872153156\\
0.725	0.119912910387022	0.474812294171664	0.632594644371964\\
0.725	0.122494107032347	0.479565853462319	0.631280689329641\\
0.725	0.125103014888515	0.484309059839721	0.629978853399058\\
0.725	0.127739580897283	0.489041676410868	0.628690460824556\\
0.73	0	0	0.710202759607373\\
0.73	1.11327767495586e-05	0.00471862581271275	0.713003157837853\\
0.73	4.46175251031912e-05	0.009446325184051	0.715846486660142\\
0.73	0.000100583311362513	0.0141829653360114	0.718788838379794\\
0.73	0.000179158434668431	0.018928411755669	0.721886092228356\\
0.73	0.000280470402701511	0.02368252819604	0.725200636220892\\
0.73	0.000404645907256436	0.0284451766772965	0.728799085274723\\
0.73	0.000551810799695644	0.0332162174883389	0.732749123983336\\
0.73	0.000722090066287311	0.037995509188729	0.737115575253328\\
0.73	0.000915607803432999	0.0427829086109896	0.741955893907011\\
0.73	0.0011324871927904	0.0475782708632729	0.747315373652289\\
0.73	0.00137285047629673	0.0523814493324038	0.753222424045811\\
0.73	0.00163681893109844	0.057192295687301	0.759684310525652\\
0.73	0.00192451284439304	0.0620106598827802	0.766683748102665\\
0.73	0.00223605148818899	0.0668363901637434	0.774176695511256\\
0.73	0.00257155309398959	0.0716693330697584	0.782091614494494\\
0.73	0.00293113482740716	0.0765093334400312	0.79033034642109\\
0.73	0.00331491276271365	0.0813562344187764	0.79877062752665\\
0.73	0.00372300185733414	0.0862098774609879	0.807270128930359\\
0.73	0.00415551592628969	0.0910701023386145	0.815671782700193\\
0.73	0.00461256761659621	0.0959367471471425	0.823810053507331\\
0.73	0.00509426838162598	0.100809648312589	0.831517746503353\\
0.73	0.00560072845543873	0.105688640598912	0.838632911419452\\
0.73	0.00613205682708918	0.11057355711583	0.845005411376293\\
0.73	0.00668836121491816	0.115464229327074	0.850502769049301\\
0.73	0.00726974804083431	0.120360487059049	0.855014975717657\\
0.73	0.00787632240459385	0.125262158509929	0.858458041096608\\
0.73	0.00850818805808555	0.13016907025918	0.860776163494182\\
0.73	0.00916544737962849	0.135081047277508	0.861942500814226\\
0.73	0.00984820134829017	0.139997912937243	0.861958614648418\\
0.73	0.0105565495182326	0.144919489023162	0.860852735617091\\
0.73	0.0112905899930939	0.149845595743738	0.858677054109649\\
0.73	0.012050419400414	0.154776051742839	0.855504274967457\\
0.73	0.0128361328661109	0.159710674111862	0.851423687967755\\
0.73	0.0136478239890177	0.164649278402306	0.846537000484342\\
0.73	0.0144855848154856	0.169591678638796	0.840954157904248\\
0.73	0.0153495058140643	0.174537687332543	0.834789345403934\\
0.73	0.0162396758502651	0.179487115495259	0.828157325806199\\
0.73	0.0171561821614173	0.184439772653511	0.821170226448696\\
0.73	0.0180991103316243	0.189395466863524	0.813934846725693\\
0.73	0.0190685442668299	0.194354004726437	0.806550519904213\\
0.73	0.0200645661700016	0.199315191403999	0.799107529846306\\
0.73	0.0210872565164405	0.204278830634725	0.791686056496196\\
0.73	0.0221366940292259	0.20924472475049	0.784355603830601\\
0.73	0.023212955654804	0.214212674693577	0.777174850281116\\
0.73	0.0243161165387281	0.219182480034174	0.770191853862644\\
0.73	0.025446250001561	0.224153938988321	0.76344454155446\\
0.73	0.0266034275149466	0.229126848436298	0.756961413910482\\
0.73	0.0277877186778607	0.234101003941464	0.750762400416763\\
0.73	0.0289991911930496	0.239076199769547	0.744859807805784\\
0.73	0.0302379108436654	0.244052228908366	0.73925931151808\\
0.73	0.0315039414701067	0.24902888308801	0.733960949044504\\
0.73	0.0327973449470747	0.254005952801449	0.728960082407261\\
0.73	0.0341181811608519	0.258983227325592	0.724248305111178\\
0.73	0.0354665079868145	0.263960494742775	0.71981427622068\\
0.73	0.0368423812671862	0.268937541962694	0.71564447061263\\
0.73	0.0382458547890421	0.273914154744766	0.711723839838041\\
0.73	0.0396769802625738	0.278890117720924	0.70803638238973\\
0.73	0.0411358072996216	0.283865214418836	0.704565625567167\\
0.73	0.0426223833924861	0.288839227285554	0.701295023640316\\
0.73	0.0441367538930258	0.293811937711588	0.698208278749352\\
0.73	0.0456789619920503	0.298783126055387	0.695289592055349\\
0.73	0.0472490486990192	0.303752571668251	0.692523853197513\\
0.73	0.0488470528220538	0.308720052919643	0.689896776228119\\
0.73	0.0504730109482718	0.313685347222913	0.687394989990281\\
0.73	0.0521269574244539	0.318648231061428	0.685006090465981\\
0.73	0.0538089243380495	0.323608480015097	0.682718662029128\\
0.73	0.0555189414985328	0.328565868787292	0.680522273853469\\
0.73	0.0572570364191156	0.333520171232163	0.678407456997849\\
0.73	0.0590232342988274	0.338471160382329	0.676365666960044\\
0.73	0.0608175580049697	0.34341860847696	0.674389235783155\\
0.73	0.0626400280559541	0.348362286990222	0.672471317134862\\
0.73	0.064490662604533	0.3533019666601	0.670605827172183\\
0.73	0.0663694774214291	0.358237417517579	0.668787383459802\\
0.73	0.0682764858793752	0.363168408916186	0.667011243730727\\
0.73	0.0702116989375697	0.368094709561873	0.66527324586365\\
0.73	0.0721751251265572	0.373016087543259	0.663569750098933\\
0.73	0.0741667705335422	0.377932310362199	0.661897584220061\\
0.73	0.0761866387881432	0.382843144964686	0.660253992184443\\
0.73	0.0782347310485954	0.38774835777208	0.658636586490996\\
0.73	0.0803110459884098	0.392647714712651	0.65704330441587\\
0.73	0.0824155797834956	0.397540981253432	0.655472368126509\\
0.73	0.084548326099754	0.402427922432375	0.653922248592745\\
0.73	0.0867092760811502	0.407308302890799	0.65239163314657\\
0.73	0.0888984183382709	0.412181886906127	0.65087939649558\\
0.73	0.0911157389373742	0.417048438424896	0.649384574964845\\
0.73	0.0933612213899392	0.421907721096044	0.647906343724685\\
0.73	0.0956348466427212	0.42675949830445	0.646443996754805\\
0.73	0.0979365930683194	0.431603533204733	0.644996929295962\\
0.73	0.100266436456264	0.436439588755293	0.643564622546847\\
0.73	0.102624350004627	0.441267427752584	0.64214663037443\\
0.73	0.105010304312169	0.446086812865616	0.640742567819433\\
0.73	0.107424267371012	0.450897506670668	0.63935210119373\\
0.73	0.109866204559871	0.455699271686218	0.637974939582443\\
0.73	0.112336078637817	0.460491870408058	0.636610827579863\\
0.73	0.114833849738607	0.465275065344603	0.635259539104381\\
0.73	0.117359475365564	0.470048619052377	0.633920872153155\\
0.73	0.119912910387023	0.474812294171664	0.632594644371966\\
0.73	0.122494107032347	0.479565853462319	0.631280689329641\\
0.73	0.125103014888515	0.484309059839721	0.629978853399059\\
0.73	0.127739580897283	0.489041676410868	0.628690460824555\\
0.735	0	0	0.710202759607373\\
0.735	1.11327767495586e-05	0.00471862581271275	0.713003157837853\\
0.735	4.46175251031912e-05	0.009446325184051	0.715846486660142\\
0.735	0.000100583311362513	0.0141829653360114	0.718788838379794\\
0.735	0.000179158434668431	0.018928411755669	0.721886092228356\\
0.735	0.000280470402701511	0.02368252819604	0.725200636220892\\
0.735	0.000404645907256436	0.0284451766772965	0.728799085274723\\
0.735	0.000551810799695644	0.0332162174883389	0.732749123983336\\
0.735	0.000722090066287311	0.037995509188729	0.737115575253328\\
0.735	0.000915607803432999	0.0427829086109896	0.741955893907011\\
0.735	0.0011324871927904	0.0475782708632729	0.747315373652289\\
0.735	0.00137285047629673	0.0523814493324038	0.753222424045811\\
0.735	0.00163681893109844	0.057192295687301	0.759684310525652\\
0.735	0.00192451284439304	0.0620106598827802	0.766683748102665\\
0.735	0.00223605148818898	0.0668363901637434	0.774176695511256\\
0.735	0.00257155309398959	0.0716693330697584	0.782091614494493\\
0.735	0.00293113482740716	0.0765093334400312	0.79033034642109\\
0.735	0.00331491276271365	0.0813562344187764	0.798770627526651\\
0.735	0.00372300185733413	0.0862098774609879	0.807270128930359\\
0.735	0.00415551592628969	0.0910701023386145	0.815671782700193\\
0.735	0.00461256761659621	0.0959367471471425	0.823810053507331\\
0.735	0.00509426838162598	0.100809648312589	0.831517746503355\\
0.735	0.00560072845543873	0.105688640598912	0.838632911419452\\
0.735	0.00613205682708918	0.11057355711583	0.845005411376292\\
0.735	0.00668836121491816	0.115464229327074	0.850502769049301\\
0.735	0.00726974804083431	0.120360487059049	0.855014975717656\\
0.735	0.00787632240459385	0.125262158509929	0.858458041096605\\
0.735	0.00850818805808554	0.13016907025918	0.860776163494183\\
0.735	0.00916544737962849	0.135081047277508	0.861942500814226\\
0.735	0.00984820134829017	0.139997912937243	0.86195861464842\\
0.735	0.0105565495182326	0.144919489023162	0.860852735617088\\
0.735	0.0112905899930939	0.149845595743738	0.858677054109652\\
0.735	0.012050419400414	0.154776051742839	0.855504274967459\\
0.735	0.0128361328661109	0.159710674111862	0.851423687967757\\
0.735	0.0136478239890177	0.164649278402306	0.846537000484341\\
0.735	0.0144855848154856	0.169591678638796	0.840954157904248\\
0.735	0.0153495058140643	0.174537687332543	0.834789345403935\\
0.735	0.0162396758502651	0.179487115495259	0.828157325806202\\
0.735	0.0171561821614173	0.184439772653511	0.821170226448697\\
0.735	0.0180991103316243	0.189395466863524	0.813934846725691\\
0.735	0.0190685442668299	0.194354004726437	0.806550519904212\\
0.735	0.0200645661700016	0.199315191403999	0.799107529846303\\
0.735	0.0210872565164405	0.204278830634725	0.791686056496194\\
0.735	0.0221366940292259	0.20924472475049	0.784355603830602\\
0.735	0.023212955654804	0.214212674693577	0.777174850281118\\
0.735	0.0243161165387281	0.219182480034174	0.770191853862644\\
0.735	0.025446250001561	0.224153938988321	0.763444541554458\\
0.735	0.0266034275149466	0.229126848436298	0.756961413910482\\
0.735	0.0277877186778607	0.234101003941464	0.750762400416764\\
0.735	0.0289991911930496	0.239076199769547	0.744859807805785\\
0.735	0.0302379108436654	0.244052228908366	0.739259311518079\\
0.735	0.0315039414701067	0.24902888308801	0.733960949044504\\
0.735	0.0327973449470747	0.254005952801449	0.728960082407262\\
0.735	0.0341181811608519	0.258983227325592	0.724248305111177\\
0.735	0.0354665079868145	0.263960494742775	0.719814276220681\\
0.735	0.0368423812671862	0.268937541962694	0.715644470612631\\
0.735	0.0382458547890421	0.273914154744766	0.711723839838041\\
0.735	0.0396769802625738	0.278890117720924	0.708036382389729\\
0.735	0.0411358072996216	0.283865214418836	0.704565625567166\\
0.735	0.0426223833924861	0.288839227285554	0.701295023640316\\
0.735	0.0441367538930258	0.293811937711588	0.698208278749352\\
0.735	0.0456789619920503	0.298783126055387	0.695289592055349\\
0.735	0.0472490486990192	0.303752571668251	0.692523853197513\\
0.735	0.0488470528220538	0.308720052919643	0.68989677622812\\
0.735	0.0504730109482718	0.313685347222913	0.687394989990281\\
0.735	0.0521269574244539	0.318648231061428	0.68500609046598\\
0.735	0.0538089243380495	0.323608480015096	0.682718662029128\\
0.735	0.0555189414985328	0.328565868787292	0.68052227385347\\
0.735	0.0572570364191156	0.333520171232163	0.678407456997849\\
0.735	0.0590232342988274	0.338471160382329	0.676365666960043\\
0.735	0.0608175580049697	0.34341860847696	0.674389235783156\\
0.735	0.0626400280559541	0.348362286990222	0.672471317134862\\
0.735	0.064490662604533	0.3533019666601	0.670605827172183\\
0.735	0.066369477421429	0.358237417517579	0.668787383459804\\
0.735	0.0682764858793752	0.363168408916186	0.667011243730727\\
0.735	0.0702116989375697	0.368094709561873	0.665273245863649\\
0.735	0.0721751251265572	0.373016087543259	0.663569750098932\\
0.735	0.0741667705335422	0.377932310362199	0.66189758422006\\
0.735	0.0761866387881432	0.382843144964686	0.660253992184442\\
0.735	0.0782347310485954	0.38774835777208	0.658636586490997\\
0.735	0.0803110459884098	0.392647714712651	0.657043304415871\\
0.735	0.0824155797834956	0.397540981253432	0.655472368126508\\
0.735	0.084548326099754	0.402427922432375	0.653922248592746\\
0.735	0.0867092760811502	0.4073083028908	0.652391633146569\\
0.735	0.0888984183382709	0.412181886906127	0.650879396495579\\
0.735	0.0911157389373742	0.417048438424897	0.649384574964844\\
0.735	0.0933612213899392	0.421907721096044	0.647906343724684\\
0.735	0.0956348466427212	0.42675949830445	0.646443996754806\\
0.735	0.0979365930683194	0.431603533204733	0.644996929295963\\
0.735	0.100266436456264	0.436439588755293	0.643564622546846\\
0.735	0.102624350004627	0.441267427752584	0.64214663037443\\
0.735	0.105010304312169	0.446086812865616	0.640742567819433\\
0.735	0.107424267371012	0.450897506670668	0.639352101193729\\
0.735	0.109866204559871	0.455699271686218	0.637974939582442\\
0.735	0.112336078637817	0.460491870408058	0.636610827579863\\
0.735	0.114833849738607	0.465275065344603	0.635259539104383\\
0.735	0.117359475365564	0.470048619052377	0.633920872153153\\
0.735	0.119912910387022	0.474812294171664	0.632594644371965\\
0.735	0.122494107032347	0.479565853462319	0.631280689329641\\
0.735	0.125103014888515	0.484309059839721	0.62997885339906\\
0.735	0.127739580897283	0.489041676410868	0.62869046082456\\
0.74	0	0	0.710202759607373\\
0.74	1.11327767495586e-05	0.00471862581271275	0.713003157837853\\
0.74	4.46175251031912e-05	0.009446325184051	0.715846486660142\\
0.74	0.000100583311362513	0.0141829653360114	0.718788838379794\\
0.74	0.000179158434668431	0.018928411755669	0.721886092228356\\
0.74	0.000280470402701511	0.02368252819604	0.725200636220892\\
0.74	0.000404645907256436	0.0284451766772965	0.728799085274723\\
0.74	0.000551810799695644	0.0332162174883388	0.732749123983336\\
0.74	0.000722090066287311	0.037995509188729	0.737115575253327\\
0.74	0.000915607803432999	0.0427829086109896	0.741955893907011\\
0.74	0.0011324871927904	0.0475782708632729	0.747315373652289\\
0.74	0.00137285047629673	0.0523814493324038	0.753222424045811\\
0.74	0.00163681893109843	0.057192295687301	0.759684310525652\\
0.74	0.00192451284439304	0.0620106598827802	0.766683748102665\\
0.74	0.00223605148818899	0.0668363901637434	0.774176695511256\\
0.74	0.00257155309398959	0.0716693330697584	0.782091614494494\\
0.74	0.00293113482740716	0.0765093334400312	0.79033034642109\\
0.74	0.00331491276271365	0.0813562344187764	0.79877062752665\\
0.74	0.00372300185733413	0.0862098774609879	0.807270128930359\\
0.74	0.00415551592628969	0.0910701023386145	0.815671782700193\\
0.74	0.00461256761659621	0.0959367471471425	0.823810053507329\\
0.74	0.00509426838162598	0.100809648312589	0.831517746503354\\
0.74	0.00560072845543873	0.105688640598912	0.838632911419452\\
0.74	0.00613205682708918	0.11057355711583	0.845005411376292\\
0.74	0.00668836121491815	0.115464229327074	0.850502769049302\\
0.74	0.00726974804083431	0.120360487059049	0.855014975717656\\
0.74	0.00787632240459385	0.125262158509929	0.858458041096605\\
0.74	0.00850818805808555	0.13016907025918	0.86077616349418\\
0.74	0.00916544737962849	0.135081047277508	0.861942500814224\\
0.74	0.00984820134829017	0.139997912937243	0.861958614648422\\
0.74	0.0105565495182326	0.144919489023162	0.860852735617092\\
0.74	0.0112905899930939	0.149845595743738	0.858677054109652\\
0.74	0.012050419400414	0.154776051742839	0.855504274967457\\
0.74	0.0128361328661109	0.159710674111862	0.851423687967754\\
0.74	0.0136478239890177	0.164649278402306	0.84653700048434\\
0.74	0.0144855848154856	0.169591678638796	0.840954157904249\\
0.74	0.0153495058140643	0.174537687332543	0.834789345403936\\
0.74	0.0162396758502651	0.179487115495259	0.828157325806201\\
0.74	0.0171561821614173	0.184439772653511	0.821170226448693\\
0.74	0.0180991103316243	0.189395466863524	0.81393484672569\\
0.74	0.0190685442668299	0.194354004726437	0.80655051990421\\
0.74	0.0200645661700016	0.199315191403999	0.799107529846305\\
0.74	0.0210872565164405	0.204278830634725	0.791686056496196\\
0.74	0.0221366940292259	0.20924472475049	0.784355603830601\\
0.74	0.023212955654804	0.214212674693577	0.777174850281117\\
0.74	0.0243161165387281	0.219182480034174	0.770191853862643\\
0.74	0.025446250001561	0.224153938988321	0.763444541554458\\
0.74	0.0266034275149466	0.229126848436298	0.756961413910484\\
0.74	0.0277877186778607	0.234101003941464	0.750762400416764\\
0.74	0.0289991911930496	0.239076199769547	0.744859807805784\\
0.74	0.0302379108436654	0.244052228908366	0.739259311518081\\
0.74	0.0315039414701067	0.24902888308801	0.733960949044505\\
0.74	0.0327973449470747	0.254005952801449	0.728960082407261\\
0.74	0.0341181811608519	0.258983227325592	0.724248305111177\\
0.74	0.0354665079868145	0.263960494742775	0.719814276220681\\
0.74	0.0368423812671862	0.268937541962694	0.71564447061263\\
0.74	0.0382458547890421	0.273914154744766	0.711723839838041\\
0.74	0.0396769802625738	0.278890117720924	0.708036382389731\\
0.74	0.0411358072996216	0.283865214418836	0.704565625567165\\
0.74	0.0426223833924861	0.288839227285554	0.701295023640316\\
0.74	0.0441367538930258	0.293811937711588	0.698208278749354\\
0.74	0.0456789619920503	0.298783126055387	0.695289592055349\\
0.74	0.0472490486990192	0.303752571668251	0.69252385319751\\
0.74	0.0488470528220538	0.308720052919643	0.689896776228118\\
0.74	0.0504730109482718	0.313685347222913	0.687394989990283\\
0.74	0.0521269574244539	0.318648231061428	0.685006090465982\\
0.74	0.0538089243380495	0.323608480015097	0.682718662029127\\
0.74	0.0555189414985328	0.328565868787292	0.68052227385347\\
0.74	0.0572570364191156	0.333520171232163	0.678407456997851\\
0.74	0.0590232342988274	0.338471160382329	0.676365666960043\\
0.74	0.0608175580049697	0.34341860847696	0.674389235783156\\
0.74	0.0626400280559541	0.348362286990222	0.672471317134861\\
0.74	0.064490662604533	0.3533019666601	0.670605827172182\\
0.74	0.0663694774214291	0.358237417517579	0.668787383459804\\
0.74	0.0682764858793752	0.363168408916186	0.667011243730728\\
0.74	0.0702116989375697	0.368094709561873	0.66527324586365\\
0.74	0.0721751251265572	0.373016087543259	0.663569750098932\\
0.74	0.0741667705335423	0.377932310362199	0.66189758422006\\
0.74	0.0761866387881432	0.382843144964686	0.660253992184442\\
0.74	0.0782347310485954	0.38774835777208	0.658636586490997\\
0.74	0.0803110459884098	0.392647714712651	0.657043304415872\\
0.74	0.0824155797834956	0.397540981253432	0.655472368126508\\
0.74	0.084548326099754	0.402427922432375	0.653922248592745\\
0.74	0.0867092760811502	0.407308302890799	0.65239163314657\\
0.74	0.0888984183382709	0.412181886906127	0.65087939649558\\
0.74	0.0911157389373742	0.417048438424896	0.649384574964845\\
0.74	0.0933612213899392	0.421907721096044	0.647906343724685\\
0.74	0.0956348466427212	0.42675949830445	0.646443996754804\\
0.74	0.0979365930683194	0.431603533204733	0.644996929295962\\
0.74	0.100266436456264	0.436439588755293	0.643564622546848\\
0.74	0.102624350004627	0.441267427752585	0.642146630374429\\
0.74	0.105010304312169	0.446086812865616	0.640742567819433\\
0.74	0.107424267371012	0.450897506670668	0.639352101193731\\
0.74	0.109866204559871	0.455699271686218	0.637974939582442\\
0.74	0.112336078637817	0.460491870408058	0.636610827579862\\
0.74	0.114833849738607	0.465275065344603	0.635259539104384\\
0.74	0.117359475365564	0.470048619052377	0.633920872153155\\
0.74	0.119912910387022	0.474812294171664	0.632594644371964\\
0.74	0.122494107032347	0.479565853462319	0.63128068932964\\
0.74	0.125103014888515	0.484309059839721	0.629978853399059\\
0.74	0.127739580897283	0.489041676410868	0.628690460824558\\
0.745	0	0	0.710202759607373\\
0.745	1.11327767495586e-05	0.00471862581271275	0.713003157837853\\
0.745	4.46175251031912e-05	0.009446325184051	0.715846486660142\\
0.745	0.000100583311362513	0.0141829653360114	0.718788838379794\\
0.745	0.000179158434668431	0.018928411755669	0.721886092228356\\
0.745	0.000280470402701511	0.02368252819604	0.725200636220892\\
0.745	0.000404645907256436	0.0284451766772965	0.728799085274723\\
0.745	0.000551810799695644	0.0332162174883389	0.732749123983336\\
0.745	0.000722090066287311	0.037995509188729	0.737115575253327\\
0.745	0.000915607803432999	0.0427829086109896	0.741955893907011\\
0.745	0.0011324871927904	0.0475782708632729	0.747315373652289\\
0.745	0.00137285047629673	0.0523814493324038	0.753222424045811\\
0.745	0.00163681893109844	0.057192295687301	0.759684310525652\\
0.745	0.00192451284439304	0.0620106598827802	0.766683748102665\\
0.745	0.00223605148818898	0.0668363901637434	0.774176695511256\\
0.745	0.00257155309398959	0.0716693330697584	0.782091614494494\\
0.745	0.00293113482740716	0.0765093334400312	0.79033034642109\\
0.745	0.00331491276271365	0.0813562344187764	0.798770627526651\\
0.745	0.00372300185733413	0.0862098774609879	0.807270128930359\\
0.745	0.00415551592628969	0.0910701023386145	0.815671782700194\\
0.745	0.00461256761659621	0.0959367471471424	0.82381005350733\\
0.745	0.00509426838162598	0.100809648312589	0.831517746503354\\
0.745	0.00560072845543873	0.105688640598912	0.838632911419452\\
0.745	0.00613205682708918	0.11057355711583	0.845005411376292\\
0.745	0.00668836121491816	0.115464229327074	0.850502769049302\\
0.745	0.00726974804083431	0.120360487059049	0.855014975717657\\
0.745	0.00787632240459385	0.125262158509929	0.858458041096606\\
0.745	0.00850818805808555	0.13016907025918	0.860776163494183\\
0.745	0.00916544737962849	0.135081047277508	0.861942500814223\\
0.745	0.00984820134829017	0.139997912937243	0.861958614648421\\
0.745	0.0105565495182326	0.144919489023162	0.860852735617091\\
0.745	0.0112905899930939	0.149845595743738	0.858677054109647\\
0.745	0.012050419400414	0.154776051742839	0.855504274967458\\
0.745	0.0128361328661109	0.159710674111862	0.851423687967755\\
0.745	0.0136478239890177	0.164649278402306	0.846537000484339\\
0.745	0.0144855848154856	0.169591678638796	0.840954157904249\\
0.745	0.0153495058140643	0.174537687332543	0.834789345403934\\
0.745	0.0162396758502651	0.179487115495259	0.8281573258062\\
0.745	0.0171561821614173	0.184439772653511	0.821170226448692\\
0.745	0.0180991103316243	0.189395466863524	0.813934846725692\\
0.745	0.0190685442668299	0.194354004726437	0.806550519904211\\
0.745	0.0200645661700016	0.199315191403999	0.799107529846308\\
0.745	0.0210872565164405	0.204278830634725	0.791686056496194\\
0.745	0.0221366940292259	0.20924472475049	0.7843556038306\\
0.745	0.023212955654804	0.214212674693577	0.777174850281117\\
0.745	0.0243161165387281	0.219182480034174	0.770191853862645\\
0.745	0.025446250001561	0.224153938988321	0.76344454155446\\
0.745	0.0266034275149466	0.229126848436298	0.756961413910482\\
0.745	0.0277877186778607	0.234101003941464	0.750762400416763\\
0.745	0.0289991911930496	0.239076199769547	0.744859807805786\\
0.745	0.0302379108436654	0.244052228908366	0.73925931151808\\
0.745	0.0315039414701067	0.24902888308801	0.733960949044503\\
0.745	0.0327973449470747	0.254005952801449	0.72896008240726\\
0.745	0.0341181811608519	0.258983227325592	0.724248305111178\\
0.745	0.0354665079868145	0.263960494742775	0.719814276220681\\
0.745	0.0368423812671862	0.268937541962694	0.715644470612629\\
0.745	0.0382458547890421	0.273914154744766	0.711723839838042\\
0.745	0.0396769802625737	0.278890117720924	0.708036382389731\\
0.745	0.0411358072996216	0.283865214418836	0.704565625567163\\
0.745	0.0426223833924861	0.288839227285554	0.701295023640316\\
0.745	0.0441367538930258	0.293811937711588	0.698208278749354\\
0.745	0.0456789619920503	0.298783126055387	0.695289592055349\\
0.745	0.0472490486990192	0.303752571668251	0.692523853197511\\
0.745	0.0488470528220538	0.308720052919643	0.689896776228119\\
0.745	0.0504730109482718	0.313685347222913	0.687394989990281\\
0.745	0.0521269574244539	0.318648231061428	0.685006090465981\\
0.745	0.0538089243380495	0.323608480015097	0.682718662029127\\
0.745	0.0555189414985328	0.328565868787292	0.680522273853469\\
0.745	0.0572570364191156	0.333520171232163	0.678407456997851\\
0.745	0.0590232342988274	0.338471160382329	0.676365666960042\\
0.745	0.0608175580049697	0.34341860847696	0.674389235783156\\
0.745	0.0626400280559541	0.348362286990222	0.672471317134863\\
0.745	0.064490662604533	0.3533019666601	0.670605827172182\\
0.745	0.0663694774214291	0.358237417517579	0.668787383459804\\
0.745	0.0682764858793752	0.363168408916186	0.667011243730728\\
0.745	0.0702116989375697	0.368094709561873	0.665273245863649\\
0.745	0.0721751251265572	0.373016087543259	0.663569750098933\\
0.745	0.0741667705335422	0.377932310362199	0.661897584220061\\
0.745	0.0761866387881432	0.382843144964686	0.660253992184443\\
0.745	0.0782347310485954	0.38774835777208	0.658636586490996\\
0.745	0.0803110459884098	0.392647714712651	0.65704330441587\\
0.745	0.0824155797834956	0.397540981253432	0.655472368126509\\
0.745	0.084548326099754	0.402427922432375	0.653922248592746\\
0.745	0.0867092760811502	0.4073083028908	0.65239163314657\\
0.745	0.0888984183382709	0.412181886906127	0.650879396495579\\
0.745	0.0911157389373742	0.417048438424896	0.649384574964844\\
0.745	0.0933612213899392	0.421907721096044	0.647906343724685\\
0.745	0.0956348466427212	0.42675949830445	0.646443996754807\\
0.745	0.0979365930683194	0.431603533204733	0.644996929295961\\
0.745	0.100266436456264	0.436439588755293	0.643564622546847\\
0.745	0.102624350004627	0.441267427752585	0.642146630374429\\
0.745	0.105010304312169	0.446086812865616	0.640742567819433\\
0.745	0.107424267371012	0.450897506670668	0.639352101193732\\
0.745	0.109866204559871	0.455699271686218	0.637974939582443\\
0.745	0.112336078637817	0.460491870408058	0.636610827579861\\
0.745	0.114833849738607	0.465275065344603	0.635259539104382\\
0.745	0.117359475365564	0.470048619052377	0.633920872153156\\
0.745	0.119912910387022	0.474812294171664	0.632594644371966\\
0.745	0.122494107032347	0.479565853462319	0.631280689329641\\
0.745	0.125103014888515	0.484309059839721	0.629978853399058\\
0.745	0.127739580897283	0.489041676410868	0.628690460824556\\
0.75	0	0	0.710202759607373\\
0.75	1.11327767495586e-05	0.00471862581271275	0.713003157837853\\
0.75	4.46175251031912e-05	0.009446325184051	0.715846486660142\\
0.75	0.000100583311362513	0.0141829653360114	0.718788838379794\\
0.75	0.000179158434668431	0.018928411755669	0.721886092228356\\
0.75	0.000280470402701511	0.02368252819604	0.725200636220892\\
0.75	0.000404645907256436	0.0284451766772965	0.728799085274723\\
0.75	0.000551810799695644	0.0332162174883388	0.732749123983336\\
0.75	0.000722090066287311	0.037995509188729	0.737115575253327\\
0.75	0.000915607803432999	0.0427829086109896	0.741955893907011\\
0.75	0.0011324871927904	0.0475782708632729	0.747315373652289\\
0.75	0.00137285047629673	0.0523814493324038	0.753222424045811\\
0.75	0.00163681893109844	0.057192295687301	0.759684310525652\\
0.75	0.00192451284439304	0.0620106598827802	0.766683748102665\\
0.75	0.00223605148818898	0.0668363901637434	0.774176695511256\\
0.75	0.00257155309398959	0.0716693330697584	0.782091614494493\\
0.75	0.00293113482740716	0.0765093334400312	0.79033034642109\\
0.75	0.00331491276271365	0.0813562344187764	0.79877062752665\\
0.75	0.00372300185733414	0.0862098774609879	0.807270128930359\\
0.75	0.00415551592628969	0.0910701023386146	0.815671782700193\\
0.75	0.00461256761659621	0.0959367471471425	0.823810053507332\\
0.75	0.00509426838162598	0.100809648312589	0.831517746503354\\
0.75	0.00560072845543873	0.105688640598912	0.838632911419453\\
0.75	0.00613205682708918	0.11057355711583	0.845005411376293\\
0.75	0.00668836121491816	0.115464229327074	0.850502769049301\\
0.75	0.00726974804083431	0.120360487059049	0.855014975717657\\
0.75	0.00787632240459385	0.125262158509929	0.858458041096606\\
0.75	0.00850818805808554	0.13016907025918	0.860776163494183\\
0.75	0.00916544737962849	0.135081047277508	0.861942500814226\\
0.75	0.00984820134829017	0.139997912937243	0.861958614648419\\
0.75	0.0105565495182326	0.144919489023162	0.860852735617091\\
0.75	0.0112905899930939	0.149845595743738	0.858677054109652\\
0.75	0.012050419400414	0.154776051742839	0.855504274967459\\
0.75	0.0128361328661109	0.159710674111862	0.851423687967756\\
0.75	0.0136478239890177	0.164649278402306	0.84653700048434\\
0.75	0.0144855848154856	0.169591678638796	0.840954157904248\\
0.75	0.0153495058140643	0.174537687332543	0.834789345403935\\
0.75	0.0162396758502651	0.179487115495259	0.828157325806201\\
0.75	0.0171561821614173	0.184439772653511	0.821170226448694\\
0.75	0.0180991103316243	0.189395466863524	0.813934846725692\\
0.75	0.0190685442668299	0.194354004726437	0.806550519904213\\
0.75	0.0200645661700016	0.199315191403999	0.799107529846305\\
0.75	0.0210872565164405	0.204278830634725	0.791686056496192\\
0.75	0.0221366940292259	0.20924472475049	0.784355603830602\\
0.75	0.023212955654804	0.214212674693577	0.777174850281119\\
0.75	0.0243161165387281	0.219182480034174	0.770191853862643\\
0.75	0.025446250001561	0.224153938988321	0.763444541554458\\
0.75	0.0266034275149466	0.229126848436298	0.756961413910482\\
0.75	0.0277877186778607	0.234101003941464	0.750762400416764\\
0.75	0.0289991911930496	0.239076199769547	0.744859807805784\\
0.75	0.0302379108436654	0.244052228908366	0.73925931151808\\
0.75	0.0315039414701067	0.24902888308801	0.733960949044503\\
0.75	0.0327973449470747	0.254005952801449	0.72896008240726\\
0.75	0.0341181811608519	0.258983227325592	0.724248305111178\\
0.75	0.0354665079868145	0.263960494742775	0.719814276220681\\
0.75	0.0368423812671862	0.268937541962694	0.71564447061263\\
0.75	0.0382458547890421	0.273914154744766	0.711723839838041\\
0.75	0.0396769802625738	0.278890117720924	0.70803638238973\\
0.75	0.0411358072996216	0.283865214418836	0.704565625567164\\
0.75	0.0426223833924861	0.288839227285554	0.701295023640317\\
0.75	0.0441367538930258	0.293811937711588	0.698208278749353\\
0.75	0.0456789619920503	0.298783126055387	0.695289592055349\\
0.75	0.0472490486990192	0.303752571668251	0.692523853197512\\
0.75	0.0488470528220538	0.308720052919643	0.689896776228119\\
0.75	0.0504730109482718	0.313685347222913	0.687394989990283\\
0.75	0.0521269574244539	0.318648231061428	0.68500609046598\\
0.75	0.0538089243380495	0.323608480015096	0.682718662029127\\
0.75	0.0555189414985328	0.328565868787292	0.680522273853469\\
0.75	0.0572570364191156	0.333520171232163	0.678407456997851\\
0.75	0.0590232342988274	0.338471160382329	0.676365666960044\\
0.75	0.0608175580049697	0.34341860847696	0.674389235783156\\
0.75	0.0626400280559541	0.348362286990222	0.672471317134862\\
0.75	0.064490662604533	0.3533019666601	0.670605827172182\\
0.75	0.0663694774214291	0.358237417517579	0.668787383459804\\
0.75	0.0682764858793752	0.363168408916186	0.667011243730729\\
0.75	0.0702116989375697	0.368094709561873	0.665273245863648\\
0.75	0.0721751251265572	0.373016087543259	0.663569750098932\\
0.75	0.0741667705335422	0.377932310362199	0.661897584220061\\
0.75	0.0761866387881432	0.382843144964686	0.660253992184442\\
0.75	0.0782347310485954	0.38774835777208	0.658636586490998\\
0.75	0.0803110459884098	0.392647714712651	0.657043304415871\\
0.75	0.0824155797834956	0.397540981253432	0.655472368126508\\
0.75	0.084548326099754	0.402427922432375	0.653922248592745\\
0.75	0.0867092760811502	0.407308302890799	0.652391633146569\\
0.75	0.0888984183382709	0.412181886906127	0.650879396495579\\
0.75	0.0911157389373742	0.417048438424896	0.649384574964845\\
0.75	0.0933612213899392	0.421907721096044	0.647906343724685\\
0.75	0.0956348466427212	0.42675949830445	0.646443996754805\\
0.75	0.0979365930683194	0.431603533204733	0.644996929295962\\
0.75	0.100266436456264	0.436439588755293	0.643564622546846\\
0.75	0.102624350004627	0.441267427752584	0.64214663037443\\
0.75	0.105010304312169	0.446086812865616	0.640742567819434\\
0.75	0.107424267371012	0.450897506670668	0.63935210119373\\
0.75	0.109866204559871	0.455699271686218	0.637974939582443\\
0.75	0.112336078637817	0.460491870408058	0.636610827579862\\
0.75	0.114833849738607	0.465275065344603	0.635259539104382\\
0.75	0.117359475365564	0.470048619052377	0.633920872153154\\
0.75	0.119912910387022	0.474812294171664	0.632594644371965\\
0.75	0.122494107032347	0.479565853462319	0.631280689329641\\
0.75	0.125103014888515	0.484309059839721	0.62997885339906\\
0.75	0.127739580897283	0.489041676410868	0.628690460824559\\
0.755	0	0	0.710202759607373\\
0.755	1.11327767495586e-05	0.00471862581271274	0.713003157837853\\
0.755	4.46175251031912e-05	0.009446325184051	0.715846486660142\\
0.755	0.000100583311362513	0.0141829653360114	0.718788838379794\\
0.755	0.000179158434668431	0.018928411755669	0.721886092228356\\
0.755	0.000280470402701511	0.02368252819604	0.725200636220892\\
0.755	0.000404645907256436	0.0284451766772965	0.728799085274723\\
0.755	0.000551810799695644	0.0332162174883389	0.732749123983336\\
0.755	0.000722090066287311	0.037995509188729	0.737115575253327\\
0.755	0.000915607803432999	0.0427829086109896	0.741955893907011\\
0.755	0.0011324871927904	0.0475782708632729	0.747315373652289\\
0.755	0.00137285047629673	0.0523814493324038	0.753222424045811\\
0.755	0.00163681893109843	0.057192295687301	0.759684310525652\\
0.755	0.00192451284439304	0.0620106598827802	0.766683748102665\\
0.755	0.00223605148818899	0.0668363901637434	0.774176695511256\\
0.755	0.00257155309398959	0.0716693330697584	0.782091614494493\\
0.755	0.00293113482740716	0.0765093334400312	0.79033034642109\\
0.755	0.00331491276271365	0.0813562344187764	0.79877062752665\\
0.755	0.00372300185733413	0.0862098774609879	0.807270128930359\\
0.755	0.00415551592628969	0.0910701023386145	0.815671782700193\\
0.755	0.00461256761659621	0.0959367471471425	0.82381005350733\\
0.755	0.00509426838162598	0.100809648312589	0.831517746503354\\
0.755	0.00560072845543873	0.105688640598912	0.838632911419451\\
0.755	0.00613205682708918	0.11057355711583	0.845005411376294\\
0.755	0.00668836121491815	0.115464229327074	0.850502769049302\\
0.755	0.00726974804083431	0.120360487059049	0.855014975717657\\
0.755	0.00787632240459385	0.125262158509929	0.858458041096606\\
0.755	0.00850818805808555	0.13016907025918	0.860776163494182\\
0.755	0.00916544737962849	0.135081047277508	0.861942500814224\\
0.755	0.00984820134829017	0.139997912937243	0.861958614648419\\
0.755	0.0105565495182326	0.144919489023162	0.860852735617092\\
0.755	0.0112905899930939	0.149845595743738	0.858677054109652\\
0.755	0.012050419400414	0.154776051742839	0.855504274967457\\
0.755	0.0128361328661109	0.159710674111862	0.851423687967753\\
0.755	0.0136478239890177	0.164649278402306	0.846537000484339\\
0.755	0.0144855848154856	0.169591678638796	0.840954157904248\\
0.755	0.0153495058140643	0.174537687332543	0.834789345403936\\
0.755	0.0162396758502651	0.179487115495259	0.828157325806202\\
0.755	0.0171561821614173	0.184439772653511	0.821170226448694\\
0.755	0.0180991103316243	0.189395466863524	0.813934846725693\\
0.755	0.0190685442668299	0.194354004726437	0.806550519904212\\
0.755	0.0200645661700016	0.199315191403999	0.799107529846304\\
0.755	0.0210872565164405	0.204278830634725	0.791686056496195\\
0.755	0.0221366940292259	0.20924472475049	0.784355603830603\\
0.755	0.023212955654804	0.214212674693577	0.777174850281118\\
0.755	0.0243161165387281	0.219182480034174	0.770191853862643\\
0.755	0.025446250001561	0.224153938988321	0.763444541554459\\
0.755	0.0266034275149466	0.229126848436298	0.756961413910483\\
0.755	0.0277877186778607	0.234101003941464	0.750762400416764\\
0.755	0.0289991911930496	0.239076199769547	0.744859807805784\\
0.755	0.0302379108436654	0.244052228908366	0.73925931151808\\
0.755	0.0315039414701067	0.24902888308801	0.733960949044504\\
0.755	0.0327973449470747	0.254005952801449	0.728960082407259\\
0.755	0.0341181811608519	0.258983227325591	0.724248305111178\\
0.755	0.0354665079868145	0.263960494742775	0.719814276220682\\
0.755	0.0368423812671862	0.268937541962694	0.715644470612629\\
0.755	0.0382458547890421	0.273914154744766	0.711723839838042\\
0.755	0.0396769802625738	0.278890117720924	0.708036382389732\\
0.755	0.0411358072996216	0.283865214418836	0.704565625567165\\
0.755	0.0426223833924861	0.288839227285554	0.701295023640315\\
0.755	0.0441367538930258	0.293811937711588	0.698208278749353\\
0.755	0.0456789619920503	0.298783126055387	0.695289592055348\\
0.755	0.0472490486990192	0.303752571668251	0.692523853197512\\
0.755	0.0488470528220538	0.308720052919643	0.68989677622812\\
0.755	0.0504730109482718	0.313685347222913	0.687394989990282\\
0.755	0.0521269574244539	0.318648231061428	0.68500609046598\\
0.755	0.0538089243380495	0.323608480015096	0.682718662029127\\
0.755	0.0555189414985328	0.328565868787292	0.680522273853469\\
0.755	0.0572570364191156	0.333520171232163	0.67840745699785\\
0.755	0.0590232342988274	0.338471160382329	0.676365666960043\\
0.755	0.0608175580049697	0.34341860847696	0.674389235783156\\
0.755	0.0626400280559541	0.348362286990222	0.672471317134863\\
0.755	0.064490662604533	0.3533019666601	0.670605827172182\\
0.755	0.0663694774214291	0.358237417517579	0.668787383459803\\
0.755	0.0682764858793752	0.363168408916186	0.667011243730728\\
0.755	0.0702116989375697	0.368094709561873	0.665273245863651\\
0.755	0.0721751251265572	0.373016087543259	0.663569750098931\\
0.755	0.0741667705335423	0.377932310362199	0.661897584220059\\
0.755	0.0761866387881432	0.382843144964686	0.660253992184443\\
0.755	0.0782347310485954	0.38774835777208	0.658636586490997\\
0.755	0.0803110459884098	0.392647714712651	0.657043304415873\\
0.755	0.0824155797834956	0.397540981253432	0.655472368126509\\
0.755	0.084548326099754	0.402427922432375	0.653922248592745\\
0.755	0.0867092760811502	0.4073083028908	0.652391633146571\\
0.755	0.0888984183382709	0.412181886906127	0.650879396495578\\
0.755	0.0911157389373742	0.417048438424896	0.649384574964843\\
0.755	0.0933612213899392	0.421907721096044	0.647906343724685\\
0.755	0.0956348466427212	0.42675949830445	0.646443996754806\\
0.755	0.0979365930683194	0.431603533204733	0.644996929295962\\
0.755	0.100266436456264	0.436439588755293	0.643564622546846\\
0.755	0.102624350004627	0.441267427752585	0.64214663037443\\
0.755	0.105010304312169	0.446086812865616	0.640742567819433\\
0.755	0.107424267371012	0.450897506670668	0.639352101193731\\
0.755	0.109866204559871	0.455699271686218	0.637974939582443\\
0.755	0.112336078637817	0.460491870408058	0.636610827579862\\
0.755	0.114833849738607	0.465275065344603	0.635259539104383\\
0.755	0.117359475365564	0.470048619052377	0.633920872153154\\
0.755	0.119912910387022	0.474812294171664	0.632594644371965\\
0.755	0.122494107032347	0.479565853462319	0.63128068932964\\
0.755	0.125103014888515	0.484309059839721	0.629978853399059\\
0.755	0.127739580897283	0.489041676410868	0.628690460824563\\
0.76	0	0	0.710202759607373\\
0.76	1.11327767495586e-05	0.00471862581271274	0.713003157837853\\
0.76	4.46175251031912e-05	0.009446325184051	0.715846486660142\\
0.76	0.000100583311362513	0.0141829653360114	0.718788838379794\\
0.76	0.000179158434668431	0.018928411755669	0.721886092228356\\
0.76	0.000280470402701511	0.02368252819604	0.725200636220892\\
0.76	0.000404645907256436	0.0284451766772965	0.728799085274723\\
0.76	0.000551810799695644	0.0332162174883388	0.732749123983336\\
0.76	0.000722090066287311	0.037995509188729	0.737115575253327\\
0.76	0.000915607803432999	0.0427829086109896	0.741955893907011\\
0.76	0.0011324871927904	0.0475782708632729	0.747315373652289\\
0.76	0.00137285047629673	0.0523814493324038	0.753222424045811\\
0.76	0.00163681893109843	0.057192295687301	0.759684310525652\\
0.76	0.00192451284439304	0.0620106598827802	0.766683748102665\\
0.76	0.00223605148818898	0.0668363901637434	0.774176695511256\\
0.76	0.00257155309398959	0.0716693330697584	0.782091614494493\\
0.76	0.00293113482740716	0.0765093334400312	0.79033034642109\\
0.76	0.00331491276271365	0.0813562344187764	0.79877062752665\\
0.76	0.00372300185733413	0.0862098774609879	0.807270128930359\\
0.76	0.00415551592628969	0.0910701023386145	0.815671782700193\\
0.76	0.00461256761659621	0.0959367471471425	0.82381005350733\\
0.76	0.00509426838162598	0.100809648312589	0.831517746503354\\
0.76	0.00560072845543873	0.105688640598912	0.838632911419452\\
0.76	0.00613205682708918	0.11057355711583	0.845005411376293\\
0.76	0.00668836121491815	0.115464229327074	0.850502769049301\\
0.76	0.00726974804083431	0.120360487059049	0.855014975717657\\
0.76	0.00787632240459385	0.125262158509929	0.858458041096605\\
0.76	0.00850818805808555	0.13016907025918	0.860776163494182\\
0.76	0.00916544737962849	0.135081047277508	0.861942500814224\\
0.76	0.00984820134829017	0.139997912937243	0.86195861464842\\
0.76	0.0105565495182326	0.144919489023162	0.860852735617092\\
0.76	0.0112905899930939	0.149845595743738	0.85867705410965\\
0.76	0.012050419400414	0.154776051742839	0.855504274967456\\
0.76	0.0128361328661109	0.159710674111862	0.851423687967753\\
0.76	0.0136478239890177	0.164649278402306	0.84653700048434\\
0.76	0.0144855848154856	0.169591678638796	0.840954157904249\\
0.76	0.0153495058140643	0.174537687332543	0.834789345403936\\
0.76	0.0162396758502651	0.179487115495259	0.828157325806199\\
0.76	0.0171561821614173	0.184439772653511	0.821170226448694\\
0.76	0.0180991103316243	0.189395466863524	0.813934846725693\\
0.76	0.0190685442668299	0.194354004726437	0.806550519904213\\
0.76	0.0200645661700016	0.199315191403999	0.799107529846307\\
0.76	0.0210872565164405	0.204278830634725	0.791686056496193\\
0.76	0.0221366940292259	0.20924472475049	0.784355603830603\\
0.76	0.023212955654804	0.214212674693577	0.777174850281117\\
0.76	0.0243161165387281	0.219182480034174	0.770191853862643\\
0.76	0.025446250001561	0.224153938988321	0.763444541554458\\
0.76	0.0266034275149466	0.229126848436298	0.756961413910483\\
0.76	0.0277877186778607	0.234101003941464	0.750762400416762\\
0.76	0.0289991911930496	0.239076199769547	0.744859807805784\\
0.76	0.0302379108436654	0.244052228908366	0.73925931151808\\
0.76	0.0315039414701067	0.24902888308801	0.733960949044503\\
0.76	0.0327973449470747	0.254005952801449	0.728960082407261\\
0.76	0.0341181811608519	0.258983227325592	0.724248305111179\\
0.76	0.0354665079868145	0.263960494742775	0.71981427622068\\
0.76	0.0368423812671862	0.268937541962694	0.71564447061263\\
0.76	0.0382458547890421	0.273914154744766	0.711723839838041\\
0.76	0.0396769802625738	0.278890117720924	0.708036382389731\\
0.76	0.0411358072996216	0.283865214418836	0.704565625567164\\
0.76	0.0426223833924861	0.288839227285554	0.701295023640316\\
0.76	0.0441367538930258	0.293811937711588	0.698208278749353\\
0.76	0.0456789619920503	0.298783126055387	0.695289592055349\\
0.76	0.0472490486990192	0.303752571668251	0.692523853197511\\
0.76	0.0488470528220538	0.308720052919643	0.689896776228119\\
0.76	0.0504730109482718	0.313685347222913	0.687394989990282\\
0.76	0.0521269574244539	0.318648231061428	0.68500609046598\\
0.76	0.0538089243380495	0.323608480015096	0.682718662029128\\
0.76	0.0555189414985328	0.328565868787292	0.680522273853469\\
0.76	0.0572570364191156	0.333520171232163	0.67840745699785\\
0.76	0.0590232342988274	0.338471160382329	0.676365666960043\\
0.76	0.0608175580049697	0.34341860847696	0.674389235783155\\
0.76	0.0626400280559541	0.348362286990222	0.672471317134862\\
0.76	0.064490662604533	0.3533019666601	0.670605827172183\\
0.76	0.0663694774214291	0.358237417517579	0.668787383459804\\
0.76	0.0682764858793752	0.363168408916186	0.667011243730727\\
0.76	0.0702116989375697	0.368094709561873	0.66527324586365\\
0.76	0.0721751251265572	0.373016087543259	0.663569750098933\\
0.76	0.0741667705335423	0.377932310362199	0.661897584220059\\
0.76	0.0761866387881432	0.382843144964686	0.660253992184442\\
0.76	0.0782347310485954	0.38774835777208	0.658636586490997\\
0.76	0.0803110459884098	0.392647714712651	0.657043304415871\\
0.76	0.0824155797834956	0.397540981253432	0.655472368126509\\
0.76	0.084548326099754	0.402427922432375	0.653922248592745\\
0.76	0.0867092760811502	0.407308302890799	0.652391633146571\\
0.76	0.0888984183382709	0.412181886906127	0.65087939649558\\
0.76	0.0911157389373742	0.417048438424896	0.649384574964842\\
0.76	0.0933612213899392	0.421907721096044	0.647906343724685\\
0.76	0.0956348466427212	0.42675949830445	0.646443996754807\\
0.76	0.0979365930683194	0.431603533204733	0.644996929295963\\
0.76	0.100266436456264	0.436439588755293	0.643564622546846\\
0.76	0.102624350004627	0.441267427752585	0.64214663037443\\
0.76	0.105010304312169	0.446086812865616	0.640742567819433\\
0.76	0.107424267371012	0.450897506670668	0.63935210119373\\
0.76	0.109866204559871	0.455699271686218	0.637974939582443\\
0.76	0.112336078637817	0.460491870408058	0.636610827579861\\
0.76	0.114833849738607	0.465275065344603	0.635259539104383\\
0.76	0.117359475365564	0.470048619052377	0.633920872153155\\
0.76	0.119912910387022	0.474812294171664	0.632594644371965\\
0.76	0.122494107032347	0.479565853462319	0.631280689329641\\
0.76	0.125103014888515	0.484309059839721	0.629978853399059\\
0.76	0.127739580897283	0.489041676410868	0.62869046082456\\
0.765	0	0	0.710202759607373\\
0.765	1.11327767495586e-05	0.00471862581271274	0.713003157837853\\
0.765	4.46175251031912e-05	0.009446325184051	0.715846486660142\\
0.765	0.000100583311362513	0.0141829653360114	0.718788838379794\\
0.765	0.000179158434668431	0.018928411755669	0.721886092228357\\
0.765	0.000280470402701511	0.02368252819604	0.725200636220892\\
0.765	0.000404645907256436	0.0284451766772965	0.728799085274723\\
0.765	0.000551810799695644	0.0332162174883389	0.732749123983336\\
0.765	0.000722090066287311	0.037995509188729	0.737115575253328\\
0.765	0.000915607803432999	0.0427829086109896	0.741955893907011\\
0.765	0.0011324871927904	0.0475782708632729	0.747315373652289\\
0.765	0.00137285047629673	0.0523814493324038	0.75322242404581\\
0.765	0.00163681893109844	0.057192295687301	0.759684310525652\\
0.765	0.00192451284439304	0.0620106598827802	0.766683748102665\\
0.765	0.00223605148818899	0.0668363901637434	0.774176695511256\\
0.765	0.00257155309398959	0.0716693330697584	0.782091614494493\\
0.765	0.00293113482740716	0.0765093334400312	0.79033034642109\\
0.765	0.00331491276271365	0.0813562344187764	0.79877062752665\\
0.765	0.00372300185733413	0.0862098774609879	0.807270128930359\\
0.765	0.00415551592628969	0.0910701023386145	0.815671782700193\\
0.765	0.00461256761659621	0.0959367471471425	0.82381005350733\\
0.765	0.00509426838162598	0.100809648312589	0.831517746503353\\
0.765	0.00560072845543873	0.105688640598912	0.838632911419452\\
0.765	0.00613205682708918	0.11057355711583	0.845005411376292\\
0.765	0.00668836121491815	0.115464229327074	0.850502769049302\\
0.765	0.00726974804083431	0.120360487059049	0.855014975717657\\
0.765	0.00787632240459385	0.125262158509929	0.858458041096606\\
0.765	0.00850818805808554	0.13016907025918	0.860776163494183\\
0.765	0.00916544737962849	0.135081047277508	0.861942500814223\\
0.765	0.00984820134829017	0.139997912937243	0.861958614648421\\
0.765	0.0105565495182326	0.144919489023162	0.860852735617092\\
0.765	0.0112905899930939	0.149845595743738	0.85867705410965\\
0.765	0.012050419400414	0.154776051742839	0.855504274967459\\
0.765	0.0128361328661109	0.159710674111862	0.851423687967755\\
0.765	0.0136478239890177	0.164649278402306	0.846537000484341\\
0.765	0.0144855848154856	0.169591678638796	0.840954157904249\\
0.765	0.0153495058140643	0.174537687332543	0.834789345403936\\
0.765	0.0162396758502651	0.179487115495259	0.828157325806198\\
0.765	0.0171561821614173	0.184439772653511	0.821170226448693\\
0.765	0.0180991103316243	0.189395466863524	0.813934846725692\\
0.765	0.0190685442668299	0.194354004726437	0.806550519904212\\
0.765	0.0200645661700016	0.199315191403999	0.799107529846302\\
0.765	0.0210872565164405	0.204278830634725	0.791686056496193\\
0.765	0.0221366940292259	0.20924472475049	0.784355603830603\\
0.765	0.023212955654804	0.214212674693577	0.777174850281116\\
0.765	0.0243161165387281	0.219182480034174	0.770191853862644\\
0.765	0.025446250001561	0.224153938988321	0.763444541554459\\
0.765	0.0266034275149466	0.229126848436298	0.756961413910481\\
0.765	0.0277877186778607	0.234101003941464	0.750762400416764\\
0.765	0.0289991911930496	0.239076199769547	0.744859807805786\\
0.765	0.0302379108436654	0.244052228908366	0.73925931151808\\
0.765	0.0315039414701067	0.24902888308801	0.733960949044503\\
0.765	0.0327973449470747	0.254005952801449	0.728960082407259\\
0.765	0.0341181811608519	0.258983227325591	0.724248305111178\\
0.765	0.0354665079868145	0.263960494742775	0.719814276220682\\
0.765	0.0368423812671862	0.268937541962694	0.715644470612629\\
0.765	0.0382458547890421	0.273914154744766	0.711723839838042\\
0.765	0.0396769802625738	0.278890117720924	0.708036382389731\\
0.765	0.0411358072996216	0.283865214418836	0.704565625567164\\
0.765	0.0426223833924861	0.288839227285554	0.701295023640316\\
0.765	0.0441367538930258	0.293811937711588	0.698208278749354\\
0.765	0.0456789619920503	0.298783126055387	0.695289592055349\\
0.765	0.0472490486990192	0.303752571668251	0.692523853197513\\
0.765	0.0488470528220538	0.308720052919643	0.689896776228119\\
0.765	0.0504730109482718	0.313685347222913	0.687394989990281\\
0.765	0.0521269574244539	0.318648231061428	0.685006090465979\\
0.765	0.0538089243380495	0.323608480015096	0.682718662029127\\
0.765	0.0555189414985328	0.328565868787292	0.680522273853471\\
0.765	0.0572570364191156	0.333520171232163	0.678407456997851\\
0.765	0.0590232342988274	0.338471160382329	0.676365666960044\\
0.765	0.0608175580049697	0.34341860847696	0.674389235783156\\
0.765	0.0626400280559541	0.348362286990222	0.672471317134862\\
0.765	0.064490662604533	0.3533019666601	0.670605827172182\\
0.765	0.0663694774214291	0.358237417517579	0.668787383459803\\
0.765	0.0682764858793752	0.363168408916186	0.667011243730728\\
0.765	0.0702116989375697	0.368094709561873	0.665273245863649\\
0.765	0.0721751251265572	0.373016087543259	0.663569750098933\\
0.765	0.0741667705335422	0.377932310362199	0.66189758422006\\
0.765	0.0761866387881432	0.382843144964686	0.660253992184441\\
0.765	0.0782347310485954	0.38774835777208	0.658636586490998\\
0.765	0.0803110459884098	0.392647714712651	0.657043304415872\\
0.765	0.0824155797834956	0.397540981253432	0.655472368126508\\
0.765	0.084548326099754	0.402427922432375	0.653922248592746\\
0.765	0.0867092760811502	0.4073083028908	0.652391633146569\\
0.765	0.0888984183382709	0.412181886906127	0.65087939649558\\
0.765	0.0911157389373742	0.417048438424897	0.649384574964844\\
0.765	0.0933612213899392	0.421907721096044	0.647906343724683\\
0.765	0.0956348466427212	0.42675949830445	0.646443996754806\\
0.765	0.0979365930683194	0.431603533204733	0.644996929295963\\
0.765	0.100266436456264	0.436439588755293	0.643564622546846\\
0.765	0.102624350004627	0.441267427752584	0.642146630374429\\
0.765	0.105010304312169	0.446086812865616	0.640742567819434\\
0.765	0.107424267371012	0.450897506670668	0.63935210119373\\
0.765	0.109866204559871	0.455699271686218	0.637974939582444\\
0.765	0.112336078637817	0.460491870408058	0.636610827579863\\
0.765	0.114833849738607	0.465275065344603	0.635259539104382\\
0.765	0.117359475365564	0.470048619052377	0.633920872153155\\
0.765	0.119912910387022	0.474812294171664	0.632594644371965\\
0.765	0.122494107032347	0.479565853462319	0.63128068932964\\
0.765	0.125103014888515	0.484309059839721	0.629978853399057\\
0.765	0.127739580897283	0.489041676410868	0.628690460824558\\
0.77	0	0	0.710202759607373\\
0.77	1.11327767495586e-05	0.00471862581271274	0.713003157837853\\
0.77	4.46175251031912e-05	0.009446325184051	0.715846486660142\\
0.77	0.000100583311362513	0.0141829653360114	0.718788838379794\\
0.77	0.000179158434668431	0.018928411755669	0.721886092228356\\
0.77	0.000280470402701511	0.02368252819604	0.725200636220892\\
0.77	0.000404645907256436	0.0284451766772965	0.728799085274723\\
0.77	0.000551810799695644	0.0332162174883389	0.732749123983336\\
0.77	0.000722090066287311	0.037995509188729	0.737115575253328\\
0.77	0.000915607803432999	0.0427829086109896	0.741955893907011\\
0.77	0.0011324871927904	0.047578270863273	0.747315373652289\\
0.77	0.00137285047629673	0.0523814493324038	0.753222424045811\\
0.77	0.00163681893109844	0.057192295687301	0.759684310525652\\
0.77	0.00192451284439304	0.0620106598827802	0.766683748102665\\
0.77	0.00223605148818899	0.0668363901637434	0.774176695511257\\
0.77	0.00257155309398959	0.0716693330697584	0.782091614494494\\
0.77	0.00293113482740716	0.0765093334400312	0.79033034642109\\
0.77	0.00331491276271365	0.0813562344187764	0.798770627526651\\
0.77	0.00372300185733414	0.0862098774609879	0.807270128930359\\
0.77	0.00415551592628969	0.0910701023386145	0.815671782700193\\
0.77	0.00461256761659621	0.0959367471471425	0.82381005350733\\
0.77	0.00509426838162598	0.100809648312589	0.831517746503354\\
0.77	0.00560072845543873	0.105688640598912	0.838632911419452\\
0.77	0.00613205682708918	0.11057355711583	0.845005411376292\\
0.77	0.00668836121491816	0.115464229327074	0.850502769049301\\
0.77	0.00726974804083431	0.120360487059049	0.855014975717656\\
0.77	0.00787632240459385	0.125262158509929	0.858458041096606\\
0.77	0.00850818805808554	0.13016907025918	0.860776163494182\\
0.77	0.00916544737962849	0.135081047277508	0.861942500814224\\
0.77	0.00984820134829017	0.139997912937243	0.861958614648421\\
0.77	0.0105565495182326	0.144919489023162	0.860852735617088\\
0.77	0.0112905899930939	0.149845595743738	0.858677054109649\\
0.77	0.012050419400414	0.154776051742839	0.855504274967458\\
0.77	0.0128361328661109	0.159710674111862	0.851423687967754\\
0.77	0.0136478239890177	0.164649278402306	0.846537000484341\\
0.77	0.0144855848154856	0.169591678638796	0.84095415790425\\
0.77	0.0153495058140643	0.174537687332543	0.834789345403936\\
0.77	0.0162396758502651	0.179487115495259	0.828157325806201\\
0.77	0.0171561821614173	0.184439772653511	0.821170226448694\\
0.77	0.0180991103316243	0.189395466863524	0.813934846725692\\
0.77	0.0190685442668299	0.194354004726437	0.806550519904209\\
0.77	0.0200645661700016	0.199315191403999	0.799107529846305\\
0.77	0.0210872565164405	0.204278830634725	0.791686056496196\\
0.77	0.0221366940292259	0.20924472475049	0.7843556038306\\
0.77	0.023212955654804	0.214212674693577	0.777174850281118\\
0.77	0.0243161165387281	0.219182480034174	0.770191853862644\\
0.77	0.025446250001561	0.224153938988321	0.763444541554457\\
0.77	0.0266034275149466	0.229126848436298	0.756961413910482\\
0.77	0.0277877186778607	0.234101003941464	0.750762400416764\\
0.77	0.0289991911930496	0.239076199769547	0.744859807805784\\
0.77	0.0302379108436654	0.244052228908366	0.739259311518079\\
0.77	0.0315039414701067	0.24902888308801	0.733960949044504\\
0.77	0.0327973449470747	0.254005952801449	0.728960082407262\\
0.77	0.0341181811608519	0.258983227325592	0.724248305111179\\
0.77	0.0354665079868145	0.263960494742775	0.71981427622068\\
0.77	0.0368423812671862	0.268937541962694	0.715644470612629\\
0.77	0.0382458547890421	0.273914154744766	0.711723839838043\\
0.77	0.0396769802625738	0.278890117720924	0.70803638238973\\
0.77	0.0411358072996216	0.283865214418836	0.704565625567165\\
0.77	0.0426223833924861	0.288839227285554	0.701295023640316\\
0.77	0.0441367538930258	0.293811937711588	0.698208278749352\\
0.77	0.0456789619920503	0.298783126055387	0.695289592055349\\
0.77	0.0472490486990192	0.303752571668251	0.692523853197512\\
0.77	0.0488470528220538	0.308720052919643	0.689896776228119\\
0.77	0.0504730109482718	0.313685347222913	0.687394989990282\\
0.77	0.0521269574244539	0.318648231061428	0.685006090465981\\
0.77	0.0538089243380495	0.323608480015097	0.682718662029127\\
0.77	0.0555189414985328	0.328565868787292	0.680522273853468\\
0.77	0.0572570364191156	0.333520171232163	0.678407456997851\\
0.77	0.0590232342988274	0.338471160382329	0.676365666960044\\
0.77	0.0608175580049697	0.34341860847696	0.674389235783156\\
0.77	0.0626400280559541	0.348362286990222	0.672471317134863\\
0.77	0.064490662604533	0.3533019666601	0.670605827172184\\
0.77	0.0663694774214291	0.358237417517579	0.668787383459803\\
0.77	0.0682764858793752	0.363168408916186	0.667011243730728\\
0.77	0.0702116989375697	0.368094709561873	0.665273245863649\\
0.77	0.0721751251265572	0.373016087543259	0.663569750098932\\
0.77	0.0741667705335422	0.377932310362199	0.661897584220061\\
0.77	0.0761866387881432	0.382843144964686	0.660253992184442\\
0.77	0.0782347310485954	0.38774835777208	0.658636586490995\\
0.77	0.0803110459884098	0.392647714712651	0.657043304415872\\
0.77	0.0824155797834956	0.397540981253432	0.655472368126511\\
0.77	0.084548326099754	0.402427922432375	0.653922248592745\\
0.77	0.0867092760811502	0.4073083028908	0.652391633146569\\
0.77	0.0888984183382709	0.412181886906127	0.650879396495578\\
0.77	0.0911157389373742	0.417048438424896	0.649384574964845\\
0.77	0.0933612213899392	0.421907721096044	0.647906343724684\\
0.77	0.0956348466427212	0.42675949830445	0.646443996754805\\
0.77	0.0979365930683194	0.431603533204733	0.644996929295962\\
0.77	0.100266436456264	0.436439588755293	0.643564622546848\\
0.77	0.102624350004627	0.441267427752585	0.642146630374429\\
0.77	0.105010304312169	0.446086812865616	0.640742567819432\\
0.77	0.107424267371012	0.450897506670668	0.639352101193729\\
0.77	0.109866204559871	0.455699271686218	0.637974939582442\\
0.77	0.112336078637817	0.460491870408058	0.636610827579863\\
0.77	0.114833849738607	0.465275065344603	0.635259539104383\\
0.77	0.117359475365564	0.470048619052377	0.633920872153155\\
0.77	0.119912910387023	0.474812294171664	0.632594644371966\\
0.77	0.122494107032347	0.479565853462319	0.631280689329641\\
0.77	0.125103014888515	0.484309059839721	0.629978853399058\\
0.77	0.127739580897283	0.489041676410868	0.628690460824557\\
0.775	0	0	0.710202759607373\\
0.775	1.11327767495586e-05	0.00471862581271274	0.713003157837853\\
0.775	4.46175251031912e-05	0.009446325184051	0.715846486660142\\
0.775	0.000100583311362513	0.0141829653360114	0.718788838379794\\
0.775	0.000179158434668431	0.018928411755669	0.721886092228356\\
0.775	0.000280470402701511	0.02368252819604	0.725200636220892\\
0.775	0.000404645907256436	0.0284451766772965	0.728799085274723\\
0.775	0.000551810799695644	0.0332162174883388	0.732749123983336\\
0.775	0.000722090066287311	0.037995509188729	0.737115575253327\\
0.775	0.000915607803432999	0.0427829086109896	0.741955893907011\\
0.775	0.0011324871927904	0.047578270863273	0.747315373652289\\
0.775	0.00137285047629673	0.0523814493324038	0.753222424045811\\
0.775	0.00163681893109844	0.057192295687301	0.759684310525652\\
0.775	0.00192451284439304	0.0620106598827802	0.766683748102665\\
0.775	0.00223605148818898	0.0668363901637434	0.774176695511257\\
0.775	0.00257155309398959	0.0716693330697584	0.782091614494493\\
0.775	0.00293113482740716	0.0765093334400312	0.79033034642109\\
0.775	0.00331491276271365	0.0813562344187764	0.79877062752665\\
0.775	0.00372300185733413	0.0862098774609879	0.80727012893036\\
0.775	0.00415551592628969	0.0910701023386145	0.815671782700193\\
0.775	0.00461256761659621	0.0959367471471425	0.823810053507331\\
0.775	0.00509426838162598	0.100809648312589	0.831517746503354\\
0.775	0.00560072845543873	0.105688640598912	0.838632911419452\\
0.775	0.00613205682708918	0.11057355711583	0.845005411376294\\
0.775	0.00668836121491816	0.115464229327074	0.850502769049302\\
0.775	0.00726974804083431	0.120360487059049	0.855014975717657\\
0.775	0.00787632240459385	0.125262158509929	0.858458041096605\\
0.775	0.00850818805808554	0.13016907025918	0.860776163494182\\
0.775	0.00916544737962849	0.135081047277508	0.861942500814226\\
0.775	0.00984820134829017	0.139997912937243	0.861958614648419\\
0.775	0.0105565495182326	0.144919489023162	0.86085273561709\\
0.775	0.0112905899930939	0.149845595743738	0.858677054109649\\
0.775	0.012050419400414	0.154776051742839	0.855504274967455\\
0.775	0.0128361328661109	0.159710674111862	0.851423687967753\\
0.775	0.0136478239890177	0.164649278402306	0.84653700048434\\
0.775	0.0144855848154856	0.169591678638796	0.840954157904248\\
0.775	0.0153495058140643	0.174537687332543	0.834789345403935\\
0.775	0.0162396758502651	0.179487115495259	0.8281573258062\\
0.775	0.0171561821614173	0.184439772653511	0.821170226448696\\
0.775	0.0180991103316243	0.189395466863524	0.813934846725691\\
0.775	0.0190685442668299	0.194354004726437	0.806550519904214\\
0.775	0.0200645661700016	0.199315191403999	0.799107529846306\\
0.775	0.0210872565164405	0.204278830634725	0.791686056496193\\
0.775	0.0221366940292259	0.20924472475049	0.7843556038306\\
0.775	0.023212955654804	0.214212674693577	0.777174850281118\\
0.775	0.0243161165387281	0.219182480034174	0.770191853862642\\
0.775	0.025446250001561	0.224153938988321	0.763444541554457\\
0.775	0.0266034275149466	0.229126848436298	0.756961413910481\\
0.775	0.0277877186778607	0.234101003941464	0.750762400416764\\
0.775	0.0289991911930496	0.239076199769547	0.744859807805785\\
0.775	0.0302379108436654	0.244052228908366	0.73925931151808\\
0.775	0.0315039414701067	0.24902888308801	0.733960949044505\\
0.775	0.0327973449470747	0.254005952801449	0.72896008240726\\
0.775	0.0341181811608519	0.258983227325591	0.724248305111177\\
0.775	0.0354665079868145	0.263960494742775	0.719814276220682\\
0.775	0.0368423812671862	0.268937541962694	0.715644470612629\\
0.775	0.0382458547890421	0.273914154744766	0.711723839838043\\
0.775	0.0396769802625738	0.278890117720924	0.708036382389731\\
0.775	0.0411358072996216	0.283865214418836	0.704565625567162\\
0.775	0.0426223833924861	0.288839227285554	0.701295023640316\\
0.775	0.0441367538930258	0.293811937711588	0.698208278749354\\
0.775	0.0456789619920503	0.298783126055387	0.695289592055348\\
0.775	0.0472490486990192	0.303752571668251	0.692523853197513\\
0.775	0.0488470528220538	0.308720052919643	0.689896776228119\\
0.775	0.0504730109482718	0.313685347222913	0.687394989990281\\
0.775	0.0521269574244539	0.318648231061428	0.685006090465981\\
0.775	0.0538089243380495	0.323608480015096	0.682718662029127\\
0.775	0.0555189414985328	0.328565868787292	0.680522273853469\\
0.775	0.0572570364191156	0.333520171232163	0.678407456997851\\
0.775	0.0590232342988274	0.338471160382329	0.676365666960043\\
0.775	0.0608175580049697	0.34341860847696	0.674389235783156\\
0.775	0.0626400280559541	0.348362286990222	0.672471317134862\\
0.775	0.064490662604533	0.3533019666601	0.670605827172182\\
0.775	0.0663694774214291	0.358237417517579	0.668787383459805\\
0.775	0.0682764858793752	0.363168408916186	0.667011243730728\\
0.775	0.0702116989375697	0.368094709561873	0.665273245863649\\
0.775	0.0721751251265572	0.373016087543259	0.663569750098933\\
0.775	0.0741667705335422	0.377932310362199	0.661897584220061\\
0.775	0.0761866387881432	0.382843144964686	0.660253992184442\\
0.775	0.0782347310485954	0.38774835777208	0.658636586490997\\
0.775	0.0803110459884098	0.392647714712651	0.65704330441587\\
0.775	0.0824155797834956	0.397540981253432	0.655472368126509\\
0.775	0.084548326099754	0.402427922432375	0.653922248592748\\
0.775	0.0867092760811502	0.4073083028908	0.652391633146572\\
0.775	0.0888984183382709	0.412181886906127	0.650879396495578\\
0.775	0.0911157389373742	0.417048438424897	0.649384574964843\\
0.775	0.0933612213899392	0.421907721096044	0.647906343724685\\
0.775	0.0956348466427212	0.42675949830445	0.646443996754806\\
0.775	0.0979365930683194	0.431603533204733	0.644996929295961\\
0.775	0.100266436456264	0.436439588755293	0.643564622546847\\
0.775	0.102624350004627	0.441267427752585	0.642146630374432\\
0.775	0.105010304312169	0.446086812865616	0.640742567819433\\
0.775	0.107424267371012	0.450897506670668	0.63935210119373\\
0.775	0.109866204559871	0.455699271686218	0.637974939582441\\
0.775	0.112336078637817	0.460491870408058	0.636610827579861\\
0.775	0.114833849738607	0.465275065344603	0.635259539104383\\
0.775	0.117359475365564	0.470048619052377	0.633920872153155\\
0.775	0.119912910387022	0.474812294171664	0.632594644371965\\
0.775	0.122494107032347	0.479565853462319	0.631280689329642\\
0.775	0.125103014888515	0.484309059839721	0.629978853399059\\
0.775	0.127739580897283	0.489041676410868	0.628690460824557\\
0.78	0	0	0.710202759607373\\
0.78	1.11327767495586e-05	0.00471862581271274	0.713003157837853\\
0.78	4.46175251031912e-05	0.009446325184051	0.715846486660142\\
0.78	0.000100583311362513	0.0141829653360114	0.718788838379794\\
0.78	0.000179158434668431	0.018928411755669	0.721886092228356\\
0.78	0.000280470402701511	0.02368252819604	0.725200636220892\\
0.78	0.000404645907256436	0.0284451766772965	0.728799085274723\\
0.78	0.000551810799695644	0.0332162174883388	0.732749123983336\\
0.78	0.000722090066287311	0.037995509188729	0.737115575253327\\
0.78	0.000915607803432999	0.0427829086109896	0.741955893907011\\
0.78	0.0011324871927904	0.0475782708632729	0.747315373652289\\
0.78	0.00137285047629673	0.0523814493324038	0.753222424045811\\
0.78	0.00163681893109844	0.057192295687301	0.759684310525653\\
0.78	0.00192451284439304	0.0620106598827802	0.766683748102665\\
0.78	0.00223605148818898	0.0668363901637434	0.774176695511256\\
0.78	0.00257155309398959	0.0716693330697584	0.782091614494494\\
0.78	0.00293113482740716	0.0765093334400312	0.79033034642109\\
0.78	0.00331491276271365	0.0813562344187764	0.79877062752665\\
0.78	0.00372300185733413	0.0862098774609879	0.807270128930359\\
0.78	0.00415551592628969	0.0910701023386145	0.815671782700194\\
0.78	0.00461256761659621	0.0959367471471425	0.82381005350733\\
0.78	0.00509426838162598	0.100809648312589	0.831517746503354\\
0.78	0.00560072845543873	0.105688640598912	0.838632911419451\\
0.78	0.00613205682708918	0.11057355711583	0.845005411376293\\
0.78	0.00668836121491816	0.115464229327074	0.850502769049303\\
0.78	0.00726974804083431	0.120360487059049	0.855014975717657\\
0.78	0.00787632240459385	0.125262158509929	0.858458041096606\\
0.78	0.00850818805808555	0.13016907025918	0.860776163494183\\
0.78	0.00916544737962849	0.135081047277508	0.861942500814224\\
0.78	0.00984820134829017	0.139997912937243	0.86195861464842\\
0.78	0.0105565495182326	0.144919489023162	0.860852735617093\\
0.78	0.0112905899930939	0.149845595743738	0.858677054109653\\
0.78	0.012050419400414	0.154776051742839	0.855504274967457\\
0.78	0.0128361328661109	0.159710674111862	0.851423687967755\\
0.78	0.0136478239890177	0.164649278402306	0.846537000484342\\
0.78	0.0144855848154856	0.169591678638796	0.840954157904247\\
0.78	0.0153495058140643	0.174537687332543	0.834789345403935\\
0.78	0.0162396758502651	0.179487115495259	0.8281573258062\\
0.78	0.0171561821614173	0.184439772653511	0.821170226448693\\
0.78	0.0180991103316243	0.189395466863524	0.813934846725692\\
0.78	0.0190685442668299	0.194354004726437	0.806550519904213\\
0.78	0.0200645661700016	0.199315191403999	0.799107529846304\\
0.78	0.0210872565164405	0.204278830634725	0.791686056496193\\
0.78	0.0221366940292259	0.20924472475049	0.784355603830601\\
0.78	0.023212955654804	0.214212674693577	0.777174850281118\\
0.78	0.0243161165387281	0.219182480034174	0.770191853862645\\
0.78	0.025446250001561	0.224153938988321	0.763444541554459\\
0.78	0.0266034275149466	0.229126848436298	0.756961413910484\\
0.78	0.0277877186778607	0.234101003941464	0.750762400416765\\
0.78	0.0289991911930496	0.239076199769547	0.744859807805784\\
0.78	0.0302379108436654	0.244052228908366	0.739259311518081\\
0.78	0.0315039414701067	0.24902888308801	0.733960949044504\\
0.78	0.0327973449470747	0.254005952801449	0.728960082407259\\
0.78	0.0341181811608519	0.258983227325591	0.724248305111179\\
0.78	0.0354665079868145	0.263960494742775	0.719814276220681\\
0.78	0.0368423812671862	0.268937541962694	0.715644470612629\\
0.78	0.0382458547890421	0.273914154744766	0.711723839838043\\
0.78	0.0396769802625738	0.278890117720924	0.70803638238973\\
0.78	0.0411358072996216	0.283865214418836	0.704565625567163\\
0.78	0.0426223833924861	0.288839227285554	0.701295023640316\\
0.78	0.0441367538930258	0.293811937711588	0.698208278749353\\
0.78	0.0456789619920503	0.298783126055387	0.69528959205535\\
0.78	0.0472490486990192	0.303752571668251	0.692523853197513\\
0.78	0.0488470528220537	0.308720052919643	0.689896776228118\\
0.78	0.0504730109482718	0.313685347222913	0.68739498999028\\
0.78	0.0521269574244539	0.318648231061427	0.685006090465981\\
0.78	0.0538089243380495	0.323608480015096	0.682718662029128\\
0.78	0.0555189414985328	0.328565868787292	0.680522273853469\\
0.78	0.0572570364191156	0.333520171232163	0.678407456997851\\
0.78	0.0590232342988274	0.338471160382329	0.676365666960045\\
0.78	0.0608175580049697	0.34341860847696	0.674389235783156\\
0.78	0.0626400280559541	0.348362286990222	0.672471317134862\\
0.78	0.064490662604533	0.3533019666601	0.670605827172182\\
0.78	0.0663694774214291	0.358237417517579	0.668787383459803\\
0.78	0.0682764858793752	0.363168408916186	0.667011243730728\\
0.78	0.0702116989375697	0.368094709561873	0.66527324586365\\
0.78	0.0721751251265572	0.373016087543259	0.663569750098932\\
0.78	0.0741667705335422	0.377932310362199	0.66189758422006\\
0.78	0.0761866387881432	0.382843144964686	0.660253992184443\\
0.78	0.0782347310485954	0.38774835777208	0.658636586490998\\
0.78	0.0803110459884098	0.392647714712651	0.65704330441587\\
0.78	0.0824155797834956	0.397540981253432	0.655472368126507\\
0.78	0.084548326099754	0.402427922432375	0.653922248592745\\
0.78	0.0867092760811502	0.407308302890799	0.652391633146571\\
0.78	0.0888984183382709	0.412181886906127	0.650879396495581\\
0.78	0.0911157389373742	0.417048438424896	0.649384574964844\\
0.78	0.0933612213899392	0.421907721096044	0.647906343724684\\
0.78	0.0956348466427212	0.42675949830445	0.646443996754806\\
0.78	0.0979365930683194	0.431603533204733	0.644996929295963\\
0.78	0.100266436456264	0.436439588755293	0.643564622546846\\
0.78	0.102624350004627	0.441267427752584	0.642146630374429\\
0.78	0.105010304312169	0.446086812865616	0.640742567819435\\
0.78	0.107424267371012	0.450897506670668	0.639352101193731\\
0.78	0.109866204559871	0.455699271686218	0.637974939582443\\
0.78	0.112336078637817	0.460491870408058	0.636610827579862\\
0.78	0.114833849738607	0.465275065344603	0.635259539104382\\
0.78	0.117359475365564	0.470048619052377	0.633920872153155\\
0.78	0.119912910387023	0.474812294171664	0.632594644371967\\
0.78	0.122494107032347	0.479565853462319	0.631280689329641\\
0.78	0.125103014888515	0.484309059839721	0.629978853399059\\
0.78	0.127739580897283	0.489041676410868	0.628690460824558\\
0.785	0	0	0.710202759607373\\
0.785	1.11327767495586e-05	0.00471862581271275	0.713003157837853\\
0.785	4.46175251031912e-05	0.009446325184051	0.715846486660142\\
0.785	0.000100583311362513	0.0141829653360114	0.718788838379794\\
0.785	0.000179158434668431	0.018928411755669	0.721886092228356\\
0.785	0.000280470402701511	0.02368252819604	0.725200636220892\\
0.785	0.000404645907256436	0.0284451766772965	0.728799085274723\\
0.785	0.000551810799695644	0.0332162174883388	0.732749123983336\\
0.785	0.000722090066287311	0.037995509188729	0.737115575253328\\
0.785	0.000915607803432999	0.0427829086109896	0.741955893907011\\
0.785	0.0011324871927904	0.047578270863273	0.747315373652289\\
0.785	0.00137285047629673	0.0523814493324038	0.75322242404581\\
0.785	0.00163681893109844	0.057192295687301	0.759684310525652\\
0.785	0.00192451284439304	0.0620106598827802	0.766683748102665\\
0.785	0.00223605148818898	0.0668363901637434	0.774176695511256\\
0.785	0.00257155309398959	0.0716693330697584	0.782091614494494\\
0.785	0.00293113482740716	0.0765093334400312	0.79033034642109\\
0.785	0.00331491276271365	0.0813562344187764	0.79877062752665\\
0.785	0.00372300185733413	0.0862098774609879	0.807270128930359\\
0.785	0.00415551592628969	0.0910701023386145	0.815671782700194\\
0.785	0.00461256761659621	0.0959367471471424	0.823810053507331\\
0.785	0.00509426838162598	0.100809648312589	0.831517746503354\\
0.785	0.00560072845543873	0.105688640598912	0.838632911419452\\
0.785	0.00613205682708918	0.11057355711583	0.845005411376292\\
0.785	0.00668836121491816	0.115464229327074	0.850502769049302\\
0.785	0.00726974804083431	0.120360487059049	0.855014975717657\\
0.785	0.00787632240459385	0.125262158509929	0.858458041096606\\
0.785	0.00850818805808555	0.13016907025918	0.860776163494182\\
0.785	0.00916544737962849	0.135081047277508	0.861942500814224\\
0.785	0.00984820134829017	0.139997912937243	0.861958614648421\\
0.785	0.0105565495182326	0.144919489023162	0.860852735617091\\
0.785	0.0112905899930939	0.149845595743738	0.858677054109651\\
0.785	0.012050419400414	0.154776051742839	0.855504274967457\\
0.785	0.0128361328661109	0.159710674111862	0.851423687967756\\
0.785	0.0136478239890177	0.164649278402306	0.84653700048434\\
0.785	0.0144855848154856	0.169591678638796	0.840954157904248\\
0.785	0.0153495058140643	0.174537687332543	0.834789345403934\\
0.785	0.0162396758502651	0.179487115495259	0.828157325806199\\
0.785	0.0171561821614173	0.184439772653511	0.821170226448692\\
0.785	0.0180991103316243	0.189395466863524	0.813934846725694\\
0.785	0.0190685442668299	0.194354004726437	0.806550519904211\\
0.785	0.0200645661700016	0.199315191403999	0.799107529846304\\
0.785	0.0210872565164405	0.204278830634725	0.791686056496192\\
0.785	0.0221366940292259	0.20924472475049	0.784355603830602\\
0.785	0.023212955654804	0.214212674693577	0.777174850281119\\
0.785	0.0243161165387281	0.219182480034174	0.770191853862646\\
0.785	0.025446250001561	0.224153938988321	0.763444541554459\\
0.785	0.0266034275149466	0.229126848436298	0.756961413910483\\
0.785	0.0277877186778607	0.234101003941464	0.750762400416763\\
0.785	0.0289991911930496	0.239076199769547	0.744859807805786\\
0.785	0.0302379108436654	0.244052228908366	0.73925931151808\\
0.785	0.0315039414701067	0.24902888308801	0.733960949044503\\
0.785	0.0327973449470747	0.254005952801449	0.728960082407262\\
0.785	0.0341181811608519	0.258983227325592	0.724248305111178\\
0.785	0.0354665079868145	0.263960494742775	0.71981427622068\\
0.785	0.0368423812671862	0.268937541962694	0.715644470612631\\
0.785	0.0382458547890421	0.273914154744766	0.711723839838041\\
0.785	0.0396769802625738	0.278890117720924	0.70803638238973\\
0.785	0.0411358072996216	0.283865214418836	0.704565625567166\\
0.785	0.0426223833924861	0.288839227285554	0.701295023640317\\
0.785	0.0441367538930258	0.293811937711588	0.698208278749353\\
0.785	0.0456789619920503	0.298783126055387	0.695289592055349\\
0.785	0.0472490486990192	0.303752571668251	0.692523853197512\\
0.785	0.0488470528220538	0.308720052919643	0.689896776228119\\
0.785	0.0504730109482718	0.313685347222913	0.687394989990281\\
0.785	0.0521269574244539	0.318648231061428	0.68500609046598\\
0.785	0.0538089243380495	0.323608480015097	0.682718662029128\\
0.785	0.0555189414985328	0.328565868787292	0.680522273853468\\
0.785	0.0572570364191156	0.333520171232163	0.678407456997848\\
0.785	0.0590232342988274	0.338471160382329	0.676365666960045\\
0.785	0.0608175580049697	0.34341860847696	0.674389235783157\\
0.785	0.0626400280559541	0.348362286990222	0.672471317134862\\
0.785	0.064490662604533	0.3533019666601	0.670605827172182\\
0.785	0.0663694774214291	0.358237417517579	0.668787383459804\\
0.785	0.0682764858793752	0.363168408916186	0.667011243730728\\
0.785	0.0702116989375697	0.368094709561873	0.66527324586365\\
0.785	0.0721751251265572	0.373016087543259	0.663569750098932\\
0.785	0.0741667705335422	0.377932310362199	0.66189758422006\\
0.785	0.0761866387881432	0.382843144964686	0.660253992184442\\
0.785	0.0782347310485954	0.38774835777208	0.658636586490997\\
0.785	0.0803110459884098	0.392647714712651	0.657043304415871\\
0.785	0.0824155797834956	0.397540981253432	0.655472368126509\\
0.785	0.084548326099754	0.402427922432375	0.653922248592745\\
0.785	0.0867092760811502	0.4073083028908	0.652391633146569\\
0.785	0.0888984183382709	0.412181886906127	0.650879396495578\\
0.785	0.0911157389373742	0.417048438424896	0.649384574964846\\
0.785	0.0933612213899392	0.421907721096044	0.647906343724685\\
0.785	0.0956348466427212	0.42675949830445	0.646443996754804\\
0.785	0.0979365930683194	0.431603533204733	0.644996929295962\\
0.785	0.100266436456264	0.436439588755293	0.643564622546847\\
0.785	0.102624350004627	0.441267427752584	0.642146630374428\\
0.785	0.105010304312169	0.446086812865616	0.640742567819433\\
0.785	0.107424267371012	0.450897506670668	0.639352101193732\\
0.785	0.109866204559871	0.455699271686218	0.637974939582443\\
0.785	0.112336078637817	0.460491870408058	0.636610827579861\\
0.785	0.114833849738607	0.465275065344603	0.635259539104383\\
0.785	0.117359475365564	0.470048619052377	0.633920872153154\\
0.785	0.119912910387022	0.474812294171664	0.632594644371966\\
0.785	0.122494107032347	0.479565853462319	0.631280689329642\\
0.785	0.125103014888515	0.484309059839721	0.62997885339906\\
0.785	0.127739580897283	0.489041676410868	0.62869046082456\\
0.79	0	0	0.710202759607373\\
0.79	1.11327767495586e-05	0.00471862581271275	0.713003157837853\\
0.79	4.46175251031912e-05	0.009446325184051	0.715846486660142\\
0.79	0.000100583311362513	0.0141829653360114	0.718788838379794\\
0.79	0.000179158434668431	0.018928411755669	0.721886092228356\\
0.79	0.000280470402701511	0.02368252819604	0.725200636220892\\
0.79	0.000404645907256436	0.0284451766772965	0.728799085274723\\
0.79	0.000551810799695644	0.0332162174883388	0.732749123983336\\
0.79	0.000722090066287311	0.037995509188729	0.737115575253327\\
0.79	0.000915607803432999	0.0427829086109896	0.741955893907011\\
0.79	0.0011324871927904	0.047578270863273	0.747315373652289\\
0.79	0.00137285047629673	0.0523814493324038	0.753222424045811\\
0.79	0.00163681893109844	0.057192295687301	0.759684310525652\\
0.79	0.00192451284439304	0.0620106598827802	0.766683748102665\\
0.79	0.00223605148818898	0.0668363901637434	0.774176695511256\\
0.79	0.00257155309398959	0.0716693330697584	0.782091614494494\\
0.79	0.00293113482740716	0.0765093334400312	0.79033034642109\\
0.79	0.00331491276271365	0.0813562344187764	0.79877062752665\\
0.79	0.00372300185733413	0.0862098774609879	0.807270128930359\\
0.79	0.00415551592628969	0.0910701023386145	0.815671782700193\\
0.79	0.00461256761659621	0.0959367471471425	0.823810053507331\\
0.79	0.00509426838162598	0.100809648312589	0.831517746503354\\
0.79	0.00560072845543873	0.105688640598912	0.838632911419452\\
0.79	0.00613205682708918	0.11057355711583	0.845005411376293\\
0.79	0.00668836121491816	0.115464229327074	0.850502769049301\\
0.79	0.00726974804083431	0.120360487059049	0.855014975717657\\
0.79	0.00787632240459385	0.125262158509929	0.858458041096607\\
0.79	0.00850818805808555	0.13016907025918	0.860776163494183\\
0.79	0.00916544737962849	0.135081047277508	0.861942500814224\\
0.79	0.00984820134829017	0.139997912937243	0.861958614648419\\
0.79	0.0105565495182326	0.144919489023162	0.860852735617089\\
0.79	0.0112905899930939	0.149845595743738	0.858677054109649\\
0.79	0.012050419400414	0.154776051742839	0.855504274967456\\
0.79	0.0128361328661109	0.159710674111862	0.851423687967754\\
0.79	0.0136478239890177	0.164649278402306	0.846537000484341\\
0.79	0.0144855848154856	0.169591678638796	0.84095415790425\\
0.79	0.0153495058140643	0.174537687332543	0.834789345403935\\
0.79	0.0162396758502651	0.179487115495259	0.8281573258062\\
0.79	0.0171561821614173	0.184439772653511	0.821170226448695\\
0.79	0.0180991103316243	0.189395466863524	0.813934846725692\\
0.79	0.0190685442668299	0.194354004726437	0.806550519904211\\
0.79	0.0200645661700016	0.199315191403999	0.799107529846304\\
0.79	0.0210872565164405	0.204278830634725	0.791686056496195\\
0.79	0.0221366940292259	0.20924472475049	0.784355603830603\\
0.79	0.023212955654804	0.214212674693577	0.77717485028112\\
0.79	0.0243161165387281	0.219182480034174	0.770191853862645\\
0.79	0.025446250001561	0.224153938988321	0.763444541554458\\
0.79	0.0266034275149466	0.229126848436298	0.756961413910482\\
0.79	0.0277877186778607	0.234101003941464	0.750762400416764\\
0.79	0.0289991911930496	0.239076199769547	0.744859807805785\\
0.79	0.0302379108436654	0.244052228908366	0.739259311518077\\
0.79	0.0315039414701067	0.24902888308801	0.733960949044504\\
0.79	0.0327973449470747	0.254005952801449	0.728960082407262\\
0.79	0.0341181811608519	0.258983227325592	0.724248305111177\\
0.79	0.0354665079868145	0.263960494742775	0.719814276220681\\
0.79	0.0368423812671862	0.268937541962694	0.715644470612629\\
0.79	0.0382458547890421	0.273914154744766	0.711723839838043\\
0.79	0.0396769802625738	0.278890117720924	0.708036382389731\\
0.79	0.0411358072996216	0.283865214418836	0.704565625567163\\
0.79	0.0426223833924861	0.288839227285554	0.701295023640316\\
0.79	0.0441367538930258	0.293811937711588	0.698208278749353\\
0.79	0.0456789619920503	0.298783126055387	0.69528959205535\\
0.79	0.0472490486990192	0.303752571668251	0.692523853197512\\
0.79	0.0488470528220538	0.308720052919643	0.689896776228119\\
0.79	0.0504730109482718	0.313685347222913	0.687394989990283\\
0.79	0.0521269574244539	0.318648231061428	0.68500609046598\\
0.79	0.0538089243380495	0.323608480015096	0.682718662029127\\
0.79	0.0555189414985328	0.328565868787292	0.68052227385347\\
0.79	0.0572570364191156	0.333520171232163	0.678407456997851\\
0.79	0.0590232342988274	0.338471160382329	0.676365666960043\\
0.79	0.0608175580049697	0.34341860847696	0.674389235783156\\
0.79	0.0626400280559541	0.348362286990222	0.672471317134862\\
0.79	0.064490662604533	0.3533019666601	0.670605827172182\\
0.79	0.0663694774214291	0.358237417517579	0.668787383459803\\
0.79	0.0682764858793752	0.363168408916186	0.667011243730728\\
0.79	0.0702116989375697	0.368094709561873	0.665273245863649\\
0.79	0.0721751251265572	0.373016087543259	0.663569750098933\\
0.79	0.0741667705335422	0.377932310362199	0.661897584220061\\
0.79	0.0761866387881432	0.382843144964686	0.660253992184442\\
0.79	0.0782347310485954	0.38774835777208	0.658636586490997\\
0.79	0.0803110459884098	0.392647714712651	0.657043304415871\\
0.79	0.0824155797834956	0.397540981253432	0.65547236812651\\
0.79	0.084548326099754	0.402427922432375	0.653922248592746\\
0.79	0.0867092760811502	0.4073083028908	0.65239163314657\\
0.79	0.0888984183382709	0.412181886906127	0.650879396495578\\
0.79	0.0911157389373742	0.417048438424896	0.649384574964842\\
0.79	0.0933612213899392	0.421907721096044	0.647906343724686\\
0.79	0.0956348466427212	0.42675949830445	0.646443996754806\\
0.79	0.0979365930683194	0.431603533204733	0.644996929295962\\
0.79	0.100266436456264	0.436439588755293	0.643564622546847\\
0.79	0.102624350004627	0.441267427752584	0.642146630374429\\
0.79	0.105010304312169	0.446086812865616	0.640742567819432\\
0.79	0.107424267371012	0.450897506670668	0.639352101193731\\
0.79	0.109866204559871	0.455699271686218	0.637974939582444\\
0.79	0.112336078637817	0.460491870408058	0.636610827579862\\
0.79	0.114833849738607	0.465275065344603	0.635259539104382\\
0.79	0.117359475365564	0.470048619052377	0.633920872153154\\
0.79	0.119912910387023	0.474812294171664	0.632594644371965\\
0.79	0.122494107032347	0.479565853462319	0.63128068932964\\
0.79	0.125103014888515	0.484309059839721	0.629978853399058\\
0.79	0.127739580897283	0.489041676410868	0.628690460824562\\
0.795	0	0	0.710202759607373\\
0.795	1.11327767495586e-05	0.00471862581271274	0.713003157837853\\
0.795	4.46175251031912e-05	0.009446325184051	0.715846486660142\\
0.795	0.000100583311362513	0.0141829653360114	0.718788838379794\\
0.795	0.000179158434668431	0.018928411755669	0.721886092228356\\
0.795	0.000280470402701511	0.02368252819604	0.725200636220892\\
0.795	0.000404645907256436	0.0284451766772965	0.728799085274723\\
0.795	0.000551810799695644	0.0332162174883388	0.732749123983336\\
0.795	0.000722090066287311	0.037995509188729	0.737115575253327\\
0.795	0.000915607803432999	0.0427829086109896	0.741955893907011\\
0.795	0.0011324871927904	0.0475782708632729	0.747315373652289\\
0.795	0.00137285047629673	0.0523814493324038	0.753222424045811\\
0.795	0.00163681893109844	0.057192295687301	0.759684310525652\\
0.795	0.00192451284439304	0.0620106598827802	0.766683748102665\\
0.795	0.00223605148818898	0.0668363901637434	0.774176695511256\\
0.795	0.00257155309398959	0.0716693330697584	0.782091614494493\\
0.795	0.00293113482740716	0.0765093334400312	0.79033034642109\\
0.795	0.00331491276271365	0.0813562344187764	0.79877062752665\\
0.795	0.00372300185733413	0.0862098774609879	0.807270128930359\\
0.795	0.00415551592628969	0.0910701023386145	0.815671782700193\\
0.795	0.00461256761659621	0.0959367471471425	0.82381005350733\\
0.795	0.00509426838162598	0.100809648312589	0.831517746503354\\
0.795	0.00560072845543873	0.105688640598912	0.838632911419451\\
0.795	0.00613205682708918	0.11057355711583	0.845005411376293\\
0.795	0.00668836121491816	0.115464229327074	0.850502769049302\\
0.795	0.00726974804083431	0.120360487059049	0.855014975717657\\
0.795	0.00787632240459385	0.125262158509929	0.858458041096606\\
0.795	0.00850818805808554	0.13016907025918	0.860776163494183\\
0.795	0.00916544737962849	0.135081047277508	0.861942500814225\\
0.795	0.00984820134829017	0.139997912937243	0.86195861464842\\
0.795	0.0105565495182326	0.144919489023162	0.860852735617092\\
0.795	0.0112905899930939	0.149845595743738	0.858677054109652\\
0.795	0.012050419400414	0.154776051742839	0.855504274967458\\
0.795	0.0128361328661109	0.159710674111862	0.851423687967756\\
0.795	0.0136478239890177	0.164649278402306	0.846537000484341\\
0.795	0.0144855848154856	0.169591678638796	0.84095415790425\\
0.795	0.0153495058140643	0.174537687332543	0.834789345403936\\
0.795	0.0162396758502651	0.179487115495259	0.828157325806199\\
0.795	0.0171561821614173	0.184439772653511	0.821170226448694\\
0.795	0.0180991103316243	0.189395466863524	0.813934846725691\\
0.795	0.0190685442668299	0.194354004726437	0.806550519904212\\
0.795	0.0200645661700016	0.199315191403999	0.799107529846306\\
0.795	0.0210872565164405	0.204278830634725	0.791686056496195\\
0.795	0.0221366940292259	0.20924472475049	0.784355603830603\\
0.795	0.023212955654804	0.214212674693577	0.77717485028112\\
0.795	0.0243161165387281	0.219182480034174	0.770191853862642\\
0.795	0.025446250001561	0.224153938988321	0.763444541554457\\
0.795	0.0266034275149466	0.229126848436298	0.756961413910482\\
0.795	0.0277877186778607	0.234101003941464	0.750762400416764\\
0.795	0.0289991911930496	0.239076199769547	0.744859807805783\\
0.795	0.0302379108436654	0.244052228908366	0.739259311518081\\
0.795	0.0315039414701067	0.24902888308801	0.733960949044504\\
0.795	0.0327973449470747	0.254005952801449	0.728960082407259\\
0.795	0.0341181811608519	0.258983227325591	0.724248305111178\\
0.795	0.0354665079868145	0.263960494742775	0.719814276220682\\
0.795	0.0368423812671862	0.268937541962694	0.715644470612629\\
0.795	0.0382458547890421	0.273914154744766	0.711723839838042\\
0.795	0.0396769802625738	0.278890117720924	0.708036382389732\\
0.795	0.0411358072996216	0.283865214418836	0.704565625567163\\
0.795	0.0426223833924861	0.288839227285554	0.701295023640315\\
0.795	0.0441367538930258	0.293811937711588	0.698208278749355\\
0.795	0.0456789619920503	0.298783126055387	0.69528959205535\\
0.795	0.0472490486990192	0.303752571668251	0.69252385319751\\
0.795	0.0488470528220538	0.308720052919643	0.689896776228119\\
0.795	0.0504730109482718	0.313685347222913	0.687394989990282\\
0.795	0.0521269574244539	0.318648231061428	0.685006090465981\\
0.795	0.0538089243380495	0.323608480015096	0.682718662029128\\
0.795	0.0555189414985328	0.328565868787292	0.680522273853469\\
0.795	0.0572570364191156	0.333520171232163	0.67840745699785\\
0.795	0.0590232342988274	0.338471160382329	0.676365666960044\\
0.795	0.0608175580049697	0.34341860847696	0.674389235783157\\
0.795	0.0626400280559541	0.348362286990222	0.672471317134862\\
0.795	0.064490662604533	0.3533019666601	0.670605827172182\\
0.795	0.0663694774214291	0.358237417517579	0.668787383459804\\
0.795	0.0682764858793752	0.363168408916186	0.667011243730728\\
0.795	0.0702116989375697	0.368094709561873	0.665273245863649\\
0.795	0.0721751251265572	0.373016087543259	0.663569750098932\\
0.795	0.0741667705335422	0.377932310362199	0.66189758422006\\
0.795	0.0761866387881432	0.382843144964686	0.660253992184442\\
0.795	0.0782347310485954	0.38774835777208	0.658636586490997\\
0.795	0.0803110459884098	0.392647714712651	0.657043304415871\\
0.795	0.0824155797834956	0.397540981253432	0.655472368126509\\
0.795	0.084548326099754	0.402427922432375	0.653922248592746\\
0.795	0.0867092760811502	0.407308302890799	0.65239163314657\\
0.795	0.0888984183382709	0.412181886906127	0.65087939649558\\
0.795	0.0911157389373742	0.417048438424897	0.649384574964843\\
0.795	0.0933612213899392	0.421907721096044	0.647906343724684\\
0.795	0.0956348466427212	0.42675949830445	0.646443996754806\\
0.795	0.0979365930683194	0.431603533204733	0.644996929295962\\
0.795	0.100266436456264	0.436439588755293	0.643564622546848\\
0.795	0.102624350004627	0.441267427752585	0.642146630374429\\
0.795	0.105010304312169	0.446086812865616	0.640742567819433\\
0.795	0.107424267371012	0.450897506670668	0.639352101193731\\
0.795	0.109866204559871	0.455699271686218	0.637974939582442\\
0.795	0.112336078637817	0.460491870408058	0.636610827579862\\
0.795	0.114833849738607	0.465275065344603	0.635259539104383\\
0.795	0.117359475365564	0.470048619052377	0.633920872153153\\
0.795	0.119912910387022	0.474812294171664	0.632594644371965\\
0.795	0.122494107032347	0.479565853462319	0.631280689329643\\
0.795	0.125103014888515	0.484309059839721	0.629978853399058\\
0.795	0.127739580897283	0.489041676410868	0.628690460824555\\
0.8	0	0	0.710202759607373\\
0.8	1.11327767495586e-05	0.00471862581271275	0.713003157837853\\
0.8	4.46175251031912e-05	0.009446325184051	0.715846486660142\\
0.8	0.000100583311362513	0.0141829653360114	0.718788838379794\\
0.8	0.000179158434668431	0.018928411755669	0.721886092228356\\
0.8	0.000280470402701511	0.02368252819604	0.725200636220892\\
0.8	0.000404645907256436	0.0284451766772965	0.728799085274723\\
0.8	0.000551810799695644	0.0332162174883388	0.732749123983336\\
0.8	0.000722090066287311	0.037995509188729	0.737115575253328\\
0.8	0.000915607803432999	0.0427829086109896	0.741955893907011\\
0.8	0.0011324871927904	0.0475782708632729	0.747315373652289\\
0.8	0.00137285047629673	0.0523814493324038	0.753222424045811\\
0.8	0.00163681893109844	0.057192295687301	0.759684310525652\\
0.8	0.00192451284439304	0.0620106598827802	0.766683748102665\\
0.8	0.00223605148818898	0.0668363901637434	0.774176695511256\\
0.8	0.00257155309398959	0.0716693330697584	0.782091614494493\\
0.8	0.00293113482740716	0.0765093334400312	0.79033034642109\\
0.8	0.00331491276271365	0.0813562344187764	0.79877062752665\\
0.8	0.00372300185733413	0.0862098774609879	0.807270128930359\\
0.8	0.00415551592628969	0.0910701023386145	0.815671782700193\\
0.8	0.00461256761659621	0.0959367471471425	0.82381005350733\\
0.8	0.00509426838162598	0.100809648312589	0.831517746503354\\
0.8	0.00560072845543873	0.105688640598912	0.838632911419452\\
0.8	0.00613205682708918	0.11057355711583	0.845005411376293\\
0.8	0.00668836121491816	0.115464229327074	0.850502769049302\\
0.8	0.00726974804083431	0.120360487059049	0.855014975717656\\
0.8	0.00787632240459385	0.125262158509929	0.858458041096605\\
0.8	0.00850818805808555	0.13016907025918	0.860776163494181\\
0.8	0.00916544737962849	0.135081047277508	0.861942500814225\\
0.8	0.00984820134829017	0.139997912937243	0.861958614648422\\
0.8	0.0105565495182326	0.144919489023162	0.860852735617092\\
0.8	0.0112905899930939	0.149845595743738	0.858677054109652\\
0.8	0.012050419400414	0.154776051742839	0.855504274967457\\
0.8	0.0128361328661109	0.159710674111862	0.851423687967754\\
0.8	0.0136478239890177	0.164649278402306	0.84653700048434\\
0.8	0.0144855848154856	0.169591678638796	0.840954157904248\\
0.8	0.0153495058140643	0.174537687332543	0.834789345403934\\
0.8	0.0162396758502651	0.179487115495259	0.828157325806199\\
0.8	0.0171561821614173	0.184439772653511	0.821170226448695\\
0.8	0.0180991103316243	0.189395466863524	0.813934846725692\\
0.8	0.0190685442668299	0.194354004726437	0.806550519904212\\
0.8	0.0200645661700016	0.199315191403999	0.799107529846305\\
0.8	0.0210872565164405	0.204278830634725	0.791686056496195\\
0.8	0.0221366940292259	0.20924472475049	0.784355603830604\\
0.8	0.023212955654804	0.214212674693577	0.777174850281115\\
0.8	0.0243161165387281	0.219182480034174	0.770191853862642\\
0.8	0.025446250001561	0.224153938988321	0.763444541554458\\
0.8	0.0266034275149466	0.229126848436298	0.756961413910484\\
0.8	0.0277877186778607	0.234101003941464	0.750762400416763\\
0.8	0.0289991911930496	0.239076199769547	0.744859807805786\\
0.8	0.0302379108436654	0.244052228908366	0.73925931151808\\
0.8	0.0315039414701067	0.24902888308801	0.733960949044503\\
0.8	0.0327973449470747	0.254005952801449	0.728960082407262\\
0.8	0.0341181811608519	0.258983227325592	0.724248305111177\\
0.8	0.0354665079868145	0.263960494742775	0.719814276220681\\
0.8	0.0368423812671862	0.268937541962694	0.71564447061263\\
0.8	0.0382458547890421	0.273914154744766	0.711723839838041\\
0.8	0.0396769802625738	0.278890117720924	0.708036382389731\\
0.8	0.0411358072996216	0.283865214418836	0.704565625567165\\
0.8	0.0426223833924861	0.288839227285554	0.701295023640316\\
0.8	0.0441367538930258	0.293811937711588	0.698208278749353\\
0.8	0.0456789619920503	0.298783126055387	0.695289592055351\\
0.8	0.0472490486990192	0.303752571668251	0.692523853197512\\
0.8	0.0488470528220538	0.308720052919643	0.689896776228118\\
0.8	0.0504730109482718	0.313685347222913	0.687394989990281\\
0.8	0.0521269574244539	0.318648231061428	0.685006090465981\\
0.8	0.0538089243380495	0.323608480015097	0.682718662029127\\
0.8	0.0555189414985328	0.328565868787292	0.680522273853469\\
0.8	0.0572570364191156	0.333520171232163	0.67840745699785\\
0.8	0.0590232342988274	0.338471160382329	0.676365666960044\\
0.8	0.0608175580049697	0.34341860847696	0.674389235783157\\
0.8	0.0626400280559541	0.348362286990222	0.672471317134862\\
0.8	0.064490662604533	0.3533019666601	0.670605827172181\\
0.8	0.0663694774214291	0.358237417517579	0.668787383459804\\
0.8	0.0682764858793752	0.363168408916186	0.667011243730729\\
0.8	0.0702116989375697	0.368094709561873	0.665273245863649\\
0.8	0.0721751251265572	0.373016087543259	0.663569750098932\\
0.8	0.0741667705335422	0.377932310362199	0.661897584220061\\
0.8	0.0761866387881432	0.382843144964686	0.660253992184442\\
0.8	0.0782347310485954	0.38774835777208	0.658636586490997\\
0.8	0.0803110459884098	0.392647714712651	0.657043304415871\\
0.8	0.0824155797834956	0.397540981253432	0.655472368126509\\
0.8	0.084548326099754	0.402427922432375	0.653922248592745\\
0.8	0.0867092760811502	0.4073083028908	0.652391633146569\\
0.8	0.0888984183382709	0.412181886906127	0.650879396495578\\
0.8	0.0911157389373742	0.417048438424896	0.649384574964844\\
0.8	0.0933612213899392	0.421907721096044	0.647906343724685\\
0.8	0.0956348466427212	0.42675949830445	0.646443996754805\\
0.8	0.0979365930683194	0.431603533204733	0.644996929295961\\
0.8	0.100266436456264	0.436439588755293	0.643564622546847\\
0.8	0.102624350004627	0.441267427752584	0.64214663037443\\
0.8	0.105010304312169	0.446086812865616	0.640742567819433\\
0.8	0.107424267371012	0.450897506670668	0.639352101193731\\
0.8	0.109866204559871	0.455699271686218	0.637974939582442\\
0.8	0.112336078637817	0.460491870408058	0.636610827579862\\
0.8	0.114833849738607	0.465275065344603	0.635259539104384\\
0.8	0.117359475365564	0.470048619052377	0.633920872153154\\
0.8	0.119912910387023	0.474812294171664	0.632594644371963\\
0.8	0.122494107032347	0.479565853462319	0.631280689329641\\
0.8	0.125103014888515	0.484309059839721	0.62997885339906\\
0.8	0.127739580897283	0.489041676410868	0.628690460824558\\
0.805	0	0	0.710202759607373\\
0.805	1.11327767495586e-05	0.00471862581271275	0.713003157837853\\
0.805	4.46175251031912e-05	0.009446325184051	0.715846486660142\\
0.805	0.000100583311362513	0.0141829653360114	0.718788838379794\\
0.805	0.000179158434668431	0.018928411755669	0.721886092228356\\
0.805	0.000280470402701511	0.02368252819604	0.725200636220892\\
0.805	0.000404645907256436	0.0284451766772965	0.728799085274723\\
0.805	0.000551810799695644	0.0332162174883388	0.732749123983336\\
0.805	0.000722090066287311	0.037995509188729	0.737115575253327\\
0.805	0.000915607803432999	0.0427829086109896	0.741955893907011\\
0.805	0.0011324871927904	0.0475782708632729	0.747315373652289\\
0.805	0.00137285047629673	0.0523814493324038	0.75322242404581\\
0.805	0.00163681893109844	0.057192295687301	0.759684310525652\\
0.805	0.00192451284439304	0.0620106598827802	0.766683748102665\\
0.805	0.00223605148818899	0.0668363901637434	0.774176695511256\\
0.805	0.00257155309398959	0.0716693330697584	0.782091614494493\\
0.805	0.00293113482740716	0.0765093334400312	0.79033034642109\\
0.805	0.00331491276271365	0.0813562344187764	0.79877062752665\\
0.805	0.00372300185733413	0.0862098774609879	0.80727012893036\\
0.805	0.00415551592628969	0.0910701023386145	0.815671782700194\\
0.805	0.00461256761659621	0.0959367471471425	0.82381005350733\\
0.805	0.00509426838162598	0.100809648312589	0.831517746503354\\
0.805	0.00560072845543873	0.105688640598912	0.838632911419452\\
0.805	0.00613205682708918	0.11057355711583	0.845005411376291\\
0.805	0.00668836121491815	0.115464229327074	0.850502769049302\\
0.805	0.00726974804083431	0.120360487059049	0.855014975717657\\
0.805	0.00787632240459385	0.125262158509929	0.858458041096606\\
0.805	0.00850818805808555	0.13016907025918	0.860776163494182\\
0.805	0.00916544737962849	0.135081047277508	0.861942500814223\\
0.805	0.00984820134829017	0.139997912937243	0.86195861464842\\
0.805	0.0105565495182326	0.144919489023162	0.860852735617091\\
0.805	0.0112905899930939	0.149845595743738	0.858677054109651\\
0.805	0.012050419400414	0.154776051742839	0.855504274967458\\
0.805	0.0128361328661109	0.159710674111862	0.851423687967756\\
0.805	0.0136478239890177	0.164649278402306	0.846537000484341\\
0.805	0.0144855848154856	0.169591678638796	0.840954157904247\\
0.805	0.0153495058140643	0.174537687332543	0.834789345403934\\
0.805	0.0162396758502651	0.179487115495259	0.8281573258062\\
0.805	0.0171561821614173	0.184439772653511	0.821170226448695\\
0.805	0.0180991103316243	0.189395466863524	0.813934846725692\\
0.805	0.0190685442668299	0.194354004726437	0.806550519904212\\
0.805	0.0200645661700016	0.199315191403999	0.799107529846305\\
0.805	0.0210872565164405	0.204278830634725	0.791686056496194\\
0.805	0.0221366940292259	0.20924472475049	0.7843556038306\\
0.805	0.023212955654804	0.214212674693577	0.777174850281117\\
0.805	0.0243161165387281	0.219182480034174	0.770191853862645\\
0.805	0.025446250001561	0.224153938988321	0.763444541554459\\
0.805	0.0266034275149466	0.229126848436298	0.756961413910484\\
0.805	0.0277877186778607	0.234101003941464	0.750762400416765\\
0.805	0.0289991911930496	0.239076199769547	0.744859807805785\\
0.805	0.0302379108436654	0.244052228908366	0.739259311518078\\
0.805	0.0315039414701067	0.24902888308801	0.733960949044503\\
0.805	0.0327973449470747	0.254005952801449	0.72896008240726\\
0.805	0.0341181811608519	0.258983227325592	0.724248305111178\\
0.805	0.0354665079868145	0.263960494742775	0.719814276220681\\
0.805	0.0368423812671862	0.268937541962694	0.715644470612629\\
0.805	0.0382458547890421	0.273914154744766	0.711723839838041\\
0.805	0.0396769802625738	0.278890117720924	0.70803638238973\\
0.805	0.0411358072996216	0.283865214418836	0.704565625567164\\
0.805	0.0426223833924861	0.288839227285554	0.701295023640316\\
0.805	0.0441367538930258	0.293811937711588	0.698208278749354\\
0.805	0.0456789619920503	0.298783126055387	0.69528959205535\\
0.805	0.0472490486990192	0.303752571668251	0.692523853197512\\
0.805	0.0488470528220538	0.308720052919643	0.689896776228118\\
0.805	0.0504730109482718	0.313685347222913	0.687394989990281\\
0.805	0.0521269574244539	0.318648231061428	0.685006090465981\\
0.805	0.0538089243380495	0.323608480015096	0.682718662029128\\
0.805	0.0555189414985328	0.328565868787292	0.68052227385347\\
0.805	0.0572570364191156	0.333520171232163	0.67840745699785\\
0.805	0.0590232342988274	0.338471160382329	0.676365666960043\\
0.805	0.0608175580049697	0.34341860847696	0.674389235783156\\
0.805	0.0626400280559541	0.348362286990222	0.672471317134862\\
0.805	0.064490662604533	0.3533019666601	0.670605827172183\\
0.805	0.066369477421429	0.358237417517579	0.668787383459803\\
0.805	0.0682764858793752	0.363168408916186	0.667011243730727\\
0.805	0.0702116989375697	0.368094709561873	0.66527324586365\\
0.805	0.0721751251265572	0.373016087543259	0.663569750098933\\
0.805	0.0741667705335422	0.377932310362199	0.66189758422006\\
0.805	0.0761866387881432	0.382843144964686	0.660253992184442\\
0.805	0.0782347310485954	0.38774835777208	0.658636586490998\\
0.805	0.0803110459884098	0.392647714712651	0.657043304415871\\
0.805	0.0824155797834956	0.397540981253432	0.655472368126509\\
0.805	0.084548326099754	0.402427922432375	0.653922248592746\\
0.805	0.0867092760811502	0.4073083028908	0.65239163314657\\
0.805	0.0888984183382708	0.412181886906127	0.650879396495579\\
0.805	0.0911157389373742	0.417048438424897	0.649384574964843\\
0.805	0.0933612213899392	0.421907721096044	0.647906343724684\\
0.805	0.0956348466427212	0.42675949830445	0.646443996754806\\
0.805	0.0979365930683194	0.431603533204733	0.644996929295962\\
0.805	0.100266436456264	0.436439588755293	0.643564622546846\\
0.805	0.102624350004627	0.441267427752585	0.64214663037443\\
0.805	0.105010304312169	0.446086812865616	0.640742567819434\\
0.805	0.107424267371012	0.450897506670668	0.639352101193731\\
0.805	0.109866204559871	0.455699271686218	0.637974939582442\\
0.805	0.112336078637817	0.460491870408058	0.636610827579861\\
0.805	0.114833849738607	0.465275065344603	0.635259539104382\\
0.805	0.117359475365564	0.470048619052377	0.633920872153155\\
0.805	0.119912910387022	0.474812294171664	0.632594644371966\\
0.805	0.122494107032347	0.479565853462319	0.63128068932964\\
0.805	0.125103014888515	0.484309059839721	0.629978853399059\\
0.805	0.127739580897283	0.489041676410868	0.628690460824563\\
0.81	0	0	0.710202759607373\\
0.81	1.11327767495586e-05	0.00471862581271275	0.713003157837853\\
0.81	4.46175251031912e-05	0.009446325184051	0.715846486660142\\
0.81	0.000100583311362513	0.0141829653360114	0.718788838379794\\
0.81	0.000179158434668431	0.018928411755669	0.721886092228357\\
0.81	0.000280470402701511	0.02368252819604	0.725200636220892\\
0.81	0.000404645907256436	0.0284451766772965	0.728799085274723\\
0.81	0.000551810799695644	0.0332162174883389	0.732749123983336\\
0.81	0.000722090066287311	0.037995509188729	0.737115575253328\\
0.81	0.000915607803432999	0.0427829086109896	0.741955893907011\\
0.81	0.0011324871927904	0.047578270863273	0.747315373652289\\
0.81	0.00137285047629673	0.0523814493324038	0.753222424045811\\
0.81	0.00163681893109844	0.057192295687301	0.759684310525652\\
0.81	0.00192451284439304	0.0620106598827802	0.766683748102665\\
0.81	0.00223605148818898	0.0668363901637434	0.774176695511256\\
0.81	0.00257155309398959	0.0716693330697584	0.782091614494494\\
0.81	0.00293113482740716	0.0765093334400312	0.79033034642109\\
0.81	0.00331491276271365	0.0813562344187764	0.79877062752665\\
0.81	0.00372300185733413	0.0862098774609879	0.807270128930359\\
0.81	0.00415551592628969	0.0910701023386145	0.815671782700193\\
0.81	0.00461256761659621	0.0959367471471425	0.82381005350733\\
0.81	0.00509426838162598	0.100809648312589	0.831517746503353\\
0.81	0.00560072845543873	0.105688640598912	0.838632911419452\\
0.81	0.00613205682708918	0.11057355711583	0.845005411376292\\
0.81	0.00668836121491816	0.115464229327074	0.850502769049301\\
0.81	0.00726974804083431	0.120360487059049	0.855014975717657\\
0.81	0.00787632240459385	0.125262158509929	0.858458041096606\\
0.81	0.00850818805808555	0.13016907025918	0.860776163494182\\
0.81	0.00916544737962849	0.135081047277508	0.861942500814224\\
0.81	0.00984820134829017	0.139997912937243	0.86195861464842\\
0.81	0.0105565495182326	0.144919489023162	0.86085273561709\\
0.81	0.0112905899930939	0.149845595743738	0.858677054109651\\
0.81	0.012050419400414	0.154776051742839	0.855504274967458\\
0.81	0.0128361328661109	0.159710674111862	0.851423687967755\\
0.81	0.0136478239890177	0.164649278402306	0.846537000484339\\
0.81	0.0144855848154856	0.169591678638796	0.840954157904248\\
0.81	0.0153495058140643	0.174537687332543	0.834789345403934\\
0.81	0.0162396758502651	0.179487115495259	0.828157325806202\\
0.81	0.0171561821614173	0.184439772653511	0.821170226448694\\
0.81	0.0180991103316243	0.189395466863524	0.813934846725691\\
0.81	0.0190685442668299	0.194354004726437	0.806550519904211\\
0.81	0.0200645661700016	0.199315191403999	0.799107529846304\\
0.81	0.0210872565164405	0.204278830634725	0.791686056496195\\
0.81	0.0221366940292259	0.20924472475049	0.784355603830601\\
0.81	0.023212955654804	0.214212674693577	0.777174850281119\\
0.81	0.0243161165387281	0.219182480034174	0.770191853862644\\
0.81	0.025446250001561	0.224153938988321	0.763444541554458\\
0.81	0.0266034275149466	0.229126848436298	0.756961413910483\\
0.81	0.0277877186778607	0.234101003941464	0.750762400416763\\
0.81	0.0289991911930496	0.239076199769547	0.744859807805782\\
0.81	0.0302379108436654	0.244052228908366	0.73925931151808\\
0.81	0.0315039414701067	0.24902888308801	0.733960949044504\\
0.81	0.0327973449470747	0.254005952801449	0.72896008240726\\
0.81	0.0341181811608519	0.258983227325592	0.724248305111177\\
0.81	0.0354665079868145	0.263960494742775	0.719814276220681\\
0.81	0.0368423812671862	0.268937541962694	0.715644470612631\\
0.81	0.0382458547890421	0.273914154744766	0.711723839838041\\
0.81	0.0396769802625738	0.278890117720924	0.708036382389729\\
0.81	0.0411358072996216	0.283865214418836	0.704565625567164\\
0.81	0.0426223833924861	0.288839227285554	0.701295023640317\\
0.81	0.0441367538930258	0.293811937711588	0.698208278749354\\
0.81	0.0456789619920503	0.298783126055387	0.69528959205535\\
0.81	0.0472490486990192	0.303752571668251	0.692523853197512\\
0.81	0.0488470528220538	0.308720052919643	0.689896776228118\\
0.81	0.0504730109482718	0.313685347222913	0.687394989990281\\
0.81	0.0521269574244539	0.318648231061428	0.68500609046598\\
0.81	0.0538089243380495	0.323608480015096	0.682718662029127\\
0.81	0.0555189414985328	0.328565868787292	0.68052227385347\\
0.81	0.0572570364191156	0.333520171232163	0.67840745699785\\
0.81	0.0590232342988274	0.338471160382329	0.676365666960043\\
0.81	0.0608175580049697	0.34341860847696	0.674389235783156\\
0.81	0.0626400280559541	0.348362286990222	0.672471317134861\\
0.81	0.064490662604533	0.3533019666601	0.670605827172182\\
0.81	0.0663694774214291	0.358237417517579	0.668787383459805\\
0.81	0.0682764858793752	0.363168408916186	0.667011243730728\\
0.81	0.0702116989375697	0.368094709561873	0.665273245863649\\
0.81	0.0721751251265572	0.373016087543259	0.663569750098932\\
0.81	0.0741667705335422	0.377932310362199	0.66189758422006\\
0.81	0.0761866387881432	0.382843144964686	0.660253992184442\\
0.81	0.0782347310485954	0.38774835777208	0.658636586490997\\
0.81	0.0803110459884098	0.392647714712651	0.657043304415871\\
0.81	0.0824155797834956	0.397540981253432	0.655472368126508\\
0.81	0.084548326099754	0.402427922432375	0.653922248592746\\
0.81	0.0867092760811502	0.4073083028908	0.65239163314657\\
0.81	0.0888984183382709	0.412181886906127	0.650879396495579\\
0.81	0.0911157389373742	0.417048438424897	0.649384574964845\\
0.81	0.0933612213899392	0.421907721096044	0.647906343724684\\
0.81	0.0956348466427212	0.42675949830445	0.646443996754806\\
0.81	0.0979365930683194	0.431603533204733	0.644996929295962\\
0.81	0.100266436456264	0.436439588755293	0.643564622546846\\
0.81	0.102624350004627	0.441267427752584	0.64214663037443\\
0.81	0.105010304312169	0.446086812865616	0.640742567819434\\
0.81	0.107424267371012	0.450897506670668	0.639352101193731\\
0.81	0.109866204559871	0.455699271686218	0.637974939582443\\
0.81	0.112336078637817	0.460491870408058	0.636610827579863\\
0.81	0.114833849738607	0.465275065344603	0.635259539104383\\
0.81	0.117359475365564	0.470048619052377	0.633920872153154\\
0.81	0.119912910387023	0.474812294171664	0.632594644371964\\
0.81	0.122494107032347	0.479565853462319	0.631280689329641\\
0.81	0.125103014888515	0.484309059839721	0.629978853399057\\
0.81	0.127739580897283	0.489041676410868	0.628690460824558\\
0.815	0	0	0.710202759607373\\
0.815	1.11327767495586e-05	0.00471862581271275	0.713003157837853\\
0.815	4.46175251031912e-05	0.009446325184051	0.715846486660142\\
0.815	0.000100583311362513	0.0141829653360114	0.718788838379794\\
0.815	0.000179158434668431	0.018928411755669	0.721886092228356\\
0.815	0.000280470402701511	0.02368252819604	0.725200636220892\\
0.815	0.000404645907256436	0.0284451766772965	0.728799085274723\\
0.815	0.000551810799695644	0.0332162174883389	0.732749123983336\\
0.815	0.000722090066287311	0.037995509188729	0.737115575253327\\
0.815	0.000915607803432999	0.0427829086109896	0.741955893907011\\
0.815	0.0011324871927904	0.0475782708632729	0.747315373652289\\
0.815	0.00137285047629673	0.0523814493324038	0.753222424045811\\
0.815	0.00163681893109844	0.057192295687301	0.759684310525652\\
0.815	0.00192451284439304	0.0620106598827802	0.766683748102665\\
0.815	0.00223605148818898	0.0668363901637434	0.774176695511256\\
0.815	0.00257155309398959	0.0716693330697584	0.782091614494494\\
0.815	0.00293113482740716	0.0765093334400312	0.79033034642109\\
0.815	0.00331491276271365	0.0813562344187764	0.798770627526651\\
0.815	0.00372300185733413	0.0862098774609879	0.807270128930359\\
0.815	0.00415551592628969	0.0910701023386145	0.815671782700194\\
0.815	0.00461256761659621	0.0959367471471424	0.82381005350733\\
0.815	0.00509426838162598	0.100809648312589	0.831517746503354\\
0.815	0.00560072845543873	0.105688640598912	0.838632911419452\\
0.815	0.00613205682708918	0.11057355711583	0.845005411376293\\
0.815	0.00668836121491816	0.115464229327074	0.850502769049302\\
0.815	0.00726974804083431	0.120360487059049	0.855014975717656\\
0.815	0.00787632240459385	0.125262158509929	0.858458041096606\\
0.815	0.00850818805808555	0.13016907025918	0.860776163494182\\
0.815	0.00916544737962849	0.135081047277508	0.861942500814225\\
0.815	0.00984820134829017	0.139997912937243	0.861958614648419\\
0.815	0.0105565495182326	0.144919489023162	0.86085273561709\\
0.815	0.0112905899930939	0.149845595743738	0.858677054109649\\
0.815	0.012050419400414	0.154776051742839	0.855504274967455\\
0.815	0.0128361328661109	0.159710674111862	0.851423687967753\\
0.815	0.0136478239890177	0.164649278402306	0.84653700048434\\
0.815	0.0144855848154856	0.169591678638796	0.84095415790425\\
0.815	0.0153495058140643	0.174537687332543	0.834789345403935\\
0.815	0.0162396758502651	0.179487115495259	0.8281573258062\\
0.815	0.0171561821614173	0.184439772653511	0.821170226448693\\
0.815	0.0180991103316243	0.189395466863524	0.813934846725692\\
0.815	0.0190685442668299	0.194354004726437	0.806550519904212\\
0.815	0.0200645661700016	0.199315191403999	0.799107529846306\\
0.815	0.0210872565164405	0.204278830634725	0.791686056496195\\
0.815	0.0221366940292259	0.20924472475049	0.784355603830601\\
0.815	0.023212955654804	0.214212674693577	0.777174850281119\\
0.815	0.0243161165387281	0.219182480034174	0.770191853862642\\
0.815	0.025446250001561	0.224153938988321	0.763444541554458\\
0.815	0.0266034275149466	0.229126848436298	0.756961413910483\\
0.815	0.0277877186778607	0.234101003941464	0.750762400416762\\
0.815	0.0289991911930496	0.239076199769547	0.744859807805785\\
0.815	0.0302379108436654	0.244052228908366	0.739259311518081\\
0.815	0.0315039414701067	0.24902888308801	0.733960949044503\\
0.815	0.0327973449470747	0.254005952801449	0.72896008240726\\
0.815	0.0341181811608519	0.258983227325592	0.724248305111177\\
0.815	0.0354665079868145	0.263960494742775	0.719814276220682\\
0.815	0.0368423812671862	0.268937541962694	0.71564447061263\\
0.815	0.0382458547890421	0.273914154744766	0.711723839838042\\
0.815	0.0396769802625738	0.278890117720924	0.708036382389731\\
0.815	0.0411358072996216	0.283865214418836	0.704565625567163\\
0.815	0.0426223833924861	0.288839227285554	0.701295023640316\\
0.815	0.0441367538930258	0.293811937711588	0.698208278749354\\
0.815	0.0456789619920503	0.298783126055387	0.69528959205535\\
0.815	0.0472490486990192	0.303752571668251	0.692523853197512\\
0.815	0.0488470528220538	0.308720052919643	0.689896776228119\\
0.815	0.0504730109482718	0.313685347222913	0.687394989990281\\
0.815	0.0521269574244539	0.318648231061428	0.68500609046598\\
0.815	0.0538089243380495	0.323608480015096	0.682718662029127\\
0.815	0.0555189414985328	0.328565868787292	0.68052227385347\\
0.815	0.0572570364191156	0.333520171232163	0.678407456997851\\
0.815	0.0590232342988274	0.338471160382329	0.676365666960043\\
0.815	0.0608175580049697	0.34341860847696	0.674389235783156\\
0.815	0.0626400280559541	0.348362286990222	0.672471317134862\\
0.815	0.064490662604533	0.3533019666601	0.670605827172182\\
0.815	0.0663694774214291	0.358237417517579	0.668787383459804\\
0.815	0.0682764858793752	0.363168408916186	0.667011243730728\\
0.815	0.0702116989375697	0.368094709561873	0.665273245863648\\
0.815	0.0721751251265572	0.373016087543259	0.663569750098932\\
0.815	0.0741667705335422	0.377932310362199	0.66189758422006\\
0.815	0.0761866387881432	0.382843144964686	0.660253992184442\\
0.815	0.0782347310485954	0.38774835777208	0.658636586490997\\
0.815	0.0803110459884098	0.392647714712651	0.657043304415873\\
0.815	0.0824155797834956	0.397540981253432	0.655472368126509\\
0.815	0.084548326099754	0.402427922432375	0.653922248592744\\
0.815	0.0867092760811502	0.4073083028908	0.65239163314657\\
0.815	0.0888984183382709	0.412181886906127	0.650879396495579\\
0.815	0.0911157389373742	0.417048438424896	0.649384574964844\\
0.815	0.0933612213899392	0.421907721096044	0.647906343724685\\
0.815	0.0956348466427212	0.42675949830445	0.646443996754804\\
0.815	0.0979365930683194	0.431603533204733	0.644996929295962\\
0.815	0.100266436456264	0.436439588755293	0.643564622546846\\
0.815	0.102624350004627	0.441267427752584	0.64214663037443\\
0.815	0.105010304312169	0.446086812865616	0.640742567819434\\
0.815	0.107424267371012	0.450897506670668	0.639352101193731\\
0.815	0.109866204559871	0.455699271686218	0.637974939582442\\
0.815	0.112336078637817	0.460491870408058	0.636610827579862\\
0.815	0.114833849738607	0.465275065344603	0.635259539104384\\
0.815	0.117359475365564	0.470048619052377	0.633920872153155\\
0.815	0.119912910387022	0.474812294171664	0.632594644371964\\
0.815	0.122494107032347	0.479565853462319	0.631280689329641\\
0.815	0.125103014888515	0.484309059839721	0.629978853399058\\
0.815	0.127739580897283	0.489041676410868	0.628690460824554\\
0.82	0	0	0.710202759607373\\
0.82	1.11327767495586e-05	0.00471862581271275	0.713003157837853\\
0.82	4.46175251031912e-05	0.009446325184051	0.715846486660142\\
0.82	0.000100583311362513	0.0141829653360114	0.718788838379794\\
0.82	0.000179158434668431	0.018928411755669	0.721886092228356\\
0.82	0.000280470402701511	0.02368252819604	0.725200636220892\\
0.82	0.000404645907256436	0.0284451766772965	0.728799085274723\\
0.82	0.000551810799695644	0.0332162174883388	0.732749123983336\\
0.82	0.000722090066287311	0.037995509188729	0.737115575253327\\
0.82	0.000915607803432999	0.0427829086109896	0.741955893907011\\
0.82	0.0011324871927904	0.0475782708632729	0.747315373652289\\
0.82	0.00137285047629673	0.0523814493324038	0.753222424045811\\
0.82	0.00163681893109843	0.057192295687301	0.759684310525652\\
0.82	0.00192451284439304	0.0620106598827802	0.766683748102665\\
0.82	0.00223605148818898	0.0668363901637434	0.774176695511256\\
0.82	0.00257155309398959	0.0716693330697584	0.782091614494494\\
0.82	0.00293113482740716	0.0765093334400312	0.79033034642109\\
0.82	0.00331491276271365	0.0813562344187764	0.79877062752665\\
0.82	0.00372300185733413	0.0862098774609879	0.807270128930359\\
0.82	0.00415551592628969	0.0910701023386145	0.815671782700193\\
0.82	0.00461256761659621	0.0959367471471424	0.823810053507331\\
0.82	0.00509426838162598	0.100809648312589	0.831517746503354\\
0.82	0.00560072845543873	0.105688640598912	0.838632911419452\\
0.82	0.00613205682708918	0.11057355711583	0.845005411376293\\
0.82	0.00668836121491815	0.115464229327074	0.850502769049302\\
0.82	0.00726974804083431	0.120360487059049	0.855014975717655\\
0.82	0.00787632240459385	0.125262158509929	0.858458041096606\\
0.82	0.00850818805808554	0.13016907025918	0.860776163494182\\
0.82	0.00916544737962849	0.135081047277508	0.861942500814224\\
0.82	0.00984820134829017	0.139997912937243	0.86195861464842\\
0.82	0.0105565495182326	0.144919489023162	0.860852735617092\\
0.82	0.0112905899930939	0.149845595743738	0.85867705410965\\
0.82	0.012050419400414	0.154776051742839	0.855504274967457\\
0.82	0.0128361328661109	0.159710674111862	0.851423687967756\\
0.82	0.0136478239890177	0.164649278402306	0.846537000484341\\
0.82	0.0144855848154856	0.169591678638796	0.840954157904249\\
0.82	0.0153495058140643	0.174537687332543	0.834789345403935\\
0.82	0.0162396758502651	0.179487115495259	0.8281573258062\\
0.82	0.0171561821614173	0.184439772653511	0.821170226448695\\
0.82	0.0180991103316243	0.189395466863524	0.813934846725693\\
0.82	0.0190685442668299	0.194354004726437	0.806550519904212\\
0.82	0.0200645661700016	0.199315191403999	0.799107529846306\\
0.82	0.0210872565164405	0.204278830634725	0.791686056496193\\
0.82	0.0221366940292259	0.20924472475049	0.784355603830604\\
0.82	0.023212955654804	0.214212674693577	0.777174850281119\\
0.82	0.0243161165387281	0.219182480034174	0.770191853862642\\
0.82	0.025446250001561	0.224153938988321	0.763444541554457\\
0.82	0.0266034275149466	0.229126848436298	0.756961413910482\\
0.82	0.0277877186778607	0.234101003941464	0.750762400416765\\
0.82	0.0289991911930496	0.239076199769547	0.744859807805785\\
0.82	0.0302379108436654	0.244052228908366	0.739259311518081\\
0.82	0.0315039414701067	0.24902888308801	0.733960949044503\\
0.82	0.0327973449470747	0.254005952801449	0.72896008240726\\
0.82	0.0341181811608519	0.258983227325592	0.724248305111179\\
0.82	0.0354665079868145	0.263960494742775	0.719814276220681\\
0.82	0.0368423812671862	0.268937541962694	0.715644470612629\\
0.82	0.0382458547890421	0.273914154744766	0.711723839838043\\
0.82	0.0396769802625738	0.278890117720924	0.70803638238973\\
0.82	0.0411358072996216	0.283865214418836	0.704565625567163\\
0.82	0.0426223833924861	0.288839227285554	0.701295023640316\\
0.82	0.0441367538930258	0.293811937711588	0.698208278749353\\
0.82	0.0456789619920503	0.298783126055387	0.69528959205535\\
0.82	0.0472490486990192	0.303752571668251	0.692523853197512\\
0.82	0.0488470528220538	0.308720052919643	0.689896776228119\\
0.82	0.0504730109482718	0.313685347222913	0.687394989990282\\
0.82	0.0521269574244539	0.318648231061428	0.685006090465981\\
0.82	0.0538089243380495	0.323608480015096	0.682718662029128\\
0.82	0.0555189414985328	0.328565868787292	0.680522273853469\\
0.82	0.0572570364191156	0.333520171232163	0.67840745699785\\
0.82	0.0590232342988274	0.338471160382329	0.676365666960043\\
0.82	0.0608175580049697	0.34341860847696	0.674389235783156\\
0.82	0.0626400280559541	0.348362286990222	0.672471317134862\\
0.82	0.064490662604533	0.3533019666601	0.670605827172182\\
0.82	0.0663694774214291	0.358237417517579	0.668787383459804\\
0.82	0.0682764858793752	0.363168408916186	0.667011243730729\\
0.82	0.0702116989375697	0.368094709561873	0.665273245863649\\
0.82	0.0721751251265572	0.373016087543259	0.663569750098933\\
0.82	0.0741667705335422	0.377932310362199	0.661897584220061\\
0.82	0.0761866387881432	0.382843144964686	0.660253992184441\\
0.82	0.0782347310485954	0.38774835777208	0.658636586490996\\
0.82	0.0803110459884098	0.392647714712651	0.657043304415872\\
0.82	0.0824155797834956	0.397540981253432	0.65547236812651\\
0.82	0.084548326099754	0.402427922432375	0.653922248592746\\
0.82	0.0867092760811502	0.4073083028908	0.65239163314657\\
0.82	0.0888984183382709	0.412181886906127	0.650879396495579\\
0.82	0.0911157389373742	0.417048438424897	0.649384574964843\\
0.82	0.0933612213899392	0.421907721096044	0.647906343724685\\
0.82	0.0956348466427212	0.42675949830445	0.646443996754806\\
0.82	0.0979365930683194	0.431603533204733	0.644996929295961\\
0.82	0.100266436456264	0.436439588755293	0.643564622546847\\
0.82	0.102624350004627	0.441267427752585	0.642146630374429\\
0.82	0.105010304312169	0.446086812865616	0.640742567819433\\
0.82	0.107424267371012	0.450897506670668	0.639352101193732\\
0.82	0.109866204559871	0.455699271686218	0.637974939582443\\
0.82	0.112336078637817	0.460491870408058	0.63661082757986\\
0.82	0.114833849738607	0.465275065344603	0.635259539104383\\
0.82	0.117359475365564	0.470048619052377	0.633920872153156\\
0.82	0.119912910387023	0.474812294171664	0.632594644371965\\
0.82	0.122494107032347	0.479565853462319	0.631280689329641\\
0.82	0.125103014888515	0.484309059839721	0.629978853399058\\
0.82	0.127739580897283	0.489041676410868	0.628690460824556\\
0.825	0	0	0.710202759607373\\
0.825	1.11327767495586e-05	0.00471862581271275	0.713003157837853\\
0.825	4.46175251031912e-05	0.009446325184051	0.715846486660142\\
0.825	0.000100583311362513	0.0141829653360114	0.718788838379794\\
0.825	0.000179158434668431	0.018928411755669	0.721886092228356\\
0.825	0.000280470402701511	0.02368252819604	0.725200636220892\\
0.825	0.000404645907256436	0.0284451766772965	0.728799085274723\\
0.825	0.000551810799695644	0.0332162174883389	0.732749123983336\\
0.825	0.000722090066287311	0.037995509188729	0.737115575253328\\
0.825	0.000915607803432999	0.0427829086109896	0.741955893907011\\
0.825	0.0011324871927904	0.0475782708632729	0.747315373652289\\
0.825	0.00137285047629673	0.0523814493324038	0.753222424045811\\
0.825	0.00163681893109844	0.057192295687301	0.759684310525652\\
0.825	0.00192451284439304	0.0620106598827802	0.766683748102665\\
0.825	0.00223605148818899	0.0668363901637434	0.774176695511256\\
0.825	0.00257155309398959	0.0716693330697584	0.782091614494494\\
0.825	0.00293113482740716	0.0765093334400312	0.79033034642109\\
0.825	0.00331491276271365	0.0813562344187764	0.79877062752665\\
0.825	0.00372300185733413	0.0862098774609879	0.80727012893036\\
0.825	0.00415551592628969	0.0910701023386145	0.815671782700193\\
0.825	0.00461256761659621	0.0959367471471425	0.823810053507331\\
0.825	0.00509426838162598	0.100809648312589	0.831517746503354\\
0.825	0.00560072845543873	0.105688640598912	0.838632911419451\\
0.825	0.00613205682708918	0.11057355711583	0.845005411376292\\
0.825	0.00668836121491816	0.115464229327074	0.850502769049301\\
0.825	0.00726974804083431	0.120360487059049	0.855014975717657\\
0.825	0.00787632240459385	0.125262158509929	0.858458041096605\\
0.825	0.00850818805808555	0.13016907025918	0.860776163494183\\
0.825	0.00916544737962849	0.135081047277508	0.861942500814224\\
0.825	0.00984820134829017	0.139997912937243	0.861958614648421\\
0.825	0.0105565495182326	0.144919489023162	0.860852735617091\\
0.825	0.0112905899930939	0.149845595743738	0.85867705410965\\
0.825	0.012050419400414	0.154776051742839	0.855504274967458\\
0.825	0.0128361328661109	0.159710674111862	0.851423687967756\\
0.825	0.0136478239890177	0.164649278402306	0.846537000484339\\
0.825	0.0144855848154856	0.169591678638796	0.840954157904249\\
0.825	0.0153495058140643	0.174537687332543	0.834789345403936\\
0.825	0.0162396758502651	0.179487115495259	0.828157325806201\\
0.825	0.0171561821614173	0.184439772653511	0.821170226448693\\
0.825	0.0180991103316243	0.189395466863524	0.813934846725692\\
0.825	0.0190685442668299	0.194354004726437	0.806550519904214\\
0.825	0.0200645661700016	0.199315191403999	0.799107529846305\\
0.825	0.0210872565164405	0.204278830634725	0.791686056496196\\
0.825	0.0221366940292259	0.20924472475049	0.784355603830602\\
0.825	0.023212955654804	0.214212674693577	0.777174850281116\\
0.825	0.0243161165387281	0.219182480034174	0.770191853862641\\
0.825	0.025446250001561	0.224153938988321	0.763444541554458\\
0.825	0.0266034275149466	0.229126848436298	0.756961413910484\\
0.825	0.0277877186778607	0.234101003941464	0.750762400416765\\
0.825	0.0289991911930496	0.239076199769547	0.744859807805784\\
0.825	0.0302379108436654	0.244052228908366	0.73925931151808\\
0.825	0.0315039414701067	0.24902888308801	0.733960949044503\\
0.825	0.0327973449470747	0.254005952801449	0.728960082407262\\
0.825	0.0341181811608519	0.258983227325592	0.724248305111177\\
0.825	0.0354665079868145	0.263960494742775	0.719814276220681\\
0.825	0.0368423812671862	0.268937541962694	0.715644470612631\\
0.825	0.0382458547890421	0.273914154744766	0.711723839838041\\
0.825	0.0396769802625738	0.278890117720924	0.70803638238973\\
0.825	0.0411358072996216	0.283865214418836	0.704565625567165\\
0.825	0.0426223833924861	0.288839227285554	0.701295023640316\\
0.825	0.0441367538930258	0.293811937711588	0.698208278749354\\
0.825	0.0456789619920503	0.298783126055387	0.695289592055349\\
0.825	0.0472490486990192	0.303752571668251	0.692523853197512\\
0.825	0.0488470528220538	0.308720052919643	0.689896776228118\\
0.825	0.0504730109482718	0.313685347222913	0.687394989990281\\
0.825	0.0521269574244539	0.318648231061428	0.685006090465981\\
0.825	0.0538089243380495	0.323608480015096	0.682718662029127\\
0.825	0.0555189414985328	0.328565868787292	0.680522273853468\\
0.825	0.0572570364191156	0.333520171232163	0.67840745699785\\
0.825	0.0590232342988274	0.338471160382329	0.676365666960045\\
0.825	0.0608175580049697	0.34341860847696	0.674389235783156\\
0.825	0.0626400280559541	0.348362286990222	0.672471317134862\\
0.825	0.064490662604533	0.3533019666601	0.670605827172183\\
0.825	0.066369477421429	0.358237417517579	0.668787383459803\\
0.825	0.0682764858793752	0.363168408916186	0.667011243730728\\
0.825	0.0702116989375697	0.368094709561873	0.665273245863649\\
0.825	0.0721751251265572	0.373016087543259	0.663569750098932\\
0.825	0.0741667705335423	0.377932310362199	0.661897584220061\\
0.825	0.0761866387881432	0.382843144964686	0.660253992184442\\
0.825	0.0782347310485954	0.38774835777208	0.658636586490996\\
0.825	0.0803110459884098	0.392647714712651	0.657043304415871\\
0.825	0.0824155797834956	0.397540981253432	0.655472368126509\\
0.825	0.084548326099754	0.402427922432375	0.653922248592745\\
0.825	0.0867092760811502	0.4073083028908	0.65239163314657\\
0.825	0.0888984183382709	0.412181886906127	0.650879396495579\\
0.825	0.0911157389373742	0.417048438424896	0.649384574964845\\
0.825	0.0933612213899392	0.421907721096044	0.647906343724684\\
0.825	0.0956348466427212	0.42675949830445	0.646443996754806\\
0.825	0.0979365930683194	0.431603533204733	0.644996929295962\\
0.825	0.100266436456264	0.436439588755293	0.643564622546847\\
0.825	0.102624350004627	0.441267427752585	0.642146630374431\\
0.825	0.105010304312169	0.446086812865616	0.640742567819433\\
0.825	0.107424267371012	0.450897506670668	0.639352101193731\\
0.825	0.109866204559871	0.455699271686218	0.637974939582443\\
0.825	0.112336078637817	0.460491870408058	0.636610827579862\\
0.825	0.114833849738607	0.465275065344603	0.635259539104382\\
0.825	0.117359475365564	0.470048619052377	0.633920872153155\\
0.825	0.119912910387023	0.474812294171664	0.632594644371965\\
0.825	0.122494107032347	0.479565853462319	0.631280689329641\\
0.825	0.125103014888515	0.484309059839721	0.62997885339906\\
0.825	0.127739580897283	0.489041676410868	0.628690460824555\\
0.83	0	0	0.710202759607373\\
0.83	1.11327767495586e-05	0.00471862581271274	0.713003157837853\\
0.83	4.46175251031912e-05	0.009446325184051	0.715846486660142\\
0.83	0.000100583311362513	0.0141829653360114	0.718788838379794\\
0.83	0.000179158434668431	0.018928411755669	0.721886092228356\\
0.83	0.000280470402701511	0.02368252819604	0.725200636220892\\
0.83	0.000404645907256436	0.0284451766772965	0.728799085274723\\
0.83	0.000551810799695644	0.0332162174883388	0.732749123983336\\
0.83	0.000722090066287311	0.037995509188729	0.737115575253327\\
0.83	0.000915607803432999	0.0427829086109896	0.741955893907011\\
0.83	0.0011324871927904	0.047578270863273	0.747315373652289\\
0.83	0.00137285047629673	0.0523814493324038	0.753222424045811\\
0.83	0.00163681893109844	0.057192295687301	0.759684310525652\\
0.83	0.00192451284439304	0.0620106598827802	0.766683748102665\\
0.83	0.00223605148818898	0.0668363901637434	0.774176695511256\\
0.83	0.00257155309398959	0.0716693330697584	0.782091614494493\\
0.83	0.00293113482740716	0.0765093334400312	0.79033034642109\\
0.83	0.00331491276271365	0.0813562344187764	0.79877062752665\\
0.83	0.00372300185733413	0.0862098774609879	0.807270128930359\\
0.83	0.00415551592628969	0.0910701023386145	0.815671782700193\\
0.83	0.00461256761659621	0.0959367471471425	0.82381005350733\\
0.83	0.00509426838162598	0.100809648312589	0.831517746503354\\
0.83	0.00560072845543873	0.105688640598912	0.838632911419452\\
0.83	0.00613205682708918	0.11057355711583	0.845005411376293\\
0.83	0.00668836121491815	0.115464229327074	0.850502769049302\\
0.83	0.00726974804083431	0.120360487059049	0.855014975717657\\
0.83	0.00787632240459385	0.125262158509929	0.858458041096607\\
0.83	0.00850818805808555	0.13016907025918	0.860776163494182\\
0.83	0.00916544737962849	0.135081047277508	0.861942500814224\\
0.83	0.00984820134829017	0.139997912937243	0.86195861464842\\
0.83	0.0105565495182326	0.144919489023162	0.86085273561709\\
0.83	0.0112905899930939	0.149845595743738	0.85867705410965\\
0.83	0.012050419400414	0.154776051742839	0.855504274967458\\
0.83	0.0128361328661109	0.159710674111862	0.851423687967751\\
0.83	0.0136478239890177	0.164649278402306	0.846537000484341\\
0.83	0.0144855848154856	0.169591678638796	0.840954157904249\\
0.83	0.0153495058140643	0.174537687332543	0.834789345403935\\
0.83	0.0162396758502651	0.179487115495259	0.828157325806201\\
0.83	0.0171561821614173	0.184439772653511	0.821170226448695\\
0.83	0.0180991103316243	0.189395466863524	0.813934846725693\\
0.83	0.0190685442668299	0.194354004726437	0.806550519904212\\
0.83	0.0200645661700016	0.199315191403999	0.799107529846303\\
0.83	0.0210872565164405	0.204278830634725	0.791686056496193\\
0.83	0.0221366940292259	0.20924472475049	0.7843556038306\\
0.83	0.023212955654804	0.214212674693577	0.777174850281117\\
0.83	0.0243161165387281	0.219182480034174	0.770191853862642\\
0.83	0.025446250001561	0.224153938988321	0.763444541554458\\
0.83	0.0266034275149466	0.229126848436298	0.756961413910483\\
0.83	0.0277877186778607	0.234101003941464	0.750762400416763\\
0.83	0.0289991911930496	0.239076199769547	0.744859807805786\\
0.83	0.0302379108436654	0.244052228908366	0.739259311518081\\
0.83	0.0315039414701067	0.24902888308801	0.733960949044503\\
0.83	0.0327973449470747	0.254005952801449	0.728960082407259\\
0.83	0.0341181811608519	0.258983227325591	0.724248305111177\\
0.83	0.0354665079868145	0.263960494742775	0.719814276220683\\
0.83	0.0368423812671862	0.268937541962694	0.715644470612631\\
0.83	0.0382458547890421	0.273914154744766	0.711723839838043\\
0.83	0.0396769802625738	0.278890117720924	0.708036382389731\\
0.83	0.0411358072996216	0.283865214418836	0.704565625567162\\
0.83	0.0426223833924861	0.288839227285554	0.701295023640315\\
0.83	0.0441367538930258	0.293811937711588	0.698208278749355\\
0.83	0.0456789619920503	0.298783126055387	0.69528959205535\\
0.83	0.0472490486990192	0.303752571668251	0.692523853197512\\
0.83	0.0488470528220538	0.308720052919643	0.689896776228118\\
0.83	0.0504730109482718	0.313685347222913	0.687394989990282\\
0.83	0.0521269574244539	0.318648231061428	0.685006090465981\\
0.83	0.0538089243380495	0.323608480015097	0.682718662029128\\
0.83	0.0555189414985328	0.328565868787292	0.680522273853469\\
0.83	0.0572570364191156	0.333520171232163	0.678407456997849\\
0.83	0.0590232342988274	0.338471160382329	0.676365666960043\\
0.83	0.0608175580049697	0.34341860847696	0.674389235783156\\
0.83	0.0626400280559541	0.348362286990222	0.672471317134862\\
0.83	0.064490662604533	0.3533019666601	0.670605827172182\\
0.83	0.0663694774214291	0.358237417517579	0.668787383459804\\
0.83	0.0682764858793752	0.363168408916186	0.667011243730728\\
0.83	0.0702116989375697	0.368094709561873	0.66527324586365\\
0.83	0.0721751251265572	0.373016087543259	0.663569750098932\\
0.83	0.0741667705335423	0.377932310362199	0.661897584220061\\
0.83	0.0761866387881432	0.382843144964686	0.660253992184441\\
0.83	0.0782347310485954	0.38774835777208	0.658636586490997\\
0.83	0.0803110459884098	0.392647714712651	0.657043304415872\\
0.83	0.0824155797834956	0.397540981253432	0.65547236812651\\
0.83	0.084548326099754	0.402427922432375	0.653922248592744\\
0.83	0.0867092760811502	0.407308302890799	0.652391633146569\\
0.83	0.0888984183382709	0.412181886906127	0.650879396495579\\
0.83	0.0911157389373742	0.417048438424896	0.649384574964844\\
0.83	0.0933612213899392	0.421907721096044	0.647906343724685\\
0.83	0.0956348466427212	0.42675949830445	0.646443996754804\\
0.83	0.0979365930683194	0.431603533204733	0.644996929295961\\
0.83	0.100266436456264	0.436439588755293	0.643564622546847\\
0.83	0.102624350004627	0.441267427752584	0.64214663037443\\
0.83	0.105010304312169	0.446086812865616	0.640742567819434\\
0.83	0.107424267371012	0.450897506670668	0.639352101193731\\
0.83	0.109866204559871	0.455699271686218	0.637974939582442\\
0.83	0.112336078637817	0.460491870408058	0.636610827579861\\
0.83	0.114833849738607	0.465275065344603	0.635259539104384\\
0.83	0.117359475365564	0.470048619052377	0.633920872153155\\
0.83	0.119912910387023	0.474812294171664	0.632594644371964\\
0.83	0.122494107032347	0.479565853462319	0.63128068932964\\
0.83	0.125103014888515	0.484309059839721	0.629978853399059\\
0.83	0.127739580897283	0.489041676410868	0.62869046082456\\
0.835	0	0	0.710202759607373\\
0.835	1.11327767495586e-05	0.00471862581271274	0.713003157837853\\
0.835	4.46175251031912e-05	0.009446325184051	0.715846486660142\\
0.835	0.000100583311362513	0.0141829653360114	0.718788838379794\\
0.835	0.000179158434668431	0.018928411755669	0.721886092228357\\
0.835	0.000280470402701511	0.02368252819604	0.725200636220892\\
0.835	0.000404645907256436	0.0284451766772965	0.728799085274723\\
0.835	0.000551810799695644	0.0332162174883389	0.732749123983336\\
0.835	0.000722090066287311	0.037995509188729	0.737115575253328\\
0.835	0.000915607803432999	0.0427829086109896	0.741955893907011\\
0.835	0.0011324871927904	0.0475782708632729	0.747315373652289\\
0.835	0.00137285047629673	0.0523814493324038	0.75322242404581\\
0.835	0.00163681893109843	0.057192295687301	0.759684310525652\\
0.835	0.00192451284439304	0.0620106598827802	0.766683748102665\\
0.835	0.00223605148818899	0.0668363901637434	0.774176695511256\\
0.835	0.00257155309398959	0.0716693330697584	0.782091614494493\\
0.835	0.00293113482740716	0.0765093334400312	0.79033034642109\\
0.835	0.00331491276271365	0.0813562344187764	0.79877062752665\\
0.835	0.00372300185733414	0.0862098774609879	0.807270128930359\\
0.835	0.00415551592628969	0.0910701023386145	0.815671782700193\\
0.835	0.00461256761659621	0.0959367471471425	0.82381005350733\\
0.835	0.00509426838162598	0.100809648312589	0.831517746503354\\
0.835	0.00560072845543873	0.105688640598912	0.838632911419452\\
0.835	0.00613205682708918	0.11057355711583	0.845005411376292\\
0.835	0.00668836121491815	0.115464229327074	0.850502769049302\\
0.835	0.00726974804083431	0.120360487059049	0.855014975717657\\
0.835	0.00787632240459385	0.125262158509929	0.858458041096606\\
0.835	0.00850818805808554	0.13016907025918	0.860776163494182\\
0.835	0.00916544737962849	0.135081047277508	0.861942500814225\\
0.835	0.00984820134829017	0.139997912937243	0.86195861464842\\
0.835	0.0105565495182326	0.144919489023162	0.860852735617091\\
0.835	0.0112905899930939	0.149845595743738	0.858677054109652\\
0.835	0.012050419400414	0.154776051742839	0.855504274967457\\
0.835	0.0128361328661109	0.159710674111862	0.851423687967753\\
0.835	0.0136478239890177	0.164649278402306	0.846537000484342\\
0.835	0.0144855848154856	0.169591678638796	0.840954157904248\\
0.835	0.0153495058140643	0.174537687332543	0.834789345403936\\
0.835	0.0162396758502651	0.179487115495259	0.8281573258062\\
0.835	0.0171561821614173	0.184439772653511	0.821170226448694\\
0.835	0.0180991103316243	0.189395466863524	0.813934846725692\\
0.835	0.0190685442668299	0.194354004726437	0.806550519904211\\
0.835	0.0200645661700016	0.199315191403999	0.799107529846305\\
0.835	0.0210872565164405	0.204278830634725	0.791686056496193\\
0.835	0.0221366940292259	0.20924472475049	0.784355603830601\\
0.835	0.023212955654804	0.214212674693577	0.777174850281117\\
0.835	0.0243161165387281	0.219182480034174	0.770191853862644\\
0.835	0.025446250001561	0.224153938988321	0.763444541554459\\
0.835	0.0266034275149466	0.229126848436298	0.756961413910483\\
0.835	0.0277877186778607	0.234101003941464	0.750762400416764\\
0.835	0.0289991911930496	0.239076199769547	0.744859807805785\\
0.835	0.0302379108436654	0.244052228908366	0.73925931151808\\
0.835	0.0315039414701067	0.24902888308801	0.733960949044504\\
0.835	0.0327973449470747	0.254005952801449	0.72896008240726\\
0.835	0.0341181811608519	0.258983227325592	0.724248305111179\\
0.835	0.0354665079868145	0.263960494742775	0.719814276220682\\
0.835	0.0368423812671862	0.268937541962694	0.71564447061263\\
0.835	0.0382458547890421	0.273914154744766	0.711723839838042\\
0.835	0.0396769802625737	0.278890117720924	0.708036382389729\\
0.835	0.0411358072996216	0.283865214418836	0.704565625567163\\
0.835	0.0426223833924861	0.288839227285554	0.701295023640316\\
0.835	0.0441367538930258	0.293811937711588	0.698208278749354\\
0.835	0.0456789619920503	0.298783126055387	0.695289592055349\\
0.835	0.0472490486990192	0.303752571668251	0.692523853197511\\
0.835	0.0488470528220537	0.308720052919643	0.689896776228118\\
0.835	0.0504730109482718	0.313685347222913	0.687394989990282\\
0.835	0.0521269574244539	0.318648231061428	0.685006090465982\\
0.835	0.0538089243380495	0.323608480015096	0.682718662029128\\
0.835	0.0555189414985328	0.328565868787292	0.68052227385347\\
0.835	0.0572570364191156	0.333520171232163	0.67840745699785\\
0.835	0.0590232342988274	0.338471160382329	0.676365666960044\\
0.835	0.0608175580049697	0.34341860847696	0.674389235783156\\
0.835	0.0626400280559541	0.348362286990222	0.672471317134862\\
0.835	0.064490662604533	0.3533019666601	0.670605827172181\\
0.835	0.0663694774214291	0.358237417517579	0.668787383459804\\
0.835	0.0682764858793752	0.363168408916186	0.667011243730728\\
0.835	0.0702116989375697	0.368094709561873	0.665273245863649\\
0.835	0.0721751251265572	0.373016087543259	0.663569750098932\\
0.835	0.0741667705335422	0.377932310362199	0.661897584220061\\
0.835	0.0761866387881432	0.382843144964686	0.660253992184444\\
0.835	0.0782347310485954	0.38774835777208	0.658636586490997\\
0.835	0.0803110459884098	0.392647714712651	0.65704330441587\\
0.835	0.0824155797834956	0.397540981253432	0.655472368126509\\
0.835	0.084548326099754	0.402427922432375	0.653922248592747\\
0.835	0.0867092760811502	0.4073083028908	0.652391633146569\\
0.835	0.0888984183382709	0.412181886906127	0.65087939649558\\
0.835	0.0911157389373742	0.417048438424897	0.649384574964844\\
0.835	0.0933612213899393	0.421907721096044	0.647906343724684\\
0.835	0.0956348466427212	0.42675949830445	0.646443996754806\\
0.835	0.0979365930683194	0.431603533204733	0.64499692929596\\
0.835	0.100266436456264	0.436439588755293	0.643564622546846\\
0.835	0.102624350004627	0.441267427752584	0.64214663037443\\
0.835	0.105010304312169	0.446086812865616	0.640742567819433\\
0.835	0.107424267371012	0.450897506670668	0.639352101193732\\
0.835	0.109866204559871	0.455699271686218	0.637974939582443\\
0.835	0.112336078637817	0.460491870408058	0.63661082757986\\
0.835	0.114833849738607	0.465275065344603	0.635259539104383\\
0.835	0.117359475365564	0.470048619052377	0.633920872153156\\
0.835	0.119912910387022	0.474812294171664	0.632594644371964\\
0.835	0.122494107032347	0.479565853462319	0.631280689329641\\
0.835	0.125103014888515	0.484309059839721	0.629978853399059\\
0.835	0.127739580897283	0.489041676410868	0.628690460824561\\
0.84	0	0	0.710202759607373\\
0.84	1.11327767495586e-05	0.00471862581271275	0.713003157837853\\
0.84	4.46175251031912e-05	0.009446325184051	0.715846486660142\\
0.84	0.000100583311362513	0.0141829653360114	0.718788838379794\\
0.84	0.000179158434668431	0.018928411755669	0.721886092228356\\
0.84	0.000280470402701511	0.02368252819604	0.725200636220892\\
0.84	0.000404645907256436	0.0284451766772965	0.728799085274723\\
0.84	0.000551810799695644	0.0332162174883388	0.732749123983336\\
0.84	0.000722090066287311	0.037995509188729	0.737115575253327\\
0.84	0.000915607803432999	0.0427829086109896	0.741955893907011\\
0.84	0.0011324871927904	0.047578270863273	0.747315373652289\\
0.84	0.00137285047629673	0.0523814493324038	0.753222424045811\\
0.84	0.00163681893109844	0.057192295687301	0.759684310525652\\
0.84	0.00192451284439304	0.0620106598827802	0.766683748102665\\
0.84	0.00223605148818899	0.0668363901637434	0.774176695511256\\
0.84	0.00257155309398959	0.0716693330697584	0.782091614494493\\
0.84	0.00293113482740716	0.0765093334400312	0.79033034642109\\
0.84	0.00331491276271365	0.0813562344187764	0.79877062752665\\
0.84	0.00372300185733414	0.0862098774609879	0.80727012893036\\
0.84	0.00415551592628969	0.0910701023386145	0.815671782700194\\
0.84	0.00461256761659621	0.0959367471471425	0.82381005350733\\
0.84	0.00509426838162598	0.100809648312589	0.831517746503354\\
0.84	0.00560072845543873	0.105688640598912	0.838632911419453\\
0.84	0.00613205682708918	0.11057355711583	0.845005411376293\\
0.84	0.00668836121491816	0.115464229327074	0.8505027690493\\
0.84	0.00726974804083431	0.120360487059049	0.855014975717655\\
0.84	0.00787632240459385	0.125262158509929	0.858458041096605\\
0.84	0.00850818805808555	0.13016907025918	0.860776163494183\\
0.84	0.00916544737962849	0.135081047277508	0.861942500814226\\
0.84	0.00984820134829017	0.139997912937243	0.86195861464842\\
0.84	0.0105565495182326	0.144919489023162	0.860852735617092\\
0.84	0.0112905899930939	0.149845595743738	0.858677054109652\\
0.84	0.012050419400414	0.154776051742839	0.855504274967456\\
0.84	0.0128361328661109	0.159710674111862	0.851423687967757\\
0.84	0.0136478239890177	0.164649278402306	0.846537000484339\\
0.84	0.0144855848154856	0.169591678638796	0.840954157904248\\
0.84	0.0153495058140643	0.174537687332543	0.834789345403934\\
0.84	0.0162396758502651	0.179487115495259	0.8281573258062\\
0.84	0.0171561821614173	0.184439772653511	0.821170226448695\\
0.84	0.0180991103316243	0.189395466863524	0.813934846725693\\
0.84	0.0190685442668299	0.194354004726437	0.806550519904212\\
0.84	0.0200645661700016	0.199315191403999	0.799107529846306\\
0.84	0.0210872565164405	0.204278830634725	0.791686056496193\\
0.84	0.0221366940292259	0.20924472475049	0.784355603830601\\
0.84	0.023212955654804	0.214212674693577	0.777174850281119\\
0.84	0.0243161165387281	0.219182480034174	0.770191853862645\\
0.84	0.025446250001561	0.224153938988321	0.763444541554459\\
0.84	0.0266034275149466	0.229126848436298	0.756961413910482\\
0.84	0.0277877186778607	0.234101003941464	0.750762400416764\\
0.84	0.0289991911930496	0.239076199769547	0.744859807805785\\
0.84	0.0302379108436654	0.244052228908366	0.73925931151808\\
0.84	0.0315039414701067	0.24902888308801	0.733960949044503\\
0.84	0.0327973449470747	0.254005952801449	0.728960082407262\\
0.84	0.0341181811608519	0.258983227325592	0.724248305111178\\
0.84	0.0354665079868145	0.263960494742775	0.71981427622068\\
0.84	0.0368423812671862	0.268937541962694	0.715644470612629\\
0.84	0.0382458547890421	0.273914154744766	0.711723839838041\\
0.84	0.0396769802625738	0.278890117720924	0.708036382389731\\
0.84	0.0411358072996216	0.283865214418836	0.704565625567164\\
0.84	0.0426223833924861	0.288839227285554	0.701295023640316\\
0.84	0.0441367538930258	0.293811937711588	0.698208278749354\\
0.84	0.0456789619920503	0.298783126055387	0.695289592055349\\
0.84	0.0472490486990192	0.303752571668251	0.692523853197512\\
0.84	0.0488470528220537	0.308720052919643	0.689896776228119\\
0.84	0.0504730109482718	0.313685347222913	0.687394989990281\\
0.84	0.0521269574244539	0.318648231061428	0.685006090465981\\
0.84	0.0538089243380495	0.323608480015097	0.682718662029127\\
0.84	0.0555189414985328	0.328565868787292	0.680522273853469\\
0.84	0.0572570364191156	0.333520171232163	0.67840745699785\\
0.84	0.0590232342988274	0.338471160382329	0.676365666960044\\
0.84	0.0608175580049697	0.34341860847696	0.674389235783157\\
0.84	0.0626400280559541	0.348362286990222	0.672471317134862\\
0.84	0.064490662604533	0.3533019666601	0.670605827172182\\
0.84	0.0663694774214291	0.358237417517579	0.668787383459802\\
0.84	0.0682764858793752	0.363168408916186	0.667011243730727\\
0.84	0.0702116989375697	0.368094709561873	0.665273245863649\\
0.84	0.0721751251265572	0.373016087543259	0.663569750098933\\
0.84	0.0741667705335423	0.377932310362199	0.66189758422006\\
0.84	0.0761866387881432	0.382843144964686	0.660253992184442\\
0.84	0.0782347310485954	0.38774835777208	0.658636586490997\\
0.84	0.0803110459884098	0.392647714712651	0.657043304415872\\
0.84	0.0824155797834956	0.397540981253432	0.655472368126509\\
0.84	0.084548326099754	0.402427922432375	0.653922248592746\\
0.84	0.0867092760811502	0.4073083028908	0.65239163314657\\
0.84	0.0888984183382709	0.412181886906127	0.650879396495578\\
0.84	0.0911157389373742	0.417048438424896	0.649384574964846\\
0.84	0.0933612213899392	0.421907721096044	0.647906343724685\\
0.84	0.0956348466427212	0.42675949830445	0.646443996754806\\
0.84	0.0979365930683194	0.431603533204733	0.644996929295961\\
0.84	0.100266436456264	0.436439588755293	0.643564622546845\\
0.84	0.102624350004627	0.441267427752584	0.64214663037443\\
0.84	0.105010304312169	0.446086812865616	0.640742567819432\\
0.84	0.107424267371012	0.450897506670668	0.639352101193731\\
0.84	0.109866204559871	0.455699271686218	0.637974939582444\\
0.84	0.112336078637817	0.460491870408058	0.636610827579862\\
0.84	0.114833849738607	0.465275065344603	0.635259539104382\\
0.84	0.117359475365564	0.470048619052377	0.633920872153155\\
0.84	0.119912910387022	0.474812294171664	0.632594644371966\\
0.84	0.122494107032347	0.479565853462319	0.63128068932964\\
0.84	0.125103014888515	0.484309059839721	0.629978853399057\\
0.84	0.127739580897283	0.489041676410868	0.62869046082456\\
0.845	0	0	0.710202759607373\\
0.845	1.11327767495586e-05	0.00471862581271275	0.713003157837853\\
0.845	4.46175251031912e-05	0.009446325184051	0.715846486660142\\
0.845	0.000100583311362513	0.0141829653360114	0.718788838379794\\
0.845	0.000179158434668431	0.018928411755669	0.721886092228356\\
0.845	0.000280470402701511	0.02368252819604	0.725200636220892\\
0.845	0.000404645907256436	0.0284451766772965	0.728799085274723\\
0.845	0.000551810799695644	0.0332162174883388	0.732749123983336\\
0.845	0.000722090066287311	0.037995509188729	0.737115575253327\\
0.845	0.000915607803432999	0.0427829086109896	0.741955893907011\\
0.845	0.0011324871927904	0.0475782708632729	0.747315373652289\\
0.845	0.00137285047629673	0.0523814493324038	0.753222424045811\\
0.845	0.00163681893109844	0.057192295687301	0.759684310525652\\
0.845	0.00192451284439304	0.0620106598827802	0.766683748102665\\
0.845	0.00223605148818898	0.0668363901637434	0.774176695511256\\
0.845	0.00257155309398959	0.0716693330697584	0.782091614494494\\
0.845	0.00293113482740716	0.0765093334400312	0.79033034642109\\
0.845	0.00331491276271365	0.0813562344187764	0.79877062752665\\
0.845	0.00372300185733414	0.0862098774609879	0.807270128930359\\
0.845	0.00415551592628969	0.0910701023386146	0.815671782700194\\
0.845	0.00461256761659621	0.0959367471471424	0.82381005350733\\
0.845	0.00509426838162598	0.100809648312589	0.831517746503354\\
0.845	0.00560072845543873	0.105688640598912	0.838632911419451\\
0.845	0.00613205682708918	0.11057355711583	0.845005411376293\\
0.845	0.00668836121491816	0.115464229327074	0.850502769049302\\
0.845	0.00726974804083431	0.120360487059049	0.855014975717657\\
0.845	0.00787632240459385	0.125262158509929	0.858458041096605\\
0.845	0.00850818805808554	0.13016907025918	0.860776163494181\\
0.845	0.00916544737962849	0.135081047277508	0.861942500814223\\
0.845	0.00984820134829017	0.139997912937243	0.861958614648421\\
0.845	0.0105565495182326	0.144919489023162	0.86085273561709\\
0.845	0.0112905899930939	0.149845595743738	0.858677054109649\\
0.845	0.012050419400414	0.154776051742839	0.855504274967456\\
0.845	0.0128361328661109	0.159710674111862	0.851423687967754\\
0.845	0.0136478239890177	0.164649278402306	0.846537000484339\\
0.845	0.0144855848154856	0.169591678638796	0.84095415790425\\
0.845	0.0153495058140643	0.174537687332543	0.834789345403935\\
0.845	0.0162396758502651	0.179487115495259	0.828157325806202\\
0.845	0.0171561821614173	0.184439772653511	0.821170226448694\\
0.845	0.0180991103316243	0.189395466863524	0.813934846725691\\
0.845	0.0190685442668299	0.194354004726437	0.806550519904212\\
0.845	0.0200645661700016	0.199315191403999	0.799107529846304\\
0.845	0.0210872565164405	0.204278830634725	0.791686056496194\\
0.845	0.0221366940292259	0.20924472475049	0.784355603830603\\
0.845	0.023212955654804	0.214212674693577	0.777174850281119\\
0.845	0.0243161165387281	0.219182480034174	0.770191853862644\\
0.845	0.025446250001561	0.224153938988321	0.763444541554456\\
0.845	0.0266034275149466	0.229126848436298	0.756961413910483\\
0.845	0.0277877186778607	0.234101003941464	0.750762400416765\\
0.845	0.0289991911930496	0.239076199769547	0.744859807805785\\
0.845	0.0302379108436654	0.244052228908366	0.739259311518078\\
0.845	0.0315039414701067	0.24902888308801	0.733960949044504\\
0.845	0.0327973449470747	0.254005952801449	0.728960082407262\\
0.845	0.0341181811608519	0.258983227325592	0.724248305111178\\
0.845	0.0354665079868145	0.263960494742775	0.719814276220681\\
0.845	0.0368423812671862	0.268937541962694	0.715644470612628\\
0.845	0.0382458547890421	0.273914154744766	0.711723839838042\\
0.845	0.0396769802625738	0.278890117720924	0.70803638238973\\
0.845	0.0411358072996216	0.283865214418836	0.704565625567163\\
0.845	0.0426223833924861	0.288839227285554	0.701295023640316\\
0.845	0.0441367538930258	0.293811937711588	0.698208278749355\\
0.845	0.0456789619920503	0.298783126055387	0.695289592055348\\
0.845	0.0472490486990192	0.303752571668251	0.692523853197512\\
0.845	0.0488470528220538	0.308720052919643	0.689896776228119\\
0.845	0.0504730109482718	0.313685347222913	0.687394989990281\\
0.845	0.0521269574244539	0.318648231061428	0.685006090465981\\
0.845	0.0538089243380495	0.323608480015097	0.682718662029127\\
0.845	0.0555189414985328	0.328565868787292	0.680522273853469\\
0.845	0.0572570364191156	0.333520171232163	0.67840745699785\\
0.845	0.0590232342988274	0.338471160382329	0.676365666960043\\
0.845	0.0608175580049697	0.34341860847696	0.674389235783157\\
0.845	0.0626400280559541	0.348362286990222	0.672471317134862\\
0.845	0.064490662604533	0.3533019666601	0.670605827172182\\
0.845	0.0663694774214291	0.358237417517579	0.668787383459805\\
0.845	0.0682764858793752	0.363168408916186	0.667011243730728\\
0.845	0.0702116989375697	0.368094709561873	0.665273245863649\\
0.845	0.0721751251265572	0.373016087543259	0.663569750098932\\
0.845	0.0741667705335422	0.377932310362199	0.66189758422006\\
0.845	0.0761866387881432	0.382843144964686	0.660253992184442\\
0.845	0.0782347310485954	0.38774835777208	0.658636586490997\\
0.845	0.0803110459884098	0.392647714712651	0.657043304415871\\
0.845	0.0824155797834956	0.397540981253432	0.65547236812651\\
0.845	0.084548326099754	0.402427922432375	0.653922248592745\\
0.845	0.0867092760811502	0.407308302890799	0.652391633146569\\
0.845	0.0888984183382709	0.412181886906127	0.650879396495579\\
0.845	0.0911157389373742	0.417048438424896	0.649384574964843\\
0.845	0.0933612213899393	0.421907721096044	0.647906343724686\\
0.845	0.0956348466427212	0.42675949830445	0.646443996754806\\
0.845	0.0979365930683194	0.431603533204733	0.644996929295962\\
0.845	0.100266436456264	0.436439588755293	0.643564622546847\\
0.845	0.102624350004627	0.441267427752585	0.64214663037443\\
0.845	0.105010304312169	0.446086812865616	0.640742567819433\\
0.845	0.107424267371012	0.450897506670668	0.63935210119373\\
0.845	0.109866204559871	0.455699271686218	0.637974939582443\\
0.845	0.112336078637817	0.460491870408058	0.636610827579862\\
0.845	0.114833849738607	0.465275065344603	0.635259539104382\\
0.845	0.117359475365564	0.470048619052377	0.633920872153156\\
0.845	0.119912910387022	0.474812294171664	0.632594644371966\\
0.845	0.122494107032347	0.479565853462319	0.63128068932964\\
0.845	0.125103014888515	0.484309059839721	0.629978853399058\\
0.845	0.127739580897283	0.489041676410868	0.62869046082456\\
0.85	0	0	0.710202759607373\\
0.85	1.11327767495586e-05	0.00471862581271275	0.713003157837853\\
0.85	4.46175251031912e-05	0.009446325184051	0.715846486660142\\
0.85	0.000100583311362513	0.0141829653360114	0.718788838379794\\
0.85	0.000179158434668431	0.018928411755669	0.721886092228356\\
0.85	0.000280470402701511	0.02368252819604	0.725200636220892\\
0.85	0.000404645907256436	0.0284451766772965	0.728799085274723\\
0.85	0.000551810799695644	0.0332162174883389	0.732749123983336\\
0.85	0.000722090066287311	0.037995509188729	0.737115575253328\\
0.85	0.000915607803432999	0.0427829086109896	0.741955893907011\\
0.85	0.0011324871927904	0.047578270863273	0.747315373652289\\
0.85	0.00137285047629673	0.0523814493324038	0.753222424045811\\
0.85	0.00163681893109844	0.057192295687301	0.759684310525652\\
0.85	0.00192451284439304	0.0620106598827802	0.766683748102665\\
0.85	0.00223605148818898	0.0668363901637434	0.774176695511257\\
0.85	0.00257155309398959	0.0716693330697584	0.782091614494493\\
0.85	0.00293113482740716	0.0765093334400312	0.79033034642109\\
0.85	0.00331491276271365	0.0813562344187764	0.79877062752665\\
0.85	0.00372300185733413	0.0862098774609879	0.807270128930359\\
0.85	0.00415551592628969	0.0910701023386145	0.815671782700193\\
0.85	0.00461256761659621	0.0959367471471424	0.82381005350733\\
0.85	0.00509426838162598	0.100809648312589	0.831517746503354\\
0.85	0.00560072845543873	0.105688640598912	0.838632911419452\\
0.85	0.00613205682708918	0.11057355711583	0.845005411376292\\
0.85	0.00668836121491816	0.115464229327074	0.850502769049303\\
0.85	0.00726974804083431	0.120360487059049	0.855014975717659\\
0.85	0.00787632240459385	0.125262158509929	0.858458041096607\\
0.85	0.00850818805808554	0.13016907025918	0.860776163494182\\
0.85	0.00916544737962849	0.135081047277508	0.861942500814224\\
0.85	0.00984820134829017	0.139997912937243	0.86195861464842\\
0.85	0.0105565495182326	0.144919489023162	0.860852735617089\\
0.85	0.0112905899930939	0.149845595743738	0.858677054109649\\
0.85	0.012050419400414	0.154776051742839	0.855504274967458\\
0.85	0.0128361328661109	0.159710674111862	0.851423687967754\\
0.85	0.0136478239890177	0.164649278402306	0.846537000484341\\
0.85	0.0144855848154856	0.169591678638796	0.840954157904249\\
0.85	0.0153495058140643	0.174537687332543	0.834789345403935\\
0.85	0.0162396758502651	0.179487115495259	0.8281573258062\\
0.85	0.0171561821614173	0.184439772653511	0.821170226448692\\
0.85	0.0180991103316243	0.189395466863524	0.813934846725691\\
0.85	0.0190685442668299	0.194354004726437	0.806550519904213\\
0.85	0.0200645661700016	0.199315191403999	0.799107529846306\\
0.85	0.0210872565164405	0.204278830634725	0.791686056496194\\
0.85	0.0221366940292259	0.20924472475049	0.784355603830602\\
0.85	0.023212955654804	0.214212674693577	0.777174850281119\\
0.85	0.0243161165387281	0.219182480034174	0.770191853862642\\
0.85	0.025446250001561	0.224153938988321	0.763444541554458\\
0.85	0.0266034275149466	0.229126848436298	0.756961413910484\\
0.85	0.0277877186778607	0.234101003941464	0.750762400416764\\
0.85	0.0289991911930496	0.239076199769547	0.744859807805784\\
0.85	0.0302379108436654	0.244052228908366	0.739259311518081\\
0.85	0.0315039414701067	0.24902888308801	0.733960949044504\\
0.85	0.0327973449470747	0.254005952801449	0.728960082407261\\
0.85	0.0341181811608519	0.258983227325592	0.72424830511118\\
0.85	0.0354665079868145	0.263960494742775	0.71981427622068\\
0.85	0.0368423812671862	0.268937541962694	0.715644470612629\\
0.85	0.0382458547890421	0.273914154744766	0.711723839838043\\
0.85	0.0396769802625738	0.278890117720924	0.70803638238973\\
0.85	0.0411358072996216	0.283865214418836	0.704565625567164\\
0.85	0.0426223833924861	0.288839227285554	0.701295023640316\\
0.85	0.0441367538930258	0.293811937711588	0.698208278749354\\
0.85	0.0456789619920503	0.298783126055387	0.695289592055349\\
0.85	0.0472490486990192	0.303752571668251	0.692523853197512\\
0.85	0.0488470528220538	0.308720052919643	0.689896776228119\\
0.85	0.0504730109482718	0.313685347222913	0.687394989990281\\
0.85	0.0521269574244539	0.318648231061428	0.68500609046598\\
0.85	0.0538089243380495	0.323608480015096	0.682718662029128\\
0.85	0.0555189414985328	0.328565868787292	0.68052227385347\\
0.85	0.0572570364191156	0.333520171232163	0.67840745699785\\
0.85	0.0590232342988274	0.338471160382329	0.676365666960042\\
0.85	0.0608175580049697	0.34341860847696	0.674389235783157\\
0.85	0.0626400280559541	0.348362286990222	0.672471317134863\\
0.85	0.064490662604533	0.3533019666601	0.670605827172181\\
0.85	0.0663694774214291	0.358237417517579	0.668787383459803\\
0.85	0.0682764858793752	0.363168408916186	0.667011243730728\\
0.85	0.0702116989375697	0.368094709561873	0.66527324586365\\
0.85	0.0721751251265572	0.373016087543259	0.663569750098933\\
0.85	0.0741667705335422	0.377932310362199	0.66189758422006\\
0.85	0.0761866387881432	0.382843144964686	0.660253992184442\\
0.85	0.0782347310485954	0.38774835777208	0.658636586490996\\
0.85	0.0803110459884098	0.392647714712651	0.657043304415871\\
0.85	0.0824155797834956	0.397540981253432	0.655472368126509\\
0.85	0.084548326099754	0.402427922432375	0.653922248592746\\
0.85	0.0867092760811502	0.4073083028908	0.65239163314657\\
0.85	0.0888984183382709	0.412181886906127	0.650879396495578\\
0.85	0.0911157389373742	0.417048438424896	0.649384574964844\\
0.85	0.0933612213899392	0.421907721096044	0.647906343724684\\
0.85	0.0956348466427212	0.42675949830445	0.646443996754806\\
0.85	0.0979365930683194	0.431603533204733	0.644996929295962\\
0.85	0.100266436456264	0.436439588755293	0.643564622546847\\
0.85	0.102624350004627	0.441267427752584	0.64214663037443\\
0.85	0.105010304312169	0.446086812865616	0.640742567819433\\
0.85	0.107424267371012	0.450897506670668	0.639352101193731\\
0.85	0.109866204559871	0.455699271686218	0.637974939582443\\
0.85	0.112336078637817	0.460491870408058	0.636610827579862\\
0.85	0.114833849738607	0.465275065344603	0.635259539104382\\
0.85	0.117359475365564	0.470048619052377	0.633920872153155\\
0.85	0.119912910387023	0.474812294171664	0.632594644371967\\
0.85	0.122494107032347	0.479565853462319	0.631280689329641\\
0.85	0.125103014888515	0.484309059839721	0.629978853399059\\
0.85	0.127739580897283	0.489041676410868	0.628690460824562\\
0.855	0	0	0.710202759607373\\
0.855	1.11327767495586e-05	0.00471862581271274	0.713003157837853\\
0.855	4.46175251031912e-05	0.009446325184051	0.715846486660142\\
0.855	0.000100583311362513	0.0141829653360114	0.718788838379794\\
0.855	0.000179158434668431	0.018928411755669	0.721886092228356\\
0.855	0.000280470402701511	0.02368252819604	0.725200636220892\\
0.855	0.000404645907256436	0.0284451766772965	0.728799085274723\\
0.855	0.000551810799695644	0.0332162174883388	0.732749123983336\\
0.855	0.000722090066287311	0.037995509188729	0.737115575253327\\
0.855	0.000915607803432999	0.0427829086109896	0.741955893907011\\
0.855	0.0011324871927904	0.0475782708632729	0.747315373652289\\
0.855	0.00137285047629673	0.0523814493324038	0.753222424045811\\
0.855	0.00163681893109844	0.057192295687301	0.759684310525652\\
0.855	0.00192451284439304	0.0620106598827802	0.766683748102665\\
0.855	0.00223605148818899	0.0668363901637434	0.774176695511256\\
0.855	0.00257155309398959	0.0716693330697584	0.782091614494494\\
0.855	0.00293113482740716	0.0765093334400312	0.79033034642109\\
0.855	0.00331491276271365	0.0813562344187764	0.79877062752665\\
0.855	0.00372300185733414	0.0862098774609879	0.807270128930359\\
0.855	0.00415551592628969	0.0910701023386145	0.815671782700193\\
0.855	0.00461256761659621	0.0959367471471424	0.823810053507331\\
0.855	0.00509426838162598	0.100809648312589	0.831517746503353\\
0.855	0.00560072845543873	0.105688640598912	0.838632911419452\\
0.855	0.00613205682708918	0.11057355711583	0.845005411376293\\
0.855	0.00668836121491816	0.115464229327074	0.850502769049301\\
0.855	0.00726974804083431	0.120360487059049	0.855014975717657\\
0.855	0.00787632240459385	0.125262158509929	0.858458041096606\\
0.855	0.00850818805808554	0.13016907025918	0.860776163494183\\
0.855	0.00916544737962849	0.135081047277508	0.861942500814224\\
0.855	0.00984820134829017	0.139997912937243	0.861958614648419\\
0.855	0.0105565495182326	0.144919489023162	0.860852735617092\\
0.855	0.0112905899930939	0.149845595743738	0.858677054109652\\
0.855	0.012050419400414	0.154776051742839	0.855504274967458\\
0.855	0.0128361328661109	0.159710674111862	0.851423687967753\\
0.855	0.0136478239890177	0.164649278402306	0.846537000484341\\
0.855	0.0144855848154856	0.169591678638796	0.840954157904246\\
0.855	0.0153495058140643	0.174537687332543	0.834789345403935\\
0.855	0.0162396758502651	0.179487115495259	0.828157325806199\\
0.855	0.0171561821614173	0.184439772653511	0.821170226448694\\
0.855	0.0180991103316243	0.189395466863524	0.813934846725693\\
0.855	0.0190685442668299	0.194354004726437	0.806550519904213\\
0.855	0.0200645661700016	0.199315191403999	0.799107529846304\\
0.855	0.0210872565164405	0.204278830634725	0.791686056496194\\
0.855	0.0221366940292259	0.20924472475049	0.784355603830603\\
0.855	0.023212955654804	0.214212674693577	0.777174850281116\\
0.855	0.0243161165387281	0.219182480034174	0.770191853862643\\
0.855	0.025446250001561	0.224153938988321	0.76344454155446\\
0.855	0.0266034275149466	0.229126848436298	0.756961413910482\\
0.855	0.0277877186778607	0.234101003941464	0.750762400416763\\
0.855	0.0289991911930496	0.239076199769547	0.744859807805787\\
0.855	0.0302379108436654	0.244052228908366	0.739259311518081\\
0.855	0.0315039414701067	0.24902888308801	0.733960949044503\\
0.855	0.0327973449470747	0.254005952801449	0.728960082407262\\
0.855	0.0341181811608519	0.258983227325592	0.724248305111178\\
0.855	0.0354665079868145	0.263960494742775	0.71981427622068\\
0.855	0.0368423812671862	0.268937541962694	0.71564447061263\\
0.855	0.0382458547890421	0.273914154744766	0.711723839838041\\
0.855	0.0396769802625738	0.278890117720924	0.708036382389732\\
0.855	0.0411358072996216	0.283865214418836	0.704565625567164\\
0.855	0.0426223833924861	0.288839227285554	0.701295023640315\\
0.855	0.0441367538930258	0.293811937711588	0.698208278749353\\
0.855	0.0456789619920503	0.298783126055387	0.69528959205535\\
0.855	0.0472490486990192	0.303752571668251	0.692523853197512\\
0.855	0.0488470528220538	0.308720052919643	0.689896776228119\\
0.855	0.0504730109482718	0.313685347222913	0.687394989990283\\
0.855	0.0521269574244539	0.318648231061428	0.685006090465981\\
0.855	0.0538089243380495	0.323608480015096	0.682718662029128\\
0.855	0.0555189414985328	0.328565868787292	0.680522273853469\\
0.855	0.0572570364191156	0.333520171232163	0.67840745699785\\
0.855	0.0590232342988274	0.338471160382329	0.676365666960044\\
0.855	0.0608175580049697	0.34341860847696	0.674389235783156\\
0.855	0.0626400280559541	0.348362286990222	0.672471317134862\\
0.855	0.064490662604533	0.3533019666601	0.670605827172182\\
0.855	0.0663694774214291	0.358237417517579	0.668787383459804\\
0.855	0.0682764858793752	0.363168408916186	0.667011243730728\\
0.855	0.0702116989375697	0.368094709561873	0.665273245863649\\
0.855	0.0721751251265572	0.373016087543259	0.663569750098933\\
0.855	0.0741667705335423	0.377932310362199	0.661897584220061\\
0.855	0.0761866387881432	0.382843144964686	0.660253992184443\\
0.855	0.0782347310485954	0.38774835777208	0.658636586490997\\
0.855	0.0803110459884098	0.392647714712651	0.65704330441587\\
0.855	0.0824155797834956	0.397540981253432	0.655472368126509\\
0.855	0.084548326099754	0.402427922432375	0.653922248592745\\
0.855	0.0867092760811502	0.407308302890799	0.65239163314657\\
0.855	0.0888984183382709	0.412181886906127	0.650879396495579\\
0.855	0.0911157389373742	0.417048438424896	0.649384574964844\\
0.855	0.0933612213899392	0.421907721096044	0.647906343724685\\
0.855	0.0956348466427212	0.42675949830445	0.646443996754806\\
0.855	0.0979365930683194	0.431603533204733	0.644996929295961\\
0.855	0.100266436456264	0.436439588755293	0.643564622546847\\
0.855	0.102624350004627	0.441267427752585	0.642146630374431\\
0.855	0.105010304312169	0.446086812865616	0.640742567819432\\
0.855	0.107424267371012	0.450897506670668	0.639352101193731\\
0.855	0.109866204559871	0.455699271686218	0.637974939582444\\
0.855	0.112336078637817	0.460491870408058	0.636610827579861\\
0.855	0.114833849738607	0.465275065344603	0.635259539104382\\
0.855	0.117359475365564	0.470048619052377	0.633920872153155\\
0.855	0.119912910387022	0.474812294171664	0.632594644371966\\
0.855	0.122494107032347	0.479565853462319	0.631280689329641\\
0.855	0.125103014888515	0.484309059839721	0.629978853399058\\
0.855	0.127739580897283	0.489041676410868	0.62869046082456\\
0.86	0	0	0.710202759607373\\
0.86	1.11327767495586e-05	0.00471862581271275	0.713003157837853\\
0.86	4.46175251031912e-05	0.009446325184051	0.715846486660142\\
0.86	0.000100583311362513	0.0141829653360114	0.718788838379794\\
0.86	0.000179158434668431	0.018928411755669	0.721886092228356\\
0.86	0.000280470402701511	0.02368252819604	0.725200636220892\\
0.86	0.000404645907256436	0.0284451766772965	0.728799085274723\\
0.86	0.000551810799695644	0.0332162174883389	0.732749123983336\\
0.86	0.000722090066287311	0.037995509188729	0.737115575253328\\
0.86	0.000915607803432999	0.0427829086109896	0.741955893907011\\
0.86	0.0011324871927904	0.0475782708632729	0.747315373652289\\
0.86	0.00137285047629673	0.0523814493324038	0.753222424045811\\
0.86	0.00163681893109844	0.057192295687301	0.759684310525652\\
0.86	0.00192451284439304	0.0620106598827802	0.766683748102665\\
0.86	0.00223605148818899	0.0668363901637434	0.774176695511256\\
0.86	0.00257155309398959	0.0716693330697584	0.782091614494494\\
0.86	0.00293113482740716	0.0765093334400312	0.79033034642109\\
0.86	0.00331491276271365	0.0813562344187764	0.79877062752665\\
0.86	0.00372300185733414	0.0862098774609879	0.807270128930359\\
0.86	0.00415551592628969	0.0910701023386145	0.815671782700193\\
0.86	0.00461256761659621	0.0959367471471425	0.82381005350733\\
0.86	0.00509426838162598	0.100809648312589	0.831517746503354\\
0.86	0.00560072845543873	0.105688640598912	0.838632911419452\\
0.86	0.00613205682708918	0.11057355711583	0.845005411376293\\
0.86	0.00668836121491815	0.115464229327074	0.850502769049303\\
0.86	0.00726974804083431	0.120360487059049	0.855014975717656\\
0.86	0.00787632240459385	0.125262158509929	0.858458041096605\\
0.86	0.00850818805808554	0.13016907025918	0.860776163494181\\
0.86	0.00916544737962849	0.135081047277508	0.861942500814223\\
0.86	0.00984820134829017	0.139997912937243	0.86195861464842\\
0.86	0.0105565495182326	0.144919489023162	0.860852735617092\\
0.86	0.0112905899930939	0.149845595743738	0.858677054109649\\
0.86	0.012050419400414	0.154776051742839	0.855504274967457\\
0.86	0.0128361328661109	0.159710674111862	0.851423687967753\\
0.86	0.0136478239890177	0.164649278402306	0.84653700048434\\
0.86	0.0144855848154856	0.169591678638796	0.840954157904247\\
0.86	0.0153495058140643	0.174537687332543	0.834789345403936\\
0.86	0.0162396758502651	0.179487115495259	0.828157325806203\\
0.86	0.0171561821614173	0.184439772653511	0.821170226448694\\
0.86	0.0180991103316243	0.189395466863524	0.813934846725691\\
0.86	0.0190685442668299	0.194354004726437	0.806550519904212\\
0.86	0.0200645661700016	0.199315191403999	0.799107529846303\\
0.86	0.0210872565164405	0.204278830634725	0.791686056496194\\
0.86	0.0221366940292259	0.20924472475049	0.7843556038306\\
0.86	0.023212955654804	0.214212674693577	0.777174850281117\\
0.86	0.0243161165387281	0.219182480034174	0.770191853862644\\
0.86	0.025446250001561	0.224153938988321	0.763444541554457\\
0.86	0.0266034275149466	0.229126848436298	0.756961413910482\\
0.86	0.0277877186778607	0.234101003941464	0.750762400416766\\
0.86	0.0289991911930496	0.239076199769547	0.744859807805785\\
0.86	0.0302379108436654	0.244052228908366	0.739259311518078\\
0.86	0.0315039414701067	0.24902888308801	0.733960949044503\\
0.86	0.0327973449470747	0.254005952801449	0.728960082407261\\
0.86	0.0341181811608519	0.258983227325592	0.724248305111178\\
0.86	0.0354665079868145	0.263960494742775	0.719814276220681\\
0.86	0.0368423812671862	0.268937541962694	0.715644470612629\\
0.86	0.0382458547890421	0.273914154744766	0.711723839838041\\
0.86	0.0396769802625738	0.278890117720924	0.708036382389729\\
0.86	0.0411358072996216	0.283865214418836	0.704565625567165\\
0.86	0.0426223833924861	0.288839227285554	0.701295023640317\\
0.86	0.0441367538930258	0.293811937711588	0.698208278749353\\
0.86	0.0456789619920503	0.298783126055387	0.695289592055348\\
0.86	0.0472490486990192	0.303752571668251	0.692523853197511\\
0.86	0.0488470528220538	0.308720052919643	0.689896776228118\\
0.86	0.0504730109482718	0.313685347222913	0.687394989990282\\
0.86	0.0521269574244539	0.318648231061428	0.685006090465981\\
0.86	0.0538089243380495	0.323608480015096	0.682718662029129\\
0.86	0.0555189414985328	0.328565868787292	0.680522273853471\\
0.86	0.0572570364191156	0.333520171232163	0.67840745699785\\
0.86	0.0590232342988274	0.338471160382329	0.676365666960043\\
0.86	0.0608175580049697	0.34341860847696	0.674389235783155\\
0.86	0.0626400280559541	0.348362286990222	0.672471317134862\\
0.86	0.064490662604533	0.3533019666601	0.670605827172182\\
0.86	0.0663694774214291	0.358237417517579	0.668787383459803\\
0.86	0.0682764858793752	0.363168408916186	0.667011243730728\\
0.86	0.0702116989375697	0.368094709561873	0.66527324586365\\
0.86	0.0721751251265572	0.373016087543259	0.663569750098931\\
0.86	0.0741667705335423	0.377932310362199	0.661897584220061\\
0.86	0.0761866387881432	0.382843144964686	0.660253992184442\\
0.86	0.0782347310485954	0.38774835777208	0.658636586490998\\
0.86	0.0803110459884098	0.392647714712651	0.657043304415871\\
0.86	0.0824155797834956	0.397540981253432	0.655472368126509\\
0.86	0.084548326099754	0.402427922432375	0.653922248592746\\
0.86	0.0867092760811502	0.407308302890799	0.65239163314657\\
0.86	0.0888984183382709	0.412181886906127	0.650879396495578\\
0.86	0.0911157389373742	0.417048438424896	0.649384574964843\\
0.86	0.0933612213899392	0.421907721096044	0.647906343724685\\
0.86	0.0956348466427212	0.42675949830445	0.646443996754806\\
0.86	0.0979365930683194	0.431603533204733	0.644996929295962\\
0.86	0.100266436456264	0.436439588755293	0.643564622546846\\
0.86	0.102624350004627	0.441267427752584	0.642146630374431\\
0.86	0.105010304312169	0.446086812865616	0.640742567819434\\
0.86	0.107424267371012	0.450897506670668	0.639352101193731\\
0.86	0.109866204559871	0.455699271686218	0.637974939582443\\
0.86	0.112336078637817	0.460491870408058	0.636610827579861\\
0.86	0.114833849738607	0.465275065344603	0.635259539104382\\
0.86	0.117359475365564	0.470048619052377	0.633920872153155\\
0.86	0.119912910387023	0.474812294171664	0.632594644371965\\
0.86	0.122494107032347	0.479565853462319	0.631280689329641\\
0.86	0.125103014888515	0.484309059839721	0.62997885339906\\
0.86	0.127739580897283	0.489041676410868	0.628690460824558\\
0.865	0	0	0.710202759607373\\
0.865	1.11327767495586e-05	0.00471862581271274	0.713003157837853\\
0.865	4.46175251031912e-05	0.009446325184051	0.715846486660142\\
0.865	0.000100583311362513	0.0141829653360114	0.718788838379794\\
0.865	0.000179158434668431	0.018928411755669	0.721886092228356\\
0.865	0.000280470402701511	0.02368252819604	0.725200636220892\\
0.865	0.000404645907256436	0.0284451766772965	0.728799085274723\\
0.865	0.000551810799695644	0.0332162174883389	0.732749123983336\\
0.865	0.000722090066287312	0.037995509188729	0.737115575253328\\
0.865	0.000915607803432999	0.0427829086109896	0.741955893907011\\
0.865	0.0011324871927904	0.0475782708632729	0.747315373652289\\
0.865	0.00137285047629673	0.0523814493324038	0.753222424045811\\
0.865	0.00163681893109843	0.057192295687301	0.759684310525652\\
0.865	0.00192451284439304	0.0620106598827802	0.766683748102665\\
0.865	0.00223605148818899	0.0668363901637434	0.774176695511256\\
0.865	0.00257155309398959	0.0716693330697584	0.782091614494493\\
0.865	0.00293113482740716	0.0765093334400313	0.79033034642109\\
0.865	0.00331491276271365	0.0813562344187764	0.79877062752665\\
0.865	0.00372300185733414	0.0862098774609879	0.807270128930359\\
0.865	0.00415551592628969	0.0910701023386145	0.815671782700193\\
0.865	0.00461256761659621	0.0959367471471425	0.82381005350733\\
0.865	0.00509426838162598	0.100809648312589	0.831517746503354\\
0.865	0.00560072845543873	0.105688640598912	0.838632911419452\\
0.865	0.00613205682708918	0.11057355711583	0.845005411376293\\
0.865	0.00668836121491815	0.115464229327074	0.850502769049302\\
0.865	0.00726974804083431	0.120360487059049	0.855014975717657\\
0.865	0.00787632240459385	0.125262158509929	0.858458041096607\\
0.865	0.00850818805808555	0.13016907025918	0.860776163494182\\
0.865	0.00916544737962849	0.135081047277508	0.861942500814225\\
0.865	0.00984820134829017	0.139997912937243	0.861958614648421\\
0.865	0.0105565495182326	0.144919489023162	0.86085273561709\\
0.865	0.0112905899930939	0.149845595743738	0.858677054109649\\
0.865	0.012050419400414	0.154776051742839	0.855504274967457\\
0.865	0.0128361328661109	0.159710674111862	0.851423687967755\\
0.865	0.0136478239890177	0.164649278402306	0.84653700048434\\
0.865	0.0144855848154856	0.169591678638796	0.840954157904249\\
0.865	0.0153495058140643	0.174537687332543	0.834789345403937\\
0.865	0.0162396758502651	0.179487115495259	0.828157325806202\\
0.865	0.0171561821614173	0.184439772653511	0.821170226448692\\
0.865	0.0180991103316243	0.189395466863524	0.813934846725692\\
0.865	0.0190685442668299	0.194354004726437	0.806550519904212\\
0.865	0.0200645661700016	0.199315191403999	0.799107529846303\\
0.865	0.0210872565164405	0.204278830634725	0.791686056496194\\
0.865	0.0221366940292259	0.20924472475049	0.784355603830601\\
0.865	0.023212955654804	0.214212674693577	0.77717485028112\\
0.865	0.0243161165387281	0.219182480034174	0.770191853862644\\
0.865	0.025446250001561	0.224153938988321	0.76344454155446\\
0.865	0.0266034275149466	0.229126848436298	0.756961413910482\\
0.865	0.0277877186778607	0.234101003941464	0.750762400416763\\
0.865	0.0289991911930496	0.239076199769547	0.744859807805784\\
0.865	0.0302379108436654	0.244052228908366	0.73925931151808\\
0.865	0.0315039414701067	0.24902888308801	0.733960949044503\\
0.865	0.0327973449470747	0.254005952801449	0.72896008240726\\
0.865	0.0341181811608519	0.258983227325591	0.724248305111179\\
0.865	0.0354665079868145	0.263960494742775	0.719814276220681\\
0.865	0.0368423812671862	0.268937541962694	0.715644470612629\\
0.865	0.0382458547890421	0.273914154744766	0.711723839838042\\
0.865	0.0396769802625738	0.278890117720924	0.70803638238973\\
0.865	0.0411358072996216	0.283865214418836	0.704565625567164\\
0.865	0.0426223833924861	0.288839227285554	0.701295023640317\\
0.865	0.0441367538930258	0.293811937711588	0.698208278749354\\
0.865	0.0456789619920503	0.298783126055387	0.695289592055349\\
0.865	0.0472490486990192	0.303752571668251	0.692523853197512\\
0.865	0.0488470528220538	0.308720052919643	0.689896776228118\\
0.865	0.0504730109482718	0.313685347222913	0.687394989990281\\
0.865	0.0521269574244539	0.318648231061427	0.685006090465981\\
0.865	0.0538089243380495	0.323608480015096	0.682718662029128\\
0.865	0.0555189414985328	0.328565868787292	0.68052227385347\\
0.865	0.0572570364191156	0.333520171232163	0.67840745699785\\
0.865	0.0590232342988274	0.338471160382329	0.676365666960044\\
0.865	0.0608175580049697	0.34341860847696	0.674389235783156\\
0.865	0.0626400280559541	0.348362286990222	0.672471317134862\\
0.865	0.064490662604533	0.3533019666601	0.670605827172181\\
0.865	0.0663694774214291	0.358237417517579	0.668787383459804\\
0.865	0.0682764858793752	0.363168408916186	0.667011243730728\\
0.865	0.0702116989375697	0.368094709561873	0.665273245863649\\
0.865	0.0721751251265572	0.373016087543259	0.663569750098931\\
0.865	0.0741667705335422	0.377932310362199	0.66189758422006\\
0.865	0.0761866387881432	0.382843144964686	0.660253992184443\\
0.865	0.0782347310485954	0.38774835777208	0.658636586490997\\
0.865	0.0803110459884098	0.392647714712651	0.657043304415872\\
0.865	0.0824155797834956	0.397540981253432	0.655472368126508\\
0.865	0.084548326099754	0.402427922432375	0.653922248592744\\
0.865	0.0867092760811502	0.407308302890799	0.652391633146571\\
0.865	0.0888984183382709	0.412181886906127	0.65087939649558\\
0.865	0.0911157389373742	0.417048438424896	0.649384574964842\\
0.865	0.0933612213899392	0.421907721096044	0.647906343724685\\
0.865	0.0956348466427212	0.42675949830445	0.646443996754806\\
0.865	0.0979365930683194	0.431603533204733	0.644996929295962\\
0.865	0.100266436456264	0.436439588755293	0.643564622546847\\
0.865	0.102624350004627	0.441267427752584	0.64214663037443\\
0.865	0.105010304312169	0.446086812865616	0.640742567819433\\
0.865	0.107424267371012	0.450897506670668	0.63935210119373\\
0.865	0.109866204559871	0.455699271686218	0.637974939582444\\
0.865	0.112336078637817	0.460491870408058	0.636610827579863\\
0.865	0.114833849738607	0.465275065344603	0.635259539104382\\
0.865	0.117359475365564	0.470048619052377	0.633920872153154\\
0.865	0.119912910387022	0.474812294171664	0.632594644371965\\
0.865	0.122494107032347	0.479565853462319	0.63128068932964\\
0.865	0.125103014888515	0.484309059839721	0.629978853399059\\
0.865	0.127739580897283	0.489041676410868	0.62869046082456\\
0.87	0	0	0.710202759607373\\
0.87	1.11327767495586e-05	0.00471862581271275	0.713003157837853\\
0.87	4.46175251031912e-05	0.009446325184051	0.715846486660142\\
0.87	0.000100583311362513	0.0141829653360114	0.718788838379794\\
0.87	0.000179158434668431	0.018928411755669	0.721886092228356\\
0.87	0.000280470402701511	0.02368252819604	0.725200636220892\\
0.87	0.000404645907256436	0.0284451766772965	0.728799085274723\\
0.87	0.000551810799695644	0.0332162174883388	0.732749123983336\\
0.87	0.000722090066287311	0.037995509188729	0.737115575253327\\
0.87	0.000915607803432999	0.0427829086109896	0.741955893907011\\
0.87	0.0011324871927904	0.047578270863273	0.747315373652289\\
0.87	0.00137285047629673	0.0523814493324038	0.753222424045811\\
0.87	0.00163681893109844	0.057192295687301	0.759684310525652\\
0.87	0.00192451284439304	0.0620106598827802	0.766683748102665\\
0.87	0.00223605148818898	0.0668363901637434	0.774176695511256\\
0.87	0.00257155309398959	0.0716693330697584	0.782091614494493\\
0.87	0.00293113482740716	0.0765093334400312	0.79033034642109\\
0.87	0.00331491276271365	0.0813562344187764	0.79877062752665\\
0.87	0.00372300185733413	0.0862098774609879	0.807270128930359\\
0.87	0.00415551592628969	0.0910701023386145	0.815671782700193\\
0.87	0.00461256761659621	0.0959367471471424	0.82381005350733\\
0.87	0.00509426838162598	0.100809648312589	0.831517746503353\\
0.87	0.00560072845543873	0.105688640598912	0.838632911419451\\
0.87	0.00613205682708918	0.11057355711583	0.845005411376293\\
0.87	0.00668836121491816	0.115464229327074	0.850502769049301\\
0.87	0.00726974804083431	0.120360487059049	0.855014975717656\\
0.87	0.00787632240459385	0.125262158509929	0.858458041096606\\
0.87	0.00850818805808555	0.13016907025918	0.860776163494183\\
0.87	0.00916544737962849	0.135081047277508	0.861942500814226\\
0.87	0.00984820134829017	0.139997912937243	0.861958614648419\\
0.87	0.0105565495182326	0.144919489023162	0.86085273561709\\
0.87	0.0112905899930939	0.149845595743738	0.858677054109652\\
0.87	0.012050419400414	0.154776051742839	0.855504274967457\\
0.87	0.0128361328661109	0.159710674111862	0.851423687967755\\
0.87	0.0136478239890177	0.164649278402306	0.846537000484341\\
0.87	0.0144855848154856	0.169591678638796	0.840954157904249\\
0.87	0.0153495058140643	0.174537687332543	0.834789345403936\\
0.87	0.0162396758502651	0.179487115495259	0.828157325806199\\
0.87	0.0171561821614173	0.184439772653511	0.821170226448694\\
0.87	0.0180991103316243	0.189395466863524	0.813934846725693\\
0.87	0.0190685442668299	0.194354004726437	0.806550519904211\\
0.87	0.0200645661700016	0.199315191403999	0.799107529846306\\
0.87	0.0210872565164405	0.204278830634725	0.791686056496194\\
0.87	0.0221366940292259	0.20924472475049	0.784355603830603\\
0.87	0.023212955654804	0.214212674693577	0.777174850281118\\
0.87	0.0243161165387281	0.219182480034174	0.770191853862643\\
0.87	0.025446250001561	0.224153938988321	0.763444541554457\\
0.87	0.0266034275149466	0.229126848436298	0.756961413910482\\
0.87	0.0277877186778607	0.234101003941464	0.750762400416763\\
0.87	0.0289991911930496	0.239076199769547	0.744859807805786\\
0.87	0.0302379108436654	0.244052228908366	0.73925931151808\\
0.87	0.0315039414701067	0.24902888308801	0.733960949044503\\
0.87	0.0327973449470747	0.254005952801449	0.728960082407261\\
0.87	0.0341181811608519	0.258983227325592	0.724248305111177\\
0.87	0.0354665079868145	0.263960494742775	0.71981427622068\\
0.87	0.0368423812671862	0.268937541962694	0.715644470612631\\
0.87	0.0382458547890421	0.273914154744766	0.711723839838041\\
0.87	0.0396769802625738	0.278890117720924	0.70803638238973\\
0.87	0.0411358072996216	0.283865214418836	0.704565625567166\\
0.87	0.0426223833924861	0.288839227285554	0.701295023640317\\
0.87	0.0441367538930258	0.293811937711588	0.698208278749353\\
0.87	0.0456789619920503	0.298783126055387	0.695289592055349\\
0.87	0.0472490486990192	0.303752571668251	0.692523853197512\\
0.87	0.0488470528220538	0.308720052919643	0.689896776228119\\
0.87	0.0504730109482718	0.313685347222913	0.687394989990281\\
0.87	0.0521269574244539	0.318648231061428	0.685006090465981\\
0.87	0.0538089243380495	0.323608480015097	0.682718662029128\\
0.87	0.0555189414985328	0.328565868787292	0.680522273853469\\
0.87	0.0572570364191156	0.333520171232163	0.67840745699785\\
0.87	0.0590232342988274	0.338471160382329	0.676365666960044\\
0.87	0.0608175580049697	0.34341860847696	0.674389235783156\\
0.87	0.0626400280559541	0.348362286990222	0.672471317134862\\
0.87	0.064490662604533	0.3533019666601	0.670605827172182\\
0.87	0.0663694774214291	0.358237417517579	0.668787383459803\\
0.87	0.0682764858793752	0.363168408916186	0.667011243730728\\
0.87	0.0702116989375697	0.368094709561873	0.66527324586365\\
0.87	0.0721751251265572	0.373016087543259	0.663569750098933\\
0.87	0.0741667705335422	0.377932310362199	0.66189758422006\\
0.87	0.0761866387881432	0.382843144964686	0.660253992184443\\
0.87	0.0782347310485954	0.38774835777208	0.658636586490996\\
0.87	0.0803110459884098	0.392647714712651	0.657043304415872\\
0.87	0.0824155797834956	0.397540981253432	0.65547236812651\\
0.87	0.084548326099754	0.402427922432375	0.653922248592745\\
0.87	0.0867092760811502	0.407308302890799	0.652391633146569\\
0.87	0.0888984183382709	0.412181886906127	0.65087939649558\\
0.87	0.0911157389373742	0.417048438424896	0.649384574964843\\
0.87	0.0933612213899392	0.421907721096044	0.647906343724685\\
0.87	0.0956348466427212	0.42675949830445	0.646443996754805\\
0.87	0.0979365930683194	0.431603533204733	0.644996929295962\\
0.87	0.100266436456264	0.436439588755293	0.643564622546848\\
0.87	0.102624350004627	0.441267427752585	0.642146630374429\\
0.87	0.105010304312169	0.446086812865616	0.640742567819433\\
0.87	0.107424267371012	0.450897506670668	0.639352101193731\\
0.87	0.109866204559871	0.455699271686218	0.637974939582442\\
0.87	0.112336078637817	0.460491870408058	0.636610827579862\\
0.87	0.114833849738607	0.465275065344603	0.635259539104384\\
0.87	0.117359475365564	0.470048619052377	0.633920872153154\\
0.87	0.119912910387022	0.474812294171664	0.632594644371965\\
0.87	0.122494107032347	0.479565853462319	0.631280689329641\\
0.87	0.125103014888515	0.484309059839721	0.62997885339906\\
0.87	0.127739580897283	0.489041676410868	0.628690460824556\\
0.875	0	0	0.710202759607373\\
0.875	1.11327767495586e-05	0.00471862581271274	0.713003157837853\\
0.875	4.46175251031912e-05	0.009446325184051	0.715846486660142\\
0.875	0.000100583311362513	0.0141829653360114	0.718788838379794\\
0.875	0.000179158434668431	0.018928411755669	0.721886092228356\\
0.875	0.000280470402701511	0.02368252819604	0.725200636220892\\
0.875	0.000404645907256436	0.0284451766772965	0.728799085274723\\
0.875	0.000551810799695644	0.0332162174883388	0.732749123983336\\
0.875	0.000722090066287311	0.037995509188729	0.737115575253327\\
0.875	0.000915607803432999	0.0427829086109896	0.741955893907011\\
0.875	0.0011324871927904	0.0475782708632729	0.747315373652289\\
0.875	0.00137285047629673	0.0523814493324038	0.75322242404581\\
0.875	0.00163681893109844	0.057192295687301	0.759684310525652\\
0.875	0.00192451284439304	0.0620106598827802	0.766683748102665\\
0.875	0.00223605148818899	0.0668363901637434	0.774176695511256\\
0.875	0.00257155309398959	0.0716693330697584	0.782091614494493\\
0.875	0.00293113482740716	0.0765093334400312	0.79033034642109\\
0.875	0.00331491276271365	0.0813562344187764	0.79877062752665\\
0.875	0.00372300185733413	0.0862098774609879	0.807270128930359\\
0.875	0.00415551592628969	0.0910701023386145	0.815671782700193\\
0.875	0.00461256761659621	0.0959367471471425	0.82381005350733\\
0.875	0.00509426838162598	0.100809648312589	0.831517746503354\\
0.875	0.00560072845543873	0.105688640598912	0.838632911419452\\
0.875	0.00613205682708918	0.11057355711583	0.845005411376293\\
0.875	0.00668836121491816	0.115464229327074	0.850502769049302\\
0.875	0.00726974804083431	0.120360487059049	0.855014975717657\\
0.875	0.00787632240459385	0.125262158509929	0.858458041096605\\
0.875	0.00850818805808554	0.13016907025918	0.860776163494182\\
0.875	0.00916544737962849	0.135081047277508	0.861942500814223\\
0.875	0.00984820134829017	0.139997912937243	0.861958614648422\\
0.875	0.0105565495182326	0.144919489023162	0.860852735617092\\
0.875	0.0112905899930939	0.149845595743738	0.858677054109651\\
0.875	0.012050419400414	0.154776051742839	0.855504274967456\\
0.875	0.0128361328661109	0.159710674111862	0.851423687967756\\
0.875	0.0136478239890177	0.164649278402306	0.846537000484339\\
0.875	0.0144855848154856	0.169591678638796	0.840954157904248\\
0.875	0.0153495058140643	0.174537687332543	0.834789345403933\\
0.875	0.0162396758502651	0.179487115495259	0.828157325806199\\
0.875	0.0171561821614173	0.184439772653511	0.821170226448693\\
0.875	0.0180991103316243	0.189395466863524	0.813934846725691\\
0.875	0.0190685442668299	0.194354004726437	0.806550519904212\\
0.875	0.0200645661700016	0.199315191403999	0.799107529846304\\
0.875	0.0210872565164405	0.204278830634725	0.791686056496193\\
0.875	0.0221366940292259	0.20924472475049	0.784355603830601\\
0.875	0.023212955654804	0.214212674693577	0.777174850281117\\
0.875	0.0243161165387281	0.219182480034174	0.770191853862643\\
0.875	0.025446250001561	0.224153938988321	0.763444541554459\\
0.875	0.0266034275149466	0.229126848436298	0.756961413910482\\
0.875	0.0277877186778607	0.234101003941464	0.750762400416763\\
0.875	0.0289991911930496	0.239076199769547	0.744859807805786\\
0.875	0.0302379108436654	0.244052228908366	0.739259311518078\\
0.875	0.0315039414701067	0.24902888308801	0.733960949044504\\
0.875	0.0327973449470747	0.254005952801449	0.728960082407261\\
0.875	0.0341181811608519	0.258983227325592	0.724248305111177\\
0.875	0.0354665079868145	0.263960494742775	0.719814276220681\\
0.875	0.0368423812671862	0.268937541962694	0.715644470612629\\
0.875	0.0382458547890421	0.273914154744766	0.711723839838041\\
0.875	0.0396769802625737	0.278890117720924	0.708036382389731\\
0.875	0.0411358072996216	0.283865214418836	0.704565625567164\\
0.875	0.0426223833924861	0.288839227285554	0.701295023640315\\
0.875	0.0441367538930258	0.293811937711588	0.698208278749353\\
0.875	0.0456789619920503	0.298783126055387	0.69528959205535\\
0.875	0.0472490486990192	0.303752571668251	0.692523853197512\\
0.875	0.0488470528220538	0.308720052919643	0.689896776228119\\
0.875	0.0504730109482718	0.313685347222913	0.687394989990282\\
0.875	0.0521269574244539	0.318648231061427	0.68500609046598\\
0.875	0.0538089243380495	0.323608480015097	0.682718662029128\\
0.875	0.0555189414985328	0.328565868787292	0.680522273853469\\
0.875	0.0572570364191156	0.333520171232163	0.67840745699785\\
0.875	0.0590232342988274	0.338471160382329	0.676365666960043\\
0.875	0.0608175580049697	0.34341860847696	0.674389235783157\\
0.875	0.0626400280559541	0.348362286990222	0.672471317134863\\
0.875	0.064490662604533	0.3533019666601	0.670605827172182\\
0.875	0.0663694774214291	0.358237417517579	0.668787383459804\\
0.875	0.0682764858793752	0.363168408916186	0.667011243730727\\
0.875	0.0702116989375697	0.368094709561873	0.665273245863649\\
0.875	0.0721751251265572	0.373016087543259	0.663569750098933\\
0.875	0.0741667705335422	0.377932310362199	0.66189758422006\\
0.875	0.0761866387881432	0.382843144964686	0.660253992184443\\
0.875	0.0782347310485954	0.38774835777208	0.658636586490997\\
0.875	0.0803110459884098	0.392647714712651	0.657043304415871\\
0.875	0.0824155797834956	0.397540981253432	0.655472368126509\\
0.875	0.084548326099754	0.402427922432375	0.653922248592747\\
0.875	0.0867092760811502	0.4073083028908	0.652391633146571\\
0.875	0.0888984183382709	0.412181886906127	0.650879396495579\\
0.875	0.0911157389373742	0.417048438424897	0.649384574964842\\
0.875	0.0933612213899392	0.421907721096044	0.647906343724684\\
0.875	0.0956348466427212	0.42675949830445	0.646443996754806\\
0.875	0.0979365930683194	0.431603533204733	0.644996929295961\\
0.875	0.100266436456264	0.436439588755293	0.643564622546847\\
0.875	0.102624350004627	0.441267427752584	0.64214663037443\\
0.875	0.105010304312169	0.446086812865616	0.640742567819433\\
0.875	0.107424267371012	0.450897506670668	0.63935210119373\\
0.875	0.109866204559871	0.455699271686218	0.637974939582442\\
0.875	0.112336078637817	0.460491870408058	0.636610827579861\\
0.875	0.114833849738607	0.465275065344603	0.635259539104384\\
0.875	0.117359475365564	0.470048619052377	0.633920872153154\\
0.875	0.119912910387022	0.474812294171664	0.632594644371966\\
0.875	0.122494107032347	0.479565853462319	0.63128068932964\\
0.875	0.125103014888515	0.484309059839721	0.629978853399059\\
0.875	0.127739580897283	0.489041676410868	0.628690460824564\\
0.88	0	0	0.710202759607373\\
0.88	1.11327767495586e-05	0.00471862581271274	0.713003157837853\\
0.88	4.46175251031912e-05	0.009446325184051	0.715846486660142\\
0.88	0.000100583311362513	0.0141829653360114	0.718788838379794\\
0.88	0.000179158434668431	0.018928411755669	0.721886092228356\\
0.88	0.000280470402701511	0.02368252819604	0.725200636220892\\
0.88	0.000404645907256436	0.0284451766772965	0.728799085274723\\
0.88	0.000551810799695644	0.0332162174883388	0.732749123983336\\
0.88	0.000722090066287311	0.037995509188729	0.737115575253327\\
0.88	0.000915607803432999	0.0427829086109896	0.741955893907011\\
0.88	0.0011324871927904	0.047578270863273	0.747315373652289\\
0.88	0.00137285047629673	0.0523814493324038	0.753222424045811\\
0.88	0.00163681893109844	0.057192295687301	0.759684310525652\\
0.88	0.00192451284439304	0.0620106598827802	0.766683748102665\\
0.88	0.00223605148818898	0.0668363901637434	0.774176695511256\\
0.88	0.00257155309398959	0.0716693330697584	0.782091614494494\\
0.88	0.00293113482740716	0.0765093334400312	0.79033034642109\\
0.88	0.00331491276271365	0.0813562344187764	0.79877062752665\\
0.88	0.00372300185733413	0.0862098774609879	0.807270128930359\\
0.88	0.00415551592628969	0.0910701023386145	0.815671782700194\\
0.88	0.00461256761659621	0.0959367471471425	0.82381005350733\\
0.88	0.00509426838162598	0.100809648312589	0.831517746503354\\
0.88	0.00560072845543873	0.105688640598912	0.838632911419452\\
0.88	0.00613205682708918	0.11057355711583	0.845005411376293\\
0.88	0.00668836121491816	0.115464229327074	0.850502769049302\\
0.88	0.00726974804083431	0.120360487059049	0.855014975717657\\
0.88	0.00787632240459385	0.125262158509929	0.858458041096607\\
0.88	0.00850818805808555	0.13016907025918	0.860776163494181\\
0.88	0.00916544737962849	0.135081047277508	0.861942500814222\\
0.88	0.00984820134829017	0.139997912937243	0.86195861464842\\
0.88	0.0105565495182326	0.144919489023162	0.86085273561709\\
0.88	0.0112905899930939	0.149845595743738	0.85867705410965\\
0.88	0.012050419400414	0.154776051742839	0.855504274967459\\
0.88	0.0128361328661109	0.159710674111862	0.851423687967754\\
0.88	0.0136478239890177	0.164649278402306	0.846537000484339\\
0.88	0.0144855848154856	0.169591678638796	0.840954157904249\\
0.88	0.0153495058140643	0.174537687332543	0.834789345403935\\
0.88	0.0162396758502651	0.179487115495259	0.8281573258062\\
0.88	0.0171561821614173	0.184439772653511	0.821170226448693\\
0.88	0.0180991103316243	0.189395466863524	0.813934846725693\\
0.88	0.0190685442668299	0.194354004726437	0.806550519904212\\
0.88	0.0200645661700016	0.199315191403999	0.799107529846304\\
0.88	0.0210872565164405	0.204278830634725	0.791686056496195\\
0.88	0.0221366940292259	0.20924472475049	0.784355603830601\\
0.88	0.023212955654804	0.214212674693577	0.777174850281118\\
0.88	0.0243161165387281	0.219182480034174	0.770191853862643\\
0.88	0.025446250001561	0.224153938988321	0.763444541554459\\
0.88	0.0266034275149466	0.229126848436298	0.756961413910482\\
0.88	0.0277877186778607	0.234101003941464	0.750762400416766\\
0.88	0.0289991911930496	0.239076199769547	0.744859807805785\\
0.88	0.0302379108436654	0.244052228908366	0.739259311518079\\
0.88	0.0315039414701067	0.24902888308801	0.733960949044503\\
0.88	0.0327973449470747	0.254005952801449	0.728960082407259\\
0.88	0.0341181811608519	0.258983227325591	0.724248305111179\\
0.88	0.0354665079868146	0.263960494742775	0.719814276220682\\
0.88	0.0368423812671862	0.268937541962694	0.71564447061263\\
0.88	0.0382458547890421	0.273914154744766	0.711723839838043\\
0.88	0.0396769802625738	0.278890117720924	0.70803638238973\\
0.88	0.0411358072996216	0.283865214418836	0.704565625567163\\
0.88	0.0426223833924861	0.288839227285554	0.701295023640316\\
0.88	0.0441367538930258	0.293811937711588	0.698208278749353\\
0.88	0.0456789619920503	0.298783126055387	0.69528959205535\\
0.88	0.0472490486990192	0.303752571668251	0.692523853197512\\
0.88	0.0488470528220538	0.308720052919643	0.689896776228118\\
0.88	0.0504730109482718	0.313685347222913	0.687394989990283\\
0.88	0.0521269574244539	0.318648231061428	0.685006090465981\\
0.88	0.0538089243380495	0.323608480015096	0.682718662029127\\
0.88	0.0555189414985328	0.328565868787292	0.68052227385347\\
0.88	0.0572570364191156	0.333520171232163	0.678407456997851\\
0.88	0.0590232342988274	0.338471160382329	0.676365666960043\\
0.88	0.0608175580049697	0.34341860847696	0.674389235783156\\
0.88	0.0626400280559541	0.348362286990222	0.672471317134862\\
0.88	0.064490662604533	0.3533019666601	0.670605827172182\\
0.88	0.0663694774214291	0.358237417517579	0.668787383459805\\
0.88	0.0682764858793752	0.363168408916186	0.667011243730728\\
0.88	0.0702116989375697	0.368094709561873	0.665273245863649\\
0.88	0.0721751251265572	0.373016087543259	0.663569750098933\\
0.88	0.0741667705335422	0.377932310362199	0.66189758422006\\
0.88	0.0761866387881432	0.382843144964686	0.660253992184441\\
0.88	0.0782347310485954	0.38774835777208	0.658636586490997\\
0.88	0.0803110459884098	0.392647714712651	0.65704330441587\\
0.88	0.0824155797834956	0.397540981253432	0.655472368126509\\
0.88	0.084548326099754	0.402427922432375	0.653922248592746\\
0.88	0.0867092760811502	0.4073083028908	0.652391633146572\\
0.88	0.0888984183382709	0.412181886906127	0.650879396495578\\
0.88	0.0911157389373742	0.417048438424896	0.649384574964843\\
0.88	0.0933612213899392	0.421907721096044	0.647906343724684\\
0.88	0.0956348466427212	0.42675949830445	0.646443996754806\\
0.88	0.0979365930683194	0.431603533204733	0.644996929295962\\
0.88	0.100266436456264	0.436439588755293	0.643564622546848\\
0.88	0.102624350004627	0.441267427752585	0.642146630374429\\
0.88	0.105010304312169	0.446086812865616	0.640742567819434\\
0.88	0.107424267371012	0.450897506670668	0.639352101193732\\
0.88	0.109866204559871	0.455699271686218	0.637974939582442\\
0.88	0.112336078637817	0.460491870408058	0.636610827579861\\
0.88	0.114833849738607	0.465275065344603	0.635259539104382\\
0.88	0.117359475365564	0.470048619052377	0.633920872153154\\
0.88	0.119912910387022	0.474812294171664	0.632594644371966\\
0.88	0.122494107032347	0.479565853462319	0.631280689329642\\
0.88	0.125103014888515	0.484309059839721	0.629978853399057\\
0.88	0.127739580897283	0.489041676410868	0.628690460824562\\
0.885	0	0	0.710202759607373\\
0.885	1.11327767495586e-05	0.00471862581271275	0.713003157837853\\
0.885	4.46175251031912e-05	0.009446325184051	0.715846486660142\\
0.885	0.000100583311362513	0.0141829653360114	0.718788838379794\\
0.885	0.000179158434668431	0.018928411755669	0.721886092228356\\
0.885	0.000280470402701511	0.02368252819604	0.725200636220892\\
0.885	0.000404645907256436	0.0284451766772965	0.728799085274723\\
0.885	0.000551810799695644	0.0332162174883388	0.732749123983336\\
0.885	0.000722090066287311	0.037995509188729	0.737115575253327\\
0.885	0.000915607803432999	0.0427829086109896	0.741955893907011\\
0.885	0.0011324871927904	0.0475782708632729	0.747315373652289\\
0.885	0.00137285047629673	0.0523814493324038	0.753222424045811\\
0.885	0.00163681893109844	0.057192295687301	0.759684310525652\\
0.885	0.00192451284439304	0.0620106598827802	0.766683748102665\\
0.885	0.00223605148818898	0.0668363901637434	0.774176695511257\\
0.885	0.00257155309398959	0.0716693330697584	0.782091614494493\\
0.885	0.00293113482740716	0.0765093334400312	0.79033034642109\\
0.885	0.00331491276271365	0.0813562344187764	0.79877062752665\\
0.885	0.00372300185733413	0.0862098774609879	0.80727012893036\\
0.885	0.00415551592628969	0.0910701023386145	0.815671782700194\\
0.885	0.00461256761659621	0.0959367471471425	0.82381005350733\\
0.885	0.00509426838162598	0.100809648312589	0.831517746503354\\
0.885	0.00560072845543873	0.105688640598912	0.838632911419452\\
0.885	0.00613205682708918	0.11057355711583	0.845005411376293\\
0.885	0.00668836121491816	0.115464229327074	0.850502769049302\\
0.885	0.00726974804083431	0.120360487059049	0.855014975717657\\
0.885	0.00787632240459385	0.125262158509929	0.858458041096607\\
0.885	0.00850818805808555	0.13016907025918	0.860776163494182\\
0.885	0.00916544737962849	0.135081047277508	0.861942500814225\\
0.885	0.00984820134829017	0.139997912937243	0.861958614648419\\
0.885	0.0105565495182326	0.144919489023162	0.86085273561709\\
0.885	0.0112905899930939	0.149845595743738	0.858677054109652\\
0.885	0.012050419400414	0.154776051742839	0.855504274967457\\
0.885	0.0128361328661109	0.159710674111862	0.851423687967753\\
0.885	0.0136478239890177	0.164649278402306	0.846537000484341\\
0.885	0.0144855848154856	0.169591678638796	0.840954157904248\\
0.885	0.0153495058140643	0.174537687332543	0.834789345403936\\
0.885	0.0162396758502651	0.179487115495259	0.8281573258062\\
0.885	0.0171561821614173	0.184439772653511	0.821170226448695\\
0.885	0.0180991103316243	0.189395466863524	0.813934846725692\\
0.885	0.0190685442668299	0.194354004726437	0.806550519904213\\
0.885	0.0200645661700016	0.199315191403999	0.799107529846306\\
0.885	0.0210872565164405	0.204278830634725	0.791686056496193\\
0.885	0.0221366940292259	0.20924472475049	0.7843556038306\\
0.885	0.023212955654804	0.214212674693577	0.777174850281118\\
0.885	0.0243161165387281	0.219182480034174	0.770191853862644\\
0.885	0.025446250001561	0.224153938988321	0.763444541554457\\
0.885	0.0266034275149466	0.229126848436298	0.756961413910483\\
0.885	0.0277877186778607	0.234101003941464	0.750762400416765\\
0.885	0.0289991911930496	0.239076199769547	0.744859807805785\\
0.885	0.0302379108436654	0.244052228908366	0.739259311518079\\
0.885	0.0315039414701067	0.24902888308801	0.733960949044504\\
0.885	0.0327973449470747	0.254005952801449	0.728960082407262\\
0.885	0.0341181811608519	0.258983227325592	0.724248305111179\\
0.885	0.0354665079868145	0.263960494742775	0.71981427622068\\
0.885	0.0368423812671862	0.268937541962694	0.715644470612629\\
0.885	0.0382458547890421	0.273914154744766	0.711723839838041\\
0.885	0.0396769802625737	0.278890117720924	0.70803638238973\\
0.885	0.0411358072996216	0.283865214418836	0.704565625567164\\
0.885	0.0426223833924861	0.288839227285554	0.701295023640316\\
0.885	0.0441367538930258	0.293811937711588	0.698208278749354\\
0.885	0.0456789619920503	0.298783126055387	0.695289592055351\\
0.885	0.0472490486990192	0.303752571668251	0.692523853197511\\
0.885	0.0488470528220538	0.308720052919643	0.689896776228118\\
0.885	0.0504730109482718	0.313685347222913	0.687394989990282\\
0.885	0.0521269574244539	0.318648231061428	0.685006090465981\\
0.885	0.0538089243380495	0.323608480015096	0.682718662029128\\
0.885	0.0555189414985328	0.328565868787292	0.680522273853469\\
0.885	0.0572570364191156	0.333520171232163	0.67840745699785\\
0.885	0.0590232342988274	0.338471160382329	0.676365666960044\\
0.885	0.0608175580049697	0.34341860847696	0.674389235783156\\
0.885	0.0626400280559541	0.348362286990222	0.672471317134863\\
0.885	0.064490662604533	0.3533019666601	0.670605827172182\\
0.885	0.0663694774214291	0.358237417517579	0.668787383459803\\
0.885	0.0682764858793752	0.363168408916186	0.667011243730728\\
0.885	0.0702116989375697	0.368094709561873	0.665273245863648\\
0.885	0.0721751251265572	0.373016087543259	0.663569750098932\\
0.885	0.0741667705335422	0.377932310362199	0.661897584220061\\
0.885	0.0761866387881432	0.382843144964686	0.660253992184442\\
0.885	0.0782347310485954	0.38774835777208	0.658636586490997\\
0.885	0.0803110459884098	0.392647714712651	0.657043304415871\\
0.885	0.0824155797834956	0.397540981253432	0.655472368126508\\
0.885	0.084548326099754	0.402427922432375	0.653922248592745\\
0.885	0.0867092760811502	0.407308302890799	0.652391633146572\\
0.885	0.0888984183382709	0.412181886906127	0.65087939649558\\
0.885	0.0911157389373742	0.417048438424896	0.649384574964842\\
0.885	0.0933612213899392	0.421907721096044	0.647906343724685\\
0.885	0.0956348466427212	0.42675949830445	0.646443996754806\\
0.885	0.0979365930683194	0.431603533204733	0.644996929295961\\
0.885	0.100266436456264	0.436439588755293	0.643564622546847\\
0.885	0.102624350004627	0.441267427752584	0.642146630374431\\
0.885	0.105010304312169	0.446086812865616	0.640742567819433\\
0.885	0.107424267371012	0.450897506670668	0.639352101193732\\
0.885	0.109866204559871	0.455699271686218	0.637974939582444\\
0.885	0.112336078637817	0.460491870408058	0.636610827579862\\
0.885	0.114833849738607	0.465275065344603	0.635259539104383\\
0.885	0.117359475365564	0.470048619052377	0.633920872153154\\
0.885	0.119912910387022	0.474812294171664	0.632594644371964\\
0.885	0.122494107032347	0.479565853462319	0.631280689329642\\
0.885	0.125103014888515	0.484309059839721	0.629978853399057\\
0.885	0.127739580897283	0.489041676410868	0.628690460824555\\
0.89	0	0	0.710202759607373\\
0.89	1.11327767495586e-05	0.00471862581271274	0.713003157837853\\
0.89	4.46175251031912e-05	0.009446325184051	0.715846486660142\\
0.89	0.000100583311362513	0.0141829653360114	0.718788838379794\\
0.89	0.000179158434668431	0.018928411755669	0.721886092228356\\
0.89	0.000280470402701511	0.02368252819604	0.725200636220892\\
0.89	0.000404645907256436	0.0284451766772965	0.728799085274723\\
0.89	0.000551810799695644	0.0332162174883389	0.732749123983336\\
0.89	0.000722090066287312	0.037995509188729	0.737115575253327\\
0.89	0.000915607803432999	0.0427829086109896	0.741955893907011\\
0.89	0.0011324871927904	0.0475782708632729	0.747315373652289\\
0.89	0.00137285047629673	0.0523814493324038	0.753222424045811\\
0.89	0.00163681893109844	0.057192295687301	0.759684310525652\\
0.89	0.00192451284439304	0.0620106598827802	0.766683748102665\\
0.89	0.00223605148818899	0.0668363901637434	0.774176695511256\\
0.89	0.00257155309398959	0.0716693330697584	0.782091614494494\\
0.89	0.00293113482740716	0.0765093334400312	0.79033034642109\\
0.89	0.00331491276271365	0.0813562344187764	0.79877062752665\\
0.89	0.00372300185733414	0.0862098774609879	0.807270128930359\\
0.89	0.00415551592628969	0.0910701023386145	0.815671782700193\\
0.89	0.00461256761659621	0.0959367471471424	0.82381005350733\\
0.89	0.00509426838162598	0.100809648312589	0.831517746503354\\
0.89	0.00560072845543873	0.105688640598912	0.838632911419452\\
0.89	0.00613205682708918	0.11057355711583	0.845005411376293\\
0.89	0.00668836121491816	0.115464229327074	0.850502769049301\\
0.89	0.00726974804083431	0.120360487059049	0.855014975717657\\
0.89	0.00787632240459385	0.125262158509929	0.858458041096606\\
0.89	0.00850818805808555	0.13016907025918	0.860776163494182\\
0.89	0.00916544737962849	0.135081047277508	0.861942500814225\\
0.89	0.00984820134829017	0.139997912937243	0.86195861464842\\
0.89	0.0105565495182326	0.144919489023162	0.860852735617092\\
0.89	0.0112905899930939	0.149845595743738	0.858677054109651\\
0.89	0.012050419400414	0.154776051742839	0.855504274967457\\
0.89	0.0128361328661109	0.159710674111862	0.851423687967754\\
0.89	0.0136478239890177	0.164649278402306	0.846537000484341\\
0.89	0.0144855848154856	0.169591678638796	0.840954157904248\\
0.89	0.0153495058140643	0.174537687332543	0.834789345403935\\
0.89	0.0162396758502651	0.179487115495259	0.8281573258062\\
0.89	0.0171561821614173	0.184439772653511	0.821170226448695\\
0.89	0.0180991103316243	0.189395466863524	0.813934846725693\\
0.89	0.0190685442668299	0.194354004726437	0.806550519904214\\
0.89	0.0200645661700016	0.199315191403999	0.799107529846304\\
0.89	0.0210872565164405	0.204278830634725	0.791686056496192\\
0.89	0.0221366940292259	0.20924472475049	0.784355603830601\\
0.89	0.023212955654804	0.214212674693577	0.77717485028112\\
0.89	0.0243161165387281	0.219182480034174	0.770191853862644\\
0.89	0.025446250001561	0.224153938988321	0.76344454155446\\
0.89	0.0266034275149466	0.229126848436298	0.756961413910484\\
0.89	0.0277877186778607	0.234101003941464	0.750762400416765\\
0.89	0.0289991911930496	0.239076199769547	0.744859807805784\\
0.89	0.0302379108436654	0.244052228908366	0.739259311518079\\
0.89	0.0315039414701067	0.24902888308801	0.733960949044503\\
0.89	0.0327973449470747	0.254005952801449	0.728960082407261\\
0.89	0.0341181811608519	0.258983227325592	0.724248305111178\\
0.89	0.0354665079868145	0.263960494742775	0.719814276220682\\
0.89	0.0368423812671862	0.268937541962694	0.715644470612629\\
0.89	0.0382458547890421	0.273914154744766	0.711723839838041\\
0.89	0.0396769802625738	0.278890117720924	0.70803638238973\\
0.89	0.0411358072996216	0.283865214418836	0.704565625567163\\
0.89	0.0426223833924861	0.288839227285554	0.701295023640316\\
0.89	0.0441367538930258	0.293811937711588	0.698208278749354\\
0.89	0.0456789619920503	0.298783126055387	0.695289592055349\\
0.89	0.0472490486990192	0.303752571668251	0.692523853197511\\
0.89	0.0488470528220538	0.308720052919643	0.689896776228119\\
0.89	0.0504730109482718	0.313685347222913	0.687394989990281\\
0.89	0.0521269574244539	0.318648231061427	0.685006090465981\\
0.89	0.0538089243380495	0.323608480015096	0.682718662029128\\
0.89	0.0555189414985328	0.328565868787292	0.680522273853469\\
0.89	0.0572570364191156	0.333520171232163	0.678407456997851\\
0.89	0.0590232342988274	0.338471160382329	0.676365666960044\\
0.89	0.0608175580049697	0.34341860847696	0.674389235783155\\
0.89	0.0626400280559541	0.348362286990222	0.672471317134862\\
0.89	0.064490662604533	0.3533019666601	0.670605827172182\\
0.89	0.0663694774214291	0.358237417517579	0.668787383459804\\
0.89	0.0682764858793752	0.363168408916186	0.667011243730728\\
0.89	0.0702116989375697	0.368094709561873	0.665273245863649\\
0.89	0.0721751251265572	0.373016087543259	0.663569750098931\\
0.89	0.0741667705335422	0.377932310362199	0.661897584220061\\
0.89	0.0761866387881432	0.382843144964686	0.660253992184444\\
0.89	0.0782347310485954	0.38774835777208	0.658636586490996\\
0.89	0.0803110459884098	0.392647714712651	0.657043304415871\\
0.89	0.0824155797834956	0.397540981253432	0.655472368126509\\
0.89	0.084548326099754	0.402427922432375	0.653922248592745\\
0.89	0.0867092760811502	0.407308302890799	0.65239163314657\\
0.89	0.0888984183382709	0.412181886906127	0.650879396495581\\
0.89	0.0911157389373742	0.417048438424897	0.649384574964843\\
0.89	0.0933612213899392	0.421907721096044	0.647906343724683\\
0.89	0.0956348466427212	0.42675949830445	0.646443996754806\\
0.89	0.0979365930683194	0.431603533204733	0.644996929295961\\
0.89	0.100266436456264	0.436439588755293	0.643564622546845\\
0.89	0.102624350004627	0.441267427752584	0.64214663037443\\
0.89	0.105010304312169	0.446086812865616	0.640742567819433\\
0.89	0.107424267371012	0.450897506670668	0.63935210119373\\
0.89	0.109866204559871	0.455699271686218	0.637974939582443\\
0.89	0.112336078637817	0.460491870408058	0.636610827579862\\
0.89	0.114833849738607	0.465275065344603	0.635259539104381\\
0.89	0.117359475365564	0.470048619052377	0.633920872153155\\
0.89	0.119912910387022	0.474812294171664	0.632594644371966\\
0.89	0.122494107032347	0.479565853462319	0.631280689329641\\
0.89	0.125103014888515	0.484309059839721	0.629978853399058\\
0.89	0.127739580897283	0.489041676410868	0.628690460824559\\
0.895	0	0	0.710202759607373\\
0.895	1.11327767495586e-05	0.00471862581271275	0.713003157837853\\
0.895	4.46175251031912e-05	0.009446325184051	0.715846486660142\\
0.895	0.000100583311362513	0.0141829653360114	0.718788838379794\\
0.895	0.000179158434668431	0.018928411755669	0.721886092228356\\
0.895	0.000280470402701511	0.02368252819604	0.725200636220892\\
0.895	0.000404645907256436	0.0284451766772965	0.728799085274723\\
0.895	0.000551810799695644	0.0332162174883389	0.732749123983336\\
0.895	0.000722090066287311	0.037995509188729	0.737115575253328\\
0.895	0.000915607803432999	0.0427829086109896	0.741955893907011\\
0.895	0.0011324871927904	0.0475782708632729	0.747315373652289\\
0.895	0.00137285047629673	0.0523814493324038	0.75322242404581\\
0.895	0.00163681893109843	0.057192295687301	0.759684310525652\\
0.895	0.00192451284439304	0.0620106598827802	0.766683748102665\\
0.895	0.00223605148818899	0.0668363901637434	0.774176695511256\\
0.895	0.00257155309398959	0.0716693330697584	0.782091614494493\\
0.895	0.00293113482740716	0.0765093334400313	0.79033034642109\\
0.895	0.00331491276271365	0.0813562344187764	0.79877062752665\\
0.895	0.00372300185733414	0.0862098774609879	0.807270128930359\\
0.895	0.00415551592628969	0.0910701023386145	0.815671782700193\\
0.895	0.00461256761659621	0.0959367471471425	0.823810053507331\\
0.895	0.00509426838162598	0.100809648312589	0.831517746503354\\
0.895	0.00560072845543873	0.105688640598912	0.838632911419452\\
0.895	0.00613205682708918	0.11057355711583	0.845005411376292\\
0.895	0.00668836121491815	0.115464229327074	0.850502769049303\\
0.895	0.00726974804083431	0.120360487059049	0.855014975717656\\
0.895	0.00787632240459385	0.125262158509929	0.858458041096606\\
0.895	0.00850818805808555	0.13016907025918	0.860776163494182\\
0.895	0.00916544737962849	0.135081047277508	0.861942500814226\\
0.895	0.00984820134829017	0.139997912937243	0.861958614648421\\
0.895	0.0105565495182326	0.144919489023162	0.860852735617092\\
0.895	0.0112905899930939	0.149845595743738	0.85867705410965\\
0.895	0.012050419400414	0.154776051742839	0.855504274967456\\
0.895	0.0128361328661109	0.159710674111862	0.851423687967756\\
0.895	0.0136478239890177	0.164649278402306	0.84653700048434\\
0.895	0.0144855848154856	0.169591678638796	0.840954157904249\\
0.895	0.0153495058140643	0.174537687332543	0.834789345403936\\
0.895	0.0162396758502651	0.179487115495259	0.828157325806199\\
0.895	0.0171561821614173	0.184439772653511	0.821170226448695\\
0.895	0.0180991103316243	0.189395466863524	0.813934846725691\\
0.895	0.0190685442668299	0.194354004726437	0.806550519904211\\
0.895	0.0200645661700016	0.199315191403999	0.799107529846304\\
0.895	0.0210872565164405	0.204278830634725	0.791686056496193\\
0.895	0.0221366940292259	0.20924472475049	0.784355603830603\\
0.895	0.023212955654804	0.214212674693577	0.77717485028112\\
0.895	0.0243161165387281	0.219182480034174	0.770191853862644\\
0.895	0.025446250001561	0.224153938988321	0.763444541554459\\
0.895	0.0266034275149466	0.229126848436298	0.756961413910482\\
0.895	0.0277877186778607	0.234101003941464	0.750762400416766\\
0.895	0.0289991911930496	0.239076199769547	0.744859807805787\\
0.895	0.0302379108436654	0.244052228908366	0.73925931151808\\
0.895	0.0315039414701067	0.24902888308801	0.733960949044501\\
0.895	0.0327973449470747	0.254005952801449	0.728960082407259\\
0.895	0.0341181811608519	0.258983227325591	0.724248305111179\\
0.895	0.0354665079868145	0.263960494742775	0.719814276220681\\
0.895	0.0368423812671862	0.268937541962694	0.715644470612629\\
0.895	0.0382458547890421	0.273914154744766	0.711723839838042\\
0.895	0.0396769802625738	0.278890117720924	0.70803638238973\\
0.895	0.0411358072996216	0.283865214418836	0.704565625567164\\
0.895	0.0426223833924861	0.288839227285554	0.701295023640316\\
0.895	0.0441367538930258	0.293811937711588	0.698208278749354\\
0.895	0.0456789619920503	0.298783126055387	0.695289592055347\\
0.895	0.0472490486990192	0.303752571668251	0.692523853197512\\
0.895	0.0488470528220538	0.308720052919643	0.68989677622812\\
0.895	0.0504730109482718	0.313685347222913	0.687394989990281\\
0.895	0.0521269574244539	0.318648231061428	0.685006090465981\\
0.895	0.0538089243380495	0.323608480015096	0.682718662029128\\
0.895	0.0555189414985328	0.328565868787292	0.68052227385347\\
0.895	0.0572570364191156	0.333520171232163	0.67840745699785\\
0.895	0.0590232342988274	0.338471160382329	0.676365666960044\\
0.895	0.0608175580049697	0.34341860847696	0.674389235783156\\
0.895	0.0626400280559541	0.348362286990222	0.672471317134862\\
0.895	0.064490662604533	0.3533019666601	0.670605827172182\\
0.895	0.066369477421429	0.358237417517579	0.668787383459804\\
0.895	0.0682764858793752	0.363168408916186	0.667011243730729\\
0.895	0.0702116989375697	0.368094709561873	0.665273245863649\\
0.895	0.0721751251265572	0.373016087543259	0.663569750098932\\
0.895	0.0741667705335422	0.377932310362199	0.661897584220059\\
0.895	0.0761866387881432	0.382843144964686	0.660253992184444\\
0.895	0.0782347310485954	0.38774835777208	0.658636586490998\\
0.895	0.0803110459884098	0.392647714712651	0.65704330441587\\
0.895	0.0824155797834956	0.397540981253432	0.655472368126509\\
0.895	0.084548326099754	0.402427922432375	0.653922248592745\\
0.895	0.0867092760811502	0.407308302890799	0.652391633146569\\
0.895	0.0888984183382709	0.412181886906127	0.65087939649558\\
0.895	0.0911157389373742	0.417048438424897	0.649384574964845\\
0.895	0.0933612213899392	0.421907721096044	0.647906343724684\\
0.895	0.0956348466427212	0.42675949830445	0.646443996754806\\
0.895	0.0979365930683194	0.431603533204733	0.644996929295962\\
0.895	0.100266436456264	0.436439588755293	0.643564622546847\\
0.895	0.102624350004627	0.441267427752585	0.642146630374429\\
0.895	0.105010304312169	0.446086812865616	0.640742567819433\\
0.895	0.107424267371012	0.450897506670668	0.639352101193731\\
0.895	0.109866204559871	0.455699271686218	0.637974939582443\\
0.895	0.112336078637817	0.460491870408058	0.636610827579862\\
0.895	0.114833849738607	0.465275065344603	0.635259539104382\\
0.895	0.117359475365564	0.470048619052377	0.633920872153153\\
0.895	0.119912910387022	0.474812294171664	0.632594644371965\\
0.895	0.122494107032347	0.479565853462319	0.631280689329641\\
0.895	0.125103014888515	0.484309059839721	0.629978853399058\\
0.895	0.127739580897283	0.489041676410868	0.628690460824561\\
0.9	0	0	0.710202759607373\\
0.9	1.11327767495586e-05	0.00471862581271274	0.713003157837853\\
0.9	4.46175251031912e-05	0.009446325184051	0.715846486660142\\
0.9	0.000100583311362513	0.0141829653360114	0.718788838379794\\
0.9	0.000179158434668431	0.018928411755669	0.721886092228356\\
0.9	0.000280470402701511	0.02368252819604	0.725200636220892\\
0.9	0.000404645907256436	0.0284451766772965	0.728799085274723\\
0.9	0.000551810799695644	0.0332162174883388	0.732749123983336\\
0.9	0.000722090066287311	0.037995509188729	0.737115575253328\\
0.9	0.000915607803432999	0.0427829086109896	0.741955893907011\\
0.9	0.0011324871927904	0.047578270863273	0.747315373652289\\
0.9	0.00137285047629673	0.0523814493324038	0.75322242404581\\
0.9	0.00163681893109844	0.057192295687301	0.759684310525652\\
0.9	0.00192451284439304	0.0620106598827802	0.766683748102665\\
0.9	0.00223605148818899	0.0668363901637434	0.774176695511256\\
0.9	0.00257155309398959	0.0716693330697584	0.782091614494493\\
0.9	0.00293113482740716	0.0765093334400312	0.79033034642109\\
0.9	0.00331491276271365	0.0813562344187764	0.79877062752665\\
0.9	0.00372300185733414	0.0862098774609879	0.807270128930359\\
0.9	0.00415551592628969	0.0910701023386145	0.815671782700193\\
0.9	0.00461256761659621	0.0959367471471425	0.823810053507331\\
0.9	0.00509426838162598	0.100809648312589	0.831517746503354\\
0.9	0.00560072845543873	0.105688640598912	0.838632911419452\\
0.9	0.00613205682708918	0.11057355711583	0.845005411376293\\
0.9	0.00668836121491815	0.115464229327074	0.850502769049302\\
0.9	0.00726974804083431	0.120360487059049	0.855014975717657\\
0.9	0.00787632240459385	0.125262158509929	0.858458041096605\\
0.9	0.00850818805808555	0.13016907025918	0.860776163494182\\
0.9	0.00916544737962849	0.135081047277508	0.861942500814224\\
0.9	0.00984820134829017	0.139997912937243	0.861958614648422\\
0.9	0.0105565495182326	0.144919489023162	0.860852735617091\\
0.9	0.0112905899930939	0.149845595743738	0.85867705410965\\
0.9	0.012050419400414	0.154776051742839	0.855504274967458\\
0.9	0.0128361328661109	0.159710674111862	0.851423687967755\\
0.9	0.0136478239890177	0.164649278402306	0.846537000484341\\
0.9	0.0144855848154856	0.169591678638796	0.840954157904248\\
0.9	0.0153495058140643	0.174537687332543	0.834789345403935\\
0.9	0.0162396758502651	0.179487115495259	0.828157325806201\\
0.9	0.0171561821614173	0.184439772653511	0.821170226448695\\
0.9	0.0180991103316243	0.189395466863524	0.813934846725691\\
0.9	0.0190685442668299	0.194354004726437	0.806550519904212\\
0.9	0.0200645661700016	0.199315191403999	0.799107529846305\\
0.9	0.0210872565164405	0.204278830634725	0.791686056496193\\
0.9	0.0221366940292259	0.20924472475049	0.784355603830605\\
0.9	0.023212955654804	0.214212674693577	0.777174850281115\\
0.9	0.0243161165387281	0.219182480034174	0.770191853862644\\
0.9	0.025446250001561	0.224153938988321	0.763444541554459\\
0.9	0.0266034275149466	0.229126848436298	0.756961413910483\\
0.9	0.0277877186778607	0.234101003941464	0.750762400416766\\
0.9	0.0289991911930496	0.239076199769547	0.744859807805785\\
0.9	0.0302379108436654	0.244052228908366	0.739259311518077\\
0.9	0.0315039414701067	0.24902888308801	0.733960949044502\\
0.9	0.0327973449470747	0.254005952801449	0.72896008240726\\
0.9	0.0341181811608519	0.258983227325592	0.724248305111178\\
0.9	0.0354665079868145	0.263960494742775	0.71981427622068\\
0.9	0.0368423812671862	0.268937541962694	0.71564447061263\\
0.9	0.0382458547890421	0.273914154744766	0.711723839838042\\
0.9	0.0396769802625738	0.278890117720924	0.708036382389732\\
0.9	0.0411358072996216	0.283865214418836	0.704565625567164\\
0.9	0.0426223833924861	0.288839227285554	0.701295023640316\\
0.9	0.0441367538930258	0.293811937711588	0.698208278749355\\
0.9	0.0456789619920503	0.298783126055387	0.69528959205535\\
0.9	0.0472490486990192	0.303752571668251	0.692523853197512\\
0.9	0.0488470528220538	0.308720052919643	0.689896776228117\\
0.9	0.0504730109482718	0.313685347222913	0.687394989990281\\
0.9	0.0521269574244539	0.318648231061428	0.685006090465982\\
0.9	0.0538089243380495	0.323608480015097	0.682718662029128\\
0.9	0.0555189414985328	0.328565868787292	0.680522273853469\\
0.9	0.0572570364191156	0.333520171232163	0.678407456997849\\
0.9	0.0590232342988274	0.338471160382329	0.676365666960044\\
0.9	0.0608175580049697	0.34341860847696	0.674389235783157\\
0.9	0.0626400280559541	0.348362286990222	0.672471317134862\\
0.9	0.064490662604533	0.3533019666601	0.670605827172182\\
0.9	0.066369477421429	0.358237417517579	0.668787383459804\\
0.9	0.0682764858793752	0.363168408916186	0.667011243730728\\
0.9	0.0702116989375697	0.368094709561873	0.665273245863649\\
0.9	0.0721751251265572	0.373016087543259	0.663569750098932\\
0.9	0.0741667705335422	0.377932310362199	0.661897584220061\\
0.9	0.0761866387881432	0.382843144964686	0.660253992184441\\
0.9	0.0782347310485954	0.38774835777208	0.658636586490997\\
0.9	0.0803110459884098	0.392647714712651	0.657043304415872\\
0.9	0.0824155797834956	0.397540981253432	0.655472368126509\\
0.9	0.084548326099754	0.402427922432375	0.653922248592746\\
0.9	0.0867092760811502	0.4073083028908	0.65239163314657\\
0.9	0.0888984183382709	0.412181886906127	0.650879396495577\\
0.9	0.0911157389373742	0.417048438424896	0.649384574964844\\
0.9	0.0933612213899392	0.421907721096044	0.647906343724685\\
0.9	0.0956348466427212	0.42675949830445	0.646443996754806\\
0.9	0.0979365930683194	0.431603533204733	0.644996929295961\\
0.9	0.100266436456264	0.436439588755293	0.643564622546847\\
0.9	0.102624350004627	0.441267427752584	0.64214663037443\\
0.9	0.105010304312169	0.446086812865616	0.640742567819433\\
0.9	0.107424267371012	0.450897506670668	0.63935210119373\\
0.9	0.109866204559871	0.455699271686218	0.637974939582444\\
0.9	0.112336078637817	0.460491870408058	0.636610827579863\\
0.9	0.114833849738607	0.465275065344603	0.635259539104383\\
0.9	0.117359475365564	0.470048619052377	0.633920872153156\\
0.9	0.119912910387023	0.474812294171664	0.632594644371964\\
0.9	0.122494107032347	0.479565853462319	0.631280689329641\\
0.9	0.125103014888515	0.484309059839721	0.629978853399059\\
0.9	0.127739580897283	0.489041676410868	0.628690460824556\\
0.905	0	0	0.710202759607373\\
0.905	1.11327767495586e-05	0.00471862581271275	0.713003157837853\\
0.905	4.46175251031912e-05	0.009446325184051	0.715846486660142\\
0.905	0.000100583311362513	0.0141829653360114	0.718788838379794\\
0.905	0.000179158434668431	0.018928411755669	0.721886092228356\\
0.905	0.000280470402701511	0.02368252819604	0.725200636220892\\
0.905	0.000404645907256436	0.0284451766772965	0.728799085274723\\
0.905	0.000551810799695644	0.0332162174883389	0.732749123983336\\
0.905	0.000722090066287311	0.037995509188729	0.737115575253328\\
0.905	0.000915607803432999	0.0427829086109896	0.741955893907011\\
0.905	0.0011324871927904	0.0475782708632729	0.747315373652289\\
0.905	0.00137285047629673	0.0523814493324038	0.753222424045811\\
0.905	0.00163681893109844	0.057192295687301	0.759684310525652\\
0.905	0.00192451284439304	0.0620106598827802	0.766683748102665\\
0.905	0.00223605148818899	0.0668363901637434	0.774176695511257\\
0.905	0.00257155309398959	0.0716693330697584	0.782091614494493\\
0.905	0.00293113482740716	0.0765093334400312	0.79033034642109\\
0.905	0.00331491276271365	0.0813562344187764	0.79877062752665\\
0.905	0.00372300185733413	0.0862098774609879	0.80727012893036\\
0.905	0.00415551592628969	0.0910701023386145	0.815671782700193\\
0.905	0.00461256761659621	0.0959367471471425	0.823810053507331\\
0.905	0.00509426838162598	0.100809648312589	0.831517746503354\\
0.905	0.00560072845543873	0.105688640598912	0.838632911419451\\
0.905	0.00613205682708918	0.11057355711583	0.845005411376293\\
0.905	0.00668836121491815	0.115464229327074	0.850502769049301\\
0.905	0.00726974804083431	0.120360487059049	0.855014975717657\\
0.905	0.00787632240459385	0.125262158509929	0.858458041096605\\
0.905	0.00850818805808555	0.13016907025918	0.860776163494182\\
0.905	0.00916544737962849	0.135081047277508	0.861942500814224\\
0.905	0.00984820134829017	0.139997912937243	0.86195861464842\\
0.905	0.0105565495182326	0.144919489023162	0.86085273561709\\
0.905	0.0112905899930939	0.149845595743738	0.858677054109652\\
0.905	0.012050419400414	0.154776051742839	0.855504274967458\\
0.905	0.0128361328661109	0.159710674111862	0.851423687967753\\
0.905	0.0136478239890177	0.164649278402306	0.846537000484341\\
0.905	0.0144855848154856	0.169591678638796	0.840954157904245\\
0.905	0.0153495058140643	0.174537687332543	0.834789345403936\\
0.905	0.0162396758502651	0.179487115495259	0.828157325806203\\
0.905	0.0171561821614173	0.184439772653511	0.821170226448693\\
0.905	0.0180991103316243	0.189395466863524	0.813934846725692\\
0.905	0.0190685442668299	0.194354004726437	0.806550519904213\\
0.905	0.0200645661700016	0.199315191403999	0.799107529846305\\
0.905	0.0210872565164405	0.204278830634725	0.791686056496194\\
0.905	0.0221366940292259	0.20924472475049	0.784355603830601\\
0.905	0.023212955654804	0.214212674693577	0.777174850281115\\
0.905	0.0243161165387281	0.219182480034174	0.770191853862645\\
0.905	0.025446250001561	0.224153938988321	0.763444541554457\\
0.905	0.0266034275149466	0.229126848436298	0.756961413910483\\
0.905	0.0277877186778607	0.234101003941464	0.750762400416764\\
0.905	0.0289991911930496	0.239076199769547	0.744859807805783\\
0.905	0.0302379108436654	0.244052228908366	0.739259311518079\\
0.905	0.0315039414701067	0.24902888308801	0.733960949044502\\
0.905	0.0327973449470747	0.254005952801449	0.72896008240726\\
0.905	0.0341181811608519	0.258983227325592	0.724248305111179\\
0.905	0.0354665079868145	0.263960494742775	0.719814276220682\\
0.905	0.0368423812671862	0.268937541962694	0.715644470612629\\
0.905	0.0382458547890421	0.273914154744766	0.711723839838041\\
0.905	0.0396769802625738	0.278890117720924	0.70803638238973\\
0.905	0.0411358072996216	0.283865214418836	0.704565625567164\\
0.905	0.0426223833924861	0.288839227285554	0.701295023640317\\
0.905	0.0441367538930258	0.293811937711588	0.698208278749354\\
0.905	0.0456789619920503	0.298783126055387	0.695289592055346\\
0.905	0.0472490486990192	0.303752571668251	0.692523853197511\\
0.905	0.0488470528220538	0.308720052919643	0.689896776228121\\
0.905	0.0504730109482718	0.313685347222913	0.687394989990282\\
0.905	0.0521269574244539	0.318648231061428	0.68500609046598\\
0.905	0.0538089243380495	0.323608480015096	0.682718662029128\\
0.905	0.0555189414985328	0.328565868787292	0.68052227385347\\
0.905	0.0572570364191156	0.333520171232163	0.67840745699785\\
0.905	0.0590232342988274	0.338471160382329	0.676365666960044\\
0.905	0.0608175580049697	0.34341860847696	0.674389235783155\\
0.905	0.0626400280559541	0.348362286990221	0.672471317134862\\
0.905	0.064490662604533	0.3533019666601	0.670605827172183\\
0.905	0.0663694774214291	0.358237417517579	0.668787383459804\\
0.905	0.0682764858793752	0.363168408916186	0.667011243730728\\
0.905	0.0702116989375697	0.368094709561873	0.665273245863648\\
0.905	0.0721751251265572	0.373016087543259	0.663569750098931\\
0.905	0.0741667705335422	0.377932310362199	0.661897584220061\\
0.905	0.0761866387881432	0.382843144964686	0.660253992184444\\
0.905	0.0782347310485954	0.38774835777208	0.658636586490998\\
0.905	0.0803110459884098	0.392647714712651	0.657043304415871\\
0.905	0.0824155797834956	0.397540981253432	0.655472368126509\\
0.905	0.084548326099754	0.402427922432375	0.653922248592745\\
0.905	0.0867092760811502	0.4073083028908	0.652391633146571\\
0.905	0.0888984183382709	0.412181886906127	0.650879396495579\\
0.905	0.0911157389373742	0.417048438424897	0.649384574964843\\
0.905	0.0933612213899393	0.421907721096044	0.647906343724685\\
0.905	0.0956348466427212	0.42675949830445	0.646443996754805\\
0.905	0.0979365930683194	0.431603533204733	0.644996929295962\\
0.905	0.100266436456264	0.436439588755293	0.643564622546847\\
0.905	0.102624350004627	0.441267427752584	0.64214663037443\\
0.905	0.105010304312169	0.446086812865616	0.640742567819433\\
0.905	0.107424267371012	0.450897506670668	0.639352101193731\\
0.905	0.109866204559871	0.455699271686218	0.637974939582443\\
0.905	0.112336078637817	0.460491870408058	0.636610827579862\\
0.905	0.114833849738607	0.465275065344603	0.635259539104382\\
0.905	0.117359475365564	0.470048619052377	0.633920872153156\\
0.905	0.119912910387022	0.474812294171664	0.632594644371966\\
0.905	0.122494107032347	0.479565853462319	0.631280689329639\\
0.905	0.125103014888515	0.484309059839721	0.629978853399059\\
0.905	0.127739580897283	0.489041676410868	0.628690460824562\\
0.91	0	0	0.710202759607373\\
0.91	1.11327767495586e-05	0.00471862581271275	0.713003157837853\\
0.91	4.46175251031912e-05	0.009446325184051	0.715846486660142\\
0.91	0.000100583311362513	0.0141829653360114	0.718788838379794\\
0.91	0.000179158434668431	0.018928411755669	0.721886092228356\\
0.91	0.000280470402701511	0.02368252819604	0.725200636220892\\
0.91	0.000404645907256436	0.0284451766772965	0.728799085274723\\
0.91	0.000551810799695644	0.0332162174883389	0.732749123983336\\
0.91	0.000722090066287311	0.037995509188729	0.737115575253327\\
0.91	0.000915607803432999	0.0427829086109896	0.741955893907011\\
0.91	0.0011324871927904	0.0475782708632729	0.747315373652289\\
0.91	0.00137285047629673	0.0523814493324038	0.753222424045811\\
0.91	0.00163681893109844	0.057192295687301	0.759684310525652\\
0.91	0.00192451284439304	0.0620106598827802	0.766683748102665\\
0.91	0.00223605148818899	0.0668363901637434	0.774176695511257\\
0.91	0.00257155309398959	0.0716693330697584	0.782091614494493\\
0.91	0.00293113482740716	0.0765093334400312	0.79033034642109\\
0.91	0.00331491276271365	0.0813562344187764	0.79877062752665\\
0.91	0.00372300185733414	0.0862098774609879	0.80727012893036\\
0.91	0.00415551592628969	0.0910701023386145	0.815671782700193\\
0.91	0.00461256761659621	0.0959367471471425	0.82381005350733\\
0.91	0.00509426838162598	0.100809648312589	0.831517746503354\\
0.91	0.00560072845543873	0.105688640598912	0.838632911419452\\
0.91	0.00613205682708918	0.11057355711583	0.845005411376292\\
0.91	0.00668836121491816	0.115464229327074	0.850502769049302\\
0.91	0.00726974804083431	0.120360487059049	0.855014975717657\\
0.91	0.00787632240459385	0.125262158509929	0.858458041096607\\
0.91	0.00850818805808555	0.13016907025918	0.860776163494181\\
0.91	0.00916544737962849	0.135081047277508	0.861942500814224\\
0.91	0.00984820134829017	0.139997912937243	0.861958614648419\\
0.91	0.0105565495182326	0.144919489023162	0.860852735617091\\
0.91	0.0112905899930939	0.149845595743738	0.858677054109651\\
0.91	0.012050419400414	0.154776051742839	0.855504274967456\\
0.91	0.0128361328661109	0.159710674111862	0.851423687967755\\
0.91	0.0136478239890177	0.164649278402306	0.846537000484339\\
0.91	0.0144855848154856	0.169591678638796	0.84095415790425\\
0.91	0.0153495058140643	0.174537687332543	0.834789345403936\\
0.91	0.0162396758502651	0.179487115495259	0.8281573258062\\
0.91	0.0171561821614173	0.184439772653511	0.821170226448692\\
0.91	0.0180991103316243	0.189395466863524	0.813934846725692\\
0.91	0.0190685442668299	0.194354004726437	0.806550519904212\\
0.91	0.0200645661700016	0.199315191403999	0.799107529846304\\
0.91	0.0210872565164405	0.204278830634725	0.791686056496193\\
0.91	0.0221366940292259	0.20924472475049	0.7843556038306\\
0.91	0.023212955654804	0.214212674693577	0.777174850281118\\
0.91	0.0243161165387281	0.219182480034174	0.770191853862643\\
0.91	0.025446250001561	0.224153938988321	0.763444541554459\\
0.91	0.0266034275149466	0.229126848436298	0.756961413910483\\
0.91	0.0277877186778607	0.234101003941464	0.750762400416764\\
0.91	0.0289991911930496	0.239076199769547	0.744859807805786\\
0.91	0.0302379108436654	0.244052228908366	0.739259311518078\\
0.91	0.0315039414701067	0.24902888308801	0.733960949044502\\
0.91	0.0327973449470747	0.254005952801449	0.728960082407261\\
0.91	0.0341181811608519	0.258983227325592	0.724248305111179\\
0.91	0.0354665079868145	0.263960494742775	0.719814276220681\\
0.91	0.0368423812671862	0.268937541962694	0.715644470612629\\
0.91	0.0382458547890421	0.273914154744766	0.711723839838042\\
0.91	0.0396769802625738	0.278890117720924	0.70803638238973\\
0.91	0.0411358072996216	0.283865214418836	0.704565625567164\\
0.91	0.0426223833924861	0.288839227285554	0.701295023640317\\
0.91	0.0441367538930258	0.293811937711588	0.698208278749353\\
0.91	0.0456789619920503	0.298783126055387	0.695289592055347\\
0.91	0.0472490486990192	0.303752571668251	0.692523853197512\\
0.91	0.0488470528220537	0.308720052919643	0.68989677622812\\
0.91	0.0504730109482718	0.313685347222913	0.687394989990281\\
0.91	0.0521269574244539	0.318648231061428	0.685006090465981\\
0.91	0.0538089243380495	0.323608480015097	0.682718662029127\\
0.91	0.0555189414985328	0.328565868787292	0.68052227385347\\
0.91	0.0572570364191156	0.333520171232163	0.67840745699785\\
0.91	0.0590232342988274	0.338471160382329	0.676365666960043\\
0.91	0.0608175580049697	0.34341860847696	0.674389235783156\\
0.91	0.0626400280559541	0.348362286990222	0.672471317134862\\
0.91	0.0644906626045329	0.3533019666601	0.670605827172183\\
0.91	0.066369477421429	0.358237417517579	0.668787383459804\\
0.91	0.0682764858793752	0.363168408916186	0.667011243730728\\
0.91	0.0702116989375697	0.368094709561873	0.665273245863649\\
0.91	0.0721751251265572	0.373016087543259	0.663569750098931\\
0.91	0.0741667705335422	0.377932310362199	0.661897584220059\\
0.91	0.0761866387881432	0.382843144964686	0.660253992184443\\
0.91	0.0782347310485954	0.38774835777208	0.658636586490998\\
0.91	0.0803110459884098	0.392647714712651	0.657043304415871\\
0.91	0.0824155797834956	0.397540981253432	0.65547236812651\\
0.91	0.084548326099754	0.402427922432375	0.653922248592745\\
0.91	0.0867092760811502	0.407308302890799	0.65239163314657\\
0.91	0.0888984183382709	0.412181886906127	0.650879396495579\\
0.91	0.0911157389373742	0.417048438424896	0.649384574964844\\
0.91	0.0933612213899392	0.421907721096044	0.647906343724686\\
0.91	0.0956348466427212	0.42675949830445	0.646443996754805\\
0.91	0.0979365930683194	0.431603533204733	0.644996929295961\\
0.91	0.100266436456264	0.436439588755293	0.643564622546847\\
0.91	0.102624350004627	0.441267427752584	0.64214663037443\\
0.91	0.105010304312169	0.446086812865616	0.640742567819433\\
0.91	0.107424267371012	0.450897506670668	0.63935210119373\\
0.91	0.109866204559871	0.455699271686218	0.637974939582443\\
0.91	0.112336078637817	0.460491870408058	0.636610827579863\\
0.91	0.114833849738607	0.465275065344603	0.635259539104382\\
0.91	0.117359475365564	0.470048619052377	0.633920872153154\\
0.91	0.119912910387022	0.474812294171664	0.632594644371966\\
0.91	0.122494107032347	0.479565853462319	0.631280689329641\\
0.91	0.125103014888515	0.484309059839721	0.629978853399057\\
0.91	0.127739580897283	0.489041676410868	0.628690460824557\\
0.915	0	0	0.710202759607373\\
0.915	1.11327767495586e-05	0.00471862581271274	0.713003157837853\\
0.915	4.46175251031912e-05	0.009446325184051	0.715846486660142\\
0.915	0.000100583311362513	0.0141829653360114	0.718788838379794\\
0.915	0.000179158434668431	0.018928411755669	0.721886092228356\\
0.915	0.000280470402701511	0.02368252819604	0.725200636220892\\
0.915	0.000404645907256436	0.0284451766772965	0.728799085274723\\
0.915	0.000551810799695644	0.0332162174883389	0.732749123983336\\
0.915	0.000722090066287311	0.037995509188729	0.737115575253328\\
0.915	0.000915607803432999	0.0427829086109896	0.741955893907011\\
0.915	0.0011324871927904	0.0475782708632729	0.747315373652289\\
0.915	0.00137285047629673	0.0523814493324038	0.75322242404581\\
0.915	0.00163681893109843	0.057192295687301	0.759684310525652\\
0.915	0.00192451284439304	0.0620106598827802	0.766683748102665\\
0.915	0.00223605148818899	0.0668363901637434	0.774176695511256\\
0.915	0.00257155309398959	0.0716693330697584	0.782091614494494\\
0.915	0.00293113482740716	0.0765093334400312	0.79033034642109\\
0.915	0.00331491276271365	0.0813562344187764	0.79877062752665\\
0.915	0.00372300185733414	0.0862098774609879	0.807270128930359\\
0.915	0.00415551592628969	0.0910701023386145	0.815671782700194\\
0.915	0.00461256761659621	0.0959367471471425	0.82381005350733\\
0.915	0.00509426838162598	0.100809648312589	0.831517746503354\\
0.915	0.00560072845543873	0.105688640598912	0.838632911419452\\
0.915	0.00613205682708918	0.11057355711583	0.845005411376292\\
0.915	0.00668836121491815	0.115464229327074	0.850502769049302\\
0.915	0.00726974804083431	0.120360487059049	0.855014975717657\\
0.915	0.00787632240459385	0.125262158509929	0.858458041096608\\
0.915	0.00850818805808554	0.13016907025918	0.860776163494182\\
0.915	0.00916544737962849	0.135081047277508	0.861942500814224\\
0.915	0.00984820134829017	0.139997912937243	0.861958614648419\\
0.915	0.0105565495182326	0.144919489023162	0.860852735617091\\
0.915	0.0112905899930939	0.149845595743738	0.858677054109649\\
0.915	0.012050419400414	0.154776051742839	0.855504274967457\\
0.915	0.0128361328661109	0.159710674111862	0.851423687967754\\
0.915	0.0136478239890177	0.164649278402306	0.846537000484339\\
0.915	0.0144855848154856	0.169591678638796	0.84095415790425\\
0.915	0.0153495058140643	0.174537687332543	0.834789345403935\\
0.915	0.0162396758502651	0.179487115495259	0.8281573258062\\
0.915	0.0171561821614173	0.184439772653511	0.821170226448694\\
0.915	0.0180991103316243	0.189395466863524	0.813934846725693\\
0.915	0.0190685442668299	0.194354004726437	0.806550519904212\\
0.915	0.0200645661700016	0.199315191403999	0.799107529846304\\
0.915	0.0210872565164405	0.204278830634725	0.791686056496193\\
0.915	0.0221366940292259	0.20924472475049	0.784355603830602\\
0.915	0.023212955654804	0.214212674693577	0.777174850281119\\
0.915	0.0243161165387281	0.219182480034174	0.770191853862643\\
0.915	0.025446250001561	0.224153938988321	0.76344454155446\\
0.915	0.0266034275149466	0.229126848436298	0.756961413910482\\
0.915	0.0277877186778607	0.234101003941464	0.750762400416764\\
0.915	0.0289991911930496	0.239076199769547	0.744859807805785\\
0.915	0.0302379108436654	0.244052228908366	0.739259311518077\\
0.915	0.0315039414701067	0.24902888308801	0.733960949044504\\
0.915	0.0327973449470747	0.254005952801449	0.728960082407262\\
0.915	0.0341181811608519	0.258983227325592	0.724248305111178\\
0.915	0.0354665079868145	0.263960494742775	0.71981427622068\\
0.915	0.0368423812671862	0.268937541962694	0.715644470612629\\
0.915	0.0382458547890421	0.273914154744766	0.711723839838042\\
0.915	0.0396769802625738	0.278890117720924	0.70803638238973\\
0.915	0.0411358072996216	0.283865214418836	0.704565625567164\\
0.915	0.0426223833924861	0.288839227285554	0.701295023640316\\
0.915	0.0441367538930258	0.293811937711588	0.698208278749353\\
0.915	0.0456789619920503	0.298783126055387	0.695289592055349\\
0.915	0.0472490486990192	0.303752571668251	0.692523853197512\\
0.915	0.0488470528220538	0.308720052919643	0.689896776228119\\
0.915	0.0504730109482718	0.313685347222913	0.687394989990281\\
0.915	0.0521269574244539	0.318648231061427	0.685006090465981\\
0.915	0.0538089243380495	0.323608480015096	0.682718662029128\\
0.915	0.0555189414985328	0.328565868787292	0.680522273853469\\
0.915	0.0572570364191156	0.333520171232163	0.67840745699785\\
0.915	0.0590232342988274	0.338471160382329	0.676365666960044\\
0.915	0.0608175580049697	0.34341860847696	0.674389235783157\\
0.915	0.0626400280559541	0.348362286990222	0.672471317134861\\
0.915	0.064490662604533	0.3533019666601	0.670605827172181\\
0.915	0.0663694774214291	0.358237417517579	0.668787383459805\\
0.915	0.0682764858793752	0.363168408916186	0.667011243730728\\
0.915	0.0702116989375697	0.368094709561873	0.66527324586365\\
0.915	0.0721751251265572	0.373016087543259	0.663569750098933\\
0.915	0.0741667705335422	0.377932310362199	0.66189758422006\\
0.915	0.0761866387881432	0.382843144964686	0.660253992184442\\
0.915	0.0782347310485954	0.38774835777208	0.658636586490997\\
0.915	0.0803110459884098	0.392647714712651	0.65704330441587\\
0.915	0.0824155797834956	0.397540981253432	0.65547236812651\\
0.915	0.084548326099754	0.402427922432375	0.653922248592746\\
0.915	0.0867092760811502	0.4073083028908	0.65239163314657\\
0.915	0.0888984183382709	0.412181886906127	0.650879396495579\\
0.915	0.0911157389373742	0.417048438424897	0.649384574964843\\
0.915	0.0933612213899392	0.421907721096044	0.647906343724685\\
0.915	0.0956348466427212	0.42675949830445	0.646443996754806\\
0.915	0.0979365930683194	0.431603533204733	0.644996929295961\\
0.915	0.100266436456264	0.436439588755293	0.643564622546846\\
0.915	0.102624350004627	0.441267427752584	0.64214663037443\\
0.915	0.105010304312169	0.446086812865616	0.640742567819434\\
0.915	0.107424267371012	0.450897506670668	0.639352101193731\\
0.915	0.109866204559871	0.455699271686218	0.637974939582442\\
0.915	0.112336078637817	0.460491870408058	0.636610827579862\\
0.915	0.114833849738607	0.465275065344603	0.635259539104383\\
0.915	0.117359475365564	0.470048619052377	0.633920872153154\\
0.915	0.119912910387022	0.474812294171664	0.632594644371965\\
0.915	0.122494107032347	0.479565853462319	0.631280689329642\\
0.915	0.125103014888515	0.484309059839721	0.629978853399059\\
0.915	0.127739580897283	0.489041676410868	0.628690460824558\\
0.92	0	0	0.710202759607373\\
0.92	1.11327767495586e-05	0.00471862581271275	0.713003157837853\\
0.92	4.46175251031912e-05	0.009446325184051	0.715846486660142\\
0.92	0.000100583311362513	0.0141829653360114	0.718788838379794\\
0.92	0.000179158434668431	0.018928411755669	0.721886092228356\\
0.92	0.000280470402701511	0.02368252819604	0.725200636220892\\
0.92	0.000404645907256436	0.0284451766772965	0.728799085274723\\
0.92	0.000551810799695644	0.0332162174883388	0.732749123983336\\
0.92	0.000722090066287311	0.037995509188729	0.737115575253327\\
0.92	0.000915607803432999	0.0427829086109896	0.741955893907011\\
0.92	0.0011324871927904	0.0475782708632729	0.747315373652289\\
0.92	0.00137285047629673	0.0523814493324038	0.753222424045811\\
0.92	0.00163681893109844	0.057192295687301	0.759684310525652\\
0.92	0.00192451284439304	0.0620106598827802	0.766683748102665\\
0.92	0.00223605148818899	0.0668363901637434	0.774176695511256\\
0.92	0.00257155309398959	0.0716693330697584	0.782091614494494\\
0.92	0.00293113482740716	0.0765093334400313	0.79033034642109\\
0.92	0.00331491276271365	0.0813562344187764	0.79877062752665\\
0.92	0.00372300185733414	0.0862098774609879	0.807270128930359\\
0.92	0.00415551592628969	0.0910701023386145	0.815671782700193\\
0.92	0.00461256761659621	0.0959367471471425	0.823810053507331\\
0.92	0.00509426838162598	0.100809648312589	0.831517746503354\\
0.92	0.00560072845543873	0.105688640598912	0.838632911419452\\
0.92	0.00613205682708918	0.11057355711583	0.845005411376292\\
0.92	0.00668836121491816	0.115464229327074	0.850502769049302\\
0.92	0.00726974804083431	0.120360487059049	0.855014975717657\\
0.92	0.00787632240459385	0.125262158509929	0.858458041096605\\
0.92	0.00850818805808555	0.13016907025918	0.860776163494183\\
0.92	0.00916544737962849	0.135081047277508	0.861942500814225\\
0.92	0.00984820134829017	0.139997912937243	0.861958614648422\\
0.92	0.0105565495182326	0.144919489023162	0.860852735617091\\
0.92	0.0112905899930939	0.149845595743738	0.858677054109651\\
0.92	0.012050419400414	0.154776051742839	0.855504274967457\\
0.92	0.0128361328661109	0.159710674111862	0.851423687967754\\
0.92	0.0136478239890177	0.164649278402306	0.846537000484341\\
0.92	0.0144855848154856	0.169591678638796	0.840954157904248\\
0.92	0.0153495058140643	0.174537687332543	0.834789345403935\\
0.92	0.0162396758502651	0.179487115495259	0.828157325806199\\
0.92	0.0171561821614173	0.184439772653511	0.821170226448694\\
0.92	0.0180991103316243	0.189395466863524	0.81393484672569\\
0.92	0.0190685442668299	0.194354004726437	0.806550519904213\\
0.92	0.0200645661700016	0.199315191403999	0.799107529846304\\
0.92	0.0210872565164405	0.204278830634725	0.791686056496194\\
0.92	0.0221366940292259	0.20924472475049	0.784355603830602\\
0.92	0.023212955654804	0.214212674693577	0.777174850281118\\
0.92	0.0243161165387281	0.219182480034174	0.770191853862644\\
0.92	0.025446250001561	0.224153938988321	0.76344454155446\\
0.92	0.0266034275149466	0.229126848436298	0.756961413910481\\
0.92	0.0277877186778607	0.234101003941464	0.750762400416766\\
0.92	0.0289991911930496	0.239076199769547	0.744859807805785\\
0.92	0.0302379108436654	0.244052228908366	0.739259311518079\\
0.92	0.0315039414701067	0.24902888308801	0.733960949044505\\
0.92	0.0327973449470747	0.254005952801449	0.72896008240726\\
0.92	0.0341181811608519	0.258983227325591	0.724248305111178\\
0.92	0.0354665079868145	0.263960494742775	0.719814276220682\\
0.92	0.0368423812671862	0.268937541962694	0.71564447061263\\
0.92	0.0382458547890421	0.273914154744766	0.711723839838042\\
0.92	0.0396769802625738	0.278890117720924	0.70803638238973\\
0.92	0.0411358072996216	0.283865214418836	0.704565625567164\\
0.92	0.0426223833924861	0.288839227285554	0.701295023640316\\
0.92	0.0441367538930258	0.293811937711588	0.698208278749354\\
0.92	0.0456789619920503	0.298783126055387	0.695289592055349\\
0.92	0.0472490486990192	0.303752571668251	0.692523853197511\\
0.92	0.0488470528220538	0.308720052919643	0.689896776228119\\
0.92	0.0504730109482718	0.313685347222913	0.687394989990282\\
0.92	0.0521269574244539	0.318648231061428	0.685006090465981\\
0.92	0.0538089243380495	0.323608480015096	0.682718662029128\\
0.92	0.0555189414985328	0.328565868787292	0.680522273853469\\
0.92	0.0572570364191156	0.333520171232163	0.67840745699785\\
0.92	0.0590232342988274	0.338471160382329	0.676365666960043\\
0.92	0.0608175580049697	0.34341860847696	0.674389235783157\\
0.92	0.0626400280559541	0.348362286990222	0.672471317134863\\
0.92	0.064490662604533	0.3533019666601	0.670605827172181\\
0.92	0.0663694774214291	0.358237417517579	0.668787383459803\\
0.92	0.0682764858793752	0.363168408916186	0.667011243730729\\
0.92	0.0702116989375697	0.368094709561873	0.665273245863649\\
0.92	0.0721751251265572	0.373016087543259	0.663569750098933\\
0.92	0.0741667705335422	0.377932310362199	0.66189758422006\\
0.92	0.0761866387881432	0.382843144964686	0.660253992184443\\
0.92	0.0782347310485954	0.38774835777208	0.658636586490998\\
0.92	0.0803110459884098	0.392647714712651	0.65704330441587\\
0.92	0.0824155797834956	0.397540981253432	0.655472368126508\\
0.92	0.084548326099754	0.402427922432375	0.653922248592745\\
0.92	0.0867092760811502	0.407308302890799	0.652391633146571\\
0.92	0.0888984183382709	0.412181886906127	0.65087939649558\\
0.92	0.0911157389373742	0.417048438424896	0.649384574964843\\
0.92	0.0933612213899392	0.421907721096044	0.647906343724685\\
0.92	0.0956348466427212	0.42675949830445	0.646443996754806\\
0.92	0.0979365930683194	0.431603533204733	0.644996929295961\\
0.92	0.100266436456264	0.436439588755293	0.643564622546846\\
0.92	0.102624350004627	0.441267427752584	0.642146630374429\\
0.92	0.105010304312169	0.446086812865616	0.640742567819433\\
0.92	0.107424267371012	0.450897506670668	0.63935210119373\\
0.92	0.109866204559871	0.455699271686218	0.637974939582444\\
0.92	0.112336078637817	0.460491870408058	0.636610827579863\\
0.92	0.114833849738607	0.465275065344603	0.635259539104381\\
0.92	0.117359475365564	0.470048619052377	0.633920872153154\\
0.92	0.119912910387022	0.474812294171664	0.632594644371966\\
0.92	0.122494107032347	0.479565853462319	0.631280689329641\\
0.92	0.125103014888515	0.484309059839721	0.629978853399058\\
0.92	0.127739580897283	0.489041676410868	0.628690460824559\\
0.925	0	0	0.710202759607373\\
0.925	1.11327767495586e-05	0.00471862581271275	0.713003157837853\\
0.925	4.46175251031912e-05	0.009446325184051	0.715846486660142\\
0.925	0.000100583311362513	0.0141829653360114	0.718788838379794\\
0.925	0.000179158434668431	0.018928411755669	0.721886092228356\\
0.925	0.000280470402701511	0.02368252819604	0.725200636220892\\
0.925	0.000404645907256436	0.0284451766772965	0.728799085274723\\
0.925	0.000551810799695644	0.0332162174883389	0.732749123983336\\
0.925	0.000722090066287311	0.037995509188729	0.737115575253327\\
0.925	0.000915607803432999	0.0427829086109896	0.741955893907011\\
0.925	0.0011324871927904	0.0475782708632729	0.747315373652289\\
0.925	0.00137285047629673	0.0523814493324038	0.75322242404581\\
0.925	0.00163681893109844	0.057192295687301	0.759684310525652\\
0.925	0.00192451284439304	0.0620106598827802	0.766683748102665\\
0.925	0.00223605148818899	0.0668363901637434	0.774176695511256\\
0.925	0.00257155309398959	0.0716693330697584	0.782091614494493\\
0.925	0.00293113482740716	0.0765093334400312	0.79033034642109\\
0.925	0.00331491276271365	0.0813562344187764	0.79877062752665\\
0.925	0.00372300185733413	0.0862098774609879	0.807270128930359\\
0.925	0.00415551592628969	0.0910701023386145	0.815671782700194\\
0.925	0.00461256761659621	0.0959367471471425	0.823810053507331\\
0.925	0.00509426838162598	0.100809648312589	0.831517746503353\\
0.925	0.00560072845543873	0.105688640598912	0.838632911419452\\
0.925	0.00613205682708918	0.11057355711583	0.845005411376293\\
0.925	0.00668836121491815	0.115464229327074	0.850502769049302\\
0.925	0.00726974804083431	0.120360487059049	0.855014975717657\\
0.925	0.00787632240459385	0.125262158509929	0.858458041096605\\
0.925	0.00850818805808555	0.13016907025918	0.860776163494182\\
0.925	0.00916544737962849	0.135081047277508	0.861942500814224\\
0.925	0.00984820134829017	0.139997912937243	0.861958614648421\\
0.925	0.0105565495182326	0.144919489023162	0.860852735617091\\
0.925	0.0112905899930939	0.149845595743738	0.858677054109652\\
0.925	0.012050419400414	0.154776051742839	0.855504274967458\\
0.925	0.0128361328661109	0.159710674111862	0.851423687967754\\
0.925	0.0136478239890177	0.164649278402306	0.84653700048434\\
0.925	0.0144855848154856	0.169591678638796	0.840954157904248\\
0.925	0.0153495058140643	0.174537687332543	0.834789345403935\\
0.925	0.0162396758502651	0.179487115495259	0.8281573258062\\
0.925	0.0171561821614173	0.184439772653511	0.821170226448693\\
0.925	0.0180991103316243	0.189395466863524	0.813934846725691\\
0.925	0.0190685442668299	0.194354004726437	0.806550519904212\\
0.925	0.0200645661700016	0.199315191403999	0.799107529846304\\
0.925	0.0210872565164405	0.204278830634725	0.791686056496193\\
0.925	0.0221366940292259	0.20924472475049	0.784355603830603\\
0.925	0.023212955654804	0.214212674693577	0.777174850281119\\
0.925	0.0243161165387281	0.219182480034174	0.770191853862645\\
0.925	0.025446250001561	0.224153938988321	0.763444541554458\\
0.925	0.0266034275149466	0.229126848436298	0.756961413910482\\
0.925	0.0277877186778607	0.234101003941464	0.750762400416766\\
0.925	0.0289991911930496	0.239076199769547	0.744859807805786\\
0.925	0.0302379108436654	0.244052228908366	0.739259311518079\\
0.925	0.0315039414701067	0.24902888308801	0.733960949044503\\
0.925	0.0327973449470747	0.254005952801449	0.728960082407261\\
0.925	0.0341181811608519	0.258983227325592	0.724248305111178\\
0.925	0.0354665079868145	0.263960494742775	0.719814276220681\\
0.925	0.0368423812671862	0.268937541962694	0.71564447061263\\
0.925	0.0382458547890421	0.273914154744766	0.71172383983804\\
0.925	0.0396769802625738	0.278890117720924	0.70803638238973\\
0.925	0.0411358072996216	0.283865214418836	0.704565625567166\\
0.925	0.0426223833924861	0.288839227285554	0.701295023640316\\
0.925	0.0441367538930258	0.293811937711588	0.698208278749353\\
0.925	0.0456789619920503	0.298783126055387	0.695289592055347\\
0.925	0.0472490486990192	0.303752571668251	0.692523853197512\\
0.925	0.0488470528220538	0.308720052919643	0.68989677622812\\
0.925	0.0504730109482718	0.313685347222913	0.687394989990281\\
0.925	0.0521269574244539	0.318648231061427	0.685006090465982\\
0.925	0.0538089243380495	0.323608480015097	0.682718662029129\\
0.925	0.0555189414985328	0.328565868787292	0.680522273853468\\
0.925	0.0572570364191156	0.333520171232163	0.67840745699785\\
0.925	0.0590232342988274	0.338471160382329	0.676365666960043\\
0.925	0.0608175580049697	0.34341860847696	0.674389235783156\\
0.925	0.0626400280559541	0.348362286990222	0.672471317134862\\
0.925	0.064490662604533	0.3533019666601	0.670605827172183\\
0.925	0.0663694774214291	0.358237417517579	0.668787383459803\\
0.925	0.0682764858793752	0.363168408916186	0.667011243730728\\
0.925	0.0702116989375697	0.368094709561873	0.665273245863649\\
0.925	0.0721751251265572	0.373016087543259	0.663569750098932\\
0.925	0.0741667705335422	0.377932310362199	0.661897584220061\\
0.925	0.0761866387881432	0.382843144964686	0.660253992184443\\
0.925	0.0782347310485954	0.38774835777208	0.658636586490998\\
0.925	0.0803110459884098	0.392647714712651	0.657043304415871\\
0.925	0.0824155797834956	0.397540981253432	0.655472368126509\\
0.925	0.084548326099754	0.402427922432375	0.653922248592745\\
0.925	0.0867092760811502	0.407308302890799	0.652391633146569\\
0.925	0.0888984183382708	0.412181886906127	0.65087939649558\\
0.925	0.0911157389373742	0.417048438424896	0.649384574964844\\
0.925	0.0933612213899392	0.421907721096044	0.647906343724684\\
0.925	0.0956348466427212	0.42675949830445	0.646443996754806\\
0.925	0.0979365930683194	0.431603533204733	0.644996929295962\\
0.925	0.100266436456264	0.436439588755293	0.643564622546847\\
0.925	0.102624350004627	0.441267427752584	0.64214663037443\\
0.925	0.105010304312169	0.446086812865616	0.640742567819433\\
0.925	0.107424267371012	0.450897506670668	0.639352101193731\\
0.925	0.109866204559871	0.455699271686218	0.637974939582442\\
0.925	0.112336078637817	0.460491870408058	0.636610827579862\\
0.925	0.114833849738607	0.465275065344603	0.635259539104383\\
0.925	0.117359475365564	0.470048619052377	0.633920872153154\\
0.925	0.119912910387022	0.474812294171664	0.632594644371966\\
0.925	0.122494107032347	0.479565853462319	0.631280689329641\\
0.925	0.125103014888515	0.484309059839721	0.629978853399059\\
0.925	0.127739580897283	0.489041676410868	0.628690460824558\\
0.93	0	0	0.710202759607373\\
0.93	1.11327767495586e-05	0.00471862581271275	0.713003157837853\\
0.93	4.46175251031912e-05	0.009446325184051	0.715846486660142\\
0.93	0.000100583311362513	0.0141829653360114	0.718788838379794\\
0.93	0.000179158434668431	0.018928411755669	0.721886092228356\\
0.93	0.000280470402701511	0.02368252819604	0.725200636220892\\
0.93	0.000404645907256436	0.0284451766772965	0.728799085274723\\
0.93	0.000551810799695644	0.0332162174883388	0.732749123983336\\
0.93	0.000722090066287311	0.037995509188729	0.737115575253327\\
0.93	0.000915607803432999	0.0427829086109896	0.741955893907011\\
0.93	0.0011324871927904	0.0475782708632729	0.747315373652289\\
0.93	0.00137285047629673	0.0523814493324038	0.75322242404581\\
0.93	0.00163681893109844	0.057192295687301	0.759684310525653\\
0.93	0.00192451284439304	0.0620106598827802	0.766683748102665\\
0.93	0.00223605148818899	0.0668363901637434	0.774176695511256\\
0.93	0.00257155309398959	0.0716693330697584	0.782091614494493\\
0.93	0.00293113482740716	0.0765093334400313	0.79033034642109\\
0.93	0.00331491276271365	0.0813562344187764	0.798770627526651\\
0.93	0.00372300185733413	0.0862098774609879	0.807270128930359\\
0.93	0.00415551592628969	0.0910701023386145	0.815671782700193\\
0.93	0.00461256761659621	0.0959367471471425	0.82381005350733\\
0.93	0.00509426838162598	0.100809648312589	0.831517746503354\\
0.93	0.00560072845543873	0.105688640598912	0.838632911419451\\
0.93	0.00613205682708918	0.11057355711583	0.845005411376292\\
0.93	0.00668836121491815	0.115464229327074	0.850502769049302\\
0.93	0.00726974804083431	0.120360487059049	0.855014975717656\\
0.93	0.00787632240459385	0.125262158509929	0.858458041096605\\
0.93	0.00850818805808555	0.13016907025918	0.860776163494182\\
0.93	0.00916544737962849	0.135081047277508	0.861942500814224\\
0.93	0.00984820134829017	0.139997912937243	0.86195861464842\\
0.93	0.0105565495182326	0.144919489023162	0.86085273561709\\
0.93	0.0112905899930939	0.149845595743738	0.858677054109649\\
0.93	0.012050419400414	0.154776051742839	0.855504274967458\\
0.93	0.0128361328661109	0.159710674111862	0.851423687967756\\
0.93	0.0136478239890177	0.164649278402306	0.846537000484339\\
0.93	0.0144855848154856	0.169591678638796	0.84095415790425\\
0.93	0.0153495058140643	0.174537687332543	0.834789345403935\\
0.93	0.0162396758502651	0.179487115495259	0.8281573258062\\
0.93	0.0171561821614173	0.184439772653511	0.821170226448694\\
0.93	0.0180991103316243	0.189395466863524	0.813934846725693\\
0.93	0.0190685442668299	0.194354004726437	0.806550519904213\\
0.93	0.0200645661700016	0.199315191403999	0.799107529846304\\
0.93	0.0210872565164405	0.204278830634725	0.791686056496194\\
0.93	0.0221366940292259	0.20924472475049	0.784355603830601\\
0.93	0.023212955654804	0.214212674693577	0.777174850281118\\
0.93	0.0243161165387281	0.219182480034174	0.770191853862643\\
0.93	0.025446250001561	0.224153938988321	0.763444541554458\\
0.93	0.0266034275149466	0.229126848436298	0.756961413910484\\
0.93	0.0277877186778607	0.234101003941464	0.750762400416766\\
0.93	0.0289991911930496	0.239076199769547	0.744859807805784\\
0.93	0.0302379108436654	0.244052228908366	0.739259311518078\\
0.93	0.0315039414701067	0.24902888308801	0.733960949044504\\
0.93	0.0327973449470747	0.254005952801449	0.72896008240726\\
0.93	0.0341181811608519	0.258983227325592	0.724248305111178\\
0.93	0.0354665079868145	0.263960494742775	0.719814276220681\\
0.93	0.0368423812671862	0.268937541962694	0.715644470612629\\
0.93	0.0382458547890421	0.273914154744766	0.711723839838042\\
0.93	0.0396769802625738	0.278890117720924	0.708036382389731\\
0.93	0.0411358072996216	0.283865214418836	0.704565625567164\\
0.93	0.0426223833924861	0.288839227285554	0.701295023640317\\
0.93	0.0441367538930258	0.293811937711588	0.698208278749354\\
0.93	0.0456789619920503	0.298783126055387	0.695289592055349\\
0.93	0.0472490486990192	0.303752571668251	0.692523853197511\\
0.93	0.0488470528220538	0.308720052919643	0.689896776228119\\
0.93	0.0504730109482718	0.313685347222913	0.687394989990281\\
0.93	0.0521269574244539	0.318648231061427	0.685006090465979\\
0.93	0.0538089243380495	0.323608480015096	0.682718662029129\\
0.929999999999999	0.0555189414985328	0.328565868787292	0.680522273853471\\
0.93	0.0572570364191156	0.333520171232163	0.67840745699785\\
0.93	0.0590232342988274	0.338471160382329	0.676365666960043\\
0.93	0.0608175580049697	0.34341860847696	0.674389235783155\\
0.929999999999999	0.0626400280559541	0.348362286990221	0.672471317134862\\
0.93	0.064490662604533	0.3533019666601	0.670605827172184\\
0.93	0.0663694774214291	0.358237417517579	0.668787383459804\\
0.93	0.0682764858793752	0.363168408916186	0.667011243730728\\
0.93	0.0702116989375697	0.368094709561873	0.665273245863648\\
0.93	0.0721751251265572	0.373016087543259	0.663569750098931\\
0.93	0.0741667705335422	0.377932310362199	0.661897584220059\\
0.93	0.0761866387881432	0.382843144964686	0.660253992184443\\
0.93	0.0782347310485954	0.38774835777208	0.658636586490998\\
0.93	0.0803110459884098	0.392647714712651	0.657043304415871\\
0.93	0.0824155797834956	0.397540981253432	0.65547236812651\\
0.93	0.084548326099754	0.402427922432375	0.653922248592745\\
0.93	0.0867092760811502	0.407308302890799	0.652391633146568\\
0.93	0.0888984183382709	0.412181886906127	0.650879396495579\\
0.93	0.0911157389373742	0.417048438424896	0.649384574964843\\
0.93	0.0933612213899392	0.421907721096044	0.647906343724686\\
0.93	0.0956348466427212	0.42675949830445	0.646443996754806\\
0.93	0.0979365930683194	0.431603533204733	0.64499692929596\\
0.93	0.100266436456264	0.436439588755293	0.643564622546847\\
0.93	0.102624350004627	0.441267427752584	0.64214663037443\\
0.93	0.105010304312169	0.446086812865616	0.640742567819433\\
0.93	0.107424267371012	0.450897506670668	0.639352101193731\\
0.929999999999999	0.109866204559871	0.455699271686218	0.637974939582444\\
0.93	0.112336078637817	0.460491870408058	0.636610827579862\\
0.93	0.114833849738607	0.465275065344603	0.635259539104383\\
0.93	0.117359475365564	0.470048619052377	0.633920872153153\\
0.93	0.119912910387022	0.474812294171664	0.632594644371966\\
0.93	0.122494107032347	0.479565853462319	0.631280689329642\\
0.93	0.125103014888515	0.484309059839721	0.629978853399059\\
0.93	0.127739580897283	0.489041676410868	0.628690460824561\\
0.935	0	0	0.710202759607373\\
0.935	1.11327767495586e-05	0.00471862581271274	0.713003157837853\\
0.935	4.46175251031912e-05	0.009446325184051	0.715846486660142\\
0.935	0.000100583311362513	0.0141829653360114	0.718788838379794\\
0.935	0.000179158434668431	0.018928411755669	0.721886092228356\\
0.935	0.000280470402701511	0.02368252819604	0.725200636220892\\
0.935	0.000404645907256436	0.0284451766772965	0.728799085274723\\
0.935	0.000551810799695644	0.0332162174883388	0.732749123983336\\
0.935	0.000722090066287311	0.037995509188729	0.737115575253327\\
0.935	0.000915607803432999	0.0427829086109896	0.741955893907011\\
0.935	0.0011324871927904	0.0475782708632729	0.747315373652289\\
0.935	0.00137285047629673	0.0523814493324038	0.75322242404581\\
0.935	0.00163681893109844	0.057192295687301	0.759684310525652\\
0.935	0.00192451284439304	0.0620106598827802	0.766683748102665\\
0.935	0.00223605148818898	0.0668363901637434	0.774176695511256\\
0.935	0.00257155309398959	0.0716693330697584	0.782091614494493\\
0.935	0.00293113482740716	0.0765093334400312	0.790330346421089\\
0.935	0.00331491276271365	0.0813562344187764	0.79877062752665\\
0.935	0.00372300185733413	0.0862098774609879	0.80727012893036\\
0.935	0.00415551592628969	0.0910701023386145	0.815671782700193\\
0.935	0.00461256761659621	0.0959367471471425	0.82381005350733\\
0.935	0.00509426838162598	0.100809648312589	0.831517746503355\\
0.935	0.00560072845543873	0.105688640598912	0.838632911419452\\
0.935	0.00613205682708918	0.11057355711583	0.845005411376293\\
0.935	0.00668836121491815	0.115464229327074	0.850502769049302\\
0.935	0.00726974804083431	0.120360487059049	0.855014975717657\\
0.935	0.00787632240459385	0.125262158509929	0.858458041096605\\
0.935	0.00850818805808555	0.13016907025918	0.860776163494182\\
0.935	0.00916544737962849	0.135081047277508	0.861942500814224\\
0.935	0.00984820134829017	0.139997912937243	0.86195861464842\\
0.935	0.0105565495182326	0.144919489023162	0.860852735617091\\
0.935	0.0112905899930939	0.149845595743738	0.858677054109651\\
0.935	0.012050419400414	0.154776051742839	0.855504274967458\\
0.935	0.0128361328661109	0.159710674111862	0.851423687967755\\
0.935	0.0136478239890177	0.164649278402306	0.84653700048434\\
0.935	0.0144855848154856	0.169591678638796	0.840954157904249\\
0.935	0.0153495058140643	0.174537687332543	0.834789345403935\\
0.935	0.0162396758502651	0.179487115495259	0.8281573258062\\
0.935	0.0171561821614173	0.184439772653511	0.821170226448695\\
0.935	0.0180991103316243	0.189395466863524	0.813934846725692\\
0.935	0.0190685442668299	0.194354004726437	0.806550519904211\\
0.935	0.0200645661700016	0.199315191403999	0.799107529846305\\
0.935	0.0210872565164405	0.204278830634725	0.791686056496194\\
0.935	0.0221366940292259	0.20924472475049	0.784355603830601\\
0.935	0.023212955654804	0.214212674693577	0.777174850281119\\
0.935	0.0243161165387281	0.219182480034174	0.770191853862644\\
0.935	0.025446250001561	0.224153938988321	0.76344454155446\\
0.935	0.0266034275149466	0.229126848436298	0.756961413910483\\
0.935	0.0277877186778607	0.234101003941464	0.750762400416763\\
0.935	0.0289991911930496	0.239076199769547	0.744859807805784\\
0.935	0.0302379108436654	0.244052228908366	0.739259311518081\\
0.935	0.0315039414701067	0.24902888308801	0.733960949044503\\
0.935	0.0327973449470747	0.254005952801449	0.72896008240726\\
0.935	0.0341181811608519	0.258983227325592	0.72424830511118\\
0.935	0.0354665079868145	0.263960494742775	0.71981427622068\\
0.935	0.0368423812671862	0.268937541962694	0.715644470612628\\
0.935	0.0382458547890421	0.273914154744766	0.711723839838043\\
0.935	0.0396769802625738	0.278890117720924	0.708036382389731\\
0.935	0.0411358072996216	0.283865214418836	0.704565625567163\\
0.935	0.0426223833924861	0.288839227285554	0.701295023640316\\
0.935	0.0441367538930258	0.293811937711588	0.698208278749353\\
0.935	0.0456789619920503	0.298783126055387	0.695289592055349\\
0.935	0.0472490486990192	0.303752571668251	0.692523853197512\\
0.935	0.0488470528220538	0.308720052919643	0.689896776228119\\
0.935	0.0504730109482718	0.313685347222913	0.687394989990282\\
0.935	0.0521269574244539	0.318648231061428	0.685006090465981\\
0.935	0.0538089243380495	0.323608480015096	0.682718662029127\\
0.935	0.0555189414985328	0.328565868787292	0.68052227385347\\
0.935	0.0572570364191156	0.333520171232163	0.678407456997851\\
0.935	0.0590232342988274	0.338471160382329	0.676365666960043\\
0.935	0.0608175580049697	0.34341860847696	0.674389235783156\\
0.935	0.0626400280559541	0.348362286990222	0.672471317134861\\
0.935	0.064490662604533	0.3533019666601	0.670605827172182\\
0.935	0.0663694774214291	0.358237417517579	0.668787383459804\\
0.935	0.0682764858793752	0.363168408916186	0.667011243730728\\
0.935	0.0702116989375697	0.368094709561873	0.665273245863649\\
0.935	0.0721751251265572	0.373016087543259	0.663569750098934\\
0.935	0.0741667705335423	0.377932310362199	0.661897584220061\\
0.935	0.0761866387881432	0.382843144964686	0.66025399218444\\
0.935	0.0782347310485954	0.38774835777208	0.658636586490996\\
0.935	0.0803110459884098	0.392647714712651	0.657043304415871\\
0.935	0.0824155797834956	0.397540981253432	0.65547236812651\\
0.935	0.084548326099754	0.402427922432375	0.653922248592747\\
0.935	0.0867092760811502	0.4073083028908	0.652391633146569\\
0.935	0.0888984183382709	0.412181886906127	0.650879396495578\\
0.935	0.0911157389373742	0.417048438424897	0.649384574964843\\
0.935	0.0933612213899392	0.421907721096044	0.647906343724684\\
0.935	0.0956348466427212	0.42675949830445	0.646443996754807\\
0.935	0.0979365930683194	0.431603533204733	0.644996929295962\\
0.935	0.100266436456264	0.436439588755293	0.643564622546846\\
0.935	0.102624350004627	0.441267427752585	0.64214663037443\\
0.935	0.105010304312169	0.446086812865616	0.640742567819433\\
0.935	0.107424267371012	0.450897506670668	0.639352101193729\\
0.935	0.109866204559871	0.455699271686218	0.637974939582444\\
0.935	0.112336078637817	0.460491870408058	0.636610827579863\\
0.935	0.114833849738607	0.465275065344603	0.635259539104382\\
0.935	0.117359475365564	0.470048619052377	0.633920872153154\\
0.935	0.119912910387023	0.474812294171664	0.632594644371965\\
0.935	0.122494107032347	0.479565853462319	0.631280689329642\\
0.935	0.125103014888515	0.484309059839721	0.629978853399058\\
0.935	0.127739580897283	0.489041676410868	0.62869046082456\\
0.94	0	0	0.710202759607373\\
0.94	1.11327767495586e-05	0.00471862581271274	0.713003157837853\\
0.94	4.46175251031912e-05	0.009446325184051	0.715846486660142\\
0.94	0.000100583311362513	0.0141829653360114	0.718788838379794\\
0.94	0.000179158434668431	0.018928411755669	0.721886092228356\\
0.94	0.000280470402701511	0.02368252819604	0.725200636220892\\
0.94	0.000404645907256436	0.0284451766772965	0.728799085274723\\
0.94	0.000551810799695644	0.0332162174883389	0.732749123983336\\
0.94	0.000722090066287311	0.037995509188729	0.737115575253328\\
0.94	0.000915607803432999	0.0427829086109896	0.741955893907011\\
0.94	0.0011324871927904	0.0475782708632729	0.747315373652289\\
0.94	0.00137285047629673	0.0523814493324038	0.753222424045811\\
0.94	0.00163681893109844	0.057192295687301	0.759684310525652\\
0.94	0.00192451284439304	0.0620106598827802	0.766683748102665\\
0.94	0.00223605148818899	0.0668363901637434	0.774176695511257\\
0.94	0.00257155309398959	0.0716693330697584	0.782091614494494\\
0.94	0.00293113482740716	0.0765093334400312	0.79033034642109\\
0.94	0.00331491276271365	0.0813562344187764	0.79877062752665\\
0.94	0.00372300185733413	0.0862098774609879	0.807270128930359\\
0.94	0.00415551592628969	0.0910701023386145	0.815671782700194\\
0.94	0.00461256761659621	0.0959367471471425	0.823810053507331\\
0.94	0.00509426838162598	0.100809648312589	0.831517746503354\\
0.94	0.00560072845543873	0.105688640598912	0.838632911419452\\
0.94	0.00613205682708918	0.11057355711583	0.845005411376294\\
0.94	0.00668836121491815	0.115464229327074	0.850502769049302\\
0.94	0.00726974804083431	0.120360487059049	0.855014975717657\\
0.94	0.00787632240459385	0.125262158509929	0.858458041096607\\
0.94	0.00850818805808555	0.13016907025918	0.860776163494182\\
0.94	0.00916544737962849	0.135081047277508	0.861942500814224\\
0.94	0.00984820134829017	0.139997912937243	0.86195861464842\\
0.94	0.0105565495182326	0.144919489023162	0.860852735617091\\
0.94	0.0112905899930939	0.149845595743738	0.858677054109652\\
0.94	0.012050419400414	0.154776051742839	0.855504274967457\\
0.94	0.0128361328661109	0.159710674111862	0.851423687967752\\
0.94	0.0136478239890177	0.164649278402306	0.846537000484343\\
0.94	0.0144855848154856	0.169591678638796	0.840954157904246\\
0.94	0.0153495058140643	0.174537687332543	0.834789345403937\\
0.94	0.0162396758502651	0.179487115495259	0.828157325806198\\
0.94	0.0171561821614173	0.184439772653511	0.821170226448693\\
0.94	0.0180991103316243	0.189395466863524	0.81393484672569\\
0.94	0.0190685442668299	0.194354004726437	0.806550519904213\\
0.94	0.0200645661700016	0.199315191403999	0.799107529846305\\
0.94	0.0210872565164405	0.204278830634725	0.791686056496192\\
0.94	0.0221366940292259	0.20924472475049	0.784355603830601\\
0.94	0.023212955654804	0.214212674693577	0.777174850281117\\
0.94	0.0243161165387281	0.219182480034174	0.770191853862643\\
0.94	0.025446250001561	0.224153938988321	0.763444541554458\\
0.94	0.0266034275149466	0.229126848436298	0.756961413910482\\
0.94	0.0277877186778607	0.234101003941464	0.750762400416762\\
0.94	0.0289991911930496	0.239076199769547	0.744859807805786\\
0.94	0.0302379108436654	0.244052228908366	0.73925931151808\\
0.94	0.0315039414701067	0.24902888308801	0.733960949044502\\
0.94	0.0327973449470747	0.254005952801449	0.728960082407261\\
0.94	0.0341181811608519	0.258983227325592	0.724248305111178\\
0.94	0.0354665079868145	0.263960494742775	0.71981427622068\\
0.94	0.0368423812671862	0.268937541962694	0.715644470612629\\
0.94	0.0382458547890421	0.273914154744766	0.711723839838042\\
0.94	0.0396769802625738	0.278890117720924	0.70803638238973\\
0.94	0.0411358072996216	0.283865214418836	0.704565625567164\\
0.94	0.0426223833924861	0.288839227285554	0.701295023640316\\
0.94	0.0441367538930258	0.293811937711588	0.698208278749354\\
0.94	0.0456789619920503	0.298783126055387	0.695289592055349\\
0.94	0.0472490486990192	0.303752571668251	0.692523853197512\\
0.94	0.0488470528220538	0.308720052919643	0.689896776228119\\
0.94	0.0504730109482718	0.313685347222913	0.68739498999028\\
0.94	0.0521269574244539	0.318648231061427	0.685006090465981\\
0.94	0.0538089243380495	0.323608480015096	0.682718662029128\\
0.94	0.0555189414985328	0.328565868787292	0.680522273853469\\
0.94	0.0572570364191156	0.333520171232163	0.67840745699785\\
0.94	0.0590232342988274	0.338471160382329	0.676365666960043\\
0.94	0.0608175580049697	0.34341860847696	0.674389235783155\\
0.94	0.0626400280559541	0.348362286990222	0.672471317134862\\
0.94	0.064490662604533	0.3533019666601	0.670605827172183\\
0.94	0.0663694774214291	0.358237417517579	0.668787383459804\\
0.94	0.0682764858793752	0.363168408916186	0.667011243730728\\
0.94	0.0702116989375697	0.368094709561873	0.665273245863648\\
0.94	0.0721751251265572	0.373016087543259	0.663569750098931\\
0.94	0.0741667705335422	0.377932310362199	0.661897584220062\\
0.94	0.0761866387881432	0.382843144964686	0.660253992184443\\
0.94	0.0782347310485954	0.38774835777208	0.658636586490996\\
0.94	0.0803110459884098	0.392647714712651	0.657043304415872\\
0.94	0.0824155797834956	0.397540981253432	0.65547236812651\\
0.94	0.084548326099754	0.402427922432375	0.653922248592745\\
0.94	0.0867092760811502	0.407308302890799	0.652391633146571\\
0.94	0.0888984183382708	0.412181886906127	0.650879396495579\\
0.94	0.0911157389373742	0.417048438424896	0.649384574964844\\
0.94	0.0933612213899392	0.421907721096044	0.647906343724685\\
0.94	0.0956348466427212	0.42675949830445	0.646443996754806\\
0.94	0.0979365930683194	0.431603533204733	0.644996929295961\\
0.94	0.100266436456264	0.436439588755293	0.643564622546847\\
0.94	0.102624350004627	0.441267427752585	0.642146630374431\\
0.94	0.105010304312169	0.446086812865616	0.640742567819433\\
0.94	0.107424267371012	0.450897506670668	0.639352101193731\\
0.94	0.109866204559871	0.455699271686218	0.637974939582443\\
0.94	0.112336078637817	0.460491870408058	0.636610827579863\\
0.94	0.114833849738607	0.465275065344603	0.635259539104383\\
0.94	0.117359475365564	0.470048619052377	0.633920872153154\\
0.94	0.119912910387022	0.474812294171664	0.632594644371965\\
0.94	0.122494107032347	0.479565853462319	0.63128068932964\\
0.94	0.125103014888515	0.484309059839721	0.629978853399058\\
0.94	0.127739580897283	0.489041676410868	0.628690460824559\\
0.945	0	0	0.710202759607373\\
0.945	1.11327767495586e-05	0.00471862581271275	0.713003157837853\\
0.945	4.46175251031912e-05	0.009446325184051	0.715846486660142\\
0.945	0.000100583311362513	0.0141829653360114	0.718788838379794\\
0.945	0.000179158434668431	0.018928411755669	0.721886092228356\\
0.945	0.000280470402701511	0.02368252819604	0.725200636220892\\
0.945	0.000404645907256436	0.0284451766772965	0.728799085274723\\
0.945	0.000551810799695644	0.0332162174883388	0.732749123983336\\
0.945	0.000722090066287311	0.037995509188729	0.737115575253327\\
0.945	0.000915607803432999	0.0427829086109896	0.741955893907011\\
0.945	0.0011324871927904	0.0475782708632729	0.747315373652289\\
0.945	0.00137285047629673	0.0523814493324038	0.753222424045811\\
0.945	0.00163681893109844	0.057192295687301	0.759684310525652\\
0.945	0.00192451284439304	0.0620106598827802	0.766683748102665\\
0.945	0.00223605148818899	0.0668363901637434	0.774176695511256\\
0.945	0.00257155309398959	0.0716693330697584	0.782091614494494\\
0.945	0.00293113482740716	0.0765093334400312	0.79033034642109\\
0.945	0.00331491276271365	0.0813562344187764	0.79877062752665\\
0.945	0.00372300185733413	0.0862098774609879	0.807270128930359\\
0.945	0.00415551592628969	0.0910701023386145	0.815671782700193\\
0.945	0.00461256761659621	0.0959367471471425	0.823810053507331\\
0.945	0.00509426838162598	0.100809648312589	0.831517746503354\\
0.945	0.00560072845543873	0.105688640598912	0.838632911419452\\
0.945	0.00613205682708918	0.11057355711583	0.845005411376293\\
0.945	0.00668836121491816	0.115464229327074	0.850502769049302\\
0.945	0.00726974804083431	0.120360487059049	0.855014975717656\\
0.945	0.00787632240459385	0.125262158509929	0.858458041096606\\
0.945	0.00850818805808555	0.13016907025918	0.860776163494181\\
0.945	0.00916544737962849	0.135081047277508	0.861942500814225\\
0.945	0.00984820134829017	0.139997912937243	0.861958614648422\\
0.945	0.0105565495182326	0.144919489023162	0.860852735617092\\
0.945	0.0112905899930939	0.149845595743738	0.858677054109652\\
0.945	0.012050419400414	0.154776051742839	0.855504274967456\\
0.945	0.0128361328661109	0.159710674111862	0.851423687967755\\
0.945	0.0136478239890177	0.164649278402306	0.84653700048434\\
0.945	0.0144855848154856	0.169591678638796	0.840954157904247\\
0.945	0.0153495058140643	0.174537687332543	0.834789345403935\\
0.945	0.0162396758502651	0.179487115495259	0.8281573258062\\
0.945	0.0171561821614173	0.184439772653511	0.821170226448694\\
0.945	0.0180991103316243	0.189395466863524	0.813934846725692\\
0.945	0.0190685442668299	0.194354004726437	0.806550519904214\\
0.945	0.0200645661700016	0.199315191403999	0.799107529846302\\
0.945	0.0210872565164405	0.204278830634725	0.791686056496193\\
0.945	0.0221366940292259	0.20924472475049	0.784355603830601\\
0.945	0.023212955654804	0.214212674693577	0.777174850281118\\
0.945	0.0243161165387281	0.219182480034174	0.770191853862644\\
0.945	0.025446250001561	0.224153938988321	0.763444541554461\\
0.945	0.0266034275149466	0.229126848436298	0.756961413910482\\
0.945	0.0277877186778607	0.234101003941464	0.750762400416764\\
0.945	0.0289991911930496	0.239076199769547	0.744859807805786\\
0.945	0.0302379108436654	0.244052228908366	0.73925931151808\\
0.945	0.0315039414701067	0.24902888308801	0.733960949044503\\
0.945	0.0327973449470747	0.254005952801449	0.72896008240726\\
0.945	0.0341181811608519	0.258983227325591	0.724248305111179\\
0.945	0.0354665079868145	0.263960494742775	0.719814276220681\\
0.945	0.0368423812671862	0.268937541962694	0.715644470612629\\
0.945	0.0382458547890421	0.273914154744766	0.711723839838043\\
0.945	0.0396769802625738	0.278890117720924	0.70803638238973\\
0.945	0.0411358072996216	0.283865214418836	0.704565625567163\\
0.945	0.0426223833924861	0.288839227285554	0.701295023640316\\
0.945	0.0441367538930258	0.293811937711588	0.698208278749354\\
0.945	0.0456789619920503	0.298783126055387	0.69528959205535\\
0.945	0.0472490486990192	0.303752571668251	0.692523853197511\\
0.945	0.0488470528220538	0.308720052919643	0.689896776228118\\
0.945	0.0504730109482718	0.313685347222913	0.687394989990282\\
0.945	0.0521269574244539	0.318648231061428	0.685006090465981\\
0.945	0.0538089243380495	0.323608480015096	0.682718662029127\\
0.945	0.0555189414985328	0.328565868787292	0.68052227385347\\
0.945	0.0572570364191156	0.333520171232163	0.678407456997851\\
0.945	0.0590232342988274	0.338471160382329	0.676365666960043\\
0.945	0.0608175580049697	0.34341860847696	0.674389235783155\\
0.945	0.0626400280559541	0.348362286990222	0.672471317134862\\
0.945	0.064490662604533	0.3533019666601	0.670605827172183\\
0.945	0.066369477421429	0.358237417517579	0.668787383459804\\
0.945	0.0682764858793752	0.363168408916186	0.667011243730727\\
0.945	0.0702116989375697	0.368094709561873	0.665273245863649\\
0.945	0.0721751251265572	0.373016087543259	0.663569750098932\\
0.945	0.0741667705335422	0.377932310362199	0.66189758422006\\
0.945	0.0761866387881432	0.382843144964686	0.660253992184444\\
0.945	0.0782347310485954	0.38774835777208	0.658636586490997\\
0.945	0.0803110459884098	0.392647714712651	0.65704330441587\\
0.945	0.0824155797834956	0.397540981253432	0.655472368126509\\
0.945	0.084548326099754	0.402427922432375	0.653922248592745\\
0.945	0.0867092760811502	0.407308302890799	0.65239163314657\\
0.945	0.0888984183382709	0.412181886906127	0.650879396495579\\
0.945	0.0911157389373742	0.417048438424896	0.649384574964844\\
0.945	0.0933612213899392	0.421907721096044	0.647906343724685\\
0.945	0.0956348466427212	0.42675949830445	0.646443996754806\\
0.945	0.0979365930683194	0.431603533204733	0.644996929295961\\
0.945	0.100266436456264	0.436439588755293	0.643564622546846\\
0.945	0.102624350004627	0.441267427752584	0.642146630374431\\
0.945	0.105010304312169	0.446086812865616	0.640742567819433\\
0.945	0.107424267371012	0.450897506670668	0.63935210119373\\
0.945	0.109866204559871	0.455699271686218	0.637974939582443\\
0.945	0.112336078637817	0.460491870408058	0.636610827579862\\
0.945	0.114833849738607	0.465275065344603	0.635259539104383\\
0.945	0.117359475365564	0.470048619052377	0.633920872153154\\
0.945	0.119912910387022	0.474812294171664	0.632594644371966\\
0.945	0.122494107032347	0.479565853462319	0.631280689329643\\
0.945	0.125103014888515	0.484309059839721	0.629978853399058\\
0.945	0.127739580897283	0.489041676410868	0.628690460824556\\
0.95	0	0	0.710202759607373\\
0.95	1.11327767495586e-05	0.00471862581271274	0.713003157837853\\
0.95	4.46175251031912e-05	0.009446325184051	0.715846486660142\\
0.95	0.000100583311362513	0.0141829653360114	0.718788838379794\\
0.95	0.000179158434668431	0.018928411755669	0.721886092228356\\
0.95	0.000280470402701511	0.02368252819604	0.725200636220892\\
0.95	0.000404645907256436	0.0284451766772965	0.728799085274723\\
0.95	0.000551810799695644	0.0332162174883389	0.732749123983336\\
0.95	0.000722090066287311	0.037995509188729	0.737115575253328\\
0.95	0.000915607803432999	0.0427829086109896	0.741955893907011\\
0.95	0.0011324871927904	0.0475782708632729	0.747315373652289\\
0.95	0.00137285047629673	0.0523814493324038	0.753222424045811\\
0.95	0.00163681893109844	0.057192295687301	0.759684310525652\\
0.95	0.00192451284439304	0.0620106598827802	0.766683748102665\\
0.95	0.00223605148818898	0.0668363901637434	0.774176695511256\\
0.95	0.00257155309398959	0.0716693330697584	0.782091614494493\\
0.95	0.00293113482740716	0.0765093334400312	0.79033034642109\\
0.95	0.00331491276271365	0.0813562344187764	0.79877062752665\\
0.95	0.00372300185733413	0.0862098774609879	0.807270128930359\\
0.95	0.00415551592628969	0.0910701023386145	0.815671782700193\\
0.95	0.00461256761659621	0.0959367471471425	0.823810053507331\\
0.95	0.00509426838162598	0.100809648312589	0.831517746503353\\
0.95	0.00560072845543873	0.105688640598912	0.838632911419452\\
0.95	0.00613205682708918	0.11057355711583	0.845005411376293\\
0.95	0.00668836121491815	0.115464229327074	0.850502769049302\\
0.95	0.00726974804083431	0.120360487059049	0.855014975717657\\
0.95	0.00787632240459385	0.125262158509929	0.858458041096605\\
0.95	0.00850818805808555	0.13016907025918	0.860776163494182\\
0.95	0.00916544737962849	0.135081047277508	0.861942500814224\\
0.95	0.00984820134829017	0.139997912937243	0.86195861464842\\
0.95	0.0105565495182326	0.144919489023162	0.860852735617091\\
0.95	0.0112905899930939	0.149845595743738	0.85867705410965\\
0.95	0.012050419400414	0.154776051742839	0.855504274967458\\
0.95	0.0128361328661109	0.159710674111862	0.851423687967754\\
0.95	0.0136478239890177	0.164649278402306	0.846537000484339\\
0.95	0.0144855848154856	0.169591678638796	0.840954157904248\\
0.95	0.0153495058140643	0.174537687332543	0.834789345403935\\
0.95	0.0162396758502651	0.179487115495259	0.828157325806202\\
0.95	0.0171561821614173	0.184439772653511	0.821170226448693\\
0.95	0.0180991103316243	0.189395466863524	0.813934846725692\\
0.95	0.0190685442668299	0.194354004726437	0.806550519904211\\
0.95	0.0200645661700016	0.199315191403999	0.799107529846304\\
0.95	0.0210872565164405	0.204278830634725	0.791686056496195\\
0.95	0.0221366940292259	0.20924472475049	0.784355603830602\\
0.95	0.023212955654804	0.214212674693577	0.77717485028112\\
0.95	0.0243161165387281	0.219182480034174	0.770191853862646\\
0.95	0.025446250001561	0.224153938988321	0.763444541554458\\
0.95	0.0266034275149466	0.229126848436298	0.756961413910481\\
0.95	0.0277877186778607	0.234101003941464	0.750762400416766\\
0.95	0.0289991911930496	0.239076199769547	0.744859807805787\\
0.95	0.0302379108436654	0.244052228908366	0.73925931151808\\
0.95	0.0315039414701067	0.24902888308801	0.733960949044503\\
0.95	0.0327973449470747	0.254005952801449	0.728960082407261\\
0.95	0.0341181811608519	0.258983227325592	0.724248305111177\\
0.95	0.0354665079868145	0.263960494742775	0.719814276220679\\
0.95	0.0368423812671862	0.268937541962694	0.715644470612631\\
0.95	0.0382458547890421	0.273914154744766	0.711723839838042\\
0.95	0.0396769802625738	0.278890117720924	0.708036382389729\\
0.95	0.0411358072996216	0.283865214418836	0.704565625567164\\
0.95	0.0426223833924861	0.288839227285554	0.701295023640317\\
0.95	0.0441367538930258	0.293811937711588	0.698208278749353\\
0.95	0.0456789619920503	0.298783126055387	0.695289592055347\\
0.95	0.0472490486990192	0.303752571668251	0.692523853197513\\
0.95	0.0488470528220537	0.308720052919643	0.68989677622812\\
0.95	0.0504730109482718	0.313685347222913	0.687394989990281\\
0.95	0.0521269574244539	0.318648231061428	0.68500609046598\\
0.95	0.0538089243380495	0.323608480015096	0.682718662029128\\
0.95	0.0555189414985328	0.328565868787292	0.68052227385347\\
0.95	0.0572570364191156	0.333520171232163	0.67840745699785\\
0.95	0.0590232342988274	0.338471160382329	0.676365666960044\\
0.95	0.0608175580049697	0.34341860847696	0.674389235783156\\
0.95	0.0626400280559541	0.348362286990222	0.672471317134861\\
0.95	0.064490662604533	0.3533019666601	0.670605827172182\\
0.95	0.0663694774214291	0.358237417517579	0.668787383459805\\
0.95	0.0682764858793752	0.363168408916186	0.667011243730729\\
0.95	0.0702116989375697	0.368094709561873	0.665273245863648\\
0.95	0.0721751251265572	0.373016087543259	0.663569750098932\\
0.95	0.0741667705335422	0.377932310362199	0.66189758422006\\
0.95	0.0761866387881432	0.382843144964686	0.660253992184442\\
0.95	0.0782347310485954	0.38774835777208	0.658636586490997\\
0.95	0.0803110459884098	0.392647714712651	0.657043304415872\\
0.95	0.0824155797834956	0.397540981253432	0.655472368126509\\
0.95	0.084548326099754	0.402427922432375	0.653922248592745\\
0.95	0.0867092760811502	0.407308302890799	0.65239163314657\\
0.95	0.0888984183382709	0.412181886906127	0.650879396495579\\
0.95	0.0911157389373742	0.417048438424896	0.649384574964843\\
0.95	0.0933612213899392	0.421907721096044	0.647906343724686\\
0.95	0.0956348466427212	0.42675949830445	0.646443996754806\\
0.95	0.0979365930683194	0.431603533204733	0.644996929295961\\
0.95	0.100266436456264	0.436439588755293	0.643564622546847\\
0.95	0.102624350004627	0.441267427752585	0.64214663037443\\
0.95	0.105010304312169	0.446086812865616	0.640742567819433\\
0.95	0.107424267371012	0.450897506670668	0.639352101193731\\
0.95	0.109866204559871	0.455699271686218	0.637974939582442\\
0.95	0.112336078637817	0.460491870408058	0.636610827579862\\
0.95	0.114833849738607	0.465275065344603	0.635259539104382\\
0.95	0.117359475365564	0.470048619052377	0.633920872153155\\
0.95	0.119912910387023	0.474812294171664	0.632594644371964\\
0.95	0.122494107032347	0.479565853462319	0.631280689329641\\
0.95	0.125103014888515	0.484309059839721	0.629978853399061\\
0.95	0.127739580897283	0.489041676410868	0.628690460824557\\
0.955	0	0	0.710202759607373\\
0.955	1.11327767495586e-05	0.00471862581271275	0.713003157837853\\
0.955	4.46175251031912e-05	0.009446325184051	0.715846486660142\\
0.955	0.000100583311362513	0.0141829653360114	0.718788838379794\\
0.955	0.000179158434668431	0.018928411755669	0.721886092228356\\
0.955	0.000280470402701511	0.02368252819604	0.725200636220892\\
0.955	0.000404645907256436	0.0284451766772965	0.728799085274723\\
0.955	0.000551810799695644	0.0332162174883388	0.732749123983336\\
0.955	0.000722090066287311	0.037995509188729	0.737115575253327\\
0.955	0.000915607803432999	0.0427829086109896	0.741955893907011\\
0.955	0.0011324871927904	0.0475782708632729	0.747315373652289\\
0.955	0.00137285047629673	0.0523814493324038	0.75322242404581\\
0.955	0.00163681893109844	0.057192295687301	0.759684310525652\\
0.955	0.00192451284439304	0.0620106598827802	0.766683748102665\\
0.955	0.00223605148818899	0.0668363901637434	0.774176695511256\\
0.955	0.00257155309398959	0.0716693330697584	0.782091614494493\\
0.955	0.00293113482740716	0.0765093334400312	0.79033034642109\\
0.955	0.00331491276271365	0.0813562344187764	0.79877062752665\\
0.955	0.00372300185733413	0.0862098774609879	0.807270128930359\\
0.955	0.00415551592628969	0.0910701023386145	0.815671782700193\\
0.955	0.00461256761659621	0.0959367471471425	0.823810053507331\\
0.955	0.00509426838162598	0.100809648312589	0.831517746503354\\
0.955	0.00560072845543873	0.105688640598912	0.838632911419452\\
0.955	0.00613205682708918	0.11057355711583	0.845005411376293\\
0.955	0.00668836121491816	0.115464229327074	0.850502769049301\\
0.955	0.00726974804083431	0.120360487059049	0.855014975717657\\
0.955	0.00787632240459385	0.125262158509929	0.858458041096606\\
0.955	0.00850818805808555	0.13016907025918	0.860776163494183\\
0.955	0.00916544737962849	0.135081047277508	0.861942500814224\\
0.955	0.00984820134829017	0.139997912937243	0.861958614648418\\
0.955	0.0105565495182326	0.144919489023162	0.860852735617089\\
0.955	0.0112905899930939	0.149845595743738	0.858677054109649\\
0.955	0.012050419400414	0.154776051742839	0.855504274967457\\
0.955	0.0128361328661109	0.159710674111862	0.851423687967756\\
0.955	0.0136478239890177	0.164649278402306	0.84653700048434\\
0.955	0.0144855848154856	0.169591678638796	0.840954157904252\\
0.955	0.0153495058140643	0.174537687332543	0.834789345403936\\
0.955	0.0162396758502651	0.179487115495259	0.828157325806199\\
0.955	0.0171561821614173	0.184439772653511	0.821170226448695\\
0.955	0.0180991103316243	0.189395466863524	0.813934846725693\\
0.955	0.0190685442668299	0.194354004726437	0.806550519904213\\
0.955	0.0200645661700016	0.199315191403999	0.799107529846306\\
0.955	0.0210872565164405	0.204278830634725	0.791686056496194\\
0.955	0.0221366940292259	0.20924472475049	0.784355603830603\\
0.955	0.023212955654804	0.214212674693577	0.777174850281119\\
0.955	0.0243161165387281	0.219182480034174	0.770191853862641\\
0.955	0.025446250001561	0.224153938988321	0.763444541554458\\
0.955	0.0266034275149466	0.229126848436298	0.756961413910484\\
0.955	0.0277877186778607	0.234101003941464	0.750762400416764\\
0.955	0.0289991911930496	0.239076199769547	0.744859807805784\\
0.955	0.0302379108436654	0.244052228908366	0.739259311518078\\
0.955	0.0315039414701067	0.24902888308801	0.733960949044504\\
0.955	0.0327973449470747	0.254005952801449	0.728960082407259\\
0.955	0.0341181811608519	0.258983227325591	0.724248305111176\\
0.955	0.0354665079868145	0.263960494742774	0.719814276220681\\
0.955	0.0368423812671862	0.268937541962694	0.71564447061263\\
0.955	0.0382458547890421	0.273914154744766	0.711723839838042\\
0.955	0.0396769802625738	0.278890117720924	0.708036382389731\\
0.955	0.0411358072996216	0.283865214418836	0.704565625567164\\
0.955	0.0426223833924861	0.288839227285554	0.701295023640316\\
0.955	0.0441367538930258	0.293811937711588	0.698208278749354\\
0.955	0.0456789619920503	0.298783126055387	0.69528959205535\\
0.955	0.0472490486990192	0.303752571668251	0.692523853197511\\
0.955	0.0488470528220538	0.308720052919643	0.689896776228119\\
0.955	0.0504730109482718	0.313685347222913	0.687394989990282\\
0.955	0.0521269574244539	0.318648231061428	0.68500609046598\\
0.955	0.0538089243380495	0.323608480015096	0.682718662029128\\
0.955	0.0555189414985328	0.328565868787292	0.68052227385347\\
0.955	0.0572570364191156	0.333520171232163	0.67840745699785\\
0.955	0.0590232342988274	0.338471160382329	0.676365666960043\\
0.955	0.0608175580049697	0.34341860847696	0.674389235783155\\
0.955	0.0626400280559541	0.348362286990222	0.672471317134862\\
0.955	0.064490662604533	0.3533019666601	0.670605827172182\\
0.955	0.0663694774214291	0.358237417517579	0.668787383459804\\
0.955	0.0682764858793752	0.363168408916186	0.667011243730728\\
0.955	0.0702116989375697	0.368094709561873	0.665273245863649\\
0.955	0.0721751251265572	0.373016087543259	0.663569750098932\\
0.955	0.0741667705335422	0.377932310362199	0.661897584220059\\
0.955	0.0761866387881432	0.382843144964686	0.660253992184442\\
0.955	0.0782347310485954	0.38774835777208	0.658636586490997\\
0.955	0.0803110459884098	0.392647714712651	0.657043304415871\\
0.955	0.0824155797834956	0.397540981253432	0.655472368126509\\
0.955	0.084548326099754	0.402427922432375	0.653922248592746\\
0.955	0.0867092760811502	0.407308302890799	0.652391633146571\\
0.955	0.0888984183382709	0.412181886906127	0.65087939649558\\
0.955	0.0911157389373742	0.417048438424897	0.649384574964843\\
0.955	0.0933612213899392	0.421907721096044	0.647906343724684\\
0.955	0.0956348466427212	0.42675949830445	0.646443996754807\\
0.955	0.0979365930683194	0.431603533204733	0.644996929295962\\
0.955	0.100266436456264	0.436439588755293	0.643564622546846\\
0.955	0.102624350004627	0.441267427752584	0.642146630374431\\
0.955	0.105010304312169	0.446086812865616	0.640742567819433\\
0.955	0.107424267371012	0.450897506670668	0.639352101193731\\
0.955	0.109866204559871	0.455699271686218	0.637974939582443\\
0.955	0.112336078637817	0.460491870408058	0.636610827579862\\
0.955	0.114833849738607	0.465275065344603	0.635259539104381\\
0.955	0.117359475365564	0.470048619052377	0.633920872153154\\
0.955	0.119912910387022	0.474812294171664	0.632594644371967\\
0.955	0.122494107032347	0.479565853462319	0.631280689329639\\
0.955	0.125103014888515	0.484309059839721	0.629978853399059\\
0.955	0.127739580897283	0.489041676410868	0.628690460824564\\
0.96	0	0	0.710202759607373\\
0.96	1.11327767495586e-05	0.00471862581271275	0.713003157837853\\
0.96	4.46175251031912e-05	0.009446325184051	0.715846486660142\\
0.96	0.000100583311362513	0.0141829653360114	0.718788838379794\\
0.96	0.000179158434668431	0.018928411755669	0.721886092228356\\
0.96	0.000280470402701511	0.02368252819604	0.725200636220892\\
0.96	0.000404645907256436	0.0284451766772965	0.728799085274723\\
0.96	0.000551810799695644	0.0332162174883389	0.732749123983336\\
0.96	0.000722090066287311	0.037995509188729	0.737115575253328\\
0.96	0.000915607803432999	0.0427829086109896	0.741955893907011\\
0.96	0.0011324871927904	0.0475782708632729	0.747315373652289\\
0.96	0.00137285047629673	0.0523814493324038	0.753222424045811\\
0.96	0.00163681893109844	0.057192295687301	0.759684310525652\\
0.96	0.00192451284439304	0.0620106598827802	0.766683748102665\\
0.96	0.00223605148818899	0.0668363901637434	0.774176695511257\\
0.96	0.00257155309398959	0.0716693330697584	0.782091614494494\\
0.96	0.00293113482740716	0.0765093334400312	0.79033034642109\\
0.96	0.00331491276271365	0.0813562344187764	0.79877062752665\\
0.96	0.00372300185733413	0.0862098774609879	0.807270128930359\\
0.96	0.00415551592628969	0.0910701023386145	0.815671782700192\\
0.96	0.00461256761659621	0.0959367471471425	0.82381005350733\\
0.96	0.00509426838162598	0.100809648312589	0.831517746503354\\
0.96	0.00560072845543873	0.105688640598912	0.838632911419452\\
0.96	0.00613205682708918	0.11057355711583	0.845005411376292\\
0.96	0.00668836121491816	0.115464229327074	0.850502769049303\\
0.96	0.00726974804083431	0.120360487059049	0.855014975717657\\
0.96	0.00787632240459385	0.125262158509929	0.858458041096607\\
0.96	0.00850818805808555	0.13016907025918	0.860776163494181\\
0.96	0.00916544737962849	0.135081047277508	0.861942500814225\\
0.96	0.00984820134829017	0.139997912937243	0.861958614648422\\
0.96	0.0105565495182326	0.144919489023162	0.860852735617091\\
0.96	0.0112905899930939	0.149845595743738	0.858677054109651\\
0.96	0.012050419400414	0.154776051742839	0.855504274967457\\
0.96	0.0128361328661109	0.159710674111862	0.851423687967756\\
0.96	0.0136478239890177	0.164649278402306	0.846537000484341\\
0.96	0.0144855848154856	0.169591678638796	0.840954157904249\\
0.96	0.0153495058140643	0.174537687332543	0.834789345403934\\
0.96	0.0162396758502651	0.179487115495259	0.828157325806202\\
0.96	0.0171561821614173	0.184439772653511	0.821170226448696\\
0.96	0.0180991103316243	0.189395466863524	0.813934846725692\\
0.96	0.0190685442668299	0.194354004726437	0.806550519904213\\
0.96	0.0200645661700016	0.199315191403999	0.799107529846303\\
0.96	0.0210872565164405	0.204278830634725	0.791686056496194\\
0.96	0.0221366940292259	0.20924472475049	0.784355603830602\\
0.96	0.023212955654804	0.214212674693577	0.777174850281117\\
0.96	0.0243161165387281	0.219182480034174	0.770191853862644\\
0.96	0.025446250001561	0.224153938988321	0.76344454155446\\
0.96	0.0266034275149466	0.229126848436298	0.756961413910481\\
0.96	0.0277877186778607	0.234101003941464	0.750762400416764\\
0.96	0.0289991911930496	0.239076199769547	0.744859807805787\\
0.96	0.0302379108436654	0.244052228908366	0.73925931151808\\
0.96	0.0315039414701067	0.24902888308801	0.733960949044502\\
0.96	0.0327973449470747	0.254005952801449	0.728960082407259\\
0.96	0.0341181811608519	0.258983227325591	0.724248305111178\\
0.96	0.0354665079868145	0.263960494742775	0.719814276220682\\
0.96	0.0368423812671862	0.268937541962694	0.715644470612629\\
0.96	0.0382458547890421	0.273914154744766	0.711723839838043\\
0.96	0.0396769802625737	0.278890117720924	0.708036382389733\\
0.96	0.0411358072996216	0.283865214418836	0.704565625567164\\
0.96	0.0426223833924861	0.288839227285554	0.701295023640315\\
0.96	0.0441367538930258	0.293811937711588	0.698208278749354\\
0.96	0.0456789619920503	0.298783126055387	0.69528959205535\\
0.96	0.0472490486990192	0.303752571668251	0.692523853197511\\
0.96	0.0488470528220538	0.308720052919643	0.689896776228118\\
0.96	0.0504730109482718	0.313685347222913	0.687394989990282\\
0.96	0.0521269574244539	0.318648231061428	0.68500609046598\\
0.96	0.0538089243380495	0.323608480015096	0.682718662029127\\
0.96	0.0555189414985328	0.328565868787292	0.68052227385347\\
0.96	0.0572570364191156	0.333520171232163	0.678407456997851\\
0.96	0.0590232342988274	0.338471160382329	0.676365666960044\\
0.96	0.0608175580049697	0.34341860847696	0.674389235783155\\
0.96	0.0626400280559541	0.348362286990222	0.672471317134862\\
0.96	0.064490662604533	0.3533019666601	0.670605827172183\\
0.96	0.0663694774214291	0.35823741751758	0.668787383459803\\
0.96	0.0682764858793752	0.363168408916186	0.667011243730728\\
0.96	0.0702116989375697	0.368094709561873	0.66527324586365\\
0.96	0.0721751251265572	0.373016087543259	0.663569750098932\\
0.96	0.0741667705335423	0.377932310362199	0.66189758422006\\
0.96	0.0761866387881432	0.382843144964686	0.660253992184442\\
0.96	0.0782347310485954	0.38774835777208	0.658636586490997\\
0.96	0.0803110459884098	0.392647714712651	0.657043304415871\\
0.96	0.0824155797834956	0.397540981253432	0.655472368126508\\
0.96	0.084548326099754	0.402427922432375	0.653922248592745\\
0.96	0.0867092760811502	0.407308302890799	0.65239163314657\\
0.96	0.0888984183382709	0.412181886906127	0.650879396495579\\
0.96	0.0911157389373742	0.417048438424896	0.649384574964845\\
0.96	0.0933612213899392	0.421907721096044	0.647906343724685\\
0.96	0.0956348466427212	0.42675949830445	0.646443996754805\\
0.96	0.0979365930683194	0.431603533204733	0.644996929295962\\
0.96	0.100266436456264	0.436439588755293	0.643564622546846\\
0.96	0.102624350004627	0.441267427752584	0.642146630374429\\
0.96	0.105010304312169	0.446086812865616	0.640742567819434\\
0.96	0.107424267371012	0.450897506670668	0.639352101193731\\
0.96	0.109866204559871	0.455699271686218	0.637974939582443\\
0.96	0.112336078637817	0.460491870408058	0.636610827579862\\
0.96	0.114833849738607	0.465275065344603	0.635259539104382\\
0.96	0.117359475365564	0.470048619052377	0.633920872153153\\
0.96	0.119912910387022	0.474812294171664	0.632594644371966\\
0.96	0.122494107032347	0.479565853462319	0.631280689329642\\
0.96	0.125103014888515	0.484309059839721	0.62997885339906\\
0.96	0.127739580897283	0.489041676410868	0.628690460824559\\
0.965	0	0	0.710202759607373\\
0.965	1.11327767495586e-05	0.00471862581271275	0.713003157837853\\
0.965	4.46175251031912e-05	0.009446325184051	0.715846486660142\\
0.965	0.000100583311362513	0.0141829653360114	0.718788838379794\\
0.965	0.000179158434668431	0.018928411755669	0.721886092228356\\
0.965	0.000280470402701511	0.02368252819604	0.725200636220892\\
0.965	0.000404645907256436	0.0284451766772965	0.728799085274723\\
0.965	0.000551810799695644	0.0332162174883388	0.732749123983336\\
0.965	0.000722090066287311	0.037995509188729	0.737115575253327\\
0.965	0.000915607803432999	0.0427829086109896	0.741955893907011\\
0.965	0.0011324871927904	0.0475782708632729	0.747315373652289\\
0.965	0.00137285047629673	0.0523814493324038	0.753222424045811\\
0.965	0.00163681893109844	0.057192295687301	0.759684310525652\\
0.965	0.00192451284439304	0.0620106598827802	0.766683748102665\\
0.965	0.00223605148818898	0.0668363901637434	0.774176695511256\\
0.965	0.00257155309398959	0.0716693330697584	0.782091614494494\\
0.965	0.00293113482740716	0.0765093334400312	0.79033034642109\\
0.965	0.00331491276271365	0.0813562344187764	0.79877062752665\\
0.965	0.00372300185733413	0.0862098774609879	0.80727012893036\\
0.965	0.00415551592628969	0.0910701023386145	0.815671782700193\\
0.965	0.00461256761659621	0.0959367471471425	0.82381005350733\\
0.965	0.00509426838162598	0.100809648312589	0.831517746503353\\
0.965	0.00560072845543873	0.105688640598912	0.838632911419452\\
0.965	0.00613205682708918	0.11057355711583	0.845005411376293\\
0.965	0.00668836121491816	0.115464229327074	0.850502769049302\\
0.965	0.00726974804083431	0.120360487059049	0.855014975717656\\
0.965	0.00787632240459385	0.125262158509929	0.858458041096606\\
0.965	0.00850818805808555	0.13016907025918	0.860776163494183\\
0.965	0.00916544737962849	0.135081047277508	0.861942500814224\\
0.965	0.00984820134829017	0.139997912937243	0.861958614648421\\
0.965	0.0105565495182326	0.144919489023162	0.860852735617091\\
0.965	0.0112905899930939	0.149845595743738	0.858677054109651\\
0.965	0.012050419400414	0.154776051742839	0.855504274967458\\
0.965	0.0128361328661109	0.159710674111862	0.851423687967754\\
0.965	0.0136478239890177	0.164649278402306	0.84653700048434\\
0.965	0.0144855848154856	0.169591678638796	0.840954157904246\\
0.965	0.0153495058140643	0.174537687332543	0.834789345403936\\
0.965	0.0162396758502651	0.179487115495259	0.8281573258062\\
0.965	0.0171561821614173	0.184439772653511	0.821170226448694\\
0.965	0.0180991103316243	0.189395466863524	0.813934846725694\\
0.965	0.0190685442668299	0.194354004726437	0.806550519904211\\
0.965	0.0200645661700016	0.199315191403999	0.799107529846303\\
0.965	0.0210872565164405	0.204278830634725	0.791686056496195\\
0.965	0.0221366940292259	0.20924472475049	0.784355603830603\\
0.965	0.023212955654804	0.214212674693577	0.777174850281117\\
0.965	0.0243161165387281	0.219182480034174	0.770191853862644\\
0.965	0.025446250001561	0.224153938988321	0.763444541554459\\
0.965	0.0266034275149466	0.229126848436298	0.756961413910483\\
0.965	0.0277877186778607	0.234101003941464	0.750762400416765\\
0.965	0.0289991911930496	0.239076199769547	0.744859807805784\\
0.965	0.0302379108436654	0.244052228908366	0.739259311518079\\
0.965	0.0315039414701067	0.24902888308801	0.733960949044504\\
0.965	0.0327973449470747	0.254005952801449	0.728960082407258\\
0.965	0.0341181811608519	0.258983227325591	0.724248305111178\\
0.965	0.0354665079868146	0.263960494742775	0.719814276220683\\
0.965	0.0368423812671862	0.268937541962694	0.71564447061263\\
0.965	0.0382458547890421	0.273914154744766	0.711723839838043\\
0.965	0.0396769802625738	0.278890117720924	0.708036382389731\\
0.965	0.0411358072996216	0.283865214418836	0.704565625567163\\
0.965	0.0426223833924861	0.288839227285554	0.701295023640315\\
0.965	0.0441367538930258	0.293811937711588	0.698208278749353\\
0.965	0.0456789619920503	0.298783126055387	0.695289592055348\\
0.965	0.0472490486990192	0.303752571668251	0.692523853197513\\
0.965	0.0488470528220538	0.308720052919643	0.68989677622812\\
0.965	0.0504730109482718	0.313685347222913	0.68739498999028\\
0.965	0.0521269574244539	0.318648231061428	0.685006090465981\\
0.965	0.0538089243380495	0.323608480015097	0.682718662029127\\
0.965	0.0555189414985328	0.328565868787292	0.680522273853468\\
0.965	0.0572570364191156	0.333520171232163	0.67840745699785\\
0.965	0.0590232342988274	0.338471160382329	0.676365666960043\\
0.965	0.0608175580049697	0.34341860847696	0.674389235783156\\
0.965	0.0626400280559541	0.348362286990222	0.672471317134862\\
0.965	0.064490662604533	0.3533019666601	0.670605827172183\\
0.965	0.0663694774214291	0.358237417517579	0.668787383459804\\
0.965	0.0682764858793752	0.363168408916186	0.667011243730728\\
0.965	0.0702116989375697	0.368094709561873	0.665273245863649\\
0.965	0.0721751251265572	0.373016087543259	0.663569750098933\\
0.965	0.0741667705335422	0.377932310362199	0.66189758422006\\
0.965	0.0761866387881432	0.382843144964686	0.660253992184442\\
0.965	0.0782347310485954	0.38774835777208	0.658636586490997\\
0.965	0.0803110459884098	0.392647714712651	0.657043304415871\\
0.965	0.0824155797834956	0.397540981253432	0.655472368126509\\
0.965	0.084548326099754	0.402427922432375	0.653922248592745\\
0.965	0.0867092760811502	0.407308302890799	0.65239163314657\\
0.965	0.0888984183382709	0.412181886906127	0.650879396495579\\
0.965	0.0911157389373742	0.417048438424896	0.649384574964842\\
0.965	0.0933612213899392	0.421907721096044	0.647906343724685\\
0.965	0.0956348466427212	0.42675949830445	0.646443996754806\\
0.965	0.0979365930683194	0.431603533204733	0.644996929295961\\
0.965	0.100266436456264	0.436439588755293	0.643564622546847\\
0.965	0.102624350004627	0.441267427752584	0.64214663037443\\
0.965	0.105010304312169	0.446086812865616	0.640742567819433\\
0.965	0.107424267371012	0.450897506670668	0.639352101193731\\
0.965	0.109866204559871	0.455699271686218	0.637974939582443\\
0.965	0.112336078637817	0.460491870408058	0.636610827579861\\
0.965	0.114833849738607	0.465275065344603	0.635259539104383\\
0.965	0.117359475365564	0.470048619052377	0.633920872153155\\
0.965	0.119912910387023	0.474812294171664	0.632594644371965\\
0.965	0.122494107032347	0.479565853462319	0.631280689329639\\
0.965	0.125103014888515	0.484309059839721	0.629978853399059\\
0.965	0.127739580897283	0.489041676410868	0.628690460824564\\
0.97	0	0	0.710202759607373\\
0.97	1.11327767495586e-05	0.00471862581271275	0.713003157837853\\
0.97	4.46175251031912e-05	0.009446325184051	0.715846486660142\\
0.97	0.000100583311362513	0.0141829653360114	0.718788838379794\\
0.97	0.000179158434668431	0.018928411755669	0.721886092228356\\
0.97	0.000280470402701511	0.02368252819604	0.725200636220892\\
0.97	0.000404645907256436	0.0284451766772965	0.728799085274723\\
0.97	0.000551810799695644	0.0332162174883388	0.732749123983336\\
0.97	0.000722090066287311	0.037995509188729	0.737115575253327\\
0.97	0.000915607803432999	0.0427829086109896	0.741955893907011\\
0.97	0.0011324871927904	0.047578270863273	0.747315373652289\\
0.97	0.00137285047629673	0.0523814493324038	0.753222424045811\\
0.97	0.00163681893109844	0.057192295687301	0.759684310525652\\
0.97	0.00192451284439304	0.0620106598827802	0.766683748102665\\
0.97	0.00223605148818898	0.0668363901637434	0.774176695511257\\
0.97	0.00257155309398959	0.0716693330697584	0.782091614494494\\
0.97	0.00293113482740716	0.0765093334400312	0.79033034642109\\
0.97	0.00331491276271365	0.0813562344187764	0.79877062752665\\
0.97	0.00372300185733413	0.0862098774609879	0.807270128930359\\
0.97	0.00415551592628969	0.0910701023386145	0.815671782700194\\
0.97	0.00461256761659621	0.0959367471471424	0.823810053507331\\
0.97	0.00509426838162598	0.100809648312589	0.831517746503354\\
0.97	0.00560072845543873	0.105688640598912	0.838632911419451\\
0.97	0.00613205682708918	0.11057355711583	0.845005411376294\\
0.97	0.00668836121491816	0.115464229327074	0.850502769049301\\
0.97	0.00726974804083431	0.120360487059049	0.855014975717657\\
0.97	0.00787632240459385	0.125262158509929	0.858458041096605\\
0.97	0.00850818805808555	0.13016907025918	0.860776163494183\\
0.97	0.00916544737962849	0.135081047277508	0.861942500814224\\
0.97	0.00984820134829017	0.139997912937243	0.86195861464842\\
0.97	0.0105565495182326	0.144919489023162	0.860852735617091\\
0.97	0.0112905899930939	0.149845595743738	0.85867705410965\\
0.97	0.012050419400414	0.154776051742839	0.855504274967459\\
0.97	0.0128361328661109	0.159710674111862	0.851423687967752\\
0.97	0.0136478239890177	0.164649278402306	0.846537000484341\\
0.97	0.0144855848154856	0.169591678638796	0.840954157904248\\
0.97	0.0153495058140643	0.174537687332543	0.834789345403934\\
0.97	0.0162396758502651	0.179487115495259	0.828157325806202\\
0.97	0.0171561821614173	0.184439772653511	0.821170226448696\\
0.97	0.0180991103316243	0.189395466863524	0.81393484672569\\
0.97	0.0190685442668299	0.194354004726437	0.80655051990421\\
0.97	0.0200645661700016	0.199315191403999	0.799107529846306\\
0.97	0.0210872565164405	0.204278830634725	0.791686056496195\\
0.97	0.0221366940292259	0.20924472475049	0.784355603830601\\
0.97	0.023212955654804	0.214212674693577	0.777174850281118\\
0.97	0.0243161165387281	0.219182480034174	0.770191853862644\\
0.97	0.025446250001561	0.224153938988321	0.763444541554458\\
0.97	0.0266034275149466	0.229126848436298	0.756961413910483\\
0.97	0.0277877186778607	0.234101003941464	0.750762400416764\\
0.97	0.0289991911930496	0.239076199769547	0.744859807805784\\
0.97	0.0302379108436654	0.244052228908366	0.739259311518079\\
0.97	0.0315039414701067	0.24902888308801	0.733960949044503\\
0.97	0.0327973449470747	0.254005952801449	0.728960082407261\\
0.97	0.0341181811608519	0.258983227325592	0.724248305111178\\
0.97	0.0354665079868145	0.263960494742774	0.719814276220681\\
0.97	0.0368423812671862	0.268937541962694	0.715644470612631\\
0.97	0.0382458547890421	0.273914154744766	0.711723839838042\\
0.97	0.0396769802625738	0.278890117720924	0.70803638238973\\
0.97	0.0411358072996216	0.283865214418836	0.704565625567163\\
0.97	0.0426223833924861	0.288839227285554	0.701295023640316\\
0.97	0.0441367538930258	0.293811937711588	0.698208278749355\\
0.97	0.0456789619920503	0.298783126055387	0.695289592055349\\
0.97	0.0472490486990192	0.303752571668251	0.692523853197512\\
0.97	0.0488470528220538	0.308720052919643	0.689896776228119\\
0.97	0.0504730109482718	0.313685347222913	0.68739498999028\\
0.97	0.0521269574244539	0.318648231061428	0.685006090465981\\
0.97	0.0538089243380495	0.323608480015097	0.682718662029128\\
0.97	0.0555189414985328	0.328565868787292	0.680522273853468\\
0.97	0.0572570364191156	0.333520171232163	0.67840745699785\\
0.97	0.0590232342988274	0.338471160382329	0.676365666960044\\
0.97	0.0608175580049697	0.34341860847696	0.674389235783156\\
0.97	0.0626400280559541	0.348362286990222	0.672471317134863\\
0.97	0.064490662604533	0.3533019666601	0.670605827172182\\
0.97	0.0663694774214291	0.358237417517579	0.668787383459803\\
0.97	0.0682764858793752	0.363168408916186	0.667011243730728\\
0.97	0.0702116989375697	0.368094709561873	0.665273245863649\\
0.97	0.0721751251265572	0.373016087543259	0.663569750098933\\
0.97	0.0741667705335423	0.377932310362199	0.661897584220061\\
0.97	0.0761866387881432	0.382843144964686	0.660253992184443\\
0.97	0.0782347310485954	0.38774835777208	0.658636586490997\\
0.97	0.0803110459884098	0.392647714712651	0.657043304415871\\
0.97	0.0824155797834956	0.397540981253432	0.655472368126509\\
0.97	0.084548326099754	0.402427922432375	0.653922248592746\\
0.97	0.0867092760811502	0.4073083028908	0.652391633146569\\
0.97	0.0888984183382709	0.412181886906127	0.650879396495579\\
0.97	0.0911157389373742	0.417048438424897	0.649384574964844\\
0.97	0.0933612213899392	0.421907721096044	0.647906343724684\\
0.97	0.0956348466427212	0.42675949830445	0.646443996754807\\
0.97	0.0979365930683194	0.431603533204733	0.644996929295962\\
0.97	0.100266436456264	0.436439588755293	0.643564622546846\\
0.97	0.102624350004627	0.441267427752584	0.642146630374429\\
0.97	0.105010304312169	0.446086812865616	0.640742567819434\\
0.97	0.107424267371012	0.450897506670668	0.639352101193731\\
0.97	0.109866204559871	0.455699271686218	0.637974939582443\\
0.97	0.112336078637817	0.460491870408058	0.636610827579861\\
0.97	0.114833849738607	0.465275065344603	0.635259539104382\\
0.97	0.117359475365564	0.470048619052377	0.633920872153155\\
0.97	0.119912910387022	0.474812294171664	0.632594644371966\\
0.97	0.122494107032347	0.479565853462319	0.63128068932964\\
0.97	0.125103014888515	0.484309059839721	0.629978853399058\\
0.97	0.127739580897283	0.489041676410868	0.62869046082456\\
0.975	0	0	0.710202759607373\\
0.975	1.11327767495586e-05	0.00471862581271275	0.713003157837853\\
0.975	4.46175251031912e-05	0.009446325184051	0.715846486660142\\
0.975	0.000100583311362513	0.0141829653360114	0.718788838379794\\
0.975	0.000179158434668431	0.018928411755669	0.721886092228356\\
0.975	0.000280470402701511	0.02368252819604	0.725200636220892\\
0.975	0.000404645907256436	0.0284451766772965	0.728799085274723\\
0.975	0.000551810799695644	0.0332162174883389	0.732749123983336\\
0.975	0.000722090066287311	0.037995509188729	0.737115575253328\\
0.975	0.000915607803432999	0.0427829086109896	0.741955893907011\\
0.975	0.0011324871927904	0.0475782708632729	0.747315373652289\\
0.975	0.00137285047629673	0.0523814493324038	0.753222424045811\\
0.975	0.00163681893109843	0.057192295687301	0.759684310525652\\
0.975	0.00192451284439304	0.0620106598827802	0.766683748102665\\
0.975	0.00223605148818899	0.0668363901637434	0.774176695511256\\
0.975	0.00257155309398959	0.0716693330697584	0.782091614494494\\
0.975	0.00293113482740716	0.0765093334400312	0.79033034642109\\
0.975	0.00331491276271365	0.0813562344187764	0.79877062752665\\
0.975	0.00372300185733414	0.0862098774609879	0.807270128930359\\
0.975	0.00415551592628969	0.0910701023386145	0.815671782700193\\
0.975	0.00461256761659621	0.0959367471471425	0.823810053507331\\
0.975	0.00509426838162598	0.100809648312589	0.831517746503354\\
0.975	0.00560072845543873	0.105688640598912	0.838632911419452\\
0.975	0.00613205682708918	0.11057355711583	0.845005411376293\\
0.975	0.00668836121491815	0.115464229327074	0.850502769049302\\
0.975	0.00726974804083431	0.120360487059049	0.855014975717657\\
0.975	0.00787632240459385	0.125262158509929	0.858458041096606\\
0.975	0.00850818805808555	0.13016907025918	0.860776163494182\\
0.975	0.00916544737962849	0.135081047277508	0.861942500814224\\
0.975	0.00984820134829017	0.139997912937243	0.861958614648419\\
0.975	0.0105565495182326	0.144919489023162	0.860852735617091\\
0.975	0.0112905899930939	0.149845595743738	0.858677054109652\\
0.975	0.012050419400414	0.154776051742839	0.855504274967457\\
0.975	0.0128361328661109	0.159710674111862	0.851423687967754\\
0.975	0.0136478239890177	0.164649278402306	0.846537000484341\\
0.975	0.0144855848154856	0.169591678638796	0.840954157904248\\
0.975	0.0153495058140643	0.174537687332543	0.834789345403938\\
0.975	0.0162396758502651	0.179487115495259	0.828157325806201\\
0.975	0.0171561821614173	0.184439772653511	0.821170226448692\\
0.975	0.0180991103316243	0.189395466863524	0.81393484672569\\
0.975	0.0190685442668299	0.194354004726437	0.806550519904212\\
0.975	0.0200645661700016	0.199315191403999	0.799107529846305\\
0.975	0.0210872565164405	0.204278830634725	0.791686056496193\\
0.975	0.0221366940292259	0.20924472475049	0.784355603830602\\
0.975	0.023212955654804	0.214212674693577	0.777174850281118\\
0.975	0.0243161165387281	0.219182480034174	0.770191853862642\\
0.975	0.025446250001561	0.224153938988321	0.763444541554458\\
0.975	0.0266034275149466	0.229126848436298	0.756961413910482\\
0.975	0.0277877186778607	0.234101003941464	0.750762400416763\\
0.975	0.0289991911930496	0.239076199769547	0.744859807805785\\
0.975	0.0302379108436654	0.244052228908366	0.73925931151808\\
0.975	0.0315039414701067	0.24902888308801	0.733960949044503\\
0.975	0.0327973449470747	0.254005952801449	0.72896008240726\\
0.975	0.0341181811608519	0.258983227325591	0.724248305111179\\
0.975	0.0354665079868145	0.263960494742775	0.719814276220682\\
0.975	0.0368423812671862	0.268937541962694	0.715644470612629\\
0.975	0.0382458547890421	0.273914154744766	0.711723839838042\\
0.975	0.0396769802625738	0.278890117720924	0.70803638238973\\
0.975	0.0411358072996216	0.283865214418836	0.704565625567164\\
0.975	0.0426223833924861	0.288839227285554	0.701295023640317\\
0.975	0.0441367538930258	0.293811937711588	0.698208278749354\\
0.975	0.0456789619920503	0.298783126055387	0.695289592055349\\
0.975	0.0472490486990192	0.303752571668251	0.692523853197511\\
0.975	0.0488470528220538	0.308720052919643	0.689896776228119\\
0.975	0.0504730109482718	0.313685347222913	0.687394989990283\\
0.975	0.0521269574244539	0.318648231061428	0.685006090465981\\
0.975	0.0538089243380495	0.323608480015097	0.682718662029127\\
0.975	0.0555189414985328	0.328565868787292	0.680522273853469\\
0.975	0.0572570364191156	0.333520171232163	0.67840745699785\\
0.975	0.0590232342988274	0.338471160382329	0.676365666960044\\
0.975	0.0608175580049697	0.34341860847696	0.674389235783155\\
0.975	0.0626400280559541	0.348362286990221	0.672471317134863\\
0.975	0.064490662604533	0.3533019666601	0.670605827172183\\
0.975	0.0663694774214291	0.358237417517579	0.668787383459803\\
0.975	0.0682764858793752	0.363168408916186	0.667011243730728\\
0.975	0.0702116989375697	0.368094709561873	0.665273245863648\\
0.975	0.0721751251265572	0.373016087543259	0.663569750098933\\
0.975	0.0741667705335422	0.377932310362199	0.661897584220061\\
0.975	0.0761866387881432	0.382843144964686	0.660253992184442\\
0.975	0.0782347310485954	0.38774835777208	0.658636586490997\\
0.975	0.0803110459884098	0.392647714712651	0.657043304415871\\
0.975	0.0824155797834956	0.397540981253432	0.65547236812651\\
0.975	0.084548326099754	0.402427922432375	0.653922248592745\\
0.975	0.0867092760811502	0.407308302890799	0.65239163314657\\
0.975	0.0888984183382709	0.412181886906127	0.650879396495579\\
0.975	0.0911157389373742	0.417048438424896	0.649384574964844\\
0.975	0.0933612213899392	0.421907721096044	0.647906343724684\\
0.975	0.0956348466427212	0.42675949830445	0.646443996754807\\
0.975	0.0979365930683194	0.431603533204733	0.644996929295961\\
0.975	0.100266436456264	0.436439588755293	0.643564622546846\\
0.975	0.102624350004627	0.441267427752584	0.64214663037443\\
0.975	0.105010304312169	0.446086812865616	0.640742567819432\\
0.975	0.107424267371012	0.450897506670668	0.639352101193731\\
0.975	0.109866204559871	0.455699271686218	0.637974939582444\\
0.975	0.112336078637817	0.460491870408058	0.636610827579862\\
0.975	0.114833849738607	0.465275065344603	0.635259539104382\\
0.975	0.117359475365564	0.470048619052377	0.633920872153153\\
0.975	0.119912910387022	0.474812294171664	0.632594644371966\\
0.975	0.122494107032347	0.479565853462319	0.631280689329642\\
0.975	0.125103014888515	0.484309059839721	0.629978853399058\\
0.975	0.127739580897283	0.489041676410868	0.628690460824561\\
0.98	0	0	0.710202759607373\\
0.98	1.11327767495586e-05	0.00471862581271275	0.713003157837853\\
0.98	4.46175251031912e-05	0.009446325184051	0.715846486660142\\
0.98	0.000100583311362513	0.0141829653360114	0.718788838379794\\
0.98	0.000179158434668431	0.018928411755669	0.721886092228356\\
0.98	0.000280470402701511	0.02368252819604	0.725200636220892\\
0.98	0.000404645907256436	0.0284451766772965	0.728799085274723\\
0.98	0.000551810799695644	0.0332162174883388	0.732749123983336\\
0.98	0.000722090066287311	0.037995509188729	0.737115575253327\\
0.98	0.000915607803432999	0.0427829086109896	0.741955893907011\\
0.98	0.0011324871927904	0.0475782708632729	0.747315373652289\\
0.98	0.00137285047629673	0.0523814493324038	0.75322242404581\\
0.98	0.00163681893109844	0.057192295687301	0.759684310525653\\
0.98	0.00192451284439304	0.0620106598827802	0.766683748102665\\
0.98	0.00223605148818898	0.0668363901637434	0.774176695511256\\
0.98	0.00257155309398959	0.0716693330697584	0.782091614494493\\
0.98	0.00293113482740716	0.0765093334400312	0.79033034642109\\
0.98	0.00331491276271365	0.0813562344187764	0.798770627526651\\
0.98	0.00372300185733413	0.0862098774609879	0.807270128930359\\
0.98	0.00415551592628969	0.0910701023386145	0.815671782700193\\
0.98	0.00461256761659621	0.0959367471471425	0.82381005350733\\
0.98	0.00509426838162598	0.100809648312589	0.831517746503354\\
0.98	0.00560072845543873	0.105688640598912	0.838632911419451\\
0.98	0.00613205682708918	0.11057355711583	0.845005411376293\\
0.98	0.00668836121491815	0.115464229327074	0.850502769049303\\
0.98	0.00726974804083431	0.120360487059049	0.855014975717656\\
0.98	0.00787632240459385	0.125262158509929	0.858458041096607\\
0.98	0.00850818805808555	0.13016907025918	0.860776163494181\\
0.98	0.00916544737962849	0.135081047277508	0.861942500814224\\
0.98	0.00984820134829017	0.139997912937243	0.86195861464842\\
0.98	0.0105565495182326	0.144919489023162	0.86085273561709\\
0.98	0.0112905899930939	0.149845595743738	0.858677054109651\\
0.98	0.012050419400414	0.154776051742839	0.855504274967457\\
0.98	0.0128361328661109	0.159710674111862	0.851423687967756\\
0.98	0.0136478239890177	0.164649278402306	0.846537000484339\\
0.98	0.0144855848154856	0.169591678638796	0.84095415790425\\
0.98	0.0153495058140643	0.174537687332543	0.834789345403937\\
0.98	0.0162396758502651	0.179487115495259	0.828157325806199\\
0.98	0.0171561821614173	0.184439772653511	0.821170226448693\\
0.98	0.0180991103316243	0.189395466863524	0.813934846725692\\
0.98	0.0190685442668299	0.194354004726437	0.806550519904212\\
0.98	0.0200645661700016	0.199315191403999	0.799107529846304\\
0.98	0.0210872565164405	0.204278830634725	0.791686056496194\\
0.98	0.0221366940292259	0.20924472475049	0.784355603830601\\
0.98	0.023212955654804	0.214212674693577	0.777174850281118\\
0.98	0.0243161165387281	0.219182480034174	0.770191853862643\\
0.98	0.025446250001561	0.224153938988321	0.763444541554458\\
0.98	0.0266034275149466	0.229126848436298	0.756961413910482\\
0.98	0.0277877186778607	0.234101003941464	0.750762400416765\\
0.98	0.0289991911930496	0.239076199769547	0.744859807805784\\
0.98	0.0302379108436654	0.244052228908366	0.739259311518081\\
0.98	0.0315039414701067	0.24902888308801	0.733960949044503\\
0.98	0.0327973449470747	0.254005952801449	0.72896008240726\\
0.98	0.0341181811608519	0.258983227325591	0.72424830511118\\
0.98	0.0354665079868145	0.263960494742775	0.719814276220681\\
0.98	0.0368423812671862	0.268937541962694	0.715644470612629\\
0.98	0.0382458547890421	0.273914154744766	0.711723839838043\\
0.98	0.0396769802625738	0.278890117720924	0.70803638238973\\
0.98	0.0411358072996216	0.283865214418836	0.704565625567163\\
0.98	0.0426223833924861	0.288839227285554	0.701295023640316\\
0.98	0.0441367538930258	0.293811937711588	0.698208278749354\\
0.98	0.0456789619920503	0.298783126055387	0.69528959205535\\
0.98	0.0472490486990192	0.303752571668251	0.692523853197512\\
0.98	0.0488470528220538	0.308720052919643	0.689896776228118\\
0.98	0.0504730109482718	0.313685347222913	0.687394989990282\\
0.98	0.0521269574244539	0.318648231061428	0.685006090465981\\
0.98	0.0538089243380495	0.323608480015096	0.682718662029128\\
0.98	0.0555189414985328	0.328565868787292	0.680522273853469\\
0.98	0.0572570364191156	0.333520171232163	0.67840745699785\\
0.98	0.0590232342988274	0.338471160382329	0.676365666960043\\
0.98	0.0608175580049697	0.34341860847696	0.674389235783155\\
0.98	0.0626400280559541	0.348362286990221	0.672471317134862\\
0.98	0.064490662604533	0.3533019666601	0.670605827172183\\
0.98	0.0663694774214291	0.358237417517579	0.668787383459803\\
0.98	0.0682764858793752	0.363168408916186	0.667011243730728\\
0.98	0.0702116989375697	0.368094709561873	0.665273245863649\\
0.98	0.0721751251265572	0.373016087543259	0.663569750098932\\
0.98	0.0741667705335422	0.377932310362199	0.661897584220062\\
0.98	0.0761866387881432	0.382843144964686	0.660253992184443\\
0.98	0.0782347310485954	0.38774835777208	0.658636586490997\\
0.98	0.0803110459884098	0.392647714712651	0.65704330441587\\
0.98	0.0824155797834956	0.397540981253432	0.65547236812651\\
0.98	0.084548326099754	0.402427922432375	0.653922248592746\\
0.98	0.0867092760811502	0.407308302890799	0.65239163314657\\
0.98	0.0888984183382709	0.412181886906127	0.650879396495579\\
0.98	0.0911157389373742	0.417048438424896	0.649384574964844\\
0.98	0.0933612213899392	0.421907721096044	0.647906343724684\\
0.98	0.0956348466427212	0.42675949830445	0.646443996754806\\
0.98	0.0979365930683194	0.431603533204733	0.644996929295962\\
0.98	0.100266436456264	0.436439588755293	0.643564622546845\\
0.98	0.102624350004627	0.441267427752584	0.642146630374431\\
0.98	0.105010304312169	0.446086812865616	0.640742567819433\\
0.98	0.107424267371012	0.450897506670668	0.639352101193729\\
0.98	0.109866204559871	0.455699271686218	0.637974939582444\\
0.98	0.112336078637817	0.460491870408058	0.636610827579863\\
0.98	0.114833849738607	0.465275065344603	0.635259539104383\\
0.98	0.117359475365564	0.470048619052377	0.633920872153153\\
0.98	0.119912910387022	0.474812294171664	0.632594644371965\\
0.98	0.122494107032347	0.479565853462319	0.631280689329643\\
0.98	0.125103014888515	0.484309059839721	0.629978853399059\\
0.98	0.127739580897283	0.489041676410868	0.628690460824557\\
0.985	0	0	0.710202759607373\\
0.985	1.11327767495586e-05	0.00471862581271275	0.713003157837853\\
0.985	4.46175251031912e-05	0.009446325184051	0.715846486660142\\
0.985	0.000100583311362513	0.0141829653360114	0.718788838379794\\
0.985	0.000179158434668431	0.018928411755669	0.721886092228357\\
0.985	0.000280470402701511	0.02368252819604	0.725200636220892\\
0.985	0.000404645907256436	0.0284451766772965	0.728799085274723\\
0.985	0.000551810799695644	0.0332162174883388	0.732749123983336\\
0.985	0.000722090066287311	0.037995509188729	0.737115575253327\\
0.985	0.000915607803432999	0.0427829086109896	0.741955893907011\\
0.985	0.0011324871927904	0.047578270863273	0.747315373652289\\
0.985	0.00137285047629673	0.0523814493324038	0.75322242404581\\
0.985	0.00163681893109844	0.057192295687301	0.759684310525653\\
0.985	0.00192451284439304	0.0620106598827802	0.766683748102665\\
0.985	0.00223605148818898	0.0668363901637434	0.774176695511256\\
0.985	0.00257155309398959	0.0716693330697584	0.782091614494494\\
0.985	0.00293113482740716	0.0765093334400312	0.79033034642109\\
0.985	0.00331491276271365	0.0813562344187764	0.79877062752665\\
0.985	0.00372300185733413	0.0862098774609879	0.80727012893036\\
0.985	0.00415551592628969	0.0910701023386145	0.815671782700193\\
0.985	0.00461256761659621	0.0959367471471425	0.82381005350733\\
0.985	0.00509426838162598	0.100809648312589	0.831517746503354\\
0.985	0.00560072845543873	0.105688640598912	0.838632911419452\\
0.985	0.00613205682708918	0.11057355711583	0.845005411376292\\
0.985	0.00668836121491815	0.115464229327074	0.850502769049302\\
0.985	0.00726974804083431	0.120360487059049	0.855014975717657\\
0.985	0.00787632240459385	0.125262158509929	0.858458041096606\\
0.985	0.00850818805808555	0.13016907025918	0.860776163494182\\
0.985	0.00916544737962849	0.135081047277508	0.861942500814224\\
0.985	0.00984820134829017	0.139997912937243	0.861958614648422\\
0.985	0.0105565495182326	0.144919489023162	0.860852735617091\\
0.985	0.0112905899930939	0.149845595743738	0.85867705410965\\
0.985	0.012050419400414	0.154776051742839	0.855504274967457\\
0.985	0.0128361328661109	0.159710674111862	0.851423687967753\\
0.985	0.0136478239890177	0.164649278402306	0.846537000484341\\
0.985	0.0144855848154856	0.169591678638796	0.840954157904249\\
0.985	0.0153495058140643	0.174537687332543	0.834789345403934\\
0.985	0.0162396758502651	0.179487115495259	0.8281573258062\\
0.985	0.0171561821614173	0.184439772653511	0.821170226448694\\
0.985	0.0180991103316243	0.189395466863524	0.813934846725693\\
0.985	0.0190685442668299	0.194354004726437	0.806550519904213\\
0.985	0.0200645661700016	0.199315191403999	0.799107529846306\\
0.985	0.0210872565164405	0.204278830634725	0.791686056496194\\
0.985	0.0221366940292259	0.20924472475049	0.784355603830602\\
0.985	0.023212955654804	0.214212674693577	0.777174850281118\\
0.985	0.0243161165387281	0.219182480034174	0.770191853862643\\
0.985	0.025446250001561	0.224153938988321	0.763444541554458\\
0.985	0.0266034275149466	0.229126848436298	0.756961413910483\\
0.985	0.0277877186778607	0.234101003941464	0.750762400416763\\
0.985	0.0289991911930496	0.239076199769547	0.744859807805785\\
0.985	0.0302379108436654	0.244052228908366	0.739259311518081\\
0.985	0.0315039414701067	0.24902888308801	0.733960949044503\\
0.985	0.0327973449470747	0.254005952801449	0.728960082407261\\
0.985	0.0341181811608519	0.258983227325592	0.724248305111177\\
0.985	0.0354665079868145	0.263960494742775	0.71981427622068\\
0.985	0.0368423812671862	0.268937541962694	0.71564447061263\\
0.985	0.0382458547890421	0.273914154744766	0.71172383983804\\
0.985	0.0396769802625738	0.278890117720924	0.708036382389731\\
0.985	0.0411358072996216	0.283865214418836	0.704565625567165\\
0.985	0.0426223833924861	0.288839227285554	0.701295023640316\\
0.985	0.0441367538930258	0.293811937711588	0.698208278749354\\
0.985	0.0456789619920503	0.298783126055387	0.695289592055349\\
0.985	0.0472490486990192	0.303752571668251	0.692523853197512\\
0.985	0.0488470528220537	0.308720052919643	0.689896776228119\\
0.985	0.0504730109482718	0.313685347222913	0.687394989990282\\
0.985	0.0521269574244539	0.318648231061428	0.685006090465981\\
0.985	0.0538089243380495	0.323608480015097	0.682718662029127\\
0.985	0.0555189414985328	0.328565868787292	0.680522273853469\\
0.985	0.0572570364191156	0.333520171232163	0.67840745699785\\
0.985	0.0590232342988274	0.338471160382329	0.676365666960043\\
0.985	0.0608175580049697	0.34341860847696	0.674389235783156\\
0.985	0.0626400280559541	0.348362286990222	0.672471317134862\\
0.985	0.064490662604533	0.3533019666601	0.670605827172183\\
0.985	0.0663694774214291	0.358237417517579	0.668787383459804\\
0.985	0.0682764858793752	0.363168408916186	0.667011243730728\\
0.985	0.0702116989375697	0.368094709561873	0.665273245863648\\
0.985	0.0721751251265572	0.373016087543259	0.663569750098931\\
0.985	0.0741667705335422	0.377932310362199	0.66189758422006\\
0.985	0.0761866387881432	0.382843144964686	0.660253992184442\\
0.985	0.0782347310485954	0.38774835777208	0.658636586490997\\
0.985	0.0803110459884098	0.392647714712651	0.65704330441587\\
0.985	0.0824155797834956	0.397540981253432	0.655472368126509\\
0.985	0.084548326099754	0.402427922432375	0.653922248592746\\
0.985	0.0867092760811502	0.4073083028908	0.652391633146569\\
0.985	0.0888984183382709	0.412181886906127	0.650879396495579\\
0.985	0.0911157389373742	0.417048438424896	0.649384574964845\\
0.985	0.0933612213899392	0.421907721096044	0.647906343724685\\
0.985	0.0956348466427212	0.42675949830445	0.646443996754805\\
0.985	0.0979365930683194	0.431603533204733	0.644996929295963\\
0.985	0.100266436456264	0.436439588755293	0.643564622546847\\
0.985	0.102624350004627	0.441267427752585	0.642146630374429\\
0.985	0.105010304312169	0.446086812865616	0.640742567819434\\
0.985	0.107424267371012	0.450897506670668	0.639352101193731\\
0.985	0.109866204559871	0.455699271686218	0.637974939582442\\
0.985	0.112336078637817	0.460491870408058	0.636610827579861\\
0.985	0.114833849738607	0.465275065344603	0.635259539104384\\
0.985	0.117359475365564	0.470048619052377	0.633920872153156\\
0.985	0.119912910387023	0.474812294171664	0.632594644371963\\
0.985	0.122494107032347	0.479565853462319	0.63128068932964\\
0.985	0.125103014888515	0.484309059839721	0.629978853399058\\
0.985	0.127739580897283	0.489041676410868	0.628690460824556\\
0.99	0	0	0.710202759607373\\
0.99	1.11327767495586e-05	0.00471862581271275	0.713003157837853\\
0.99	4.46175251031912e-05	0.009446325184051	0.715846486660142\\
0.99	0.000100583311362513	0.0141829653360114	0.718788838379794\\
0.99	0.000179158434668431	0.018928411755669	0.721886092228356\\
0.99	0.000280470402701511	0.02368252819604	0.725200636220892\\
0.99	0.000404645907256436	0.0284451766772965	0.728799085274723\\
0.99	0.000551810799695644	0.0332162174883388	0.732749123983336\\
0.99	0.000722090066287311	0.037995509188729	0.737115575253327\\
0.99	0.000915607803432999	0.0427829086109896	0.741955893907011\\
0.99	0.0011324871927904	0.0475782708632729	0.747315373652289\\
0.99	0.00137285047629673	0.0523814493324038	0.753222424045811\\
0.99	0.00163681893109844	0.057192295687301	0.759684310525652\\
0.99	0.00192451284439304	0.0620106598827802	0.766683748102665\\
0.99	0.00223605148818899	0.0668363901637434	0.774176695511256\\
0.99	0.00257155309398959	0.0716693330697584	0.782091614494494\\
0.99	0.00293113482740716	0.0765093334400312	0.79033034642109\\
0.99	0.00331491276271365	0.0813562344187764	0.79877062752665\\
0.99	0.00372300185733414	0.0862098774609879	0.807270128930359\\
0.99	0.00415551592628969	0.0910701023386145	0.815671782700194\\
0.99	0.00461256761659621	0.0959367471471425	0.82381005350733\\
0.99	0.00509426838162598	0.100809648312589	0.831517746503353\\
0.99	0.00560072845543873	0.105688640598912	0.838632911419452\\
0.99	0.00613205682708918	0.11057355711583	0.845005411376293\\
0.99	0.00668836121491816	0.115464229327074	0.850502769049301\\
0.99	0.00726974804083431	0.120360487059049	0.855014975717657\\
0.99	0.00787632240459385	0.125262158509929	0.858458041096605\\
0.99	0.00850818805808555	0.13016907025918	0.860776163494183\\
0.99	0.00916544737962849	0.135081047277508	0.861942500814224\\
0.99	0.00984820134829017	0.139997912937243	0.861958614648421\\
0.99	0.0105565495182326	0.144919489023162	0.860852735617092\\
0.99	0.0112905899930939	0.149845595743738	0.85867705410965\\
0.99	0.012050419400414	0.154776051742839	0.855504274967457\\
0.99	0.0128361328661109	0.159710674111862	0.851423687967752\\
0.99	0.0136478239890177	0.164649278402306	0.846537000484342\\
0.99	0.0144855848154856	0.169591678638796	0.840954157904247\\
0.99	0.0153495058140643	0.174537687332543	0.834789345403934\\
0.99	0.0162396758502651	0.179487115495259	0.828157325806199\\
0.99	0.0171561821614173	0.184439772653511	0.821170226448694\\
0.99	0.0180991103316243	0.189395466863524	0.813934846725692\\
0.99	0.0190685442668299	0.194354004726437	0.806550519904213\\
0.99	0.0200645661700016	0.199315191403999	0.799107529846304\\
0.99	0.0210872565164405	0.204278830634725	0.791686056496192\\
0.99	0.0221366940292259	0.20924472475049	0.7843556038306\\
0.99	0.023212955654804	0.214212674693577	0.777174850281116\\
0.99	0.0243161165387281	0.219182480034174	0.770191853862643\\
0.99	0.025446250001561	0.224153938988321	0.763444541554458\\
0.99	0.0266034275149466	0.229126848436298	0.756961413910482\\
0.99	0.0277877186778607	0.234101003941464	0.750762400416764\\
0.99	0.0289991911930496	0.239076199769547	0.744859807805785\\
0.99	0.0302379108436654	0.244052228908366	0.739259311518079\\
0.99	0.0315039414701067	0.24902888308801	0.733960949044504\\
0.99	0.0327973449470747	0.254005952801449	0.72896008240726\\
0.99	0.0341181811608519	0.258983227325591	0.724248305111178\\
0.99	0.0354665079868145	0.263960494742775	0.719814276220682\\
0.99	0.0368423812671862	0.268937541962694	0.715644470612629\\
0.99	0.0382458547890421	0.273914154744766	0.711723839838042\\
0.99	0.0396769802625738	0.278890117720924	0.70803638238973\\
0.99	0.0411358072996216	0.283865214418836	0.704565625567164\\
0.99	0.0426223833924861	0.288839227285554	0.701295023640316\\
0.99	0.0441367538930258	0.293811937711588	0.698208278749355\\
0.99	0.0456789619920503	0.298783126055387	0.695289592055349\\
0.99	0.0472490486990192	0.303752571668251	0.692523853197512\\
0.99	0.0488470528220538	0.308720052919643	0.689896776228119\\
0.99	0.0504730109482718	0.313685347222913	0.68739498999028\\
0.99	0.0521269574244539	0.318648231061428	0.685006090465981\\
0.99	0.0538089243380495	0.323608480015096	0.682718662029128\\
0.99	0.0555189414985328	0.328565868787292	0.680522273853469\\
0.99	0.0572570364191156	0.333520171232163	0.678407456997851\\
0.99	0.0590232342988274	0.338471160382329	0.676365666960043\\
0.99	0.0608175580049697	0.34341860847696	0.674389235783155\\
0.99	0.0626400280559541	0.348362286990222	0.672471317134862\\
0.99	0.064490662604533	0.3533019666601	0.670605827172183\\
0.99	0.0663694774214291	0.358237417517579	0.668787383459805\\
0.99	0.0682764858793752	0.363168408916186	0.667011243730728\\
0.99	0.0702116989375697	0.368094709561873	0.66527324586365\\
0.99	0.0721751251265572	0.373016087543259	0.663569750098932\\
0.99	0.0741667705335423	0.377932310362199	0.661897584220059\\
0.99	0.0761866387881432	0.382843144964686	0.660253992184443\\
0.99	0.0782347310485954	0.38774835777208	0.658636586490997\\
0.99	0.0803110459884098	0.392647714712651	0.657043304415871\\
0.99	0.0824155797834956	0.397540981253432	0.655472368126508\\
0.99	0.084548326099754	0.402427922432375	0.653922248592746\\
0.99	0.0867092760811502	0.4073083028908	0.65239163314657\\
0.99	0.0888984183382708	0.412181886906127	0.650879396495579\\
0.99	0.0911157389373742	0.417048438424896	0.649384574964843\\
0.99	0.0933612213899392	0.421907721096044	0.647906343724685\\
0.99	0.0956348466427212	0.42675949830445	0.646443996754805\\
0.99	0.0979365930683194	0.431603533204733	0.644996929295961\\
0.99	0.100266436456264	0.436439588755293	0.643564622546847\\
0.99	0.102624350004627	0.441267427752584	0.642146630374431\\
0.99	0.105010304312169	0.446086812865616	0.640742567819433\\
0.99	0.107424267371012	0.450897506670668	0.639352101193731\\
0.99	0.109866204559871	0.455699271686218	0.637974939582444\\
0.99	0.112336078637817	0.460491870408058	0.636610827579862\\
0.99	0.114833849738607	0.465275065344603	0.635259539104381\\
0.99	0.117359475365564	0.470048619052377	0.633920872153154\\
0.99	0.119912910387022	0.474812294171664	0.632594644371967\\
0.99	0.122494107032347	0.479565853462319	0.631280689329642\\
0.99	0.125103014888515	0.484309059839721	0.629978853399059\\
0.99	0.127739580897283	0.489041676410868	0.628690460824557\\
0.995	0	0	0.710202759607373\\
0.995	1.11327767495586e-05	0.00471862581271274	0.713003157837853\\
0.995	4.46175251031912e-05	0.009446325184051	0.715846486660142\\
0.995	0.000100583311362513	0.0141829653360114	0.718788838379794\\
0.995	0.000179158434668431	0.018928411755669	0.721886092228356\\
0.995	0.000280470402701511	0.02368252819604	0.725200636220892\\
0.995	0.000404645907256436	0.0284451766772965	0.728799085274723\\
0.995	0.000551810799695644	0.0332162174883389	0.732749123983336\\
0.995	0.000722090066287311	0.037995509188729	0.737115575253328\\
0.995	0.000915607803432999	0.0427829086109896	0.741955893907011\\
0.995	0.0011324871927904	0.0475782708632729	0.747315373652289\\
0.995	0.00137285047629673	0.0523814493324038	0.753222424045811\\
0.995	0.00163681893109844	0.057192295687301	0.759684310525653\\
0.995	0.00192451284439304	0.0620106598827802	0.766683748102665\\
0.995	0.00223605148818899	0.0668363901637434	0.774176695511256\\
0.995	0.00257155309398959	0.0716693330697584	0.782091614494494\\
0.995	0.00293113482740716	0.0765093334400312	0.79033034642109\\
0.995	0.00331491276271365	0.0813562344187764	0.79877062752665\\
0.995	0.00372300185733414	0.0862098774609879	0.807270128930359\\
0.995	0.00415551592628969	0.0910701023386145	0.815671782700193\\
0.995	0.00461256761659621	0.0959367471471425	0.82381005350733\\
0.995	0.00509426838162598	0.100809648312589	0.831517746503354\\
0.995	0.00560072845543873	0.105688640598912	0.838632911419452\\
0.995	0.00613205682708918	0.11057355711583	0.845005411376293\\
0.995	0.00668836121491815	0.115464229327074	0.850502769049302\\
0.995	0.00726974804083431	0.120360487059049	0.855014975717657\\
0.995	0.00787632240459385	0.125262158509929	0.858458041096606\\
0.995	0.00850818805808555	0.13016907025918	0.860776163494183\\
0.995	0.00916544737962849	0.135081047277508	0.861942500814225\\
0.995	0.00984820134829017	0.139997912937243	0.861958614648421\\
0.995	0.0105565495182326	0.144919489023162	0.86085273561709\\
0.995	0.0112905899930939	0.149845595743738	0.858677054109649\\
0.995	0.012050419400414	0.154776051742839	0.855504274967456\\
0.995	0.0128361328661109	0.159710674111862	0.851423687967757\\
0.995	0.0136478239890177	0.164649278402306	0.846537000484341\\
0.995	0.0144855848154856	0.169591678638796	0.840954157904251\\
0.995	0.0153495058140643	0.174537687332543	0.834789345403936\\
0.995	0.0162396758502651	0.179487115495259	0.828157325806201\\
0.995	0.0171561821614173	0.184439772653511	0.821170226448695\\
0.995	0.0180991103316243	0.189395466863524	0.813934846725693\\
0.995	0.0190685442668299	0.194354004726437	0.806550519904213\\
0.995	0.0200645661700016	0.199315191403999	0.799107529846303\\
0.995	0.0210872565164405	0.204278830634725	0.791686056496192\\
0.995	0.0221366940292259	0.20924472475049	0.784355603830601\\
0.995	0.023212955654804	0.214212674693577	0.77717485028112\\
0.995	0.0243161165387281	0.219182480034174	0.770191853862644\\
0.995	0.025446250001561	0.224153938988321	0.76344454155446\\
0.995	0.0266034275149466	0.229126848436298	0.756961413910484\\
0.995	0.0277877186778607	0.234101003941464	0.750762400416764\\
0.995	0.0289991911930496	0.239076199769547	0.744859807805784\\
0.995	0.0302379108436654	0.244052228908366	0.739259311518081\\
0.995	0.0315039414701067	0.24902888308801	0.733960949044505\\
0.995	0.0327973449470747	0.254005952801449	0.728960082407261\\
0.995	0.0341181811608519	0.258983227325592	0.724248305111178\\
0.995	0.0354665079868145	0.263960494742775	0.71981427622068\\
0.995	0.0368423812671862	0.268937541962694	0.71564447061263\\
0.995	0.0382458547890421	0.273914154744766	0.711723839838042\\
0.995	0.0396769802625738	0.278890117720924	0.708036382389729\\
0.995	0.0411358072996216	0.283865214418836	0.704565625567164\\
0.995	0.0426223833924861	0.288839227285554	0.701295023640316\\
0.995	0.0441367538930258	0.293811937711588	0.698208278749354\\
0.995	0.0456789619920503	0.298783126055387	0.695289592055347\\
0.995	0.0472490486990192	0.303752571668251	0.692523853197511\\
0.995	0.0488470528220538	0.308720052919643	0.68989677622812\\
0.995	0.0504730109482718	0.313685347222913	0.687394989990282\\
0.995	0.0521269574244539	0.318648231061428	0.685006090465981\\
0.995	0.0538089243380495	0.323608480015096	0.682718662029128\\
0.995	0.0555189414985328	0.328565868787292	0.68052227385347\\
0.995	0.0572570364191156	0.333520171232163	0.67840745699785\\
0.995	0.0590232342988274	0.338471160382329	0.676365666960043\\
0.995	0.0608175580049697	0.34341860847696	0.674389235783155\\
0.995	0.0626400280559541	0.348362286990221	0.672471317134861\\
0.995	0.064490662604533	0.3533019666601	0.670605827172183\\
0.995	0.0663694774214291	0.358237417517579	0.668787383459805\\
0.995	0.0682764858793752	0.363168408916186	0.667011243730728\\
0.995	0.0702116989375697	0.368094709561873	0.66527324586365\\
0.995	0.0721751251265572	0.373016087543259	0.663569750098933\\
0.995	0.0741667705335422	0.377932310362199	0.66189758422006\\
0.995	0.0761866387881432	0.382843144964686	0.660253992184441\\
0.995	0.0782347310485954	0.38774835777208	0.658636586490998\\
0.995	0.0803110459884098	0.392647714712651	0.657043304415871\\
0.995	0.0824155797834956	0.397540981253432	0.655472368126509\\
0.995	0.084548326099754	0.402427922432375	0.653922248592745\\
0.995	0.0867092760811502	0.407308302890799	0.65239163314657\\
0.995	0.0888984183382708	0.412181886906127	0.650879396495579\\
0.995	0.0911157389373742	0.417048438424896	0.649384574964843\\
0.995	0.0933612213899392	0.421907721096044	0.647906343724685\\
0.995	0.0956348466427212	0.42675949830445	0.646443996754806\\
0.995	0.0979365930683194	0.431603533204733	0.644996929295961\\
0.995	0.100266436456264	0.436439588755293	0.643564622546846\\
0.995	0.102624350004627	0.441267427752584	0.64214663037443\\
0.995	0.105010304312169	0.446086812865616	0.640742567819433\\
0.995	0.107424267371012	0.450897506670668	0.63935210119373\\
0.995	0.109866204559871	0.455699271686218	0.637974939582444\\
0.995	0.112336078637817	0.460491870408058	0.636610827579863\\
0.995	0.114833849738607	0.465275065344603	0.635259539104382\\
0.995	0.117359475365564	0.470048619052377	0.633920872153153\\
0.995	0.119912910387022	0.474812294171664	0.632594644371964\\
0.995	0.122494107032347	0.479565853462319	0.631280689329642\\
0.995	0.125103014888515	0.484309059839721	0.62997885339906\\
0.995	0.127739580897283	0.489041676410868	0.628690460824559\\
1	0	0	0.710202759607373\\
1	1.11327767495586e-05	0.00471862581271275	0.713003157837853\\
1	4.46175251031912e-05	0.009446325184051	0.715846486660142\\
1	0.000100583311362513	0.0141829653360114	0.718788838379794\\
1	0.000179158434668431	0.018928411755669	0.721886092228356\\
1	0.000280470402701511	0.02368252819604	0.725200636220892\\
1	0.000404645907256436	0.0284451766772965	0.728799085274723\\
1	0.000551810799695644	0.0332162174883388	0.732749123983336\\
1	0.000722090066287311	0.037995509188729	0.737115575253327\\
1	0.000915607803432999	0.0427829086109896	0.741955893907011\\
1	0.0011324871927904	0.047578270863273	0.747315373652289\\
1	0.00137285047629673	0.0523814493324038	0.75322242404581\\
1	0.00163681893109844	0.057192295687301	0.759684310525652\\
1	0.00192451284439304	0.0620106598827802	0.766683748102665\\
1	0.00223605148818898	0.0668363901637434	0.774176695511256\\
1	0.00257155309398959	0.0716693330697584	0.782091614494494\\
1	0.00293113482740716	0.0765093334400312	0.79033034642109\\
1	0.00331491276271365	0.0813562344187764	0.79877062752665\\
1	0.00372300185733413	0.0862098774609879	0.807270128930359\\
1	0.00415551592628969	0.0910701023386145	0.815671782700192\\
1	0.00461256761659621	0.0959367471471425	0.823810053507328\\
1	0.00509426838162598	0.100809648312589	0.831517746503355\\
1	0.00560072845543873	0.105688640598912	0.838632911419453\\
1	0.00613205682708918	0.11057355711583	0.845005411376294\\
1	0.00668836121491816	0.115464229327074	0.850502769049302\\
1	0.00726974804083431	0.120360487059049	0.855014975717657\\
1	0.00787632240459385	0.125262158509929	0.858458041096606\\
1	0.00850818805808554	0.13016907025918	0.860776163494183\\
1	0.00916544737962849	0.135081047277508	0.861942500814227\\
1	0.00984820134829017	0.139997912937243	0.86195861464842\\
1	0.0105565495182326	0.144919489023162	0.86085273561709\\
1	0.0112905899930939	0.149845595743738	0.858677054109648\\
1	0.012050419400414	0.154776051742839	0.855504274967455\\
1	0.0128361328661109	0.159710674111862	0.851423687967763\\
1	0.0136478239890177	0.164649278402306	0.846537000484335\\
1	0.0144855848154856	0.169591678638796	0.840954157904254\\
1	0.0153495058140643	0.174537687332543	0.834789345403943\\
1	0.0162396758502651	0.179487115495259	0.828157325806207\\
1	0.0171561821614173	0.184439772653511	0.821170226448698\\
1	0.0180991103316243	0.189395466863524	0.813934846725698\\
1	0.0190685442668299	0.194354004726437	0.806550519904213\\
1	0.0200645661700016	0.199315191403999	0.7991075298463\\
1	0.0210872565164405	0.204278830634725	0.791686056496194\\
1	0.0221366940292259	0.20924472475049	0.784355603830603\\
1	0.023212955654804	0.214212674693577	0.77717485028112\\
1	0.0243161165387281	0.219182480034174	0.770191853862642\\
1	0.025446250001561	0.224153938988321	0.76344454155446\\
1	0.0266034275149466	0.229126848436298	0.756961413910482\\
1	0.0277877186778607	0.234101003941464	0.750762400416763\\
1	0.0289991911930496	0.239076199769547	0.744859807805785\\
1	0.0302379108436654	0.244052228908366	0.739259311518082\\
1	0.0315039414701067	0.24902888308801	0.733960949044504\\
1	0.0327973449470747	0.254005952801449	0.72896008240726\\
1	0.0341181811608519	0.258983227325592	0.724248305111177\\
1	0.0354665079868145	0.263960494742775	0.71981427622068\\
1	0.0368423812671862	0.268937541962694	0.71564447061263\\
1	0.0382458547890421	0.273914154744766	0.711723839838042\\
1	0.0396769802625738	0.278890117720924	0.708036382389732\\
1	0.0411358072996216	0.283865214418836	0.704565625567164\\
1	0.0426223833924861	0.288839227285554	0.701295023640316\\
1	0.0441367538930258	0.293811937711588	0.698208278749354\\
1	0.0456789619920503	0.298783126055387	0.69528959205535\\
1	0.0472490486990192	0.303752571668251	0.692523853197511\\
1	0.0488470528220538	0.308720052919643	0.689896776228118\\
1	0.0504730109482718	0.313685347222913	0.687394989990281\\
1	0.0521269574244539	0.318648231061428	0.685006090465981\\
1	0.0538089243380495	0.323608480015097	0.682718662029127\\
1	0.0555189414985328	0.328565868787292	0.680522273853469\\
1	0.0572570364191156	0.333520171232163	0.678407456997852\\
1	0.0590232342988274	0.338471160382329	0.676365666960042\\
1	0.0608175580049697	0.34341860847696	0.674389235783156\\
1	0.0626400280559541	0.348362286990222	0.672471317134859\\
1	0.064490662604533	0.3533019666601	0.670605827172181\\
1	0.0663694774214291	0.358237417517579	0.668787383459803\\
1	0.0682764858793752	0.363168408916186	0.667011243730729\\
1	0.0702116989375697	0.368094709561873	0.665273245863648\\
1	0.0721751251265572	0.373016087543259	0.663569750098933\\
1	0.0741667705335423	0.377932310362199	0.661897584220062\\
1	0.0761866387881432	0.382843144964686	0.660253992184441\\
1	0.0782347310485954	0.38774835777208	0.658636586490997\\
1	0.0803110459884098	0.392647714712651	0.657043304415871\\
1	0.0824155797834956	0.397540981253432	0.655472368126511\\
1	0.084548326099754	0.402427922432375	0.653922248592746\\
1	0.0867092760811502	0.4073083028908	0.652391633146568\\
1	0.0888984183382709	0.412181886906127	0.650879396495577\\
1	0.0911157389373742	0.417048438424897	0.649384574964844\\
1	0.0933612213899392	0.421907721096044	0.647906343724682\\
1	0.0956348466427212	0.42675949830445	0.646443996754806\\
1	0.0979365930683194	0.431603533204733	0.644996929295963\\
1	0.100266436456264	0.436439588755293	0.643564622546846\\
1	0.102624350004627	0.441267427752584	0.642146630374429\\
1	0.105010304312169	0.446086812865616	0.640742567819434\\
1	0.107424267371012	0.450897506670668	0.63935210119373\\
1	0.109866204559871	0.455699271686218	0.637974939582442\\
1	0.112336078637817	0.460491870408058	0.636610827579863\\
1	0.114833849738607	0.465275065344603	0.635259539104384\\
1	0.117359475365564	0.470048619052377	0.633920872153155\\
1	0.119912910387022	0.474812294171664	0.632594644371962\\
1	0.122494107032347	0.479565853462319	0.63128068932964\\
1	0.125103014888515	0.484309059839721	0.629978853399059\\
1	0.127739580897283	0.489041676410868	0.628690460824559\\
};
\addplot3 [color=mycolor1,solid,line width=1.5pt]
 table[row sep=crcr] {%
0	0	0\\
0.015	0	0\\
0.03	0	0\\
0.045	0	0\\
0.06	0	0\\
0.075	0	0\\
0.09	0	0\\
0.105	0	0\\
0.12	0	0\\
0.135	0	0\\
0.15	0	0\\
0.165	0	0\\
0.18	0	0\\
0.195	0	0\\
0.21	0	0\\
0.225	0	0\\
0.24	0	0\\
0.255	0	0\\
0.27	0	0\\
0.285	0	0\\
0.3	0	0\\
0.315	0	0\\
0.33	0	0\\
0.345	0	0\\
0.36	0	0\\
0.375	0	0\\
0.39	0	0\\
0.405	0	0\\
0.42	0	0\\
0.435	0	0\\
0.45	0	0\\
0.465	0	0\\
0.48	0	0\\
0.495	0	0\\
0.51	0	0\\
0.525	0	0\\
0.54	0	0\\
0.555	0	0\\
0.57	0	0\\
0.585	0	0\\
0.6	0	0\\
0.615	0	0\\
0.63	0	0\\
0.645	0	0\\
0.66	0	0\\
0.675	0	0\\
0.69	0	0\\
0.705	0	0\\
0.72	0	0\\
0.735	0	0\\
0.75	0	0\\
0.765	0	0\\
0.78	0	0\\
0.795	0	0\\
0.81	0	0\\
0.825	0	0\\
0.84	0	0\\
0.855	0	0\\
0.87	0	0\\
0.885	0	0\\
0.9	0	0\\
0.915	0	0\\
0.93	0	0\\
0.945	0	0\\
0.96	0	0\\
0.975	0	0\\
0.99	0	0\\
1.005	0	0\\
1.02	0	0\\
1.035	0	0\\
1.05	0	0\\
1.065	0	0\\
1.08	0	0\\
1.095	0	0\\
1.11	0	0\\
1.125	0	0\\
1.14	0	0\\
1.155	0	0\\
1.17	0	0\\
1.185	0	0\\
1.2	0	0\\
1.215	0	0\\
1.23	0	0\\
1.245	0	0\\
1.26	0	0\\
1.275	0	0\\
1.29	0	0\\
1.305	0	0\\
1.32	0	0\\
1.335	0	0\\
1.35	0	0\\
1.365	0	0\\
1.38	0	0\\
1.395	0	0\\
1.41	0	0\\
1.425	0	0\\
1.44	0	0\\
1.455	0	0\\
1.47	0	0\\
1.485	0	0\\
1.5	0	0\\
};
 \addplot3 [color=mycolor1,solid,line width=1.5pt]
 table[row sep=crcr] {%
0	0.127739580897283	0.489041676410868\\
0.015	0.127739580897283	0.489041676410868\\
0.03	0.127739580897283	0.489041676410868\\
0.045	0.127739580897283	0.489041676410868\\
0.06	0.127739580897283	0.489041676410868\\
0.075	0.127739580897283	0.489041676410868\\
0.09	0.127739580897283	0.489041676410868\\
0.105	0.127739580897283	0.489041676410868\\
0.12	0.127739580897283	0.489041676410868\\
0.135	0.127739580897283	0.489041676410868\\
0.15	0.127739580897283	0.489041676410868\\
0.165	0.127739580897283	0.489041676410868\\
0.18	0.127739580897283	0.489041676410868\\
0.195	0.127739580897283	0.489041676410868\\
0.21	0.127739580897283	0.489041676410868\\
0.225	0.127739580897283	0.489041676410868\\
0.24	0.127739580897283	0.489041676410868\\
0.255	0.127739580897283	0.489041676410868\\
0.27	0.127739580897283	0.489041676410868\\
0.285	0.127739580897283	0.489041676410868\\
0.3	0.127739580897283	0.489041676410868\\
0.315	0.127739580897283	0.489041676410868\\
0.33	0.127739580897283	0.489041676410868\\
0.345	0.127739580897283	0.489041676410868\\
0.36	0.127739580897283	0.489041676410868\\
0.375	0.127739580897283	0.489041676410868\\
0.39	0.127739580897283	0.489041676410868\\
0.405	0.127739580897283	0.489041676410868\\
0.42	0.127739580897283	0.489041676410868\\
0.435	0.127739580897283	0.489041676410868\\
0.45	0.127739580897283	0.489041676410868\\
0.465	0.127739580897283	0.489041676410868\\
0.48	0.127739580897283	0.489041676410868\\
0.495	0.127739580897283	0.489041676410868\\
0.51	0.127739580897283	0.489041676410868\\
0.525	0.127739580897283	0.489041676410868\\
0.54	0.127739580897283	0.489041676410868\\
0.555	0.127739580897283	0.489041676410868\\
0.57	0.127739580897283	0.489041676410868\\
0.585	0.127739580897283	0.489041676410868\\
0.6	0.127739580897283	0.489041676410868\\
0.615	0.127739580897283	0.489041676410868\\
0.63	0.127739580897283	0.489041676410868\\
0.645	0.127739580897283	0.489041676410868\\
0.66	0.127739580897283	0.489041676410868\\
0.675	0.127739580897283	0.489041676410868\\
0.69	0.127739580897283	0.489041676410868\\
0.705	0.127739580897283	0.489041676410868\\
0.72	0.127739580897283	0.489041676410868\\
0.735	0.127739580897283	0.489041676410868\\
0.75	0.127739580897283	0.489041676410868\\
0.765	0.127739580897283	0.489041676410868\\
0.78	0.127739580897283	0.489041676410868\\
0.795	0.127739580897283	0.489041676410868\\
0.81	0.127739580897283	0.489041676410868\\
0.825	0.127739580897283	0.489041676410868\\
0.84	0.127739580897283	0.489041676410868\\
0.855	0.127739580897283	0.489041676410868\\
0.87	0.127739580897283	0.489041676410868\\
0.885	0.127739580897283	0.489041676410868\\
0.9	0.127739580897283	0.489041676410868\\
0.915	0.127739580897283	0.489041676410868\\
0.93	0.127739580897283	0.489041676410868\\
0.945	0.127739580897283	0.489041676410868\\
0.96	0.127739580897283	0.489041676410868\\
0.975	0.127739580897283	0.489041676410868\\
0.99	0.127739580897283	0.489041676410868\\
1.005	0.127739580897283	0.489041676410868\\
1.02	0.127739580897283	0.489041676410868\\
1.035	0.127739580897283	0.489041676410868\\
1.05	0.127739580897283	0.489041676410868\\
1.065	0.127739580897283	0.489041676410868\\
1.08	0.127739580897283	0.489041676410868\\
1.095	0.127739580897283	0.489041676410868\\
1.11	0.127739580897283	0.489041676410868\\
1.125	0.127739580897283	0.489041676410868\\
1.14	0.127739580897283	0.489041676410868\\
1.155	0.127739580897283	0.489041676410868\\
1.17	0.127739580897283	0.489041676410868\\
1.185	0.127739580897283	0.489041676410868\\
1.2	0.127739580897283	0.489041676410868\\
1.215	0.127739580897283	0.489041676410868\\
1.23	0.127739580897283	0.489041676410868\\
1.245	0.127739580897283	0.489041676410868\\
1.26	0.127739580897283	0.489041676410868\\
1.275	0.127739580897283	0.489041676410868\\
1.29	0.127739580897283	0.489041676410868\\
1.305	0.127739580897283	0.489041676410868\\
1.32	0.127739580897283	0.489041676410868\\
1.335	0.127739580897283	0.489041676410868\\
1.35	0.127739580897283	0.489041676410868\\
1.365	0.127739580897283	0.489041676410868\\
1.38	0.127739580897283	0.489041676410868\\
1.395	0.127739580897283	0.489041676410868\\
1.41	0.127739580897283	0.489041676410868\\
1.425	0.127739580897283	0.489041676410868\\
1.44	0.127739580897283	0.489041676410868\\
1.455	0.127739580897283	0.489041676410868\\
1.47	0.127739580897283	0.489041676410868\\
1.485	0.127739580897283	0.489041676410868\\
1.5	0.127739580897283	0.489041676410868\\
};
 \addplot3 [color=mycolor1,solid,line width=1.5pt]
 table[row sep=crcr] {%
0	0.510958323589132	0.872260419102717\\
0.015	0.510958323589132	0.872260419102717\\
0.03	0.510958323589132	0.872260419102717\\
0.045	0.510958323589132	0.872260419102717\\
0.06	0.510958323589132	0.872260419102717\\
0.075	0.510958323589132	0.872260419102717\\
0.09	0.510958323589132	0.872260419102717\\
0.105	0.510958323589132	0.872260419102717\\
0.12	0.510958323589132	0.872260419102717\\
0.135	0.510958323589132	0.872260419102717\\
0.15	0.510958323589132	0.872260419102717\\
0.165	0.510958323589132	0.872260419102717\\
0.18	0.510958323589132	0.872260419102717\\
0.195	0.510958323589132	0.872260419102717\\
0.21	0.510958323589132	0.872260419102717\\
0.225	0.510958323589132	0.872260419102717\\
0.24	0.510958323589132	0.872260419102717\\
0.255	0.510958323589132	0.872260419102717\\
0.27	0.510958323589132	0.872260419102717\\
0.285	0.510958323589132	0.872260419102717\\
0.3	0.510958323589132	0.872260419102717\\
0.315	0.510958323589132	0.872260419102717\\
0.33	0.510958323589132	0.872260419102717\\
0.345	0.510958323589132	0.872260419102717\\
0.36	0.510958323589132	0.872260419102717\\
0.375	0.510958323589132	0.872260419102717\\
0.39	0.510958323589132	0.872260419102717\\
0.405	0.510958323589132	0.872260419102717\\
0.42	0.510958323589132	0.872260419102717\\
0.435	0.510958323589132	0.872260419102717\\
0.45	0.510958323589132	0.872260419102717\\
0.465	0.510958323589132	0.872260419102717\\
0.48	0.510958323589132	0.872260419102717\\
0.495	0.510958323589132	0.872260419102717\\
0.51	0.510958323589132	0.872260419102717\\
0.525	0.510958323589132	0.872260419102717\\
0.54	0.510958323589132	0.872260419102717\\
0.555	0.510958323589132	0.872260419102717\\
0.57	0.510958323589132	0.872260419102717\\
0.585	0.510958323589132	0.872260419102717\\
0.6	0.510958323589132	0.872260419102717\\
0.615	0.510958323589132	0.872260419102717\\
0.63	0.510958323589132	0.872260419102717\\
0.645	0.510958323589132	0.872260419102717\\
0.66	0.510958323589132	0.872260419102717\\
0.675	0.510958323589132	0.872260419102717\\
0.69	0.510958323589132	0.872260419102717\\
0.705	0.510958323589132	0.872260419102717\\
0.72	0.510958323589132	0.872260419102717\\
0.735	0.510958323589132	0.872260419102717\\
0.75	0.510958323589132	0.872260419102717\\
0.765	0.510958323589132	0.872260419102717\\
0.78	0.510958323589132	0.872260419102717\\
0.795	0.510958323589132	0.872260419102717\\
0.81	0.510958323589132	0.872260419102717\\
0.825	0.510958323589132	0.872260419102717\\
0.84	0.510958323589132	0.872260419102717\\
0.855	0.510958323589132	0.872260419102717\\
0.87	0.510958323589132	0.872260419102717\\
0.885	0.510958323589132	0.872260419102717\\
0.9	0.510958323589132	0.872260419102717\\
0.915	0.510958323589132	0.872260419102717\\
0.93	0.510958323589132	0.872260419102717\\
0.945	0.510958323589132	0.872260419102717\\
0.96	0.510958323589132	0.872260419102717\\
0.975	0.510958323589132	0.872260419102717\\
0.99	0.510958323589132	0.872260419102717\\
1.005	0.510958323589132	0.872260419102717\\
1.02	0.510958323589132	0.872260419102717\\
1.035	0.510958323589132	0.872260419102717\\
1.05	0.510958323589132	0.872260419102717\\
1.065	0.510958323589132	0.872260419102717\\
1.08	0.510958323589132	0.872260419102717\\
1.095	0.510958323589132	0.872260419102717\\
1.11	0.510958323589132	0.872260419102717\\
1.125	0.510958323589132	0.872260419102717\\
1.14	0.510958323589132	0.872260419102717\\
1.155	0.510958323589132	0.872260419102717\\
1.17	0.510958323589132	0.872260419102717\\
1.185	0.510958323589132	0.872260419102717\\
1.2	0.510958323589132	0.872260419102717\\
1.215	0.510958323589132	0.872260419102717\\
1.23	0.510958323589132	0.872260419102717\\
1.245	0.510958323589132	0.872260419102717\\
1.26	0.510958323589132	0.872260419102717\\
1.275	0.510958323589132	0.872260419102717\\
1.29	0.510958323589132	0.872260419102717\\
1.305	0.510958323589132	0.872260419102717\\
1.32	0.510958323589132	0.872260419102717\\
1.335	0.510958323589132	0.872260419102717\\
1.35	0.510958323589132	0.872260419102717\\
1.365	0.510958323589132	0.872260419102717\\
1.38	0.510958323589132	0.872260419102717\\
1.395	0.510958323589132	0.872260419102717\\
1.41	0.510958323589132	0.872260419102717\\
1.425	0.510958323589132	0.872260419102717\\
1.44	0.510958323589132	0.872260419102717\\
1.455	0.510958323589132	0.872260419102717\\
1.47	0.510958323589132	0.872260419102717\\
1.485	0.510958323589132	0.872260419102717\\
1.5	0.510958323589132	0.872260419102717\\
};
 \addplot3 [color=mycolor1,solid,line width=1.5pt]
 table[row sep=crcr] {%
0	1	1\\
0.015	1	1\\
0.03	1	1\\
0.045	1	1\\
0.06	1	1\\
0.075	1	1\\
0.09	1	1\\
0.105	1	1\\
0.12	1	1\\
0.135	1	1\\
0.15	1	1\\
0.165	1	1\\
0.18	1	1\\
0.195	1	1\\
0.21	1	1\\
0.225	1	1\\
0.24	1	1\\
0.255	1	1\\
0.27	1	1\\
0.285	1	1\\
0.3	1	1\\
0.315	1	1\\
0.33	1	1\\
0.345	1	1\\
0.36	1	1\\
0.375	1	1\\
0.39	1	1\\
0.405	1	1\\
0.42	1	1\\
0.435	1	1\\
0.45	1	1\\
0.465	1	1\\
0.48	1	1\\
0.495	1	1\\
0.51	1	1\\
0.525	1	1\\
0.54	1	1\\
0.555	1	1\\
0.57	1	1\\
0.585	1	1\\
0.6	1	1\\
0.615	1	1\\
0.63	1	1\\
0.645	1	1\\
0.66	1	1\\
0.675	1	1\\
0.69	1	1\\
0.705	1	1\\
0.72	1	1\\
0.735	1	1\\
0.75	1	1\\
0.765	1	1\\
0.78	1	1\\
0.795	1	1\\
0.81	1	1\\
0.825	1	1\\
0.84	1	1\\
0.855	1	1\\
0.87	1	1\\
0.885	1	1\\
0.9	1	1\\
0.915	1	1\\
0.93	1	1\\
0.945	1	1\\
0.96	1	1\\
0.975	1	1\\
0.99	1	1\\
1.005	1	1\\
1.02	1	1\\
1.035	1	1\\
1.05	1	1\\
1.065	1	1\\
1.08	1	1\\
1.095	1	1\\
1.11	1	1\\
1.125	1	1\\
1.14	1	1\\
1.155	1	1\\
1.17	1	1\\
1.185	1	1\\
1.2	1	1\\
1.215	1	1\\
1.23	1	1\\
1.245	1	1\\
1.26	1	1\\
1.275	1	1\\
1.29	1	1\\
1.305	1	1\\
1.32	1	1\\
1.335	1	1\\
1.35	1	1\\
1.365	1	1\\
1.38	1	1\\
1.395	1	1\\
1.41	1	1\\
1.425	1	1\\
1.44	1	1\\
1.455	1	1\\
1.47	1	1\\
1.485	1	1\\
1.5	1	1\\
};
 \addplot3 [color=mycolor1,solid,line width=1.5pt]
 table[row sep=crcr] {%
0	0	0\\
0	0.000100583311362513	0.0141829653360114\\
0	0.000404645907256436	0.0284451766772965\\
0	0.000915607803432999	0.0427829086109896\\
0	0.00163681893109844	0.057192295687301\\
0	0.00257155309398959	0.0716693330697585\\
0	0.00372300185733413	0.0862098774609879\\
0	0.00509426838162598	0.100809648312589\\
0	0.00668836121491815	0.115464229327074\\
0	0.00850818805808554	0.13016907025918\\
0	0.0105565495182326	0.144919489023162\\
0	0.0128361328661109	0.159710674111862\\
0	0.0153495058140643	0.174537687332543\\
0	0.0180991103316243	0.189395466863524\\
0	0.0210872565164405	0.204278830634725\\
0	0.0243161165387281	0.219182480034174\\
0	0.0277877186778607	0.234101003941464\\
0	0.0315039414701067	0.24902888308801\\
0	0.0354665079868145	0.263960494742775\\
0	0.0396769802625738	0.278890117720924\\
0	0.0441367538930258	0.293811937711588\\
0	0.0488470528220538	0.308720052919643\\
0	0.0538089243380495	0.323608480015096\\
0	0.0590232342988274	0.338471160382329\\
0	0.064490662604533	0.3533019666601\\
0	0.0702116989375697	0.368094709561873\\
0	0.0761866387881432	0.382843144964686\\
0	0.0824155797834956	0.397540981253432\\
0	0.0888984183382709	0.412181886906127\\
0	0.0956348466427212	0.42675949830445\\
0	0.102624350004627	0.441267427752584\\
0	0.109866204559871	0.455699271686218\\
0	0.117359475365564	0.470048619052377\\
0	0.125103014888515	0.484309059839721\\
0	0.133095461900593	0.498474193737904\\
0	0.14133524079124	0.512537638903683\\
0	0.149820561306021	0.526493040810599\\
0	0.158549418718625	0.540334081158348\\
0	0.167519594442213	0.554054486817265\\
0	0.176728657084455	0.567648038782867\\
0	0.186173963948925	0.581108581114919\\
0	0.195852662983903	0.594430029835236\\
0	0.205761695177907	0.607606381758231\\
0	0.215897797399558	0.620631723228143\\
0	0.226257505677663	0.633500238737009\\
0	0.236837158915675	0.646206219397571\\
0	0.247632903032949	0.658744071245709\\
0	0.258640695523527	0.671108323347402\\
0	0.269856310421543	0.683293635685828\\
0	0.28127534366066	0.695294806804925\\
0	0.292893218813452	0.707106781186547\\
0	0.304705193195075	0.71872465633934\\
0	0.316706364314172	0.730143689578457\\
0	0.328891676652598	0.741359304476472\\
0	0.341255928754291	0.752367096967051\\
0	0.353793780602429	0.763162841084325\\
0	0.366499761262991	0.773742494322337\\
0	0.379368276771857	0.784102202600442\\
0	0.392393618241769	0.794238304822092\\
0	0.405569970164763	0.804147337016097\\
0	0.418891418885081	0.813826036051075\\
0	0.432351961217132	0.823271342915544\\
0	0.445945513182735	0.832480405557786\\
0	0.459665918841652	0.841450581281375\\
0	0.473506959189401	0.850179438693979\\
0	0.487462361096317	0.85866475920876\\
0	0.501525806262096	0.866904538099407\\
0	0.515690940160279	0.874896985111485\\
0	0.529951380947623	0.882640524634436\\
0	0.544300728313782	0.890133795440129\\
0	0.558732572247415	0.897375649995373\\
0	0.57324050169555	0.904365153357279\\
0	0.587818113093873	0.911101581661729\\
0	0.602459018746568	0.917584420216504\\
0	0.617156855035314	0.923813361211857\\
0	0.631905290438127	0.92978830106243\\
0	0.6466980333399	0.935509337395467\\
0	0.661528839617671	0.940976765701173\\
0	0.676391519984904	0.94619107566195\\
0	0.691279947080357	0.951152947177946\\
0	0.706188062288412	0.955863246106974\\
0	0.721109882279076	0.960323019737426\\
0	0.736039505257226	0.964533492013186\\
0	0.75097111691199	0.968496058529893\\
0	0.765898996058536	0.972212281322139\\
0	0.780817519965826	0.975683883461272\\
0	0.795721169365275	0.97891274348356\\
0	0.810604533136476	0.981900889668376\\
0	0.825462312667457	0.984650494185936\\
0	0.840289325888138	0.987163867133889\\
0	0.855080510976839	0.989443450481767\\
0	0.86983092974082	0.991491811941914\\
0	0.884535770672926	0.993311638785082\\
0	0.899190351687411	0.994905731618374\\
0	0.913790122539012	0.996276998142666\\
0	0.928330666930242	0.997428446906011\\
0	0.942807704312699	0.998363181068902\\
0	0.95721709138901	0.999084392196567\\
0	0.971554823322703	0.999595354092743\\
0	0.985817034663989	0.999899416688637\\
0	1	1\\
};
 \addplot3 [color=mycolor1,solid,line width=1.5pt]
 table[row sep=crcr] {%
0.5	0	0\\
0.5	0.000100583311362513	0.0141829653360114\\
0.5	0.000404645907256436	0.0284451766772965\\
0.5	0.000915607803432999	0.0427829086109896\\
0.5	0.00163681893109844	0.057192295687301\\
0.5	0.00257155309398959	0.0716693330697584\\
0.5	0.00372300185733413	0.0862098774609879\\
0.5	0.00509426838162598	0.100809648312589\\
0.5	0.00668836121491816	0.115464229327074\\
0.5	0.00850818805808555	0.13016907025918\\
0.5	0.0105565495182326	0.144919489023162\\
0.5	0.0128361328661109	0.159710674111862\\
0.5	0.0153495058140643	0.174537687332543\\
0.5	0.0180991103316243	0.189395466863524\\
0.5	0.0210872565164405	0.204278830634725\\
0.5	0.0243161165387281	0.219182480034174\\
0.5	0.0277877186778607	0.234101003941464\\
0.5	0.0315039414701067	0.24902888308801\\
0.5	0.0354665079868145	0.263960494742775\\
0.5	0.0396769802625738	0.278890117720924\\
0.5	0.0441367538930258	0.293811937711588\\
0.5	0.0488470528220538	0.308720052919643\\
0.5	0.0538089243380495	0.323608480015096\\
0.5	0.0590232342988274	0.338471160382329\\
0.5	0.064490662604533	0.3533019666601\\
0.5	0.0702116989375697	0.368094709561873\\
0.5	0.0761866387881432	0.382843144964686\\
0.5	0.0824155797834956	0.397540981253432\\
0.5	0.0888984183382709	0.412181886906127\\
0.5	0.0956348466427212	0.42675949830445\\
0.5	0.102624350004627	0.441267427752584\\
0.5	0.109866204559871	0.455699271686218\\
0.5	0.117359475365564	0.470048619052377\\
0.5	0.125103014888515	0.484309059839721\\
0.5	0.133095461900593	0.498474193737904\\
0.5	0.14133524079124	0.512537638903683\\
0.5	0.149820561306021	0.526493040810599\\
0.5	0.158549418718625	0.540334081158348\\
0.5	0.167519594442214	0.554054486817265\\
0.5	0.176728657084455	0.567648038782867\\
0.5	0.186173963948925	0.581108581114919\\
0.5	0.195852662983903	0.594430029835236\\
0.5	0.205761695177907	0.607606381758231\\
0.5	0.215897797399558	0.620631723228143\\
0.5	0.226257505677663	0.633500238737009\\
0.5	0.236837158915675	0.646206219397571\\
0.5	0.247632903032949	0.658744071245709\\
0.5	0.258640695523527	0.671108323347402\\
0.5	0.269856310421543	0.683293635685828\\
0.5	0.28127534366066	0.695294806804924\\
0.5	0.292893218813452	0.707106781186547\\
0.5	0.304705193195075	0.71872465633934\\
0.5	0.316706364314172	0.730143689578457\\
0.5	0.328891676652598	0.741359304476472\\
0.5	0.341255928754291	0.752367096967051\\
0.5	0.353793780602429	0.763162841084325\\
0.5	0.366499761262991	0.773742494322337\\
0.5	0.379368276771857	0.784102202600442\\
0.5	0.392393618241769	0.794238304822092\\
0.5	0.405569970164763	0.804147337016097\\
0.5	0.418891418885081	0.813826036051075\\
0.5	0.432351961217132	0.823271342915544\\
0.5	0.445945513182735	0.832480405557786\\
0.5	0.459665918841652	0.841450581281375\\
0.5	0.473506959189401	0.850179438693979\\
0.5	0.487462361096317	0.85866475920876\\
0.5	0.501525806262096	0.866904538099407\\
0.5	0.515690940160279	0.874896985111485\\
0.5	0.529951380947623	0.882640524634436\\
0.5	0.544300728313782	0.890133795440129\\
0.5	0.558732572247415	0.897375649995373\\
0.5	0.57324050169555	0.904365153357279\\
0.5	0.587818113093873	0.911101581661729\\
0.5	0.602459018746568	0.917584420216504\\
0.5	0.617156855035314	0.923813361211857\\
0.5	0.631905290438127	0.92978830106243\\
0.5	0.6466980333399	0.935509337395467\\
0.5	0.661528839617671	0.940976765701173\\
0.5	0.676391519984904	0.94619107566195\\
0.5	0.691279947080357	0.951152947177946\\
0.5	0.706188062288412	0.955863246106974\\
0.5	0.721109882279076	0.960323019737426\\
0.5	0.736039505257226	0.964533492013186\\
0.5	0.75097111691199	0.968496058529893\\
0.5	0.765898996058536	0.972212281322139\\
0.5	0.780817519965826	0.975683883461272\\
0.5	0.795721169365275	0.97891274348356\\
0.5	0.810604533136476	0.981900889668376\\
0.5	0.825462312667457	0.984650494185936\\
0.5	0.840289325888138	0.987163867133889\\
0.5	0.855080510976839	0.989443450481767\\
0.5	0.86983092974082	0.991491811941914\\
0.5	0.884535770672926	0.993311638785082\\
0.5	0.899190351687411	0.994905731618374\\
0.5	0.913790122539012	0.996276998142666\\
0.5	0.928330666930242	0.997428446906011\\
0.5	0.942807704312699	0.998363181068902\\
0.5	0.95721709138901	0.999084392196567\\
0.5	0.971554823322703	0.999595354092743\\
0.5	0.985817034663989	0.999899416688637\\
0.5	1	1\\
};
 \addplot3 [color=mycolor1,solid,line width=1.5pt]
 table[row sep=crcr] {%
1	0	0\\
1	0.000100583311362513	0.0141829653360114\\
1	0.000404645907256436	0.0284451766772965\\
1	0.000915607803432999	0.0427829086109896\\
1	0.00163681893109844	0.057192295687301\\
1	0.00257155309398959	0.0716693330697584\\
1	0.00372300185733413	0.0862098774609879\\
1	0.00509426838162598	0.100809648312589\\
1	0.00668836121491816	0.115464229327074\\
1	0.00850818805808554	0.13016907025918\\
1	0.0105565495182326	0.144919489023162\\
1	0.0128361328661109	0.159710674111862\\
1	0.0153495058140643	0.174537687332543\\
1	0.0180991103316243	0.189395466863524\\
1	0.0210872565164405	0.204278830634725\\
1	0.0243161165387281	0.219182480034174\\
1	0.0277877186778607	0.234101003941464\\
1	0.0315039414701067	0.24902888308801\\
1	0.0354665079868145	0.263960494742775\\
1	0.0396769802625738	0.278890117720924\\
1	0.0441367538930258	0.293811937711588\\
1	0.0488470528220538	0.308720052919643\\
1	0.0538089243380495	0.323608480015097\\
1	0.0590232342988274	0.338471160382329\\
1	0.064490662604533	0.3533019666601\\
1	0.0702116989375697	0.368094709561873\\
1	0.0761866387881432	0.382843144964686\\
1	0.0824155797834956	0.397540981253432\\
1	0.0888984183382709	0.412181886906127\\
1	0.0956348466427212	0.42675949830445\\
1	0.102624350004627	0.441267427752584\\
1	0.109866204559871	0.455699271686218\\
1	0.117359475365564	0.470048619052377\\
1	0.125103014888515	0.484309059839721\\
1	0.133095461900593	0.498474193737904\\
1	0.14133524079124	0.512537638903683\\
1	0.149820561306021	0.526493040810599\\
1	0.158549418718625	0.540334081158348\\
1	0.167519594442213	0.554054486817265\\
1	0.176728657084455	0.567648038782867\\
1	0.186173963948925	0.581108581114919\\
1	0.195852662983903	0.594430029835237\\
1	0.205761695177907	0.607606381758231\\
1	0.215897797399558	0.620631723228144\\
1	0.226257505677663	0.633500238737009\\
1	0.236837158915675	0.646206219397571\\
1	0.247632903032949	0.658744071245709\\
1	0.258640695523527	0.671108323347402\\
1	0.269856310421543	0.683293635685828\\
1	0.28127534366066	0.695294806804924\\
1	0.292893218813452	0.707106781186547\\
1	0.304705193195075	0.71872465633934\\
1	0.316706364314172	0.730143689578457\\
1	0.328891676652598	0.741359304476472\\
1	0.341255928754291	0.752367096967051\\
1	0.353793780602429	0.763162841084325\\
1	0.366499761262991	0.773742494322337\\
1	0.379368276771857	0.784102202600443\\
1	0.392393618241769	0.794238304822092\\
1	0.405569970164763	0.804147337016097\\
1	0.418891418885081	0.813826036051075\\
1	0.432351961217133	0.823271342915545\\
1	0.445945513182735	0.832480405557786\\
1	0.459665918841652	0.841450581281375\\
1	0.473506959189401	0.850179438693979\\
1	0.487462361096317	0.85866475920876\\
1	0.501525806262096	0.866904538099407\\
1	0.515690940160279	0.874896985111485\\
1	0.529951380947623	0.882640524634436\\
1	0.544300728313782	0.890133795440129\\
1	0.558732572247415	0.897375649995373\\
1	0.57324050169555	0.904365153357279\\
1	0.587818113093873	0.911101581661729\\
1	0.602459018746568	0.917584420216504\\
1	0.617156855035314	0.923813361211857\\
1	0.631905290438127	0.92978830106243\\
1	0.6466980333399	0.935509337395467\\
1	0.661528839617671	0.940976765701173\\
1	0.676391519984903	0.94619107566195\\
1	0.691279947080357	0.951152947177946\\
1	0.706188062288412	0.955863246106974\\
1	0.721109882279076	0.960323019737426\\
1	0.736039505257226	0.964533492013186\\
1	0.75097111691199	0.968496058529893\\
1	0.765898996058536	0.972212281322139\\
1	0.780817519965826	0.975683883461272\\
1	0.795721169365275	0.978912743483559\\
1	0.810604533136476	0.981900889668376\\
1	0.825462312667457	0.984650494185936\\
1	0.840289325888138	0.987163867133889\\
1	0.855080510976838	0.989443450481767\\
1	0.86983092974082	0.991491811941914\\
1	0.884535770672926	0.993311638785082\\
1	0.899190351687411	0.994905731618374\\
1	0.913790122539012	0.996276998142666\\
1	0.928330666930242	0.997428446906011\\
1	0.942807704312699	0.998363181068902\\
1	0.95721709138901	0.999084392196567\\
1	0.971554823322703	0.999595354092743\\
1	0.985817034663989	0.999899416688637\\
1	1	1\\
};
 \addplot3 [color=mycolor1,solid,line width=1.5pt]
 table[row sep=crcr] {%
1.5	0	0\\
1.5	0.000100583311362513	0.0141829653360114\\
1.5	0.000404645907256436	0.0284451766772965\\
1.5	0.000915607803432999	0.0427829086109896\\
1.5	0.00163681893109844	0.057192295687301\\
1.5	0.00257155309398959	0.0716693330697585\\
1.5	0.00372300185733413	0.0862098774609879\\
1.5	0.00509426838162598	0.100809648312589\\
1.5	0.00668836121491815	0.115464229327074\\
1.5	0.00850818805808554	0.13016907025918\\
1.5	0.0105565495182326	0.144919489023162\\
1.5	0.0128361328661109	0.159710674111862\\
1.5	0.0153495058140643	0.174537687332543\\
1.5	0.0180991103316243	0.189395466863524\\
1.5	0.0210872565164405	0.204278830634725\\
1.5	0.0243161165387281	0.219182480034174\\
1.5	0.0277877186778607	0.234101003941464\\
1.5	0.0315039414701067	0.24902888308801\\
1.5	0.0354665079868145	0.263960494742775\\
1.5	0.0396769802625738	0.278890117720924\\
1.5	0.0441367538930258	0.293811937711588\\
1.5	0.0488470528220538	0.308720052919643\\
1.5	0.0538089243380495	0.323608480015096\\
1.5	0.0590232342988274	0.338471160382329\\
1.5	0.064490662604533	0.3533019666601\\
1.5	0.0702116989375697	0.368094709561873\\
1.5	0.0761866387881432	0.382843144964686\\
1.5	0.0824155797834956	0.397540981253432\\
1.5	0.0888984183382709	0.412181886906127\\
1.5	0.0956348466427212	0.42675949830445\\
1.5	0.102624350004627	0.441267427752584\\
1.5	0.109866204559871	0.455699271686218\\
1.5	0.117359475365564	0.470048619052377\\
1.5	0.125103014888515	0.484309059839721\\
1.5	0.133095461900593	0.498474193737904\\
1.5	0.14133524079124	0.512537638903683\\
1.5	0.149820561306021	0.526493040810599\\
1.5	0.158549418718625	0.540334081158348\\
1.5	0.167519594442213	0.554054486817265\\
1.5	0.176728657084455	0.567648038782867\\
1.5	0.186173963948925	0.581108581114919\\
1.5	0.195852662983903	0.594430029835236\\
1.5	0.205761695177907	0.607606381758231\\
1.5	0.215897797399558	0.620631723228143\\
1.5	0.226257505677663	0.633500238737009\\
1.5	0.236837158915675	0.646206219397571\\
1.5	0.247632903032949	0.658744071245709\\
1.5	0.258640695523527	0.671108323347402\\
1.5	0.269856310421543	0.683293635685828\\
1.5	0.28127534366066	0.695294806804925\\
1.5	0.292893218813452	0.707106781186547\\
1.5	0.304705193195075	0.71872465633934\\
1.5	0.316706364314172	0.730143689578457\\
1.5	0.328891676652598	0.741359304476472\\
1.5	0.341255928754291	0.752367096967051\\
1.5	0.353793780602429	0.763162841084325\\
1.5	0.366499761262991	0.773742494322337\\
1.5	0.379368276771857	0.784102202600442\\
1.5	0.392393618241769	0.794238304822092\\
1.5	0.405569970164763	0.804147337016097\\
1.5	0.418891418885081	0.813826036051075\\
1.5	0.432351961217132	0.823271342915544\\
1.5	0.445945513182735	0.832480405557786\\
1.5	0.459665918841652	0.841450581281375\\
1.5	0.473506959189401	0.850179438693979\\
1.5	0.487462361096317	0.85866475920876\\
1.5	0.501525806262096	0.866904538099407\\
1.5	0.515690940160279	0.874896985111485\\
1.5	0.529951380947623	0.882640524634436\\
1.5	0.544300728313782	0.890133795440129\\
1.5	0.558732572247415	0.897375649995373\\
1.5	0.57324050169555	0.904365153357279\\
1.5	0.587818113093873	0.911101581661729\\
1.5	0.602459018746568	0.917584420216504\\
1.5	0.617156855035314	0.923813361211857\\
1.5	0.631905290438127	0.92978830106243\\
1.5	0.6466980333399	0.935509337395467\\
1.5	0.661528839617671	0.940976765701173\\
1.5	0.676391519984904	0.94619107566195\\
1.5	0.691279947080357	0.951152947177946\\
1.5	0.706188062288412	0.955863246106974\\
1.5	0.721109882279076	0.960323019737426\\
1.5	0.736039505257226	0.964533492013186\\
1.5	0.75097111691199	0.968496058529893\\
1.5	0.765898996058536	0.972212281322139\\
1.5	0.780817519965826	0.975683883461272\\
1.5	0.795721169365275	0.97891274348356\\
1.5	0.810604533136476	0.981900889668376\\
1.5	0.825462312667457	0.984650494185936\\
1.5	0.840289325888138	0.987163867133889\\
1.5	0.855080510976839	0.989443450481767\\
1.5	0.86983092974082	0.991491811941914\\
1.5	0.884535770672926	0.993311638785082\\
1.5	0.899190351687411	0.994905731618374\\
1.5	0.913790122539012	0.996276998142666\\
1.5	0.928330666930242	0.997428446906011\\
1.5	0.942807704312699	0.998363181068902\\
1.5	0.95721709138901	0.999084392196567\\
1.5	0.971554823322703	0.999595354092743\\
1.5	0.985817034663989	0.999899416688637\\
1.5	1	1\\
};
 \addplot3 [draw=none, mark size=3.3pt, scatter,mark=ball,scatter/use mapped color={ball color=red},scatter src=rand,only marks,z buffer=sort]
 table[row sep=crcr] {%
0	0	0\\
0	0	0.261203874963741\\
0	0.265409196609864	0.734590803390136\\
0	0.738796125036259	1\\
0	1	1\\
0.25	0	0\\
0.25	0	0.261203874963741\\
0.25	0.265409196609864	0.734590803390136\\
0.25	0.738796125036259	1\\
0.25	1	1\\
0.75	0	0\\
0.75	0	0.261203874963741\\
0.75	0.265409196609864	0.734590803390136\\
0.75	0.738796125036259	1\\
0.75	1	1\\
1.25	0	0\\
1.25	0	0.261203874963741\\
1.25	0.265409196609864	0.734590803390136\\
1.25	0.738796125036259	1\\
1.25	1	1\\
1.5	0	0\\
1.5	0	0.261203874963741\\
1.5	0.265409196609864	0.734590803390136\\
1.5	0.738796125036259	1\\
1.5	1	1\\
};
 \addplot3 [color=black,dashed]
 table[row sep=crcr] {%
0	0	0\\
0.25	0	0\\
0.75	0	0\\
1.25	0	0\\
1.5	0	0\\
};
 \addplot3 [color=black,dashed]
 table[row sep=crcr] {%
0	0	0.261203874963741\\
0.25	0	0.261203874963741\\
0.75	0	0.261203874963741\\
1.25	0	0.261203874963741\\
1.5	0	0.261203874963741\\
};
 \addplot3 [color=black,dashed]
 table[row sep=crcr] {%
0	0.265409196609864	0.734590803390136\\
0.25	0.265409196609864	0.734590803390136\\
0.75	0.265409196609864	0.734590803390136\\
1.25	0.265409196609864	0.734590803390136\\
1.5	0.265409196609864	0.734590803390136\\
};
 \addplot3 [color=black,dashed]
 table[row sep=crcr] {%
0	0.738796125036259	1\\
0.25	0.738796125036259	1\\
0.75	0.738796125036259	1\\
1.25	0.738796125036259	1\\
1.5	0.738796125036259	1\\
};
 \addplot3 [color=black,dashed]
 table[row sep=crcr] {%
0	1	1\\
0.25	1	1\\
0.75	1	1\\
1.25	1	1\\
1.5	1	1\\
};
 \addplot3 [color=black,dashed]
 table[row sep=crcr] {%
0	0	0\\
0	0	0.261203874963741\\
0	0.265409196609864	0.734590803390136\\
0	0.738796125036259	1\\
0	1	1\\
};
 \addplot3 [color=black,dashed]
 table[row sep=crcr] {%
0.25	0	0\\
0.25	0	0.261203874963741\\
0.25	0.265409196609864	0.734590803390136\\
0.25	0.738796125036259	1\\
0.25	1	1\\
};
 \addplot3 [color=black,dashed]
 table[row sep=crcr] {%
0.75	0	0\\
0.75	0	0.261203874963741\\
0.75	0.265409196609864	0.734590803390136\\
0.75	0.738796125036259	1\\
0.75	1	1\\
};
 \addplot3 [color=black,dashed]
 table[row sep=crcr] {%
1.25	0	0\\
1.25	0	0.261203874963741\\
1.25	0.265409196609864	0.734590803390136\\
1.25	0.738796125036259	1\\
1.25	1	1\\
};
 \addplot3 [color=black,dashed]
 table[row sep=crcr] {%
1.5	0	0\\
1.5	0	0.261203874963741\\
1.5	0.265409196609864	0.734590803390136\\
1.5	0.738796125036259	1\\
1.5	1	1\\
};
\coordinate (A01) at (axis cs: 0, 0, 0);
\coordinate (A02) at (axis cs: 0, 0, 0.261203874963741);
\coordinate (A03) at (axis cs: 0, 0.265409196609864, 0.734590803390136);
\coordinate (A04) at (axis cs: 0, 0.738796125036259, 1);
\coordinate (A05) at (axis cs: 0, 1, 1);
\coordinate (A06) at (axis cs: 0.25, 0, 0);
\coordinate (A07) at (axis cs: 0.25, 0, 0.261203874963741);
\coordinate (A08) at (axis cs: 0.25, 0.265409196609864, 0.734590803390136);
\coordinate (A09) at (axis cs: 0.25, 0.738796125036259, 1);
\coordinate (A10) at (axis cs: 0.25, 1, 1);
\coordinate (A11) at (axis cs: 0.75, 0, 0);
\coordinate (A12) at (axis cs: 0.75, 0, 0.261203874963741);
\coordinate (A13) at (axis cs: 0.75, 0.265409196609864, 0.734590803390136);
\coordinate (A14) at (axis cs: 0.75, 0.738796125036259, 1);
\coordinate (A15) at (axis cs: 0.75, 1, 1);
\coordinate (A16) at (axis cs: 1.25, 0, 0);
\coordinate (A17) at (axis cs: 1.25, 0, 0.261203874963741);
\coordinate (A18) at (axis cs: 1.25, 0.265409196609864, 0.734590803390136);
\coordinate (A19) at (axis cs: 1.25, 0.738796125036259, 1);
\coordinate (A20) at (axis cs: 1.25, 1, 1);
\coordinate (A21) at (axis cs: 1.5, 0, 0);
\coordinate (A22) at (axis cs: 1.5, 0, 0.261203874963741);
\coordinate (A23) at (axis cs: 1.5, 0.265409196609864, 0.734590803390136);
\coordinate (A24) at (axis cs: 1.5, 0.738796125036259, 1);
\coordinate (A25) at (axis cs: 1.5, 1, 1);

\addplot3 [color=cyan, dashed, -stealth,line width=1.5pt, postaction={decorate,decoration={text along path,
              text={$\xi$} {--} direction, raise=1ex, text align={center}, text color={cyan},
          }}]
 table[row sep=crcr] {%
0	0	0\\
0	0.000100583311362513	0.0141829653360114\\
0	0.000404645907256436	0.0284451766772965\\
0	0.000915607803433	0.0427829086109896\\
0	0.00163681893109844	0.057192295687301\\
0	0.00257155309398959	0.0716693330697585\\
0	0.00372300185733414	0.086209877460988\\
0	0.00509426838162598	0.100809648312589\\
0	0.00668836121491816	0.115464229327074\\
0	0.00850818805808555	0.13016907025918\\
0	0.0105565495182326	0.144919489023162\\
0	0.0128361328661109	0.159710674111862\\
0	0.0153495058140643	0.174537687332543\\
0	0.0180991103316243	0.189395466863524\\
0	0.0210872565164405	0.204278830634725\\
0	0.0243161165387281	0.219182480034174\\
0	0.0277877186778607	0.234101003941464\\
0	0.0315039414701067	0.24902888308801\\
0	0.0354665079868145	0.263960494742775\\
0	0.0396769802625738	0.278890117720924\\
0	0.0441367538930258	0.293811937711588\\
0	0.0488470528220538	0.308720052919643\\
0	0.0538089243380495	0.323608480015096\\
0	0.0590232342988274	0.338471160382329\\
0	0.064490662604533	0.3533019666601\\
0	0.0702116989375697	0.368094709561873\\
0	0.0761866387881432	0.382843144964686\\
0	0.0824155797834956	0.397540981253432\\
0	0.0888984183382709	0.412181886906127\\
0	0.0956348466427212	0.42675949830445\\
0	0.102624350004627	0.441267427752584\\
0	0.109866204559871	0.455699271686219\\
0	0.117359475365564	0.470048619052377\\
0	0.125103014888515	0.484309059839721\\
0	0.133095461900593	0.498474193737904\\
0	0.14133524079124	0.512537638903683\\
0	0.149820561306021	0.526493040810599\\
0	0.158549418718625	0.540334081158348\\
0	0.167519594442214	0.554054486817265\\
0	0.176728657084455	0.567648038782867\\
0	0.186173963948925	0.581108581114919\\
0	0.195852662983903	0.594430029835237\\
0	0.205761695177907	0.607606381758231\\
0	0.215897797399558	0.620631723228143\\
0	0.226257505677663	0.633500238737009\\
0	0.236837158915675	0.646206219397571\\
0	0.247632903032949	0.658744071245709\\
0	0.258640695523528	0.671108323347402\\
0	0.269856310421543	0.683293635685828\\
0	0.28127534366066	0.695294806804925\\
0	0.292893218813452	0.707106781186547\\
0	0.304705193195075	0.71872465633934\\
};
\coordinate (A) at (axis cs: 0, 0, 0);
\coordinate (X) at (axis cs: 0.75, 0, 0);
\end{axis}
\node[below] at (A01) {$\mathbf{P}_{1}$};
\node[below] at (A02) {$\mathbf{P}_{2}$};
\node[below] at (A03) {$\mathbf{P}_{3}$};
\node[below] at (A04) {$\mathbf{P}_{4}$};
\node[above] at (A05) {$\mathbf{P}_{5}$};
\node[below] at (A06) {$\mathbf{P}_{6}$};
\node[below] at (A07) {$\mathbf{P}_{7}$};
\node[below] at (A08) {$\mathbf{P}_{8}$};
\node[below] at (A09) {$\mathbf{P}_{9}$};
\node[above] at (A10) {$\mathbf{P}_{10}$};
\node[below] at (A11) {$\mathbf{P}_{11}$};
\node[below] at (A12) {$\mathbf{P}_{12}$};
\node[below] at (A13) {$\mathbf{P}_{13}$};
\node[below] at (A14) {$\mathbf{P}_{14}$};
\node[above] at (A15) {$\mathbf{P}_{15}$};
\node[below] at (A16) {$\mathbf{P}_{16}$};
\node[below] at (A17) {$\mathbf{P}_{17}$};
\node[below] at (A18) {$\mathbf{P}_{18}$};
\node[below] at (A19) {$\mathbf{P}_{19}$};
\node[above] at (A20) {$\mathbf{P}_{20}$};
\node[below] at (A21) {$\mathbf{P}_{21}$};
\node[below] at (A22) {$\mathbf{P}_{22}$};
\node[below] at (A23) {$\mathbf{P}_{23}$};
\node[below] at (A24) {$\mathbf{P}_{24}$};
\node[above] at (A25) {$\mathbf{P}_{25}$};

\node[above, fill=yellow!80!black, opacity=0.7, transform shape, rotate = 21] at (A07) {$e = 1$};
\node[above, fill=yellow!80!black, opacity=0.7, transform shape, rotate = 21] at (A08) {$e = 2$};
\node[above, fill=yellow!80!black, opacity=0.7, transform shape, rotate = 21] at (A09) {$e = 3$};

\node[above, fill=yellow!80!black, opacity=0.7, transform shape, rotate = 21] at (A12) {$e = 4$};
\node[above, fill=yellow!80!black, opacity=0.7, transform shape, rotate = 21] at (A13) {$e = 5$};
\node[above, fill=yellow!80!black, opacity=0.7, transform shape, rotate = 21] at (A14) {$e = 6$};

\node[above, fill=yellow!80!black, opacity=0.7, transform shape, rotate = 21] at (A17) {$e = 7$};
\node[above, fill=yellow!80!black, opacity=0.7, transform shape, rotate = 21] at (A18) {$e = 8$};
\node[above, fill=yellow!80!black, opacity=0.7, transform shape, rotate = 21] at (A19) {$e = 9$};

%\draw[help lines,xstep=1,ystep=1] (0,0) grid (10, 10);
%\foreach \x in {0,1,...,10} { \node [anchor=north] at (\x, 0) {\x}; }
%\foreach \y in {0,1,...,10} { \node [anchor=east] at (0, \y) {\y}; }

\begin{scope}[shift = {(10.2cm, 1.0cm)}, scale=1, thin]
\draw[fill=gray!10, fill opacity=0.70] (0, 0) rectangle (5.0, 5.0);
\fill[fill=mycolor2, fill opacity=0.50] (0.0, 2.75) rectangle (5.0, 3.25);
{\fontencoding{T1}\selectfont \ttfamily
    \node [rectangle, anchor=west, fill=red!50, fill opacity=0.70, align=center, text width=2cm, left] at (0, 4.5) {\footnotesize 1$^\text{th}$ element};
    \node [rectangle, anchor=west, fill=red!50, fill opacity=0.70, align=center, text width=2cm, left] at (0, 4.0) {\footnotesize 2$^\text{th}$ element};
    \node [rectangle, anchor=west, fill=red!50, fill opacity=0.70, align=center, text width=2cm, left] at (0, 3.5) {\footnotesize 3$^\text{th}$ element};
    \node [rectangle, anchor=west, fill=red!50, fill opacity=0.70, align=center, text width=2cm, left] at (0, 3.0) {\footnotesize 4$^\text{th}$ element};
    \node [rectangle, anchor=west, fill=red!50, fill opacity=0.70, align=center, text width=2cm, left] at (0, 2.5) {\footnotesize 5$^\text{th}$ element};
    \node [rectangle, anchor=west, fill=red!50, fill opacity=0.70, align=center, text width=2cm, left] at (0, 2.0) {\footnotesize 6$^\text{th}$ element};
    \node [rectangle, anchor=west, fill=red!50, fill opacity=0.70, align=center, text width=2cm, left] at (0, 1.5) {\footnotesize 7$^\text{th}$ element};
    \node [rectangle, anchor=west, fill=red!50, fill opacity=0.70, align=center, text width=2cm, left] at (0, 1.0) {\footnotesize 8$^\text{th}$ element};
    \node [rectangle, anchor=west, fill=red!50, fill opacity=0.70, align=center, text width=2cm, left] at (0, 0.5) {\footnotesize 9$^\text{th}$ element};

    \node [rectangle, anchor=west, fill=red!50, fill opacity=0.70, align=center, above] at (2.5, 5.0) {\footnotesize Control Point Connectivities};

    \node at (0.5, 4.5) {1};
    \node at (1.0, 4.5) {2};
    \node at (1.5, 4.5) {3};
    \node at (2.0, 4.5) {6};
    \node at (2.5, 4.5) {7};
    \node at (3.0, 4.5) {8};
    \node at (3.5, 4.5) {11};
    \node at (4.0, 4.5) {12};
    \node at (4.5, 4.5) {13};

    \node at (0.5, 4.0) {2};
    \node at (1.0, 4.0) {3};
    \node at (1.5, 4.0) {4};
    \node at (2.0, 4.0) {7};
    \node at (2.5, 4.0) {8};
    \node at (3.0, 4.0) {9};
    \node at (3.5, 4.0) {12};
    \node at (4.0, 4.0) {13};
    \node at (4.5, 4.0) {14};

    \node at (0.5, 3.5) {3};
    \node at (1.0, 3.5) {4};
    \node at (1.5, 3.5) {5};
    \node at (2.0, 3.5) {8};
    \node at (2.5, 3.5) {9};
    \node at (3.0, 3.5) {10};
    \node at (3.5, 3.5) {13};
    \node at (4.0, 3.5) {14};
    \node at (4.5, 3.5) {15};

    \node at (0.5, 3.0) {6};
    \node at (1.0, 3.0) {7};
    \node at (1.5, 3.0) {8};
    \node at (2.0, 3.0) {11};
    \node at (2.5, 3.0) {12};
    \node at (3.0, 3.0) {13};
    \node at (3.5, 3.0) {16};
    \node at (4.0, 3.0) {17};
    \node at (4.5, 3.0) {18};

    \node at (0.5, 2.5) {7};
    \node at (1.0, 2.5) {8};
    \node at (1.5, 2.5) {9};
    \node at (2.0, 2.5) {12};
    \node at (2.5, 2.5) {13};
    \node at (3.0, 2.5) {14};
    \node at (3.5, 2.5) {17};
    \node at (4.0, 2.5) {18};
    \node at (4.5, 2.5) {19};

    \node at (0.5, 2.0) {8};
    \node at (1.0, 2.0) {9};
    \node at (1.5, 2.0) {10};
    \node at (2.0, 2.0) {13};
    \node at (2.5, 2.0) {14};
    \node at (3.0, 2.0) {15};
    \node at (3.5, 2.0) {18};
    \node at (4.0, 2.0) {19};
    \node at (4.5, 2.0) {20};

    \node at (0.5, 1.5) {11};
    \node at (1.0, 1.5) {12};
    \node at (1.5, 1.5) {13};
    \node at (2.0, 1.5) {16};
    \node at (2.5, 1.5) {17};
    \node at (3.0, 1.5) {18};
    \node at (3.5, 1.5) {21};
    \node at (4.0, 1.5) {22};
    \node at (4.5, 1.5) {23};

    \node at (0.5, 1.0) {12};
    \node at (1.0, 1.0) {13};
    \node at (1.5, 1.0) {14};
    \node at (2.0, 1.0) {17};
    \node at (2.5, 1.0) {18};
    \node at (3.0, 1.0) {19};
    \node at (3.5, 1.0) {22};
    \node at (4.0, 1.0) {23};
    \node at (4.5, 1.0) {24};

    \node at (0.5, 0.5) {13};
    \node at (1.0, 0.5) {14};
    \node at (1.5, 0.5) {15};
    \node at (2.0, 0.5) {18};
    \node at (2.5, 0.5) {19};
    \node at (3.0, 0.5) {20};
    \node at (3.5, 0.5) {23};
    \node at (4.0, 0.5) {24};
    \node at (4.5, 0.5) {25};
    }
\end{scope}
\draw[-stealth, dashed, line width=1.5pt, color=cyan] (A) -- node[below, midway, sloped] {$\eta$ -- direction} (X);
\end{tikzpicture}% 
    \caption{Global control point and element numbers.}
    \label{fig:Ch3SurfAQuarterOfACylinderEnum}
\end{figure}
In SIMO Package, to take advantage of MATLAB programming and simplify the code, the first two data and the last one are implemented implicitly, that means we do not need to create any MATLAB array to store these information. Let's take an example as depicted in Fig.~\ref{fig:Ch3SurfAQuarterOfACylinderEnum}. In this example, the NURBS geometry is parameterized by two knot vectors $\boldsymbol{\Xi} = \{0, 0, 0, 1, 1, 1\}$ and $\mathbf{H} = \{0, 0, 1, 1 \}$, respectively with $p = 2$ and $q = 1$. For the global control point numbers, the ordering is determined by the indices of the array that stores the control point's coordinates, moving along the $\xi$ direction first (horizontally), followed by $\eta$ (vertically) (row by row starting at the lower left corner of the parametric space) as in Fig.~\ref{fig:Ch3SurfAQuarterOfACylinderEnum}. Therefore, we have to follow strictly the procedure as mentioned in the previous section to set up control point's coordinates correctly.

The element numbers are assigned in the same manner as we did for global control point numbers. Due to the fact that only the non-zero knot spans are taken into account in analysis which we call ``elements'' in IsoGeometric Analysis and we are using open knot vectors that possibly have repeated knot values, some of these knot spans may result in zero interiors. In case of 2D isogeometric structured mesh, the computational domain is parameterized by two knot vectors $\boldsymbol{\Xi}$ and $\mathbf{H}$. The discretization of the physical domain is, therefore, obtained by excluding the multiple knot values in each knot vector. This implementation is incorporated as a member of the NURBS struct which is mentioned before in the Data Representation section, namely ``uqKntVect''. The number of elements in each direction is then calculated by taking the number of unique knot values minus 1 (\lstinline{numel(NURBS.uqKntVect) - 1;}). Finally, the numbering of elements can be illustrated as in Fig.~\ref{fig:Ch3SurfAQuarterOfACylinderEnum}.

\begin{figure}[H]
    \centering
    \tikzsetnextfilename{Ch3SurfAQuarterOfACylinderLocalEnum}
    \normalsize
    \definecolor{mycolor1}{rgb}{0.38000,0.54800,0.24000}%
\definecolor{mycolor2}{rgb}{1,0.4,0}
\begin{tikzpicture}

\begin{axis}[%
width=10cm,
scale only axis,
plot box ratio=1 1 1,
point meta min=0,
point meta max=1,
xmin=0,
xmax=1.5,
tick align=outside,
%xlabel={x},
ymin=0,
ymax=1,
%ylabel={y},
zmin=0,
zmax=1,
%zlabel={z},
ticks=none,
%yticklabels={,,},
view={-37.5}{30},
axis background/.style={fill=white},
axis x line*=bottom,
axis y line*=left,
axis z line*=left,
unit vector ratio=1 1 1,%
]
\addplot3[%
surf,
shader=interp,colormap={mymap}{[1pt] rgb(0pt)=(1,0.4,0); rgb(2pt)=(1,0.4,0)},mesh/rows=101]
table[row sep=crcr, point meta=\thisrow{c}] {%
%
x	y	z	c\\
0.5	0	0	0.710202759607373\\
0.5	1.11327767495586e-05	0.00471862581271275	0.713003157837853\\
0.5	4.46175251031912e-05	0.009446325184051	0.715846486660142\\
0.5	0.000100583311362513	0.0141829653360114	0.718788838379794\\
0.5	0.000179158434668431	0.018928411755669	0.721886092228356\\
0.5	0.000280470402701511	0.02368252819604	0.725200636220892\\
0.5	0.000404645907256436	0.0284451766772965	0.728799085274723\\
0.5	0.000551810799695644	0.0332162174883388	0.732749123983336\\
0.5	0.000722090066287311	0.037995509188729	0.737115575253328\\
0.5	0.000915607803432999	0.0427829086109896	0.741955893907011\\
0.5	0.0011324871927904	0.0475782708632729	0.747315373652289\\
0.5	0.00137285047629673	0.0523814493324038	0.753222424045811\\
0.5	0.00163681893109844	0.057192295687301	0.759684310525652\\
0.5	0.00192451284439304	0.0620106598827802	0.766683748102665\\
0.5	0.00223605148818898	0.0668363901637434	0.774176695511256\\
0.5	0.00257155309398959	0.0716693330697584	0.782091614494494\\
0.5	0.00293113482740716	0.0765093334400312	0.790330346421089\\
0.5	0.00331491276271365	0.0813562344187764	0.79877062752665\\
0.5	0.00372300185733413	0.0862098774609879	0.807270128930359\\
0.5	0.00415551592628969	0.0910701023386145	0.815671782700195\\
0.5	0.00461256761659621	0.0959367471471425	0.82381005350733\\
0.5	0.00509426838162598	0.100809648312589	0.831517746503351\\
0.5	0.00560072845543873	0.105688640598912	0.838632911419452\\
0.5	0.00613205682708918	0.11057355711583	0.845005411376292\\
0.5	0.00668836121491816	0.115464229327074	0.850502769049297\\
0.5	0.00726974804083431	0.120360487059049	0.855014975717658\\
0.5	0.00787632240459385	0.125262158509929	0.858458041096608\\
0.5	0.00850818805808555	0.13016907025918	0.860776163494182\\
0.5	0.00916544737962849	0.135081047277508	0.861942500814233\\
0.5	0.00984820134829017	0.139997912937243	0.861958614648411\\
0.5	0.0105565495182326	0.144919489023162	0.860852735617087\\
0.5	0.0112905899930939	0.149845595743738	0.858677054109646\\
0.5	0.012050419400414	0.154776051742839	0.855504274967457\\
0.5	0.0128361328661109	0.159710674111862	0.851423687967759\\
0.5	0.0136478239890177	0.164649278402306	0.84653700048434\\
0.5	0.0144855848154856	0.169591678638796	0.840954157904247\\
0.5	0.0153495058140643	0.174537687332543	0.834789345403929\\
0.5	0.0162396758502651	0.179487115495259	0.828157325806199\\
0.5	0.0171561821614173	0.184439772653511	0.821170226448696\\
0.5	0.0180991103316243	0.189395466863524	0.813934846725693\\
0.5	0.0190685442668299	0.194354004726437	0.806550519904208\\
0.5	0.0200645661700016	0.199315191403999	0.799107529846312\\
0.5	0.0210872565164405	0.204278830634725	0.791686056496189\\
0.5	0.0221366940292259	0.20924472475049	0.784355603830603\\
0.5	0.023212955654804	0.214212674693577	0.777174850281121\\
0.5	0.0243161165387281	0.219182480034174	0.770191853862643\\
0.5	0.025446250001561	0.224153938988321	0.763444541554457\\
0.5	0.0266034275149466	0.229126848436298	0.756961413910482\\
0.5	0.0277877186778607	0.234101003941464	0.750762400416764\\
0.5	0.0289991911930496	0.239076199769547	0.744859807805785\\
0.5	0.0302379108436654	0.244052228908366	0.73925931151808\\
0.5	0.0315039414701067	0.24902888308801	0.733960949044504\\
0.5	0.0327973449470747	0.254005952801449	0.728960082407259\\
0.5	0.0341181811608519	0.258983227325592	0.724248305111179\\
0.5	0.0354665079868145	0.263960494742775	0.719814276220681\\
0.5	0.0368423812671862	0.268937541962694	0.715644470612632\\
0.5	0.0382458547890421	0.273914154744766	0.711723839838041\\
0.5	0.0396769802625738	0.278890117720924	0.708036382389731\\
0.5	0.0411358072996216	0.283865214418836	0.704565625567164\\
0.5	0.0426223833924861	0.288839227285554	0.701295023640316\\
0.5	0.0441367538930258	0.293811937711588	0.698208278749354\\
0.5	0.0456789619920503	0.298783126055387	0.695289592055349\\
0.5	0.0472490486990192	0.303752571668251	0.692523853197512\\
0.5	0.0488470528220538	0.308720052919643	0.689896776228118\\
0.5	0.0504730109482718	0.313685347222913	0.687394989990281\\
0.5	0.0521269574244539	0.318648231061428	0.685006090465981\\
0.5	0.0538089243380495	0.323608480015096	0.682718662029128\\
0.5	0.0555189414985328	0.328565868787292	0.680522273853469\\
0.5	0.0572570364191156	0.333520171232163	0.67840745699785\\
0.5	0.0590232342988274	0.338471160382329	0.676365666960044\\
0.5	0.0608175580049697	0.34341860847696	0.674389235783157\\
0.5	0.0626400280559541	0.348362286990222	0.672471317134862\\
0.5	0.064490662604533	0.3533019666601	0.670605827172182\\
0.5	0.0663694774214291	0.358237417517579	0.668787383459802\\
0.5	0.0682764858793752	0.363168408916186	0.667011243730727\\
0.5	0.0702116989375697	0.368094709561873	0.665273245863648\\
0.5	0.0721751251265572	0.373016087543259	0.663569750098933\\
0.5	0.0741667705335422	0.377932310362199	0.66189758422006\\
0.5	0.0761866387881432	0.382843144964686	0.660253992184443\\
0.5	0.0782347310485954	0.38774835777208	0.658636586490997\\
0.5	0.0803110459884098	0.392647714712651	0.657043304415871\\
0.5	0.0824155797834956	0.397540981253432	0.65547236812651\\
0.5	0.084548326099754	0.402427922432375	0.653922248592746\\
0.5	0.0867092760811502	0.4073083028908	0.652391633146571\\
0.5	0.0888984183382709	0.412181886906127	0.65087939649558\\
0.5	0.0911157389373742	0.417048438424896	0.649384574964844\\
0.5	0.0933612213899392	0.421907721096044	0.647906343724683\\
0.5	0.0956348466427212	0.42675949830445	0.646443996754803\\
0.5	0.0979365930683194	0.431603533204733	0.644996929295963\\
0.5	0.100266436456264	0.436439588755293	0.643564622546847\\
0.5	0.102624350004627	0.441267427752584	0.642146630374431\\
0.5	0.105010304312169	0.446086812865616	0.640742567819433\\
0.5	0.107424267371012	0.450897506670668	0.639352101193731\\
0.5	0.109866204559871	0.455699271686218	0.637974939582442\\
0.5	0.112336078637817	0.460491870408058	0.636610827579862\\
0.5	0.114833849738607	0.465275065344603	0.635259539104384\\
0.5	0.117359475365564	0.470048619052377	0.633920872153155\\
0.5	0.119912910387022	0.474812294171664	0.632594644371965\\
0.5	0.122494107032347	0.479565853462319	0.631280689329641\\
0.5	0.125103014888515	0.484309059839721	0.629978853399059\\
0.5	0.127739580897283	0.489041676410868	0.628690460824559\\
0.505	0	0	0.710202759607373\\
0.505	1.11327767495586e-05	0.00471862581271274	0.713003157837853\\
0.505	4.46175251031912e-05	0.009446325184051	0.715846486660142\\
0.505	0.000100583311362513	0.0141829653360114	0.718788838379794\\
0.505	0.000179158434668431	0.018928411755669	0.721886092228356\\
0.505	0.000280470402701511	0.02368252819604	0.725200636220892\\
0.505	0.000404645907256436	0.0284451766772965	0.728799085274723\\
0.505	0.000551810799695644	0.0332162174883388	0.732749123983336\\
0.505	0.000722090066287311	0.037995509188729	0.737115575253328\\
0.505	0.000915607803432999	0.0427829086109896	0.741955893907011\\
0.505	0.0011324871927904	0.0475782708632729	0.747315373652289\\
0.505	0.00137285047629673	0.0523814493324038	0.75322242404581\\
0.505	0.00163681893109844	0.057192295687301	0.759684310525652\\
0.505	0.00192451284439304	0.0620106598827802	0.766683748102665\\
0.505	0.00223605148818898	0.0668363901637434	0.774176695511256\\
0.505	0.00257155309398959	0.0716693330697584	0.782091614494494\\
0.505	0.00293113482740716	0.0765093334400312	0.79033034642109\\
0.505	0.00331491276271365	0.0813562344187764	0.79877062752665\\
0.505	0.00372300185733414	0.0862098774609879	0.807270128930359\\
0.505	0.00415551592628969	0.0910701023386145	0.815671782700194\\
0.505	0.00461256761659621	0.0959367471471425	0.823810053507331\\
0.505	0.00509426838162598	0.100809648312589	0.831517746503354\\
0.505	0.00560072845543873	0.105688640598912	0.838632911419452\\
0.505	0.00613205682708918	0.11057355711583	0.845005411376294\\
0.505	0.00668836121491815	0.115464229327074	0.850502769049302\\
0.505	0.00726974804083431	0.120360487059049	0.855014975717657\\
0.505	0.00787632240459385	0.125262158509929	0.858458041096606\\
0.505	0.00850818805808555	0.13016907025918	0.860776163494182\\
0.505	0.00916544737962849	0.135081047277508	0.861942500814225\\
0.505	0.00984820134829017	0.139997912937243	0.861958614648421\\
0.505	0.0105565495182326	0.144919489023162	0.860852735617091\\
0.505	0.0112905899930939	0.149845595743738	0.858677054109652\\
0.505	0.012050419400414	0.154776051742839	0.855504274967457\\
0.505	0.0128361328661109	0.159710674111862	0.851423687967754\\
0.505	0.0136478239890177	0.164649278402306	0.84653700048434\\
0.505	0.0144855848154856	0.169591678638796	0.840954157904249\\
0.505	0.0153495058140643	0.174537687332543	0.834789345403935\\
0.505	0.0162396758502651	0.179487115495259	0.828157325806202\\
0.505	0.0171561821614173	0.184439772653511	0.821170226448694\\
0.505	0.0180991103316243	0.189395466863524	0.81393484672569\\
0.505	0.0190685442668299	0.194354004726437	0.806550519904212\\
0.505	0.0200645661700016	0.199315191403999	0.799107529846305\\
0.505	0.0210872565164405	0.204278830634725	0.791686056496193\\
0.505	0.0221366940292259	0.20924472475049	0.784355603830604\\
0.505	0.023212955654804	0.214212674693577	0.777174850281118\\
0.505	0.0243161165387281	0.219182480034174	0.770191853862644\\
0.505	0.025446250001561	0.224153938988321	0.763444541554458\\
0.505	0.0266034275149466	0.229126848436298	0.756961413910483\\
0.505	0.0277877186778607	0.234101003941464	0.750762400416764\\
0.505	0.0289991911930496	0.239076199769547	0.744859807805784\\
0.505	0.0302379108436654	0.244052228908366	0.73925931151808\\
0.505	0.0315039414701067	0.24902888308801	0.733960949044503\\
0.505	0.0327973449470747	0.254005952801449	0.728960082407262\\
0.505	0.0341181811608519	0.258983227325592	0.724248305111177\\
0.505	0.0354665079868145	0.263960494742775	0.719814276220681\\
0.505	0.0368423812671862	0.268937541962694	0.715644470612631\\
0.505	0.0382458547890421	0.273914154744766	0.711723839838042\\
0.505	0.0396769802625738	0.278890117720924	0.70803638238973\\
0.505	0.0411358072996216	0.283865214418836	0.704565625567163\\
0.505	0.0426223833924861	0.288839227285554	0.701295023640317\\
0.505	0.0441367538930258	0.293811937711588	0.698208278749353\\
0.505	0.0456789619920503	0.298783126055387	0.695289592055349\\
0.505	0.0472490486990192	0.303752571668251	0.692523853197513\\
0.505	0.0488470528220538	0.308720052919643	0.68989677622812\\
0.505	0.0504730109482718	0.313685347222913	0.687394989990281\\
0.505	0.0521269574244539	0.318648231061428	0.68500609046598\\
0.505	0.0538089243380495	0.323608480015096	0.682718662029128\\
0.505	0.0555189414985328	0.328565868787292	0.68052227385347\\
0.505	0.0572570364191156	0.333520171232163	0.678407456997849\\
0.505	0.0590232342988274	0.338471160382329	0.676365666960043\\
0.505	0.0608175580049697	0.34341860847696	0.674389235783156\\
0.505	0.0626400280559541	0.348362286990222	0.672471317134862\\
0.505	0.064490662604533	0.3533019666601	0.670605827172183\\
0.505	0.0663694774214291	0.358237417517579	0.668787383459804\\
0.505	0.0682764858793752	0.363168408916186	0.667011243730728\\
0.505	0.0702116989375697	0.368094709561873	0.665273245863648\\
0.505	0.0721751251265572	0.373016087543259	0.663569750098932\\
0.505	0.0741667705335423	0.377932310362199	0.661897584220061\\
0.505	0.0761866387881432	0.382843144964686	0.660253992184443\\
0.505	0.0782347310485954	0.38774835777208	0.658636586490997\\
0.505	0.0803110459884098	0.392647714712651	0.657043304415871\\
0.505	0.0824155797834956	0.397540981253432	0.655472368126509\\
0.505	0.084548326099754	0.402427922432375	0.653922248592745\\
0.505	0.0867092760811502	0.407308302890799	0.65239163314657\\
0.505	0.0888984183382709	0.412181886906127	0.650879396495579\\
0.505	0.0911157389373742	0.417048438424896	0.649384574964845\\
0.505	0.0933612213899392	0.421907721096044	0.647906343724686\\
0.505	0.0956348466427212	0.42675949830445	0.646443996754804\\
0.505	0.0979365930683194	0.431603533204733	0.644996929295961\\
0.505	0.100266436456264	0.436439588755293	0.643564622546846\\
0.505	0.102624350004627	0.441267427752584	0.64214663037443\\
0.505	0.105010304312169	0.446086812865616	0.640742567819433\\
0.505	0.107424267371012	0.450897506670668	0.639352101193731\\
0.505	0.109866204559871	0.455699271686218	0.637974939582444\\
0.505	0.112336078637817	0.460491870408058	0.636610827579861\\
0.505	0.114833849738607	0.465275065344603	0.635259539104383\\
0.505	0.117359475365564	0.470048619052377	0.633920872153155\\
0.505	0.119912910387022	0.474812294171664	0.632594644371966\\
0.505	0.122494107032347	0.479565853462319	0.631280689329641\\
0.505	0.125103014888515	0.484309059839721	0.629978853399059\\
0.505	0.127739580897283	0.489041676410868	0.62869046082456\\
0.51	0	0	0.710202759607373\\
0.51	1.11327767495586e-05	0.00471862581271274	0.713003157837853\\
0.51	4.46175251031912e-05	0.009446325184051	0.715846486660142\\
0.51	0.000100583311362513	0.0141829653360114	0.718788838379794\\
0.51	0.000179158434668431	0.018928411755669	0.721886092228356\\
0.51	0.000280470402701511	0.02368252819604	0.725200636220892\\
0.51	0.000404645907256436	0.0284451766772965	0.728799085274723\\
0.51	0.000551810799695644	0.0332162174883388	0.732749123983336\\
0.51	0.000722090066287311	0.037995509188729	0.737115575253327\\
0.51	0.000915607803432999	0.0427829086109896	0.741955893907011\\
0.51	0.0011324871927904	0.0475782708632729	0.747315373652289\\
0.51	0.00137285047629673	0.0523814493324038	0.75322242404581\\
0.51	0.00163681893109844	0.057192295687301	0.759684310525652\\
0.51	0.00192451284439304	0.0620106598827802	0.766683748102665\\
0.51	0.00223605148818898	0.0668363901637434	0.774176695511256\\
0.51	0.00257155309398959	0.0716693330697584	0.782091614494493\\
0.51	0.00293113482740716	0.0765093334400312	0.79033034642109\\
0.51	0.00331491276271365	0.0813562344187764	0.79877062752665\\
0.51	0.00372300185733413	0.0862098774609879	0.807270128930359\\
0.51	0.00415551592628969	0.0910701023386145	0.815671782700194\\
0.51	0.00461256761659621	0.0959367471471425	0.82381005350733\\
0.51	0.00509426838162598	0.100809648312589	0.831517746503354\\
0.51	0.00560072845543873	0.105688640598912	0.838632911419452\\
0.51	0.00613205682708918	0.11057355711583	0.845005411376292\\
0.51	0.00668836121491816	0.115464229327074	0.850502769049303\\
0.51	0.00726974804083431	0.120360487059049	0.855014975717657\\
0.51	0.00787632240459385	0.125262158509929	0.858458041096606\\
0.51	0.00850818805808555	0.13016907025918	0.860776163494181\\
0.51	0.00916544737962849	0.135081047277508	0.861942500814222\\
0.51	0.00984820134829017	0.139997912937243	0.861958614648422\\
0.51	0.0105565495182326	0.144919489023162	0.860852735617092\\
0.51	0.0112905899930939	0.149845595743738	0.858677054109651\\
0.51	0.012050419400414	0.154776051742839	0.855504274967456\\
0.51	0.0128361328661109	0.159710674111862	0.851423687967753\\
0.51	0.0136478239890177	0.164649278402306	0.846537000484341\\
0.51	0.0144855848154856	0.169591678638796	0.840954157904249\\
0.51	0.0153495058140643	0.174537687332543	0.834789345403937\\
0.51	0.0162396758502651	0.179487115495259	0.828157325806201\\
0.51	0.0171561821614173	0.184439772653511	0.821170226448692\\
0.51	0.0180991103316243	0.189395466863524	0.81393484672569\\
0.51	0.0190685442668299	0.194354004726437	0.806550519904214\\
0.51	0.0200645661700016	0.199315191403999	0.799107529846304\\
0.51	0.0210872565164405	0.204278830634725	0.791686056496195\\
0.51	0.0221366940292259	0.20924472475049	0.784355603830602\\
0.51	0.023212955654804	0.214212674693577	0.777174850281118\\
0.51	0.0243161165387281	0.219182480034174	0.770191853862643\\
0.51	0.025446250001561	0.224153938988321	0.763444541554457\\
0.51	0.0266034275149466	0.229126848436298	0.756961413910483\\
0.51	0.0277877186778607	0.234101003941464	0.750762400416763\\
0.51	0.0289991911930496	0.239076199769547	0.744859807805784\\
0.51	0.0302379108436654	0.244052228908366	0.73925931151808\\
0.51	0.0315039414701067	0.24902888308801	0.733960949044504\\
0.51	0.0327973449470747	0.254005952801449	0.728960082407259\\
0.51	0.0341181811608519	0.258983227325591	0.724248305111178\\
0.51	0.0354665079868145	0.263960494742775	0.719814276220682\\
0.51	0.0368423812671862	0.268937541962694	0.715644470612629\\
0.51	0.0382458547890421	0.273914154744766	0.711723839838042\\
0.51	0.0396769802625738	0.278890117720924	0.70803638238973\\
0.51	0.0411358072996216	0.283865214418836	0.704565625567166\\
0.51	0.0426223833924861	0.288839227285554	0.701295023640315\\
0.51	0.0441367538930258	0.293811937711588	0.698208278749353\\
0.51	0.0456789619920503	0.298783126055387	0.69528959205535\\
0.51	0.0472490486990192	0.303752571668251	0.69252385319751\\
0.51	0.0488470528220538	0.308720052919643	0.689896776228119\\
0.51	0.0504730109482718	0.313685347222913	0.687394989990282\\
0.51	0.0521269574244539	0.318648231061428	0.685006090465981\\
0.51	0.0538089243380495	0.323608480015096	0.682718662029128\\
0.51	0.0555189414985328	0.328565868787292	0.680522273853469\\
0.51	0.0572570364191156	0.333520171232163	0.67840745699785\\
0.51	0.0590232342988274	0.338471160382329	0.676365666960045\\
0.51	0.0608175580049697	0.34341860847696	0.674389235783156\\
0.51	0.0626400280559541	0.348362286990222	0.672471317134861\\
0.51	0.064490662604533	0.3533019666601	0.670605827172182\\
0.51	0.0663694774214291	0.358237417517579	0.668787383459805\\
0.51	0.0682764858793752	0.363168408916186	0.667011243730728\\
0.51	0.0702116989375697	0.368094709561873	0.665273245863648\\
0.51	0.0721751251265572	0.373016087543259	0.663569750098932\\
0.51	0.0741667705335422	0.377932310362199	0.66189758422006\\
0.51	0.0761866387881432	0.382843144964686	0.660253992184443\\
0.51	0.0782347310485954	0.38774835777208	0.658636586490998\\
0.51	0.0803110459884098	0.392647714712651	0.657043304415871\\
0.51	0.0824155797834956	0.397540981253432	0.65547236812651\\
0.51	0.084548326099754	0.402427922432375	0.653922248592745\\
0.51	0.0867092760811502	0.407308302890799	0.652391633146569\\
0.51	0.0888984183382709	0.412181886906127	0.650879396495579\\
0.51	0.0911157389373742	0.417048438424897	0.649384574964844\\
0.51	0.0933612213899392	0.421907721096044	0.647906343724685\\
0.51	0.0956348466427212	0.42675949830445	0.646443996754807\\
0.51	0.0979365930683194	0.431603533204733	0.644996929295962\\
0.51	0.100266436456264	0.436439588755293	0.643564622546847\\
0.51	0.102624350004627	0.441267427752584	0.642146630374428\\
0.51	0.105010304312169	0.446086812865616	0.640742567819433\\
0.51	0.107424267371012	0.450897506670668	0.63935210119373\\
0.51	0.109866204559871	0.455699271686218	0.637974939582443\\
0.51	0.112336078637817	0.460491870408058	0.636610827579862\\
0.51	0.114833849738607	0.465275065344603	0.635259539104382\\
0.51	0.117359475365564	0.470048619052377	0.633920872153154\\
0.51	0.119912910387023	0.474812294171664	0.632594644371965\\
0.51	0.122494107032347	0.479565853462319	0.631280689329641\\
0.51	0.125103014888515	0.484309059839721	0.629978853399059\\
0.51	0.127739580897283	0.489041676410868	0.628690460824559\\
0.515	0	0	0.710202759607373\\
0.515	1.11327767495586e-05	0.00471862581271275	0.713003157837853\\
0.515	4.46175251031912e-05	0.009446325184051	0.715846486660142\\
0.515	0.000100583311362513	0.0141829653360114	0.718788838379794\\
0.515	0.000179158434668431	0.018928411755669	0.721886092228356\\
0.515	0.000280470402701511	0.02368252819604	0.725200636220892\\
0.515	0.000404645907256436	0.0284451766772965	0.728799085274723\\
0.515	0.000551810799695644	0.0332162174883389	0.732749123983336\\
0.515	0.000722090066287311	0.037995509188729	0.737115575253327\\
0.515	0.000915607803432999	0.0427829086109896	0.741955893907011\\
0.515	0.0011324871927904	0.0475782708632729	0.747315373652289\\
0.515	0.00137285047629673	0.0523814493324038	0.753222424045811\\
0.515	0.00163681893109844	0.057192295687301	0.759684310525652\\
0.515	0.00192451284439304	0.0620106598827802	0.766683748102665\\
0.515	0.00223605148818898	0.0668363901637434	0.774176695511256\\
0.515	0.00257155309398959	0.0716693330697584	0.782091614494494\\
0.515	0.00293113482740716	0.0765093334400312	0.79033034642109\\
0.515	0.00331491276271365	0.0813562344187764	0.79877062752665\\
0.515	0.00372300185733413	0.0862098774609879	0.807270128930359\\
0.515	0.00415551592628969	0.0910701023386145	0.815671782700193\\
0.515	0.00461256761659621	0.0959367471471425	0.823810053507331\\
0.515	0.00509426838162598	0.100809648312589	0.831517746503354\\
0.515	0.00560072845543873	0.105688640598912	0.838632911419452\\
0.515	0.00613205682708918	0.11057355711583	0.845005411376293\\
0.515	0.00668836121491816	0.115464229327074	0.850502769049302\\
0.515	0.00726974804083431	0.120360487059049	0.855014975717656\\
0.515	0.00787632240459385	0.125262158509929	0.858458041096607\\
0.515	0.00850818805808555	0.13016907025918	0.860776163494181\\
0.515	0.00916544737962849	0.135081047277508	0.861942500814224\\
0.515	0.00984820134829017	0.139997912937243	0.861958614648418\\
0.515	0.0105565495182326	0.144919489023162	0.860852735617091\\
0.515	0.0112905899930939	0.149845595743738	0.85867705410965\\
0.515	0.012050419400414	0.154776051742839	0.855504274967457\\
0.515	0.0128361328661109	0.159710674111862	0.851423687967756\\
0.515	0.0136478239890177	0.164649278402306	0.84653700048434\\
0.515	0.0144855848154856	0.169591678638796	0.840954157904249\\
0.515	0.0153495058140643	0.174537687332543	0.834789345403935\\
0.515	0.0162396758502651	0.179487115495259	0.828157325806199\\
0.515	0.0171561821614173	0.184439772653511	0.821170226448693\\
0.515	0.0180991103316243	0.189395466863524	0.813934846725693\\
0.515	0.0190685442668299	0.194354004726437	0.806550519904216\\
0.515	0.0200645661700016	0.199315191403999	0.799107529846304\\
0.515	0.0210872565164405	0.204278830634725	0.791686056496193\\
0.515	0.0221366940292259	0.20924472475049	0.784355603830602\\
0.515	0.023212955654804	0.214212674693577	0.777174850281118\\
0.515	0.0243161165387281	0.219182480034174	0.770191853862642\\
0.515	0.025446250001561	0.224153938988321	0.76344454155446\\
0.515	0.0266034275149466	0.229126848436298	0.756961413910481\\
0.515	0.0277877186778607	0.234101003941464	0.750762400416761\\
0.515	0.0289991911930496	0.239076199769547	0.744859807805786\\
0.515	0.0302379108436654	0.244052228908366	0.739259311518081\\
0.515	0.0315039414701067	0.24902888308801	0.733960949044503\\
0.515	0.0327973449470747	0.254005952801449	0.72896008240726\\
0.515	0.0341181811608519	0.258983227325592	0.724248305111177\\
0.515	0.0354665079868145	0.263960494742775	0.719814276220681\\
0.515	0.0368423812671862	0.268937541962694	0.71564447061263\\
0.515	0.0382458547890421	0.273914154744766	0.711723839838041\\
0.515	0.0396769802625738	0.278890117720924	0.70803638238973\\
0.515	0.0411358072996216	0.283865214418836	0.704565625567165\\
0.515	0.0426223833924861	0.288839227285554	0.701295023640316\\
0.515	0.0441367538930258	0.293811937711588	0.698208278749354\\
0.515	0.0456789619920503	0.298783126055387	0.695289592055349\\
0.515	0.0472490486990192	0.303752571668251	0.692523853197511\\
0.515	0.0488470528220538	0.308720052919643	0.68989677622812\\
0.515	0.0504730109482718	0.313685347222913	0.68739498999028\\
0.515	0.0521269574244539	0.318648231061427	0.68500609046598\\
0.515	0.0538089243380495	0.323608480015097	0.682718662029128\\
0.515	0.0555189414985328	0.328565868787292	0.680522273853469\\
0.515	0.0572570364191156	0.333520171232163	0.67840745699785\\
0.515	0.0590232342988274	0.338471160382329	0.676365666960044\\
0.515	0.0608175580049697	0.34341860847696	0.674389235783156\\
0.515	0.0626400280559541	0.348362286990222	0.672471317134861\\
0.515	0.064490662604533	0.3533019666601	0.670605827172182\\
0.515	0.0663694774214291	0.358237417517579	0.668787383459804\\
0.515	0.0682764858793752	0.363168408916186	0.667011243730729\\
0.515	0.0702116989375697	0.368094709561873	0.665273245863649\\
0.515	0.0721751251265572	0.373016087543259	0.663569750098931\\
0.515	0.0741667705335422	0.377932310362199	0.66189758422006\\
0.515	0.0761866387881432	0.382843144964686	0.660253992184442\\
0.515	0.0782347310485954	0.38774835777208	0.658636586490997\\
0.515	0.0803110459884098	0.392647714712651	0.657043304415871\\
0.515	0.0824155797834956	0.397540981253432	0.655472368126509\\
0.515	0.084548326099754	0.402427922432375	0.653922248592747\\
0.515	0.0867092760811501	0.407308302890799	0.652391633146571\\
0.515	0.0888984183382709	0.412181886906127	0.650879396495578\\
0.515	0.0911157389373742	0.417048438424896	0.649384574964843\\
0.515	0.0933612213899392	0.421907721096044	0.647906343724685\\
0.515	0.0956348466427212	0.42675949830445	0.646443996754805\\
0.515	0.0979365930683194	0.431603533204733	0.644996929295962\\
0.515	0.100266436456264	0.436439588755293	0.643564622546847\\
0.515	0.102624350004627	0.441267427752585	0.64214663037443\\
0.515	0.105010304312169	0.446086812865616	0.640742567819432\\
0.515	0.107424267371012	0.450897506670668	0.639352101193731\\
0.515	0.109866204559871	0.455699271686218	0.637974939582443\\
0.515	0.112336078637817	0.460491870408058	0.636610827579862\\
0.515	0.114833849738607	0.465275065344603	0.635259539104383\\
0.515	0.117359475365564	0.470048619052377	0.633920872153155\\
0.515	0.119912910387022	0.474812294171664	0.632594644371964\\
0.515	0.122494107032347	0.479565853462319	0.63128068932964\\
0.515	0.125103014888515	0.484309059839721	0.629978853399057\\
0.515	0.127739580897283	0.489041676410868	0.628690460824557\\
0.52	0	0	0.710202759607373\\
0.52	1.11327767495586e-05	0.00471862581271275	0.713003157837853\\
0.52	4.46175251031912e-05	0.009446325184051	0.715846486660142\\
0.52	0.000100583311362513	0.0141829653360114	0.718788838379794\\
0.52	0.000179158434668431	0.018928411755669	0.721886092228356\\
0.52	0.000280470402701511	0.02368252819604	0.725200636220892\\
0.52	0.000404645907256436	0.0284451766772965	0.728799085274723\\
0.52	0.000551810799695644	0.0332162174883389	0.732749123983336\\
0.52	0.000722090066287311	0.037995509188729	0.737115575253328\\
0.52	0.000915607803432999	0.0427829086109896	0.741955893907011\\
0.52	0.0011324871927904	0.0475782708632729	0.747315373652289\\
0.52	0.00137285047629673	0.0523814493324038	0.753222424045811\\
0.52	0.00163681893109844	0.057192295687301	0.759684310525652\\
0.52	0.00192451284439304	0.0620106598827802	0.766683748102665\\
0.52	0.00223605148818898	0.0668363901637434	0.774176695511256\\
0.52	0.00257155309398959	0.0716693330697584	0.782091614494493\\
0.52	0.00293113482740716	0.0765093334400312	0.79033034642109\\
0.52	0.00331491276271365	0.0813562344187764	0.79877062752665\\
0.52	0.00372300185733413	0.0862098774609879	0.807270128930359\\
0.52	0.00415551592628969	0.0910701023386145	0.815671782700193\\
0.52	0.00461256761659621	0.0959367471471425	0.82381005350733\\
0.52	0.00509426838162598	0.100809648312589	0.831517746503354\\
0.52	0.00560072845543873	0.105688640598912	0.838632911419452\\
0.52	0.00613205682708918	0.11057355711583	0.845005411376293\\
0.52	0.00668836121491816	0.115464229327074	0.850502769049301\\
0.52	0.00726974804083431	0.120360487059049	0.855014975717657\\
0.52	0.00787632240459385	0.125262158509929	0.858458041096605\\
0.52	0.00850818805808554	0.13016907025918	0.860776163494183\\
0.52	0.00916544737962849	0.135081047277508	0.861942500814225\\
0.52	0.00984820134829017	0.139997912937243	0.86195861464842\\
0.52	0.0105565495182326	0.144919489023162	0.86085273561709\\
0.52	0.0112905899930939	0.149845595743738	0.858677054109651\\
0.52	0.012050419400414	0.154776051742839	0.855504274967457\\
0.52	0.0128361328661109	0.159710674111862	0.851423687967756\\
0.52	0.0136478239890177	0.164649278402306	0.846537000484339\\
0.52	0.0144855848154856	0.169591678638796	0.840954157904249\\
0.52	0.0153495058140643	0.174537687332543	0.834789345403935\\
0.52	0.0162396758502651	0.179487115495259	0.828157325806199\\
0.52	0.0171561821614173	0.184439772653511	0.821170226448693\\
0.52	0.0180991103316243	0.189395466863524	0.813934846725693\\
0.52	0.0190685442668299	0.194354004726437	0.806550519904212\\
0.52	0.0200645661700016	0.199315191403999	0.799107529846303\\
0.52	0.0210872565164405	0.204278830634725	0.791686056496195\\
0.52	0.0221366940292259	0.20924472475049	0.784355603830603\\
0.52	0.023212955654804	0.214212674693577	0.777174850281116\\
0.52	0.0243161165387281	0.219182480034174	0.770191853862644\\
0.52	0.025446250001561	0.224153938988321	0.763444541554457\\
0.52	0.0266034275149466	0.229126848436298	0.756961413910481\\
0.52	0.0277877186778607	0.234101003941464	0.750762400416763\\
0.52	0.0289991911930496	0.239076199769547	0.744859807805786\\
0.52	0.0302379108436654	0.244052228908366	0.73925931151808\\
0.52	0.0315039414701067	0.24902888308801	0.733960949044504\\
0.52	0.0327973449470747	0.254005952801449	0.72896008240726\\
0.52	0.0341181811608519	0.258983227325591	0.724248305111177\\
0.52	0.0354665079868145	0.263960494742775	0.719814276220682\\
0.52	0.0368423812671862	0.268937541962694	0.715644470612629\\
0.52	0.0382458547890421	0.273914154744766	0.711723839838041\\
0.52	0.0396769802625738	0.278890117720924	0.708036382389732\\
0.52	0.0411358072996216	0.283865214418836	0.704565625567165\\
0.52	0.0426223833924861	0.288839227285554	0.701295023640315\\
0.52	0.0441367538930258	0.293811937711588	0.698208278749353\\
0.52	0.0456789619920503	0.298783126055387	0.695289592055351\\
0.52	0.0472490486990192	0.303752571668251	0.692523853197512\\
0.52	0.0488470528220538	0.308720052919643	0.689896776228118\\
0.52	0.0504730109482718	0.313685347222913	0.687394989990282\\
0.52	0.0521269574244539	0.318648231061428	0.68500609046598\\
0.52	0.0538089243380495	0.323608480015096	0.682718662029128\\
0.52	0.0555189414985328	0.328565868787292	0.68052227385347\\
0.52	0.0572570364191156	0.333520171232163	0.67840745699785\\
0.52	0.0590232342988274	0.338471160382329	0.676365666960045\\
0.52	0.0608175580049697	0.34341860847696	0.674389235783157\\
0.52	0.0626400280559541	0.348362286990222	0.672471317134861\\
0.52	0.064490662604533	0.3533019666601	0.670605827172182\\
0.52	0.0663694774214291	0.358237417517579	0.668787383459804\\
0.52	0.0682764858793752	0.363168408916186	0.667011243730729\\
0.52	0.0702116989375697	0.368094709561873	0.66527324586365\\
0.52	0.0721751251265572	0.373016087543259	0.663569750098931\\
0.52	0.0741667705335422	0.377932310362199	0.66189758422006\\
0.52	0.0761866387881432	0.382843144964686	0.660253992184443\\
0.52	0.0782347310485954	0.38774835777208	0.658636586490997\\
0.52	0.0803110459884098	0.392647714712651	0.657043304415871\\
0.52	0.0824155797834956	0.397540981253432	0.655472368126508\\
0.52	0.084548326099754	0.402427922432375	0.653922248592746\\
0.52	0.0867092760811502	0.407308302890799	0.652391633146571\\
0.52	0.0888984183382709	0.412181886906127	0.65087939649558\\
0.52	0.0911157389373742	0.417048438424896	0.649384574964843\\
0.52	0.0933612213899392	0.421907721096044	0.647906343724684\\
0.52	0.0956348466427212	0.42675949830445	0.646443996754806\\
0.52	0.0979365930683194	0.431603533204733	0.644996929295963\\
0.52	0.100266436456264	0.436439588755293	0.643564622546847\\
0.52	0.102624350004627	0.441267427752584	0.64214663037443\\
0.52	0.105010304312169	0.446086812865616	0.640742567819433\\
0.52	0.107424267371012	0.450897506670668	0.63935210119373\\
0.52	0.109866204559871	0.455699271686218	0.637974939582443\\
0.52	0.112336078637817	0.460491870408058	0.636610827579862\\
0.52	0.114833849738607	0.465275065344603	0.635259539104382\\
0.52	0.117359475365564	0.470048619052377	0.633920872153155\\
0.52	0.119912910387022	0.474812294171664	0.632594644371965\\
0.52	0.122494107032347	0.479565853462319	0.631280689329641\\
0.52	0.125103014888515	0.484309059839721	0.629978853399058\\
0.52	0.127739580897283	0.489041676410868	0.628690460824553\\
0.525	0	0	0.710202759607373\\
0.525	1.11327767495586e-05	0.00471862581271275	0.713003157837853\\
0.525	4.46175251031912e-05	0.009446325184051	0.715846486660142\\
0.525	0.000100583311362513	0.0141829653360114	0.718788838379794\\
0.525	0.000179158434668431	0.018928411755669	0.721886092228356\\
0.525	0.000280470402701511	0.02368252819604	0.725200636220892\\
0.525	0.000404645907256436	0.0284451766772965	0.728799085274723\\
0.525	0.000551810799695644	0.0332162174883388	0.732749123983336\\
0.525	0.000722090066287311	0.037995509188729	0.737115575253327\\
0.525	0.000915607803432999	0.0427829086109896	0.741955893907011\\
0.525	0.0011324871927904	0.0475782708632729	0.747315373652289\\
0.525	0.00137285047629673	0.0523814493324038	0.753222424045811\\
0.525	0.00163681893109844	0.057192295687301	0.759684310525652\\
0.525	0.00192451284439304	0.0620106598827802	0.766683748102665\\
0.525	0.00223605148818898	0.0668363901637434	0.774176695511256\\
0.525	0.00257155309398959	0.0716693330697584	0.782091614494494\\
0.525	0.00293113482740716	0.0765093334400312	0.79033034642109\\
0.525	0.00331491276271365	0.0813562344187764	0.79877062752665\\
0.525	0.00372300185733414	0.0862098774609879	0.80727012893036\\
0.525	0.00415551592628969	0.0910701023386145	0.815671782700193\\
0.525	0.00461256761659621	0.0959367471471425	0.82381005350733\\
0.525	0.00509426838162598	0.100809648312589	0.831517746503354\\
0.525	0.00560072845543873	0.105688640598912	0.838632911419452\\
0.525	0.00613205682708918	0.11057355711583	0.845005411376293\\
0.525	0.00668836121491816	0.115464229327074	0.850502769049302\\
0.525	0.00726974804083431	0.120360487059049	0.855014975717656\\
0.525	0.00787632240459385	0.125262158509929	0.858458041096605\\
0.525	0.00850818805808555	0.13016907025918	0.860776163494182\\
0.525	0.00916544737962849	0.135081047277508	0.861942500814223\\
0.525	0.00984820134829017	0.139997912937243	0.861958614648422\\
0.525	0.0105565495182326	0.144919489023162	0.86085273561709\\
0.525	0.0112905899930939	0.149845595743738	0.85867705410965\\
0.525	0.012050419400414	0.154776051742839	0.855504274967459\\
0.525	0.0128361328661109	0.159710674111862	0.851423687967754\\
0.525	0.0136478239890177	0.164649278402306	0.84653700048434\\
0.525	0.0144855848154856	0.169591678638796	0.840954157904247\\
0.525	0.0153495058140643	0.174537687332543	0.834789345403933\\
0.525	0.0162396758502651	0.179487115495259	0.828157325806198\\
0.525	0.0171561821614173	0.184439772653511	0.821170226448693\\
0.525	0.0180991103316243	0.189395466863524	0.813934846725692\\
0.525	0.0190685442668299	0.194354004726437	0.806550519904211\\
0.525	0.0200645661700016	0.199315191403999	0.799107529846304\\
0.525	0.0210872565164405	0.204278830634725	0.791686056496195\\
0.525	0.0221366940292259	0.20924472475049	0.7843556038306\\
0.525	0.023212955654804	0.214212674693577	0.777174850281118\\
0.525	0.0243161165387281	0.219182480034174	0.770191853862643\\
0.525	0.025446250001561	0.224153938988321	0.763444541554458\\
0.525	0.0266034275149466	0.229126848436298	0.756961413910482\\
0.525	0.0277877186778607	0.234101003941464	0.750762400416764\\
0.525	0.0289991911930496	0.239076199769547	0.744859807805788\\
0.525	0.0302379108436654	0.244052228908366	0.739259311518081\\
0.525	0.0315039414701067	0.24902888308801	0.733960949044501\\
0.525	0.0327973449470747	0.254005952801449	0.728960082407259\\
0.525	0.0341181811608519	0.258983227325591	0.724248305111178\\
0.525	0.0354665079868145	0.263960494742775	0.719814276220682\\
0.525	0.0368423812671862	0.268937541962694	0.715644470612631\\
0.525	0.0382458547890421	0.273914154744766	0.711723839838043\\
0.525	0.0396769802625737	0.278890117720924	0.70803638238973\\
0.525	0.0411358072996216	0.283865214418836	0.704565625567163\\
0.525	0.0426223833924861	0.288839227285554	0.701295023640317\\
0.525	0.0441367538930258	0.293811937711588	0.698208278749354\\
0.525	0.0456789619920503	0.298783126055387	0.69528959205535\\
0.525	0.0472490486990192	0.303752571668251	0.69252385319751\\
0.525	0.0488470528220538	0.308720052919643	0.689896776228117\\
0.525	0.0504730109482718	0.313685347222913	0.687394989990282\\
0.525	0.0521269574244539	0.318648231061428	0.685006090465982\\
0.525	0.0538089243380495	0.323608480015097	0.682718662029128\\
0.525	0.0555189414985328	0.328565868787292	0.680522273853468\\
0.525	0.0572570364191156	0.333520171232163	0.678407456997851\\
0.525	0.0590232342988274	0.338471160382329	0.676365666960044\\
0.525	0.0608175580049697	0.34341860847696	0.674389235783156\\
0.525	0.0626400280559541	0.348362286990222	0.672471317134862\\
0.525	0.064490662604533	0.3533019666601	0.670605827172181\\
0.525	0.0663694774214291	0.358237417517579	0.668787383459803\\
0.525	0.0682764858793752	0.363168408916186	0.667011243730729\\
0.525	0.0702116989375697	0.368094709561873	0.66527324586365\\
0.525	0.0721751251265572	0.373016087543259	0.663569750098932\\
0.525	0.0741667705335423	0.377932310362199	0.661897584220059\\
0.525	0.0761866387881432	0.382843144964686	0.660253992184442\\
0.525	0.0782347310485954	0.38774835777208	0.658636586490997\\
0.525	0.0803110459884098	0.392647714712651	0.657043304415871\\
0.525	0.0824155797834956	0.397540981253432	0.655472368126508\\
0.525	0.084548326099754	0.402427922432375	0.653922248592745\\
0.525	0.0867092760811502	0.4073083028908	0.652391633146569\\
0.525	0.0888984183382709	0.412181886906127	0.650879396495579\\
0.525	0.0911157389373742	0.417048438424896	0.649384574964845\\
0.525	0.0933612213899392	0.421907721096044	0.647906343724685\\
0.525	0.0956348466427212	0.42675949830445	0.646443996754804\\
0.525	0.0979365930683194	0.431603533204733	0.644996929295962\\
0.525	0.100266436456264	0.436439588755293	0.643564622546848\\
0.525	0.102624350004627	0.441267427752584	0.64214663037443\\
0.525	0.105010304312169	0.446086812865616	0.640742567819433\\
0.525	0.107424267371012	0.450897506670668	0.639352101193731\\
0.525	0.109866204559871	0.455699271686218	0.637974939582442\\
0.525	0.112336078637817	0.460491870408058	0.636610827579862\\
0.525	0.114833849738607	0.465275065344603	0.635259539104382\\
0.525	0.117359475365564	0.470048619052377	0.633920872153155\\
0.525	0.119912910387022	0.474812294171664	0.632594644371966\\
0.525	0.122494107032347	0.479565853462319	0.63128068932964\\
0.525	0.125103014888515	0.484309059839721	0.629978853399059\\
0.525	0.127739580897283	0.489041676410868	0.628690460824559\\
0.53	0	0	0.710202759607373\\
0.53	1.11327767495586e-05	0.00471862581271274	0.713003157837853\\
0.53	4.46175251031912e-05	0.009446325184051	0.715846486660142\\
0.53	0.000100583311362513	0.0141829653360114	0.718788838379794\\
0.53	0.000179158434668431	0.018928411755669	0.721886092228356\\
0.53	0.000280470402701511	0.02368252819604	0.725200636220892\\
0.53	0.000404645907256436	0.0284451766772965	0.728799085274723\\
0.53	0.000551810799695644	0.0332162174883388	0.732749123983336\\
0.53	0.000722090066287311	0.037995509188729	0.737115575253328\\
0.53	0.000915607803432999	0.0427829086109896	0.741955893907011\\
0.53	0.0011324871927904	0.0475782708632729	0.747315373652289\\
0.53	0.00137285047629673	0.0523814493324038	0.753222424045811\\
0.53	0.00163681893109844	0.057192295687301	0.759684310525652\\
0.53	0.00192451284439304	0.0620106598827802	0.766683748102665\\
0.53	0.00223605148818898	0.0668363901637434	0.774176695511256\\
0.53	0.00257155309398959	0.0716693330697584	0.782091614494493\\
0.53	0.00293113482740716	0.0765093334400312	0.79033034642109\\
0.53	0.00331491276271365	0.0813562344187764	0.79877062752665\\
0.53	0.00372300185733413	0.0862098774609879	0.807270128930359\\
0.53	0.00415551592628969	0.0910701023386145	0.815671782700194\\
0.53	0.00461256761659621	0.0959367471471425	0.82381005350733\\
0.53	0.00509426838162598	0.100809648312589	0.831517746503354\\
0.53	0.00560072845543873	0.105688640598912	0.838632911419452\\
0.53	0.00613205682708918	0.11057355711583	0.845005411376292\\
0.53	0.00668836121491815	0.115464229327074	0.850502769049303\\
0.53	0.00726974804083431	0.120360487059049	0.855014975717658\\
0.53	0.00787632240459385	0.125262158509929	0.858458041096606\\
0.53	0.00850818805808555	0.13016907025918	0.860776163494181\\
0.53	0.00916544737962849	0.135081047277508	0.861942500814225\\
0.53	0.00984820134829017	0.139997912937243	0.861958614648418\\
0.53	0.0105565495182326	0.144919489023162	0.86085273561709\\
0.53	0.0112905899930939	0.149845595743738	0.85867705410965\\
0.53	0.012050419400414	0.154776051742839	0.855504274967458\\
0.53	0.0128361328661109	0.159710674111862	0.851423687967754\\
0.53	0.0136478239890177	0.164649278402306	0.846537000484342\\
0.53	0.0144855848154856	0.169591678638796	0.840954157904249\\
0.53	0.0153495058140643	0.174537687332543	0.834789345403934\\
0.53	0.0162396758502651	0.179487115495259	0.828157325806201\\
0.53	0.0171561821614173	0.184439772653511	0.821170226448696\\
0.53	0.0180991103316243	0.189395466863524	0.813934846725694\\
0.53	0.0190685442668299	0.194354004726437	0.806550519904213\\
0.53	0.0200645661700016	0.199315191403999	0.799107529846304\\
0.53	0.0210872565164405	0.204278830634725	0.791686056496196\\
0.53	0.0221366940292259	0.20924472475049	0.784355603830602\\
0.53	0.023212955654804	0.214212674693577	0.77717485028112\\
0.53	0.0243161165387281	0.219182480034174	0.770191853862643\\
0.53	0.025446250001561	0.224153938988321	0.763444541554458\\
0.53	0.0266034275149466	0.229126848436298	0.756961413910482\\
0.53	0.0277877186778607	0.234101003941464	0.750762400416766\\
0.53	0.0289991911930496	0.239076199769547	0.744859807805786\\
0.53	0.0302379108436654	0.244052228908366	0.739259311518078\\
0.53	0.0315039414701067	0.24902888308801	0.733960949044502\\
0.53	0.0327973449470747	0.254005952801449	0.72896008240726\\
0.53	0.0341181811608519	0.258983227325591	0.724248305111177\\
0.53	0.0354665079868145	0.263960494742775	0.719814276220683\\
0.53	0.0368423812671862	0.268937541962694	0.71564447061263\\
0.53	0.0382458547890421	0.273914154744766	0.711723839838042\\
0.53	0.0396769802625738	0.278890117720924	0.70803638238973\\
0.53	0.0411358072996216	0.283865214418836	0.704565625567165\\
0.53	0.0426223833924861	0.288839227285554	0.701295023640315\\
0.53	0.0441367538930258	0.293811937711588	0.698208278749353\\
0.53	0.0456789619920503	0.298783126055387	0.695289592055351\\
0.53	0.0472490486990192	0.303752571668251	0.69252385319751\\
0.53	0.0488470528220538	0.308720052919643	0.689896776228118\\
0.53	0.0504730109482718	0.313685347222913	0.687394989990282\\
0.53	0.0521269574244539	0.318648231061428	0.68500609046598\\
0.53	0.0538089243380495	0.323608480015096	0.682718662029128\\
0.53	0.0555189414985328	0.328565868787292	0.680522273853469\\
0.53	0.0572570364191156	0.333520171232163	0.678407456997849\\
0.53	0.0590232342988274	0.338471160382329	0.676365666960044\\
0.53	0.0608175580049697	0.34341860847696	0.674389235783156\\
0.53	0.0626400280559541	0.348362286990222	0.672471317134862\\
0.53	0.064490662604533	0.3533019666601	0.670605827172182\\
0.53	0.0663694774214291	0.358237417517579	0.668787383459803\\
0.53	0.0682764858793752	0.363168408916186	0.667011243730728\\
0.53	0.0702116989375697	0.368094709561873	0.66527324586365\\
0.53	0.0721751251265572	0.373016087543259	0.663569750098933\\
0.53	0.0741667705335423	0.377932310362199	0.66189758422006\\
0.53	0.0761866387881432	0.382843144964686	0.660253992184442\\
0.53	0.0782347310485954	0.38774835777208	0.658636586490997\\
0.53	0.0803110459884098	0.392647714712651	0.657043304415872\\
0.53	0.0824155797834956	0.397540981253432	0.655472368126509\\
0.53	0.084548326099754	0.402427922432375	0.653922248592744\\
0.53	0.0867092760811502	0.407308302890799	0.652391633146571\\
0.53	0.0888984183382709	0.412181886906127	0.650879396495579\\
0.53	0.0911157389373742	0.417048438424897	0.649384574964844\\
0.53	0.0933612213899392	0.421907721096044	0.647906343724685\\
0.53	0.0956348466427212	0.42675949830445	0.646443996754805\\
0.53	0.0979365930683194	0.431603533204733	0.644996929295961\\
0.53	0.100266436456264	0.436439588755293	0.643564622546847\\
0.53	0.102624350004627	0.441267427752584	0.64214663037443\\
0.53	0.105010304312169	0.446086812865616	0.640742567819433\\
0.53	0.107424267371012	0.450897506670668	0.639352101193731\\
0.53	0.109866204559871	0.455699271686218	0.637974939582444\\
0.53	0.112336078637817	0.460491870408058	0.636610827579861\\
0.53	0.114833849738607	0.465275065344603	0.635259539104382\\
0.53	0.117359475365564	0.470048619052377	0.633920872153154\\
0.53	0.119912910387022	0.474812294171664	0.632594644371966\\
0.53	0.122494107032347	0.479565853462319	0.63128068932964\\
0.53	0.125103014888515	0.484309059839721	0.629978853399057\\
0.53	0.127739580897283	0.489041676410868	0.628690460824558\\
0.535	0	0	0.710202759607373\\
0.535	1.11327767495586e-05	0.00471862581271275	0.713003157837853\\
0.535	4.46175251031912e-05	0.009446325184051	0.715846486660142\\
0.535	0.000100583311362513	0.0141829653360114	0.718788838379794\\
0.535	0.000179158434668431	0.018928411755669	0.721886092228356\\
0.535	0.000280470402701511	0.02368252819604	0.725200636220892\\
0.535	0.000404645907256436	0.0284451766772965	0.728799085274723\\
0.535	0.000551810799695644	0.0332162174883388	0.732749123983336\\
0.535	0.000722090066287311	0.037995509188729	0.737115575253328\\
0.535	0.000915607803432999	0.0427829086109896	0.741955893907011\\
0.535	0.0011324871927904	0.0475782708632729	0.747315373652289\\
0.535	0.00137285047629673	0.0523814493324038	0.753222424045811\\
0.535	0.00163681893109844	0.057192295687301	0.759684310525652\\
0.535	0.00192451284439304	0.0620106598827802	0.766683748102665\\
0.535	0.00223605148818899	0.0668363901637434	0.774176695511257\\
0.535	0.00257155309398959	0.0716693330697584	0.782091614494493\\
0.535	0.00293113482740716	0.0765093334400312	0.79033034642109\\
0.535	0.00331491276271365	0.0813562344187764	0.79877062752665\\
0.535	0.00372300185733414	0.0862098774609879	0.807270128930359\\
0.535	0.00415551592628969	0.0910701023386145	0.815671782700193\\
0.535	0.00461256761659621	0.0959367471471425	0.823810053507331\\
0.535	0.00509426838162598	0.100809648312589	0.831517746503354\\
0.535	0.00560072845543873	0.105688640598912	0.838632911419452\\
0.535	0.00613205682708918	0.11057355711583	0.845005411376292\\
0.535	0.00668836121491816	0.115464229327074	0.850502769049302\\
0.535	0.00726974804083431	0.120360487059049	0.855014975717658\\
0.535	0.00787632240459385	0.125262158509929	0.858458041096606\\
0.535	0.00850818805808555	0.13016907025918	0.860776163494182\\
0.535	0.00916544737962849	0.135081047277508	0.861942500814227\\
0.535	0.00984820134829017	0.139997912937243	0.861958614648419\\
0.535	0.0105565495182326	0.144919489023162	0.86085273561709\\
0.535	0.0112905899930939	0.149845595743738	0.85867705410965\\
0.535	0.012050419400414	0.154776051742839	0.855504274967458\\
0.535	0.0128361328661109	0.159710674111862	0.851423687967754\\
0.535	0.0136478239890177	0.164649278402306	0.84653700048434\\
0.535	0.0144855848154856	0.169591678638796	0.840954157904248\\
0.535	0.0153495058140643	0.174537687332543	0.834789345403935\\
0.535	0.0162396758502651	0.179487115495259	0.8281573258062\\
0.535	0.0171561821614173	0.184439772653511	0.821170226448695\\
0.535	0.0180991103316243	0.189395466863524	0.813934846725692\\
0.535	0.0190685442668299	0.194354004726437	0.806550519904213\\
0.535	0.0200645661700016	0.199315191403999	0.799107529846306\\
0.535	0.0210872565164405	0.204278830634725	0.791686056496195\\
0.535	0.0221366940292259	0.20924472475049	0.784355603830601\\
0.535	0.023212955654804	0.214212674693577	0.777174850281118\\
0.535	0.0243161165387281	0.219182480034174	0.770191853862644\\
0.535	0.025446250001561	0.224153938988321	0.763444541554458\\
0.535	0.0266034275149466	0.229126848436298	0.756961413910483\\
0.535	0.0277877186778607	0.234101003941464	0.750762400416764\\
0.535	0.0289991911930496	0.239076199769547	0.744859807805785\\
0.535	0.0302379108436654	0.244052228908366	0.73925931151808\\
0.535	0.0315039414701067	0.24902888308801	0.733960949044503\\
0.535	0.0327973449470747	0.254005952801449	0.728960082407259\\
0.535	0.0341181811608519	0.258983227325591	0.724248305111178\\
0.535	0.0354665079868145	0.263960494742775	0.719814276220682\\
0.535	0.0368423812671862	0.268937541962694	0.71564447061263\\
0.535	0.0382458547890421	0.273914154744766	0.711723839838043\\
0.535	0.0396769802625738	0.278890117720924	0.708036382389731\\
0.535	0.0411358072996216	0.283865214418836	0.704565625567163\\
0.535	0.0426223833924861	0.288839227285554	0.701295023640315\\
0.535	0.0441367538930258	0.293811937711588	0.698208278749354\\
0.535	0.0456789619920503	0.298783126055387	0.695289592055349\\
0.535	0.0472490486990192	0.303752571668251	0.692523853197511\\
0.535	0.0488470528220538	0.308720052919643	0.689896776228119\\
0.535	0.0504730109482718	0.313685347222913	0.687394989990281\\
0.535	0.0521269574244539	0.318648231061428	0.685006090465979\\
0.535	0.0538089243380495	0.323608480015096	0.682718662029128\\
0.535	0.0555189414985328	0.328565868787292	0.68052227385347\\
0.535	0.0572570364191156	0.333520171232163	0.67840745699785\\
0.535	0.0590232342988274	0.338471160382329	0.676365666960044\\
0.535	0.0608175580049697	0.34341860847696	0.674389235783156\\
0.535	0.0626400280559541	0.348362286990222	0.672471317134861\\
0.535	0.064490662604533	0.3533019666601	0.670605827172182\\
0.535	0.066369477421429	0.358237417517579	0.668787383459804\\
0.535	0.0682764858793752	0.363168408916186	0.667011243730728\\
0.535	0.0702116989375697	0.368094709561873	0.665273245863649\\
0.535	0.0721751251265572	0.373016087543259	0.663569750098932\\
0.535	0.0741667705335423	0.377932310362199	0.66189758422006\\
0.535	0.0761866387881432	0.382843144964686	0.660253992184442\\
0.535	0.0782347310485954	0.38774835777208	0.658636586490998\\
0.535	0.0803110459884098	0.392647714712651	0.657043304415871\\
0.535	0.0824155797834956	0.397540981253432	0.65547236812651\\
0.535	0.084548326099754	0.402427922432375	0.653922248592745\\
0.535	0.0867092760811502	0.407308302890799	0.65239163314657\\
0.535	0.0888984183382709	0.412181886906127	0.650879396495579\\
0.535	0.0911157389373742	0.417048438424896	0.649384574964845\\
0.535	0.0933612213899392	0.421907721096044	0.647906343724685\\
0.535	0.0956348466427212	0.42675949830445	0.646443996754804\\
0.535	0.0979365930683194	0.431603533204733	0.644996929295961\\
0.535	0.100266436456264	0.436439588755293	0.643564622546847\\
0.535	0.102624350004627	0.441267427752584	0.64214663037443\\
0.535	0.105010304312169	0.446086812865616	0.640742567819433\\
0.535	0.107424267371012	0.450897506670668	0.639352101193729\\
0.535	0.109866204559871	0.455699271686218	0.637974939582444\\
0.535	0.112336078637817	0.460491870408058	0.636610827579863\\
0.535	0.114833849738607	0.465275065344603	0.635259539104383\\
0.535	0.117359475365564	0.470048619052377	0.633920872153155\\
0.535	0.119912910387022	0.474812294171664	0.632594644371965\\
0.535	0.122494107032347	0.479565853462319	0.631280689329641\\
0.535	0.125103014888515	0.484309059839721	0.629978853399058\\
0.535	0.127739580897283	0.489041676410868	0.628690460824557\\
0.54	0	0	0.710202759607373\\
0.54	1.11327767495586e-05	0.00471862581271275	0.713003157837853\\
0.54	4.46175251031912e-05	0.009446325184051	0.715846486660142\\
0.54	0.000100583311362513	0.0141829653360114	0.718788838379794\\
0.54	0.000179158434668431	0.018928411755669	0.721886092228356\\
0.54	0.000280470402701511	0.02368252819604	0.725200636220892\\
0.54	0.000404645907256436	0.0284451766772965	0.728799085274723\\
0.54	0.000551810799695644	0.0332162174883388	0.732749123983336\\
0.54	0.000722090066287311	0.037995509188729	0.737115575253327\\
0.54	0.000915607803432999	0.0427829086109896	0.741955893907011\\
0.54	0.0011324871927904	0.0475782708632729	0.747315373652289\\
0.54	0.00137285047629673	0.0523814493324038	0.753222424045811\\
0.54	0.00163681893109844	0.057192295687301	0.759684310525652\\
0.54	0.00192451284439304	0.0620106598827802	0.766683748102665\\
0.54	0.00223605148818898	0.0668363901637434	0.774176695511256\\
0.54	0.00257155309398959	0.0716693330697584	0.782091614494494\\
0.54	0.00293113482740716	0.0765093334400312	0.79033034642109\\
0.54	0.00331491276271365	0.0813562344187764	0.79877062752665\\
0.54	0.00372300185733413	0.0862098774609879	0.807270128930359\\
0.54	0.00415551592628969	0.0910701023386145	0.815671782700193\\
0.54	0.00461256761659621	0.0959367471471425	0.82381005350733\\
0.54	0.00509426838162598	0.100809648312589	0.831517746503355\\
0.54	0.00560072845543873	0.105688640598912	0.838632911419452\\
0.54	0.00613205682708918	0.11057355711583	0.845005411376294\\
0.54	0.00668836121491815	0.115464229327074	0.850502769049301\\
0.54	0.00726974804083431	0.120360487059049	0.855014975717656\\
0.54	0.00787632240459385	0.125262158509929	0.858458041096605\\
0.54	0.00850818805808555	0.13016907025918	0.860776163494183\\
0.54	0.00916544737962849	0.135081047277508	0.861942500814223\\
0.54	0.00984820134829017	0.139997912937243	0.86195861464842\\
0.54	0.0105565495182326	0.144919489023162	0.860852735617092\\
0.54	0.0112905899930939	0.149845595743738	0.858677054109649\\
0.54	0.012050419400414	0.154776051742839	0.855504274967457\\
0.54	0.0128361328661109	0.159710674111862	0.851423687967753\\
0.54	0.0136478239890177	0.164649278402306	0.846537000484341\\
0.54	0.0144855848154856	0.169591678638796	0.840954157904249\\
0.54	0.0153495058140643	0.174537687332543	0.834789345403934\\
0.54	0.0162396758502651	0.179487115495259	0.828157325806204\\
0.54	0.0171561821614173	0.184439772653511	0.821170226448693\\
0.54	0.0180991103316243	0.189395466863524	0.813934846725693\\
0.54	0.0190685442668299	0.194354004726437	0.806550519904211\\
0.54	0.0200645661700016	0.199315191403999	0.799107529846303\\
0.54	0.0210872565164405	0.204278830634725	0.791686056496192\\
0.54	0.0221366940292259	0.20924472475049	0.784355603830602\\
0.54	0.023212955654804	0.214212674693577	0.777174850281119\\
0.54	0.0243161165387281	0.219182480034174	0.770191853862643\\
0.54	0.025446250001561	0.224153938988321	0.763444541554457\\
0.54	0.0266034275149466	0.229126848436298	0.756961413910483\\
0.54	0.0277877186778607	0.234101003941464	0.750762400416766\\
0.54	0.0289991911930496	0.239076199769547	0.744859807805786\\
0.54	0.0302379108436654	0.244052228908366	0.739259311518077\\
0.54	0.0315039414701067	0.24902888308801	0.733960949044503\\
0.54	0.0327973449470747	0.254005952801449	0.72896008240726\\
0.54	0.0341181811608519	0.258983227325591	0.724248305111178\\
0.54	0.0354665079868145	0.263960494742775	0.719814276220681\\
0.54	0.0368423812671862	0.268937541962694	0.71564447061263\\
0.54	0.0382458547890421	0.273914154744766	0.711723839838042\\
0.54	0.0396769802625738	0.278890117720924	0.708036382389731\\
0.54	0.0411358072996216	0.283865214418836	0.704565625567164\\
0.54	0.0426223833924861	0.288839227285554	0.701295023640315\\
0.54	0.0441367538930258	0.293811937711588	0.698208278749354\\
0.54	0.0456789619920503	0.298783126055387	0.695289592055349\\
0.54	0.0472490486990192	0.303752571668251	0.692523853197512\\
0.54	0.0488470528220538	0.308720052919643	0.68989677622812\\
0.54	0.0504730109482718	0.313685347222913	0.687394989990281\\
0.54	0.0521269574244539	0.318648231061428	0.68500609046598\\
0.54	0.0538089243380495	0.323608480015096	0.682718662029128\\
0.54	0.0555189414985328	0.328565868787292	0.68052227385347\\
0.54	0.0572570364191156	0.333520171232163	0.67840745699785\\
0.54	0.0590232342988274	0.338471160382329	0.676365666960044\\
0.54	0.0608175580049697	0.34341860847696	0.674389235783157\\
0.54	0.0626400280559541	0.348362286990222	0.672471317134861\\
0.54	0.064490662604533	0.3533019666601	0.670605827172182\\
0.54	0.0663694774214291	0.358237417517579	0.668787383459805\\
0.54	0.0682764858793752	0.363168408916186	0.667011243730728\\
0.54	0.0702116989375697	0.368094709561873	0.665273245863649\\
0.54	0.0721751251265572	0.373016087543259	0.663569750098932\\
0.54	0.0741667705335423	0.377932310362199	0.661897584220061\\
0.54	0.0761866387881432	0.382843144964686	0.660253992184441\\
0.54	0.0782347310485954	0.38774835777208	0.658636586490997\\
0.54	0.0803110459884098	0.392647714712651	0.657043304415873\\
0.54	0.0824155797834956	0.397540981253432	0.655472368126509\\
0.54	0.084548326099754	0.402427922432375	0.653922248592744\\
0.54	0.0867092760811502	0.407308302890799	0.652391633146571\\
0.54	0.0888984183382709	0.412181886906127	0.65087939649558\\
0.54	0.0911157389373742	0.417048438424896	0.649384574964842\\
0.54	0.0933612213899392	0.421907721096044	0.647906343724686\\
0.54	0.0956348466427212	0.42675949830445	0.646443996754806\\
0.54	0.0979365930683194	0.431603533204733	0.64499692929596\\
0.54	0.100266436456264	0.436439588755293	0.643564622546846\\
0.54	0.102624350004627	0.441267427752584	0.64214663037443\\
0.54	0.105010304312169	0.446086812865616	0.640742567819433\\
0.54	0.107424267371012	0.450897506670668	0.639352101193731\\
0.54	0.109866204559871	0.455699271686218	0.637974939582443\\
0.54	0.112336078637817	0.460491870408058	0.636610827579862\\
0.54	0.114833849738607	0.465275065344603	0.635259539104382\\
0.54	0.117359475365564	0.470048619052377	0.633920872153155\\
0.54	0.119912910387022	0.474812294171664	0.632594644371965\\
0.54	0.122494107032347	0.479565853462319	0.631280689329642\\
0.54	0.125103014888515	0.484309059839721	0.629978853399061\\
0.54	0.127739580897283	0.489041676410868	0.628690460824557\\
0.545	0	0	0.710202759607373\\
0.545	1.11327767495586e-05	0.00471862581271275	0.713003157837853\\
0.545	4.46175251031912e-05	0.009446325184051	0.715846486660142\\
0.545	0.000100583311362513	0.0141829653360114	0.718788838379794\\
0.545	0.000179158434668431	0.018928411755669	0.721886092228356\\
0.545	0.000280470402701511	0.02368252819604	0.725200636220892\\
0.545	0.000404645907256436	0.0284451766772965	0.728799085274723\\
0.545	0.000551810799695644	0.0332162174883389	0.732749123983336\\
0.545	0.000722090066287311	0.037995509188729	0.737115575253328\\
0.545	0.000915607803432999	0.0427829086109896	0.741955893907011\\
0.545	0.0011324871927904	0.0475782708632729	0.747315373652289\\
0.545	0.00137285047629673	0.0523814493324038	0.75322242404581\\
0.545	0.00163681893109844	0.057192295687301	0.759684310525652\\
0.545	0.00192451284439304	0.0620106598827802	0.766683748102665\\
0.545	0.00223605148818898	0.0668363901637434	0.774176695511257\\
0.545	0.00257155309398959	0.0716693330697584	0.782091614494494\\
0.545	0.00293113482740716	0.0765093334400312	0.79033034642109\\
0.545	0.00331491276271365	0.0813562344187764	0.798770627526651\\
0.545	0.00372300185733413	0.0862098774609879	0.807270128930359\\
0.545	0.00415551592628969	0.0910701023386145	0.815671782700193\\
0.545	0.00461256761659621	0.0959367471471425	0.823810053507331\\
0.545	0.00509426838162598	0.100809648312589	0.831517746503353\\
0.545	0.00560072845543873	0.105688640598912	0.838632911419452\\
0.545	0.00613205682708918	0.11057355711583	0.845005411376293\\
0.545	0.00668836121491815	0.115464229327074	0.850502769049302\\
0.545	0.00726974804083431	0.120360487059049	0.855014975717656\\
0.545	0.00787632240459385	0.125262158509929	0.858458041096606\\
0.545	0.00850818805808555	0.13016907025918	0.860776163494181\\
0.545	0.00916544737962849	0.135081047277508	0.861942500814224\\
0.545	0.00984820134829017	0.139997912937243	0.861958614648422\\
0.545	0.0105565495182326	0.144919489023162	0.860852735617091\\
0.545	0.0112905899930939	0.149845595743738	0.858677054109651\\
0.545	0.012050419400414	0.154776051742839	0.855504274967457\\
0.545	0.0128361328661109	0.159710674111862	0.851423687967756\\
0.545	0.0136478239890177	0.164649278402306	0.846537000484343\\
0.545	0.0144855848154856	0.169591678638796	0.840954157904249\\
0.545	0.0153495058140643	0.174537687332543	0.834789345403937\\
0.545	0.0162396758502651	0.179487115495259	0.828157325806203\\
0.545	0.0171561821614173	0.184439772653511	0.821170226448691\\
0.545	0.0180991103316243	0.189395466863524	0.813934846725692\\
0.545	0.0190685442668299	0.194354004726437	0.806550519904211\\
0.545	0.0200645661700016	0.199315191403999	0.799107529846304\\
0.545	0.0210872565164405	0.204278830634725	0.791686056496194\\
0.545	0.0221366940292259	0.20924472475049	0.784355603830604\\
0.545	0.023212955654804	0.214212674693577	0.777174850281118\\
0.545	0.0243161165387281	0.219182480034174	0.770191853862641\\
0.545	0.025446250001561	0.224153938988321	0.763444541554459\\
0.545	0.0266034275149466	0.229126848436298	0.756961413910484\\
0.545	0.0277877186778607	0.234101003941464	0.750762400416766\\
0.545	0.0289991911930496	0.239076199769547	0.744859807805783\\
0.545	0.0302379108436654	0.244052228908366	0.739259311518077\\
0.545	0.0315039414701067	0.24902888308801	0.733960949044503\\
0.545	0.0327973449470747	0.254005952801449	0.728960082407263\\
0.545	0.0341181811608519	0.258983227325592	0.724248305111178\\
0.545	0.0354665079868145	0.263960494742775	0.71981427622068\\
0.545	0.0368423812671862	0.268937541962694	0.715644470612631\\
0.545	0.0382458547890421	0.273914154744766	0.711723839838043\\
0.545	0.0396769802625738	0.278890117720924	0.708036382389731\\
0.545	0.0411358072996216	0.283865214418836	0.704565625567163\\
0.545	0.0426223833924861	0.288839227285554	0.701295023640316\\
0.545	0.0441367538930258	0.293811937711588	0.698208278749354\\
0.545	0.0456789619920503	0.298783126055387	0.695289592055348\\
0.545	0.0472490486990192	0.303752571668251	0.692523853197512\\
0.545	0.0488470528220538	0.308720052919643	0.689896776228121\\
0.545	0.0504730109482718	0.313685347222913	0.687394989990281\\
0.545	0.0521269574244539	0.318648231061428	0.685006090465979\\
0.545	0.0538089243380495	0.323608480015096	0.682718662029128\\
0.545	0.0555189414985328	0.328565868787292	0.68052227385347\\
0.545	0.0572570364191156	0.333520171232163	0.67840745699785\\
0.545	0.0590232342988274	0.338471160382329	0.676365666960043\\
0.545	0.0608175580049697	0.34341860847696	0.674389235783156\\
0.545	0.0626400280559541	0.348362286990222	0.672471317134862\\
0.545	0.064490662604533	0.3533019666601	0.670605827172182\\
0.545	0.0663694774214291	0.358237417517579	0.668787383459805\\
0.545	0.0682764858793752	0.363168408916186	0.667011243730728\\
0.545	0.0702116989375697	0.368094709561873	0.665273245863649\\
0.545	0.0721751251265572	0.373016087543259	0.663569750098933\\
0.545	0.0741667705335422	0.377932310362199	0.66189758422006\\
0.545	0.0761866387881432	0.382843144964686	0.660253992184442\\
0.545	0.0782347310485954	0.38774835777208	0.658636586490997\\
0.545	0.0803110459884098	0.392647714712651	0.657043304415872\\
0.545	0.0824155797834956	0.397540981253432	0.65547236812651\\
0.545	0.084548326099754	0.402427922432375	0.653922248592745\\
0.545	0.0867092760811502	0.407308302890799	0.652391633146569\\
0.545	0.0888984183382709	0.412181886906127	0.65087939649558\\
0.545	0.0911157389373742	0.417048438424896	0.649384574964844\\
0.545	0.0933612213899392	0.421907721096044	0.647906343724685\\
0.545	0.0956348466427212	0.42675949830445	0.646443996754807\\
0.545	0.0979365930683194	0.431603533204733	0.644996929295961\\
0.545	0.100266436456264	0.436439588755293	0.643564622546846\\
0.545	0.102624350004627	0.441267427752584	0.64214663037443\\
0.545	0.105010304312169	0.446086812865616	0.640742567819433\\
0.545	0.107424267371012	0.450897506670668	0.63935210119373\\
0.545	0.109866204559871	0.455699271686218	0.637974939582443\\
0.545	0.112336078637817	0.460491870408058	0.636610827579862\\
0.545	0.114833849738607	0.465275065344603	0.635259539104384\\
0.545	0.117359475365564	0.470048619052377	0.633920872153155\\
0.545	0.119912910387022	0.474812294171664	0.632594644371964\\
0.545	0.122494107032347	0.479565853462319	0.63128068932964\\
0.545	0.125103014888515	0.484309059839721	0.629978853399061\\
0.545	0.127739580897283	0.489041676410868	0.628690460824564\\
0.55	0	0	0.710202759607373\\
0.55	1.11327767495586e-05	0.00471862581271274	0.713003157837853\\
0.55	4.46175251031912e-05	0.009446325184051	0.715846486660142\\
0.55	0.000100583311362513	0.0141829653360114	0.718788838379794\\
0.55	0.000179158434668431	0.018928411755669	0.721886092228356\\
0.55	0.000280470402701511	0.02368252819604	0.725200636220892\\
0.55	0.000404645907256436	0.0284451766772965	0.728799085274723\\
0.55	0.000551810799695644	0.0332162174883388	0.732749123983336\\
0.55	0.000722090066287311	0.037995509188729	0.737115575253328\\
0.55	0.000915607803432999	0.0427829086109896	0.741955893907011\\
0.55	0.0011324871927904	0.0475782708632729	0.747315373652289\\
0.55	0.00137285047629673	0.0523814493324038	0.75322242404581\\
0.55	0.00163681893109844	0.057192295687301	0.759684310525652\\
0.55	0.00192451284439304	0.0620106598827802	0.766683748102665\\
0.55	0.00223605148818899	0.0668363901637434	0.774176695511256\\
0.55	0.00257155309398959	0.0716693330697584	0.782091614494493\\
0.55	0.00293113482740716	0.0765093334400313	0.79033034642109\\
0.55	0.00331491276271365	0.0813562344187764	0.798770627526651\\
0.55	0.00372300185733413	0.0862098774609879	0.807270128930359\\
0.55	0.00415551592628969	0.0910701023386145	0.815671782700193\\
0.55	0.00461256761659621	0.0959367471471425	0.82381005350733\\
0.55	0.00509426838162598	0.100809648312589	0.831517746503354\\
0.55	0.00560072845543873	0.105688640598912	0.838632911419451\\
0.55	0.00613205682708918	0.11057355711583	0.845005411376292\\
0.55	0.00668836121491816	0.115464229327074	0.850502769049302\\
0.55	0.00726974804083431	0.120360487059049	0.855014975717658\\
0.55	0.00787632240459385	0.125262158509929	0.858458041096606\\
0.55	0.00850818805808555	0.13016907025918	0.860776163494181\\
0.55	0.00916544737962849	0.135081047277508	0.861942500814225\\
0.55	0.00984820134829017	0.139997912937243	0.86195861464842\\
0.55	0.0105565495182326	0.144919489023162	0.860852735617091\\
0.55	0.0112905899930939	0.149845595743738	0.858677054109652\\
0.55	0.012050419400414	0.154776051742839	0.855504274967456\\
0.55	0.0128361328661109	0.159710674111862	0.851423687967757\\
0.55	0.0136478239890177	0.164649278402306	0.846537000484341\\
0.55	0.0144855848154856	0.169591678638796	0.840954157904247\\
0.55	0.0153495058140643	0.174537687332543	0.834789345403936\\
0.55	0.0162396758502651	0.179487115495259	0.828157325806196\\
0.55	0.0171561821614173	0.184439772653511	0.821170226448693\\
0.55	0.0180991103316243	0.189395466863524	0.813934846725693\\
0.55	0.0190685442668299	0.194354004726437	0.806550519904212\\
0.55	0.0200645661700016	0.199315191403999	0.799107529846304\\
0.55	0.0210872565164405	0.204278830634725	0.791686056496197\\
0.55	0.0221366940292259	0.20924472475049	0.784355603830603\\
0.55	0.023212955654804	0.214212674693577	0.777174850281116\\
0.55	0.0243161165387281	0.219182480034174	0.770191853862644\\
0.55	0.025446250001561	0.224153938988321	0.76344454155446\\
0.55	0.0266034275149466	0.229126848436298	0.756961413910482\\
0.55	0.0277877186778607	0.234101003941464	0.750762400416764\\
0.55	0.0289991911930496	0.239076199769547	0.744859807805783\\
0.55	0.0302379108436654	0.244052228908366	0.739259311518079\\
0.55	0.0315039414701067	0.24902888308801	0.733960949044505\\
0.55	0.0327973449470747	0.254005952801449	0.728960082407261\\
0.55	0.0341181811608519	0.258983227325591	0.724248305111176\\
0.55	0.0354665079868145	0.263960494742775	0.719814276220683\\
0.55	0.0368423812671862	0.268937541962694	0.715644470612631\\
0.55	0.0382458547890421	0.273914154744766	0.711723839838041\\
0.55	0.0396769802625738	0.278890117720924	0.70803638238973\\
0.55	0.0411358072996216	0.283865214418836	0.704565625567166\\
0.55	0.0426223833924861	0.288839227285554	0.701295023640316\\
0.55	0.0441367538930258	0.293811937711588	0.698208278749352\\
0.55	0.0456789619920503	0.298783126055387	0.695289592055351\\
0.55	0.0472490486990192	0.303752571668251	0.692523853197511\\
0.55	0.0488470528220538	0.308720052919643	0.689896776228118\\
0.55	0.0504730109482718	0.313685347222913	0.687394989990282\\
0.55	0.0521269574244539	0.318648231061428	0.685006090465981\\
0.55	0.0538089243380495	0.323608480015096	0.682718662029127\\
0.55	0.0555189414985328	0.328565868787292	0.68052227385347\\
0.55	0.0572570364191156	0.333520171232163	0.67840745699785\\
0.55	0.0590232342988274	0.338471160382329	0.676365666960043\\
0.55	0.0608175580049697	0.34341860847696	0.674389235783157\\
0.55	0.0626400280559541	0.348362286990222	0.672471317134862\\
0.55	0.064490662604533	0.3533019666601	0.670605827172181\\
0.55	0.0663694774214291	0.358237417517579	0.668787383459803\\
0.55	0.0682764858793752	0.363168408916186	0.667011243730728\\
0.55	0.0702116989375697	0.368094709561873	0.665273245863649\\
0.55	0.0721751251265572	0.373016087543259	0.663569750098932\\
0.55	0.0741667705335423	0.377932310362199	0.661897584220061\\
0.55	0.0761866387881432	0.382843144964686	0.660253992184442\\
0.55	0.0782347310485954	0.38774835777208	0.658636586490996\\
0.55	0.0803110459884098	0.392647714712651	0.657043304415871\\
0.55	0.0824155797834956	0.397540981253432	0.655472368126509\\
0.55	0.084548326099754	0.402427922432375	0.653922248592746\\
0.55	0.0867092760811502	0.4073083028908	0.65239163314657\\
0.55	0.0888984183382709	0.412181886906127	0.65087939649558\\
0.55	0.0911157389373742	0.417048438424897	0.649384574964844\\
0.55	0.0933612213899392	0.421907721096044	0.647906343724683\\
0.55	0.0956348466427212	0.42675949830445	0.646443996754806\\
0.55	0.0979365930683194	0.431603533204733	0.644996929295963\\
0.55	0.100266436456264	0.436439588755293	0.643564622546847\\
0.55	0.102624350004627	0.441267427752585	0.642146630374429\\
0.55	0.105010304312169	0.446086812865616	0.640742567819434\\
0.55	0.107424267371012	0.450897506670668	0.63935210119373\\
0.55	0.109866204559871	0.455699271686218	0.637974939582443\\
0.55	0.112336078637817	0.460491870408058	0.636610827579862\\
0.55	0.114833849738607	0.465275065344603	0.635259539104382\\
0.55	0.117359475365564	0.470048619052377	0.633920872153156\\
0.55	0.119912910387022	0.474812294171664	0.632594644371966\\
0.55	0.122494107032347	0.479565853462319	0.63128068932964\\
0.55	0.125103014888515	0.484309059839721	0.629978853399058\\
0.55	0.127739580897283	0.489041676410868	0.628690460824562\\
0.555	0	0	0.710202759607373\\
0.555	1.11327767495586e-05	0.00471862581271275	0.713003157837853\\
0.555	4.46175251031912e-05	0.009446325184051	0.715846486660142\\
0.555	0.000100583311362513	0.0141829653360114	0.718788838379794\\
0.555	0.000179158434668431	0.018928411755669	0.721886092228356\\
0.555	0.000280470402701511	0.02368252819604	0.725200636220892\\
0.555	0.000404645907256436	0.0284451766772965	0.728799085274723\\
0.555	0.000551810799695644	0.0332162174883388	0.732749123983336\\
0.555	0.000722090066287311	0.037995509188729	0.737115575253327\\
0.555	0.000915607803432999	0.0427829086109896	0.741955893907011\\
0.555	0.0011324871927904	0.0475782708632729	0.747315373652289\\
0.555	0.00137285047629673	0.0523814493324038	0.753222424045811\\
0.555	0.00163681893109844	0.057192295687301	0.759684310525652\\
0.555	0.00192451284439304	0.0620106598827802	0.766683748102665\\
0.555	0.00223605148818898	0.0668363901637434	0.774176695511256\\
0.555	0.00257155309398959	0.0716693330697584	0.782091614494493\\
0.555	0.00293113482740716	0.0765093334400312	0.79033034642109\\
0.555	0.00331491276271365	0.0813562344187764	0.79877062752665\\
0.555	0.00372300185733413	0.0862098774609879	0.807270128930359\\
0.555	0.00415551592628969	0.0910701023386145	0.815671782700194\\
0.555	0.00461256761659621	0.0959367471471425	0.823810053507329\\
0.555	0.00509426838162598	0.100809648312589	0.831517746503354\\
0.555	0.00560072845543873	0.105688640598912	0.838632911419452\\
0.555	0.00613205682708918	0.11057355711583	0.845005411376292\\
0.555	0.00668836121491816	0.115464229327074	0.850502769049301\\
0.555	0.00726974804083431	0.120360487059049	0.855014975717657\\
0.555	0.00787632240459385	0.125262158509929	0.858458041096606\\
0.555	0.00850818805808554	0.13016907025918	0.860776163494183\\
0.555	0.00916544737962849	0.135081047277508	0.861942500814223\\
0.555	0.00984820134829017	0.139997912937243	0.86195861464842\\
0.555	0.0105565495182326	0.144919489023162	0.860852735617092\\
0.555	0.0112905899930939	0.149845595743738	0.85867705410965\\
0.555	0.012050419400414	0.154776051742839	0.855504274967458\\
0.555	0.0128361328661109	0.159710674111862	0.851423687967757\\
0.555	0.0136478239890177	0.164649278402306	0.846537000484339\\
0.555	0.0144855848154856	0.169591678638796	0.840954157904248\\
0.555	0.0153495058140643	0.174537687332543	0.834789345403934\\
0.555	0.0162396758502651	0.179487115495259	0.828157325806199\\
0.555	0.0171561821614173	0.184439772653511	0.821170226448694\\
0.555	0.0180991103316243	0.189395466863524	0.813934846725692\\
0.555	0.0190685442668299	0.194354004726437	0.806550519904211\\
0.555	0.0200645661700016	0.199315191403999	0.799107529846306\\
0.555	0.0210872565164405	0.204278830634725	0.791686056496193\\
0.555	0.0221366940292259	0.20924472475049	0.7843556038306\\
0.555	0.023212955654804	0.214212674693577	0.777174850281117\\
0.555	0.0243161165387281	0.219182480034174	0.770191853862645\\
0.555	0.025446250001561	0.224153938988321	0.763444541554458\\
0.555	0.0266034275149466	0.229126848436298	0.756961413910483\\
0.555	0.0277877186778607	0.234101003941464	0.750762400416763\\
0.555	0.0289991911930496	0.239076199769547	0.744859807805785\\
0.555	0.0302379108436654	0.244052228908366	0.739259311518081\\
0.555	0.0315039414701067	0.24902888308801	0.733960949044503\\
0.555	0.0327973449470747	0.254005952801449	0.72896008240726\\
0.555	0.0341181811608519	0.258983227325592	0.72424830511118\\
0.555	0.0354665079868145	0.263960494742775	0.719814276220682\\
0.555	0.0368423812671862	0.268937541962694	0.715644470612628\\
0.555	0.0382458547890421	0.273914154744766	0.711723839838042\\
0.555	0.0396769802625737	0.278890117720924	0.708036382389731\\
0.555	0.0411358072996216	0.283865214418836	0.704565625567162\\
0.555	0.0426223833924861	0.288839227285554	0.701295023640315\\
0.555	0.0441367538930258	0.293811937711588	0.698208278749354\\
0.555	0.0456789619920503	0.298783126055387	0.69528959205535\\
0.555	0.0472490486990192	0.303752571668251	0.692523853197511\\
0.555	0.0488470528220538	0.308720052919643	0.689896776228119\\
0.555	0.0504730109482718	0.313685347222913	0.687394989990282\\
0.555	0.0521269574244539	0.318648231061428	0.68500609046598\\
0.555	0.0538089243380495	0.323608480015096	0.682718662029127\\
0.555	0.0555189414985328	0.328565868787292	0.680522273853469\\
0.555	0.0572570364191156	0.333520171232163	0.678407456997851\\
0.555	0.0590232342988274	0.338471160382329	0.676365666960043\\
0.555	0.0608175580049697	0.34341860847696	0.674389235783155\\
0.555	0.0626400280559541	0.348362286990222	0.672471317134863\\
0.555	0.064490662604533	0.3533019666601	0.670605827172182\\
0.555	0.0663694774214291	0.358237417517579	0.668787383459804\\
0.555	0.0682764858793752	0.363168408916186	0.667011243730728\\
0.555	0.0702116989375697	0.368094709561873	0.665273245863649\\
0.555	0.0721751251265572	0.373016087543259	0.663569750098932\\
0.555	0.0741667705335423	0.377932310362199	0.661897584220061\\
0.555	0.0761866387881432	0.382843144964686	0.660253992184442\\
0.555	0.0782347310485954	0.38774835777208	0.658636586490997\\
0.555	0.0803110459884098	0.392647714712651	0.657043304415871\\
0.555	0.0824155797834956	0.397540981253432	0.655472368126509\\
0.555	0.084548326099754	0.402427922432375	0.653922248592744\\
0.555	0.0867092760811502	0.407308302890799	0.652391633146571\\
0.555	0.0888984183382709	0.412181886906127	0.65087939649558\\
0.555	0.0911157389373742	0.417048438424896	0.649384574964845\\
0.555	0.0933612213899392	0.421907721096044	0.647906343724685\\
0.555	0.0956348466427212	0.42675949830445	0.646443996754805\\
0.555	0.0979365930683194	0.431603533204733	0.644996929295962\\
0.555	0.100266436456264	0.436439588755293	0.643564622546847\\
0.555	0.102624350004627	0.441267427752585	0.64214663037443\\
0.555	0.105010304312169	0.446086812865616	0.640742567819434\\
0.555	0.107424267371012	0.450897506670668	0.639352101193731\\
0.555	0.109866204559871	0.455699271686218	0.637974939582443\\
0.555	0.112336078637817	0.460491870408058	0.636610827579862\\
0.555	0.114833849738607	0.465275065344603	0.635259539104382\\
0.555	0.117359475365564	0.470048619052377	0.633920872153155\\
0.555	0.119912910387022	0.474812294171664	0.632594644371966\\
0.555	0.122494107032347	0.479565853462319	0.63128068932964\\
0.555	0.125103014888515	0.484309059839721	0.629978853399059\\
0.555	0.127739580897283	0.489041676410868	0.628690460824559\\
0.56	0	0	0.710202759607373\\
0.56	1.11327767495586e-05	0.00471862581271275	0.713003157837853\\
0.56	4.46175251031912e-05	0.009446325184051	0.715846486660142\\
0.56	0.000100583311362513	0.0141829653360114	0.718788838379794\\
0.56	0.000179158434668431	0.018928411755669	0.721886092228356\\
0.56	0.000280470402701511	0.02368252819604	0.725200636220892\\
0.56	0.000404645907256436	0.0284451766772965	0.728799085274723\\
0.56	0.000551810799695644	0.0332162174883388	0.732749123983336\\
0.56	0.000722090066287311	0.037995509188729	0.737115575253328\\
0.56	0.000915607803432999	0.0427829086109896	0.741955893907011\\
0.56	0.0011324871927904	0.047578270863273	0.747315373652289\\
0.56	0.00137285047629673	0.0523814493324038	0.753222424045811\\
0.56	0.00163681893109844	0.057192295687301	0.759684310525652\\
0.56	0.00192451284439304	0.0620106598827802	0.766683748102665\\
0.56	0.00223605148818898	0.0668363901637434	0.774176695511256\\
0.56	0.00257155309398959	0.0716693330697584	0.782091614494493\\
0.56	0.00293113482740716	0.0765093334400312	0.79033034642109\\
0.56	0.00331491276271365	0.0813562344187764	0.798770627526651\\
0.56	0.00372300185733413	0.0862098774609879	0.807270128930359\\
0.56	0.00415551592628969	0.0910701023386145	0.815671782700193\\
0.56	0.00461256761659621	0.0959367471471424	0.82381005350733\\
0.56	0.00509426838162598	0.100809648312589	0.831517746503353\\
0.56	0.00560072845543873	0.105688640598912	0.838632911419452\\
0.56	0.00613205682708918	0.11057355711583	0.845005411376293\\
0.56	0.00668836121491815	0.115464229327074	0.850502769049302\\
0.56	0.00726974804083431	0.120360487059049	0.855014975717657\\
0.56	0.00787632240459385	0.125262158509929	0.858458041096605\\
0.56	0.00850818805808555	0.13016907025918	0.860776163494183\\
0.56	0.00916544737962849	0.135081047277508	0.861942500814224\\
0.56	0.00984820134829017	0.139997912937243	0.861958614648421\\
0.56	0.0105565495182326	0.144919489023162	0.86085273561709\\
0.56	0.0112905899930939	0.149845595743738	0.858677054109652\\
0.56	0.012050419400414	0.154776051742839	0.855504274967458\\
0.56	0.0128361328661109	0.159710674111862	0.851423687967753\\
0.56	0.0136478239890177	0.164649278402306	0.84653700048434\\
0.56	0.0144855848154856	0.169591678638796	0.840954157904248\\
0.56	0.0153495058140643	0.174537687332543	0.834789345403933\\
0.56	0.0162396758502651	0.179487115495259	0.8281573258062\\
0.56	0.0171561821614173	0.184439772653511	0.821170226448694\\
0.56	0.0180991103316243	0.189395466863524	0.813934846725692\\
0.56	0.0190685442668299	0.194354004726437	0.806550519904211\\
0.56	0.0200645661700016	0.199315191403999	0.799107529846303\\
0.56	0.0210872565164405	0.204278830634725	0.791686056496193\\
0.56	0.0221366940292259	0.20924472475049	0.784355603830602\\
0.56	0.023212955654804	0.214212674693577	0.777174850281118\\
0.56	0.0243161165387281	0.219182480034174	0.770191853862643\\
0.56	0.025446250001561	0.224153938988321	0.763444541554459\\
0.56	0.0266034275149466	0.229126848436298	0.756961413910484\\
0.56	0.0277877186778607	0.234101003941464	0.750762400416763\\
0.56	0.0289991911930496	0.239076199769547	0.744859807805785\\
0.56	0.0302379108436654	0.244052228908366	0.739259311518078\\
0.56	0.0315039414701067	0.24902888308801	0.733960949044504\\
0.56	0.0327973449470747	0.254005952801449	0.728960082407263\\
0.56	0.0341181811608519	0.258983227325592	0.724248305111176\\
0.56	0.0354665079868145	0.263960494742775	0.719814276220679\\
0.56	0.0368423812671862	0.268937541962694	0.715644470612631\\
0.56	0.0382458547890421	0.273914154744766	0.711723839838043\\
0.56	0.0396769802625737	0.278890117720924	0.70803638238973\\
0.56	0.0411358072996216	0.283865214418836	0.704565625567164\\
0.56	0.0426223833924861	0.288839227285554	0.701295023640317\\
0.56	0.0441367538930258	0.293811937711588	0.698208278749354\\
0.56	0.0456789619920503	0.298783126055387	0.695289592055349\\
0.56	0.0472490486990192	0.303752571668251	0.692523853197512\\
0.56	0.0488470528220538	0.308720052919643	0.689896776228118\\
0.56	0.0504730109482718	0.313685347222913	0.687394989990282\\
0.56	0.0521269574244539	0.318648231061428	0.685006090465982\\
0.56	0.0538089243380495	0.323608480015097	0.682718662029128\\
0.56	0.0555189414985328	0.328565868787292	0.680522273853469\\
0.56	0.0572570364191156	0.333520171232163	0.678407456997849\\
0.56	0.0590232342988274	0.338471160382329	0.676365666960043\\
0.56	0.0608175580049696	0.34341860847696	0.674389235783157\\
0.56	0.0626400280559541	0.348362286990222	0.672471317134862\\
0.56	0.064490662604533	0.3533019666601	0.670605827172182\\
0.56	0.0663694774214291	0.358237417517579	0.668787383459804\\
0.56	0.0682764858793752	0.363168408916186	0.667011243730728\\
0.56	0.0702116989375697	0.368094709561873	0.66527324586365\\
0.56	0.0721751251265572	0.373016087543259	0.663569750098933\\
0.56	0.0741667705335423	0.377932310362199	0.66189758422006\\
0.56	0.0761866387881432	0.382843144964686	0.660253992184442\\
0.56	0.0782347310485954	0.38774835777208	0.658636586490997\\
0.56	0.0803110459884098	0.392647714712651	0.657043304415871\\
0.56	0.0824155797834956	0.397540981253432	0.655472368126508\\
0.56	0.084548326099754	0.402427922432375	0.653922248592744\\
0.56	0.0867092760811502	0.407308302890799	0.652391633146569\\
0.56	0.0888984183382709	0.412181886906127	0.65087939649558\\
0.56	0.0911157389373742	0.417048438424896	0.649384574964844\\
0.56	0.0933612213899392	0.421907721096044	0.647906343724686\\
0.56	0.0956348466427212	0.42675949830445	0.646443996754805\\
0.56	0.0979365930683194	0.431603533204733	0.644996929295961\\
0.56	0.100266436456264	0.436439588755293	0.643564622546848\\
0.56	0.102624350004627	0.441267427752585	0.642146630374429\\
0.56	0.105010304312169	0.446086812865616	0.640742567819432\\
0.56	0.107424267371012	0.450897506670668	0.639352101193731\\
0.56	0.109866204559871	0.455699271686218	0.637974939582444\\
0.56	0.112336078637817	0.460491870408058	0.636610827579862\\
0.56	0.114833849738607	0.465275065344603	0.635259539104382\\
0.56	0.117359475365564	0.470048619052377	0.633920872153154\\
0.56	0.119912910387022	0.474812294171664	0.632594644371966\\
0.56	0.122494107032347	0.479565853462319	0.631280689329641\\
0.56	0.125103014888515	0.484309059839721	0.629978853399057\\
0.56	0.127739580897283	0.489041676410868	0.628690460824561\\
0.565	0	0	0.710202759607373\\
0.565	1.11327767495586e-05	0.00471862581271275	0.713003157837853\\
0.565	4.46175251031912e-05	0.009446325184051	0.715846486660142\\
0.565	0.000100583311362513	0.0141829653360114	0.718788838379794\\
0.565	0.000179158434668431	0.018928411755669	0.721886092228356\\
0.565	0.000280470402701511	0.02368252819604	0.725200636220892\\
0.565	0.000404645907256436	0.0284451766772965	0.728799085274723\\
0.565	0.000551810799695644	0.0332162174883388	0.732749123983336\\
0.565	0.000722090066287311	0.037995509188729	0.737115575253327\\
0.565	0.000915607803432999	0.0427829086109896	0.741955893907011\\
0.565	0.0011324871927904	0.0475782708632729	0.747315373652289\\
0.565	0.00137285047629673	0.0523814493324038	0.753222424045811\\
0.565	0.00163681893109844	0.057192295687301	0.759684310525652\\
0.565	0.00192451284439304	0.0620106598827802	0.766683748102665\\
0.565	0.00223605148818898	0.0668363901637434	0.774176695511256\\
0.565	0.00257155309398959	0.0716693330697584	0.782091614494494\\
0.565	0.00293113482740716	0.0765093334400312	0.79033034642109\\
0.565	0.00331491276271365	0.0813562344187764	0.79877062752665\\
0.565	0.00372300185733413	0.0862098774609879	0.80727012893036\\
0.565	0.00415551592628969	0.0910701023386145	0.815671782700194\\
0.565	0.00461256761659621	0.0959367471471424	0.823810053507331\\
0.565	0.00509426838162598	0.100809648312589	0.831517746503353\\
0.565	0.00560072845543873	0.105688640598912	0.838632911419452\\
0.565	0.00613205682708918	0.11057355711583	0.845005411376293\\
0.565	0.00668836121491816	0.115464229327074	0.850502769049302\\
0.565	0.00726974804083431	0.120360487059049	0.855014975717657\\
0.565	0.00787632240459385	0.125262158509929	0.858458041096606\\
0.565	0.00850818805808555	0.13016907025918	0.860776163494182\\
0.565	0.00916544737962849	0.135081047277508	0.861942500814225\\
0.565	0.00984820134829017	0.139997912937243	0.86195861464842\\
0.565	0.0105565495182326	0.144919489023162	0.860852735617089\\
0.565	0.0112905899930939	0.149845595743738	0.858677054109649\\
0.565	0.012050419400414	0.154776051742839	0.855504274967455\\
0.565	0.0128361328661109	0.159710674111862	0.851423687967753\\
0.565	0.0136478239890177	0.164649278402306	0.84653700048434\\
0.565	0.0144855848154856	0.169591678638796	0.840954157904248\\
0.565	0.0153495058140643	0.174537687332543	0.834789345403936\\
0.565	0.0162396758502651	0.179487115495259	0.828157325806201\\
0.565	0.0171561821614173	0.184439772653511	0.821170226448695\\
0.565	0.0180991103316243	0.189395466863524	0.813934846725692\\
0.565	0.0190685442668299	0.194354004726437	0.806550519904212\\
0.565	0.0200645661700016	0.199315191403999	0.799107529846305\\
0.565	0.0210872565164405	0.204278830634725	0.791686056496197\\
0.565	0.0221366940292259	0.20924472475049	0.784355603830602\\
0.565	0.023212955654804	0.214212674693577	0.777174850281118\\
0.565	0.0243161165387281	0.219182480034174	0.770191853862644\\
0.565	0.025446250001561	0.224153938988321	0.763444541554461\\
0.565	0.0266034275149466	0.229126848436298	0.756961413910482\\
0.565	0.0277877186778607	0.234101003941464	0.750762400416763\\
0.565	0.0289991911930496	0.239076199769547	0.744859807805785\\
0.565	0.0302379108436654	0.244052228908366	0.73925931151808\\
0.565	0.0315039414701067	0.24902888308801	0.733960949044504\\
0.565	0.0327973449470747	0.254005952801449	0.728960082407259\\
0.565	0.0341181811608519	0.258983227325591	0.724248305111179\\
0.565	0.0354665079868146	0.263960494742775	0.719814276220683\\
0.565	0.0368423812671862	0.268937541962694	0.715644470612629\\
0.565	0.0382458547890421	0.273914154744766	0.711723839838041\\
0.565	0.0396769802625737	0.278890117720924	0.70803638238973\\
0.565	0.0411358072996216	0.283865214418836	0.704565625567165\\
0.565	0.0426223833924861	0.288839227285554	0.701295023640316\\
0.565	0.0441367538930258	0.293811937711588	0.698208278749353\\
0.565	0.0456789619920503	0.298783126055387	0.69528959205535\\
0.565	0.0472490486990192	0.303752571668251	0.69252385319751\\
0.565	0.0488470528220538	0.308720052919643	0.689896776228119\\
0.565	0.0504730109482718	0.313685347222913	0.687394989990282\\
0.565	0.0521269574244539	0.318648231061428	0.68500609046598\\
0.565	0.0538089243380495	0.323608480015097	0.682718662029127\\
0.565	0.0555189414985328	0.328565868787292	0.680522273853469\\
0.565	0.0572570364191156	0.333520171232163	0.678407456997851\\
0.565	0.0590232342988274	0.338471160382329	0.676365666960042\\
0.565	0.0608175580049696	0.34341860847696	0.674389235783156\\
0.565	0.0626400280559541	0.348362286990222	0.672471317134862\\
0.565	0.064490662604533	0.3533019666601	0.670605827172181\\
0.565	0.0663694774214291	0.358237417517579	0.668787383459804\\
0.565	0.0682764858793752	0.363168408916186	0.667011243730728\\
0.565	0.0702116989375697	0.368094709561873	0.665273245863648\\
0.565	0.0721751251265572	0.373016087543259	0.663569750098933\\
0.565	0.0741667705335423	0.377932310362199	0.661897584220061\\
0.565	0.0761866387881432	0.382843144964686	0.660253992184442\\
0.565	0.0782347310485954	0.38774835777208	0.658636586490997\\
0.565	0.0803110459884098	0.392647714712651	0.657043304415873\\
0.565	0.0824155797834956	0.397540981253432	0.65547236812651\\
0.565	0.084548326099754	0.402427922432375	0.653922248592745\\
0.565	0.0867092760811502	0.4073083028908	0.652391633146569\\
0.565	0.0888984183382709	0.412181886906127	0.650879396495579\\
0.565	0.0911157389373742	0.417048438424897	0.649384574964843\\
0.565	0.0933612213899392	0.421907721096044	0.647906343724684\\
0.565	0.0956348466427212	0.42675949830445	0.646443996754806\\
0.565	0.0979365930683194	0.431603533204733	0.644996929295961\\
0.565	0.100266436456264	0.436439588755293	0.643564622546846\\
0.565	0.102624350004627	0.441267427752584	0.642146630374431\\
0.565	0.105010304312169	0.446086812865616	0.640742567819433\\
0.565	0.107424267371012	0.450897506670668	0.63935210119373\\
0.565	0.109866204559871	0.455699271686218	0.637974939582443\\
0.565	0.112336078637817	0.460491870408058	0.636610827579862\\
0.565	0.114833849738607	0.465275065344603	0.635259539104381\\
0.565	0.117359475365564	0.470048619052377	0.633920872153154\\
0.565	0.119912910387023	0.474812294171664	0.632594644371965\\
0.565	0.122494107032347	0.479565853462319	0.631280689329642\\
0.565	0.125103014888515	0.484309059839721	0.629978853399059\\
0.565	0.127739580897283	0.489041676410868	0.628690460824556\\
0.57	0	0	0.710202759607373\\
0.57	1.11327767495586e-05	0.00471862581271275	0.713003157837853\\
0.57	4.46175251031912e-05	0.009446325184051	0.715846486660142\\
0.57	0.000100583311362513	0.0141829653360114	0.718788838379794\\
0.57	0.000179158434668431	0.018928411755669	0.721886092228356\\
0.57	0.000280470402701511	0.02368252819604	0.725200636220892\\
0.57	0.000404645907256436	0.0284451766772965	0.728799085274723\\
0.57	0.000551810799695644	0.0332162174883389	0.732749123983336\\
0.57	0.000722090066287311	0.037995509188729	0.737115575253328\\
0.57	0.000915607803432999	0.0427829086109896	0.741955893907011\\
0.57	0.0011324871927904	0.0475782708632729	0.747315373652289\\
0.57	0.00137285047629673	0.0523814493324038	0.753222424045811\\
0.57	0.00163681893109844	0.057192295687301	0.759684310525653\\
0.57	0.00192451284439304	0.0620106598827801	0.766683748102665\\
0.57	0.00223605148818898	0.0668363901637434	0.774176695511256\\
0.57	0.00257155309398959	0.0716693330697584	0.782091614494493\\
0.57	0.00293113482740716	0.0765093334400313	0.79033034642109\\
0.57	0.00331491276271365	0.0813562344187764	0.79877062752665\\
0.57	0.00372300185733413	0.0862098774609879	0.807270128930359\\
0.57	0.00415551592628969	0.0910701023386145	0.815671782700194\\
0.57	0.00461256761659621	0.0959367471471425	0.823810053507331\\
0.57	0.00509426838162598	0.100809648312589	0.831517746503355\\
0.57	0.00560072845543873	0.105688640598912	0.838632911419451\\
0.57	0.00613205682708918	0.11057355711583	0.845005411376293\\
0.57	0.00668836121491815	0.115464229327074	0.850502769049302\\
0.57	0.00726974804083431	0.120360487059049	0.855014975717657\\
0.57	0.00787632240459385	0.125262158509929	0.858458041096607\\
0.57	0.00850818805808555	0.13016907025918	0.860776163494183\\
0.57	0.00916544737962849	0.135081047277508	0.861942500814225\\
0.57	0.00984820134829017	0.139997912937243	0.86195861464842\\
0.57	0.0105565495182326	0.144919489023162	0.860852735617091\\
0.57	0.0112905899930939	0.149845595743738	0.858677054109649\\
0.57	0.012050419400414	0.154776051742839	0.855504274967456\\
0.57	0.0128361328661109	0.159710674111862	0.851423687967755\\
0.57	0.0136478239890177	0.164649278402306	0.846537000484338\\
0.57	0.0144855848154856	0.169591678638796	0.840954157904247\\
0.57	0.0153495058140643	0.174537687332543	0.834789345403936\\
0.57	0.0162396758502651	0.179487115495259	0.8281573258062\\
0.57	0.0171561821614173	0.184439772653511	0.821170226448695\\
0.57	0.0180991103316243	0.189395466863524	0.813934846725692\\
0.57	0.0190685442668299	0.194354004726437	0.806550519904212\\
0.57	0.0200645661700016	0.199315191403999	0.799107529846306\\
0.57	0.0210872565164405	0.204278830634725	0.791686056496194\\
0.57	0.0221366940292259	0.20924472475049	0.784355603830602\\
0.57	0.023212955654804	0.214212674693577	0.777174850281119\\
0.57	0.0243161165387281	0.219182480034174	0.770191853862644\\
0.57	0.025446250001561	0.224153938988321	0.763444541554458\\
0.57	0.0266034275149466	0.229126848436298	0.756961413910482\\
0.57	0.0277877186778607	0.234101003941464	0.750762400416764\\
0.57	0.0289991911930496	0.239076199769547	0.744859807805785\\
0.57	0.0302379108436654	0.244052228908366	0.739259311518081\\
0.57	0.0315039414701067	0.24902888308801	0.733960949044503\\
0.57	0.0327973449470747	0.254005952801449	0.728960082407261\\
0.57	0.0341181811608519	0.258983227325592	0.724248305111179\\
0.57	0.0354665079868146	0.263960494742775	0.719814276220682\\
0.57	0.0368423812671862	0.268937541962694	0.71564447061263\\
0.57	0.0382458547890421	0.273914154744766	0.711723839838041\\
0.57	0.0396769802625738	0.278890117720924	0.70803638238973\\
0.57	0.0411358072996216	0.283865214418836	0.704565625567163\\
0.57	0.0426223833924861	0.288839227285554	0.701295023640317\\
0.57	0.0441367538930258	0.293811937711588	0.698208278749354\\
0.57	0.0456789619920503	0.298783126055387	0.695289592055349\\
0.57	0.0472490486990192	0.303752571668251	0.692523853197512\\
0.57	0.0488470528220538	0.308720052919643	0.689896776228117\\
0.57	0.0504730109482718	0.313685347222913	0.687394989990282\\
0.57	0.0521269574244539	0.318648231061428	0.685006090465981\\
0.57	0.0538089243380495	0.323608480015096	0.682718662029128\\
0.57	0.0555189414985328	0.328565868787292	0.68052227385347\\
0.57	0.0572570364191156	0.333520171232163	0.678407456997851\\
0.57	0.0590232342988274	0.338471160382329	0.676365666960043\\
0.57	0.0608175580049697	0.34341860847696	0.674389235783155\\
0.57	0.0626400280559541	0.348362286990222	0.672471317134863\\
0.57	0.064490662604533	0.3533019666601	0.670605827172182\\
0.57	0.0663694774214291	0.358237417517579	0.668787383459804\\
0.57	0.0682764858793752	0.363168408916186	0.667011243730728\\
0.57	0.0702116989375697	0.368094709561873	0.66527324586365\\
0.57	0.0721751251265572	0.373016087543259	0.663569750098933\\
0.57	0.0741667705335422	0.377932310362199	0.661897584220059\\
0.57	0.0761866387881432	0.382843144964686	0.660253992184443\\
0.57	0.0782347310485954	0.38774835777208	0.658636586490997\\
0.57	0.0803110459884098	0.392647714712651	0.657043304415871\\
0.57	0.0824155797834956	0.397540981253432	0.655472368126511\\
0.57	0.084548326099754	0.402427922432375	0.653922248592747\\
0.57	0.0867092760811502	0.407308302890799	0.652391633146571\\
0.57	0.0888984183382709	0.412181886906127	0.650879396495579\\
0.57	0.0911157389373742	0.417048438424896	0.649384574964844\\
0.57	0.0933612213899392	0.421907721096044	0.647906343724684\\
0.57	0.0956348466427212	0.42675949830445	0.646443996754806\\
0.57	0.0979365930683194	0.431603533204733	0.644996929295962\\
0.57	0.100266436456264	0.436439588755293	0.643564622546847\\
0.57	0.102624350004627	0.441267427752584	0.64214663037443\\
0.57	0.105010304312169	0.446086812865616	0.640742567819434\\
0.57	0.107424267371012	0.450897506670668	0.63935210119373\\
0.57	0.109866204559871	0.455699271686218	0.637974939582443\\
0.57	0.112336078637817	0.460491870408058	0.636610827579862\\
0.57	0.114833849738607	0.465275065344603	0.635259539104382\\
0.57	0.117359475365564	0.470048619052377	0.633920872153153\\
0.57	0.119912910387023	0.474812294171664	0.632594644371965\\
0.57	0.122494107032347	0.479565853462319	0.631280689329641\\
0.57	0.125103014888515	0.484309059839721	0.629978853399059\\
0.57	0.127739580897283	0.489041676410868	0.628690460824559\\
0.575	0	0	0.710202759607373\\
0.575	1.11327767495586e-05	0.00471862581271274	0.713003157837853\\
0.575	4.46175251031912e-05	0.009446325184051	0.715846486660142\\
0.575	0.000100583311362513	0.0141829653360114	0.718788838379794\\
0.575	0.000179158434668431	0.018928411755669	0.721886092228356\\
0.575	0.000280470402701511	0.02368252819604	0.725200636220892\\
0.575	0.000404645907256436	0.0284451766772965	0.728799085274723\\
0.575	0.000551810799695644	0.0332162174883389	0.732749123983336\\
0.575	0.000722090066287311	0.037995509188729	0.737115575253328\\
0.575	0.000915607803432999	0.0427829086109896	0.741955893907011\\
0.575	0.0011324871927904	0.047578270863273	0.747315373652289\\
0.575	0.00137285047629673	0.0523814493324038	0.75322242404581\\
0.575	0.00163681893109844	0.057192295687301	0.759684310525652\\
0.575	0.00192451284439304	0.0620106598827802	0.766683748102665\\
0.575	0.00223605148818898	0.0668363901637434	0.774176695511257\\
0.575	0.00257155309398959	0.0716693330697584	0.782091614494493\\
0.575	0.00293113482740716	0.0765093334400312	0.79033034642109\\
0.575	0.00331491276271365	0.0813562344187764	0.79877062752665\\
0.575	0.00372300185733413	0.0862098774609879	0.80727012893036\\
0.575	0.00415551592628969	0.0910701023386145	0.815671782700193\\
0.575	0.00461256761659621	0.0959367471471425	0.82381005350733\\
0.575	0.00509426838162598	0.100809648312589	0.831517746503354\\
0.575	0.00560072845543873	0.105688640598912	0.838632911419452\\
0.575	0.00613205682708918	0.11057355711583	0.845005411376292\\
0.575	0.00668836121491815	0.115464229327074	0.850502769049301\\
0.575	0.00726974804083431	0.120360487059049	0.855014975717657\\
0.575	0.00787632240459385	0.125262158509929	0.858458041096606\\
0.575	0.00850818805808554	0.13016907025918	0.860776163494183\\
0.575	0.00916544737962849	0.135081047277508	0.861942500814225\\
0.575	0.00984820134829017	0.139997912937243	0.861958614648419\\
0.575	0.0105565495182326	0.144919489023162	0.860852735617091\\
0.575	0.0112905899930939	0.149845595743738	0.858677054109652\\
0.575	0.012050419400414	0.154776051742839	0.855504274967458\\
0.575	0.0128361328661109	0.159710674111862	0.851423687967753\\
0.575	0.0136478239890177	0.164649278402306	0.846537000484339\\
0.575	0.0144855848154856	0.169591678638796	0.840954157904251\\
0.575	0.0153495058140643	0.174537687332543	0.834789345403935\\
0.575	0.0162396758502651	0.179487115495259	0.8281573258062\\
0.575	0.0171561821614173	0.184439772653511	0.821170226448693\\
0.575	0.0180991103316243	0.189395466863524	0.813934846725691\\
0.575	0.0190685442668299	0.194354004726437	0.806550519904213\\
0.575	0.0200645661700016	0.199315191403999	0.799107529846305\\
0.575	0.0210872565164405	0.204278830634725	0.791686056496194\\
0.575	0.0221366940292259	0.20924472475049	0.784355603830603\\
0.575	0.023212955654804	0.214212674693577	0.777174850281118\\
0.575	0.0243161165387281	0.219182480034174	0.770191853862641\\
0.575	0.025446250001561	0.224153938988321	0.763444541554457\\
0.575	0.0266034275149466	0.229126848436298	0.756961413910483\\
0.575	0.0277877186778607	0.234101003941464	0.750762400416765\\
0.575	0.0289991911930496	0.239076199769547	0.744859807805786\\
0.575	0.0302379108436654	0.244052228908366	0.739259311518077\\
0.575	0.0315039414701067	0.24902888308801	0.733960949044504\\
0.575	0.0327973449470747	0.254005952801449	0.728960082407263\\
0.575	0.0341181811608519	0.258983227325592	0.724248305111177\\
0.575	0.0354665079868145	0.263960494742775	0.71981427622068\\
0.575	0.0368423812671862	0.268937541962694	0.71564447061263\\
0.575	0.0382458547890421	0.273914154744766	0.711723839838041\\
0.575	0.0396769802625738	0.278890117720924	0.70803638238973\\
0.575	0.0411358072996216	0.283865214418836	0.704565625567163\\
0.575	0.0426223833924861	0.288839227285554	0.701295023640316\\
0.575	0.0441367538930258	0.293811937711588	0.698208278749354\\
0.575	0.0456789619920503	0.298783126055387	0.695289592055349\\
0.575	0.0472490486990192	0.303752571668251	0.692523853197513\\
0.575	0.0488470528220538	0.308720052919643	0.689896776228119\\
0.575	0.0504730109482718	0.313685347222913	0.687394989990281\\
0.575	0.0521269574244539	0.318648231061428	0.68500609046598\\
0.575	0.0538089243380495	0.323608480015096	0.682718662029127\\
0.575	0.0555189414985328	0.328565868787292	0.680522273853469\\
0.575	0.0572570364191156	0.333520171232163	0.678407456997851\\
0.575	0.0590232342988274	0.338471160382329	0.676365666960044\\
0.575	0.0608175580049697	0.34341860847696	0.674389235783156\\
0.575	0.0626400280559541	0.348362286990222	0.672471317134862\\
0.575	0.064490662604533	0.3533019666601	0.670605827172182\\
0.575	0.0663694774214291	0.358237417517579	0.668787383459804\\
0.575	0.0682764858793752	0.363168408916186	0.667011243730727\\
0.575	0.0702116989375697	0.368094709561873	0.665273245863648\\
0.575	0.0721751251265572	0.373016087543259	0.663569750098933\\
0.575	0.0741667705335422	0.377932310362199	0.66189758422006\\
0.575	0.0761866387881432	0.382843144964686	0.660253992184441\\
0.575	0.0782347310485954	0.38774835777208	0.658636586490997\\
0.575	0.0803110459884098	0.392647714712651	0.657043304415871\\
0.575	0.0824155797834956	0.397540981253432	0.655472368126508\\
0.575	0.084548326099754	0.402427922432375	0.653922248592745\\
0.575	0.0867092760811502	0.407308302890799	0.652391633146571\\
0.575	0.0888984183382709	0.412181886906127	0.65087939649558\\
0.575	0.0911157389373742	0.417048438424897	0.649384574964845\\
0.575	0.0933612213899392	0.421907721096044	0.647906343724684\\
0.575	0.0956348466427212	0.42675949830445	0.646443996754805\\
0.575	0.0979365930683194	0.431603533204733	0.644996929295962\\
0.575	0.100266436456264	0.436439588755293	0.643564622546847\\
0.575	0.102624350004627	0.441267427752585	0.64214663037443\\
0.575	0.105010304312169	0.446086812865616	0.640742567819433\\
0.575	0.107424267371012	0.450897506670668	0.639352101193731\\
0.575	0.109866204559871	0.455699271686218	0.637974939582443\\
0.575	0.112336078637817	0.460491870408058	0.636610827579862\\
0.575	0.114833849738607	0.465275065344603	0.635259539104383\\
0.575	0.117359475365564	0.470048619052377	0.633920872153154\\
0.575	0.119912910387022	0.474812294171664	0.632594644371964\\
0.575	0.122494107032347	0.479565853462319	0.631280689329641\\
0.575	0.125103014888515	0.484309059839721	0.62997885339906\\
0.575	0.127739580897283	0.489041676410868	0.628690460824557\\
0.58	0	0	0.710202759607373\\
0.58	1.11327767495586e-05	0.00471862581271274	0.713003157837853\\
0.58	4.46175251031912e-05	0.009446325184051	0.715846486660142\\
0.58	0.000100583311362513	0.0141829653360114	0.718788838379794\\
0.58	0.000179158434668431	0.018928411755669	0.721886092228356\\
0.58	0.000280470402701511	0.02368252819604	0.725200636220892\\
0.58	0.000404645907256436	0.0284451766772965	0.728799085274723\\
0.58	0.000551810799695644	0.0332162174883388	0.732749123983336\\
0.58	0.000722090066287311	0.037995509188729	0.737115575253327\\
0.58	0.000915607803432999	0.0427829086109896	0.741955893907011\\
0.58	0.0011324871927904	0.0475782708632729	0.747315373652289\\
0.58	0.00137285047629673	0.0523814493324038	0.75322242404581\\
0.58	0.00163681893109844	0.057192295687301	0.759684310525652\\
0.58	0.00192451284439304	0.0620106598827802	0.766683748102665\\
0.58	0.00223605148818899	0.0668363901637434	0.774176695511256\\
0.58	0.00257155309398959	0.0716693330697584	0.782091614494494\\
0.58	0.00293113482740716	0.0765093334400312	0.79033034642109\\
0.58	0.00331491276271365	0.0813562344187764	0.79877062752665\\
0.58	0.00372300185733413	0.0862098774609879	0.807270128930359\\
0.58	0.00415551592628969	0.0910701023386145	0.815671782700194\\
0.58	0.00461256761659621	0.0959367471471425	0.82381005350733\\
0.58	0.00509426838162598	0.100809648312589	0.831517746503353\\
0.58	0.00560072845543873	0.105688640598912	0.838632911419452\\
0.58	0.00613205682708918	0.11057355711583	0.845005411376293\\
0.58	0.00668836121491816	0.115464229327074	0.850502769049301\\
0.58	0.00726974804083431	0.120360487059049	0.855014975717657\\
0.58	0.00787632240459385	0.125262158509929	0.858458041096606\\
0.58	0.00850818805808554	0.13016907025918	0.860776163494182\\
0.58	0.00916544737962849	0.135081047277508	0.861942500814223\\
0.58	0.00984820134829017	0.139997912937243	0.86195861464842\\
0.58	0.0105565495182326	0.144919489023162	0.86085273561709\\
0.58	0.0112905899930939	0.149845595743738	0.858677054109649\\
0.58	0.012050419400414	0.154776051742839	0.855504274967459\\
0.58	0.0128361328661109	0.159710674111862	0.851423687967753\\
0.58	0.0136478239890177	0.164649278402306	0.846537000484344\\
0.58	0.0144855848154856	0.169591678638796	0.840954157904251\\
0.58	0.0153495058140643	0.174537687332543	0.834789345403934\\
0.58	0.0162396758502651	0.179487115495259	0.8281573258062\\
0.58	0.0171561821614173	0.184439772653511	0.821170226448694\\
0.58	0.0180991103316243	0.189395466863524	0.813934846725693\\
0.58	0.0190685442668299	0.194354004726437	0.806550519904213\\
0.58	0.0200645661700016	0.199315191403999	0.799107529846305\\
0.58	0.0210872565164405	0.204278830634725	0.791686056496195\\
0.58	0.0221366940292259	0.20924472475049	0.784355603830601\\
0.58	0.023212955654804	0.214212674693577	0.777174850281116\\
0.58	0.0243161165387281	0.219182480034174	0.770191853862643\\
0.58	0.025446250001561	0.224153938988321	0.76344454155446\\
0.58	0.0266034275149466	0.229126848436298	0.756961413910483\\
0.58	0.0277877186778607	0.234101003941464	0.750762400416766\\
0.58	0.0289991911930496	0.239076199769547	0.744859807805783\\
0.58	0.0302379108436654	0.244052228908366	0.73925931151808\\
0.58	0.0315039414701067	0.24902888308801	0.733960949044505\\
0.58	0.0327973449470747	0.254005952801449	0.72896008240726\\
0.58	0.0341181811608519	0.258983227325592	0.724248305111176\\
0.58	0.0354665079868145	0.263960494742775	0.719814276220681\\
0.58	0.0368423812671862	0.268937541962694	0.715644470612629\\
0.58	0.0382458547890421	0.273914154744766	0.711723839838041\\
0.58	0.0396769802625738	0.278890117720924	0.708036382389731\\
0.58	0.0411358072996216	0.283865214418836	0.704565625567165\\
0.58	0.0426223833924861	0.288839227285554	0.701295023640317\\
0.58	0.0441367538930258	0.293811937711588	0.698208278749353\\
0.58	0.0456789619920503	0.298783126055387	0.695289592055349\\
0.58	0.0472490486990192	0.303752571668251	0.692523853197512\\
0.58	0.0488470528220538	0.308720052919643	0.689896776228118\\
0.58	0.0504730109482718	0.313685347222913	0.687394989990282\\
0.58	0.0521269574244539	0.318648231061428	0.685006090465981\\
0.58	0.0538089243380495	0.323608480015096	0.682718662029127\\
0.58	0.0555189414985328	0.328565868787292	0.680522273853468\\
0.58	0.0572570364191156	0.333520171232163	0.678407456997849\\
0.58	0.0590232342988274	0.338471160382329	0.676365666960044\\
0.58	0.0608175580049697	0.34341860847696	0.674389235783157\\
0.58	0.0626400280559541	0.348362286990222	0.672471317134862\\
0.58	0.064490662604533	0.3533019666601	0.670605827172181\\
0.58	0.0663694774214291	0.358237417517579	0.668787383459804\\
0.58	0.0682764858793752	0.363168408916186	0.667011243730727\\
0.58	0.0702116989375697	0.368094709561873	0.665273245863648\\
0.58	0.0721751251265572	0.373016087543259	0.663569750098932\\
0.58	0.0741667705335422	0.377932310362199	0.661897584220061\\
0.58	0.0761866387881432	0.382843144964686	0.660253992184443\\
0.58	0.0782347310485954	0.38774835777208	0.658636586490998\\
0.58	0.0803110459884098	0.392647714712651	0.65704330441587\\
0.58	0.0824155797834956	0.397540981253432	0.655472368126508\\
0.58	0.084548326099754	0.402427922432375	0.653922248592745\\
0.58	0.0867092760811502	0.407308302890799	0.65239163314657\\
0.58	0.0888984183382709	0.412181886906127	0.65087939649558\\
0.58	0.0911157389373742	0.417048438424897	0.649384574964844\\
0.58	0.0933612213899392	0.421907721096044	0.647906343724685\\
0.58	0.0956348466427212	0.42675949830445	0.646443996754806\\
0.58	0.0979365930683194	0.431603533204733	0.644996929295961\\
0.58	0.100266436456264	0.436439588755293	0.643564622546846\\
0.58	0.102624350004627	0.441267427752584	0.64214663037443\\
0.58	0.105010304312169	0.446086812865616	0.640742567819434\\
0.58	0.107424267371012	0.450897506670668	0.639352101193731\\
0.58	0.109866204559871	0.455699271686218	0.637974939582443\\
0.58	0.112336078637817	0.460491870408058	0.636610827579862\\
0.58	0.114833849738607	0.465275065344603	0.635259539104383\\
0.58	0.117359475365564	0.470048619052377	0.633920872153155\\
0.58	0.119912910387022	0.474812294171664	0.632594644371965\\
0.58	0.122494107032347	0.479565853462319	0.631280689329639\\
0.58	0.125103014888515	0.484309059839721	0.629978853399059\\
0.58	0.127739580897283	0.489041676410868	0.628690460824563\\
0.585	0	0	0.710202759607373\\
0.585	1.11327767495586e-05	0.00471862581271275	0.713003157837853\\
0.585	4.46175251031912e-05	0.009446325184051	0.715846486660142\\
0.585	0.000100583311362513	0.0141829653360114	0.718788838379794\\
0.585	0.000179158434668431	0.018928411755669	0.721886092228357\\
0.585	0.000280470402701511	0.02368252819604	0.725200636220892\\
0.585	0.000404645907256436	0.0284451766772965	0.728799085274723\\
0.585	0.000551810799695644	0.0332162174883389	0.732749123983336\\
0.585	0.000722090066287311	0.037995509188729	0.737115575253327\\
0.585	0.000915607803432999	0.0427829086109896	0.741955893907011\\
0.585	0.0011324871927904	0.0475782708632729	0.747315373652289\\
0.585	0.00137285047629673	0.0523814493324038	0.753222424045811\\
0.585	0.00163681893109844	0.057192295687301	0.759684310525652\\
0.585	0.00192451284439304	0.0620106598827802	0.766683748102665\\
0.585	0.00223605148818898	0.0668363901637434	0.774176695511256\\
0.585	0.00257155309398959	0.0716693330697584	0.782091614494493\\
0.585	0.00293113482740716	0.0765093334400312	0.79033034642109\\
0.585	0.00331491276271365	0.0813562344187764	0.79877062752665\\
0.585	0.00372300185733413	0.0862098774609879	0.807270128930359\\
0.585	0.00415551592628969	0.0910701023386145	0.815671782700194\\
0.585	0.00461256761659621	0.0959367471471425	0.82381005350733\\
0.585	0.00509426838162598	0.100809648312589	0.831517746503353\\
0.585	0.00560072845543873	0.105688640598912	0.838632911419452\\
0.585	0.00613205682708918	0.11057355711583	0.845005411376293\\
0.585	0.00668836121491816	0.115464229327074	0.850502769049302\\
0.585	0.00726974804083431	0.120360487059049	0.855014975717657\\
0.585	0.00787632240459385	0.125262158509929	0.858458041096605\\
0.585	0.00850818805808554	0.13016907025918	0.860776163494182\\
0.585	0.00916544737962849	0.135081047277508	0.861942500814224\\
0.585	0.00984820134829017	0.139997912937243	0.861958614648422\\
0.585	0.0105565495182326	0.144919489023162	0.860852735617092\\
0.585	0.0112905899930939	0.149845595743738	0.858677054109649\\
0.585	0.012050419400414	0.154776051742839	0.855504274967458\\
0.585	0.0128361328661109	0.159710674111862	0.851423687967755\\
0.585	0.0136478239890177	0.164649278402306	0.846537000484341\\
0.585	0.0144855848154856	0.169591678638796	0.840954157904248\\
0.585	0.0153495058140643	0.174537687332543	0.834789345403933\\
0.585	0.0162396758502651	0.179487115495259	0.828157325806201\\
0.585	0.0171561821614173	0.184439772653511	0.821170226448696\\
0.585	0.0180991103316243	0.189395466863524	0.813934846725692\\
0.585	0.0190685442668299	0.194354004726437	0.806550519904213\\
0.585	0.0200645661700016	0.199315191403999	0.799107529846306\\
0.585	0.0210872565164405	0.204278830634725	0.791686056496195\\
0.585	0.0221366940292259	0.20924472475049	0.784355603830601\\
0.585	0.023212955654804	0.214212674693577	0.777174850281119\\
0.585	0.0243161165387281	0.219182480034174	0.770191853862645\\
0.585	0.025446250001561	0.224153938988321	0.763444541554457\\
0.585	0.0266034275149466	0.229126848436298	0.756961413910484\\
0.585	0.0277877186778607	0.234101003941464	0.750762400416764\\
0.585	0.0289991911930496	0.239076199769547	0.744859807805783\\
0.585	0.0302379108436654	0.244052228908366	0.73925931151808\\
0.585	0.0315039414701067	0.24902888308801	0.733960949044503\\
0.585	0.0327973449470747	0.254005952801449	0.72896008240726\\
0.585	0.0341181811608519	0.258983227325591	0.724248305111179\\
0.585	0.0354665079868145	0.263960494742775	0.719814276220681\\
0.585	0.0368423812671862	0.268937541962694	0.715644470612629\\
0.585	0.0382458547890421	0.273914154744766	0.711723839838043\\
0.585	0.0396769802625738	0.278890117720924	0.708036382389731\\
0.585	0.0411358072996216	0.283865214418836	0.704565625567163\\
0.585	0.0426223833924861	0.288839227285554	0.701295023640315\\
0.585	0.0441367538930258	0.293811937711588	0.698208278749355\\
0.585	0.0456789619920503	0.298783126055387	0.69528959205535\\
0.585	0.0472490486990192	0.303752571668251	0.692523853197511\\
0.585	0.0488470528220538	0.308720052919643	0.689896776228119\\
0.585	0.0504730109482718	0.313685347222913	0.687394989990281\\
0.585	0.0521269574244539	0.318648231061428	0.685006090465981\\
0.585	0.0538089243380495	0.323608480015096	0.682718662029128\\
0.585	0.0555189414985328	0.328565868787292	0.680522273853469\\
0.585	0.0572570364191156	0.333520171232163	0.678407456997849\\
0.585	0.0590232342988274	0.338471160382329	0.676365666960044\\
0.585	0.0608175580049697	0.34341860847696	0.674389235783156\\
0.585	0.0626400280559541	0.348362286990222	0.672471317134863\\
0.585	0.064490662604533	0.3533019666601	0.670605827172182\\
0.585	0.0663694774214291	0.358237417517579	0.668787383459803\\
0.585	0.0682764858793752	0.363168408916186	0.667011243730729\\
0.585	0.0702116989375697	0.368094709561873	0.665273245863648\\
0.585	0.0721751251265572	0.373016087543259	0.663569750098932\\
0.585	0.0741667705335422	0.377932310362199	0.66189758422006\\
0.585	0.0761866387881432	0.382843144964686	0.660253992184442\\
0.585	0.0782347310485954	0.38774835777208	0.658636586490998\\
0.585	0.0803110459884098	0.392647714712651	0.657043304415871\\
0.585	0.0824155797834956	0.397540981253432	0.655472368126508\\
0.585	0.084548326099754	0.402427922432375	0.653922248592745\\
0.585	0.0867092760811502	0.407308302890799	0.65239163314657\\
0.585	0.0888984183382709	0.412181886906127	0.650879396495579\\
0.585	0.0911157389373742	0.417048438424896	0.649384574964844\\
0.585	0.0933612213899392	0.421907721096044	0.647906343724685\\
0.585	0.0956348466427212	0.42675949830445	0.646443996754807\\
0.585	0.0979365930683194	0.431603533204733	0.644996929295963\\
0.585	0.100266436456264	0.436439588755293	0.643564622546846\\
0.585	0.102624350004627	0.441267427752584	0.642146630374429\\
0.585	0.105010304312169	0.446086812865616	0.640742567819433\\
0.585	0.107424267371012	0.450897506670668	0.63935210119373\\
0.585	0.109866204559871	0.455699271686218	0.637974939582443\\
0.585	0.112336078637817	0.460491870408058	0.636610827579862\\
0.585	0.114833849738607	0.465275065344603	0.635259539104383\\
0.585	0.117359475365564	0.470048619052377	0.633920872153155\\
0.585	0.119912910387023	0.474812294171664	0.632594644371966\\
0.585	0.122494107032347	0.479565853462319	0.631280689329641\\
0.585	0.125103014888515	0.484309059839721	0.629978853399058\\
0.585	0.127739580897283	0.489041676410868	0.628690460824559\\
0.59	0	0	0.710202759607373\\
0.59	1.11327767495586e-05	0.00471862581271274	0.713003157837853\\
0.59	4.46175251031912e-05	0.009446325184051	0.715846486660142\\
0.59	0.000100583311362513	0.0141829653360114	0.718788838379794\\
0.59	0.000179158434668431	0.018928411755669	0.721886092228356\\
0.59	0.000280470402701511	0.02368252819604	0.725200636220892\\
0.59	0.000404645907256436	0.0284451766772965	0.728799085274723\\
0.59	0.000551810799695644	0.0332162174883388	0.732749123983336\\
0.59	0.000722090066287311	0.037995509188729	0.737115575253327\\
0.59	0.000915607803432999	0.0427829086109896	0.741955893907011\\
0.59	0.0011324871927904	0.047578270863273	0.747315373652289\\
0.59	0.00137285047629673	0.0523814493324038	0.753222424045811\\
0.59	0.00163681893109844	0.057192295687301	0.759684310525652\\
0.59	0.00192451284439304	0.0620106598827802	0.766683748102665\\
0.59	0.00223605148818898	0.0668363901637434	0.774176695511257\\
0.59	0.00257155309398959	0.0716693330697584	0.782091614494493\\
0.59	0.00293113482740716	0.0765093334400312	0.79033034642109\\
0.59	0.00331491276271365	0.0813562344187764	0.798770627526651\\
0.59	0.00372300185733414	0.0862098774609879	0.807270128930359\\
0.59	0.00415551592628969	0.0910701023386145	0.815671782700193\\
0.59	0.00461256761659621	0.0959367471471424	0.823810053507331\\
0.59	0.00509426838162598	0.100809648312589	0.831517746503354\\
0.59	0.00560072845543873	0.105688640598912	0.838632911419451\\
0.59	0.00613205682708918	0.11057355711583	0.845005411376293\\
0.59	0.00668836121491816	0.115464229327074	0.850502769049302\\
0.59	0.00726974804083431	0.120360487059049	0.855014975717657\\
0.59	0.00787632240459385	0.125262158509929	0.858458041096607\\
0.59	0.00850818805808555	0.13016907025918	0.860776163494182\\
0.59	0.00916544737962849	0.135081047277508	0.861942500814225\\
0.59	0.00984820134829017	0.139997912937243	0.861958614648421\\
0.59	0.0105565495182326	0.144919489023162	0.860852735617092\\
0.59	0.0112905899930939	0.149845595743738	0.858677054109651\\
0.59	0.012050419400414	0.154776051742839	0.855504274967458\\
0.59	0.0128361328661109	0.159710674111862	0.851423687967755\\
0.59	0.0136478239890177	0.164649278402306	0.84653700048434\\
0.59	0.0144855848154856	0.169591678638796	0.840954157904246\\
0.59	0.0153495058140643	0.174537687332543	0.834789345403935\\
0.59	0.0162396758502651	0.179487115495259	0.828157325806202\\
0.59	0.0171561821614173	0.184439772653511	0.821170226448694\\
0.59	0.0180991103316243	0.189395466863524	0.813934846725692\\
0.59	0.0190685442668299	0.194354004726437	0.806550519904214\\
0.59	0.0200645661700016	0.199315191403999	0.799107529846306\\
0.59	0.0210872565164405	0.204278830634725	0.791686056496193\\
0.59	0.0221366940292259	0.20924472475049	0.784355603830601\\
0.59	0.023212955654804	0.214212674693577	0.777174850281118\\
0.59	0.0243161165387281	0.219182480034174	0.770191853862642\\
0.59	0.025446250001561	0.224153938988321	0.763444541554458\\
0.59	0.0266034275149466	0.229126848436298	0.756961413910485\\
0.59	0.0277877186778607	0.234101003941464	0.750762400416763\\
0.59	0.0289991911930496	0.239076199769547	0.744859807805786\\
0.59	0.0302379108436654	0.244052228908366	0.739259311518079\\
0.59	0.0315039414701067	0.24902888308801	0.733960949044504\\
0.59	0.0327973449470747	0.254005952801449	0.728960082407262\\
0.59	0.0341181811608519	0.258983227325592	0.724248305111177\\
0.59	0.0354665079868145	0.263960494742775	0.719814276220681\\
0.59	0.0368423812671862	0.268937541962694	0.715644470612631\\
0.59	0.0382458547890421	0.273914154744766	0.711723839838042\\
0.59	0.0396769802625738	0.278890117720924	0.708036382389731\\
0.59	0.0411358072996216	0.283865214418836	0.704565625567165\\
0.59	0.0426223833924861	0.288839227285554	0.701295023640316\\
0.59	0.0441367538930258	0.293811937711588	0.698208278749353\\
0.59	0.0456789619920503	0.298783126055387	0.69528959205535\\
0.59	0.0472490486990192	0.303752571668251	0.69252385319751\\
0.59	0.0488470528220538	0.308720052919643	0.689896776228118\\
0.59	0.0504730109482718	0.313685347222913	0.687394989990282\\
0.59	0.0521269574244539	0.318648231061427	0.68500609046598\\
0.59	0.0538089243380495	0.323608480015096	0.682718662029128\\
0.59	0.0555189414985328	0.328565868787292	0.68052227385347\\
0.59	0.0572570364191156	0.333520171232163	0.67840745699785\\
0.59	0.0590232342988274	0.338471160382329	0.676365666960043\\
0.59	0.0608175580049697	0.34341860847696	0.674389235783156\\
0.59	0.0626400280559541	0.348362286990222	0.672471317134862\\
0.59	0.064490662604533	0.3533019666601	0.670605827172183\\
0.59	0.0663694774214291	0.358237417517579	0.668787383459804\\
0.59	0.0682764858793752	0.363168408916186	0.667011243730728\\
0.59	0.0702116989375697	0.368094709561873	0.665273245863649\\
0.59	0.0721751251265572	0.373016087543259	0.663569750098933\\
0.59	0.0741667705335422	0.377932310362199	0.661897584220061\\
0.59	0.0761866387881432	0.382843144964686	0.660253992184442\\
0.59	0.0782347310485954	0.38774835777208	0.658636586490997\\
0.59	0.0803110459884098	0.392647714712651	0.657043304415872\\
0.59	0.0824155797834956	0.397540981253432	0.655472368126508\\
0.59	0.084548326099754	0.402427922432375	0.653922248592744\\
0.59	0.0867092760811502	0.407308302890799	0.65239163314657\\
0.59	0.0888984183382709	0.412181886906127	0.650879396495579\\
0.59	0.0911157389373742	0.417048438424896	0.649384574964843\\
0.59	0.0933612213899392	0.421907721096044	0.647906343724684\\
0.59	0.0956348466427212	0.42675949830445	0.646443996754805\\
0.59	0.0979365930683194	0.431603533204733	0.644996929295962\\
0.59	0.100266436456264	0.436439588755293	0.643564622546847\\
0.59	0.102624350004627	0.441267427752584	0.642146630374431\\
0.59	0.105010304312169	0.446086812865616	0.640742567819434\\
0.59	0.107424267371012	0.450897506670668	0.63935210119373\\
0.59	0.109866204559871	0.455699271686218	0.637974939582443\\
0.59	0.112336078637817	0.460491870408058	0.636610827579861\\
0.59	0.114833849738607	0.465275065344603	0.635259539104382\\
0.59	0.117359475365564	0.470048619052377	0.633920872153154\\
0.59	0.119912910387022	0.474812294171664	0.632594644371965\\
0.59	0.122494107032347	0.479565853462319	0.631280689329641\\
0.59	0.125103014888515	0.484309059839721	0.62997885339906\\
0.59	0.127739580897283	0.489041676410868	0.628690460824563\\
0.595	0	0	0.710202759607373\\
0.595	1.11327767495586e-05	0.00471862581271275	0.713003157837853\\
0.595	4.46175251031912e-05	0.009446325184051	0.715846486660142\\
0.595	0.000100583311362513	0.0141829653360114	0.718788838379794\\
0.595	0.000179158434668431	0.018928411755669	0.721886092228356\\
0.595	0.000280470402701511	0.02368252819604	0.725200636220892\\
0.595	0.000404645907256436	0.0284451766772965	0.728799085274723\\
0.595	0.000551810799695644	0.0332162174883388	0.732749123983336\\
0.595	0.000722090066287311	0.037995509188729	0.737115575253327\\
0.595	0.000915607803432999	0.0427829086109896	0.741955893907011\\
0.595	0.0011324871927904	0.0475782708632729	0.747315373652289\\
0.595	0.00137285047629673	0.0523814493324038	0.753222424045811\\
0.595	0.00163681893109844	0.057192295687301	0.759684310525652\\
0.595	0.00192451284439304	0.0620106598827802	0.766683748102665\\
0.595	0.00223605148818898	0.0668363901637434	0.774176695511256\\
0.595	0.00257155309398959	0.0716693330697584	0.782091614494494\\
0.595	0.00293113482740716	0.0765093334400312	0.79033034642109\\
0.595	0.00331491276271365	0.0813562344187764	0.79877062752665\\
0.595	0.00372300185733413	0.0862098774609879	0.80727012893036\\
0.595	0.00415551592628969	0.0910701023386145	0.815671782700193\\
0.595	0.00461256761659621	0.0959367471471425	0.823810053507331\\
0.595	0.00509426838162598	0.100809648312589	0.831517746503354\\
0.595	0.00560072845543873	0.105688640598912	0.838632911419452\\
0.595	0.00613205682708918	0.11057355711583	0.845005411376292\\
0.595	0.00668836121491816	0.115464229327074	0.850502769049302\\
0.595	0.00726974804083431	0.120360487059049	0.855014975717656\\
0.595	0.00787632240459385	0.125262158509929	0.858458041096608\\
0.595	0.00850818805808555	0.13016907025918	0.860776163494181\\
0.595	0.00916544737962849	0.135081047277508	0.861942500814226\\
0.595	0.00984820134829017	0.139997912937243	0.861958614648419\\
0.595	0.0105565495182326	0.144919489023162	0.860852735617091\\
0.595	0.0112905899930939	0.149845595743738	0.858677054109651\\
0.595	0.012050419400414	0.154776051742839	0.855504274967457\\
0.595	0.0128361328661109	0.159710674111862	0.851423687967756\\
0.595	0.0136478239890177	0.164649278402306	0.846537000484339\\
0.595	0.0144855848154856	0.169591678638796	0.840954157904247\\
0.595	0.0153495058140643	0.174537687332543	0.834789345403936\\
0.595	0.0162396758502651	0.179487115495259	0.8281573258062\\
0.595	0.0171561821614173	0.184439772653511	0.821170226448694\\
0.595	0.0180991103316243	0.189395466863524	0.813934846725692\\
0.595	0.0190685442668299	0.194354004726437	0.806550519904212\\
0.595	0.0200645661700016	0.199315191403999	0.799107529846304\\
0.595	0.0210872565164405	0.204278830634725	0.791686056496192\\
0.595	0.0221366940292259	0.20924472475049	0.784355603830602\\
0.595	0.023212955654804	0.214212674693577	0.777174850281117\\
0.595	0.0243161165387281	0.219182480034174	0.770191853862642\\
0.595	0.025446250001561	0.224153938988321	0.763444541554461\\
0.595	0.0266034275149466	0.229126848436298	0.756961413910484\\
0.595	0.0277877186778607	0.234101003941464	0.750762400416765\\
0.595	0.0289991911930496	0.239076199769547	0.744859807805785\\
0.595	0.0302379108436654	0.244052228908366	0.739259311518078\\
0.595	0.0315039414701067	0.24902888308801	0.733960949044504\\
0.595	0.0327973449470747	0.254005952801449	0.728960082407259\\
0.595	0.0341181811608519	0.258983227325591	0.724248305111178\\
0.595	0.0354665079868145	0.263960494742775	0.719814276220683\\
0.595	0.0368423812671862	0.268937541962694	0.715644470612629\\
0.595	0.0382458547890421	0.273914154744766	0.711723839838042\\
0.595	0.0396769802625738	0.278890117720924	0.708036382389729\\
0.595	0.0411358072996216	0.283865214418836	0.704565625567163\\
0.595	0.0426223833924861	0.288839227285554	0.701295023640317\\
0.595	0.0441367538930258	0.293811937711588	0.698208278749353\\
0.595	0.0456789619920503	0.298783126055387	0.695289592055349\\
0.595	0.0472490486990192	0.303752571668251	0.692523853197512\\
0.595	0.0488470528220537	0.308720052919643	0.689896776228118\\
0.595	0.0504730109482718	0.313685347222913	0.687394989990283\\
0.595	0.0521269574244539	0.318648231061428	0.68500609046598\\
0.595	0.0538089243380495	0.323608480015096	0.682718662029127\\
0.595	0.0555189414985328	0.328565868787292	0.680522273853471\\
0.595	0.0572570364191156	0.333520171232163	0.67840745699785\\
0.595	0.0590232342988274	0.338471160382329	0.676365666960042\\
0.595	0.0608175580049697	0.34341860847696	0.674389235783156\\
0.595	0.0626400280559541	0.348362286990222	0.672471317134863\\
0.595	0.064490662604533	0.3533019666601	0.670605827172181\\
0.595	0.0663694774214291	0.358237417517579	0.668787383459804\\
0.595	0.0682764858793752	0.363168408916186	0.667011243730729\\
0.595	0.0702116989375697	0.368094709561873	0.665273245863648\\
0.595	0.0721751251265572	0.373016087543259	0.663569750098932\\
0.595	0.0741667705335422	0.377932310362199	0.661897584220061\\
0.595	0.0761866387881432	0.382843144964686	0.660253992184442\\
0.595	0.0782347310485954	0.38774835777208	0.658636586490998\\
0.595	0.0803110459884098	0.392647714712651	0.657043304415872\\
0.595	0.0824155797834956	0.397540981253432	0.65547236812651\\
0.595	0.084548326099754	0.402427922432375	0.653922248592746\\
0.595	0.0867092760811502	0.4073083028908	0.65239163314657\\
0.595	0.0888984183382709	0.412181886906127	0.650879396495579\\
0.595	0.0911157389373742	0.417048438424897	0.649384574964844\\
0.595	0.0933612213899392	0.421907721096044	0.647906343724684\\
0.595	0.0956348466427212	0.42675949830445	0.646443996754805\\
0.595	0.0979365930683194	0.431603533204733	0.644996929295961\\
0.595	0.100266436456264	0.436439588755293	0.643564622546846\\
0.595	0.102624350004627	0.441267427752584	0.642146630374429\\
0.595	0.105010304312169	0.446086812865616	0.640742567819434\\
0.595	0.107424267371012	0.450897506670668	0.63935210119373\\
0.595	0.109866204559871	0.455699271686218	0.637974939582442\\
0.595	0.112336078637817	0.460491870408058	0.636610827579862\\
0.595	0.114833849738607	0.465275065344603	0.635259539104382\\
0.595	0.117359475365564	0.470048619052377	0.633920872153154\\
0.595	0.119912910387022	0.474812294171664	0.632594644371965\\
0.595	0.122494107032347	0.479565853462319	0.631280689329641\\
0.595	0.125103014888515	0.484309059839721	0.629978853399058\\
0.595	0.127739580897283	0.489041676410868	0.628690460824561\\
0.6	0	0	0.710202759607373\\
0.6	1.11327767495586e-05	0.00471862581271275	0.713003157837853\\
0.6	4.46175251031912e-05	0.009446325184051	0.715846486660142\\
0.6	0.000100583311362513	0.0141829653360114	0.718788838379794\\
0.6	0.000179158434668431	0.018928411755669	0.721886092228356\\
0.6	0.000280470402701511	0.02368252819604	0.725200636220892\\
0.6	0.000404645907256436	0.0284451766772965	0.728799085274723\\
0.6	0.000551810799695644	0.0332162174883389	0.732749123983336\\
0.6	0.000722090066287311	0.037995509188729	0.737115575253328\\
0.6	0.000915607803432999	0.0427829086109896	0.741955893907011\\
0.6	0.0011324871927904	0.047578270863273	0.747315373652289\\
0.6	0.00137285047629673	0.0523814493324038	0.75322242404581\\
0.6	0.00163681893109844	0.057192295687301	0.759684310525652\\
0.6	0.00192451284439304	0.0620106598827802	0.766683748102665\\
0.6	0.00223605148818898	0.0668363901637434	0.774176695511256\\
0.6	0.00257155309398959	0.0716693330697584	0.782091614494493\\
0.6	0.00293113482740716	0.0765093334400313	0.79033034642109\\
0.6	0.00331491276271365	0.0813562344187764	0.79877062752665\\
0.6	0.00372300185733413	0.0862098774609879	0.80727012893036\\
0.6	0.00415551592628969	0.0910701023386145	0.815671782700193\\
0.6	0.00461256761659621	0.0959367471471425	0.82381005350733\\
0.6	0.00509426838162598	0.100809648312589	0.831517746503354\\
0.6	0.00560072845543873	0.105688640598912	0.838632911419453\\
0.6	0.00613205682708918	0.11057355711583	0.845005411376292\\
0.6	0.00668836121491816	0.115464229327074	0.850502769049301\\
0.6	0.00726974804083431	0.120360487059049	0.855014975717657\\
0.6	0.00787632240459385	0.125262158509929	0.858458041096605\\
0.6	0.00850818805808555	0.13016907025918	0.860776163494182\\
0.6	0.00916544737962849	0.135081047277508	0.861942500814225\\
0.6	0.00984820134829017	0.139997912937243	0.861958614648418\\
0.6	0.0105565495182326	0.144919489023162	0.860852735617091\\
0.6	0.0112905899930939	0.149845595743738	0.858677054109652\\
0.6	0.012050419400414	0.154776051742839	0.855504274967456\\
0.6	0.0128361328661109	0.159710674111862	0.851423687967755\\
0.6	0.0136478239890177	0.164649278402306	0.846537000484337\\
0.6	0.0144855848154856	0.169591678638796	0.840954157904249\\
0.6	0.0153495058140643	0.174537687332543	0.834789345403935\\
0.6	0.0162396758502651	0.179487115495259	0.828157325806199\\
0.6	0.0171561821614173	0.184439772653511	0.821170226448694\\
0.6	0.0180991103316243	0.189395466863524	0.813934846725692\\
0.6	0.0190685442668299	0.194354004726437	0.806550519904212\\
0.6	0.0200645661700016	0.199315191403999	0.799107529846303\\
0.6	0.0210872565164405	0.204278830634725	0.791686056496194\\
0.6	0.0221366940292259	0.20924472475049	0.784355603830602\\
0.6	0.023212955654804	0.214212674693577	0.777174850281117\\
0.6	0.0243161165387281	0.219182480034174	0.770191853862645\\
0.6	0.025446250001561	0.224153938988321	0.76344454155446\\
0.6	0.0266034275149466	0.229126848436298	0.756961413910481\\
0.6	0.0277877186778607	0.234101003941464	0.750762400416763\\
0.6	0.0289991911930496	0.239076199769547	0.744859807805784\\
0.6	0.0302379108436654	0.244052228908366	0.739259311518081\\
0.6	0.0315039414701067	0.24902888308801	0.733960949044503\\
0.6	0.0327973449470747	0.254005952801449	0.728960082407262\\
0.6	0.0341181811608519	0.258983227325592	0.724248305111177\\
0.6	0.0354665079868145	0.263960494742775	0.71981427622068\\
0.6	0.0368423812671862	0.268937541962694	0.715644470612631\\
0.6	0.0382458547890421	0.273914154744766	0.711723839838041\\
0.6	0.0396769802625738	0.278890117720924	0.70803638238973\\
0.6	0.0411358072996216	0.283865214418836	0.704565625567166\\
0.6	0.0426223833924861	0.288839227285554	0.701295023640315\\
0.6	0.0441367538930258	0.293811937711588	0.698208278749352\\
0.6	0.0456789619920503	0.298783126055387	0.69528959205535\\
0.6	0.0472490486990192	0.303752571668251	0.692523853197511\\
0.6	0.0488470528220538	0.308720052919643	0.689896776228118\\
0.6	0.0504730109482718	0.313685347222913	0.687394989990282\\
0.6	0.0521269574244539	0.318648231061428	0.685006090465981\\
0.6	0.0538089243380495	0.323608480015096	0.682718662029127\\
0.6	0.0555189414985328	0.328565868787292	0.680522273853469\\
0.6	0.0572570364191156	0.333520171232163	0.678407456997851\\
0.6	0.0590232342988274	0.338471160382329	0.676365666960045\\
0.6	0.0608175580049697	0.34341860847696	0.674389235783155\\
0.6	0.0626400280559541	0.348362286990222	0.672471317134861\\
0.6	0.064490662604533	0.3533019666601	0.670605827172183\\
0.6	0.066369477421429	0.358237417517579	0.668787383459803\\
0.6	0.0682764858793752	0.363168408916186	0.667011243730729\\
0.6	0.0702116989375697	0.368094709561873	0.66527324586365\\
0.6	0.0721751251265572	0.373016087543259	0.663569750098931\\
0.6	0.0741667705335422	0.377932310362199	0.661897584220061\\
0.6	0.0761866387881432	0.382843144964686	0.660253992184443\\
0.6	0.0782347310485954	0.38774835777208	0.658636586490996\\
0.6	0.0803110459884098	0.392647714712651	0.657043304415871\\
0.6	0.0824155797834956	0.397540981253432	0.65547236812651\\
0.6	0.084548326099754	0.402427922432375	0.653922248592746\\
0.6	0.0867092760811502	0.4073083028908	0.65239163314657\\
0.6	0.0888984183382709	0.412181886906127	0.650879396495579\\
0.6	0.0911157389373742	0.417048438424896	0.649384574964845\\
0.6	0.0933612213899392	0.421907721096044	0.647906343724684\\
0.6	0.0956348466427212	0.42675949830445	0.646443996754805\\
0.6	0.0979365930683194	0.431603533204733	0.644996929295962\\
0.6	0.100266436456264	0.436439588755293	0.643564622546846\\
0.6	0.102624350004627	0.441267427752584	0.642146630374429\\
0.6	0.105010304312169	0.446086812865616	0.640742567819433\\
0.6	0.107424267371012	0.450897506670668	0.639352101193731\\
0.6	0.109866204559871	0.455699271686218	0.637974939582443\\
0.6	0.112336078637817	0.460491870408058	0.636610827579862\\
0.6	0.114833849738607	0.465275065344603	0.635259539104383\\
0.6	0.117359475365564	0.470048619052377	0.633920872153155\\
0.6	0.119912910387022	0.474812294171664	0.632594644371964\\
0.6	0.122494107032347	0.479565853462319	0.631280689329641\\
0.6	0.125103014888515	0.484309059839721	0.629978853399058\\
0.6	0.127739580897283	0.489041676410868	0.628690460824559\\
0.605	0	0	0.710202759607373\\
0.605	1.11327767495586e-05	0.00471862581271275	0.713003157837853\\
0.605	4.46175251031912e-05	0.009446325184051	0.715846486660142\\
0.605	0.000100583311362513	0.0141829653360114	0.718788838379794\\
0.605	0.000179158434668431	0.018928411755669	0.721886092228356\\
0.605	0.000280470402701511	0.02368252819604	0.725200636220892\\
0.605	0.000404645907256436	0.0284451766772965	0.728799085274723\\
0.605	0.000551810799695644	0.0332162174883389	0.732749123983336\\
0.605	0.000722090066287311	0.037995509188729	0.737115575253328\\
0.605	0.000915607803432999	0.0427829086109896	0.741955893907011\\
0.605	0.0011324871927904	0.0475782708632729	0.747315373652289\\
0.605	0.00137285047629673	0.0523814493324038	0.753222424045811\\
0.605	0.00163681893109844	0.057192295687301	0.759684310525652\\
0.605	0.00192451284439304	0.0620106598827802	0.766683748102665\\
0.605	0.00223605148818899	0.0668363901637434	0.774176695511256\\
0.605	0.00257155309398959	0.0716693330697584	0.782091614494494\\
0.605	0.00293113482740716	0.0765093334400312	0.79033034642109\\
0.605	0.00331491276271365	0.0813562344187764	0.798770627526651\\
0.605	0.00372300185733414	0.0862098774609879	0.807270128930359\\
0.605	0.00415551592628969	0.0910701023386145	0.815671782700194\\
0.605	0.00461256761659621	0.0959367471471425	0.82381005350733\\
0.605	0.00509426838162598	0.100809648312589	0.831517746503354\\
0.605	0.00560072845543873	0.105688640598912	0.838632911419452\\
0.605	0.00613205682708918	0.11057355711583	0.845005411376294\\
0.605	0.00668836121491816	0.115464229327074	0.850502769049301\\
0.605	0.00726974804083431	0.120360487059049	0.855014975717658\\
0.605	0.00787632240459385	0.125262158509929	0.858458041096604\\
0.605	0.00850818805808555	0.13016907025918	0.860776163494181\\
0.605	0.00916544737962849	0.135081047277508	0.861942500814223\\
0.605	0.00984820134829017	0.139997912937243	0.86195861464842\\
0.605	0.0105565495182326	0.144919489023162	0.860852735617091\\
0.605	0.0112905899930939	0.149845595743738	0.858677054109652\\
0.605	0.012050419400414	0.154776051742839	0.855504274967458\\
0.605	0.0128361328661109	0.159710674111862	0.851423687967754\\
0.605	0.0136478239890177	0.164649278402306	0.84653700048434\\
0.605	0.0144855848154856	0.169591678638796	0.84095415790425\\
0.605	0.0153495058140643	0.174537687332543	0.834789345403934\\
0.605	0.0162396758502651	0.179487115495259	0.8281573258062\\
0.605	0.0171561821614173	0.184439772653511	0.821170226448695\\
0.605	0.0180991103316243	0.189395466863524	0.813934846725692\\
0.605	0.0190685442668299	0.194354004726437	0.806550519904212\\
0.605	0.0200645661700016	0.199315191403999	0.799107529846306\\
0.605	0.0210872565164405	0.204278830634725	0.791686056496196\\
0.605	0.0221366940292259	0.20924472475049	0.7843556038306\\
0.605	0.023212955654804	0.214212674693577	0.777174850281118\\
0.605	0.0243161165387281	0.219182480034174	0.770191853862643\\
0.605	0.025446250001561	0.224153938988321	0.763444541554457\\
0.605	0.0266034275149466	0.229126848436298	0.756961413910482\\
0.605	0.0277877186778607	0.234101003941464	0.750762400416763\\
0.605	0.0289991911930496	0.239076199769547	0.744859807805787\\
0.605	0.0302379108436654	0.244052228908366	0.739259311518079\\
0.605	0.0315039414701067	0.24902888308801	0.733960949044502\\
0.605	0.0327973449470747	0.254005952801449	0.728960082407261\\
0.605	0.0341181811608519	0.258983227325592	0.724248305111178\\
0.605	0.0354665079868145	0.263960494742775	0.719814276220682\\
0.605	0.0368423812671862	0.268937541962694	0.715644470612629\\
0.605	0.0382458547890421	0.273914154744766	0.711723839838041\\
0.605	0.0396769802625738	0.278890117720924	0.70803638238973\\
0.605	0.0411358072996216	0.283865214418836	0.704565625567164\\
0.605	0.0426223833924861	0.288839227285554	0.701295023640317\\
0.605	0.0441367538930258	0.293811937711588	0.698208278749355\\
0.605	0.0456789619920503	0.298783126055387	0.695289592055347\\
0.605	0.0472490486990191	0.303752571668251	0.692523853197512\\
0.605	0.0488470528220538	0.308720052919643	0.689896776228121\\
0.605	0.0504730109482718	0.313685347222913	0.687394989990281\\
0.605	0.0521269574244539	0.318648231061428	0.68500609046598\\
0.605	0.0538089243380495	0.323608480015097	0.682718662029128\\
0.605	0.0555189414985328	0.328565868787292	0.680522273853469\\
0.605	0.0572570364191156	0.333520171232163	0.678407456997849\\
0.605	0.0590232342988274	0.338471160382329	0.676365666960045\\
0.605	0.0608175580049697	0.34341860847696	0.674389235783157\\
0.605	0.0626400280559541	0.348362286990222	0.67247131713486\\
0.605	0.064490662604533	0.3533019666601	0.670605827172182\\
0.605	0.0663694774214291	0.358237417517579	0.668787383459804\\
0.605	0.0682764858793752	0.363168408916186	0.667011243730727\\
0.605	0.0702116989375697	0.368094709561873	0.665273245863649\\
0.605	0.0721751251265572	0.373016087543259	0.663569750098932\\
0.605	0.0741667705335422	0.377932310362199	0.661897584220061\\
0.605	0.0761866387881432	0.382843144964686	0.660253992184443\\
0.605	0.0782347310485954	0.38774835777208	0.658636586490997\\
0.605	0.0803110459884098	0.392647714712651	0.65704330441587\\
0.605	0.0824155797834956	0.397540981253432	0.655472368126509\\
0.605	0.084548326099754	0.402427922432375	0.653922248592746\\
0.605	0.0867092760811502	0.4073083028908	0.652391633146569\\
0.605	0.0888984183382709	0.412181886906127	0.650879396495579\\
0.605	0.0911157389373742	0.417048438424896	0.649384574964845\\
0.605	0.0933612213899392	0.421907721096044	0.647906343724685\\
0.605	0.0956348466427212	0.42675949830445	0.646443996754804\\
0.605	0.0979365930683194	0.431603533204733	0.644996929295963\\
0.605	0.100266436456264	0.436439588755293	0.643564622546847\\
0.605	0.102624350004627	0.441267427752584	0.642146630374429\\
0.605	0.105010304312169	0.446086812865616	0.640742567819434\\
0.605	0.107424267371012	0.450897506670668	0.639352101193731\\
0.605	0.109866204559871	0.455699271686218	0.637974939582442\\
0.605	0.112336078637817	0.460491870408058	0.636610827579862\\
0.605	0.114833849738607	0.465275065344603	0.635259539104383\\
0.605	0.117359475365564	0.470048619052377	0.633920872153155\\
0.605	0.119912910387023	0.474812294171664	0.632594644371965\\
0.605	0.122494107032347	0.479565853462319	0.631280689329641\\
0.605	0.125103014888515	0.484309059839721	0.629978853399058\\
0.605	0.127739580897283	0.489041676410868	0.628690460824558\\
0.61	0	0	0.710202759607373\\
0.61	1.11327767495586e-05	0.00471862581271275	0.713003157837853\\
0.61	4.46175251031912e-05	0.009446325184051	0.715846486660142\\
0.61	0.000100583311362513	0.0141829653360114	0.718788838379794\\
0.61	0.000179158434668431	0.018928411755669	0.721886092228356\\
0.61	0.000280470402701511	0.02368252819604	0.725200636220892\\
0.61	0.000404645907256436	0.0284451766772965	0.728799085274723\\
0.61	0.000551810799695644	0.0332162174883389	0.732749123983336\\
0.61	0.000722090066287311	0.037995509188729	0.737115575253328\\
0.61	0.000915607803432999	0.0427829086109896	0.741955893907011\\
0.61	0.0011324871927904	0.0475782708632729	0.747315373652289\\
0.61	0.00137285047629673	0.0523814493324038	0.75322242404581\\
0.61	0.00163681893109844	0.057192295687301	0.759684310525652\\
0.61	0.00192451284439304	0.0620106598827802	0.766683748102665\\
0.61	0.00223605148818898	0.0668363901637434	0.774176695511256\\
0.61	0.00257155309398959	0.0716693330697584	0.782091614494493\\
0.61	0.00293113482740716	0.0765093334400312	0.79033034642109\\
0.61	0.00331491276271365	0.0813562344187764	0.79877062752665\\
0.61	0.00372300185733414	0.0862098774609879	0.807270128930359\\
0.61	0.00415551592628969	0.0910701023386145	0.815671782700194\\
0.61	0.00461256761659621	0.0959367471471425	0.82381005350733\\
0.61	0.00509426838162598	0.100809648312589	0.831517746503354\\
0.61	0.00560072845543873	0.105688640598912	0.838632911419452\\
0.61	0.00613205682708918	0.11057355711583	0.845005411376293\\
0.61	0.00668836121491816	0.115464229327074	0.850502769049303\\
0.61	0.00726974804083431	0.120360487059049	0.855014975717657\\
0.61	0.00787632240459385	0.125262158509929	0.858458041096608\\
0.61	0.00850818805808555	0.13016907025918	0.860776163494183\\
0.61	0.00916544737962849	0.135081047277508	0.861942500814224\\
0.61	0.00984820134829017	0.139997912937243	0.861958614648422\\
0.61	0.0105565495182326	0.144919489023162	0.860852735617092\\
0.61	0.0112905899930939	0.149845595743738	0.858677054109648\\
0.61	0.012050419400414	0.154776051742839	0.855504274967459\\
0.61	0.0128361328661109	0.159710674111862	0.851423687967755\\
0.61	0.0136478239890177	0.164649278402306	0.846537000484341\\
0.61	0.0144855848154856	0.169591678638796	0.840954157904248\\
0.61	0.0153495058140643	0.174537687332543	0.834789345403934\\
0.61	0.0162396758502651	0.179487115495259	0.8281573258062\\
0.61	0.0171561821614173	0.184439772653511	0.821170226448695\\
0.61	0.0180991103316243	0.189395466863524	0.813934846725693\\
0.61	0.0190685442668299	0.194354004726437	0.806550519904212\\
0.61	0.0200645661700016	0.199315191403999	0.799107529846307\\
0.61	0.0210872565164405	0.204278830634725	0.791686056496192\\
0.61	0.0221366940292259	0.20924472475049	0.784355603830601\\
0.61	0.023212955654804	0.214212674693577	0.777174850281119\\
0.61	0.0243161165387281	0.219182480034174	0.770191853862644\\
0.61	0.025446250001561	0.224153938988321	0.76344454155446\\
0.61	0.0266034275149466	0.229126848436298	0.756961413910484\\
0.61	0.0277877186778607	0.234101003941464	0.750762400416765\\
0.61	0.0289991911930496	0.239076199769547	0.744859807805784\\
0.61	0.0302379108436654	0.244052228908366	0.739259311518077\\
0.61	0.0315039414701067	0.24902888308801	0.733960949044504\\
0.61	0.0327973449470747	0.254005952801449	0.728960082407263\\
0.61	0.0341181811608519	0.258983227325592	0.724248305111178\\
0.61	0.0354665079868145	0.263960494742775	0.71981427622068\\
0.61	0.0368423812671862	0.268937541962694	0.715644470612628\\
0.61	0.0382458547890421	0.273914154744766	0.711723839838041\\
0.61	0.0396769802625738	0.278890117720924	0.708036382389731\\
0.61	0.0411358072996216	0.283865214418836	0.704565625567165\\
0.61	0.0426223833924861	0.288839227285554	0.701295023640317\\
0.61	0.0441367538930258	0.293811937711588	0.698208278749353\\
0.61	0.0456789619920503	0.298783126055387	0.695289592055349\\
0.61	0.0472490486990192	0.303752571668251	0.692523853197511\\
0.61	0.0488470528220537	0.308720052919643	0.689896776228119\\
0.61	0.0504730109482718	0.313685347222913	0.687394989990283\\
0.61	0.0521269574244539	0.318648231061428	0.68500609046598\\
0.61	0.0538089243380495	0.323608480015096	0.682718662029129\\
0.61	0.0555189414985328	0.328565868787292	0.68052227385347\\
0.61	0.0572570364191156	0.333520171232163	0.678407456997849\\
0.61	0.0590232342988274	0.338471160382329	0.676365666960043\\
0.61	0.0608175580049697	0.34341860847696	0.674389235783157\\
0.61	0.0626400280559541	0.348362286990222	0.672471317134861\\
0.61	0.064490662604533	0.3533019666601	0.670605827172181\\
0.61	0.0663694774214291	0.358237417517579	0.668787383459804\\
0.61	0.0682764858793752	0.363168408916186	0.667011243730728\\
0.61	0.0702116989375697	0.368094709561873	0.665273245863649\\
0.61	0.0721751251265572	0.373016087543259	0.663569750098931\\
0.61	0.0741667705335422	0.377932310362199	0.66189758422006\\
0.61	0.0761866387881432	0.382843144964686	0.660253992184443\\
0.61	0.0782347310485954	0.38774835777208	0.658636586490997\\
0.61	0.0803110459884098	0.392647714712651	0.657043304415872\\
0.61	0.0824155797834956	0.397540981253432	0.655472368126509\\
0.61	0.084548326099754	0.402427922432375	0.653922248592746\\
0.61	0.0867092760811502	0.4073083028908	0.65239163314657\\
0.61	0.0888984183382709	0.412181886906127	0.650879396495578\\
0.61	0.0911157389373742	0.417048438424896	0.649384574964845\\
0.61	0.0933612213899392	0.421907721096044	0.647906343724686\\
0.61	0.0956348466427212	0.42675949830445	0.646443996754804\\
0.61	0.0979365930683194	0.431603533204733	0.644996929295961\\
0.61	0.100266436456264	0.436439588755293	0.643564622546847\\
0.61	0.102624350004627	0.441267427752584	0.64214663037443\\
0.61	0.105010304312169	0.446086812865616	0.640742567819433\\
0.61	0.107424267371012	0.450897506670668	0.639352101193731\\
0.61	0.109866204559871	0.455699271686218	0.637974939582444\\
0.61	0.112336078637817	0.460491870408058	0.636610827579862\\
0.61	0.114833849738607	0.465275065344603	0.635259539104383\\
0.61	0.117359475365564	0.470048619052377	0.633920872153154\\
0.61	0.119912910387022	0.474812294171664	0.632594644371965\\
0.61	0.122494107032347	0.479565853462319	0.63128068932964\\
0.61	0.125103014888515	0.484309059839721	0.629978853399058\\
0.61	0.127739580897283	0.489041676410868	0.628690460824559\\
0.615	0	0	0.710202759607373\\
0.615	1.11327767495586e-05	0.00471862581271274	0.713003157837853\\
0.615	4.46175251031912e-05	0.009446325184051	0.715846486660142\\
0.615	0.000100583311362513	0.0141829653360114	0.718788838379794\\
0.615	0.000179158434668431	0.018928411755669	0.721886092228356\\
0.615	0.000280470402701511	0.02368252819604	0.725200636220892\\
0.615	0.000404645907256436	0.0284451766772965	0.728799085274723\\
0.615	0.000551810799695644	0.0332162174883388	0.732749123983336\\
0.615	0.000722090066287311	0.037995509188729	0.737115575253327\\
0.615	0.000915607803432999	0.0427829086109896	0.741955893907011\\
0.615	0.0011324871927904	0.0475782708632729	0.747315373652289\\
0.615	0.00137285047629673	0.0523814493324038	0.753222424045811\\
0.615	0.00163681893109844	0.057192295687301	0.759684310525652\\
0.615	0.00192451284439304	0.0620106598827802	0.766683748102665\\
0.615	0.00223605148818899	0.0668363901637434	0.774176695511257\\
0.615	0.00257155309398959	0.0716693330697584	0.782091614494493\\
0.615	0.00293113482740716	0.0765093334400312	0.79033034642109\\
0.615	0.00331491276271365	0.0813562344187764	0.79877062752665\\
0.615	0.00372300185733414	0.0862098774609879	0.807270128930358\\
0.615	0.00415551592628969	0.0910701023386145	0.815671782700193\\
0.615	0.00461256761659621	0.0959367471471425	0.823810053507331\\
0.615	0.00509426838162598	0.100809648312589	0.831517746503353\\
0.615	0.00560072845543873	0.105688640598912	0.838632911419452\\
0.615	0.00613205682708918	0.11057355711583	0.845005411376292\\
0.615	0.00668836121491816	0.115464229327074	0.850502769049302\\
0.615	0.00726974804083431	0.120360487059049	0.855014975717656\\
0.615	0.00787632240459385	0.125262158509929	0.858458041096606\\
0.615	0.00850818805808555	0.13016907025918	0.860776163494183\\
0.615	0.00916544737962849	0.135081047277508	0.861942500814225\\
0.615	0.00984820134829017	0.139997912937243	0.86195861464842\\
0.615	0.0105565495182326	0.144919489023162	0.860852735617091\\
0.615	0.0112905899930939	0.149845595743738	0.858677054109647\\
0.615	0.012050419400414	0.154776051742839	0.855504274967457\\
0.615	0.0128361328661109	0.159710674111862	0.851423687967754\\
0.615	0.0136478239890177	0.164649278402306	0.846537000484338\\
0.615	0.0144855848154856	0.169591678638796	0.840954157904249\\
0.615	0.0153495058140643	0.174537687332543	0.834789345403934\\
0.615	0.0162396758502651	0.179487115495259	0.8281573258062\\
0.615	0.0171561821614173	0.184439772653511	0.821170226448694\\
0.615	0.0180991103316243	0.189395466863524	0.81393484672569\\
0.615	0.0190685442668299	0.194354004726437	0.806550519904211\\
0.615	0.0200645661700016	0.199315191403999	0.799107529846304\\
0.615	0.0210872565164405	0.204278830634725	0.791686056496193\\
0.615	0.0221366940292259	0.20924472475049	0.784355603830602\\
0.615	0.023212955654804	0.214212674693577	0.777174850281119\\
0.615	0.0243161165387281	0.219182480034174	0.770191853862645\\
0.615	0.025446250001561	0.224153938988321	0.763444541554458\\
0.615	0.0266034275149466	0.229126848436298	0.756961413910482\\
0.615	0.0277877186778607	0.234101003941464	0.750762400416763\\
0.615	0.0289991911930496	0.239076199769547	0.744859807805784\\
0.615	0.0302379108436654	0.244052228908366	0.739259311518081\\
0.615	0.0315039414701067	0.24902888308801	0.733960949044504\\
0.615	0.0327973449470747	0.254005952801449	0.72896008240726\\
0.615	0.0341181811608519	0.258983227325592	0.724248305111177\\
0.615	0.0354665079868145	0.263960494742775	0.71981427622068\\
0.615	0.0368423812671862	0.268937541962694	0.715644470612631\\
0.615	0.0382458547890421	0.273914154744766	0.711723839838043\\
0.615	0.0396769802625738	0.278890117720924	0.708036382389729\\
0.615	0.0411358072996216	0.283865214418836	0.704565625567164\\
0.615	0.0426223833924861	0.288839227285554	0.701295023640316\\
0.615	0.0441367538930258	0.293811937711588	0.698208278749354\\
0.615	0.0456789619920503	0.298783126055387	0.69528959205535\\
0.615	0.0472490486990192	0.303752571668251	0.692523853197512\\
0.615	0.0488470528220538	0.308720052919643	0.689896776228119\\
0.615	0.0504730109482718	0.313685347222913	0.687394989990281\\
0.615	0.0521269574244539	0.318648231061428	0.68500609046598\\
0.615	0.0538089243380495	0.323608480015096	0.682718662029127\\
0.615	0.0555189414985328	0.328565868787292	0.680522273853469\\
0.615	0.0572570364191156	0.333520171232163	0.67840745699785\\
0.615	0.0590232342988274	0.338471160382329	0.676365666960043\\
0.615	0.0608175580049697	0.34341860847696	0.674389235783156\\
0.615	0.0626400280559541	0.348362286990222	0.672471317134862\\
0.615	0.064490662604533	0.3533019666601	0.670605827172182\\
0.615	0.0663694774214291	0.358237417517579	0.668787383459804\\
0.615	0.0682764858793752	0.363168408916186	0.667011243730728\\
0.615	0.0702116989375697	0.368094709561873	0.665273245863649\\
0.615	0.0721751251265572	0.373016087543259	0.663569750098933\\
0.615	0.0741667705335422	0.377932310362199	0.661897584220059\\
0.615	0.0761866387881432	0.382843144964686	0.660253992184441\\
0.615	0.0782347310485954	0.38774835777208	0.658636586490997\\
0.615	0.0803110459884098	0.392647714712651	0.657043304415872\\
0.615	0.0824155797834956	0.397540981253432	0.655472368126509\\
0.615	0.084548326099754	0.402427922432375	0.653922248592744\\
0.615	0.0867092760811502	0.407308302890799	0.65239163314657\\
0.615	0.0888984183382709	0.412181886906127	0.650879396495579\\
0.615	0.0911157389373742	0.417048438424896	0.649384574964843\\
0.615	0.0933612213899392	0.421907721096044	0.647906343724686\\
0.615	0.0956348466427212	0.42675949830445	0.646443996754806\\
0.615	0.0979365930683194	0.431603533204733	0.644996929295961\\
0.615	0.100266436456264	0.436439588755293	0.643564622546847\\
0.615	0.102624350004627	0.441267427752584	0.642146630374429\\
0.615	0.105010304312169	0.446086812865616	0.640742567819433\\
0.615	0.107424267371012	0.450897506670668	0.63935210119373\\
0.615	0.109866204559871	0.455699271686218	0.637974939582443\\
0.615	0.112336078637817	0.460491870408058	0.636610827579863\\
0.615	0.114833849738607	0.465275065344603	0.635259539104384\\
0.615	0.117359475365564	0.470048619052377	0.633920872153155\\
0.615	0.119912910387022	0.474812294171664	0.632594644371964\\
0.615	0.122494107032347	0.479565853462319	0.631280689329641\\
0.615	0.125103014888515	0.484309059839721	0.629978853399059\\
0.615	0.127739580897283	0.489041676410868	0.628690460824559\\
0.62	0	0	0.710202759607373\\
0.62	1.11327767495586e-05	0.00471862581271274	0.713003157837853\\
0.62	4.46175251031912e-05	0.009446325184051	0.715846486660142\\
0.62	0.000100583311362513	0.0141829653360114	0.718788838379794\\
0.62	0.000179158434668431	0.018928411755669	0.721886092228356\\
0.62	0.000280470402701511	0.02368252819604	0.725200636220892\\
0.62	0.000404645907256436	0.0284451766772965	0.728799085274723\\
0.62	0.000551810799695644	0.0332162174883388	0.732749123983336\\
0.62	0.000722090066287311	0.037995509188729	0.737115575253328\\
0.62	0.000915607803432999	0.0427829086109896	0.741955893907011\\
0.62	0.0011324871927904	0.0475782708632729	0.747315373652289\\
0.62	0.00137285047629673	0.0523814493324038	0.75322242404581\\
0.62	0.00163681893109844	0.057192295687301	0.759684310525652\\
0.62	0.00192451284439304	0.0620106598827802	0.766683748102665\\
0.62	0.00223605148818898	0.0668363901637434	0.774176695511256\\
0.62	0.00257155309398959	0.0716693330697584	0.782091614494494\\
0.62	0.00293113482740716	0.0765093334400312	0.79033034642109\\
0.62	0.00331491276271365	0.0813562344187764	0.798770627526651\\
0.62	0.00372300185733414	0.0862098774609879	0.80727012893036\\
0.62	0.00415551592628969	0.0910701023386145	0.815671782700193\\
0.62	0.00461256761659621	0.0959367471471425	0.823810053507331\\
0.62	0.00509426838162598	0.100809648312589	0.831517746503354\\
0.62	0.00560072845543873	0.105688640598912	0.838632911419452\\
0.62	0.00613205682708918	0.11057355711583	0.845005411376292\\
0.62	0.00668836121491816	0.115464229327074	0.850502769049302\\
0.62	0.00726974804083431	0.120360487059049	0.855014975717657\\
0.62	0.00787632240459385	0.125262158509929	0.858458041096605\\
0.62	0.00850818805808555	0.13016907025918	0.860776163494182\\
0.62	0.00916544737962849	0.135081047277508	0.861942500814224\\
0.62	0.00984820134829017	0.139997912937243	0.86195861464842\\
0.62	0.0105565495182326	0.144919489023162	0.860852735617089\\
0.62	0.0112905899930939	0.149845595743738	0.85867705410965\\
0.62	0.012050419400414	0.154776051742839	0.855504274967458\\
0.62	0.0128361328661109	0.159710674111862	0.851423687967753\\
0.62	0.0136478239890177	0.164649278402306	0.846537000484341\\
0.62	0.0144855848154856	0.169591678638796	0.84095415790425\\
0.62	0.0153495058140643	0.174537687332543	0.834789345403934\\
0.62	0.0162396758502651	0.179487115495259	0.828157325806201\\
0.62	0.0171561821614173	0.184439772653511	0.821170226448693\\
0.62	0.0180991103316243	0.189395466863524	0.813934846725691\\
0.62	0.0190685442668299	0.194354004726437	0.806550519904213\\
0.62	0.0200645661700016	0.199315191403999	0.799107529846304\\
0.62	0.0210872565164405	0.204278830634725	0.791686056496193\\
0.62	0.0221366940292259	0.20924472475049	0.784355603830602\\
0.62	0.023212955654804	0.214212674693577	0.77717485028112\\
0.62	0.0243161165387281	0.219182480034174	0.770191853862643\\
0.62	0.025446250001561	0.224153938988321	0.763444541554457\\
0.62	0.0266034275149466	0.229126848436298	0.756961413910483\\
0.62	0.0277877186778607	0.234101003941464	0.750762400416764\\
0.62	0.0289991911930496	0.239076199769547	0.744859807805785\\
0.62	0.0302379108436654	0.244052228908366	0.73925931151808\\
0.62	0.0315039414701067	0.24902888308801	0.733960949044503\\
0.62	0.0327973449470747	0.254005952801449	0.728960082407261\\
0.62	0.0341181811608519	0.258983227325592	0.724248305111177\\
0.62	0.0354665079868145	0.263960494742775	0.719814276220682\\
0.62	0.0368423812671862	0.268937541962694	0.715644470612631\\
0.62	0.0382458547890421	0.273914154744766	0.711723839838041\\
0.62	0.0396769802625738	0.278890117720924	0.70803638238973\\
0.62	0.0411358072996216	0.283865214418836	0.704565625567165\\
0.62	0.0426223833924861	0.288839227285554	0.701295023640315\\
0.62	0.0441367538930258	0.293811937711588	0.698208278749354\\
0.62	0.0456789619920503	0.298783126055387	0.695289592055351\\
0.62	0.0472490486990192	0.303752571668251	0.692523853197512\\
0.62	0.0488470528220538	0.308720052919643	0.689896776228118\\
0.62	0.0504730109482718	0.313685347222913	0.687394989990281\\
0.62	0.0521269574244539	0.318648231061428	0.685006090465981\\
0.62	0.0538089243380495	0.323608480015097	0.682718662029128\\
0.62	0.0555189414985328	0.328565868787292	0.68052227385347\\
0.62	0.0572570364191156	0.333520171232163	0.67840745699785\\
0.62	0.0590232342988274	0.338471160382329	0.676365666960043\\
0.62	0.0608175580049697	0.34341860847696	0.674389235783156\\
0.62	0.0626400280559541	0.348362286990222	0.672471317134861\\
0.62	0.064490662604533	0.3533019666601	0.670605827172182\\
0.62	0.0663694774214291	0.358237417517579	0.668787383459804\\
0.62	0.0682764858793752	0.363168408916186	0.667011243730728\\
0.62	0.0702116989375697	0.368094709561873	0.665273245863649\\
0.62	0.0721751251265572	0.373016087543259	0.663569750098932\\
0.62	0.0741667705335423	0.377932310362199	0.66189758422006\\
0.62	0.0761866387881432	0.382843144964686	0.660253992184442\\
0.62	0.0782347310485954	0.38774835777208	0.658636586490997\\
0.62	0.0803110459884098	0.392647714712651	0.657043304415871\\
0.62	0.0824155797834956	0.397540981253432	0.655472368126509\\
0.62	0.084548326099754	0.402427922432375	0.653922248592746\\
0.62	0.0867092760811502	0.4073083028908	0.65239163314657\\
0.62	0.0888984183382709	0.412181886906127	0.650879396495579\\
0.62	0.0911157389373742	0.417048438424896	0.649384574964844\\
0.62	0.0933612213899392	0.421907721096044	0.647906343724685\\
0.62	0.0956348466427212	0.42675949830445	0.646443996754805\\
0.62	0.0979365930683194	0.431603533204733	0.644996929295961\\
0.62	0.100266436456264	0.436439588755293	0.643564622546847\\
0.62	0.102624350004627	0.441267427752584	0.642146630374429\\
0.62	0.105010304312169	0.446086812865616	0.640742567819432\\
0.62	0.107424267371012	0.450897506670668	0.639352101193731\\
0.62	0.109866204559871	0.455699271686218	0.637974939582442\\
0.62	0.112336078637817	0.460491870408058	0.636610827579862\\
0.62	0.114833849738607	0.465275065344603	0.635259539104383\\
0.62	0.117359475365564	0.470048619052377	0.633920872153156\\
0.62	0.119912910387022	0.474812294171664	0.632594644371965\\
0.62	0.122494107032347	0.479565853462319	0.63128068932964\\
0.62	0.125103014888515	0.484309059839721	0.629978853399059\\
0.62	0.127739580897283	0.489041676410868	0.628690460824561\\
0.625	0	0	0.710202759607373\\
0.625	1.11327767495586e-05	0.00471862581271274	0.713003157837853\\
0.625	4.46175251031912e-05	0.009446325184051	0.715846486660142\\
0.625	0.000100583311362513	0.0141829653360114	0.718788838379794\\
0.625	0.000179158434668431	0.018928411755669	0.721886092228356\\
0.625	0.000280470402701511	0.02368252819604	0.725200636220892\\
0.625	0.000404645907256436	0.0284451766772965	0.728799085274723\\
0.625	0.000551810799695644	0.0332162174883389	0.732749123983336\\
0.625	0.000722090066287311	0.037995509188729	0.737115575253328\\
0.625	0.000915607803432999	0.0427829086109896	0.741955893907011\\
0.625	0.0011324871927904	0.047578270863273	0.747315373652289\\
0.625	0.00137285047629673	0.0523814493324038	0.753222424045811\\
0.625	0.00163681893109844	0.057192295687301	0.759684310525652\\
0.625	0.00192451284439304	0.0620106598827802	0.766683748102665\\
0.625	0.00223605148818899	0.0668363901637434	0.774176695511256\\
0.625	0.00257155309398959	0.0716693330697584	0.782091614494493\\
0.625	0.00293113482740716	0.0765093334400312	0.79033034642109\\
0.625	0.00331491276271365	0.0813562344187764	0.798770627526651\\
0.625	0.00372300185733413	0.0862098774609879	0.80727012893036\\
0.625	0.00415551592628969	0.0910701023386145	0.815671782700193\\
0.625	0.00461256761659621	0.0959367471471425	0.82381005350733\\
0.625	0.00509426838162598	0.100809648312589	0.831517746503354\\
0.625	0.00560072845543873	0.105688640598912	0.838632911419452\\
0.625	0.00613205682708918	0.11057355711583	0.845005411376292\\
0.625	0.00668836121491815	0.115464229327074	0.850502769049302\\
0.625	0.00726974804083431	0.120360487059049	0.855014975717656\\
0.625	0.00787632240459385	0.125262158509929	0.858458041096606\\
0.625	0.00850818805808554	0.13016907025918	0.860776163494182\\
0.625	0.00916544737962849	0.135081047277508	0.861942500814224\\
0.625	0.00984820134829017	0.139997912937243	0.86195861464842\\
0.625	0.0105565495182326	0.144919489023162	0.860852735617091\\
0.625	0.0112905899930939	0.149845595743738	0.858677054109652\\
0.625	0.012050419400414	0.154776051742839	0.85550427496746\\
0.625	0.0128361328661109	0.159710674111862	0.851423687967755\\
0.625	0.0136478239890177	0.164649278402306	0.846537000484343\\
0.625	0.0144855848154856	0.169591678638796	0.840954157904247\\
0.625	0.0153495058140643	0.174537687332543	0.834789345403936\\
0.625	0.0162396758502651	0.179487115495259	0.828157325806201\\
0.625	0.0171561821614173	0.184439772653511	0.821170226448695\\
0.625	0.0180991103316243	0.189395466863524	0.813934846725693\\
0.625	0.0190685442668299	0.194354004726437	0.806550519904213\\
0.625	0.0200645661700016	0.199315191403999	0.799107529846304\\
0.625	0.0210872565164405	0.204278830634725	0.791686056496194\\
0.625	0.0221366940292259	0.20924472475049	0.784355603830605\\
0.625	0.023212955654804	0.214212674693577	0.777174850281119\\
0.625	0.0243161165387281	0.219182480034174	0.770191853862642\\
0.625	0.025446250001561	0.224153938988321	0.763444541554458\\
0.625	0.0266034275149466	0.229126848436298	0.756961413910484\\
0.625	0.0277877186778607	0.234101003941464	0.750762400416764\\
0.625	0.0289991911930496	0.239076199769547	0.744859807805785\\
0.625	0.0302379108436654	0.244052228908366	0.739259311518079\\
0.625	0.0315039414701067	0.24902888308801	0.733960949044504\\
0.625	0.0327973449470747	0.254005952801449	0.728960082407262\\
0.625	0.0341181811608519	0.258983227325592	0.724248305111179\\
0.625	0.0354665079868145	0.263960494742775	0.719814276220682\\
0.625	0.0368423812671862	0.268937541962694	0.715644470612628\\
0.625	0.0382458547890421	0.273914154744766	0.711723839838041\\
0.625	0.0396769802625738	0.278890117720924	0.708036382389731\\
0.625	0.0411358072996216	0.283865214418836	0.704565625567163\\
0.625	0.0426223833924861	0.288839227285554	0.701295023640316\\
0.625	0.0441367538930258	0.293811937711588	0.698208278749354\\
0.625	0.0456789619920503	0.298783126055387	0.69528959205535\\
0.625	0.0472490486990192	0.303752571668251	0.692523853197511\\
0.625	0.0488470528220538	0.308720052919643	0.689896776228119\\
0.625	0.0504730109482718	0.313685347222913	0.687394989990283\\
0.625	0.0521269574244539	0.318648231061428	0.68500609046598\\
0.625	0.0538089243380495	0.323608480015096	0.682718662029127\\
0.625	0.0555189414985328	0.328565868787292	0.68052227385347\\
0.625	0.0572570364191156	0.333520171232163	0.678407456997851\\
0.625	0.0590232342988274	0.338471160382329	0.676365666960044\\
0.625	0.0608175580049697	0.34341860847696	0.674389235783156\\
0.625	0.0626400280559541	0.348362286990222	0.672471317134862\\
0.625	0.064490662604533	0.3533019666601	0.670605827172181\\
0.625	0.0663694774214291	0.358237417517579	0.668787383459804\\
0.625	0.0682764858793752	0.363168408916186	0.667011243730728\\
0.625	0.0702116989375697	0.368094709561873	0.66527324586365\\
0.625	0.0721751251265572	0.373016087543259	0.663569750098933\\
0.625	0.0741667705335422	0.377932310362199	0.66189758422006\\
0.625	0.0761866387881432	0.382843144964686	0.660253992184442\\
0.625	0.0782347310485954	0.38774835777208	0.658636586490998\\
0.625	0.0803110459884098	0.392647714712651	0.657043304415871\\
0.625	0.0824155797834956	0.397540981253432	0.655472368126508\\
0.625	0.084548326099754	0.402427922432375	0.653922248592745\\
0.625	0.0867092760811502	0.407308302890799	0.652391633146571\\
0.625	0.0888984183382709	0.412181886906127	0.65087939649558\\
0.625	0.0911157389373742	0.417048438424896	0.649384574964843\\
0.625	0.0933612213899392	0.421907721096044	0.647906343724685\\
0.625	0.0956348466427212	0.42675949830445	0.646443996754807\\
0.625	0.0979365930683194	0.431603533204733	0.644996929295962\\
0.625	0.100266436456264	0.436439588755293	0.643564622546847\\
0.625	0.102624350004627	0.441267427752584	0.642146630374431\\
0.625	0.105010304312169	0.446086812865616	0.640742567819433\\
0.625	0.107424267371012	0.450897506670668	0.63935210119373\\
0.625	0.109866204559871	0.455699271686218	0.637974939582443\\
0.625	0.112336078637817	0.460491870408058	0.636610827579862\\
0.625	0.114833849738607	0.465275065344603	0.635259539104383\\
0.625	0.117359475365564	0.470048619052377	0.633920872153155\\
0.625	0.119912910387022	0.474812294171664	0.632594644371966\\
0.625	0.122494107032347	0.479565853462319	0.63128068932964\\
0.625	0.125103014888515	0.484309059839721	0.629978853399058\\
0.625	0.127739580897283	0.489041676410868	0.628690460824558\\
0.63	0	0	0.710202759607373\\
0.63	1.11327767495586e-05	0.00471862581271275	0.713003157837853\\
0.63	4.46175251031912e-05	0.009446325184051	0.715846486660142\\
0.63	0.000100583311362513	0.0141829653360114	0.718788838379794\\
0.63	0.000179158434668431	0.018928411755669	0.721886092228357\\
0.63	0.000280470402701511	0.02368252819604	0.725200636220892\\
0.63	0.000404645907256436	0.0284451766772965	0.728799085274723\\
0.63	0.000551810799695644	0.0332162174883388	0.732749123983336\\
0.63	0.000722090066287311	0.037995509188729	0.737115575253328\\
0.63	0.000915607803432999	0.0427829086109896	0.741955893907011\\
0.63	0.0011324871927904	0.0475782708632729	0.747315373652289\\
0.63	0.00137285047629673	0.0523814493324038	0.75322242404581\\
0.63	0.00163681893109844	0.057192295687301	0.759684310525652\\
0.63	0.00192451284439304	0.0620106598827802	0.766683748102665\\
0.63	0.00223605148818898	0.0668363901637434	0.774176695511256\\
0.63	0.00257155309398959	0.0716693330697584	0.782091614494494\\
0.63	0.00293113482740716	0.0765093334400312	0.79033034642109\\
0.63	0.00331491276271365	0.0813562344187764	0.79877062752665\\
0.63	0.00372300185733414	0.0862098774609879	0.80727012893036\\
0.63	0.00415551592628969	0.0910701023386145	0.815671782700193\\
0.63	0.00461256761659621	0.0959367471471424	0.823810053507331\\
0.63	0.00509426838162598	0.100809648312589	0.831517746503353\\
0.63	0.00560072845543873	0.105688640598912	0.838632911419451\\
0.63	0.00613205682708918	0.11057355711583	0.845005411376293\\
0.63	0.00668836121491816	0.115464229327074	0.850502769049302\\
0.63	0.00726974804083431	0.120360487059049	0.855014975717657\\
0.63	0.00787632240459385	0.125262158509929	0.858458041096606\\
0.63	0.00850818805808555	0.13016907025918	0.860776163494182\\
0.63	0.00916544737962849	0.135081047277508	0.861942500814222\\
0.63	0.00984820134829017	0.139997912937243	0.86195861464842\\
0.63	0.0105565495182326	0.144919489023162	0.860852735617091\\
0.63	0.0112905899930939	0.149845595743738	0.858677054109652\\
0.63	0.012050419400414	0.154776051742839	0.855504274967458\\
0.63	0.0128361328661109	0.159710674111862	0.851423687967754\\
0.63	0.0136478239890177	0.164649278402306	0.846537000484341\\
0.63	0.0144855848154856	0.169591678638796	0.840954157904247\\
0.63	0.0153495058140643	0.174537687332543	0.834789345403936\\
0.63	0.0162396758502651	0.179487115495259	0.828157325806199\\
0.63	0.0171561821614173	0.184439772653511	0.821170226448695\\
0.63	0.0180991103316243	0.189395466863524	0.813934846725691\\
0.63	0.0190685442668299	0.194354004726437	0.806550519904211\\
0.63	0.0200645661700016	0.199315191403999	0.799107529846306\\
0.63	0.0210872565164405	0.204278830634725	0.791686056496196\\
0.63	0.0221366940292259	0.20924472475049	0.784355603830602\\
0.63	0.023212955654804	0.214212674693577	0.777174850281116\\
0.63	0.0243161165387281	0.219182480034174	0.770191853862643\\
0.63	0.025446250001561	0.224153938988321	0.763444541554459\\
0.63	0.0266034275149466	0.229126848436298	0.756961413910483\\
0.63	0.0277877186778607	0.234101003941464	0.750762400416764\\
0.63	0.0289991911930496	0.239076199769547	0.744859807805785\\
0.63	0.0302379108436654	0.244052228908366	0.739259311518079\\
0.63	0.0315039414701067	0.24902888308801	0.733960949044504\\
0.63	0.0327973449470747	0.254005952801449	0.72896008240726\\
0.63	0.0341181811608519	0.258983227325592	0.724248305111178\\
0.63	0.0354665079868145	0.263960494742775	0.719814276220681\\
0.63	0.0368423812671862	0.268937541962694	0.715644470612629\\
0.63	0.0382458547890421	0.273914154744766	0.711723839838041\\
0.63	0.0396769802625738	0.278890117720924	0.70803638238973\\
0.63	0.0411358072996216	0.283865214418836	0.704565625567165\\
0.63	0.0426223833924861	0.288839227285554	0.701295023640316\\
0.63	0.0441367538930258	0.293811937711588	0.698208278749354\\
0.63	0.0456789619920503	0.298783126055387	0.69528959205535\\
0.63	0.0472490486990192	0.303752571668251	0.692523853197511\\
0.63	0.0488470528220538	0.308720052919643	0.689896776228119\\
0.63	0.0504730109482718	0.313685347222913	0.687394989990281\\
0.63	0.0521269574244539	0.318648231061428	0.685006090465981\\
0.63	0.0538089243380495	0.323608480015096	0.682718662029128\\
0.63	0.0555189414985328	0.328565868787292	0.680522273853469\\
0.63	0.0572570364191156	0.333520171232163	0.67840745699785\\
0.63	0.0590232342988274	0.338471160382329	0.676365666960044\\
0.63	0.0608175580049697	0.34341860847696	0.674389235783157\\
0.63	0.0626400280559541	0.348362286990222	0.672471317134862\\
0.63	0.064490662604533	0.3533019666601	0.670605827172183\\
0.63	0.066369477421429	0.358237417517579	0.668787383459804\\
0.63	0.0682764858793752	0.363168408916186	0.667011243730727\\
0.63	0.0702116989375697	0.368094709561873	0.665273245863649\\
0.63	0.0721751251265572	0.373016087543259	0.663569750098932\\
0.63	0.0741667705335422	0.377932310362199	0.661897584220061\\
0.63	0.0761866387881432	0.382843144964686	0.660253992184442\\
0.63	0.0782347310485954	0.38774835777208	0.658636586490996\\
0.63	0.0803110459884098	0.392647714712651	0.657043304415872\\
0.63	0.0824155797834956	0.397540981253432	0.655472368126509\\
0.63	0.084548326099754	0.402427922432375	0.653922248592745\\
0.63	0.0867092760811502	0.4073083028908	0.652391633146569\\
0.63	0.0888984183382709	0.412181886906127	0.650879396495578\\
0.63	0.0911157389373742	0.417048438424896	0.649384574964844\\
0.63	0.0933612213899392	0.421907721096044	0.647906343724684\\
0.63	0.0956348466427212	0.42675949830445	0.646443996754804\\
0.63	0.0979365930683194	0.431603533204733	0.644996929295963\\
0.63	0.100266436456264	0.436439588755293	0.643564622546847\\
0.63	0.102624350004627	0.441267427752584	0.64214663037443\\
0.63	0.105010304312169	0.446086812865616	0.640742567819434\\
0.63	0.107424267371012	0.450897506670668	0.639352101193731\\
0.63	0.109866204559871	0.455699271686218	0.637974939582442\\
0.63	0.112336078637817	0.460491870408058	0.636610827579862\\
0.63	0.114833849738607	0.465275065344603	0.635259539104383\\
0.63	0.117359475365564	0.470048619052377	0.633920872153154\\
0.63	0.119912910387023	0.474812294171664	0.632594644371966\\
0.63	0.122494107032347	0.479565853462319	0.631280689329641\\
0.63	0.125103014888515	0.484309059839721	0.629978853399059\\
0.63	0.127739580897283	0.489041676410868	0.62869046082456\\
0.635	0	0	0.710202759607373\\
0.635	1.11327767495586e-05	0.00471862581271275	0.713003157837853\\
0.635	4.46175251031912e-05	0.009446325184051	0.715846486660142\\
0.635	0.000100583311362513	0.0141829653360114	0.718788838379794\\
0.635	0.000179158434668431	0.018928411755669	0.721886092228356\\
0.635	0.000280470402701511	0.02368252819604	0.725200636220892\\
0.635	0.000404645907256436	0.0284451766772965	0.728799085274723\\
0.635	0.000551810799695644	0.0332162174883388	0.732749123983336\\
0.635	0.000722090066287311	0.037995509188729	0.737115575253327\\
0.635	0.000915607803432999	0.0427829086109896	0.741955893907011\\
0.635	0.0011324871927904	0.0475782708632729	0.747315373652289\\
0.635	0.00137285047629673	0.0523814493324038	0.75322242404581\\
0.635	0.00163681893109844	0.057192295687301	0.759684310525652\\
0.635	0.00192451284439304	0.0620106598827802	0.766683748102665\\
0.635	0.00223605148818898	0.0668363901637434	0.774176695511257\\
0.635	0.00257155309398959	0.0716693330697584	0.782091614494494\\
0.635	0.00293113482740716	0.0765093334400312	0.79033034642109\\
0.635	0.00331491276271365	0.0813562344187764	0.79877062752665\\
0.635	0.00372300185733414	0.0862098774609879	0.807270128930359\\
0.635	0.00415551592628969	0.0910701023386145	0.815671782700194\\
0.635	0.00461256761659621	0.0959367471471425	0.823810053507331\\
0.635	0.00509426838162598	0.100809648312589	0.831517746503354\\
0.635	0.00560072845543873	0.105688640598912	0.838632911419452\\
0.635	0.00613205682708918	0.11057355711583	0.845005411376293\\
0.635	0.00668836121491816	0.115464229327074	0.850502769049302\\
0.635	0.00726974804083431	0.120360487059049	0.855014975717657\\
0.635	0.00787632240459385	0.125262158509929	0.858458041096605\\
0.635	0.00850818805808555	0.13016907025918	0.860776163494181\\
0.635	0.00916544737962849	0.135081047277508	0.861942500814224\\
0.635	0.00984820134829017	0.139997912937243	0.86195861464842\\
0.635	0.0105565495182326	0.144919489023162	0.860852735617091\\
0.635	0.0112905899930939	0.149845595743738	0.858677054109649\\
0.635	0.012050419400414	0.154776051742839	0.855504274967456\\
0.635	0.0128361328661109	0.159710674111862	0.851423687967754\\
0.635	0.0136478239890177	0.164649278402306	0.84653700048434\\
0.635	0.0144855848154856	0.169591678638796	0.84095415790425\\
0.635	0.0153495058140643	0.174537687332543	0.834789345403935\\
0.635	0.0162396758502651	0.179487115495259	0.8281573258062\\
0.635	0.0171561821614173	0.184439772653511	0.821170226448694\\
0.635	0.0180991103316243	0.189395466863524	0.813934846725692\\
0.635	0.0190685442668299	0.194354004726437	0.806550519904214\\
0.635	0.0200645661700016	0.199315191403999	0.799107529846306\\
0.635	0.0210872565164405	0.204278830634725	0.791686056496195\\
0.635	0.0221366940292259	0.20924472475049	0.7843556038306\\
0.635	0.023212955654804	0.214212674693577	0.777174850281117\\
0.635	0.0243161165387281	0.219182480034174	0.770191853862644\\
0.635	0.025446250001561	0.224153938988321	0.763444541554458\\
0.635	0.0266034275149466	0.229126848436298	0.756961413910482\\
0.635	0.0277877186778607	0.234101003941464	0.750762400416765\\
0.635	0.0289991911930496	0.239076199769547	0.744859807805785\\
0.635	0.0302379108436654	0.244052228908366	0.739259311518078\\
0.635	0.0315039414701067	0.24902888308801	0.733960949044503\\
0.635	0.0327973449470747	0.254005952801449	0.728960082407261\\
0.635	0.0341181811608519	0.258983227325592	0.724248305111178\\
0.635	0.0354665079868145	0.263960494742775	0.719814276220682\\
0.635	0.0368423812671862	0.268937541962694	0.71564447061263\\
0.635	0.0382458547890421	0.273914154744766	0.71172383983804\\
0.635	0.0396769802625738	0.278890117720924	0.708036382389729\\
0.635	0.0411358072996216	0.283865214418836	0.704565625567166\\
0.635	0.0426223833924861	0.288839227285554	0.701295023640317\\
0.635	0.0441367538930258	0.293811937711588	0.698208278749352\\
0.635	0.0456789619920503	0.298783126055387	0.695289592055349\\
0.635	0.0472490486990192	0.303752571668251	0.692523853197512\\
0.635	0.0488470528220538	0.308720052919643	0.689896776228119\\
0.635	0.0504730109482718	0.313685347222913	0.68739498999028\\
0.635	0.0521269574244539	0.318648231061427	0.685006090465981\\
0.635	0.0538089243380495	0.323608480015097	0.682718662029129\\
0.635	0.0555189414985328	0.328565868787292	0.680522273853469\\
0.635	0.0572570364191156	0.333520171232163	0.67840745699785\\
0.635	0.0590232342988274	0.338471160382329	0.676365666960043\\
0.635	0.0608175580049697	0.34341860847696	0.674389235783156\\
0.635	0.0626400280559541	0.348362286990222	0.672471317134862\\
0.635	0.064490662604533	0.3533019666601	0.670605827172182\\
0.635	0.0663694774214291	0.358237417517579	0.668787383459804\\
0.635	0.0682764858793752	0.363168408916186	0.667011243730728\\
0.635	0.0702116989375697	0.368094709561873	0.665273245863649\\
0.635	0.0721751251265572	0.373016087543259	0.663569750098933\\
0.635	0.0741667705335422	0.377932310362199	0.661897584220061\\
0.635	0.0761866387881432	0.382843144964686	0.660253992184443\\
0.635	0.0782347310485954	0.38774835777208	0.658636586490996\\
0.635	0.0803110459884098	0.392647714712651	0.657043304415871\\
0.635	0.0824155797834956	0.397540981253432	0.655472368126509\\
0.635	0.084548326099754	0.402427922432375	0.653922248592745\\
0.635	0.0867092760811502	0.407308302890799	0.652391633146571\\
0.635	0.0888984183382709	0.412181886906127	0.650879396495578\\
0.635	0.0911157389373742	0.417048438424896	0.649384574964843\\
0.635	0.0933612213899392	0.421907721096044	0.647906343724685\\
0.635	0.0956348466427212	0.42675949830445	0.646443996754805\\
0.635	0.0979365930683194	0.431603533204733	0.644996929295961\\
0.635	0.100266436456264	0.436439588755293	0.643564622546847\\
0.635	0.102624350004627	0.441267427752584	0.64214663037443\\
0.635	0.105010304312169	0.446086812865616	0.640742567819433\\
0.635	0.107424267371012	0.450897506670668	0.63935210119373\\
0.635	0.109866204559871	0.455699271686218	0.637974939582443\\
0.635	0.112336078637817	0.460491870408058	0.636610827579861\\
0.635	0.114833849738607	0.465275065344603	0.635259539104383\\
0.635	0.117359475365564	0.470048619052377	0.633920872153154\\
0.635	0.119912910387022	0.474812294171664	0.632594644371967\\
0.635	0.122494107032347	0.479565853462319	0.631280689329641\\
0.635	0.125103014888515	0.484309059839721	0.629978853399058\\
0.635	0.127739580897283	0.489041676410868	0.628690460824558\\
0.64	0	0	0.710202759607373\\
0.64	1.11327767495586e-05	0.00471862581271275	0.713003157837853\\
0.64	4.46175251031912e-05	0.009446325184051	0.715846486660142\\
0.64	0.000100583311362513	0.0141829653360114	0.718788838379794\\
0.64	0.000179158434668431	0.018928411755669	0.721886092228356\\
0.64	0.000280470402701511	0.02368252819604	0.725200636220892\\
0.64	0.000404645907256436	0.0284451766772965	0.728799085274723\\
0.64	0.000551810799695644	0.0332162174883389	0.732749123983336\\
0.64	0.000722090066287311	0.037995509188729	0.737115575253328\\
0.64	0.000915607803432999	0.0427829086109896	0.741955893907011\\
0.64	0.0011324871927904	0.0475782708632729	0.747315373652289\\
0.64	0.00137285047629673	0.0523814493324038	0.753222424045811\\
0.64	0.00163681893109844	0.057192295687301	0.759684310525652\\
0.64	0.00192451284439304	0.0620106598827802	0.766683748102665\\
0.64	0.00223605148818898	0.0668363901637434	0.774176695511256\\
0.64	0.00257155309398959	0.0716693330697584	0.782091614494494\\
0.64	0.00293113482740716	0.0765093334400312	0.79033034642109\\
0.64	0.00331491276271365	0.0813562344187764	0.79877062752665\\
0.64	0.00372300185733413	0.0862098774609879	0.807270128930359\\
0.64	0.00415551592628969	0.0910701023386145	0.815671782700194\\
0.64	0.00461256761659621	0.0959367471471424	0.82381005350733\\
0.64	0.00509426838162598	0.100809648312589	0.831517746503353\\
0.64	0.00560072845543873	0.105688640598912	0.838632911419452\\
0.64	0.00613205682708918	0.11057355711583	0.845005411376293\\
0.64	0.00668836121491816	0.115464229327074	0.850502769049302\\
0.64	0.00726974804083431	0.120360487059049	0.855014975717657\\
0.64	0.00787632240459385	0.125262158509929	0.858458041096606\\
0.64	0.00850818805808555	0.13016907025918	0.860776163494181\\
0.64	0.00916544737962849	0.135081047277508	0.861942500814227\\
0.64	0.00984820134829017	0.139997912937243	0.86195861464842\\
0.64	0.0105565495182326	0.144919489023162	0.860852735617091\\
0.64	0.0112905899930939	0.149845595743738	0.858677054109652\\
0.64	0.012050419400414	0.154776051742839	0.855504274967456\\
0.64	0.0128361328661109	0.159710674111862	0.851423687967755\\
0.64	0.0136478239890177	0.164649278402306	0.846537000484341\\
0.64	0.0144855848154856	0.169591678638796	0.84095415790425\\
0.64	0.0153495058140643	0.174537687332543	0.834789345403935\\
0.64	0.0162396758502651	0.179487115495259	0.828157325806199\\
0.64	0.0171561821614173	0.184439772653511	0.821170226448693\\
0.64	0.0180991103316243	0.189395466863524	0.813934846725692\\
0.64	0.0190685442668299	0.194354004726437	0.806550519904213\\
0.64	0.0200645661700016	0.199315191403999	0.799107529846306\\
0.64	0.0210872565164405	0.204278830634725	0.791686056496192\\
0.64	0.0221366940292259	0.20924472475049	0.784355603830601\\
0.64	0.023212955654804	0.214212674693577	0.777174850281119\\
0.64	0.0243161165387281	0.219182480034174	0.770191853862644\\
0.64	0.025446250001561	0.224153938988321	0.763444541554458\\
0.64	0.0266034275149466	0.229126848436298	0.756961413910483\\
0.64	0.0277877186778607	0.234101003941464	0.750762400416764\\
0.64	0.0289991911930496	0.239076199769547	0.744859807805784\\
0.64	0.0302379108436654	0.244052228908366	0.739259311518079\\
0.64	0.0315039414701067	0.24902888308801	0.733960949044504\\
0.64	0.0327973449470747	0.254005952801449	0.728960082407259\\
0.64	0.0341181811608519	0.258983227325591	0.724248305111179\\
0.64	0.0354665079868145	0.263960494742775	0.719814276220683\\
0.64	0.0368423812671862	0.268937541962694	0.715644470612629\\
0.64	0.0382458547890421	0.273914154744766	0.711723839838041\\
0.64	0.0396769802625738	0.278890117720924	0.70803638238973\\
0.64	0.0411358072996216	0.283865214418836	0.704565625567165\\
0.64	0.0426223833924861	0.288839227285554	0.701295023640316\\
0.64	0.0441367538930258	0.293811937711588	0.698208278749354\\
0.64	0.0456789619920503	0.298783126055387	0.695289592055348\\
0.64	0.0472490486990192	0.303752571668251	0.69252385319751\\
0.64	0.0488470528220538	0.308720052919643	0.68989677622812\\
0.64	0.0504730109482718	0.313685347222913	0.687394989990282\\
0.64	0.0521269574244539	0.318648231061428	0.68500609046598\\
0.64	0.0538089243380495	0.323608480015096	0.682718662029128\\
0.64	0.0555189414985328	0.328565868787292	0.68052227385347\\
0.64	0.0572570364191156	0.333520171232163	0.67840745699785\\
0.64	0.0590232342988274	0.338471160382329	0.676365666960044\\
0.64	0.0608175580049697	0.34341860847696	0.674389235783156\\
0.64	0.0626400280559541	0.348362286990222	0.672471317134862\\
0.64	0.064490662604533	0.3533019666601	0.670605827172182\\
0.64	0.066369477421429	0.358237417517579	0.668787383459803\\
0.64	0.0682764858793752	0.363168408916186	0.667011243730727\\
0.64	0.0702116989375697	0.368094709561873	0.665273245863649\\
0.64	0.0721751251265572	0.373016087543259	0.663569750098932\\
0.64	0.0741667705335422	0.377932310362199	0.66189758422006\\
0.64	0.0761866387881432	0.382843144964686	0.660253992184443\\
0.64	0.0782347310485954	0.38774835777208	0.658636586490998\\
0.64	0.0803110459884098	0.392647714712651	0.657043304415871\\
0.64	0.0824155797834956	0.397540981253432	0.65547236812651\\
0.64	0.084548326099754	0.402427922432375	0.653922248592746\\
0.64	0.0867092760811502	0.4073083028908	0.652391633146571\\
0.64	0.0888984183382709	0.412181886906127	0.650879396495578\\
0.64	0.0911157389373742	0.417048438424896	0.649384574964844\\
0.64	0.0933612213899392	0.421907721096044	0.647906343724685\\
0.64	0.0956348466427212	0.42675949830445	0.646443996754805\\
0.64	0.0979365930683194	0.431603533204733	0.644996929295962\\
0.64	0.100266436456264	0.436439588755293	0.643564622546847\\
0.64	0.102624350004627	0.441267427752585	0.64214663037443\\
0.64	0.105010304312169	0.446086812865616	0.640742567819433\\
0.64	0.107424267371012	0.450897506670668	0.639352101193729\\
0.64	0.109866204559871	0.455699271686218	0.637974939582443\\
0.64	0.112336078637817	0.460491870408058	0.636610827579862\\
0.64	0.114833849738607	0.465275065344603	0.635259539104383\\
0.64	0.117359475365564	0.470048619052377	0.633920872153154\\
0.64	0.119912910387023	0.474812294171664	0.632594644371965\\
0.64	0.122494107032347	0.479565853462319	0.631280689329642\\
0.64	0.125103014888515	0.484309059839721	0.629978853399059\\
0.64	0.127739580897283	0.489041676410868	0.628690460824559\\
0.645	0	0	0.710202759607373\\
0.645	1.11327767495586e-05	0.00471862581271275	0.713003157837853\\
0.645	4.46175251031912e-05	0.009446325184051	0.715846486660142\\
0.645	0.000100583311362513	0.0141829653360114	0.718788838379794\\
0.645	0.000179158434668431	0.018928411755669	0.721886092228356\\
0.645	0.000280470402701511	0.02368252819604	0.725200636220892\\
0.645	0.000404645907256436	0.0284451766772965	0.728799085274723\\
0.645	0.000551810799695644	0.0332162174883388	0.732749123983336\\
0.645	0.000722090066287311	0.037995509188729	0.737115575253328\\
0.645	0.000915607803432999	0.0427829086109896	0.741955893907011\\
0.645	0.0011324871927904	0.047578270863273	0.747315373652289\\
0.645	0.00137285047629673	0.0523814493324038	0.75322242404581\\
0.645	0.00163681893109844	0.057192295687301	0.759684310525652\\
0.645	0.00192451284439304	0.0620106598827802	0.766683748102665\\
0.645	0.00223605148818898	0.0668363901637434	0.774176695511256\\
0.645	0.00257155309398959	0.0716693330697584	0.782091614494493\\
0.645	0.00293113482740716	0.0765093334400312	0.79033034642109\\
0.645	0.00331491276271365	0.0813562344187764	0.79877062752665\\
0.645	0.00372300185733414	0.0862098774609879	0.807270128930359\\
0.645	0.00415551592628969	0.0910701023386145	0.815671782700193\\
0.645	0.00461256761659621	0.0959367471471424	0.823810053507331\\
0.645	0.00509426838162598	0.100809648312589	0.831517746503354\\
0.645	0.00560072845543873	0.105688640598912	0.838632911419451\\
0.645	0.00613205682708918	0.11057355711583	0.845005411376292\\
0.645	0.00668836121491816	0.115464229327074	0.850502769049302\\
0.645	0.00726974804083431	0.120360487059049	0.855014975717657\\
0.645	0.00787632240459385	0.125262158509929	0.858458041096607\\
0.645	0.00850818805808555	0.13016907025918	0.860776163494183\\
0.645	0.00916544737962849	0.135081047277508	0.861942500814225\\
0.645	0.00984820134829017	0.139997912937243	0.86195861464842\\
0.645	0.0105565495182326	0.144919489023162	0.86085273561709\\
0.645	0.0112905899930939	0.149845595743738	0.85867705410965\\
0.645	0.012050419400414	0.154776051742839	0.855504274967455\\
0.645	0.0128361328661109	0.159710674111862	0.851423687967754\\
0.645	0.0136478239890177	0.164649278402306	0.84653700048434\\
0.645	0.0144855848154856	0.169591678638796	0.840954157904248\\
0.645	0.0153495058140643	0.174537687332543	0.834789345403934\\
0.645	0.0162396758502651	0.179487115495259	0.8281573258062\\
0.645	0.0171561821614173	0.184439772653511	0.821170226448695\\
0.645	0.0180991103316243	0.189395466863524	0.813934846725693\\
0.645	0.0190685442668299	0.194354004726437	0.806550519904212\\
0.645	0.0200645661700016	0.199315191403999	0.799107529846303\\
0.645	0.0210872565164405	0.204278830634725	0.791686056496195\\
0.645	0.0221366940292259	0.20924472475049	0.784355603830605\\
0.645	0.023212955654804	0.214212674693577	0.777174850281116\\
0.645	0.0243161165387281	0.219182480034174	0.770191853862643\\
0.645	0.025446250001561	0.224153938988321	0.763444541554457\\
0.645	0.0266034275149466	0.229126848436298	0.756961413910482\\
0.645	0.0277877186778607	0.234101003941464	0.750762400416763\\
0.645	0.0289991911930496	0.239076199769547	0.744859807805784\\
0.645	0.0302379108436654	0.244052228908366	0.73925931151808\\
0.645	0.0315039414701067	0.24902888308801	0.733960949044504\\
0.645	0.0327973449470747	0.254005952801449	0.728960082407261\\
0.645	0.0341181811608519	0.258983227325592	0.724248305111178\\
0.645	0.0354665079868145	0.263960494742775	0.71981427622068\\
0.645	0.0368423812671862	0.268937541962694	0.71564447061263\\
0.645	0.0382458547890421	0.273914154744766	0.711723839838042\\
0.645	0.0396769802625738	0.278890117720924	0.708036382389731\\
0.645	0.0411358072996216	0.283865214418836	0.704565625567166\\
0.645	0.0426223833924861	0.288839227285554	0.701295023640315\\
0.645	0.0441367538930258	0.293811937711588	0.698208278749352\\
0.645	0.0456789619920503	0.298783126055387	0.69528959205535\\
0.645	0.0472490486990192	0.303752571668251	0.692523853197512\\
0.645	0.0488470528220538	0.308720052919643	0.689896776228119\\
0.645	0.0504730109482718	0.313685347222913	0.687394989990281\\
0.645	0.0521269574244539	0.318648231061428	0.685006090465981\\
0.645	0.0538089243380495	0.323608480015096	0.682718662029128\\
0.645	0.0555189414985328	0.328565868787292	0.68052227385347\\
0.645	0.0572570364191156	0.333520171232163	0.678407456997848\\
0.645	0.0590232342988274	0.338471160382329	0.676365666960044\\
0.645	0.0608175580049697	0.34341860847696	0.674389235783157\\
0.645	0.0626400280559541	0.348362286990222	0.672471317134862\\
0.645	0.064490662604533	0.3533019666601	0.670605827172183\\
0.645	0.0663694774214291	0.35823741751758	0.668787383459804\\
0.645	0.0682764858793752	0.363168408916186	0.667011243730728\\
0.645	0.0702116989375697	0.368094709561873	0.665273245863649\\
0.645	0.0721751251265572	0.373016087543259	0.663569750098932\\
0.645	0.0741667705335422	0.377932310362199	0.661897584220059\\
0.645	0.0761866387881432	0.382843144964686	0.660253992184442\\
0.645	0.0782347310485954	0.38774835777208	0.658636586490997\\
0.645	0.0803110459884098	0.392647714712651	0.657043304415871\\
0.645	0.0824155797834956	0.397540981253432	0.655472368126509\\
0.645	0.084548326099754	0.402427922432375	0.653922248592745\\
0.645	0.0867092760811502	0.407308302890799	0.65239163314657\\
0.645	0.0888984183382709	0.412181886906127	0.65087939649558\\
0.645	0.0911157389373742	0.417048438424896	0.649384574964845\\
0.645	0.0933612213899392	0.421907721096044	0.647906343724686\\
0.645	0.0956348466427212	0.42675949830445	0.646443996754804\\
0.645	0.0979365930683194	0.431603533204733	0.64499692929596\\
0.645	0.100266436456264	0.436439588755293	0.643564622546847\\
0.645	0.102624350004627	0.441267427752584	0.642146630374431\\
0.645	0.105010304312169	0.446086812865616	0.640742567819433\\
0.645	0.107424267371012	0.450897506670668	0.639352101193731\\
0.645	0.109866204559871	0.455699271686218	0.637974939582444\\
0.645	0.112336078637817	0.460491870408058	0.636610827579862\\
0.645	0.114833849738607	0.465275065344603	0.635259539104382\\
0.645	0.117359475365564	0.470048619052377	0.633920872153154\\
0.645	0.119912910387022	0.474812294171664	0.632594644371964\\
0.645	0.122494107032347	0.479565853462319	0.631280689329641\\
0.645	0.125103014888515	0.484309059839721	0.629978853399059\\
0.645	0.127739580897283	0.489041676410868	0.628690460824563\\
0.65	0	0	0.710202759607373\\
0.65	1.11327767495586e-05	0.00471862581271275	0.713003157837853\\
0.65	4.46175251031912e-05	0.009446325184051	0.715846486660142\\
0.65	0.000100583311362513	0.0141829653360114	0.718788838379794\\
0.65	0.000179158434668431	0.018928411755669	0.721886092228356\\
0.65	0.000280470402701511	0.02368252819604	0.725200636220892\\
0.65	0.000404645907256436	0.0284451766772965	0.728799085274723\\
0.65	0.000551810799695644	0.0332162174883389	0.732749123983336\\
0.65	0.000722090066287311	0.037995509188729	0.737115575253328\\
0.65	0.000915607803432999	0.0427829086109896	0.741955893907011\\
0.65	0.0011324871927904	0.047578270863273	0.747315373652289\\
0.65	0.00137285047629673	0.0523814493324038	0.753222424045811\\
0.65	0.00163681893109844	0.057192295687301	0.759684310525652\\
0.65	0.00192451284439304	0.0620106598827802	0.766683748102665\\
0.65	0.00223605148818898	0.0668363901637434	0.774176695511257\\
0.65	0.00257155309398959	0.0716693330697584	0.782091614494494\\
0.65	0.00293113482740716	0.0765093334400312	0.79033034642109\\
0.65	0.00331491276271365	0.0813562344187764	0.79877062752665\\
0.65	0.00372300185733413	0.0862098774609879	0.807270128930359\\
0.65	0.00415551592628969	0.0910701023386145	0.815671782700193\\
0.65	0.00461256761659621	0.0959367471471424	0.823810053507331\\
0.65	0.00509426838162598	0.100809648312589	0.831517746503354\\
0.65	0.00560072845543873	0.105688640598912	0.838632911419451\\
0.65	0.00613205682708918	0.11057355711583	0.845005411376293\\
0.65	0.00668836121491816	0.115464229327074	0.850502769049301\\
0.65	0.00726974804083431	0.120360487059049	0.855014975717657\\
0.65	0.00787632240459385	0.125262158509929	0.858458041096605\\
0.65	0.00850818805808555	0.13016907025918	0.860776163494184\\
0.65	0.00916544737962849	0.135081047277508	0.861942500814224\\
0.65	0.00984820134829017	0.139997912937243	0.86195861464842\\
0.65	0.0105565495182326	0.144919489023162	0.860852735617091\\
0.65	0.0112905899930939	0.149845595743738	0.858677054109648\\
0.65	0.012050419400414	0.154776051742839	0.855504274967459\\
0.65	0.0128361328661109	0.159710674111862	0.851423687967754\\
0.65	0.0136478239890177	0.164649278402306	0.846537000484341\\
0.65	0.0144855848154856	0.169591678638796	0.840954157904247\\
0.65	0.0153495058140643	0.174537687332543	0.834789345403935\\
0.65	0.0162396758502651	0.179487115495259	0.828157325806201\\
0.65	0.0171561821614173	0.184439772653511	0.821170226448695\\
0.65	0.0180991103316243	0.189395466863524	0.813934846725692\\
0.65	0.0190685442668299	0.194354004726437	0.806550519904211\\
0.65	0.0200645661700016	0.199315191403999	0.799107529846306\\
0.65	0.0210872565164405	0.204278830634725	0.791686056496197\\
0.65	0.0221366940292259	0.20924472475049	0.7843556038306\\
0.65	0.023212955654804	0.214212674693577	0.777174850281117\\
0.65	0.0243161165387281	0.219182480034174	0.770191853862643\\
0.65	0.025446250001561	0.224153938988321	0.763444541554458\\
0.65	0.0266034275149466	0.229126848436298	0.756961413910484\\
0.65	0.0277877186778607	0.234101003941464	0.750762400416764\\
0.65	0.0289991911930496	0.239076199769547	0.744859807805786\\
0.65	0.0302379108436654	0.244052228908366	0.73925931151808\\
0.65	0.0315039414701067	0.24902888308801	0.733960949044503\\
0.65	0.0327973449470747	0.254005952801449	0.728960082407261\\
0.65	0.0341181811608519	0.258983227325592	0.724248305111177\\
0.65	0.0354665079868145	0.263960494742775	0.719814276220679\\
0.65	0.0368423812671862	0.268937541962694	0.715644470612631\\
0.65	0.0382458547890421	0.273914154744766	0.711723839838044\\
0.65	0.0396769802625738	0.278890117720924	0.708036382389731\\
0.65	0.0411358072996216	0.283865214418836	0.704565625567163\\
0.65	0.0426223833924861	0.288839227285554	0.701295023640316\\
0.65	0.0441367538930258	0.293811937711588	0.698208278749354\\
0.65	0.0456789619920503	0.298783126055387	0.695289592055349\\
0.65	0.0472490486990192	0.303752571668251	0.692523853197512\\
0.65	0.0488470528220538	0.308720052919643	0.68989677622812\\
0.65	0.0504730109482718	0.313685347222913	0.68739498999028\\
0.65	0.0521269574244539	0.318648231061428	0.685006090465981\\
0.65	0.0538089243380495	0.323608480015097	0.682718662029128\\
0.65	0.0555189414985328	0.328565868787292	0.68052227385347\\
0.65	0.0572570364191156	0.333520171232163	0.678407456997851\\
0.65	0.0590232342988274	0.338471160382329	0.676365666960044\\
0.65	0.0608175580049697	0.34341860847696	0.674389235783156\\
0.65	0.0626400280559541	0.348362286990222	0.672471317134862\\
0.65	0.064490662604533	0.3533019666601	0.670605827172183\\
0.65	0.0663694774214291	0.358237417517579	0.668787383459804\\
0.65	0.0682764858793752	0.363168408916186	0.667011243730727\\
0.65	0.0702116989375697	0.368094709561873	0.665273245863649\\
0.65	0.0721751251265572	0.373016087543259	0.663569750098934\\
0.65	0.0741667705335422	0.377932310362199	0.66189758422006\\
0.65	0.0761866387881432	0.382843144964686	0.660253992184441\\
0.65	0.0782347310485954	0.38774835777208	0.658636586490998\\
0.65	0.0803110459884098	0.392647714712651	0.657043304415871\\
0.65	0.0824155797834956	0.397540981253432	0.655472368126509\\
0.65	0.084548326099754	0.402427922432375	0.653922248592745\\
0.65	0.0867092760811502	0.4073083028908	0.652391633146569\\
0.65	0.0888984183382709	0.412181886906127	0.650879396495578\\
0.65	0.0911157389373742	0.417048438424896	0.649384574964843\\
0.65	0.0933612213899392	0.421907721096044	0.647906343724686\\
0.65	0.0956348466427212	0.42675949830445	0.646443996754807\\
0.65	0.0979365930683194	0.431603533204733	0.64499692929596\\
0.65	0.100266436456264	0.436439588755293	0.643564622546845\\
0.65	0.102624350004627	0.441267427752584	0.64214663037443\\
0.65	0.105010304312169	0.446086812865616	0.640742567819433\\
0.65	0.107424267371012	0.450897506670668	0.63935210119373\\
0.65	0.109866204559871	0.455699271686218	0.637974939582443\\
0.65	0.112336078637817	0.460491870408058	0.636610827579863\\
0.65	0.114833849738607	0.465275065344603	0.635259539104383\\
0.65	0.117359475365564	0.470048619052377	0.633920872153155\\
0.65	0.119912910387023	0.474812294171664	0.632594644371964\\
0.65	0.122494107032347	0.479565853462319	0.631280689329642\\
0.65	0.125103014888515	0.484309059839721	0.629978853399058\\
0.65	0.127739580897283	0.489041676410868	0.628690460824559\\
0.655	0	0	0.710202759607373\\
0.655	1.11327767495586e-05	0.00471862581271275	0.713003157837853\\
0.655	4.46175251031912e-05	0.009446325184051	0.715846486660142\\
0.655	0.000100583311362513	0.0141829653360114	0.718788838379794\\
0.655	0.000179158434668431	0.018928411755669	0.721886092228357\\
0.655	0.000280470402701511	0.02368252819604	0.725200636220892\\
0.655	0.000404645907256436	0.0284451766772965	0.728799085274723\\
0.655	0.000551810799695644	0.0332162174883388	0.732749123983336\\
0.655	0.000722090066287311	0.037995509188729	0.737115575253327\\
0.655	0.000915607803432999	0.0427829086109896	0.741955893907011\\
0.655	0.0011324871927904	0.0475782708632729	0.747315373652289\\
0.655	0.00137285047629673	0.0523814493324038	0.753222424045811\\
0.655	0.00163681893109844	0.057192295687301	0.759684310525652\\
0.655	0.00192451284439304	0.0620106598827802	0.766683748102665\\
0.655	0.00223605148818899	0.0668363901637434	0.774176695511256\\
0.655	0.00257155309398959	0.0716693330697584	0.782091614494494\\
0.655	0.00293113482740716	0.0765093334400312	0.79033034642109\\
0.655	0.00331491276271365	0.0813562344187764	0.79877062752665\\
0.655	0.00372300185733413	0.0862098774609879	0.807270128930359\\
0.655	0.00415551592628969	0.0910701023386145	0.815671782700193\\
0.655	0.00461256761659621	0.0959367471471424	0.82381005350733\\
0.655	0.00509426838162598	0.100809648312589	0.831517746503354\\
0.655	0.00560072845543873	0.105688640598912	0.838632911419452\\
0.655	0.00613205682708918	0.11057355711583	0.845005411376293\\
0.655	0.00668836121491815	0.115464229327074	0.850502769049302\\
0.655	0.00726974804083431	0.120360487059049	0.855014975717656\\
0.655	0.00787632240459385	0.125262158509929	0.858458041096605\\
0.655	0.00850818805808554	0.13016907025918	0.860776163494181\\
0.655	0.00916544737962849	0.135081047277508	0.861942500814225\\
0.655	0.00984820134829017	0.139997912937243	0.861958614648422\\
0.655	0.0105565495182326	0.144919489023162	0.860852735617091\\
0.655	0.0112905899930939	0.149845595743738	0.858677054109652\\
0.655	0.012050419400414	0.154776051742839	0.855504274967458\\
0.655	0.0128361328661109	0.159710674111862	0.851423687967755\\
0.655	0.0136478239890177	0.164649278402306	0.846537000484338\\
0.655	0.0144855848154856	0.169591678638796	0.840954157904247\\
0.655	0.0153495058140643	0.174537687332543	0.834789345403935\\
0.655	0.0162396758502651	0.179487115495259	0.828157325806199\\
0.655	0.0171561821614173	0.184439772653511	0.821170226448695\\
0.655	0.0180991103316243	0.189395466863524	0.81393484672569\\
0.655	0.0190685442668299	0.194354004726437	0.806550519904213\\
0.655	0.0200645661700016	0.199315191403999	0.799107529846307\\
0.655	0.0210872565164405	0.204278830634725	0.791686056496195\\
0.655	0.0221366940292259	0.20924472475049	0.7843556038306\\
0.655	0.023212955654804	0.214212674693577	0.777174850281118\\
0.655	0.0243161165387281	0.219182480034174	0.770191853862643\\
0.655	0.025446250001561	0.224153938988321	0.76344454155446\\
0.655	0.0266034275149466	0.229126848436298	0.756961413910483\\
0.655	0.0277877186778607	0.234101003941464	0.750762400416765\\
0.655	0.0289991911930496	0.239076199769547	0.744859807805785\\
0.655	0.0302379108436654	0.244052228908366	0.739259311518078\\
0.655	0.0315039414701067	0.24902888308801	0.733960949044504\\
0.655	0.0327973449470747	0.254005952801449	0.728960082407262\\
0.655	0.0341181811608519	0.258983227325592	0.724248305111179\\
0.655	0.0354665079868145	0.263960494742775	0.719814276220681\\
0.655	0.0368423812671862	0.268937541962694	0.715644470612629\\
0.655	0.0382458547890421	0.273914154744766	0.711723839838041\\
0.655	0.0396769802625738	0.278890117720924	0.708036382389729\\
0.655	0.0411358072996216	0.283865214418836	0.704565625567166\\
0.655	0.0426223833924861	0.288839227285554	0.701295023640317\\
0.655	0.0441367538930258	0.293811937711588	0.698208278749353\\
0.655	0.0456789619920503	0.298783126055387	0.695289592055349\\
0.655	0.0472490486990192	0.303752571668251	0.692523853197511\\
0.655	0.0488470528220538	0.308720052919643	0.68989677622812\\
0.655	0.0504730109482718	0.313685347222913	0.68739498999028\\
0.655	0.0521269574244539	0.318648231061427	0.68500609046598\\
0.655	0.0538089243380495	0.323608480015096	0.682718662029128\\
0.655	0.0555189414985328	0.328565868787292	0.680522273853468\\
0.655	0.0572570364191156	0.333520171232163	0.67840745699785\\
0.655	0.0590232342988274	0.338471160382329	0.676365666960045\\
0.655	0.0608175580049697	0.34341860847696	0.674389235783157\\
0.655	0.0626400280559541	0.348362286990222	0.672471317134862\\
0.655	0.064490662604533	0.3533019666601	0.670605827172182\\
0.655	0.0663694774214291	0.358237417517579	0.668787383459804\\
0.655	0.0682764858793752	0.363168408916186	0.667011243730729\\
0.655	0.0702116989375697	0.368094709561873	0.665273245863649\\
0.655	0.0721751251265572	0.373016087543259	0.663569750098932\\
0.655	0.0741667705335422	0.377932310362199	0.661897584220061\\
0.655	0.0761866387881432	0.382843144964686	0.660253992184441\\
0.655	0.0782347310485954	0.38774835777208	0.658636586490997\\
0.655	0.0803110459884098	0.392647714712651	0.657043304415872\\
0.655	0.0824155797834956	0.397540981253432	0.655472368126509\\
0.655	0.084548326099754	0.402427922432375	0.653922248592746\\
0.655	0.0867092760811502	0.4073083028908	0.65239163314657\\
0.655	0.0888984183382709	0.412181886906127	0.650879396495579\\
0.655	0.0911157389373742	0.417048438424897	0.649384574964843\\
0.655	0.0933612213899392	0.421907721096044	0.647906343724684\\
0.655	0.0956348466427212	0.42675949830445	0.646443996754807\\
0.655	0.0979365930683194	0.431603533204733	0.644996929295962\\
0.655	0.100266436456264	0.436439588755293	0.643564622546846\\
0.655	0.102624350004627	0.441267427752585	0.64214663037443\\
0.655	0.105010304312169	0.446086812865616	0.640742567819433\\
0.655	0.107424267371012	0.450897506670668	0.63935210119373\\
0.655	0.109866204559871	0.455699271686218	0.637974939582442\\
0.655	0.112336078637817	0.460491870408058	0.636610827579862\\
0.655	0.114833849738607	0.465275065344603	0.635259539104383\\
0.655	0.117359475365564	0.470048619052377	0.633920872153154\\
0.655	0.119912910387022	0.474812294171664	0.632594644371964\\
0.655	0.122494107032347	0.479565853462319	0.631280689329641\\
0.655	0.125103014888515	0.484309059839721	0.629978853399059\\
0.655	0.127739580897283	0.489041676410868	0.628690460824557\\
0.66	0	0	0.710202759607373\\
0.66	1.11327767495586e-05	0.00471862581271275	0.713003157837853\\
0.66	4.46175251031912e-05	0.009446325184051	0.715846486660142\\
0.66	0.000100583311362513	0.0141829653360114	0.718788838379794\\
0.66	0.000179158434668431	0.018928411755669	0.721886092228356\\
0.66	0.000280470402701511	0.02368252819604	0.725200636220892\\
0.66	0.000404645907256436	0.0284451766772965	0.728799085274723\\
0.66	0.000551810799695644	0.0332162174883388	0.732749123983336\\
0.66	0.000722090066287311	0.037995509188729	0.737115575253328\\
0.66	0.000915607803432999	0.0427829086109896	0.741955893907011\\
0.66	0.0011324871927904	0.0475782708632729	0.747315373652289\\
0.66	0.00137285047629673	0.0523814493324038	0.75322242404581\\
0.66	0.00163681893109844	0.057192295687301	0.759684310525652\\
0.66	0.00192451284439304	0.0620106598827802	0.766683748102665\\
0.66	0.00223605148818898	0.0668363901637434	0.774176695511257\\
0.66	0.00257155309398959	0.0716693330697584	0.782091614494494\\
0.66	0.00293113482740716	0.0765093334400312	0.79033034642109\\
0.66	0.00331491276271365	0.0813562344187764	0.79877062752665\\
0.66	0.00372300185733413	0.0862098774609879	0.807270128930359\\
0.66	0.00415551592628969	0.0910701023386145	0.815671782700194\\
0.66	0.00461256761659621	0.0959367471471425	0.823810053507331\\
0.66	0.00509426838162598	0.100809648312589	0.831517746503354\\
0.66	0.00560072845543873	0.105688640598912	0.838632911419452\\
0.66	0.00613205682708918	0.11057355711583	0.845005411376293\\
0.66	0.00668836121491816	0.115464229327074	0.850502769049302\\
0.66	0.00726974804083431	0.120360487059049	0.855014975717656\\
0.66	0.00787632240459385	0.125262158509929	0.858458041096606\\
0.66	0.00850818805808555	0.13016907025918	0.860776163494181\\
0.66	0.00916544737962849	0.135081047277508	0.861942500814223\\
0.66	0.00984820134829017	0.139997912937243	0.861958614648422\\
0.66	0.0105565495182326	0.144919489023162	0.86085273561709\\
0.66	0.0112905899930939	0.149845595743738	0.858677054109649\\
0.66	0.012050419400414	0.154776051742839	0.855504274967457\\
0.66	0.0128361328661109	0.159710674111862	0.851423687967753\\
0.66	0.0136478239890177	0.164649278402306	0.84653700048434\\
0.66	0.0144855848154856	0.169591678638796	0.84095415790425\\
0.66	0.0153495058140643	0.174537687332543	0.834789345403933\\
0.66	0.0162396758502651	0.179487115495259	0.828157325806201\\
0.66	0.0171561821614173	0.184439772653511	0.821170226448694\\
0.66	0.0180991103316243	0.189395466863524	0.813934846725691\\
0.66	0.0190685442668299	0.194354004726437	0.806550519904214\\
0.66	0.0200645661700016	0.199315191403999	0.799107529846305\\
0.66	0.0210872565164405	0.204278830634725	0.791686056496194\\
0.66	0.0221366940292259	0.20924472475049	0.7843556038306\\
0.66	0.023212955654804	0.214212674693577	0.777174850281116\\
0.66	0.0243161165387281	0.219182480034174	0.770191853862643\\
0.66	0.025446250001561	0.224153938988321	0.763444541554457\\
0.66	0.0266034275149466	0.229126848436298	0.756961413910483\\
0.66	0.0277877186778607	0.234101003941464	0.750762400416764\\
0.66	0.0289991911930496	0.239076199769547	0.744859807805783\\
0.66	0.0302379108436654	0.244052228908366	0.739259311518079\\
0.66	0.0315039414701067	0.24902888308801	0.733960949044504\\
0.66	0.0327973449470747	0.254005952801449	0.72896008240726\\
0.66	0.0341181811608519	0.258983227325592	0.724248305111176\\
0.66	0.0354665079868146	0.263960494742775	0.71981427622068\\
0.66	0.0368423812671862	0.268937541962694	0.71564447061263\\
0.66	0.0382458547890421	0.273914154744766	0.711723839838042\\
0.66	0.0396769802625738	0.278890117720924	0.70803638238973\\
0.66	0.0411358072996216	0.283865214418836	0.704565625567164\\
0.66	0.0426223833924861	0.288839227285554	0.701295023640316\\
0.66	0.0441367538930258	0.293811937711588	0.698208278749354\\
0.66	0.0456789619920503	0.298783126055387	0.695289592055349\\
0.66	0.0472490486990192	0.303752571668251	0.692523853197512\\
0.66	0.0488470528220538	0.308720052919643	0.689896776228119\\
0.66	0.0504730109482718	0.313685347222913	0.687394989990281\\
0.66	0.0521269574244539	0.318648231061428	0.68500609046598\\
0.66	0.0538089243380495	0.323608480015096	0.682718662029128\\
0.66	0.0555189414985328	0.328565868787292	0.68052227385347\\
0.66	0.0572570364191156	0.333520171232163	0.678407456997849\\
0.66	0.0590232342988274	0.338471160382329	0.676365666960043\\
0.66	0.0608175580049697	0.34341860847696	0.674389235783156\\
0.66	0.0626400280559541	0.348362286990222	0.672471317134861\\
0.66	0.064490662604533	0.3533019666601	0.670605827172183\\
0.66	0.0663694774214291	0.358237417517579	0.668787383459804\\
0.66	0.0682764858793752	0.363168408916186	0.667011243730728\\
0.66	0.0702116989375697	0.368094709561873	0.665273245863649\\
0.66	0.0721751251265572	0.373016087543259	0.663569750098932\\
0.66	0.0741667705335422	0.377932310362199	0.661897584220061\\
0.66	0.0761866387881432	0.382843144964686	0.660253992184442\\
0.66	0.0782347310485954	0.38774835777208	0.658636586490996\\
0.66	0.0803110459884098	0.392647714712651	0.657043304415871\\
0.66	0.0824155797834956	0.397540981253432	0.655472368126509\\
0.66	0.084548326099754	0.402427922432375	0.653922248592745\\
0.66	0.0867092760811502	0.4073083028908	0.65239163314657\\
0.66	0.0888984183382709	0.412181886906127	0.650879396495578\\
0.66	0.0911157389373742	0.417048438424896	0.649384574964845\\
0.66	0.0933612213899392	0.421907721096044	0.647906343724685\\
0.66	0.0956348466427212	0.42675949830445	0.646443996754806\\
0.66	0.0979365930683194	0.431603533204733	0.644996929295962\\
0.66	0.100266436456264	0.436439588755293	0.643564622546847\\
0.66	0.102624350004627	0.441267427752585	0.64214663037443\\
0.66	0.105010304312169	0.446086812865616	0.640742567819433\\
0.66	0.107424267371012	0.450897506670668	0.639352101193731\\
0.66	0.109866204559871	0.455699271686218	0.637974939582442\\
0.66	0.112336078637817	0.460491870408058	0.636610827579861\\
0.66	0.114833849738607	0.465275065344603	0.635259539104383\\
0.66	0.117359475365564	0.470048619052377	0.633920872153155\\
0.66	0.119912910387022	0.474812294171664	0.632594644371966\\
0.66	0.122494107032347	0.479565853462319	0.63128068932964\\
0.66	0.125103014888515	0.484309059839721	0.62997885339906\\
0.66	0.127739580897283	0.489041676410868	0.62869046082456\\
0.665	0	0	0.710202759607373\\
0.665	1.11327767495586e-05	0.00471862581271274	0.713003157837853\\
0.665	4.46175251031912e-05	0.009446325184051	0.715846486660142\\
0.665	0.000100583311362513	0.0141829653360114	0.718788838379794\\
0.665	0.000179158434668431	0.018928411755669	0.721886092228357\\
0.665	0.000280470402701511	0.02368252819604	0.725200636220892\\
0.665	0.000404645907256436	0.0284451766772965	0.728799085274723\\
0.665	0.000551810799695644	0.0332162174883389	0.732749123983336\\
0.665	0.000722090066287311	0.037995509188729	0.737115575253328\\
0.665	0.000915607803432999	0.0427829086109896	0.741955893907011\\
0.665	0.0011324871927904	0.0475782708632729	0.747315373652289\\
0.665	0.00137285047629673	0.0523814493324038	0.753222424045811\\
0.665	0.00163681893109844	0.057192295687301	0.759684310525652\\
0.665	0.00192451284439304	0.0620106598827802	0.766683748102665\\
0.665	0.00223605148818899	0.0668363901637434	0.774176695511256\\
0.665	0.00257155309398959	0.0716693330697584	0.782091614494494\\
0.665	0.00293113482740716	0.0765093334400312	0.79033034642109\\
0.665	0.00331491276271365	0.0813562344187764	0.79877062752665\\
0.665	0.00372300185733414	0.0862098774609879	0.807270128930359\\
0.665	0.00415551592628969	0.0910701023386145	0.815671782700194\\
0.665	0.00461256761659621	0.0959367471471425	0.82381005350733\\
0.665	0.00509426838162598	0.100809648312589	0.831517746503354\\
0.665	0.00560072845543873	0.105688640598912	0.838632911419451\\
0.665	0.00613205682708918	0.11057355711583	0.845005411376293\\
0.665	0.00668836121491816	0.115464229327074	0.850502769049302\\
0.665	0.00726974804083431	0.120360487059049	0.855014975717658\\
0.665	0.00787632240459385	0.125262158509929	0.858458041096605\\
0.665	0.00850818805808554	0.13016907025918	0.860776163494182\\
0.665	0.00916544737962849	0.135081047277508	0.861942500814224\\
0.665	0.00984820134829017	0.139997912937243	0.861958614648419\\
0.665	0.0105565495182326	0.144919489023162	0.86085273561709\\
0.665	0.0112905899930939	0.149845595743738	0.85867705410965\\
0.665	0.012050419400414	0.154776051742839	0.855504274967458\\
0.665	0.0128361328661109	0.159710674111862	0.851423687967754\\
0.665	0.0136478239890177	0.164649278402306	0.84653700048434\\
0.665	0.0144855848154856	0.169591678638796	0.840954157904248\\
0.665	0.0153495058140643	0.174537687332543	0.834789345403935\\
0.665	0.0162396758502651	0.179487115495259	0.828157325806201\\
0.665	0.0171561821614173	0.184439772653511	0.821170226448694\\
0.665	0.0180991103316243	0.189395466863524	0.813934846725692\\
0.665	0.0190685442668299	0.194354004726437	0.806550519904212\\
0.665	0.0200645661700016	0.199315191403999	0.799107529846304\\
0.665	0.0210872565164405	0.204278830634725	0.791686056496194\\
0.665	0.0221366940292259	0.20924472475049	0.784355603830599\\
0.665	0.023212955654804	0.214212674693577	0.777174850281117\\
0.665	0.0243161165387281	0.219182480034174	0.770191853862644\\
0.665	0.025446250001561	0.224153938988321	0.763444541554458\\
0.665	0.0266034275149466	0.229126848436298	0.756961413910482\\
0.665	0.0277877186778607	0.234101003941464	0.750762400416764\\
0.665	0.0289991911930496	0.239076199769547	0.744859807805785\\
0.665	0.0302379108436654	0.244052228908366	0.73925931151808\\
0.665	0.0315039414701067	0.24902888308801	0.733960949044504\\
0.665	0.0327973449470747	0.254005952801449	0.728960082407259\\
0.665	0.0341181811608519	0.258983227325591	0.724248305111178\\
0.665	0.0354665079868145	0.263960494742775	0.719814276220681\\
0.665	0.0368423812671862	0.268937541962694	0.715644470612629\\
0.665	0.0382458547890421	0.273914154744766	0.711723839838042\\
0.665	0.0396769802625738	0.278890117720924	0.70803638238973\\
0.665	0.0411358072996216	0.283865214418836	0.704565625567164\\
0.665	0.0426223833924861	0.288839227285554	0.701295023640316\\
0.665	0.0441367538930258	0.293811937711588	0.698208278749353\\
0.665	0.0456789619920503	0.298783126055387	0.695289592055349\\
0.665	0.0472490486990192	0.303752571668251	0.692523853197512\\
0.665	0.0488470528220538	0.308720052919643	0.689896776228119\\
0.665	0.0504730109482718	0.313685347222913	0.687394989990282\\
0.665	0.0521269574244539	0.318648231061428	0.685006090465982\\
0.665	0.0538089243380495	0.323608480015097	0.682718662029128\\
0.665	0.0555189414985328	0.328565868787292	0.680522273853468\\
0.665	0.0572570364191156	0.333520171232163	0.67840745699785\\
0.665	0.0590232342988274	0.338471160382329	0.676365666960044\\
0.665	0.0608175580049697	0.34341860847696	0.674389235783157\\
0.665	0.0626400280559541	0.348362286990222	0.672471317134861\\
0.665	0.064490662604533	0.3533019666601	0.670605827172182\\
0.665	0.066369477421429	0.358237417517579	0.668787383459804\\
0.665	0.0682764858793752	0.363168408916186	0.667011243730727\\
0.665	0.0702116989375697	0.368094709561873	0.66527324586365\\
0.665	0.0721751251265572	0.373016087543259	0.663569750098933\\
0.665	0.0741667705335422	0.377932310362199	0.66189758422006\\
0.665	0.0761866387881432	0.382843144964686	0.660253992184442\\
0.665	0.0782347310485954	0.38774835777208	0.658636586490997\\
0.665	0.0803110459884098	0.392647714712651	0.657043304415871\\
0.665	0.0824155797834956	0.397540981253432	0.65547236812651\\
0.665	0.084548326099754	0.402427922432375	0.653922248592745\\
0.665	0.0867092760811502	0.4073083028908	0.65239163314657\\
0.665	0.0888984183382709	0.412181886906127	0.650879396495578\\
0.665	0.0911157389373742	0.417048438424897	0.649384574964843\\
0.665	0.0933612213899392	0.421907721096044	0.647906343724685\\
0.665	0.0956348466427212	0.42675949830445	0.646443996754806\\
0.665	0.0979365930683194	0.431603533204733	0.644996929295962\\
0.665	0.100266436456264	0.436439588755293	0.643564622546845\\
0.665	0.102624350004627	0.441267427752584	0.64214663037443\\
0.665	0.105010304312169	0.446086812865616	0.640742567819434\\
0.665	0.107424267371012	0.450897506670668	0.639352101193731\\
0.665	0.109866204559871	0.455699271686218	0.637974939582443\\
0.665	0.112336078637817	0.460491870408058	0.636610827579862\\
0.665	0.114833849738607	0.465275065344603	0.635259539104382\\
0.665	0.117359475365564	0.470048619052377	0.633920872153154\\
0.665	0.119912910387022	0.474812294171664	0.632594644371965\\
0.665	0.122494107032347	0.479565853462319	0.631280689329641\\
0.665	0.125103014888515	0.484309059839721	0.62997885339906\\
0.665	0.127739580897283	0.489041676410868	0.628690460824563\\
0.67	0	0	0.710202759607373\\
0.67	1.11327767495586e-05	0.00471862581271274	0.713003157837853\\
0.67	4.46175251031912e-05	0.009446325184051	0.715846486660142\\
0.67	0.000100583311362513	0.0141829653360114	0.718788838379794\\
0.67	0.000179158434668431	0.018928411755669	0.721886092228356\\
0.67	0.000280470402701511	0.02368252819604	0.725200636220892\\
0.67	0.000404645907256436	0.0284451766772965	0.728799085274723\\
0.67	0.000551810799695644	0.0332162174883388	0.732749123983336\\
0.67	0.000722090066287311	0.037995509188729	0.737115575253327\\
0.67	0.000915607803432999	0.0427829086109896	0.741955893907011\\
0.67	0.0011324871927904	0.0475782708632729	0.747315373652289\\
0.67	0.00137285047629673	0.0523814493324038	0.75322242404581\\
0.67	0.00163681893109844	0.057192295687301	0.759684310525652\\
0.67	0.00192451284439304	0.0620106598827802	0.766683748102665\\
0.67	0.00223605148818898	0.0668363901637434	0.774176695511256\\
0.67	0.00257155309398959	0.0716693330697584	0.782091614494493\\
0.67	0.00293113482740716	0.0765093334400312	0.79033034642109\\
0.67	0.00331491276271365	0.0813562344187764	0.79877062752665\\
0.67	0.00372300185733413	0.0862098774609879	0.807270128930358\\
0.67	0.00415551592628969	0.0910701023386145	0.815671782700194\\
0.67	0.00461256761659621	0.0959367471471425	0.82381005350733\\
0.67	0.00509426838162598	0.100809648312589	0.831517746503354\\
0.67	0.00560072845543873	0.105688640598912	0.838632911419452\\
0.67	0.00613205682708918	0.11057355711583	0.845005411376292\\
0.67	0.00668836121491816	0.115464229327074	0.850502769049301\\
0.67	0.00726974804083431	0.120360487059049	0.855014975717657\\
0.67	0.00787632240459385	0.125262158509929	0.858458041096607\\
0.67	0.00850818805808554	0.13016907025918	0.860776163494182\\
0.67	0.00916544737962849	0.135081047277508	0.861942500814224\\
0.67	0.00984820134829017	0.139997912937243	0.86195861464842\\
0.67	0.0105565495182326	0.144919489023162	0.86085273561709\\
0.67	0.0112905899930939	0.149845595743738	0.858677054109652\\
0.67	0.012050419400414	0.154776051742839	0.855504274967457\\
0.67	0.0128361328661109	0.159710674111862	0.851423687967754\\
0.67	0.0136478239890177	0.164649278402306	0.84653700048434\\
0.67	0.0144855848154856	0.169591678638796	0.840954157904248\\
0.67	0.0153495058140643	0.174537687332543	0.834789345403935\\
0.67	0.0162396758502651	0.179487115495259	0.828157325806201\\
0.67	0.0171561821614173	0.184439772653511	0.821170226448693\\
0.67	0.0180991103316243	0.189395466863524	0.813934846725691\\
0.67	0.0190685442668299	0.194354004726437	0.806550519904213\\
0.67	0.0200645661700016	0.199315191403999	0.799107529846304\\
0.67	0.0210872565164405	0.204278830634725	0.791686056496193\\
0.67	0.0221366940292259	0.20924472475049	0.784355603830603\\
0.67	0.023212955654804	0.214212674693577	0.77717485028112\\
0.67	0.0243161165387281	0.219182480034174	0.770191853862644\\
0.67	0.025446250001561	0.224153938988321	0.763444541554458\\
0.67	0.0266034275149466	0.229126848436298	0.756961413910483\\
0.67	0.0277877186778607	0.234101003941464	0.750762400416766\\
0.67	0.0289991911930496	0.239076199769547	0.744859807805786\\
0.67	0.0302379108436654	0.244052228908366	0.73925931151808\\
0.67	0.0315039414701067	0.24902888308801	0.733960949044502\\
0.67	0.0327973449470747	0.254005952801449	0.728960082407261\\
0.67	0.0341181811608519	0.258983227325592	0.724248305111178\\
0.67	0.0354665079868145	0.263960494742775	0.719814276220681\\
0.67	0.0368423812671862	0.268937541962694	0.71564447061263\\
0.67	0.0382458547890421	0.273914154744766	0.711723839838042\\
0.67	0.0396769802625738	0.278890117720924	0.70803638238973\\
0.67	0.0411358072996216	0.283865214418836	0.704565625567164\\
0.67	0.0426223833924861	0.288839227285554	0.701295023640316\\
0.67	0.0441367538930258	0.293811937711588	0.698208278749355\\
0.67	0.0456789619920503	0.298783126055387	0.69528959205535\\
0.67	0.0472490486990192	0.303752571668251	0.692523853197511\\
0.67	0.0488470528220538	0.308720052919643	0.689896776228118\\
0.67	0.0504730109482718	0.313685347222913	0.687394989990281\\
0.67	0.0521269574244539	0.318648231061428	0.685006090465982\\
0.67	0.0538089243380495	0.323608480015097	0.682718662029128\\
0.67	0.0555189414985328	0.328565868787292	0.680522273853468\\
0.67	0.0572570364191156	0.333520171232163	0.67840745699785\\
0.67	0.0590232342988274	0.338471160382329	0.676365666960043\\
0.67	0.0608175580049697	0.34341860847696	0.674389235783156\\
0.67	0.0626400280559541	0.348362286990222	0.672471317134863\\
0.67	0.064490662604533	0.3533019666601	0.670605827172183\\
0.67	0.0663694774214291	0.35823741751758	0.668787383459803\\
0.67	0.0682764858793752	0.363168408916186	0.667011243730728\\
0.67	0.0702116989375697	0.368094709561873	0.665273245863649\\
0.67	0.0721751251265572	0.373016087543259	0.663569750098933\\
0.67	0.0741667705335422	0.377932310362199	0.661897584220059\\
0.67	0.0761866387881432	0.382843144964686	0.660253992184441\\
0.67	0.0782347310485954	0.38774835777208	0.658636586490997\\
0.67	0.0803110459884098	0.392647714712651	0.657043304415871\\
0.67	0.0824155797834956	0.397540981253432	0.655472368126509\\
0.67	0.084548326099754	0.402427922432375	0.653922248592746\\
0.67	0.0867092760811502	0.407308302890799	0.65239163314657\\
0.67	0.0888984183382709	0.412181886906127	0.65087939649558\\
0.67	0.0911157389373742	0.417048438424896	0.649384574964843\\
0.67	0.0933612213899392	0.421907721096044	0.647906343724684\\
0.67	0.0956348466427212	0.42675949830445	0.646443996754806\\
0.67	0.0979365930683194	0.431603533204733	0.644996929295962\\
0.67	0.100266436456264	0.436439588755293	0.643564622546846\\
0.67	0.102624350004627	0.441267427752584	0.642146630374428\\
0.67	0.105010304312169	0.446086812865616	0.640742567819433\\
0.67	0.107424267371012	0.450897506670668	0.639352101193731\\
0.67	0.109866204559871	0.455699271686218	0.637974939582442\\
0.67	0.112336078637817	0.460491870408058	0.636610827579862\\
0.67	0.114833849738607	0.465275065344603	0.635259539104383\\
0.67	0.117359475365564	0.470048619052377	0.633920872153155\\
0.67	0.119912910387023	0.474812294171664	0.632594644371965\\
0.67	0.122494107032347	0.479565853462319	0.631280689329641\\
0.67	0.125103014888515	0.484309059839721	0.629978853399058\\
0.67	0.127739580897283	0.489041676410868	0.62869046082456\\
0.675	0	0	0.710202759607373\\
0.675	1.11327767495586e-05	0.00471862581271275	0.713003157837853\\
0.675	4.46175251031912e-05	0.009446325184051	0.715846486660142\\
0.675	0.000100583311362513	0.0141829653360114	0.718788838379794\\
0.675	0.000179158434668431	0.018928411755669	0.721886092228356\\
0.675	0.000280470402701511	0.02368252819604	0.725200636220892\\
0.675	0.000404645907256436	0.0284451766772965	0.728799085274723\\
0.675	0.000551810799695644	0.0332162174883389	0.732749123983336\\
0.675	0.000722090066287311	0.037995509188729	0.737115575253328\\
0.675	0.000915607803432999	0.0427829086109896	0.741955893907011\\
0.675	0.0011324871927904	0.0475782708632729	0.747315373652289\\
0.675	0.00137285047629673	0.0523814493324038	0.753222424045811\\
0.675	0.00163681893109844	0.057192295687301	0.759684310525652\\
0.675	0.00192451284439304	0.0620106598827802	0.766683748102665\\
0.675	0.00223605148818899	0.0668363901637434	0.774176695511256\\
0.675	0.00257155309398959	0.0716693330697584	0.782091614494493\\
0.675	0.00293113482740716	0.0765093334400312	0.79033034642109\\
0.675	0.00331491276271365	0.0813562344187764	0.798770627526651\\
0.675	0.00372300185733413	0.0862098774609879	0.807270128930359\\
0.675	0.00415551592628969	0.0910701023386145	0.815671782700193\\
0.675	0.00461256761659621	0.0959367471471425	0.82381005350733\\
0.675	0.00509426838162598	0.100809648312589	0.831517746503354\\
0.675	0.00560072845543873	0.105688640598912	0.838632911419452\\
0.675	0.00613205682708918	0.11057355711583	0.845005411376292\\
0.675	0.00668836121491816	0.115464229327074	0.850502769049302\\
0.675	0.00726974804083431	0.120360487059049	0.855014975717655\\
0.675	0.00787632240459385	0.125262158509929	0.858458041096607\\
0.675	0.00850818805808554	0.13016907025918	0.860776163494183\\
0.675	0.00916544737962849	0.135081047277508	0.861942500814224\\
0.675	0.00984820134829017	0.139997912937243	0.86195861464842\\
0.675	0.0105565495182326	0.144919489023162	0.860852735617091\\
0.675	0.0112905899930939	0.149845595743738	0.858677054109652\\
0.675	0.012050419400414	0.154776051742839	0.855504274967457\\
0.675	0.0128361328661109	0.159710674111862	0.851423687967756\\
0.675	0.0136478239890177	0.164649278402306	0.846537000484342\\
0.675	0.0144855848154856	0.169591678638796	0.840954157904248\\
0.675	0.0153495058140643	0.174537687332543	0.834789345403935\\
0.675	0.0162396758502651	0.179487115495259	0.8281573258062\\
0.675	0.0171561821614173	0.184439772653511	0.821170226448693\\
0.675	0.0180991103316243	0.189395466863524	0.813934846725694\\
0.675	0.0190685442668299	0.194354004726437	0.806550519904212\\
0.675	0.0200645661700016	0.199315191403999	0.799107529846304\\
0.675	0.0210872565164405	0.204278830634725	0.791686056496195\\
0.675	0.0221366940292259	0.20924472475049	0.784355603830603\\
0.675	0.023212955654804	0.214212674693577	0.777174850281117\\
0.675	0.0243161165387281	0.219182480034174	0.770191853862643\\
0.675	0.025446250001561	0.224153938988321	0.763444541554458\\
0.675	0.0266034275149466	0.229126848436298	0.756961413910482\\
0.675	0.0277877186778607	0.234101003941464	0.750762400416764\\
0.675	0.0289991911930496	0.239076199769547	0.744859807805785\\
0.675	0.0302379108436654	0.244052228908366	0.73925931151808\\
0.675	0.0315039414701067	0.24902888308801	0.733960949044503\\
0.675	0.0327973449470747	0.254005952801449	0.728960082407259\\
0.675	0.0341181811608519	0.258983227325591	0.724248305111178\\
0.675	0.0354665079868145	0.263960494742775	0.719814276220682\\
0.675	0.0368423812671862	0.268937541962694	0.715644470612629\\
0.675	0.0382458547890421	0.273914154744766	0.711723839838042\\
0.675	0.0396769802625738	0.278890117720924	0.70803638238973\\
0.675	0.0411358072996216	0.283865214418836	0.704565625567164\\
0.675	0.0426223833924861	0.288839227285554	0.701295023640316\\
0.675	0.0441367538930258	0.293811937711588	0.698208278749354\\
0.675	0.0456789619920503	0.298783126055387	0.69528959205535\\
0.675	0.0472490486990192	0.303752571668251	0.692523853197513\\
0.675	0.0488470528220538	0.308720052919643	0.689896776228119\\
0.675	0.0504730109482718	0.313685347222913	0.687394989990279\\
0.675	0.0521269574244539	0.318648231061427	0.68500609046598\\
0.675	0.0538089243380495	0.323608480015096	0.682718662029129\\
0.675	0.0555189414985328	0.328565868787292	0.680522273853469\\
0.675	0.0572570364191156	0.333520171232163	0.67840745699785\\
0.675	0.0590232342988274	0.338471160382329	0.676365666960044\\
0.675	0.0608175580049697	0.34341860847696	0.674389235783157\\
0.675	0.0626400280559541	0.348362286990222	0.672471317134862\\
0.675	0.064490662604533	0.3533019666601	0.670605827172182\\
0.675	0.0663694774214291	0.358237417517579	0.668787383459804\\
0.675	0.0682764858793752	0.363168408916186	0.667011243730728\\
0.675	0.0702116989375697	0.368094709561873	0.66527324586365\\
0.675	0.0721751251265572	0.373016087543259	0.663569750098933\\
0.675	0.0741667705335422	0.377932310362199	0.66189758422006\\
0.675	0.0761866387881432	0.382843144964686	0.660253992184441\\
0.675	0.0782347310485954	0.38774835777208	0.658636586490997\\
0.675	0.0803110459884098	0.392647714712651	0.657043304415872\\
0.675	0.0824155797834956	0.397540981253432	0.655472368126508\\
0.675	0.084548326099754	0.402427922432375	0.653922248592745\\
0.675	0.0867092760811502	0.407308302890799	0.65239163314657\\
0.675	0.0888984183382709	0.412181886906127	0.650879396495579\\
0.675	0.0911157389373742	0.417048438424896	0.649384574964844\\
0.675	0.0933612213899392	0.421907721096044	0.647906343724686\\
0.675	0.0956348466427212	0.42675949830445	0.646443996754805\\
0.675	0.0979365930683194	0.431603533204733	0.644996929295962\\
0.675	0.100266436456264	0.436439588755293	0.643564622546846\\
0.675	0.102624350004627	0.441267427752584	0.642146630374429\\
0.675	0.105010304312169	0.446086812865616	0.640742567819432\\
0.675	0.107424267371012	0.450897506670668	0.639352101193731\\
0.675	0.109866204559871	0.455699271686218	0.637974939582443\\
0.675	0.112336078637817	0.460491870408058	0.636610827579861\\
0.675	0.114833849738607	0.465275065344603	0.635259539104383\\
0.675	0.117359475365564	0.470048619052377	0.633920872153155\\
0.675	0.119912910387022	0.474812294171664	0.632594644371964\\
0.675	0.122494107032347	0.479565853462319	0.631280689329641\\
0.675	0.125103014888515	0.484309059839721	0.62997885339906\\
0.675	0.127739580897283	0.489041676410868	0.628690460824557\\
0.68	0	0	0.710202759607373\\
0.68	1.11327767495586e-05	0.00471862581271275	0.713003157837853\\
0.68	4.46175251031912e-05	0.009446325184051	0.715846486660142\\
0.68	0.000100583311362513	0.0141829653360114	0.718788838379794\\
0.68	0.000179158434668431	0.018928411755669	0.721886092228356\\
0.68	0.000280470402701511	0.02368252819604	0.725200636220892\\
0.68	0.000404645907256436	0.0284451766772965	0.728799085274723\\
0.68	0.000551810799695644	0.0332162174883388	0.732749123983336\\
0.68	0.000722090066287311	0.037995509188729	0.737115575253327\\
0.68	0.000915607803432999	0.0427829086109896	0.741955893907011\\
0.68	0.0011324871927904	0.047578270863273	0.747315373652289\\
0.68	0.00137285047629673	0.0523814493324038	0.753222424045811\\
0.68	0.00163681893109844	0.057192295687301	0.759684310525652\\
0.68	0.00192451284439304	0.0620106598827802	0.766683748102665\\
0.68	0.00223605148818898	0.0668363901637434	0.774176695511256\\
0.68	0.00257155309398959	0.0716693330697584	0.782091614494493\\
0.68	0.00293113482740716	0.0765093334400312	0.79033034642109\\
0.68	0.00331491276271365	0.0813562344187764	0.798770627526651\\
0.68	0.00372300185733413	0.0862098774609879	0.80727012893036\\
0.68	0.00415551592628969	0.0910701023386145	0.815671782700194\\
0.68	0.00461256761659621	0.0959367471471425	0.82381005350733\\
0.68	0.00509426838162598	0.100809648312589	0.831517746503354\\
0.68	0.00560072845543873	0.105688640598912	0.838632911419452\\
0.68	0.00613205682708918	0.11057355711583	0.845005411376293\\
0.68	0.00668836121491816	0.115464229327074	0.850502769049303\\
0.68	0.00726974804083431	0.120360487059049	0.855014975717656\\
0.68	0.00787632240459385	0.125262158509929	0.858458041096605\\
0.68	0.00850818805808554	0.13016907025918	0.860776163494183\\
0.68	0.00916544737962849	0.135081047277508	0.861942500814225\\
0.68	0.00984820134829017	0.139997912937243	0.86195861464842\\
0.68	0.0105565495182326	0.144919489023162	0.860852735617092\\
0.68	0.0112905899930939	0.149845595743738	0.858677054109651\\
0.68	0.012050419400414	0.154776051742839	0.855504274967457\\
0.68	0.0128361328661109	0.159710674111862	0.851423687967754\\
0.68	0.0136478239890177	0.164649278402306	0.846537000484339\\
0.68	0.0144855848154856	0.169591678638796	0.840954157904248\\
0.68	0.0153495058140643	0.174537687332543	0.834789345403935\\
0.68	0.0162396758502651	0.179487115495259	0.828157325806201\\
0.68	0.0171561821614173	0.184439772653511	0.821170226448695\\
0.68	0.0180991103316243	0.189395466863524	0.813934846725691\\
0.68	0.0190685442668299	0.194354004726437	0.806550519904212\\
0.68	0.0200645661700016	0.199315191403999	0.799107529846305\\
0.68	0.0210872565164405	0.204278830634725	0.791686056496194\\
0.68	0.0221366940292259	0.20924472475049	0.784355603830601\\
0.68	0.023212955654804	0.214212674693577	0.777174850281119\\
0.68	0.0243161165387281	0.219182480034174	0.770191853862643\\
0.68	0.025446250001561	0.224153938988321	0.763444541554458\\
0.68	0.0266034275149466	0.229126848436298	0.756961413910483\\
0.68	0.0277877186778607	0.234101003941464	0.750762400416764\\
0.68	0.0289991911930496	0.239076199769547	0.744859807805785\\
0.68	0.0302379108436654	0.244052228908366	0.739259311518081\\
0.68	0.0315039414701067	0.24902888308801	0.733960949044502\\
0.68	0.0327973449470747	0.254005952801449	0.728960082407259\\
0.68	0.0341181811608519	0.258983227325592	0.72424830511118\\
0.68	0.0354665079868145	0.263960494742775	0.71981427622068\\
0.68	0.0368423812671862	0.268937541962694	0.715644470612631\\
0.68	0.0382458547890421	0.273914154744766	0.711723839838042\\
0.68	0.0396769802625738	0.278890117720924	0.70803638238973\\
0.68	0.0411358072996216	0.283865214418836	0.704565625567164\\
0.68	0.0426223833924861	0.288839227285554	0.701295023640315\\
0.68	0.0441367538930258	0.293811937711588	0.698208278749355\\
0.68	0.0456789619920503	0.298783126055387	0.695289592055348\\
0.68	0.0472490486990192	0.303752571668251	0.692523853197512\\
0.68	0.0488470528220538	0.308720052919643	0.689896776228121\\
0.68	0.0504730109482718	0.313685347222913	0.687394989990282\\
0.68	0.0521269574244539	0.318648231061428	0.68500609046598\\
0.68	0.0538089243380495	0.323608480015096	0.682718662029127\\
0.68	0.0555189414985328	0.328565868787292	0.68052227385347\\
0.68	0.0572570364191156	0.333520171232163	0.678407456997851\\
0.68	0.0590232342988274	0.338471160382329	0.676365666960043\\
0.68	0.0608175580049697	0.34341860847696	0.674389235783155\\
0.68	0.0626400280559541	0.348362286990222	0.672471317134862\\
0.68	0.064490662604533	0.3533019666601	0.670605827172183\\
0.68	0.066369477421429	0.358237417517579	0.668787383459804\\
0.68	0.0682764858793752	0.363168408916186	0.667011243730728\\
0.68	0.0702116989375697	0.368094709561873	0.66527324586365\\
0.68	0.0721751251265572	0.373016087543259	0.663569750098932\\
0.68	0.0741667705335422	0.377932310362199	0.661897584220061\\
0.68	0.0761866387881432	0.382843144964686	0.660253992184442\\
0.68	0.0782347310485954	0.38774835777208	0.658636586490996\\
0.68	0.0803110459884098	0.392647714712651	0.657043304415872\\
0.68	0.0824155797834956	0.397540981253432	0.65547236812651\\
0.68	0.084548326099754	0.402427922432375	0.653922248592746\\
0.68	0.0867092760811502	0.4073083028908	0.652391633146569\\
0.68	0.0888984183382709	0.412181886906127	0.650879396495579\\
0.68	0.0911157389373742	0.417048438424897	0.649384574964843\\
0.68	0.0933612213899392	0.421907721096044	0.647906343724685\\
0.68	0.0956348466427212	0.42675949830445	0.646443996754806\\
0.68	0.0979365930683194	0.431603533204733	0.644996929295962\\
0.68	0.100266436456264	0.436439588755293	0.643564622546848\\
0.68	0.102624350004627	0.441267427752585	0.64214663037443\\
0.68	0.105010304312169	0.446086812865616	0.640742567819433\\
0.68	0.107424267371012	0.450897506670668	0.63935210119373\\
0.68	0.109866204559871	0.455699271686218	0.637974939582443\\
0.68	0.112336078637817	0.460491870408058	0.636610827579861\\
0.68	0.114833849738607	0.465275065344603	0.635259539104382\\
0.68	0.117359475365564	0.470048619052377	0.633920872153155\\
0.68	0.119912910387023	0.474812294171664	0.632594644371965\\
0.68	0.122494107032347	0.479565853462319	0.631280689329641\\
0.68	0.125103014888515	0.484309059839721	0.629978853399059\\
0.68	0.127739580897283	0.489041676410868	0.628690460824557\\
0.685	0	0	0.710202759607373\\
0.685	1.11327767495586e-05	0.00471862581271275	0.713003157837853\\
0.685	4.46175251031912e-05	0.009446325184051	0.715846486660142\\
0.685	0.000100583311362513	0.0141829653360114	0.718788838379794\\
0.685	0.000179158434668431	0.018928411755669	0.721886092228356\\
0.685	0.000280470402701511	0.02368252819604	0.725200636220892\\
0.685	0.000404645907256436	0.0284451766772965	0.728799085274723\\
0.685	0.000551810799695644	0.0332162174883388	0.732749123983336\\
0.685	0.000722090066287311	0.037995509188729	0.737115575253327\\
0.685	0.000915607803432999	0.0427829086109896	0.741955893907011\\
0.685	0.0011324871927904	0.0475782708632729	0.747315373652289\\
0.685	0.00137285047629673	0.0523814493324038	0.753222424045811\\
0.685	0.00163681893109844	0.057192295687301	0.759684310525652\\
0.685	0.00192451284439304	0.0620106598827802	0.766683748102665\\
0.685	0.00223605148818898	0.0668363901637434	0.774176695511257\\
0.685	0.00257155309398959	0.0716693330697584	0.782091614494493\\
0.685	0.00293113482740716	0.0765093334400312	0.790330346421089\\
0.685	0.00331491276271365	0.0813562344187764	0.79877062752665\\
0.685	0.00372300185733413	0.0862098774609879	0.807270128930359\\
0.685	0.00415551592628969	0.0910701023386146	0.815671782700193\\
0.685	0.00461256761659621	0.0959367471471424	0.82381005350733\\
0.685	0.00509426838162598	0.100809648312589	0.831517746503354\\
0.685	0.00560072845543873	0.105688640598912	0.838632911419451\\
0.685	0.00613205682708918	0.11057355711583	0.845005411376294\\
0.685	0.00668836121491815	0.115464229327074	0.850502769049302\\
0.685	0.00726974804083431	0.120360487059049	0.855014975717657\\
0.685	0.00787632240459385	0.125262158509929	0.858458041096606\\
0.685	0.00850818805808555	0.13016907025918	0.860776163494183\\
0.685	0.00916544737962849	0.135081047277508	0.861942500814224\\
0.685	0.00984820134829017	0.139997912937243	0.861958614648421\\
0.685	0.0105565495182326	0.144919489023162	0.86085273561709\\
0.685	0.0112905899930939	0.149845595743738	0.858677054109649\\
0.685	0.012050419400414	0.154776051742839	0.855504274967456\\
0.685	0.0128361328661109	0.159710674111862	0.851423687967753\\
0.685	0.0136478239890177	0.164649278402306	0.84653700048434\\
0.685	0.0144855848154856	0.169591678638796	0.840954157904248\\
0.685	0.0153495058140643	0.174537687332543	0.834789345403936\\
0.685	0.0162396758502651	0.179487115495259	0.828157325806202\\
0.685	0.0171561821614173	0.184439772653511	0.821170226448692\\
0.685	0.0180991103316243	0.189395466863524	0.813934846725691\\
0.685	0.0190685442668299	0.194354004726437	0.806550519904213\\
0.685	0.0200645661700016	0.199315191403999	0.799107529846305\\
0.685	0.0210872565164405	0.204278830634725	0.791686056496193\\
0.685	0.0221366940292259	0.20924472475049	0.784355603830604\\
0.685	0.023212955654804	0.214212674693577	0.777174850281119\\
0.685	0.0243161165387281	0.219182480034174	0.770191853862643\\
0.685	0.025446250001561	0.224153938988321	0.763444541554458\\
0.685	0.0266034275149466	0.229126848436298	0.756961413910483\\
0.685	0.0277877186778607	0.234101003941464	0.750762400416763\\
0.685	0.0289991911930496	0.239076199769547	0.744859807805784\\
0.685	0.0302379108436654	0.244052228908366	0.73925931151808\\
0.685	0.0315039414701067	0.24902888308801	0.733960949044502\\
0.685	0.0327973449470747	0.254005952801449	0.728960082407262\\
0.685	0.0341181811608519	0.258983227325592	0.724248305111179\\
0.685	0.0354665079868145	0.263960494742775	0.71981427622068\\
0.685	0.0368423812671862	0.268937541962694	0.715644470612629\\
0.685	0.0382458547890421	0.273914154744766	0.711723839838043\\
0.685	0.0396769802625738	0.278890117720924	0.708036382389731\\
0.685	0.0411358072996216	0.283865214418836	0.704565625567163\\
0.685	0.0426223833924861	0.288839227285554	0.701295023640316\\
0.685	0.0441367538930258	0.293811937711588	0.698208278749355\\
0.685	0.0456789619920503	0.298783126055387	0.69528959205535\\
0.685	0.0472490486990192	0.303752571668251	0.692523853197511\\
0.685	0.0488470528220537	0.308720052919643	0.689896776228118\\
0.685	0.0504730109482718	0.313685347222913	0.687394989990281\\
0.685	0.0521269574244539	0.318648231061428	0.68500609046598\\
0.685	0.0538089243380495	0.323608480015096	0.682718662029127\\
0.685	0.0555189414985328	0.328565868787292	0.680522273853471\\
0.685	0.0572570364191156	0.333520171232163	0.67840745699785\\
0.685	0.0590232342988274	0.338471160382329	0.676365666960044\\
0.685	0.0608175580049697	0.34341860847696	0.674389235783157\\
0.685	0.0626400280559541	0.348362286990222	0.672471317134861\\
0.685	0.064490662604533	0.3533019666601	0.670605827172181\\
0.685	0.0663694774214291	0.358237417517579	0.668787383459804\\
0.685	0.0682764858793752	0.363168408916186	0.667011243730728\\
0.685	0.0702116989375697	0.368094709561873	0.665273245863649\\
0.685	0.0721751251265572	0.373016087543259	0.663569750098933\\
0.685	0.0741667705335422	0.377932310362199	0.661897584220059\\
0.685	0.0761866387881432	0.382843144964686	0.660253992184442\\
0.685	0.0782347310485954	0.38774835777208	0.658636586490997\\
0.685	0.0803110459884098	0.392647714712651	0.657043304415872\\
0.685	0.0824155797834956	0.397540981253432	0.655472368126509\\
0.685	0.084548326099754	0.402427922432375	0.653922248592745\\
0.685	0.0867092760811502	0.4073083028908	0.652391633146571\\
0.685	0.0888984183382709	0.412181886906127	0.650879396495579\\
0.685	0.0911157389373742	0.417048438424896	0.649384574964845\\
0.685	0.0933612213899392	0.421907721096044	0.647906343724684\\
0.685	0.0956348466427212	0.42675949830445	0.646443996754805\\
0.685	0.0979365930683194	0.431603533204733	0.644996929295961\\
0.685	0.100266436456264	0.436439588755293	0.643564622546847\\
0.685	0.102624350004627	0.441267427752584	0.64214663037443\\
0.685	0.105010304312169	0.446086812865616	0.640742567819433\\
0.685	0.107424267371012	0.450897506670668	0.639352101193731\\
0.685	0.109866204559871	0.455699271686218	0.637974939582444\\
0.685	0.112336078637817	0.460491870408058	0.636610827579862\\
0.685	0.114833849738607	0.465275065344603	0.635259539104381\\
0.685	0.117359475365564	0.470048619052377	0.633920872153155\\
0.685	0.119912910387022	0.474812294171664	0.632594644371965\\
0.685	0.122494107032347	0.479565853462319	0.631280689329641\\
0.685	0.125103014888515	0.484309059839721	0.629978853399059\\
0.685	0.127739580897283	0.489041676410868	0.628690460824555\\
0.69	0	0	0.710202759607373\\
0.69	1.11327767495586e-05	0.00471862581271275	0.713003157837853\\
0.69	4.46175251031912e-05	0.009446325184051	0.715846486660142\\
0.69	0.000100583311362513	0.0141829653360114	0.718788838379794\\
0.69	0.000179158434668431	0.018928411755669	0.721886092228356\\
0.69	0.000280470402701511	0.02368252819604	0.725200636220892\\
0.69	0.000404645907256436	0.0284451766772965	0.728799085274723\\
0.69	0.000551810799695644	0.0332162174883389	0.732749123983336\\
0.69	0.000722090066287311	0.037995509188729	0.737115575253328\\
0.69	0.000915607803432999	0.0427829086109896	0.741955893907011\\
0.69	0.0011324871927904	0.0475782708632729	0.747315373652289\\
0.69	0.00137285047629673	0.0523814493324038	0.75322242404581\\
0.69	0.00163681893109844	0.057192295687301	0.759684310525653\\
0.69	0.00192451284439304	0.0620106598827802	0.766683748102665\\
0.69	0.00223605148818899	0.0668363901637434	0.774176695511256\\
0.69	0.00257155309398959	0.0716693330697584	0.782091614494494\\
0.69	0.00293113482740716	0.0765093334400312	0.79033034642109\\
0.69	0.00331491276271365	0.0813562344187764	0.79877062752665\\
0.69	0.00372300185733413	0.0862098774609879	0.80727012893036\\
0.69	0.00415551592628969	0.0910701023386145	0.815671782700193\\
0.69	0.00461256761659621	0.0959367471471425	0.82381005350733\\
0.69	0.00509426838162598	0.100809648312589	0.831517746503354\\
0.69	0.00560072845543873	0.105688640598912	0.838632911419452\\
0.69	0.00613205682708918	0.11057355711583	0.845005411376293\\
0.69	0.00668836121491816	0.115464229327074	0.850502769049302\\
0.69	0.00726974804083431	0.120360487059049	0.855014975717657\\
0.69	0.00787632240459385	0.125262158509929	0.858458041096606\\
0.69	0.00850818805808555	0.13016907025918	0.860776163494182\\
0.69	0.00916544737962849	0.135081047277508	0.861942500814225\\
0.69	0.00984820134829017	0.139997912937243	0.861958614648422\\
0.69	0.0105565495182326	0.144919489023162	0.86085273561709\\
0.69	0.0112905899930939	0.149845595743738	0.858677054109649\\
0.69	0.012050419400414	0.154776051742839	0.855504274967456\\
0.69	0.0128361328661109	0.159710674111862	0.851423687967756\\
0.69	0.0136478239890177	0.164649278402306	0.84653700048434\\
0.69	0.0144855848154856	0.169591678638796	0.840954157904248\\
0.69	0.0153495058140643	0.174537687332543	0.834789345403936\\
0.69	0.0162396758502651	0.179487115495259	0.8281573258062\\
0.69	0.0171561821614173	0.184439772653511	0.821170226448692\\
0.69	0.0180991103316243	0.189395466863524	0.813934846725693\\
0.69	0.0190685442668299	0.194354004726437	0.806550519904214\\
0.69	0.0200645661700016	0.199315191403999	0.799107529846305\\
0.69	0.0210872565164405	0.204278830634725	0.791686056496194\\
0.69	0.0221366940292259	0.20924472475049	0.784355603830603\\
0.69	0.023212955654804	0.214212674693577	0.777174850281117\\
0.69	0.0243161165387281	0.219182480034174	0.770191853862642\\
0.69	0.025446250001561	0.224153938988321	0.763444541554457\\
0.69	0.0266034275149466	0.229126848436298	0.756961413910482\\
0.69	0.0277877186778607	0.234101003941464	0.750762400416764\\
0.69	0.0289991911930496	0.239076199769547	0.744859807805786\\
0.69	0.0302379108436654	0.244052228908366	0.73925931151808\\
0.69	0.0315039414701067	0.24902888308801	0.733960949044504\\
0.69	0.0327973449470747	0.254005952801449	0.728960082407259\\
0.69	0.0341181811608519	0.258983227325591	0.724248305111177\\
0.69	0.0354665079868145	0.263960494742775	0.719814276220683\\
0.69	0.0368423812671862	0.268937541962694	0.71564447061263\\
0.69	0.0382458547890421	0.273914154744766	0.711723839838042\\
0.69	0.0396769802625738	0.278890117720924	0.70803638238973\\
0.69	0.0411358072996216	0.283865214418836	0.704565625567162\\
0.69	0.0426223833924861	0.288839227285554	0.701295023640317\\
0.69	0.0441367538930258	0.293811937711588	0.698208278749355\\
0.69	0.0456789619920503	0.298783126055387	0.695289592055349\\
0.69	0.0472490486990192	0.303752571668251	0.692523853197512\\
0.69	0.0488470528220538	0.308720052919643	0.689896776228119\\
0.69	0.0504730109482718	0.313685347222913	0.687394989990282\\
0.69	0.0521269574244539	0.318648231061428	0.685006090465981\\
0.69	0.0538089243380495	0.323608480015097	0.682718662029127\\
0.69	0.0555189414985328	0.328565868787292	0.680522273853468\\
0.69	0.0572570364191156	0.333520171232163	0.67840745699785\\
0.69	0.0590232342988274	0.338471160382329	0.676365666960044\\
0.69	0.0608175580049697	0.34341860847696	0.674389235783156\\
0.69	0.0626400280559541	0.348362286990222	0.672471317134863\\
0.69	0.064490662604533	0.3533019666601	0.670605827172183\\
0.69	0.0663694774214291	0.358237417517579	0.668787383459804\\
0.69	0.0682764858793752	0.363168408916186	0.667011243730728\\
0.69	0.0702116989375697	0.368094709561873	0.665273245863649\\
0.69	0.0721751251265572	0.373016087543259	0.663569750098932\\
0.69	0.0741667705335422	0.377932310362199	0.661897584220061\\
0.69	0.0761866387881432	0.382843144964686	0.660253992184441\\
0.69	0.0782347310485954	0.38774835777208	0.658636586490996\\
0.69	0.0803110459884098	0.392647714712651	0.657043304415871\\
0.69	0.0824155797834956	0.397540981253432	0.65547236812651\\
0.69	0.084548326099754	0.402427922432375	0.653922248592744\\
0.69	0.0867092760811502	0.407308302890799	0.65239163314657\\
0.69	0.0888984183382709	0.412181886906127	0.650879396495579\\
0.69	0.0911157389373742	0.417048438424896	0.649384574964844\\
0.69	0.0933612213899392	0.421907721096044	0.647906343724686\\
0.69	0.0956348466427212	0.42675949830445	0.646443996754806\\
0.69	0.0979365930683194	0.431603533204733	0.644996929295961\\
0.69	0.100266436456264	0.436439588755293	0.643564622546846\\
0.69	0.102624350004627	0.441267427752585	0.642146630374429\\
0.69	0.105010304312169	0.446086812865616	0.640742567819433\\
0.69	0.107424267371012	0.450897506670668	0.63935210119373\\
0.69	0.109866204559871	0.455699271686218	0.637974939582442\\
0.69	0.112336078637817	0.460491870408058	0.636610827579863\\
0.69	0.114833849738607	0.465275065344603	0.635259539104384\\
0.69	0.117359475365564	0.470048619052377	0.633920872153155\\
0.69	0.119912910387023	0.474812294171664	0.632594644371964\\
0.69	0.122494107032347	0.479565853462319	0.63128068932964\\
0.69	0.125103014888515	0.484309059839721	0.62997885339906\\
0.69	0.127739580897283	0.489041676410868	0.628690460824558\\
0.695	0	0	0.710202759607373\\
0.695	1.11327767495586e-05	0.00471862581271275	0.713003157837853\\
0.695	4.46175251031912e-05	0.009446325184051	0.715846486660142\\
0.695	0.000100583311362513	0.0141829653360114	0.718788838379794\\
0.695	0.000179158434668431	0.018928411755669	0.721886092228356\\
0.695	0.000280470402701511	0.02368252819604	0.725200636220892\\
0.695	0.000404645907256436	0.0284451766772965	0.728799085274723\\
0.695	0.000551810799695644	0.0332162174883388	0.732749123983336\\
0.695	0.000722090066287311	0.037995509188729	0.737115575253327\\
0.695	0.000915607803432999	0.0427829086109896	0.741955893907011\\
0.695	0.0011324871927904	0.0475782708632729	0.747315373652289\\
0.695	0.00137285047629673	0.0523814493324038	0.753222424045811\\
0.695	0.00163681893109844	0.057192295687301	0.759684310525652\\
0.695	0.00192451284439304	0.0620106598827802	0.766683748102665\\
0.695	0.00223605148818899	0.0668363901637434	0.774176695511257\\
0.695	0.00257155309398959	0.0716693330697584	0.782091614494493\\
0.695	0.00293113482740716	0.0765093334400312	0.79033034642109\\
0.695	0.00331491276271365	0.0813562344187764	0.79877062752665\\
0.695	0.00372300185733413	0.0862098774609879	0.807270128930359\\
0.695	0.00415551592628969	0.0910701023386145	0.815671782700194\\
0.695	0.00461256761659621	0.0959367471471425	0.82381005350733\\
0.695	0.00509426838162598	0.100809648312589	0.831517746503354\\
0.695	0.00560072845543873	0.105688640598912	0.838632911419451\\
0.695	0.00613205682708918	0.11057355711583	0.845005411376292\\
0.695	0.00668836121491815	0.115464229327074	0.850502769049302\\
0.695	0.00726974804083431	0.120360487059049	0.855014975717659\\
0.695	0.00787632240459385	0.125262158509929	0.858458041096605\\
0.695	0.00850818805808555	0.13016907025918	0.860776163494181\\
0.695	0.00916544737962849	0.135081047277508	0.861942500814226\\
0.695	0.00984820134829017	0.139997912937243	0.86195861464842\\
0.695	0.0105565495182326	0.144919489023162	0.860852735617091\\
0.695	0.0112905899930939	0.149845595743738	0.858677054109651\\
0.695	0.012050419400414	0.154776051742839	0.85550427496746\\
0.695	0.0128361328661109	0.159710674111862	0.851423687967755\\
0.695	0.0136478239890177	0.164649278402306	0.846537000484341\\
0.695	0.0144855848154856	0.169591678638796	0.84095415790425\\
0.695	0.0153495058140643	0.174537687332543	0.834789345403935\\
0.695	0.0162396758502651	0.179487115495259	0.828157325806199\\
0.695	0.0171561821614173	0.184439772653511	0.821170226448694\\
0.695	0.0180991103316243	0.189395466863524	0.813934846725695\\
0.695	0.0190685442668299	0.194354004726437	0.806550519904213\\
0.695	0.0200645661700016	0.199315191403999	0.799107529846303\\
0.695	0.0210872565164405	0.204278830634725	0.791686056496195\\
0.695	0.0221366940292259	0.20924472475049	0.784355603830601\\
0.695	0.023212955654804	0.214212674693577	0.777174850281117\\
0.695	0.0243161165387281	0.219182480034174	0.770191853862644\\
0.695	0.025446250001561	0.224153938988321	0.763444541554459\\
0.695	0.0266034275149466	0.229126848436298	0.756961413910484\\
0.695	0.0277877186778607	0.234101003941464	0.750762400416766\\
0.695	0.0289991911930496	0.239076199769547	0.744859807805785\\
0.695	0.0302379108436654	0.244052228908366	0.739259311518079\\
0.695	0.0315039414701067	0.24902888308801	0.733960949044502\\
0.695	0.0327973449470747	0.254005952801449	0.728960082407262\\
0.695	0.0341181811608519	0.258983227325592	0.724248305111179\\
0.695	0.0354665079868145	0.263960494742775	0.71981427622068\\
0.695	0.0368423812671862	0.268937541962694	0.71564447061263\\
0.695	0.0382458547890421	0.273914154744766	0.711723839838041\\
0.695	0.0396769802625738	0.278890117720924	0.70803638238973\\
0.695	0.0411358072996216	0.283865214418836	0.704565625567164\\
0.695	0.0426223833924861	0.288839227285554	0.701295023640316\\
0.695	0.0441367538930258	0.293811937711588	0.698208278749353\\
0.695	0.0456789619920503	0.298783126055387	0.69528959205535\\
0.695	0.0472490486990192	0.303752571668251	0.692523853197512\\
0.695	0.0488470528220538	0.308720052919643	0.689896776228118\\
0.695	0.0504730109482718	0.313685347222913	0.687394989990282\\
0.695	0.0521269574244539	0.318648231061428	0.685006090465981\\
0.695	0.0538089243380495	0.323608480015096	0.682718662029127\\
0.695	0.0555189414985328	0.328565868787292	0.680522273853471\\
0.695	0.0572570364191156	0.333520171232163	0.678407456997849\\
0.695	0.0590232342988274	0.338471160382329	0.676365666960042\\
0.695	0.0608175580049697	0.34341860847696	0.674389235783155\\
0.695	0.0626400280559541	0.348362286990221	0.672471317134861\\
0.695	0.064490662604533	0.3533019666601	0.670605827172184\\
0.695	0.066369477421429	0.358237417517579	0.668787383459805\\
0.695	0.0682764858793752	0.363168408916186	0.667011243730727\\
0.695	0.0702116989375697	0.368094709561873	0.66527324586365\\
0.695	0.0721751251265572	0.373016087543259	0.663569750098933\\
0.695	0.0741667705335422	0.377932310362199	0.66189758422006\\
0.695	0.0761866387881432	0.382843144964686	0.660253992184443\\
0.695	0.0782347310485954	0.38774835777208	0.658636586490998\\
0.695	0.0803110459884098	0.392647714712651	0.657043304415871\\
0.695	0.0824155797834956	0.397540981253432	0.655472368126509\\
0.695	0.084548326099754	0.402427922432375	0.653922248592746\\
0.695	0.0867092760811502	0.407308302890799	0.652391633146568\\
0.695	0.0888984183382709	0.412181886906127	0.65087939649558\\
0.695	0.0911157389373742	0.417048438424896	0.649384574964843\\
0.695	0.0933612213899392	0.421907721096044	0.647906343724684\\
0.695	0.0956348466427212	0.42675949830445	0.646443996754807\\
0.695	0.0979365930683194	0.431603533204733	0.644996929295963\\
0.695	0.100266436456264	0.436439588755293	0.643564622546847\\
0.695	0.102624350004627	0.441267427752584	0.64214663037443\\
0.695	0.105010304312169	0.446086812865616	0.640742567819435\\
0.695	0.107424267371012	0.450897506670668	0.639352101193731\\
0.695	0.109866204559871	0.455699271686218	0.637974939582441\\
0.695	0.112336078637817	0.460491870408058	0.636610827579861\\
0.695	0.114833849738607	0.465275065344603	0.635259539104383\\
0.695	0.117359475365564	0.470048619052377	0.633920872153155\\
0.695	0.119912910387022	0.474812294171664	0.632594644371966\\
0.695	0.122494107032347	0.479565853462319	0.63128068932964\\
0.695	0.125103014888515	0.484309059839721	0.629978853399059\\
0.695	0.127739580897283	0.489041676410868	0.628690460824563\\
0.7	0	0	0.710202759607373\\
0.7	1.11327767495586e-05	0.00471862581271274	0.713003157837853\\
0.7	4.46175251031912e-05	0.009446325184051	0.715846486660142\\
0.7	0.000100583311362513	0.0141829653360114	0.718788838379794\\
0.7	0.000179158434668431	0.018928411755669	0.721886092228356\\
0.7	0.000280470402701511	0.02368252819604	0.725200636220892\\
0.7	0.000404645907256436	0.0284451766772965	0.728799085274723\\
0.7	0.000551810799695644	0.0332162174883388	0.732749123983336\\
0.7	0.000722090066287311	0.037995509188729	0.737115575253327\\
0.7	0.000915607803432999	0.0427829086109896	0.741955893907011\\
0.7	0.0011324871927904	0.047578270863273	0.747315373652289\\
0.7	0.00137285047629673	0.0523814493324038	0.75322242404581\\
0.7	0.00163681893109844	0.057192295687301	0.759684310525652\\
0.7	0.00192451284439304	0.0620106598827802	0.766683748102665\\
0.7	0.00223605148818898	0.0668363901637434	0.774176695511256\\
0.7	0.00257155309398959	0.0716693330697584	0.782091614494493\\
0.7	0.00293113482740716	0.0765093334400312	0.79033034642109\\
0.7	0.00331491276271365	0.0813562344187764	0.79877062752665\\
0.7	0.00372300185733414	0.0862098774609879	0.807270128930359\\
0.7	0.00415551592628969	0.0910701023386145	0.815671782700193\\
0.7	0.00461256761659621	0.0959367471471424	0.823810053507331\\
0.7	0.00509426838162598	0.100809648312589	0.831517746503353\\
0.7	0.00560072845543873	0.105688640598912	0.838632911419451\\
0.7	0.00613205682708918	0.11057355711583	0.845005411376291\\
0.7	0.00668836121491816	0.115464229327074	0.850502769049301\\
0.7	0.00726974804083431	0.120360487059049	0.855014975717655\\
0.7	0.00787632240459385	0.125262158509929	0.858458041096607\\
0.7	0.00850818805808554	0.13016907025918	0.860776163494183\\
0.7	0.00916544737962849	0.135081047277508	0.861942500814223\\
0.7	0.00984820134829017	0.139997912937243	0.86195861464842\\
0.7	0.0105565495182326	0.144919489023162	0.860852735617091\\
0.7	0.0112905899930939	0.149845595743738	0.85867705410965\\
0.7	0.012050419400414	0.154776051742839	0.855504274967459\\
0.7	0.0128361328661109	0.159710674111862	0.851423687967755\\
0.7	0.0136478239890177	0.164649278402306	0.846537000484339\\
0.7	0.0144855848154856	0.169591678638796	0.840954157904248\\
0.7	0.0153495058140643	0.174537687332543	0.834789345403936\\
0.7	0.0162396758502651	0.179487115495259	0.828157325806199\\
0.7	0.0171561821614173	0.184439772653511	0.821170226448695\\
0.7	0.0180991103316243	0.189395466863524	0.813934846725692\\
0.7	0.0190685442668299	0.194354004726437	0.806550519904212\\
0.7	0.0200645661700016	0.199315191403999	0.799107529846304\\
0.7	0.0210872565164405	0.204278830634725	0.791686056496196\\
0.7	0.0221366940292259	0.20924472475049	0.7843556038306\\
0.7	0.023212955654804	0.214212674693577	0.777174850281119\\
0.7	0.0243161165387281	0.219182480034174	0.770191853862643\\
0.7	0.025446250001561	0.224153938988321	0.763444541554459\\
0.7	0.0266034275149466	0.229126848436298	0.756961413910484\\
0.7	0.0277877186778607	0.234101003941464	0.750762400416763\\
0.7	0.0289991911930496	0.239076199769547	0.744859807805784\\
0.7	0.0302379108436654	0.244052228908366	0.739259311518079\\
0.7	0.0315039414701067	0.24902888308801	0.733960949044503\\
0.7	0.0327973449470747	0.254005952801449	0.728960082407261\\
0.7	0.0341181811608519	0.258983227325592	0.724248305111176\\
0.7	0.0354665079868145	0.263960494742775	0.719814276220681\\
0.7	0.0368423812671862	0.268937541962694	0.715644470612629\\
0.7	0.0382458547890421	0.273914154744766	0.711723839838042\\
0.7	0.0396769802625738	0.278890117720924	0.708036382389731\\
0.7	0.0411358072996216	0.283865214418836	0.704565625567165\\
0.7	0.0426223833924861	0.288839227285554	0.701295023640316\\
0.7	0.0441367538930258	0.293811937711588	0.698208278749353\\
0.7	0.0456789619920503	0.298783126055387	0.69528959205535\\
0.7	0.0472490486990192	0.303752571668251	0.692523853197512\\
0.7	0.0488470528220538	0.308720052919643	0.689896776228119\\
0.7	0.0504730109482718	0.313685347222913	0.687394989990281\\
0.7	0.0521269574244539	0.318648231061428	0.685006090465981\\
0.7	0.0538089243380495	0.323608480015097	0.682718662029127\\
0.7	0.0555189414985328	0.328565868787292	0.680522273853469\\
0.7	0.0572570364191156	0.333520171232163	0.678407456997851\\
0.7	0.0590232342988274	0.338471160382329	0.676365666960044\\
0.7	0.0608175580049697	0.34341860847696	0.674389235783156\\
0.7	0.0626400280559541	0.348362286990222	0.67247131713486\\
0.7	0.064490662604533	0.3533019666601	0.670605827172182\\
0.7	0.066369477421429	0.358237417517579	0.668787383459805\\
0.7	0.0682764858793752	0.363168408916186	0.667011243730727\\
0.7	0.0702116989375697	0.368094709561873	0.665273245863647\\
0.7	0.0721751251265572	0.373016087543259	0.663569750098933\\
0.7	0.0741667705335422	0.377932310362199	0.66189758422006\\
0.7	0.0761866387881432	0.382843144964686	0.660253992184442\\
0.7	0.0782347310485954	0.38774835777208	0.658636586490997\\
0.7	0.0803110459884098	0.392647714712651	0.657043304415872\\
0.7	0.0824155797834956	0.397540981253432	0.655472368126509\\
0.7	0.084548326099754	0.402427922432375	0.653922248592746\\
0.7	0.0867092760811502	0.4073083028908	0.65239163314657\\
0.7	0.0888984183382709	0.412181886906127	0.650879396495578\\
0.7	0.0911157389373742	0.417048438424897	0.649384574964844\\
0.7	0.0933612213899392	0.421907721096044	0.647906343724683\\
0.7	0.0956348466427212	0.42675949830445	0.646443996754805\\
0.7	0.0979365930683194	0.431603533204733	0.644996929295961\\
0.7	0.100266436456264	0.436439588755293	0.643564622546848\\
0.7	0.102624350004627	0.441267427752584	0.64214663037443\\
0.7	0.105010304312169	0.446086812865616	0.640742567819433\\
0.7	0.107424267371012	0.450897506670668	0.639352101193732\\
0.7	0.109866204559871	0.455699271686218	0.637974939582444\\
0.7	0.112336078637817	0.460491870408058	0.636610827579861\\
0.7	0.114833849738607	0.465275065344603	0.635259539104383\\
0.7	0.117359475365564	0.470048619052377	0.633920872153156\\
0.7	0.119912910387023	0.474812294171664	0.632594644371964\\
0.7	0.122494107032347	0.479565853462319	0.631280689329641\\
0.7	0.125103014888515	0.484309059839721	0.629978853399058\\
0.7	0.127739580897283	0.489041676410868	0.628690460824557\\
0.705	0	0	0.710202759607373\\
0.705	1.11327767495586e-05	0.00471862581271275	0.713003157837853\\
0.705	4.46175251031912e-05	0.009446325184051	0.715846486660142\\
0.705	0.000100583311362513	0.0141829653360114	0.718788838379794\\
0.705	0.000179158434668431	0.018928411755669	0.721886092228356\\
0.705	0.000280470402701511	0.02368252819604	0.725200636220892\\
0.705	0.000404645907256436	0.0284451766772965	0.728799085274723\\
0.705	0.000551810799695644	0.0332162174883389	0.732749123983336\\
0.705	0.000722090066287311	0.037995509188729	0.737115575253328\\
0.705	0.000915607803432999	0.0427829086109896	0.741955893907011\\
0.705	0.0011324871927904	0.0475782708632729	0.747315373652289\\
0.705	0.00137285047629673	0.0523814493324038	0.753222424045811\\
0.705	0.00163681893109844	0.057192295687301	0.759684310525652\\
0.705	0.00192451284439304	0.0620106598827802	0.766683748102665\\
0.705	0.00223605148818898	0.0668363901637434	0.774176695511257\\
0.705	0.00257155309398959	0.0716693330697584	0.782091614494494\\
0.705	0.00293113482740716	0.0765093334400312	0.79033034642109\\
0.705	0.00331491276271365	0.0813562344187764	0.79877062752665\\
0.705	0.00372300185733413	0.0862098774609879	0.807270128930359\\
0.705	0.00415551592628969	0.0910701023386145	0.815671782700194\\
0.705	0.00461256761659621	0.0959367471471425	0.823810053507331\\
0.705	0.00509426838162598	0.100809648312589	0.831517746503353\\
0.705	0.00560072845543873	0.105688640598912	0.838632911419452\\
0.705	0.00613205682708918	0.11057355711583	0.845005411376293\\
0.705	0.00668836121491816	0.115464229327074	0.850502769049302\\
0.705	0.00726974804083431	0.120360487059049	0.855014975717657\\
0.705	0.00787632240459385	0.125262158509929	0.858458041096605\\
0.705	0.00850818805808554	0.13016907025918	0.860776163494182\\
0.705	0.00916544737962849	0.135081047277508	0.861942500814222\\
0.705	0.00984820134829017	0.139997912937243	0.861958614648418\\
0.705	0.0105565495182326	0.144919489023162	0.86085273561709\\
0.705	0.0112905899930939	0.149845595743738	0.85867705410965\\
0.705	0.012050419400414	0.154776051742839	0.855504274967455\\
0.705	0.0128361328661109	0.159710674111862	0.851423687967754\\
0.705	0.0136478239890177	0.164649278402306	0.846537000484341\\
0.705	0.0144855848154856	0.169591678638796	0.840954157904248\\
0.705	0.0153495058140643	0.174537687332543	0.834789345403936\\
0.705	0.0162396758502651	0.179487115495259	0.8281573258062\\
0.705	0.0171561821614173	0.184439772653511	0.821170226448695\\
0.705	0.0180991103316243	0.189395466863524	0.813934846725694\\
0.705	0.0190685442668299	0.194354004726437	0.806550519904211\\
0.705	0.0200645661700016	0.199315191403999	0.799107529846304\\
0.705	0.0210872565164405	0.204278830634725	0.791686056496193\\
0.705	0.0221366940292259	0.20924472475049	0.7843556038306\\
0.705	0.023212955654804	0.214212674693577	0.777174850281117\\
0.705	0.0243161165387281	0.219182480034174	0.770191853862644\\
0.705	0.025446250001561	0.224153938988321	0.76344454155446\\
0.705	0.0266034275149466	0.229126848436298	0.756961413910482\\
0.705	0.0277877186778607	0.234101003941464	0.750762400416763\\
0.705	0.0289991911930496	0.239076199769547	0.744859807805786\\
0.705	0.0302379108436654	0.244052228908366	0.739259311518079\\
0.705	0.0315039414701067	0.24902888308801	0.733960949044504\\
0.705	0.0327973449470747	0.254005952801449	0.72896008240726\\
0.705	0.0341181811608519	0.258983227325591	0.724248305111178\\
0.705	0.0354665079868145	0.263960494742775	0.719814276220683\\
0.705	0.0368423812671862	0.268937541962694	0.71564447061263\\
0.705	0.0382458547890421	0.273914154744766	0.711723839838041\\
0.705	0.0396769802625738	0.278890117720924	0.70803638238973\\
0.705	0.0411358072996216	0.283865214418836	0.704565625567166\\
0.705	0.0426223833924861	0.288839227285554	0.701295023640316\\
0.705	0.0441367538930258	0.293811937711588	0.698208278749353\\
0.705	0.0456789619920503	0.298783126055387	0.69528959205535\\
0.705	0.0472490486990192	0.303752571668251	0.692523853197512\\
0.705	0.0488470528220538	0.308720052919643	0.689896776228119\\
0.705	0.0504730109482718	0.313685347222913	0.687394989990281\\
0.705	0.0521269574244539	0.318648231061428	0.685006090465981\\
0.705	0.0538089243380495	0.323608480015097	0.682718662029128\\
0.705	0.0555189414985328	0.328565868787292	0.680522273853469\\
0.705	0.0572570364191156	0.333520171232163	0.67840745699785\\
0.705	0.0590232342988274	0.338471160382329	0.676365666960044\\
0.705	0.0608175580049697	0.34341860847696	0.674389235783156\\
0.705	0.0626400280559541	0.348362286990222	0.672471317134862\\
0.705	0.064490662604533	0.3533019666601	0.670605827172182\\
0.705	0.0663694774214291	0.358237417517579	0.668787383459803\\
0.705	0.0682764858793752	0.363168408916186	0.667011243730729\\
0.705	0.0702116989375697	0.368094709561873	0.665273245863649\\
0.705	0.0721751251265572	0.373016087543259	0.663569750098931\\
0.705	0.0741667705335422	0.377932310362199	0.661897584220061\\
0.705	0.0761866387881432	0.382843144964686	0.660253992184442\\
0.705	0.0782347310485954	0.38774835777208	0.658636586490997\\
0.705	0.0803110459884098	0.392647714712651	0.657043304415871\\
0.705	0.0824155797834956	0.397540981253432	0.655472368126509\\
0.705	0.084548326099754	0.402427922432375	0.653922248592746\\
0.705	0.0867092760811502	0.4073083028908	0.65239163314657\\
0.705	0.0888984183382709	0.412181886906127	0.650879396495579\\
0.705	0.0911157389373742	0.417048438424896	0.649384574964845\\
0.705	0.0933612213899392	0.421907721096044	0.647906343724685\\
0.705	0.0956348466427212	0.42675949830445	0.646443996754805\\
0.705	0.0979365930683194	0.431603533204733	0.644996929295962\\
0.705	0.100266436456264	0.436439588755293	0.643564622546846\\
0.705	0.102624350004627	0.441267427752584	0.64214663037443\\
0.705	0.105010304312169	0.446086812865616	0.640742567819433\\
0.705	0.107424267371012	0.450897506670668	0.63935210119373\\
0.705	0.109866204559871	0.455699271686218	0.637974939582444\\
0.705	0.112336078637817	0.460491870408058	0.636610827579863\\
0.705	0.114833849738607	0.465275065344603	0.635259539104381\\
0.705	0.117359475365564	0.470048619052377	0.633920872153154\\
0.705	0.119912910387023	0.474812294171664	0.632594644371966\\
0.705	0.122494107032347	0.479565853462319	0.631280689329641\\
0.705	0.125103014888515	0.484309059839721	0.62997885339906\\
0.705	0.127739580897283	0.489041676410868	0.628690460824556\\
0.71	0	0	0.710202759607373\\
0.71	1.11327767495586e-05	0.00471862581271274	0.713003157837853\\
0.71	4.46175251031912e-05	0.009446325184051	0.715846486660142\\
0.71	0.000100583311362513	0.0141829653360114	0.718788838379794\\
0.71	0.000179158434668431	0.018928411755669	0.721886092228356\\
0.71	0.000280470402701511	0.02368252819604	0.725200636220892\\
0.71	0.000404645907256436	0.0284451766772965	0.728799085274723\\
0.71	0.000551810799695644	0.0332162174883388	0.732749123983336\\
0.71	0.000722090066287311	0.037995509188729	0.737115575253327\\
0.71	0.000915607803432999	0.0427829086109896	0.741955893907011\\
0.71	0.0011324871927904	0.047578270863273	0.747315373652289\\
0.71	0.00137285047629673	0.0523814493324038	0.753222424045811\\
0.71	0.00163681893109844	0.057192295687301	0.759684310525652\\
0.71	0.00192451284439304	0.0620106598827802	0.766683748102665\\
0.71	0.00223605148818898	0.0668363901637434	0.774176695511256\\
0.71	0.00257155309398959	0.0716693330697584	0.782091614494494\\
0.71	0.00293113482740716	0.0765093334400312	0.79033034642109\\
0.71	0.00331491276271365	0.0813562344187764	0.79877062752665\\
0.71	0.00372300185733413	0.0862098774609879	0.80727012893036\\
0.71	0.00415551592628969	0.0910701023386145	0.815671782700194\\
0.71	0.00461256761659621	0.0959367471471425	0.82381005350733\\
0.71	0.00509426838162598	0.100809648312589	0.831517746503354\\
0.71	0.00560072845543873	0.105688640598912	0.838632911419451\\
0.71	0.00613205682708918	0.11057355711583	0.845005411376293\\
0.71	0.00668836121491816	0.115464229327074	0.850502769049302\\
0.71	0.00726974804083431	0.120360487059049	0.855014975717658\\
0.71	0.00787632240459385	0.125262158509929	0.858458041096605\\
0.71	0.00850818805808555	0.13016907025918	0.860776163494182\\
0.71	0.00916544737962849	0.135081047277508	0.861942500814224\\
0.71	0.00984820134829017	0.139997912937243	0.86195861464842\\
0.71	0.0105565495182326	0.144919489023162	0.860852735617091\\
0.71	0.0112905899930939	0.149845595743738	0.858677054109651\\
0.71	0.012050419400414	0.154776051742839	0.855504274967457\\
0.71	0.0128361328661109	0.159710674111862	0.851423687967756\\
0.71	0.0136478239890177	0.164649278402306	0.84653700048434\\
0.71	0.0144855848154856	0.169591678638796	0.84095415790425\\
0.71	0.0153495058140643	0.174537687332543	0.834789345403934\\
0.71	0.0162396758502651	0.179487115495259	0.828157325806202\\
0.71	0.0171561821614173	0.184439772653511	0.821170226448694\\
0.71	0.0180991103316243	0.189395466863524	0.813934846725691\\
0.71	0.0190685442668299	0.194354004726437	0.806550519904208\\
0.71	0.0200645661700016	0.199315191403999	0.799107529846305\\
0.71	0.0210872565164405	0.204278830634725	0.791686056496194\\
0.71	0.0221366940292259	0.20924472475049	0.784355603830602\\
0.71	0.023212955654804	0.214212674693577	0.777174850281119\\
0.71	0.0243161165387281	0.219182480034174	0.770191853862645\\
0.71	0.025446250001561	0.224153938988321	0.763444541554458\\
0.71	0.0266034275149466	0.229126848436298	0.756961413910482\\
0.71	0.0277877186778607	0.234101003941464	0.750762400416765\\
0.71	0.0289991911930496	0.239076199769547	0.744859807805784\\
0.71	0.0302379108436654	0.244052228908366	0.739259311518078\\
0.71	0.0315039414701067	0.24902888308801	0.733960949044503\\
0.71	0.0327973449470747	0.254005952801449	0.728960082407262\\
0.71	0.0341181811608519	0.258983227325592	0.724248305111178\\
0.71	0.0354665079868145	0.263960494742775	0.71981427622068\\
0.71	0.0368423812671862	0.268937541962694	0.71564447061263\\
0.71	0.0382458547890421	0.273914154744766	0.711723839838043\\
0.71	0.0396769802625738	0.278890117720924	0.70803638238973\\
0.71	0.0411358072996216	0.283865214418836	0.704565625567164\\
0.71	0.0426223833924861	0.288839227285554	0.701295023640316\\
0.71	0.0441367538930258	0.293811937711588	0.698208278749355\\
0.71	0.0456789619920503	0.298783126055387	0.69528959205535\\
0.71	0.0472490486990192	0.303752571668251	0.69252385319751\\
0.71	0.0488470528220538	0.308720052919643	0.689896776228117\\
0.71	0.0504730109482718	0.313685347222913	0.687394989990282\\
0.71	0.0521269574244539	0.318648231061428	0.685006090465982\\
0.71	0.0538089243380495	0.323608480015097	0.682718662029128\\
0.71	0.0555189414985328	0.328565868787292	0.680522273853469\\
0.71	0.0572570364191156	0.333520171232163	0.67840745699785\\
0.71	0.0590232342988274	0.338471160382329	0.676365666960045\\
0.71	0.0608175580049697	0.34341860847696	0.674389235783156\\
0.71	0.0626400280559541	0.348362286990222	0.672471317134861\\
0.71	0.064490662604533	0.3533019666601	0.670605827172183\\
0.71	0.0663694774214291	0.358237417517579	0.668787383459804\\
0.71	0.0682764858793752	0.363168408916186	0.667011243730728\\
0.71	0.0702116989375697	0.368094709561873	0.66527324586365\\
0.71	0.0721751251265572	0.373016087543259	0.663569750098932\\
0.71	0.0741667705335422	0.377932310362199	0.66189758422006\\
0.71	0.0761866387881432	0.382843144964686	0.660253992184442\\
0.71	0.0782347310485954	0.38774835777208	0.658636586490997\\
0.71	0.0803110459884098	0.392647714712651	0.657043304415871\\
0.71	0.0824155797834956	0.397540981253432	0.655472368126509\\
0.71	0.084548326099754	0.402427922432375	0.653922248592746\\
0.71	0.0867092760811502	0.407308302890799	0.65239163314657\\
0.71	0.0888984183382709	0.412181886906127	0.650879396495579\\
0.71	0.0911157389373742	0.417048438424896	0.649384574964842\\
0.71	0.0933612213899392	0.421907721096044	0.647906343724685\\
0.71	0.0956348466427212	0.42675949830445	0.646443996754806\\
0.71	0.0979365930683194	0.431603533204733	0.644996929295962\\
0.71	0.100266436456264	0.436439588755293	0.643564622546847\\
0.71	0.102624350004627	0.441267427752585	0.642146630374429\\
0.71	0.105010304312169	0.446086812865616	0.640742567819433\\
0.71	0.107424267371012	0.450897506670668	0.639352101193731\\
0.71	0.109866204559871	0.455699271686218	0.637974939582443\\
0.71	0.112336078637817	0.460491870408058	0.636610827579862\\
0.71	0.114833849738607	0.465275065344603	0.635259539104383\\
0.71	0.117359475365564	0.470048619052377	0.633920872153154\\
0.71	0.119912910387023	0.474812294171664	0.632594644371964\\
0.71	0.122494107032347	0.479565853462319	0.63128068932964\\
0.71	0.125103014888515	0.484309059839721	0.629978853399059\\
0.71	0.127739580897283	0.489041676410868	0.628690460824562\\
0.715	0	0	0.710202759607373\\
0.715	1.11327767495586e-05	0.00471862581271275	0.713003157837853\\
0.715	4.46175251031912e-05	0.009446325184051	0.715846486660142\\
0.715	0.000100583311362513	0.0141829653360114	0.718788838379794\\
0.715	0.000179158434668431	0.018928411755669	0.721886092228357\\
0.715	0.000280470402701511	0.02368252819604	0.725200636220892\\
0.715	0.000404645907256436	0.0284451766772965	0.728799085274723\\
0.715	0.000551810799695644	0.0332162174883388	0.732749123983336\\
0.715	0.000722090066287311	0.037995509188729	0.737115575253328\\
0.715	0.000915607803432999	0.0427829086109896	0.741955893907011\\
0.715	0.0011324871927904	0.047578270863273	0.747315373652289\\
0.715	0.00137285047629673	0.0523814493324038	0.75322242404581\\
0.715	0.00163681893109844	0.057192295687301	0.759684310525652\\
0.715	0.00192451284439304	0.0620106598827802	0.766683748102665\\
0.715	0.00223605148818898	0.0668363901637434	0.774176695511256\\
0.715	0.00257155309398959	0.0716693330697584	0.782091614494493\\
0.715	0.00293113482740716	0.0765093334400312	0.79033034642109\\
0.715	0.00331491276271365	0.0813562344187764	0.79877062752665\\
0.715	0.00372300185733414	0.0862098774609879	0.807270128930359\\
0.715	0.00415551592628969	0.0910701023386145	0.815671782700194\\
0.715	0.00461256761659621	0.0959367471471424	0.823810053507331\\
0.715	0.00509426838162598	0.100809648312589	0.831517746503354\\
0.715	0.00560072845543873	0.105688640598912	0.838632911419452\\
0.715	0.00613205682708918	0.11057355711583	0.845005411376292\\
0.715	0.00668836121491816	0.115464229327074	0.850502769049301\\
0.715	0.00726974804083431	0.120360487059049	0.855014975717657\\
0.715	0.00787632240459385	0.125262158509929	0.858458041096606\\
0.715	0.00850818805808555	0.13016907025918	0.860776163494182\\
0.715	0.00916544737962849	0.135081047277508	0.861942500814226\\
0.715	0.00984820134829017	0.139997912937243	0.861958614648422\\
0.715	0.0105565495182326	0.144919489023162	0.860852735617091\\
0.715	0.0112905899930939	0.149845595743738	0.858677054109652\\
0.715	0.012050419400414	0.154776051742839	0.855504274967458\\
0.715	0.0128361328661109	0.159710674111862	0.851423687967752\\
0.715	0.0136478239890177	0.164649278402306	0.846537000484339\\
0.715	0.0144855848154856	0.169591678638796	0.840954157904248\\
0.715	0.0153495058140643	0.174537687332543	0.834789345403935\\
0.715	0.0162396758502651	0.179487115495259	0.8281573258062\\
0.715	0.0171561821614173	0.184439772653511	0.821170226448694\\
0.715	0.0180991103316243	0.189395466863524	0.81393484672569\\
0.715	0.0190685442668299	0.194354004726437	0.806550519904211\\
0.715	0.0200645661700016	0.199315191403999	0.799107529846308\\
0.715	0.0210872565164405	0.204278830634725	0.791686056496194\\
0.715	0.0221366940292259	0.20924472475049	0.784355603830601\\
0.715	0.023212955654804	0.214212674693577	0.777174850281119\\
0.715	0.0243161165387281	0.219182480034174	0.770191853862644\\
0.715	0.025446250001561	0.224153938988321	0.763444541554457\\
0.715	0.0266034275149466	0.229126848436298	0.756961413910481\\
0.715	0.0277877186778607	0.234101003941464	0.750762400416763\\
0.715	0.0289991911930496	0.239076199769547	0.744859807805784\\
0.715	0.0302379108436654	0.244052228908366	0.73925931151808\\
0.715	0.0315039414701067	0.24902888308801	0.733960949044504\\
0.715	0.0327973449470747	0.254005952801449	0.728960082407261\\
0.715	0.0341181811608519	0.258983227325592	0.724248305111177\\
0.715	0.0354665079868145	0.263960494742775	0.719814276220682\\
0.715	0.0368423812671862	0.268937541962694	0.715644470612631\\
0.715	0.0382458547890421	0.273914154744766	0.711723839838041\\
0.715	0.0396769802625738	0.278890117720924	0.70803638238973\\
0.715	0.0411358072996216	0.283865214418836	0.704565625567166\\
0.715	0.0426223833924861	0.288839227285554	0.701295023640316\\
0.715	0.0441367538930258	0.293811937711588	0.698208278749353\\
0.715	0.0456789619920503	0.298783126055387	0.69528959205535\\
0.715	0.0472490486990192	0.303752571668251	0.692523853197512\\
0.715	0.0488470528220538	0.308720052919643	0.689896776228118\\
0.715	0.0504730109482718	0.313685347222913	0.687394989990281\\
0.715	0.0521269574244539	0.318648231061428	0.685006090465981\\
0.715	0.0538089243380495	0.323608480015097	0.682718662029128\\
0.715	0.0555189414985328	0.328565868787292	0.680522273853469\\
0.715	0.0572570364191156	0.333520171232163	0.678407456997849\\
0.715	0.0590232342988274	0.338471160382329	0.676365666960043\\
0.715	0.0608175580049697	0.34341860847696	0.674389235783157\\
0.715	0.0626400280559541	0.348362286990222	0.672471317134861\\
0.715	0.064490662604533	0.3533019666601	0.670605827172181\\
0.715	0.0663694774214291	0.358237417517579	0.668787383459804\\
0.715	0.0682764858793752	0.363168408916186	0.667011243730728\\
0.715	0.0702116989375697	0.368094709561873	0.665273245863649\\
0.715	0.0721751251265572	0.373016087543259	0.663569750098932\\
0.715	0.0741667705335422	0.377932310362199	0.661897584220061\\
0.715	0.0761866387881432	0.382843144964686	0.660253992184442\\
0.715	0.0782347310485954	0.38774835777208	0.658636586490997\\
0.715	0.0803110459884098	0.392647714712651	0.657043304415871\\
0.715	0.0824155797834956	0.397540981253432	0.655472368126508\\
0.715	0.084548326099754	0.402427922432375	0.653922248592745\\
0.715	0.0867092760811502	0.4073083028908	0.65239163314657\\
0.715	0.0888984183382709	0.412181886906127	0.650879396495578\\
0.715	0.0911157389373742	0.417048438424896	0.649384574964845\\
0.715	0.0933612213899392	0.421907721096044	0.647906343724685\\
0.715	0.0956348466427212	0.42675949830445	0.646443996754804\\
0.715	0.0979365930683194	0.431603533204733	0.644996929295962\\
0.715	0.100266436456264	0.436439588755293	0.643564622546847\\
0.715	0.102624350004627	0.441267427752584	0.64214663037443\\
0.715	0.105010304312169	0.446086812865616	0.640742567819433\\
0.715	0.107424267371012	0.450897506670668	0.639352101193732\\
0.715	0.109866204559871	0.455699271686218	0.637974939582443\\
0.715	0.112336078637817	0.460491870408058	0.636610827579862\\
0.715	0.114833849738607	0.465275065344603	0.635259539104382\\
0.715	0.117359475365564	0.470048619052377	0.633920872153155\\
0.715	0.119912910387023	0.474812294171664	0.632594644371966\\
0.715	0.122494107032347	0.479565853462319	0.631280689329641\\
0.715	0.125103014888515	0.484309059839721	0.629978853399058\\
0.715	0.127739580897283	0.489041676410868	0.628690460824558\\
0.72	0	0	0.710202759607373\\
0.72	1.11327767495586e-05	0.00471862581271275	0.713003157837853\\
0.72	4.46175251031912e-05	0.009446325184051	0.715846486660142\\
0.72	0.000100583311362513	0.0141829653360114	0.718788838379794\\
0.72	0.000179158434668431	0.018928411755669	0.721886092228356\\
0.72	0.000280470402701511	0.02368252819604	0.725200636220892\\
0.72	0.000404645907256436	0.0284451766772965	0.728799085274723\\
0.72	0.000551810799695644	0.0332162174883389	0.732749123983336\\
0.72	0.000722090066287311	0.037995509188729	0.737115575253327\\
0.72	0.000915607803432999	0.0427829086109896	0.741955893907011\\
0.72	0.0011324871927904	0.0475782708632729	0.747315373652289\\
0.72	0.00137285047629673	0.0523814493324038	0.753222424045811\\
0.72	0.00163681893109844	0.057192295687301	0.759684310525652\\
0.72	0.00192451284439304	0.0620106598827802	0.766683748102665\\
0.72	0.00223605148818899	0.0668363901637434	0.774176695511256\\
0.72	0.00257155309398959	0.0716693330697584	0.782091614494493\\
0.72	0.00293113482740716	0.0765093334400312	0.79033034642109\\
0.72	0.00331491276271365	0.0813562344187764	0.79877062752665\\
0.72	0.00372300185733413	0.0862098774609879	0.80727012893036\\
0.72	0.00415551592628969	0.0910701023386145	0.815671782700193\\
0.72	0.00461256761659621	0.0959367471471425	0.82381005350733\\
0.72	0.00509426838162598	0.100809648312589	0.831517746503354\\
0.72	0.00560072845543873	0.105688640598912	0.838632911419452\\
0.72	0.00613205682708918	0.11057355711583	0.845005411376292\\
0.72	0.00668836121491816	0.115464229327074	0.850502769049302\\
0.72	0.00726974804083431	0.120360487059049	0.855014975717657\\
0.72	0.00787632240459385	0.125262158509929	0.858458041096606\\
0.72	0.00850818805808555	0.13016907025918	0.860776163494182\\
0.72	0.00916544737962849	0.135081047277508	0.861942500814223\\
0.72	0.00984820134829017	0.139997912937243	0.86195861464842\\
0.72	0.0105565495182326	0.144919489023162	0.86085273561709\\
0.72	0.0112905899930939	0.149845595743738	0.858677054109649\\
0.72	0.012050419400414	0.154776051742839	0.855504274967456\\
0.72	0.0128361328661109	0.159710674111862	0.851423687967753\\
0.72	0.0136478239890177	0.164649278402306	0.846537000484341\\
0.72	0.0144855848154856	0.169591678638796	0.840954157904249\\
0.72	0.0153495058140643	0.174537687332543	0.834789345403936\\
0.72	0.0162396758502651	0.179487115495259	0.8281573258062\\
0.72	0.0171561821614173	0.184439772653511	0.821170226448695\\
0.72	0.0180991103316243	0.189395466863524	0.813934846725692\\
0.72	0.0190685442668299	0.194354004726437	0.806550519904213\\
0.72	0.0200645661700016	0.199315191403999	0.799107529846305\\
0.72	0.0210872565164405	0.204278830634725	0.791686056496193\\
0.72	0.0221366940292259	0.20924472475049	0.784355603830604\\
0.72	0.023212955654804	0.214212674693577	0.777174850281118\\
0.72	0.0243161165387281	0.219182480034174	0.770191853862641\\
0.72	0.025446250001561	0.224153938988321	0.763444541554457\\
0.72	0.0266034275149466	0.229126848436298	0.756961413910483\\
0.72	0.0277877186778607	0.234101003941464	0.750762400416765\\
0.72	0.0289991911930496	0.239076199769547	0.744859807805785\\
0.72	0.0302379108436654	0.244052228908366	0.739259311518079\\
0.72	0.0315039414701067	0.24902888308801	0.733960949044503\\
0.72	0.0327973449470747	0.254005952801449	0.728960082407262\\
0.72	0.0341181811608519	0.258983227325592	0.724248305111178\\
0.72	0.0354665079868145	0.263960494742775	0.719814276220681\\
0.72	0.0368423812671862	0.268937541962694	0.715644470612629\\
0.72	0.0382458547890421	0.273914154744766	0.711723839838041\\
0.72	0.0396769802625738	0.278890117720924	0.708036382389731\\
0.72	0.0411358072996216	0.283865214418836	0.704565625567164\\
0.72	0.0426223833924861	0.288839227285554	0.701295023640316\\
0.72	0.0441367538930258	0.293811937711588	0.698208278749355\\
0.72	0.0456789619920503	0.298783126055387	0.695289592055349\\
0.72	0.0472490486990192	0.303752571668251	0.692523853197511\\
0.72	0.0488470528220537	0.308720052919643	0.689896776228118\\
0.72	0.0504730109482718	0.313685347222913	0.687394989990281\\
0.72	0.0521269574244539	0.318648231061428	0.685006090465982\\
0.72	0.0538089243380495	0.323608480015097	0.682718662029128\\
0.72	0.0555189414985328	0.328565868787292	0.680522273853469\\
0.72	0.0572570364191156	0.333520171232163	0.678407456997851\\
0.72	0.0590232342988274	0.338471160382329	0.676365666960044\\
0.72	0.0608175580049697	0.34341860847696	0.674389235783156\\
0.72	0.0626400280559541	0.348362286990222	0.67247131713486\\
0.72	0.064490662604533	0.3533019666601	0.670605827172181\\
0.72	0.066369477421429	0.358237417517579	0.668787383459805\\
0.72	0.0682764858793752	0.363168408916186	0.667011243730728\\
0.72	0.0702116989375697	0.368094709561873	0.665273245863648\\
0.72	0.0721751251265572	0.373016087543259	0.663569750098931\\
0.72	0.0741667705335422	0.377932310362199	0.661897584220061\\
0.72	0.0761866387881432	0.382843144964686	0.660253992184443\\
0.72	0.0782347310485954	0.38774835777208	0.658636586490997\\
0.72	0.0803110459884098	0.392647714712651	0.657043304415872\\
0.72	0.0824155797834956	0.397540981253432	0.655472368126511\\
0.72	0.084548326099754	0.402427922432375	0.653922248592744\\
0.72	0.0867092760811502	0.407308302890799	0.652391633146569\\
0.72	0.0888984183382709	0.412181886906127	0.650879396495579\\
0.72	0.0911157389373742	0.417048438424896	0.649384574964843\\
0.72	0.0933612213899392	0.421907721096044	0.647906343724684\\
0.72	0.0956348466427212	0.42675949830445	0.646443996754806\\
0.72	0.0979365930683194	0.431603533204733	0.644996929295963\\
0.72	0.100266436456264	0.436439588755293	0.643564622546846\\
0.72	0.102624350004627	0.441267427752584	0.64214663037443\\
0.72	0.105010304312169	0.446086812865616	0.640742567819432\\
0.72	0.107424267371012	0.450897506670668	0.63935210119373\\
0.72	0.109866204559871	0.455699271686218	0.637974939582443\\
0.72	0.112336078637817	0.460491870408058	0.636610827579862\\
0.72	0.114833849738607	0.465275065344603	0.635259539104383\\
0.72	0.117359475365564	0.470048619052377	0.633920872153154\\
0.72	0.119912910387022	0.474812294171664	0.632594644371965\\
0.72	0.122494107032347	0.479565853462319	0.631280689329641\\
0.72	0.125103014888515	0.484309059839721	0.629978853399058\\
0.72	0.127739580897283	0.489041676410868	0.628690460824559\\
0.725	0	0	0.710202759607373\\
0.725	1.11327767495586e-05	0.00471862581271275	0.713003157837853\\
0.725	4.46175251031912e-05	0.009446325184051	0.715846486660142\\
0.725	0.000100583311362513	0.0141829653360114	0.718788838379794\\
0.725	0.000179158434668431	0.018928411755669	0.721886092228356\\
0.725	0.000280470402701511	0.02368252819604	0.725200636220892\\
0.725	0.000404645907256436	0.0284451766772965	0.728799085274723\\
0.725	0.000551810799695644	0.0332162174883389	0.732749123983336\\
0.725	0.000722090066287311	0.037995509188729	0.737115575253328\\
0.725	0.000915607803432999	0.0427829086109896	0.741955893907011\\
0.725	0.0011324871927904	0.0475782708632729	0.747315373652289\\
0.725	0.00137285047629673	0.0523814493324038	0.753222424045811\\
0.725	0.00163681893109844	0.057192295687301	0.759684310525652\\
0.725	0.00192451284439304	0.0620106598827802	0.766683748102665\\
0.725	0.00223605148818899	0.0668363901637434	0.774176695511257\\
0.725	0.00257155309398959	0.0716693330697584	0.782091614494493\\
0.725	0.00293113482740716	0.0765093334400312	0.79033034642109\\
0.725	0.00331491276271365	0.0813562344187764	0.79877062752665\\
0.725	0.00372300185733413	0.0862098774609879	0.807270128930359\\
0.725	0.00415551592628969	0.0910701023386145	0.815671782700193\\
0.725	0.00461256761659621	0.0959367471471425	0.82381005350733\\
0.725	0.00509426838162598	0.100809648312589	0.831517746503354\\
0.725	0.00560072845543873	0.105688640598912	0.838632911419452\\
0.725	0.00613205682708918	0.11057355711583	0.845005411376293\\
0.725	0.00668836121491816	0.115464229327074	0.850502769049302\\
0.725	0.00726974804083431	0.120360487059049	0.855014975717657\\
0.725	0.00787632240459385	0.125262158509929	0.858458041096607\\
0.725	0.00850818805808554	0.13016907025918	0.860776163494181\\
0.725	0.00916544737962849	0.135081047277508	0.861942500814221\\
0.725	0.00984820134829017	0.139997912937243	0.861958614648419\\
0.725	0.0105565495182326	0.144919489023162	0.860852735617091\\
0.725	0.0112905899930939	0.149845595743738	0.858677054109649\\
0.725	0.012050419400414	0.154776051742839	0.855504274967456\\
0.725	0.0128361328661109	0.159710674111862	0.851423687967755\\
0.725	0.0136478239890177	0.164649278402306	0.84653700048434\\
0.725	0.0144855848154856	0.169591678638796	0.840954157904249\\
0.725	0.0153495058140643	0.174537687332543	0.834789345403935\\
0.725	0.0162396758502651	0.179487115495259	0.8281573258062\\
0.725	0.0171561821614173	0.184439772653511	0.821170226448693\\
0.725	0.0180991103316243	0.189395466863524	0.813934846725693\\
0.725	0.0190685442668299	0.194354004726437	0.806550519904212\\
0.725	0.0200645661700016	0.199315191403999	0.799107529846305\\
0.725	0.0210872565164405	0.204278830634725	0.791686056496196\\
0.725	0.0221366940292259	0.20924472475049	0.784355603830604\\
0.725	0.023212955654804	0.214212674693577	0.777174850281116\\
0.725	0.0243161165387281	0.219182480034174	0.770191853862643\\
0.725	0.025446250001561	0.224153938988321	0.763444541554459\\
0.725	0.0266034275149466	0.229126848436298	0.756961413910485\\
0.725	0.0277877186778607	0.234101003941464	0.750762400416764\\
0.725	0.0289991911930496	0.239076199769547	0.744859807805784\\
0.725	0.0302379108436654	0.244052228908366	0.739259311518079\\
0.725	0.0315039414701067	0.24902888308801	0.733960949044505\\
0.725	0.0327973449470747	0.254005952801449	0.72896008240726\\
0.725	0.0341181811608519	0.258983227325591	0.724248305111179\\
0.725	0.0354665079868145	0.263960494742775	0.719814276220682\\
0.725	0.0368423812671862	0.268937541962694	0.715644470612628\\
0.725	0.0382458547890421	0.273914154744766	0.711723839838043\\
0.725	0.0396769802625738	0.278890117720924	0.708036382389731\\
0.725	0.0411358072996216	0.283865214418836	0.704565625567163\\
0.725	0.0426223833924861	0.288839227285554	0.701295023640316\\
0.725	0.0441367538930258	0.293811937711588	0.698208278749353\\
0.725	0.0456789619920503	0.298783126055387	0.69528959205535\\
0.725	0.0472490486990192	0.303752571668251	0.692523853197513\\
0.725	0.0488470528220538	0.308720052919643	0.689896776228118\\
0.725	0.0504730109482718	0.313685347222913	0.68739498999028\\
0.725	0.0521269574244539	0.318648231061427	0.68500609046598\\
0.725	0.0538089243380495	0.323608480015096	0.682718662029129\\
0.725	0.0555189414985328	0.328565868787292	0.680522273853469\\
0.725	0.0572570364191156	0.333520171232163	0.678407456997849\\
0.725	0.0590232342988274	0.338471160382329	0.676365666960044\\
0.725	0.0608175580049697	0.34341860847696	0.674389235783157\\
0.725	0.0626400280559541	0.348362286990222	0.672471317134863\\
0.725	0.064490662604533	0.3533019666601	0.67060582717218\\
0.725	0.0663694774214291	0.358237417517579	0.668787383459804\\
0.725	0.0682764858793752	0.363168408916186	0.667011243730729\\
0.725	0.0702116989375697	0.368094709561873	0.66527324586365\\
0.725	0.0721751251265572	0.373016087543259	0.663569750098932\\
0.725	0.0741667705335422	0.377932310362199	0.66189758422006\\
0.725	0.0761866387881432	0.382843144964686	0.660253992184442\\
0.725	0.0782347310485954	0.38774835777208	0.658636586490996\\
0.725	0.0803110459884098	0.392647714712651	0.657043304415871\\
0.725	0.0824155797834956	0.397540981253432	0.65547236812651\\
0.725	0.084548326099754	0.402427922432375	0.653922248592747\\
0.725	0.0867092760811502	0.4073083028908	0.65239163314657\\
0.725	0.0888984183382709	0.412181886906127	0.650879396495579\\
0.725	0.0911157389373742	0.417048438424897	0.649384574964845\\
0.725	0.0933612213899392	0.421907721096044	0.647906343724683\\
0.725	0.0956348466427212	0.42675949830445	0.646443996754804\\
0.725	0.0979365930683194	0.431603533204733	0.644996929295962\\
0.725	0.100266436456264	0.436439588755293	0.643564622546848\\
0.725	0.102624350004627	0.441267427752585	0.642146630374429\\
0.725	0.105010304312169	0.446086812865616	0.640742567819434\\
0.725	0.107424267371012	0.450897506670668	0.63935210119373\\
0.725	0.109866204559871	0.455699271686218	0.637974939582442\\
0.725	0.112336078637817	0.460491870408058	0.636610827579861\\
0.725	0.114833849738607	0.465275065344603	0.635259539104383\\
0.725	0.117359475365564	0.470048619052377	0.633920872153156\\
0.725	0.119912910387022	0.474812294171664	0.632594644371964\\
0.725	0.122494107032347	0.479565853462319	0.631280689329641\\
0.725	0.125103014888515	0.484309059839721	0.629978853399058\\
0.725	0.127739580897283	0.489041676410868	0.628690460824556\\
0.73	0	0	0.710202759607373\\
0.73	1.11327767495586e-05	0.00471862581271275	0.713003157837853\\
0.73	4.46175251031912e-05	0.009446325184051	0.715846486660142\\
0.73	0.000100583311362513	0.0141829653360114	0.718788838379794\\
0.73	0.000179158434668431	0.018928411755669	0.721886092228356\\
0.73	0.000280470402701511	0.02368252819604	0.725200636220892\\
0.73	0.000404645907256436	0.0284451766772965	0.728799085274723\\
0.73	0.000551810799695644	0.0332162174883389	0.732749123983336\\
0.73	0.000722090066287311	0.037995509188729	0.737115575253328\\
0.73	0.000915607803432999	0.0427829086109896	0.741955893907011\\
0.73	0.0011324871927904	0.0475782708632729	0.747315373652289\\
0.73	0.00137285047629673	0.0523814493324038	0.753222424045811\\
0.73	0.00163681893109844	0.057192295687301	0.759684310525652\\
0.73	0.00192451284439304	0.0620106598827802	0.766683748102665\\
0.73	0.00223605148818899	0.0668363901637434	0.774176695511256\\
0.73	0.00257155309398959	0.0716693330697584	0.782091614494494\\
0.73	0.00293113482740716	0.0765093334400312	0.79033034642109\\
0.73	0.00331491276271365	0.0813562344187764	0.79877062752665\\
0.73	0.00372300185733414	0.0862098774609879	0.807270128930359\\
0.73	0.00415551592628969	0.0910701023386145	0.815671782700193\\
0.73	0.00461256761659621	0.0959367471471425	0.823810053507331\\
0.73	0.00509426838162598	0.100809648312589	0.831517746503353\\
0.73	0.00560072845543873	0.105688640598912	0.838632911419452\\
0.73	0.00613205682708918	0.11057355711583	0.845005411376293\\
0.73	0.00668836121491816	0.115464229327074	0.850502769049301\\
0.73	0.00726974804083431	0.120360487059049	0.855014975717657\\
0.73	0.00787632240459385	0.125262158509929	0.858458041096608\\
0.73	0.00850818805808555	0.13016907025918	0.860776163494182\\
0.73	0.00916544737962849	0.135081047277508	0.861942500814226\\
0.73	0.00984820134829017	0.139997912937243	0.861958614648418\\
0.73	0.0105565495182326	0.144919489023162	0.860852735617091\\
0.73	0.0112905899930939	0.149845595743738	0.858677054109649\\
0.73	0.012050419400414	0.154776051742839	0.855504274967457\\
0.73	0.0128361328661109	0.159710674111862	0.851423687967755\\
0.73	0.0136478239890177	0.164649278402306	0.846537000484342\\
0.73	0.0144855848154856	0.169591678638796	0.840954157904248\\
0.73	0.0153495058140643	0.174537687332543	0.834789345403934\\
0.73	0.0162396758502651	0.179487115495259	0.828157325806199\\
0.73	0.0171561821614173	0.184439772653511	0.821170226448696\\
0.73	0.0180991103316243	0.189395466863524	0.813934846725693\\
0.73	0.0190685442668299	0.194354004726437	0.806550519904213\\
0.73	0.0200645661700016	0.199315191403999	0.799107529846306\\
0.73	0.0210872565164405	0.204278830634725	0.791686056496196\\
0.73	0.0221366940292259	0.20924472475049	0.784355603830601\\
0.73	0.023212955654804	0.214212674693577	0.777174850281116\\
0.73	0.0243161165387281	0.219182480034174	0.770191853862644\\
0.73	0.025446250001561	0.224153938988321	0.76344454155446\\
0.73	0.0266034275149466	0.229126848436298	0.756961413910482\\
0.73	0.0277877186778607	0.234101003941464	0.750762400416763\\
0.73	0.0289991911930496	0.239076199769547	0.744859807805784\\
0.73	0.0302379108436654	0.244052228908366	0.73925931151808\\
0.73	0.0315039414701067	0.24902888308801	0.733960949044504\\
0.73	0.0327973449470747	0.254005952801449	0.728960082407261\\
0.73	0.0341181811608519	0.258983227325592	0.724248305111178\\
0.73	0.0354665079868145	0.263960494742775	0.71981427622068\\
0.73	0.0368423812671862	0.268937541962694	0.71564447061263\\
0.73	0.0382458547890421	0.273914154744766	0.711723839838041\\
0.73	0.0396769802625738	0.278890117720924	0.70803638238973\\
0.73	0.0411358072996216	0.283865214418836	0.704565625567167\\
0.73	0.0426223833924861	0.288839227285554	0.701295023640316\\
0.73	0.0441367538930258	0.293811937711588	0.698208278749352\\
0.73	0.0456789619920503	0.298783126055387	0.695289592055349\\
0.73	0.0472490486990192	0.303752571668251	0.692523853197513\\
0.73	0.0488470528220538	0.308720052919643	0.689896776228119\\
0.73	0.0504730109482718	0.313685347222913	0.687394989990281\\
0.73	0.0521269574244539	0.318648231061428	0.685006090465981\\
0.73	0.0538089243380495	0.323608480015097	0.682718662029128\\
0.73	0.0555189414985328	0.328565868787292	0.680522273853469\\
0.73	0.0572570364191156	0.333520171232163	0.678407456997849\\
0.73	0.0590232342988274	0.338471160382329	0.676365666960044\\
0.73	0.0608175580049697	0.34341860847696	0.674389235783155\\
0.73	0.0626400280559541	0.348362286990222	0.672471317134862\\
0.73	0.064490662604533	0.3533019666601	0.670605827172183\\
0.73	0.0663694774214291	0.358237417517579	0.668787383459802\\
0.73	0.0682764858793752	0.363168408916186	0.667011243730727\\
0.73	0.0702116989375697	0.368094709561873	0.66527324586365\\
0.73	0.0721751251265572	0.373016087543259	0.663569750098933\\
0.73	0.0741667705335422	0.377932310362199	0.661897584220061\\
0.73	0.0761866387881432	0.382843144964686	0.660253992184443\\
0.73	0.0782347310485954	0.38774835777208	0.658636586490996\\
0.73	0.0803110459884098	0.392647714712651	0.65704330441587\\
0.73	0.0824155797834956	0.397540981253432	0.655472368126509\\
0.73	0.084548326099754	0.402427922432375	0.653922248592745\\
0.73	0.0867092760811502	0.407308302890799	0.65239163314657\\
0.73	0.0888984183382709	0.412181886906127	0.65087939649558\\
0.73	0.0911157389373742	0.417048438424896	0.649384574964845\\
0.73	0.0933612213899392	0.421907721096044	0.647906343724685\\
0.73	0.0956348466427212	0.42675949830445	0.646443996754805\\
0.73	0.0979365930683194	0.431603533204733	0.644996929295962\\
0.73	0.100266436456264	0.436439588755293	0.643564622546847\\
0.73	0.102624350004627	0.441267427752584	0.64214663037443\\
0.73	0.105010304312169	0.446086812865616	0.640742567819433\\
0.73	0.107424267371012	0.450897506670668	0.63935210119373\\
0.73	0.109866204559871	0.455699271686218	0.637974939582443\\
0.73	0.112336078637817	0.460491870408058	0.636610827579863\\
0.73	0.114833849738607	0.465275065344603	0.635259539104381\\
0.73	0.117359475365564	0.470048619052377	0.633920872153155\\
0.73	0.119912910387023	0.474812294171664	0.632594644371966\\
0.73	0.122494107032347	0.479565853462319	0.631280689329641\\
0.73	0.125103014888515	0.484309059839721	0.629978853399059\\
0.73	0.127739580897283	0.489041676410868	0.628690460824555\\
0.735	0	0	0.710202759607373\\
0.735	1.11327767495586e-05	0.00471862581271275	0.713003157837853\\
0.735	4.46175251031912e-05	0.009446325184051	0.715846486660142\\
0.735	0.000100583311362513	0.0141829653360114	0.718788838379794\\
0.735	0.000179158434668431	0.018928411755669	0.721886092228356\\
0.735	0.000280470402701511	0.02368252819604	0.725200636220892\\
0.735	0.000404645907256436	0.0284451766772965	0.728799085274723\\
0.735	0.000551810799695644	0.0332162174883389	0.732749123983336\\
0.735	0.000722090066287311	0.037995509188729	0.737115575253328\\
0.735	0.000915607803432999	0.0427829086109896	0.741955893907011\\
0.735	0.0011324871927904	0.0475782708632729	0.747315373652289\\
0.735	0.00137285047629673	0.0523814493324038	0.753222424045811\\
0.735	0.00163681893109844	0.057192295687301	0.759684310525652\\
0.735	0.00192451284439304	0.0620106598827802	0.766683748102665\\
0.735	0.00223605148818898	0.0668363901637434	0.774176695511256\\
0.735	0.00257155309398959	0.0716693330697584	0.782091614494493\\
0.735	0.00293113482740716	0.0765093334400312	0.79033034642109\\
0.735	0.00331491276271365	0.0813562344187764	0.798770627526651\\
0.735	0.00372300185733413	0.0862098774609879	0.807270128930359\\
0.735	0.00415551592628969	0.0910701023386145	0.815671782700193\\
0.735	0.00461256761659621	0.0959367471471425	0.823810053507331\\
0.735	0.00509426838162598	0.100809648312589	0.831517746503355\\
0.735	0.00560072845543873	0.105688640598912	0.838632911419452\\
0.735	0.00613205682708918	0.11057355711583	0.845005411376292\\
0.735	0.00668836121491816	0.115464229327074	0.850502769049301\\
0.735	0.00726974804083431	0.120360487059049	0.855014975717656\\
0.735	0.00787632240459385	0.125262158509929	0.858458041096605\\
0.735	0.00850818805808554	0.13016907025918	0.860776163494183\\
0.735	0.00916544737962849	0.135081047277508	0.861942500814226\\
0.735	0.00984820134829017	0.139997912937243	0.86195861464842\\
0.735	0.0105565495182326	0.144919489023162	0.860852735617088\\
0.735	0.0112905899930939	0.149845595743738	0.858677054109652\\
0.735	0.012050419400414	0.154776051742839	0.855504274967459\\
0.735	0.0128361328661109	0.159710674111862	0.851423687967757\\
0.735	0.0136478239890177	0.164649278402306	0.846537000484341\\
0.735	0.0144855848154856	0.169591678638796	0.840954157904248\\
0.735	0.0153495058140643	0.174537687332543	0.834789345403935\\
0.735	0.0162396758502651	0.179487115495259	0.828157325806202\\
0.735	0.0171561821614173	0.184439772653511	0.821170226448697\\
0.735	0.0180991103316243	0.189395466863524	0.813934846725691\\
0.735	0.0190685442668299	0.194354004726437	0.806550519904212\\
0.735	0.0200645661700016	0.199315191403999	0.799107529846303\\
0.735	0.0210872565164405	0.204278830634725	0.791686056496194\\
0.735	0.0221366940292259	0.20924472475049	0.784355603830602\\
0.735	0.023212955654804	0.214212674693577	0.777174850281118\\
0.735	0.0243161165387281	0.219182480034174	0.770191853862644\\
0.735	0.025446250001561	0.224153938988321	0.763444541554458\\
0.735	0.0266034275149466	0.229126848436298	0.756961413910482\\
0.735	0.0277877186778607	0.234101003941464	0.750762400416764\\
0.735	0.0289991911930496	0.239076199769547	0.744859807805785\\
0.735	0.0302379108436654	0.244052228908366	0.739259311518079\\
0.735	0.0315039414701067	0.24902888308801	0.733960949044504\\
0.735	0.0327973449470747	0.254005952801449	0.728960082407262\\
0.735	0.0341181811608519	0.258983227325592	0.724248305111177\\
0.735	0.0354665079868145	0.263960494742775	0.719814276220681\\
0.735	0.0368423812671862	0.268937541962694	0.715644470612631\\
0.735	0.0382458547890421	0.273914154744766	0.711723839838041\\
0.735	0.0396769802625738	0.278890117720924	0.708036382389729\\
0.735	0.0411358072996216	0.283865214418836	0.704565625567166\\
0.735	0.0426223833924861	0.288839227285554	0.701295023640316\\
0.735	0.0441367538930258	0.293811937711588	0.698208278749352\\
0.735	0.0456789619920503	0.298783126055387	0.695289592055349\\
0.735	0.0472490486990192	0.303752571668251	0.692523853197513\\
0.735	0.0488470528220538	0.308720052919643	0.68989677622812\\
0.735	0.0504730109482718	0.313685347222913	0.687394989990281\\
0.735	0.0521269574244539	0.318648231061428	0.68500609046598\\
0.735	0.0538089243380495	0.323608480015096	0.682718662029128\\
0.735	0.0555189414985328	0.328565868787292	0.68052227385347\\
0.735	0.0572570364191156	0.333520171232163	0.678407456997849\\
0.735	0.0590232342988274	0.338471160382329	0.676365666960043\\
0.735	0.0608175580049697	0.34341860847696	0.674389235783156\\
0.735	0.0626400280559541	0.348362286990222	0.672471317134862\\
0.735	0.064490662604533	0.3533019666601	0.670605827172183\\
0.735	0.066369477421429	0.358237417517579	0.668787383459804\\
0.735	0.0682764858793752	0.363168408916186	0.667011243730727\\
0.735	0.0702116989375697	0.368094709561873	0.665273245863649\\
0.735	0.0721751251265572	0.373016087543259	0.663569750098932\\
0.735	0.0741667705335422	0.377932310362199	0.66189758422006\\
0.735	0.0761866387881432	0.382843144964686	0.660253992184442\\
0.735	0.0782347310485954	0.38774835777208	0.658636586490997\\
0.735	0.0803110459884098	0.392647714712651	0.657043304415871\\
0.735	0.0824155797834956	0.397540981253432	0.655472368126508\\
0.735	0.084548326099754	0.402427922432375	0.653922248592746\\
0.735	0.0867092760811502	0.4073083028908	0.652391633146569\\
0.735	0.0888984183382709	0.412181886906127	0.650879396495579\\
0.735	0.0911157389373742	0.417048438424897	0.649384574964844\\
0.735	0.0933612213899392	0.421907721096044	0.647906343724684\\
0.735	0.0956348466427212	0.42675949830445	0.646443996754806\\
0.735	0.0979365930683194	0.431603533204733	0.644996929295963\\
0.735	0.100266436456264	0.436439588755293	0.643564622546846\\
0.735	0.102624350004627	0.441267427752584	0.64214663037443\\
0.735	0.105010304312169	0.446086812865616	0.640742567819433\\
0.735	0.107424267371012	0.450897506670668	0.639352101193729\\
0.735	0.109866204559871	0.455699271686218	0.637974939582442\\
0.735	0.112336078637817	0.460491870408058	0.636610827579863\\
0.735	0.114833849738607	0.465275065344603	0.635259539104383\\
0.735	0.117359475365564	0.470048619052377	0.633920872153153\\
0.735	0.119912910387022	0.474812294171664	0.632594644371965\\
0.735	0.122494107032347	0.479565853462319	0.631280689329641\\
0.735	0.125103014888515	0.484309059839721	0.62997885339906\\
0.735	0.127739580897283	0.489041676410868	0.62869046082456\\
0.74	0	0	0.710202759607373\\
0.74	1.11327767495586e-05	0.00471862581271275	0.713003157837853\\
0.74	4.46175251031912e-05	0.009446325184051	0.715846486660142\\
0.74	0.000100583311362513	0.0141829653360114	0.718788838379794\\
0.74	0.000179158434668431	0.018928411755669	0.721886092228356\\
0.74	0.000280470402701511	0.02368252819604	0.725200636220892\\
0.74	0.000404645907256436	0.0284451766772965	0.728799085274723\\
0.74	0.000551810799695644	0.0332162174883388	0.732749123983336\\
0.74	0.000722090066287311	0.037995509188729	0.737115575253327\\
0.74	0.000915607803432999	0.0427829086109896	0.741955893907011\\
0.74	0.0011324871927904	0.0475782708632729	0.747315373652289\\
0.74	0.00137285047629673	0.0523814493324038	0.753222424045811\\
0.74	0.00163681893109843	0.057192295687301	0.759684310525652\\
0.74	0.00192451284439304	0.0620106598827802	0.766683748102665\\
0.74	0.00223605148818899	0.0668363901637434	0.774176695511256\\
0.74	0.00257155309398959	0.0716693330697584	0.782091614494494\\
0.74	0.00293113482740716	0.0765093334400312	0.79033034642109\\
0.74	0.00331491276271365	0.0813562344187764	0.79877062752665\\
0.74	0.00372300185733413	0.0862098774609879	0.807270128930359\\
0.74	0.00415551592628969	0.0910701023386145	0.815671782700193\\
0.74	0.00461256761659621	0.0959367471471425	0.823810053507329\\
0.74	0.00509426838162598	0.100809648312589	0.831517746503354\\
0.74	0.00560072845543873	0.105688640598912	0.838632911419452\\
0.74	0.00613205682708918	0.11057355711583	0.845005411376292\\
0.74	0.00668836121491815	0.115464229327074	0.850502769049302\\
0.74	0.00726974804083431	0.120360487059049	0.855014975717656\\
0.74	0.00787632240459385	0.125262158509929	0.858458041096605\\
0.74	0.00850818805808555	0.13016907025918	0.86077616349418\\
0.74	0.00916544737962849	0.135081047277508	0.861942500814224\\
0.74	0.00984820134829017	0.139997912937243	0.861958614648422\\
0.74	0.0105565495182326	0.144919489023162	0.860852735617092\\
0.74	0.0112905899930939	0.149845595743738	0.858677054109652\\
0.74	0.012050419400414	0.154776051742839	0.855504274967457\\
0.74	0.0128361328661109	0.159710674111862	0.851423687967754\\
0.74	0.0136478239890177	0.164649278402306	0.84653700048434\\
0.74	0.0144855848154856	0.169591678638796	0.840954157904249\\
0.74	0.0153495058140643	0.174537687332543	0.834789345403936\\
0.74	0.0162396758502651	0.179487115495259	0.828157325806201\\
0.74	0.0171561821614173	0.184439772653511	0.821170226448693\\
0.74	0.0180991103316243	0.189395466863524	0.81393484672569\\
0.74	0.0190685442668299	0.194354004726437	0.80655051990421\\
0.74	0.0200645661700016	0.199315191403999	0.799107529846305\\
0.74	0.0210872565164405	0.204278830634725	0.791686056496196\\
0.74	0.0221366940292259	0.20924472475049	0.784355603830601\\
0.74	0.023212955654804	0.214212674693577	0.777174850281117\\
0.74	0.0243161165387281	0.219182480034174	0.770191853862643\\
0.74	0.025446250001561	0.224153938988321	0.763444541554458\\
0.74	0.0266034275149466	0.229126848436298	0.756961413910484\\
0.74	0.0277877186778607	0.234101003941464	0.750762400416764\\
0.74	0.0289991911930496	0.239076199769547	0.744859807805784\\
0.74	0.0302379108436654	0.244052228908366	0.739259311518081\\
0.74	0.0315039414701067	0.24902888308801	0.733960949044505\\
0.74	0.0327973449470747	0.254005952801449	0.728960082407261\\
0.74	0.0341181811608519	0.258983227325592	0.724248305111177\\
0.74	0.0354665079868145	0.263960494742775	0.719814276220681\\
0.74	0.0368423812671862	0.268937541962694	0.71564447061263\\
0.74	0.0382458547890421	0.273914154744766	0.711723839838041\\
0.74	0.0396769802625738	0.278890117720924	0.708036382389731\\
0.74	0.0411358072996216	0.283865214418836	0.704565625567165\\
0.74	0.0426223833924861	0.288839227285554	0.701295023640316\\
0.74	0.0441367538930258	0.293811937711588	0.698208278749354\\
0.74	0.0456789619920503	0.298783126055387	0.695289592055349\\
0.74	0.0472490486990192	0.303752571668251	0.69252385319751\\
0.74	0.0488470528220538	0.308720052919643	0.689896776228118\\
0.74	0.0504730109482718	0.313685347222913	0.687394989990283\\
0.74	0.0521269574244539	0.318648231061428	0.685006090465982\\
0.74	0.0538089243380495	0.323608480015097	0.682718662029127\\
0.74	0.0555189414985328	0.328565868787292	0.68052227385347\\
0.74	0.0572570364191156	0.333520171232163	0.678407456997851\\
0.74	0.0590232342988274	0.338471160382329	0.676365666960043\\
0.74	0.0608175580049697	0.34341860847696	0.674389235783156\\
0.74	0.0626400280559541	0.348362286990222	0.672471317134861\\
0.74	0.064490662604533	0.3533019666601	0.670605827172182\\
0.74	0.0663694774214291	0.358237417517579	0.668787383459804\\
0.74	0.0682764858793752	0.363168408916186	0.667011243730728\\
0.74	0.0702116989375697	0.368094709561873	0.66527324586365\\
0.74	0.0721751251265572	0.373016087543259	0.663569750098932\\
0.74	0.0741667705335423	0.377932310362199	0.66189758422006\\
0.74	0.0761866387881432	0.382843144964686	0.660253992184442\\
0.74	0.0782347310485954	0.38774835777208	0.658636586490997\\
0.74	0.0803110459884098	0.392647714712651	0.657043304415872\\
0.74	0.0824155797834956	0.397540981253432	0.655472368126508\\
0.74	0.084548326099754	0.402427922432375	0.653922248592745\\
0.74	0.0867092760811502	0.407308302890799	0.65239163314657\\
0.74	0.0888984183382709	0.412181886906127	0.65087939649558\\
0.74	0.0911157389373742	0.417048438424896	0.649384574964845\\
0.74	0.0933612213899392	0.421907721096044	0.647906343724685\\
0.74	0.0956348466427212	0.42675949830445	0.646443996754804\\
0.74	0.0979365930683194	0.431603533204733	0.644996929295962\\
0.74	0.100266436456264	0.436439588755293	0.643564622546848\\
0.74	0.102624350004627	0.441267427752585	0.642146630374429\\
0.74	0.105010304312169	0.446086812865616	0.640742567819433\\
0.74	0.107424267371012	0.450897506670668	0.639352101193731\\
0.74	0.109866204559871	0.455699271686218	0.637974939582442\\
0.74	0.112336078637817	0.460491870408058	0.636610827579862\\
0.74	0.114833849738607	0.465275065344603	0.635259539104384\\
0.74	0.117359475365564	0.470048619052377	0.633920872153155\\
0.74	0.119912910387022	0.474812294171664	0.632594644371964\\
0.74	0.122494107032347	0.479565853462319	0.63128068932964\\
0.74	0.125103014888515	0.484309059839721	0.629978853399059\\
0.74	0.127739580897283	0.489041676410868	0.628690460824558\\
0.745	0	0	0.710202759607373\\
0.745	1.11327767495586e-05	0.00471862581271275	0.713003157837853\\
0.745	4.46175251031912e-05	0.009446325184051	0.715846486660142\\
0.745	0.000100583311362513	0.0141829653360114	0.718788838379794\\
0.745	0.000179158434668431	0.018928411755669	0.721886092228356\\
0.745	0.000280470402701511	0.02368252819604	0.725200636220892\\
0.745	0.000404645907256436	0.0284451766772965	0.728799085274723\\
0.745	0.000551810799695644	0.0332162174883389	0.732749123983336\\
0.745	0.000722090066287311	0.037995509188729	0.737115575253327\\
0.745	0.000915607803432999	0.0427829086109896	0.741955893907011\\
0.745	0.0011324871927904	0.0475782708632729	0.747315373652289\\
0.745	0.00137285047629673	0.0523814493324038	0.753222424045811\\
0.745	0.00163681893109844	0.057192295687301	0.759684310525652\\
0.745	0.00192451284439304	0.0620106598827802	0.766683748102665\\
0.745	0.00223605148818898	0.0668363901637434	0.774176695511256\\
0.745	0.00257155309398959	0.0716693330697584	0.782091614494494\\
0.745	0.00293113482740716	0.0765093334400312	0.79033034642109\\
0.745	0.00331491276271365	0.0813562344187764	0.798770627526651\\
0.745	0.00372300185733413	0.0862098774609879	0.807270128930359\\
0.745	0.00415551592628969	0.0910701023386145	0.815671782700194\\
0.745	0.00461256761659621	0.0959367471471424	0.82381005350733\\
0.745	0.00509426838162598	0.100809648312589	0.831517746503354\\
0.745	0.00560072845543873	0.105688640598912	0.838632911419452\\
0.745	0.00613205682708918	0.11057355711583	0.845005411376292\\
0.745	0.00668836121491816	0.115464229327074	0.850502769049302\\
0.745	0.00726974804083431	0.120360487059049	0.855014975717657\\
0.745	0.00787632240459385	0.125262158509929	0.858458041096606\\
0.745	0.00850818805808555	0.13016907025918	0.860776163494183\\
0.745	0.00916544737962849	0.135081047277508	0.861942500814223\\
0.745	0.00984820134829017	0.139997912937243	0.861958614648421\\
0.745	0.0105565495182326	0.144919489023162	0.860852735617091\\
0.745	0.0112905899930939	0.149845595743738	0.858677054109647\\
0.745	0.012050419400414	0.154776051742839	0.855504274967458\\
0.745	0.0128361328661109	0.159710674111862	0.851423687967755\\
0.745	0.0136478239890177	0.164649278402306	0.846537000484339\\
0.745	0.0144855848154856	0.169591678638796	0.840954157904249\\
0.745	0.0153495058140643	0.174537687332543	0.834789345403934\\
0.745	0.0162396758502651	0.179487115495259	0.8281573258062\\
0.745	0.0171561821614173	0.184439772653511	0.821170226448692\\
0.745	0.0180991103316243	0.189395466863524	0.813934846725692\\
0.745	0.0190685442668299	0.194354004726437	0.806550519904211\\
0.745	0.0200645661700016	0.199315191403999	0.799107529846308\\
0.745	0.0210872565164405	0.204278830634725	0.791686056496194\\
0.745	0.0221366940292259	0.20924472475049	0.7843556038306\\
0.745	0.023212955654804	0.214212674693577	0.777174850281117\\
0.745	0.0243161165387281	0.219182480034174	0.770191853862645\\
0.745	0.025446250001561	0.224153938988321	0.76344454155446\\
0.745	0.0266034275149466	0.229126848436298	0.756961413910482\\
0.745	0.0277877186778607	0.234101003941464	0.750762400416763\\
0.745	0.0289991911930496	0.239076199769547	0.744859807805786\\
0.745	0.0302379108436654	0.244052228908366	0.73925931151808\\
0.745	0.0315039414701067	0.24902888308801	0.733960949044503\\
0.745	0.0327973449470747	0.254005952801449	0.72896008240726\\
0.745	0.0341181811608519	0.258983227325592	0.724248305111178\\
0.745	0.0354665079868145	0.263960494742775	0.719814276220681\\
0.745	0.0368423812671862	0.268937541962694	0.715644470612629\\
0.745	0.0382458547890421	0.273914154744766	0.711723839838042\\
0.745	0.0396769802625737	0.278890117720924	0.708036382389731\\
0.745	0.0411358072996216	0.283865214418836	0.704565625567163\\
0.745	0.0426223833924861	0.288839227285554	0.701295023640316\\
0.745	0.0441367538930258	0.293811937711588	0.698208278749354\\
0.745	0.0456789619920503	0.298783126055387	0.695289592055349\\
0.745	0.0472490486990192	0.303752571668251	0.692523853197511\\
0.745	0.0488470528220538	0.308720052919643	0.689896776228119\\
0.745	0.0504730109482718	0.313685347222913	0.687394989990281\\
0.745	0.0521269574244539	0.318648231061428	0.685006090465981\\
0.745	0.0538089243380495	0.323608480015097	0.682718662029127\\
0.745	0.0555189414985328	0.328565868787292	0.680522273853469\\
0.745	0.0572570364191156	0.333520171232163	0.678407456997851\\
0.745	0.0590232342988274	0.338471160382329	0.676365666960042\\
0.745	0.0608175580049697	0.34341860847696	0.674389235783156\\
0.745	0.0626400280559541	0.348362286990222	0.672471317134863\\
0.745	0.064490662604533	0.3533019666601	0.670605827172182\\
0.745	0.0663694774214291	0.358237417517579	0.668787383459804\\
0.745	0.0682764858793752	0.363168408916186	0.667011243730728\\
0.745	0.0702116989375697	0.368094709561873	0.665273245863649\\
0.745	0.0721751251265572	0.373016087543259	0.663569750098933\\
0.745	0.0741667705335422	0.377932310362199	0.661897584220061\\
0.745	0.0761866387881432	0.382843144964686	0.660253992184443\\
0.745	0.0782347310485954	0.38774835777208	0.658636586490996\\
0.745	0.0803110459884098	0.392647714712651	0.65704330441587\\
0.745	0.0824155797834956	0.397540981253432	0.655472368126509\\
0.745	0.084548326099754	0.402427922432375	0.653922248592746\\
0.745	0.0867092760811502	0.4073083028908	0.65239163314657\\
0.745	0.0888984183382709	0.412181886906127	0.650879396495579\\
0.745	0.0911157389373742	0.417048438424896	0.649384574964844\\
0.745	0.0933612213899392	0.421907721096044	0.647906343724685\\
0.745	0.0956348466427212	0.42675949830445	0.646443996754807\\
0.745	0.0979365930683194	0.431603533204733	0.644996929295961\\
0.745	0.100266436456264	0.436439588755293	0.643564622546847\\
0.745	0.102624350004627	0.441267427752585	0.642146630374429\\
0.745	0.105010304312169	0.446086812865616	0.640742567819433\\
0.745	0.107424267371012	0.450897506670668	0.639352101193732\\
0.745	0.109866204559871	0.455699271686218	0.637974939582443\\
0.745	0.112336078637817	0.460491870408058	0.636610827579861\\
0.745	0.114833849738607	0.465275065344603	0.635259539104382\\
0.745	0.117359475365564	0.470048619052377	0.633920872153156\\
0.745	0.119912910387022	0.474812294171664	0.632594644371966\\
0.745	0.122494107032347	0.479565853462319	0.631280689329641\\
0.745	0.125103014888515	0.484309059839721	0.629978853399058\\
0.745	0.127739580897283	0.489041676410868	0.628690460824556\\
0.75	0	0	0.710202759607373\\
0.75	1.11327767495586e-05	0.00471862581271275	0.713003157837853\\
0.75	4.46175251031912e-05	0.009446325184051	0.715846486660142\\
0.75	0.000100583311362513	0.0141829653360114	0.718788838379794\\
0.75	0.000179158434668431	0.018928411755669	0.721886092228356\\
0.75	0.000280470402701511	0.02368252819604	0.725200636220892\\
0.75	0.000404645907256436	0.0284451766772965	0.728799085274723\\
0.75	0.000551810799695644	0.0332162174883388	0.732749123983336\\
0.75	0.000722090066287311	0.037995509188729	0.737115575253327\\
0.75	0.000915607803432999	0.0427829086109896	0.741955893907011\\
0.75	0.0011324871927904	0.0475782708632729	0.747315373652289\\
0.75	0.00137285047629673	0.0523814493324038	0.753222424045811\\
0.75	0.00163681893109844	0.057192295687301	0.759684310525652\\
0.75	0.00192451284439304	0.0620106598827802	0.766683748102665\\
0.75	0.00223605148818898	0.0668363901637434	0.774176695511256\\
0.75	0.00257155309398959	0.0716693330697584	0.782091614494493\\
0.75	0.00293113482740716	0.0765093334400312	0.79033034642109\\
0.75	0.00331491276271365	0.0813562344187764	0.79877062752665\\
0.75	0.00372300185733414	0.0862098774609879	0.807270128930359\\
0.75	0.00415551592628969	0.0910701023386146	0.815671782700193\\
0.75	0.00461256761659621	0.0959367471471425	0.823810053507332\\
0.75	0.00509426838162598	0.100809648312589	0.831517746503354\\
0.75	0.00560072845543873	0.105688640598912	0.838632911419453\\
0.75	0.00613205682708918	0.11057355711583	0.845005411376293\\
0.75	0.00668836121491816	0.115464229327074	0.850502769049301\\
0.75	0.00726974804083431	0.120360487059049	0.855014975717657\\
0.75	0.00787632240459385	0.125262158509929	0.858458041096606\\
0.75	0.00850818805808554	0.13016907025918	0.860776163494183\\
0.75	0.00916544737962849	0.135081047277508	0.861942500814226\\
0.75	0.00984820134829017	0.139997912937243	0.861958614648419\\
0.75	0.0105565495182326	0.144919489023162	0.860852735617091\\
0.75	0.0112905899930939	0.149845595743738	0.858677054109652\\
0.75	0.012050419400414	0.154776051742839	0.855504274967459\\
0.75	0.0128361328661109	0.159710674111862	0.851423687967756\\
0.75	0.0136478239890177	0.164649278402306	0.84653700048434\\
0.75	0.0144855848154856	0.169591678638796	0.840954157904248\\
0.75	0.0153495058140643	0.174537687332543	0.834789345403935\\
0.75	0.0162396758502651	0.179487115495259	0.828157325806201\\
0.75	0.0171561821614173	0.184439772653511	0.821170226448694\\
0.75	0.0180991103316243	0.189395466863524	0.813934846725692\\
0.75	0.0190685442668299	0.194354004726437	0.806550519904213\\
0.75	0.0200645661700016	0.199315191403999	0.799107529846305\\
0.75	0.0210872565164405	0.204278830634725	0.791686056496192\\
0.75	0.0221366940292259	0.20924472475049	0.784355603830602\\
0.75	0.023212955654804	0.214212674693577	0.777174850281119\\
0.75	0.0243161165387281	0.219182480034174	0.770191853862643\\
0.75	0.025446250001561	0.224153938988321	0.763444541554458\\
0.75	0.0266034275149466	0.229126848436298	0.756961413910482\\
0.75	0.0277877186778607	0.234101003941464	0.750762400416764\\
0.75	0.0289991911930496	0.239076199769547	0.744859807805784\\
0.75	0.0302379108436654	0.244052228908366	0.73925931151808\\
0.75	0.0315039414701067	0.24902888308801	0.733960949044503\\
0.75	0.0327973449470747	0.254005952801449	0.72896008240726\\
0.75	0.0341181811608519	0.258983227325592	0.724248305111178\\
0.75	0.0354665079868145	0.263960494742775	0.719814276220681\\
0.75	0.0368423812671862	0.268937541962694	0.71564447061263\\
0.75	0.0382458547890421	0.273914154744766	0.711723839838041\\
0.75	0.0396769802625738	0.278890117720924	0.70803638238973\\
0.75	0.0411358072996216	0.283865214418836	0.704565625567164\\
0.75	0.0426223833924861	0.288839227285554	0.701295023640317\\
0.75	0.0441367538930258	0.293811937711588	0.698208278749353\\
0.75	0.0456789619920503	0.298783126055387	0.695289592055349\\
0.75	0.0472490486990192	0.303752571668251	0.692523853197512\\
0.75	0.0488470528220538	0.308720052919643	0.689896776228119\\
0.75	0.0504730109482718	0.313685347222913	0.687394989990283\\
0.75	0.0521269574244539	0.318648231061428	0.68500609046598\\
0.75	0.0538089243380495	0.323608480015096	0.682718662029127\\
0.75	0.0555189414985328	0.328565868787292	0.680522273853469\\
0.75	0.0572570364191156	0.333520171232163	0.678407456997851\\
0.75	0.0590232342988274	0.338471160382329	0.676365666960044\\
0.75	0.0608175580049697	0.34341860847696	0.674389235783156\\
0.75	0.0626400280559541	0.348362286990222	0.672471317134862\\
0.75	0.064490662604533	0.3533019666601	0.670605827172182\\
0.75	0.0663694774214291	0.358237417517579	0.668787383459804\\
0.75	0.0682764858793752	0.363168408916186	0.667011243730729\\
0.75	0.0702116989375697	0.368094709561873	0.665273245863648\\
0.75	0.0721751251265572	0.373016087543259	0.663569750098932\\
0.75	0.0741667705335422	0.377932310362199	0.661897584220061\\
0.75	0.0761866387881432	0.382843144964686	0.660253992184442\\
0.75	0.0782347310485954	0.38774835777208	0.658636586490998\\
0.75	0.0803110459884098	0.392647714712651	0.657043304415871\\
0.75	0.0824155797834956	0.397540981253432	0.655472368126508\\
0.75	0.084548326099754	0.402427922432375	0.653922248592745\\
0.75	0.0867092760811502	0.407308302890799	0.652391633146569\\
0.75	0.0888984183382709	0.412181886906127	0.650879396495579\\
0.75	0.0911157389373742	0.417048438424896	0.649384574964845\\
0.75	0.0933612213899392	0.421907721096044	0.647906343724685\\
0.75	0.0956348466427212	0.42675949830445	0.646443996754805\\
0.75	0.0979365930683194	0.431603533204733	0.644996929295962\\
0.75	0.100266436456264	0.436439588755293	0.643564622546846\\
0.75	0.102624350004627	0.441267427752584	0.64214663037443\\
0.75	0.105010304312169	0.446086812865616	0.640742567819434\\
0.75	0.107424267371012	0.450897506670668	0.63935210119373\\
0.75	0.109866204559871	0.455699271686218	0.637974939582443\\
0.75	0.112336078637817	0.460491870408058	0.636610827579862\\
0.75	0.114833849738607	0.465275065344603	0.635259539104382\\
0.75	0.117359475365564	0.470048619052377	0.633920872153154\\
0.75	0.119912910387022	0.474812294171664	0.632594644371965\\
0.75	0.122494107032347	0.479565853462319	0.631280689329641\\
0.75	0.125103014888515	0.484309059839721	0.62997885339906\\
0.75	0.127739580897283	0.489041676410868	0.628690460824559\\
0.755	0	0	0.710202759607373\\
0.755	1.11327767495586e-05	0.00471862581271274	0.713003157837853\\
0.755	4.46175251031912e-05	0.009446325184051	0.715846486660142\\
0.755	0.000100583311362513	0.0141829653360114	0.718788838379794\\
0.755	0.000179158434668431	0.018928411755669	0.721886092228356\\
0.755	0.000280470402701511	0.02368252819604	0.725200636220892\\
0.755	0.000404645907256436	0.0284451766772965	0.728799085274723\\
0.755	0.000551810799695644	0.0332162174883389	0.732749123983336\\
0.755	0.000722090066287311	0.037995509188729	0.737115575253327\\
0.755	0.000915607803432999	0.0427829086109896	0.741955893907011\\
0.755	0.0011324871927904	0.0475782708632729	0.747315373652289\\
0.755	0.00137285047629673	0.0523814493324038	0.753222424045811\\
0.755	0.00163681893109843	0.057192295687301	0.759684310525652\\
0.755	0.00192451284439304	0.0620106598827802	0.766683748102665\\
0.755	0.00223605148818899	0.0668363901637434	0.774176695511256\\
0.755	0.00257155309398959	0.0716693330697584	0.782091614494493\\
0.755	0.00293113482740716	0.0765093334400312	0.79033034642109\\
0.755	0.00331491276271365	0.0813562344187764	0.79877062752665\\
0.755	0.00372300185733413	0.0862098774609879	0.807270128930359\\
0.755	0.00415551592628969	0.0910701023386145	0.815671782700193\\
0.755	0.00461256761659621	0.0959367471471425	0.82381005350733\\
0.755	0.00509426838162598	0.100809648312589	0.831517746503354\\
0.755	0.00560072845543873	0.105688640598912	0.838632911419451\\
0.755	0.00613205682708918	0.11057355711583	0.845005411376294\\
0.755	0.00668836121491815	0.115464229327074	0.850502769049302\\
0.755	0.00726974804083431	0.120360487059049	0.855014975717657\\
0.755	0.00787632240459385	0.125262158509929	0.858458041096606\\
0.755	0.00850818805808555	0.13016907025918	0.860776163494182\\
0.755	0.00916544737962849	0.135081047277508	0.861942500814224\\
0.755	0.00984820134829017	0.139997912937243	0.861958614648419\\
0.755	0.0105565495182326	0.144919489023162	0.860852735617092\\
0.755	0.0112905899930939	0.149845595743738	0.858677054109652\\
0.755	0.012050419400414	0.154776051742839	0.855504274967457\\
0.755	0.0128361328661109	0.159710674111862	0.851423687967753\\
0.755	0.0136478239890177	0.164649278402306	0.846537000484339\\
0.755	0.0144855848154856	0.169591678638796	0.840954157904248\\
0.755	0.0153495058140643	0.174537687332543	0.834789345403936\\
0.755	0.0162396758502651	0.179487115495259	0.828157325806202\\
0.755	0.0171561821614173	0.184439772653511	0.821170226448694\\
0.755	0.0180991103316243	0.189395466863524	0.813934846725693\\
0.755	0.0190685442668299	0.194354004726437	0.806550519904212\\
0.755	0.0200645661700016	0.199315191403999	0.799107529846304\\
0.755	0.0210872565164405	0.204278830634725	0.791686056496195\\
0.755	0.0221366940292259	0.20924472475049	0.784355603830603\\
0.755	0.023212955654804	0.214212674693577	0.777174850281118\\
0.755	0.0243161165387281	0.219182480034174	0.770191853862643\\
0.755	0.025446250001561	0.224153938988321	0.763444541554459\\
0.755	0.0266034275149466	0.229126848436298	0.756961413910483\\
0.755	0.0277877186778607	0.234101003941464	0.750762400416764\\
0.755	0.0289991911930496	0.239076199769547	0.744859807805784\\
0.755	0.0302379108436654	0.244052228908366	0.73925931151808\\
0.755	0.0315039414701067	0.24902888308801	0.733960949044504\\
0.755	0.0327973449470747	0.254005952801449	0.728960082407259\\
0.755	0.0341181811608519	0.258983227325591	0.724248305111178\\
0.755	0.0354665079868145	0.263960494742775	0.719814276220682\\
0.755	0.0368423812671862	0.268937541962694	0.715644470612629\\
0.755	0.0382458547890421	0.273914154744766	0.711723839838042\\
0.755	0.0396769802625738	0.278890117720924	0.708036382389732\\
0.755	0.0411358072996216	0.283865214418836	0.704565625567165\\
0.755	0.0426223833924861	0.288839227285554	0.701295023640315\\
0.755	0.0441367538930258	0.293811937711588	0.698208278749353\\
0.755	0.0456789619920503	0.298783126055387	0.695289592055348\\
0.755	0.0472490486990192	0.303752571668251	0.692523853197512\\
0.755	0.0488470528220538	0.308720052919643	0.68989677622812\\
0.755	0.0504730109482718	0.313685347222913	0.687394989990282\\
0.755	0.0521269574244539	0.318648231061428	0.68500609046598\\
0.755	0.0538089243380495	0.323608480015096	0.682718662029127\\
0.755	0.0555189414985328	0.328565868787292	0.680522273853469\\
0.755	0.0572570364191156	0.333520171232163	0.67840745699785\\
0.755	0.0590232342988274	0.338471160382329	0.676365666960043\\
0.755	0.0608175580049697	0.34341860847696	0.674389235783156\\
0.755	0.0626400280559541	0.348362286990222	0.672471317134863\\
0.755	0.064490662604533	0.3533019666601	0.670605827172182\\
0.755	0.0663694774214291	0.358237417517579	0.668787383459803\\
0.755	0.0682764858793752	0.363168408916186	0.667011243730728\\
0.755	0.0702116989375697	0.368094709561873	0.665273245863651\\
0.755	0.0721751251265572	0.373016087543259	0.663569750098931\\
0.755	0.0741667705335423	0.377932310362199	0.661897584220059\\
0.755	0.0761866387881432	0.382843144964686	0.660253992184443\\
0.755	0.0782347310485954	0.38774835777208	0.658636586490997\\
0.755	0.0803110459884098	0.392647714712651	0.657043304415873\\
0.755	0.0824155797834956	0.397540981253432	0.655472368126509\\
0.755	0.084548326099754	0.402427922432375	0.653922248592745\\
0.755	0.0867092760811502	0.4073083028908	0.652391633146571\\
0.755	0.0888984183382709	0.412181886906127	0.650879396495578\\
0.755	0.0911157389373742	0.417048438424896	0.649384574964843\\
0.755	0.0933612213899392	0.421907721096044	0.647906343724685\\
0.755	0.0956348466427212	0.42675949830445	0.646443996754806\\
0.755	0.0979365930683194	0.431603533204733	0.644996929295962\\
0.755	0.100266436456264	0.436439588755293	0.643564622546846\\
0.755	0.102624350004627	0.441267427752585	0.64214663037443\\
0.755	0.105010304312169	0.446086812865616	0.640742567819433\\
0.755	0.107424267371012	0.450897506670668	0.639352101193731\\
0.755	0.109866204559871	0.455699271686218	0.637974939582443\\
0.755	0.112336078637817	0.460491870408058	0.636610827579862\\
0.755	0.114833849738607	0.465275065344603	0.635259539104383\\
0.755	0.117359475365564	0.470048619052377	0.633920872153154\\
0.755	0.119912910387022	0.474812294171664	0.632594644371965\\
0.755	0.122494107032347	0.479565853462319	0.63128068932964\\
0.755	0.125103014888515	0.484309059839721	0.629978853399059\\
0.755	0.127739580897283	0.489041676410868	0.628690460824563\\
0.76	0	0	0.710202759607373\\
0.76	1.11327767495586e-05	0.00471862581271274	0.713003157837853\\
0.76	4.46175251031912e-05	0.009446325184051	0.715846486660142\\
0.76	0.000100583311362513	0.0141829653360114	0.718788838379794\\
0.76	0.000179158434668431	0.018928411755669	0.721886092228356\\
0.76	0.000280470402701511	0.02368252819604	0.725200636220892\\
0.76	0.000404645907256436	0.0284451766772965	0.728799085274723\\
0.76	0.000551810799695644	0.0332162174883388	0.732749123983336\\
0.76	0.000722090066287311	0.037995509188729	0.737115575253327\\
0.76	0.000915607803432999	0.0427829086109896	0.741955893907011\\
0.76	0.0011324871927904	0.0475782708632729	0.747315373652289\\
0.76	0.00137285047629673	0.0523814493324038	0.753222424045811\\
0.76	0.00163681893109843	0.057192295687301	0.759684310525652\\
0.76	0.00192451284439304	0.0620106598827802	0.766683748102665\\
0.76	0.00223605148818898	0.0668363901637434	0.774176695511256\\
0.76	0.00257155309398959	0.0716693330697584	0.782091614494493\\
0.76	0.00293113482740716	0.0765093334400312	0.79033034642109\\
0.76	0.00331491276271365	0.0813562344187764	0.79877062752665\\
0.76	0.00372300185733413	0.0862098774609879	0.807270128930359\\
0.76	0.00415551592628969	0.0910701023386145	0.815671782700193\\
0.76	0.00461256761659621	0.0959367471471425	0.82381005350733\\
0.76	0.00509426838162598	0.100809648312589	0.831517746503354\\
0.76	0.00560072845543873	0.105688640598912	0.838632911419452\\
0.76	0.00613205682708918	0.11057355711583	0.845005411376293\\
0.76	0.00668836121491815	0.115464229327074	0.850502769049301\\
0.76	0.00726974804083431	0.120360487059049	0.855014975717657\\
0.76	0.00787632240459385	0.125262158509929	0.858458041096605\\
0.76	0.00850818805808555	0.13016907025918	0.860776163494182\\
0.76	0.00916544737962849	0.135081047277508	0.861942500814224\\
0.76	0.00984820134829017	0.139997912937243	0.86195861464842\\
0.76	0.0105565495182326	0.144919489023162	0.860852735617092\\
0.76	0.0112905899930939	0.149845595743738	0.85867705410965\\
0.76	0.012050419400414	0.154776051742839	0.855504274967456\\
0.76	0.0128361328661109	0.159710674111862	0.851423687967753\\
0.76	0.0136478239890177	0.164649278402306	0.84653700048434\\
0.76	0.0144855848154856	0.169591678638796	0.840954157904249\\
0.76	0.0153495058140643	0.174537687332543	0.834789345403936\\
0.76	0.0162396758502651	0.179487115495259	0.828157325806199\\
0.76	0.0171561821614173	0.184439772653511	0.821170226448694\\
0.76	0.0180991103316243	0.189395466863524	0.813934846725693\\
0.76	0.0190685442668299	0.194354004726437	0.806550519904213\\
0.76	0.0200645661700016	0.199315191403999	0.799107529846307\\
0.76	0.0210872565164405	0.204278830634725	0.791686056496193\\
0.76	0.0221366940292259	0.20924472475049	0.784355603830603\\
0.76	0.023212955654804	0.214212674693577	0.777174850281117\\
0.76	0.0243161165387281	0.219182480034174	0.770191853862643\\
0.76	0.025446250001561	0.224153938988321	0.763444541554458\\
0.76	0.0266034275149466	0.229126848436298	0.756961413910483\\
0.76	0.0277877186778607	0.234101003941464	0.750762400416762\\
0.76	0.0289991911930496	0.239076199769547	0.744859807805784\\
0.76	0.0302379108436654	0.244052228908366	0.73925931151808\\
0.76	0.0315039414701067	0.24902888308801	0.733960949044503\\
0.76	0.0327973449470747	0.254005952801449	0.728960082407261\\
0.76	0.0341181811608519	0.258983227325592	0.724248305111179\\
0.76	0.0354665079868145	0.263960494742775	0.71981427622068\\
0.76	0.0368423812671862	0.268937541962694	0.71564447061263\\
0.76	0.0382458547890421	0.273914154744766	0.711723839838041\\
0.76	0.0396769802625738	0.278890117720924	0.708036382389731\\
0.76	0.0411358072996216	0.283865214418836	0.704565625567164\\
0.76	0.0426223833924861	0.288839227285554	0.701295023640316\\
0.76	0.0441367538930258	0.293811937711588	0.698208278749353\\
0.76	0.0456789619920503	0.298783126055387	0.695289592055349\\
0.76	0.0472490486990192	0.303752571668251	0.692523853197511\\
0.76	0.0488470528220538	0.308720052919643	0.689896776228119\\
0.76	0.0504730109482718	0.313685347222913	0.687394989990282\\
0.76	0.0521269574244539	0.318648231061428	0.68500609046598\\
0.76	0.0538089243380495	0.323608480015096	0.682718662029128\\
0.76	0.0555189414985328	0.328565868787292	0.680522273853469\\
0.76	0.0572570364191156	0.333520171232163	0.67840745699785\\
0.76	0.0590232342988274	0.338471160382329	0.676365666960043\\
0.76	0.0608175580049697	0.34341860847696	0.674389235783155\\
0.76	0.0626400280559541	0.348362286990222	0.672471317134862\\
0.76	0.064490662604533	0.3533019666601	0.670605827172183\\
0.76	0.0663694774214291	0.358237417517579	0.668787383459804\\
0.76	0.0682764858793752	0.363168408916186	0.667011243730727\\
0.76	0.0702116989375697	0.368094709561873	0.66527324586365\\
0.76	0.0721751251265572	0.373016087543259	0.663569750098933\\
0.76	0.0741667705335423	0.377932310362199	0.661897584220059\\
0.76	0.0761866387881432	0.382843144964686	0.660253992184442\\
0.76	0.0782347310485954	0.38774835777208	0.658636586490997\\
0.76	0.0803110459884098	0.392647714712651	0.657043304415871\\
0.76	0.0824155797834956	0.397540981253432	0.655472368126509\\
0.76	0.084548326099754	0.402427922432375	0.653922248592745\\
0.76	0.0867092760811502	0.407308302890799	0.652391633146571\\
0.76	0.0888984183382709	0.412181886906127	0.65087939649558\\
0.76	0.0911157389373742	0.417048438424896	0.649384574964842\\
0.76	0.0933612213899392	0.421907721096044	0.647906343724685\\
0.76	0.0956348466427212	0.42675949830445	0.646443996754807\\
0.76	0.0979365930683194	0.431603533204733	0.644996929295963\\
0.76	0.100266436456264	0.436439588755293	0.643564622546846\\
0.76	0.102624350004627	0.441267427752585	0.64214663037443\\
0.76	0.105010304312169	0.446086812865616	0.640742567819433\\
0.76	0.107424267371012	0.450897506670668	0.63935210119373\\
0.76	0.109866204559871	0.455699271686218	0.637974939582443\\
0.76	0.112336078637817	0.460491870408058	0.636610827579861\\
0.76	0.114833849738607	0.465275065344603	0.635259539104383\\
0.76	0.117359475365564	0.470048619052377	0.633920872153155\\
0.76	0.119912910387022	0.474812294171664	0.632594644371965\\
0.76	0.122494107032347	0.479565853462319	0.631280689329641\\
0.76	0.125103014888515	0.484309059839721	0.629978853399059\\
0.76	0.127739580897283	0.489041676410868	0.62869046082456\\
0.765	0	0	0.710202759607373\\
0.765	1.11327767495586e-05	0.00471862581271274	0.713003157837853\\
0.765	4.46175251031912e-05	0.009446325184051	0.715846486660142\\
0.765	0.000100583311362513	0.0141829653360114	0.718788838379794\\
0.765	0.000179158434668431	0.018928411755669	0.721886092228357\\
0.765	0.000280470402701511	0.02368252819604	0.725200636220892\\
0.765	0.000404645907256436	0.0284451766772965	0.728799085274723\\
0.765	0.000551810799695644	0.0332162174883389	0.732749123983336\\
0.765	0.000722090066287311	0.037995509188729	0.737115575253328\\
0.765	0.000915607803432999	0.0427829086109896	0.741955893907011\\
0.765	0.0011324871927904	0.0475782708632729	0.747315373652289\\
0.765	0.00137285047629673	0.0523814493324038	0.75322242404581\\
0.765	0.00163681893109844	0.057192295687301	0.759684310525652\\
0.765	0.00192451284439304	0.0620106598827802	0.766683748102665\\
0.765	0.00223605148818899	0.0668363901637434	0.774176695511256\\
0.765	0.00257155309398959	0.0716693330697584	0.782091614494493\\
0.765	0.00293113482740716	0.0765093334400312	0.79033034642109\\
0.765	0.00331491276271365	0.0813562344187764	0.79877062752665\\
0.765	0.00372300185733413	0.0862098774609879	0.807270128930359\\
0.765	0.00415551592628969	0.0910701023386145	0.815671782700193\\
0.765	0.00461256761659621	0.0959367471471425	0.82381005350733\\
0.765	0.00509426838162598	0.100809648312589	0.831517746503353\\
0.765	0.00560072845543873	0.105688640598912	0.838632911419452\\
0.765	0.00613205682708918	0.11057355711583	0.845005411376292\\
0.765	0.00668836121491815	0.115464229327074	0.850502769049302\\
0.765	0.00726974804083431	0.120360487059049	0.855014975717657\\
0.765	0.00787632240459385	0.125262158509929	0.858458041096606\\
0.765	0.00850818805808554	0.13016907025918	0.860776163494183\\
0.765	0.00916544737962849	0.135081047277508	0.861942500814223\\
0.765	0.00984820134829017	0.139997912937243	0.861958614648421\\
0.765	0.0105565495182326	0.144919489023162	0.860852735617092\\
0.765	0.0112905899930939	0.149845595743738	0.85867705410965\\
0.765	0.012050419400414	0.154776051742839	0.855504274967459\\
0.765	0.0128361328661109	0.159710674111862	0.851423687967755\\
0.765	0.0136478239890177	0.164649278402306	0.846537000484341\\
0.765	0.0144855848154856	0.169591678638796	0.840954157904249\\
0.765	0.0153495058140643	0.174537687332543	0.834789345403936\\
0.765	0.0162396758502651	0.179487115495259	0.828157325806198\\
0.765	0.0171561821614173	0.184439772653511	0.821170226448693\\
0.765	0.0180991103316243	0.189395466863524	0.813934846725692\\
0.765	0.0190685442668299	0.194354004726437	0.806550519904212\\
0.765	0.0200645661700016	0.199315191403999	0.799107529846302\\
0.765	0.0210872565164405	0.204278830634725	0.791686056496193\\
0.765	0.0221366940292259	0.20924472475049	0.784355603830603\\
0.765	0.023212955654804	0.214212674693577	0.777174850281116\\
0.765	0.0243161165387281	0.219182480034174	0.770191853862644\\
0.765	0.025446250001561	0.224153938988321	0.763444541554459\\
0.765	0.0266034275149466	0.229126848436298	0.756961413910481\\
0.765	0.0277877186778607	0.234101003941464	0.750762400416764\\
0.765	0.0289991911930496	0.239076199769547	0.744859807805786\\
0.765	0.0302379108436654	0.244052228908366	0.73925931151808\\
0.765	0.0315039414701067	0.24902888308801	0.733960949044503\\
0.765	0.0327973449470747	0.254005952801449	0.728960082407259\\
0.765	0.0341181811608519	0.258983227325591	0.724248305111178\\
0.765	0.0354665079868145	0.263960494742775	0.719814276220682\\
0.765	0.0368423812671862	0.268937541962694	0.715644470612629\\
0.765	0.0382458547890421	0.273914154744766	0.711723839838042\\
0.765	0.0396769802625738	0.278890117720924	0.708036382389731\\
0.765	0.0411358072996216	0.283865214418836	0.704565625567164\\
0.765	0.0426223833924861	0.288839227285554	0.701295023640316\\
0.765	0.0441367538930258	0.293811937711588	0.698208278749354\\
0.765	0.0456789619920503	0.298783126055387	0.695289592055349\\
0.765	0.0472490486990192	0.303752571668251	0.692523853197513\\
0.765	0.0488470528220538	0.308720052919643	0.689896776228119\\
0.765	0.0504730109482718	0.313685347222913	0.687394989990281\\
0.765	0.0521269574244539	0.318648231061428	0.685006090465979\\
0.765	0.0538089243380495	0.323608480015096	0.682718662029127\\
0.765	0.0555189414985328	0.328565868787292	0.680522273853471\\
0.765	0.0572570364191156	0.333520171232163	0.678407456997851\\
0.765	0.0590232342988274	0.338471160382329	0.676365666960044\\
0.765	0.0608175580049697	0.34341860847696	0.674389235783156\\
0.765	0.0626400280559541	0.348362286990222	0.672471317134862\\
0.765	0.064490662604533	0.3533019666601	0.670605827172182\\
0.765	0.0663694774214291	0.358237417517579	0.668787383459803\\
0.765	0.0682764858793752	0.363168408916186	0.667011243730728\\
0.765	0.0702116989375697	0.368094709561873	0.665273245863649\\
0.765	0.0721751251265572	0.373016087543259	0.663569750098933\\
0.765	0.0741667705335422	0.377932310362199	0.66189758422006\\
0.765	0.0761866387881432	0.382843144964686	0.660253992184441\\
0.765	0.0782347310485954	0.38774835777208	0.658636586490998\\
0.765	0.0803110459884098	0.392647714712651	0.657043304415872\\
0.765	0.0824155797834956	0.397540981253432	0.655472368126508\\
0.765	0.084548326099754	0.402427922432375	0.653922248592746\\
0.765	0.0867092760811502	0.4073083028908	0.652391633146569\\
0.765	0.0888984183382709	0.412181886906127	0.65087939649558\\
0.765	0.0911157389373742	0.417048438424897	0.649384574964844\\
0.765	0.0933612213899392	0.421907721096044	0.647906343724683\\
0.765	0.0956348466427212	0.42675949830445	0.646443996754806\\
0.765	0.0979365930683194	0.431603533204733	0.644996929295963\\
0.765	0.100266436456264	0.436439588755293	0.643564622546846\\
0.765	0.102624350004627	0.441267427752584	0.642146630374429\\
0.765	0.105010304312169	0.446086812865616	0.640742567819434\\
0.765	0.107424267371012	0.450897506670668	0.63935210119373\\
0.765	0.109866204559871	0.455699271686218	0.637974939582444\\
0.765	0.112336078637817	0.460491870408058	0.636610827579863\\
0.765	0.114833849738607	0.465275065344603	0.635259539104382\\
0.765	0.117359475365564	0.470048619052377	0.633920872153155\\
0.765	0.119912910387022	0.474812294171664	0.632594644371965\\
0.765	0.122494107032347	0.479565853462319	0.63128068932964\\
0.765	0.125103014888515	0.484309059839721	0.629978853399057\\
0.765	0.127739580897283	0.489041676410868	0.628690460824558\\
0.77	0	0	0.710202759607373\\
0.77	1.11327767495586e-05	0.00471862581271274	0.713003157837853\\
0.77	4.46175251031912e-05	0.009446325184051	0.715846486660142\\
0.77	0.000100583311362513	0.0141829653360114	0.718788838379794\\
0.77	0.000179158434668431	0.018928411755669	0.721886092228356\\
0.77	0.000280470402701511	0.02368252819604	0.725200636220892\\
0.77	0.000404645907256436	0.0284451766772965	0.728799085274723\\
0.77	0.000551810799695644	0.0332162174883389	0.732749123983336\\
0.77	0.000722090066287311	0.037995509188729	0.737115575253328\\
0.77	0.000915607803432999	0.0427829086109896	0.741955893907011\\
0.77	0.0011324871927904	0.047578270863273	0.747315373652289\\
0.77	0.00137285047629673	0.0523814493324038	0.753222424045811\\
0.77	0.00163681893109844	0.057192295687301	0.759684310525652\\
0.77	0.00192451284439304	0.0620106598827802	0.766683748102665\\
0.77	0.00223605148818899	0.0668363901637434	0.774176695511257\\
0.77	0.00257155309398959	0.0716693330697584	0.782091614494494\\
0.77	0.00293113482740716	0.0765093334400312	0.79033034642109\\
0.77	0.00331491276271365	0.0813562344187764	0.798770627526651\\
0.77	0.00372300185733414	0.0862098774609879	0.807270128930359\\
0.77	0.00415551592628969	0.0910701023386145	0.815671782700193\\
0.77	0.00461256761659621	0.0959367471471425	0.82381005350733\\
0.77	0.00509426838162598	0.100809648312589	0.831517746503354\\
0.77	0.00560072845543873	0.105688640598912	0.838632911419452\\
0.77	0.00613205682708918	0.11057355711583	0.845005411376292\\
0.77	0.00668836121491816	0.115464229327074	0.850502769049301\\
0.77	0.00726974804083431	0.120360487059049	0.855014975717656\\
0.77	0.00787632240459385	0.125262158509929	0.858458041096606\\
0.77	0.00850818805808554	0.13016907025918	0.860776163494182\\
0.77	0.00916544737962849	0.135081047277508	0.861942500814224\\
0.77	0.00984820134829017	0.139997912937243	0.861958614648421\\
0.77	0.0105565495182326	0.144919489023162	0.860852735617088\\
0.77	0.0112905899930939	0.149845595743738	0.858677054109649\\
0.77	0.012050419400414	0.154776051742839	0.855504274967458\\
0.77	0.0128361328661109	0.159710674111862	0.851423687967754\\
0.77	0.0136478239890177	0.164649278402306	0.846537000484341\\
0.77	0.0144855848154856	0.169591678638796	0.84095415790425\\
0.77	0.0153495058140643	0.174537687332543	0.834789345403936\\
0.77	0.0162396758502651	0.179487115495259	0.828157325806201\\
0.77	0.0171561821614173	0.184439772653511	0.821170226448694\\
0.77	0.0180991103316243	0.189395466863524	0.813934846725692\\
0.77	0.0190685442668299	0.194354004726437	0.806550519904209\\
0.77	0.0200645661700016	0.199315191403999	0.799107529846305\\
0.77	0.0210872565164405	0.204278830634725	0.791686056496196\\
0.77	0.0221366940292259	0.20924472475049	0.7843556038306\\
0.77	0.023212955654804	0.214212674693577	0.777174850281118\\
0.77	0.0243161165387281	0.219182480034174	0.770191853862644\\
0.77	0.025446250001561	0.224153938988321	0.763444541554457\\
0.77	0.0266034275149466	0.229126848436298	0.756961413910482\\
0.77	0.0277877186778607	0.234101003941464	0.750762400416764\\
0.77	0.0289991911930496	0.239076199769547	0.744859807805784\\
0.77	0.0302379108436654	0.244052228908366	0.739259311518079\\
0.77	0.0315039414701067	0.24902888308801	0.733960949044504\\
0.77	0.0327973449470747	0.254005952801449	0.728960082407262\\
0.77	0.0341181811608519	0.258983227325592	0.724248305111179\\
0.77	0.0354665079868145	0.263960494742775	0.71981427622068\\
0.77	0.0368423812671862	0.268937541962694	0.715644470612629\\
0.77	0.0382458547890421	0.273914154744766	0.711723839838043\\
0.77	0.0396769802625738	0.278890117720924	0.70803638238973\\
0.77	0.0411358072996216	0.283865214418836	0.704565625567165\\
0.77	0.0426223833924861	0.288839227285554	0.701295023640316\\
0.77	0.0441367538930258	0.293811937711588	0.698208278749352\\
0.77	0.0456789619920503	0.298783126055387	0.695289592055349\\
0.77	0.0472490486990192	0.303752571668251	0.692523853197512\\
0.77	0.0488470528220538	0.308720052919643	0.689896776228119\\
0.77	0.0504730109482718	0.313685347222913	0.687394989990282\\
0.77	0.0521269574244539	0.318648231061428	0.685006090465981\\
0.77	0.0538089243380495	0.323608480015097	0.682718662029127\\
0.77	0.0555189414985328	0.328565868787292	0.680522273853468\\
0.77	0.0572570364191156	0.333520171232163	0.678407456997851\\
0.77	0.0590232342988274	0.338471160382329	0.676365666960044\\
0.77	0.0608175580049697	0.34341860847696	0.674389235783156\\
0.77	0.0626400280559541	0.348362286990222	0.672471317134863\\
0.77	0.064490662604533	0.3533019666601	0.670605827172184\\
0.77	0.0663694774214291	0.358237417517579	0.668787383459803\\
0.77	0.0682764858793752	0.363168408916186	0.667011243730728\\
0.77	0.0702116989375697	0.368094709561873	0.665273245863649\\
0.77	0.0721751251265572	0.373016087543259	0.663569750098932\\
0.77	0.0741667705335422	0.377932310362199	0.661897584220061\\
0.77	0.0761866387881432	0.382843144964686	0.660253992184442\\
0.77	0.0782347310485954	0.38774835777208	0.658636586490995\\
0.77	0.0803110459884098	0.392647714712651	0.657043304415872\\
0.77	0.0824155797834956	0.397540981253432	0.655472368126511\\
0.77	0.084548326099754	0.402427922432375	0.653922248592745\\
0.77	0.0867092760811502	0.4073083028908	0.652391633146569\\
0.77	0.0888984183382709	0.412181886906127	0.650879396495578\\
0.77	0.0911157389373742	0.417048438424896	0.649384574964845\\
0.77	0.0933612213899392	0.421907721096044	0.647906343724684\\
0.77	0.0956348466427212	0.42675949830445	0.646443996754805\\
0.77	0.0979365930683194	0.431603533204733	0.644996929295962\\
0.77	0.100266436456264	0.436439588755293	0.643564622546848\\
0.77	0.102624350004627	0.441267427752585	0.642146630374429\\
0.77	0.105010304312169	0.446086812865616	0.640742567819432\\
0.77	0.107424267371012	0.450897506670668	0.639352101193729\\
0.77	0.109866204559871	0.455699271686218	0.637974939582442\\
0.77	0.112336078637817	0.460491870408058	0.636610827579863\\
0.77	0.114833849738607	0.465275065344603	0.635259539104383\\
0.77	0.117359475365564	0.470048619052377	0.633920872153155\\
0.77	0.119912910387023	0.474812294171664	0.632594644371966\\
0.77	0.122494107032347	0.479565853462319	0.631280689329641\\
0.77	0.125103014888515	0.484309059839721	0.629978853399058\\
0.77	0.127739580897283	0.489041676410868	0.628690460824557\\
0.775	0	0	0.710202759607373\\
0.775	1.11327767495586e-05	0.00471862581271274	0.713003157837853\\
0.775	4.46175251031912e-05	0.009446325184051	0.715846486660142\\
0.775	0.000100583311362513	0.0141829653360114	0.718788838379794\\
0.775	0.000179158434668431	0.018928411755669	0.721886092228356\\
0.775	0.000280470402701511	0.02368252819604	0.725200636220892\\
0.775	0.000404645907256436	0.0284451766772965	0.728799085274723\\
0.775	0.000551810799695644	0.0332162174883388	0.732749123983336\\
0.775	0.000722090066287311	0.037995509188729	0.737115575253327\\
0.775	0.000915607803432999	0.0427829086109896	0.741955893907011\\
0.775	0.0011324871927904	0.047578270863273	0.747315373652289\\
0.775	0.00137285047629673	0.0523814493324038	0.753222424045811\\
0.775	0.00163681893109844	0.057192295687301	0.759684310525652\\
0.775	0.00192451284439304	0.0620106598827802	0.766683748102665\\
0.775	0.00223605148818898	0.0668363901637434	0.774176695511257\\
0.775	0.00257155309398959	0.0716693330697584	0.782091614494493\\
0.775	0.00293113482740716	0.0765093334400312	0.79033034642109\\
0.775	0.00331491276271365	0.0813562344187764	0.79877062752665\\
0.775	0.00372300185733413	0.0862098774609879	0.80727012893036\\
0.775	0.00415551592628969	0.0910701023386145	0.815671782700193\\
0.775	0.00461256761659621	0.0959367471471425	0.823810053507331\\
0.775	0.00509426838162598	0.100809648312589	0.831517746503354\\
0.775	0.00560072845543873	0.105688640598912	0.838632911419452\\
0.775	0.00613205682708918	0.11057355711583	0.845005411376294\\
0.775	0.00668836121491816	0.115464229327074	0.850502769049302\\
0.775	0.00726974804083431	0.120360487059049	0.855014975717657\\
0.775	0.00787632240459385	0.125262158509929	0.858458041096605\\
0.775	0.00850818805808554	0.13016907025918	0.860776163494182\\
0.775	0.00916544737962849	0.135081047277508	0.861942500814226\\
0.775	0.00984820134829017	0.139997912937243	0.861958614648419\\
0.775	0.0105565495182326	0.144919489023162	0.86085273561709\\
0.775	0.0112905899930939	0.149845595743738	0.858677054109649\\
0.775	0.012050419400414	0.154776051742839	0.855504274967455\\
0.775	0.0128361328661109	0.159710674111862	0.851423687967753\\
0.775	0.0136478239890177	0.164649278402306	0.84653700048434\\
0.775	0.0144855848154856	0.169591678638796	0.840954157904248\\
0.775	0.0153495058140643	0.174537687332543	0.834789345403935\\
0.775	0.0162396758502651	0.179487115495259	0.8281573258062\\
0.775	0.0171561821614173	0.184439772653511	0.821170226448696\\
0.775	0.0180991103316243	0.189395466863524	0.813934846725691\\
0.775	0.0190685442668299	0.194354004726437	0.806550519904214\\
0.775	0.0200645661700016	0.199315191403999	0.799107529846306\\
0.775	0.0210872565164405	0.204278830634725	0.791686056496193\\
0.775	0.0221366940292259	0.20924472475049	0.7843556038306\\
0.775	0.023212955654804	0.214212674693577	0.777174850281118\\
0.775	0.0243161165387281	0.219182480034174	0.770191853862642\\
0.775	0.025446250001561	0.224153938988321	0.763444541554457\\
0.775	0.0266034275149466	0.229126848436298	0.756961413910481\\
0.775	0.0277877186778607	0.234101003941464	0.750762400416764\\
0.775	0.0289991911930496	0.239076199769547	0.744859807805785\\
0.775	0.0302379108436654	0.244052228908366	0.73925931151808\\
0.775	0.0315039414701067	0.24902888308801	0.733960949044505\\
0.775	0.0327973449470747	0.254005952801449	0.72896008240726\\
0.775	0.0341181811608519	0.258983227325591	0.724248305111177\\
0.775	0.0354665079868145	0.263960494742775	0.719814276220682\\
0.775	0.0368423812671862	0.268937541962694	0.715644470612629\\
0.775	0.0382458547890421	0.273914154744766	0.711723839838043\\
0.775	0.0396769802625738	0.278890117720924	0.708036382389731\\
0.775	0.0411358072996216	0.283865214418836	0.704565625567162\\
0.775	0.0426223833924861	0.288839227285554	0.701295023640316\\
0.775	0.0441367538930258	0.293811937711588	0.698208278749354\\
0.775	0.0456789619920503	0.298783126055387	0.695289592055348\\
0.775	0.0472490486990192	0.303752571668251	0.692523853197513\\
0.775	0.0488470528220538	0.308720052919643	0.689896776228119\\
0.775	0.0504730109482718	0.313685347222913	0.687394989990281\\
0.775	0.0521269574244539	0.318648231061428	0.685006090465981\\
0.775	0.0538089243380495	0.323608480015096	0.682718662029127\\
0.775	0.0555189414985328	0.328565868787292	0.680522273853469\\
0.775	0.0572570364191156	0.333520171232163	0.678407456997851\\
0.775	0.0590232342988274	0.338471160382329	0.676365666960043\\
0.775	0.0608175580049697	0.34341860847696	0.674389235783156\\
0.775	0.0626400280559541	0.348362286990222	0.672471317134862\\
0.775	0.064490662604533	0.3533019666601	0.670605827172182\\
0.775	0.0663694774214291	0.358237417517579	0.668787383459805\\
0.775	0.0682764858793752	0.363168408916186	0.667011243730728\\
0.775	0.0702116989375697	0.368094709561873	0.665273245863649\\
0.775	0.0721751251265572	0.373016087543259	0.663569750098933\\
0.775	0.0741667705335422	0.377932310362199	0.661897584220061\\
0.775	0.0761866387881432	0.382843144964686	0.660253992184442\\
0.775	0.0782347310485954	0.38774835777208	0.658636586490997\\
0.775	0.0803110459884098	0.392647714712651	0.65704330441587\\
0.775	0.0824155797834956	0.397540981253432	0.655472368126509\\
0.775	0.084548326099754	0.402427922432375	0.653922248592748\\
0.775	0.0867092760811502	0.4073083028908	0.652391633146572\\
0.775	0.0888984183382709	0.412181886906127	0.650879396495578\\
0.775	0.0911157389373742	0.417048438424897	0.649384574964843\\
0.775	0.0933612213899392	0.421907721096044	0.647906343724685\\
0.775	0.0956348466427212	0.42675949830445	0.646443996754806\\
0.775	0.0979365930683194	0.431603533204733	0.644996929295961\\
0.775	0.100266436456264	0.436439588755293	0.643564622546847\\
0.775	0.102624350004627	0.441267427752585	0.642146630374432\\
0.775	0.105010304312169	0.446086812865616	0.640742567819433\\
0.775	0.107424267371012	0.450897506670668	0.63935210119373\\
0.775	0.109866204559871	0.455699271686218	0.637974939582441\\
0.775	0.112336078637817	0.460491870408058	0.636610827579861\\
0.775	0.114833849738607	0.465275065344603	0.635259539104383\\
0.775	0.117359475365564	0.470048619052377	0.633920872153155\\
0.775	0.119912910387022	0.474812294171664	0.632594644371965\\
0.775	0.122494107032347	0.479565853462319	0.631280689329642\\
0.775	0.125103014888515	0.484309059839721	0.629978853399059\\
0.775	0.127739580897283	0.489041676410868	0.628690460824557\\
0.78	0	0	0.710202759607373\\
0.78	1.11327767495586e-05	0.00471862581271274	0.713003157837853\\
0.78	4.46175251031912e-05	0.009446325184051	0.715846486660142\\
0.78	0.000100583311362513	0.0141829653360114	0.718788838379794\\
0.78	0.000179158434668431	0.018928411755669	0.721886092228356\\
0.78	0.000280470402701511	0.02368252819604	0.725200636220892\\
0.78	0.000404645907256436	0.0284451766772965	0.728799085274723\\
0.78	0.000551810799695644	0.0332162174883388	0.732749123983336\\
0.78	0.000722090066287311	0.037995509188729	0.737115575253327\\
0.78	0.000915607803432999	0.0427829086109896	0.741955893907011\\
0.78	0.0011324871927904	0.0475782708632729	0.747315373652289\\
0.78	0.00137285047629673	0.0523814493324038	0.753222424045811\\
0.78	0.00163681893109844	0.057192295687301	0.759684310525653\\
0.78	0.00192451284439304	0.0620106598827802	0.766683748102665\\
0.78	0.00223605148818898	0.0668363901637434	0.774176695511256\\
0.78	0.00257155309398959	0.0716693330697584	0.782091614494494\\
0.78	0.00293113482740716	0.0765093334400312	0.79033034642109\\
0.78	0.00331491276271365	0.0813562344187764	0.79877062752665\\
0.78	0.00372300185733413	0.0862098774609879	0.807270128930359\\
0.78	0.00415551592628969	0.0910701023386145	0.815671782700194\\
0.78	0.00461256761659621	0.0959367471471425	0.82381005350733\\
0.78	0.00509426838162598	0.100809648312589	0.831517746503354\\
0.78	0.00560072845543873	0.105688640598912	0.838632911419451\\
0.78	0.00613205682708918	0.11057355711583	0.845005411376293\\
0.78	0.00668836121491816	0.115464229327074	0.850502769049303\\
0.78	0.00726974804083431	0.120360487059049	0.855014975717657\\
0.78	0.00787632240459385	0.125262158509929	0.858458041096606\\
0.78	0.00850818805808555	0.13016907025918	0.860776163494183\\
0.78	0.00916544737962849	0.135081047277508	0.861942500814224\\
0.78	0.00984820134829017	0.139997912937243	0.86195861464842\\
0.78	0.0105565495182326	0.144919489023162	0.860852735617093\\
0.78	0.0112905899930939	0.149845595743738	0.858677054109653\\
0.78	0.012050419400414	0.154776051742839	0.855504274967457\\
0.78	0.0128361328661109	0.159710674111862	0.851423687967755\\
0.78	0.0136478239890177	0.164649278402306	0.846537000484342\\
0.78	0.0144855848154856	0.169591678638796	0.840954157904247\\
0.78	0.0153495058140643	0.174537687332543	0.834789345403935\\
0.78	0.0162396758502651	0.179487115495259	0.8281573258062\\
0.78	0.0171561821614173	0.184439772653511	0.821170226448693\\
0.78	0.0180991103316243	0.189395466863524	0.813934846725692\\
0.78	0.0190685442668299	0.194354004726437	0.806550519904213\\
0.78	0.0200645661700016	0.199315191403999	0.799107529846304\\
0.78	0.0210872565164405	0.204278830634725	0.791686056496193\\
0.78	0.0221366940292259	0.20924472475049	0.784355603830601\\
0.78	0.023212955654804	0.214212674693577	0.777174850281118\\
0.78	0.0243161165387281	0.219182480034174	0.770191853862645\\
0.78	0.025446250001561	0.224153938988321	0.763444541554459\\
0.78	0.0266034275149466	0.229126848436298	0.756961413910484\\
0.78	0.0277877186778607	0.234101003941464	0.750762400416765\\
0.78	0.0289991911930496	0.239076199769547	0.744859807805784\\
0.78	0.0302379108436654	0.244052228908366	0.739259311518081\\
0.78	0.0315039414701067	0.24902888308801	0.733960949044504\\
0.78	0.0327973449470747	0.254005952801449	0.728960082407259\\
0.78	0.0341181811608519	0.258983227325591	0.724248305111179\\
0.78	0.0354665079868145	0.263960494742775	0.719814276220681\\
0.78	0.0368423812671862	0.268937541962694	0.715644470612629\\
0.78	0.0382458547890421	0.273914154744766	0.711723839838043\\
0.78	0.0396769802625738	0.278890117720924	0.70803638238973\\
0.78	0.0411358072996216	0.283865214418836	0.704565625567163\\
0.78	0.0426223833924861	0.288839227285554	0.701295023640316\\
0.78	0.0441367538930258	0.293811937711588	0.698208278749353\\
0.78	0.0456789619920503	0.298783126055387	0.69528959205535\\
0.78	0.0472490486990192	0.303752571668251	0.692523853197513\\
0.78	0.0488470528220537	0.308720052919643	0.689896776228118\\
0.78	0.0504730109482718	0.313685347222913	0.68739498999028\\
0.78	0.0521269574244539	0.318648231061427	0.685006090465981\\
0.78	0.0538089243380495	0.323608480015096	0.682718662029128\\
0.78	0.0555189414985328	0.328565868787292	0.680522273853469\\
0.78	0.0572570364191156	0.333520171232163	0.678407456997851\\
0.78	0.0590232342988274	0.338471160382329	0.676365666960045\\
0.78	0.0608175580049697	0.34341860847696	0.674389235783156\\
0.78	0.0626400280559541	0.348362286990222	0.672471317134862\\
0.78	0.064490662604533	0.3533019666601	0.670605827172182\\
0.78	0.0663694774214291	0.358237417517579	0.668787383459803\\
0.78	0.0682764858793752	0.363168408916186	0.667011243730728\\
0.78	0.0702116989375697	0.368094709561873	0.66527324586365\\
0.78	0.0721751251265572	0.373016087543259	0.663569750098932\\
0.78	0.0741667705335422	0.377932310362199	0.66189758422006\\
0.78	0.0761866387881432	0.382843144964686	0.660253992184443\\
0.78	0.0782347310485954	0.38774835777208	0.658636586490998\\
0.78	0.0803110459884098	0.392647714712651	0.65704330441587\\
0.78	0.0824155797834956	0.397540981253432	0.655472368126507\\
0.78	0.084548326099754	0.402427922432375	0.653922248592745\\
0.78	0.0867092760811502	0.407308302890799	0.652391633146571\\
0.78	0.0888984183382709	0.412181886906127	0.650879396495581\\
0.78	0.0911157389373742	0.417048438424896	0.649384574964844\\
0.78	0.0933612213899392	0.421907721096044	0.647906343724684\\
0.78	0.0956348466427212	0.42675949830445	0.646443996754806\\
0.78	0.0979365930683194	0.431603533204733	0.644996929295963\\
0.78	0.100266436456264	0.436439588755293	0.643564622546846\\
0.78	0.102624350004627	0.441267427752584	0.642146630374429\\
0.78	0.105010304312169	0.446086812865616	0.640742567819435\\
0.78	0.107424267371012	0.450897506670668	0.639352101193731\\
0.78	0.109866204559871	0.455699271686218	0.637974939582443\\
0.78	0.112336078637817	0.460491870408058	0.636610827579862\\
0.78	0.114833849738607	0.465275065344603	0.635259539104382\\
0.78	0.117359475365564	0.470048619052377	0.633920872153155\\
0.78	0.119912910387023	0.474812294171664	0.632594644371967\\
0.78	0.122494107032347	0.479565853462319	0.631280689329641\\
0.78	0.125103014888515	0.484309059839721	0.629978853399059\\
0.78	0.127739580897283	0.489041676410868	0.628690460824558\\
0.785	0	0	0.710202759607373\\
0.785	1.11327767495586e-05	0.00471862581271275	0.713003157837853\\
0.785	4.46175251031912e-05	0.009446325184051	0.715846486660142\\
0.785	0.000100583311362513	0.0141829653360114	0.718788838379794\\
0.785	0.000179158434668431	0.018928411755669	0.721886092228356\\
0.785	0.000280470402701511	0.02368252819604	0.725200636220892\\
0.785	0.000404645907256436	0.0284451766772965	0.728799085274723\\
0.785	0.000551810799695644	0.0332162174883388	0.732749123983336\\
0.785	0.000722090066287311	0.037995509188729	0.737115575253328\\
0.785	0.000915607803432999	0.0427829086109896	0.741955893907011\\
0.785	0.0011324871927904	0.047578270863273	0.747315373652289\\
0.785	0.00137285047629673	0.0523814493324038	0.75322242404581\\
0.785	0.00163681893109844	0.057192295687301	0.759684310525652\\
0.785	0.00192451284439304	0.0620106598827802	0.766683748102665\\
0.785	0.00223605148818898	0.0668363901637434	0.774176695511256\\
0.785	0.00257155309398959	0.0716693330697584	0.782091614494494\\
0.785	0.00293113482740716	0.0765093334400312	0.79033034642109\\
0.785	0.00331491276271365	0.0813562344187764	0.79877062752665\\
0.785	0.00372300185733413	0.0862098774609879	0.807270128930359\\
0.785	0.00415551592628969	0.0910701023386145	0.815671782700194\\
0.785	0.00461256761659621	0.0959367471471424	0.823810053507331\\
0.785	0.00509426838162598	0.100809648312589	0.831517746503354\\
0.785	0.00560072845543873	0.105688640598912	0.838632911419452\\
0.785	0.00613205682708918	0.11057355711583	0.845005411376292\\
0.785	0.00668836121491816	0.115464229327074	0.850502769049302\\
0.785	0.00726974804083431	0.120360487059049	0.855014975717657\\
0.785	0.00787632240459385	0.125262158509929	0.858458041096606\\
0.785	0.00850818805808555	0.13016907025918	0.860776163494182\\
0.785	0.00916544737962849	0.135081047277508	0.861942500814224\\
0.785	0.00984820134829017	0.139997912937243	0.861958614648421\\
0.785	0.0105565495182326	0.144919489023162	0.860852735617091\\
0.785	0.0112905899930939	0.149845595743738	0.858677054109651\\
0.785	0.012050419400414	0.154776051742839	0.855504274967457\\
0.785	0.0128361328661109	0.159710674111862	0.851423687967756\\
0.785	0.0136478239890177	0.164649278402306	0.84653700048434\\
0.785	0.0144855848154856	0.169591678638796	0.840954157904248\\
0.785	0.0153495058140643	0.174537687332543	0.834789345403934\\
0.785	0.0162396758502651	0.179487115495259	0.828157325806199\\
0.785	0.0171561821614173	0.184439772653511	0.821170226448692\\
0.785	0.0180991103316243	0.189395466863524	0.813934846725694\\
0.785	0.0190685442668299	0.194354004726437	0.806550519904211\\
0.785	0.0200645661700016	0.199315191403999	0.799107529846304\\
0.785	0.0210872565164405	0.204278830634725	0.791686056496192\\
0.785	0.0221366940292259	0.20924472475049	0.784355603830602\\
0.785	0.023212955654804	0.214212674693577	0.777174850281119\\
0.785	0.0243161165387281	0.219182480034174	0.770191853862646\\
0.785	0.025446250001561	0.224153938988321	0.763444541554459\\
0.785	0.0266034275149466	0.229126848436298	0.756961413910483\\
0.785	0.0277877186778607	0.234101003941464	0.750762400416763\\
0.785	0.0289991911930496	0.239076199769547	0.744859807805786\\
0.785	0.0302379108436654	0.244052228908366	0.73925931151808\\
0.785	0.0315039414701067	0.24902888308801	0.733960949044503\\
0.785	0.0327973449470747	0.254005952801449	0.728960082407262\\
0.785	0.0341181811608519	0.258983227325592	0.724248305111178\\
0.785	0.0354665079868145	0.263960494742775	0.71981427622068\\
0.785	0.0368423812671862	0.268937541962694	0.715644470612631\\
0.785	0.0382458547890421	0.273914154744766	0.711723839838041\\
0.785	0.0396769802625738	0.278890117720924	0.70803638238973\\
0.785	0.0411358072996216	0.283865214418836	0.704565625567166\\
0.785	0.0426223833924861	0.288839227285554	0.701295023640317\\
0.785	0.0441367538930258	0.293811937711588	0.698208278749353\\
0.785	0.0456789619920503	0.298783126055387	0.695289592055349\\
0.785	0.0472490486990192	0.303752571668251	0.692523853197512\\
0.785	0.0488470528220538	0.308720052919643	0.689896776228119\\
0.785	0.0504730109482718	0.313685347222913	0.687394989990281\\
0.785	0.0521269574244539	0.318648231061428	0.68500609046598\\
0.785	0.0538089243380495	0.323608480015097	0.682718662029128\\
0.785	0.0555189414985328	0.328565868787292	0.680522273853468\\
0.785	0.0572570364191156	0.333520171232163	0.678407456997848\\
0.785	0.0590232342988274	0.338471160382329	0.676365666960045\\
0.785	0.0608175580049697	0.34341860847696	0.674389235783157\\
0.785	0.0626400280559541	0.348362286990222	0.672471317134862\\
0.785	0.064490662604533	0.3533019666601	0.670605827172182\\
0.785	0.0663694774214291	0.358237417517579	0.668787383459804\\
0.785	0.0682764858793752	0.363168408916186	0.667011243730728\\
0.785	0.0702116989375697	0.368094709561873	0.66527324586365\\
0.785	0.0721751251265572	0.373016087543259	0.663569750098932\\
0.785	0.0741667705335422	0.377932310362199	0.66189758422006\\
0.785	0.0761866387881432	0.382843144964686	0.660253992184442\\
0.785	0.0782347310485954	0.38774835777208	0.658636586490997\\
0.785	0.0803110459884098	0.392647714712651	0.657043304415871\\
0.785	0.0824155797834956	0.397540981253432	0.655472368126509\\
0.785	0.084548326099754	0.402427922432375	0.653922248592745\\
0.785	0.0867092760811502	0.4073083028908	0.652391633146569\\
0.785	0.0888984183382709	0.412181886906127	0.650879396495578\\
0.785	0.0911157389373742	0.417048438424896	0.649384574964846\\
0.785	0.0933612213899392	0.421907721096044	0.647906343724685\\
0.785	0.0956348466427212	0.42675949830445	0.646443996754804\\
0.785	0.0979365930683194	0.431603533204733	0.644996929295962\\
0.785	0.100266436456264	0.436439588755293	0.643564622546847\\
0.785	0.102624350004627	0.441267427752584	0.642146630374428\\
0.785	0.105010304312169	0.446086812865616	0.640742567819433\\
0.785	0.107424267371012	0.450897506670668	0.639352101193732\\
0.785	0.109866204559871	0.455699271686218	0.637974939582443\\
0.785	0.112336078637817	0.460491870408058	0.636610827579861\\
0.785	0.114833849738607	0.465275065344603	0.635259539104383\\
0.785	0.117359475365564	0.470048619052377	0.633920872153154\\
0.785	0.119912910387022	0.474812294171664	0.632594644371966\\
0.785	0.122494107032347	0.479565853462319	0.631280689329642\\
0.785	0.125103014888515	0.484309059839721	0.62997885339906\\
0.785	0.127739580897283	0.489041676410868	0.62869046082456\\
0.79	0	0	0.710202759607373\\
0.79	1.11327767495586e-05	0.00471862581271275	0.713003157837853\\
0.79	4.46175251031912e-05	0.009446325184051	0.715846486660142\\
0.79	0.000100583311362513	0.0141829653360114	0.718788838379794\\
0.79	0.000179158434668431	0.018928411755669	0.721886092228356\\
0.79	0.000280470402701511	0.02368252819604	0.725200636220892\\
0.79	0.000404645907256436	0.0284451766772965	0.728799085274723\\
0.79	0.000551810799695644	0.0332162174883388	0.732749123983336\\
0.79	0.000722090066287311	0.037995509188729	0.737115575253327\\
0.79	0.000915607803432999	0.0427829086109896	0.741955893907011\\
0.79	0.0011324871927904	0.047578270863273	0.747315373652289\\
0.79	0.00137285047629673	0.0523814493324038	0.753222424045811\\
0.79	0.00163681893109844	0.057192295687301	0.759684310525652\\
0.79	0.00192451284439304	0.0620106598827802	0.766683748102665\\
0.79	0.00223605148818898	0.0668363901637434	0.774176695511256\\
0.79	0.00257155309398959	0.0716693330697584	0.782091614494494\\
0.79	0.00293113482740716	0.0765093334400312	0.79033034642109\\
0.79	0.00331491276271365	0.0813562344187764	0.79877062752665\\
0.79	0.00372300185733413	0.0862098774609879	0.807270128930359\\
0.79	0.00415551592628969	0.0910701023386145	0.815671782700193\\
0.79	0.00461256761659621	0.0959367471471425	0.823810053507331\\
0.79	0.00509426838162598	0.100809648312589	0.831517746503354\\
0.79	0.00560072845543873	0.105688640598912	0.838632911419452\\
0.79	0.00613205682708918	0.11057355711583	0.845005411376293\\
0.79	0.00668836121491816	0.115464229327074	0.850502769049301\\
0.79	0.00726974804083431	0.120360487059049	0.855014975717657\\
0.79	0.00787632240459385	0.125262158509929	0.858458041096607\\
0.79	0.00850818805808555	0.13016907025918	0.860776163494183\\
0.79	0.00916544737962849	0.135081047277508	0.861942500814224\\
0.79	0.00984820134829017	0.139997912937243	0.861958614648419\\
0.79	0.0105565495182326	0.144919489023162	0.860852735617089\\
0.79	0.0112905899930939	0.149845595743738	0.858677054109649\\
0.79	0.012050419400414	0.154776051742839	0.855504274967456\\
0.79	0.0128361328661109	0.159710674111862	0.851423687967754\\
0.79	0.0136478239890177	0.164649278402306	0.846537000484341\\
0.79	0.0144855848154856	0.169591678638796	0.84095415790425\\
0.79	0.0153495058140643	0.174537687332543	0.834789345403935\\
0.79	0.0162396758502651	0.179487115495259	0.8281573258062\\
0.79	0.0171561821614173	0.184439772653511	0.821170226448695\\
0.79	0.0180991103316243	0.189395466863524	0.813934846725692\\
0.79	0.0190685442668299	0.194354004726437	0.806550519904211\\
0.79	0.0200645661700016	0.199315191403999	0.799107529846304\\
0.79	0.0210872565164405	0.204278830634725	0.791686056496195\\
0.79	0.0221366940292259	0.20924472475049	0.784355603830603\\
0.79	0.023212955654804	0.214212674693577	0.77717485028112\\
0.79	0.0243161165387281	0.219182480034174	0.770191853862645\\
0.79	0.025446250001561	0.224153938988321	0.763444541554458\\
0.79	0.0266034275149466	0.229126848436298	0.756961413910482\\
0.79	0.0277877186778607	0.234101003941464	0.750762400416764\\
0.79	0.0289991911930496	0.239076199769547	0.744859807805785\\
0.79	0.0302379108436654	0.244052228908366	0.739259311518077\\
0.79	0.0315039414701067	0.24902888308801	0.733960949044504\\
0.79	0.0327973449470747	0.254005952801449	0.728960082407262\\
0.79	0.0341181811608519	0.258983227325592	0.724248305111177\\
0.79	0.0354665079868145	0.263960494742775	0.719814276220681\\
0.79	0.0368423812671862	0.268937541962694	0.715644470612629\\
0.79	0.0382458547890421	0.273914154744766	0.711723839838043\\
0.79	0.0396769802625738	0.278890117720924	0.708036382389731\\
0.79	0.0411358072996216	0.283865214418836	0.704565625567163\\
0.79	0.0426223833924861	0.288839227285554	0.701295023640316\\
0.79	0.0441367538930258	0.293811937711588	0.698208278749353\\
0.79	0.0456789619920503	0.298783126055387	0.69528959205535\\
0.79	0.0472490486990192	0.303752571668251	0.692523853197512\\
0.79	0.0488470528220538	0.308720052919643	0.689896776228119\\
0.79	0.0504730109482718	0.313685347222913	0.687394989990283\\
0.79	0.0521269574244539	0.318648231061428	0.68500609046598\\
0.79	0.0538089243380495	0.323608480015096	0.682718662029127\\
0.79	0.0555189414985328	0.328565868787292	0.68052227385347\\
0.79	0.0572570364191156	0.333520171232163	0.678407456997851\\
0.79	0.0590232342988274	0.338471160382329	0.676365666960043\\
0.79	0.0608175580049697	0.34341860847696	0.674389235783156\\
0.79	0.0626400280559541	0.348362286990222	0.672471317134862\\
0.79	0.064490662604533	0.3533019666601	0.670605827172182\\
0.79	0.0663694774214291	0.358237417517579	0.668787383459803\\
0.79	0.0682764858793752	0.363168408916186	0.667011243730728\\
0.79	0.0702116989375697	0.368094709561873	0.665273245863649\\
0.79	0.0721751251265572	0.373016087543259	0.663569750098933\\
0.79	0.0741667705335422	0.377932310362199	0.661897584220061\\
0.79	0.0761866387881432	0.382843144964686	0.660253992184442\\
0.79	0.0782347310485954	0.38774835777208	0.658636586490997\\
0.79	0.0803110459884098	0.392647714712651	0.657043304415871\\
0.79	0.0824155797834956	0.397540981253432	0.65547236812651\\
0.79	0.084548326099754	0.402427922432375	0.653922248592746\\
0.79	0.0867092760811502	0.4073083028908	0.65239163314657\\
0.79	0.0888984183382709	0.412181886906127	0.650879396495578\\
0.79	0.0911157389373742	0.417048438424896	0.649384574964842\\
0.79	0.0933612213899392	0.421907721096044	0.647906343724686\\
0.79	0.0956348466427212	0.42675949830445	0.646443996754806\\
0.79	0.0979365930683194	0.431603533204733	0.644996929295962\\
0.79	0.100266436456264	0.436439588755293	0.643564622546847\\
0.79	0.102624350004627	0.441267427752584	0.642146630374429\\
0.79	0.105010304312169	0.446086812865616	0.640742567819432\\
0.79	0.107424267371012	0.450897506670668	0.639352101193731\\
0.79	0.109866204559871	0.455699271686218	0.637974939582444\\
0.79	0.112336078637817	0.460491870408058	0.636610827579862\\
0.79	0.114833849738607	0.465275065344603	0.635259539104382\\
0.79	0.117359475365564	0.470048619052377	0.633920872153154\\
0.79	0.119912910387023	0.474812294171664	0.632594644371965\\
0.79	0.122494107032347	0.479565853462319	0.63128068932964\\
0.79	0.125103014888515	0.484309059839721	0.629978853399058\\
0.79	0.127739580897283	0.489041676410868	0.628690460824562\\
0.795	0	0	0.710202759607373\\
0.795	1.11327767495586e-05	0.00471862581271274	0.713003157837853\\
0.795	4.46175251031912e-05	0.009446325184051	0.715846486660142\\
0.795	0.000100583311362513	0.0141829653360114	0.718788838379794\\
0.795	0.000179158434668431	0.018928411755669	0.721886092228356\\
0.795	0.000280470402701511	0.02368252819604	0.725200636220892\\
0.795	0.000404645907256436	0.0284451766772965	0.728799085274723\\
0.795	0.000551810799695644	0.0332162174883388	0.732749123983336\\
0.795	0.000722090066287311	0.037995509188729	0.737115575253327\\
0.795	0.000915607803432999	0.0427829086109896	0.741955893907011\\
0.795	0.0011324871927904	0.0475782708632729	0.747315373652289\\
0.795	0.00137285047629673	0.0523814493324038	0.753222424045811\\
0.795	0.00163681893109844	0.057192295687301	0.759684310525652\\
0.795	0.00192451284439304	0.0620106598827802	0.766683748102665\\
0.795	0.00223605148818898	0.0668363901637434	0.774176695511256\\
0.795	0.00257155309398959	0.0716693330697584	0.782091614494493\\
0.795	0.00293113482740716	0.0765093334400312	0.79033034642109\\
0.795	0.00331491276271365	0.0813562344187764	0.79877062752665\\
0.795	0.00372300185733413	0.0862098774609879	0.807270128930359\\
0.795	0.00415551592628969	0.0910701023386145	0.815671782700193\\
0.795	0.00461256761659621	0.0959367471471425	0.82381005350733\\
0.795	0.00509426838162598	0.100809648312589	0.831517746503354\\
0.795	0.00560072845543873	0.105688640598912	0.838632911419451\\
0.795	0.00613205682708918	0.11057355711583	0.845005411376293\\
0.795	0.00668836121491816	0.115464229327074	0.850502769049302\\
0.795	0.00726974804083431	0.120360487059049	0.855014975717657\\
0.795	0.00787632240459385	0.125262158509929	0.858458041096606\\
0.795	0.00850818805808554	0.13016907025918	0.860776163494183\\
0.795	0.00916544737962849	0.135081047277508	0.861942500814225\\
0.795	0.00984820134829017	0.139997912937243	0.86195861464842\\
0.795	0.0105565495182326	0.144919489023162	0.860852735617092\\
0.795	0.0112905899930939	0.149845595743738	0.858677054109652\\
0.795	0.012050419400414	0.154776051742839	0.855504274967458\\
0.795	0.0128361328661109	0.159710674111862	0.851423687967756\\
0.795	0.0136478239890177	0.164649278402306	0.846537000484341\\
0.795	0.0144855848154856	0.169591678638796	0.84095415790425\\
0.795	0.0153495058140643	0.174537687332543	0.834789345403936\\
0.795	0.0162396758502651	0.179487115495259	0.828157325806199\\
0.795	0.0171561821614173	0.184439772653511	0.821170226448694\\
0.795	0.0180991103316243	0.189395466863524	0.813934846725691\\
0.795	0.0190685442668299	0.194354004726437	0.806550519904212\\
0.795	0.0200645661700016	0.199315191403999	0.799107529846306\\
0.795	0.0210872565164405	0.204278830634725	0.791686056496195\\
0.795	0.0221366940292259	0.20924472475049	0.784355603830603\\
0.795	0.023212955654804	0.214212674693577	0.77717485028112\\
0.795	0.0243161165387281	0.219182480034174	0.770191853862642\\
0.795	0.025446250001561	0.224153938988321	0.763444541554457\\
0.795	0.0266034275149466	0.229126848436298	0.756961413910482\\
0.795	0.0277877186778607	0.234101003941464	0.750762400416764\\
0.795	0.0289991911930496	0.239076199769547	0.744859807805783\\
0.795	0.0302379108436654	0.244052228908366	0.739259311518081\\
0.795	0.0315039414701067	0.24902888308801	0.733960949044504\\
0.795	0.0327973449470747	0.254005952801449	0.728960082407259\\
0.795	0.0341181811608519	0.258983227325591	0.724248305111178\\
0.795	0.0354665079868145	0.263960494742775	0.719814276220682\\
0.795	0.0368423812671862	0.268937541962694	0.715644470612629\\
0.795	0.0382458547890421	0.273914154744766	0.711723839838042\\
0.795	0.0396769802625738	0.278890117720924	0.708036382389732\\
0.795	0.0411358072996216	0.283865214418836	0.704565625567163\\
0.795	0.0426223833924861	0.288839227285554	0.701295023640315\\
0.795	0.0441367538930258	0.293811937711588	0.698208278749355\\
0.795	0.0456789619920503	0.298783126055387	0.69528959205535\\
0.795	0.0472490486990192	0.303752571668251	0.69252385319751\\
0.795	0.0488470528220538	0.308720052919643	0.689896776228119\\
0.795	0.0504730109482718	0.313685347222913	0.687394989990282\\
0.795	0.0521269574244539	0.318648231061428	0.685006090465981\\
0.795	0.0538089243380495	0.323608480015096	0.682718662029128\\
0.795	0.0555189414985328	0.328565868787292	0.680522273853469\\
0.795	0.0572570364191156	0.333520171232163	0.67840745699785\\
0.795	0.0590232342988274	0.338471160382329	0.676365666960044\\
0.795	0.0608175580049697	0.34341860847696	0.674389235783157\\
0.795	0.0626400280559541	0.348362286990222	0.672471317134862\\
0.795	0.064490662604533	0.3533019666601	0.670605827172182\\
0.795	0.0663694774214291	0.358237417517579	0.668787383459804\\
0.795	0.0682764858793752	0.363168408916186	0.667011243730728\\
0.795	0.0702116989375697	0.368094709561873	0.665273245863649\\
0.795	0.0721751251265572	0.373016087543259	0.663569750098932\\
0.795	0.0741667705335422	0.377932310362199	0.66189758422006\\
0.795	0.0761866387881432	0.382843144964686	0.660253992184442\\
0.795	0.0782347310485954	0.38774835777208	0.658636586490997\\
0.795	0.0803110459884098	0.392647714712651	0.657043304415871\\
0.795	0.0824155797834956	0.397540981253432	0.655472368126509\\
0.795	0.084548326099754	0.402427922432375	0.653922248592746\\
0.795	0.0867092760811502	0.407308302890799	0.65239163314657\\
0.795	0.0888984183382709	0.412181886906127	0.65087939649558\\
0.795	0.0911157389373742	0.417048438424897	0.649384574964843\\
0.795	0.0933612213899392	0.421907721096044	0.647906343724684\\
0.795	0.0956348466427212	0.42675949830445	0.646443996754806\\
0.795	0.0979365930683194	0.431603533204733	0.644996929295962\\
0.795	0.100266436456264	0.436439588755293	0.643564622546848\\
0.795	0.102624350004627	0.441267427752585	0.642146630374429\\
0.795	0.105010304312169	0.446086812865616	0.640742567819433\\
0.795	0.107424267371012	0.450897506670668	0.639352101193731\\
0.795	0.109866204559871	0.455699271686218	0.637974939582442\\
0.795	0.112336078637817	0.460491870408058	0.636610827579862\\
0.795	0.114833849738607	0.465275065344603	0.635259539104383\\
0.795	0.117359475365564	0.470048619052377	0.633920872153153\\
0.795	0.119912910387022	0.474812294171664	0.632594644371965\\
0.795	0.122494107032347	0.479565853462319	0.631280689329643\\
0.795	0.125103014888515	0.484309059839721	0.629978853399058\\
0.795	0.127739580897283	0.489041676410868	0.628690460824555\\
0.8	0	0	0.710202759607373\\
0.8	1.11327767495586e-05	0.00471862581271275	0.713003157837853\\
0.8	4.46175251031912e-05	0.009446325184051	0.715846486660142\\
0.8	0.000100583311362513	0.0141829653360114	0.718788838379794\\
0.8	0.000179158434668431	0.018928411755669	0.721886092228356\\
0.8	0.000280470402701511	0.02368252819604	0.725200636220892\\
0.8	0.000404645907256436	0.0284451766772965	0.728799085274723\\
0.8	0.000551810799695644	0.0332162174883388	0.732749123983336\\
0.8	0.000722090066287311	0.037995509188729	0.737115575253328\\
0.8	0.000915607803432999	0.0427829086109896	0.741955893907011\\
0.8	0.0011324871927904	0.0475782708632729	0.747315373652289\\
0.8	0.00137285047629673	0.0523814493324038	0.753222424045811\\
0.8	0.00163681893109844	0.057192295687301	0.759684310525652\\
0.8	0.00192451284439304	0.0620106598827802	0.766683748102665\\
0.8	0.00223605148818898	0.0668363901637434	0.774176695511256\\
0.8	0.00257155309398959	0.0716693330697584	0.782091614494493\\
0.8	0.00293113482740716	0.0765093334400312	0.79033034642109\\
0.8	0.00331491276271365	0.0813562344187764	0.79877062752665\\
0.8	0.00372300185733413	0.0862098774609879	0.807270128930359\\
0.8	0.00415551592628969	0.0910701023386145	0.815671782700193\\
0.8	0.00461256761659621	0.0959367471471425	0.82381005350733\\
0.8	0.00509426838162598	0.100809648312589	0.831517746503354\\
0.8	0.00560072845543873	0.105688640598912	0.838632911419452\\
0.8	0.00613205682708918	0.11057355711583	0.845005411376293\\
0.8	0.00668836121491816	0.115464229327074	0.850502769049302\\
0.8	0.00726974804083431	0.120360487059049	0.855014975717656\\
0.8	0.00787632240459385	0.125262158509929	0.858458041096605\\
0.8	0.00850818805808555	0.13016907025918	0.860776163494181\\
0.8	0.00916544737962849	0.135081047277508	0.861942500814225\\
0.8	0.00984820134829017	0.139997912937243	0.861958614648422\\
0.8	0.0105565495182326	0.144919489023162	0.860852735617092\\
0.8	0.0112905899930939	0.149845595743738	0.858677054109652\\
0.8	0.012050419400414	0.154776051742839	0.855504274967457\\
0.8	0.0128361328661109	0.159710674111862	0.851423687967754\\
0.8	0.0136478239890177	0.164649278402306	0.84653700048434\\
0.8	0.0144855848154856	0.169591678638796	0.840954157904248\\
0.8	0.0153495058140643	0.174537687332543	0.834789345403934\\
0.8	0.0162396758502651	0.179487115495259	0.828157325806199\\
0.8	0.0171561821614173	0.184439772653511	0.821170226448695\\
0.8	0.0180991103316243	0.189395466863524	0.813934846725692\\
0.8	0.0190685442668299	0.194354004726437	0.806550519904212\\
0.8	0.0200645661700016	0.199315191403999	0.799107529846305\\
0.8	0.0210872565164405	0.204278830634725	0.791686056496195\\
0.8	0.0221366940292259	0.20924472475049	0.784355603830604\\
0.8	0.023212955654804	0.214212674693577	0.777174850281115\\
0.8	0.0243161165387281	0.219182480034174	0.770191853862642\\
0.8	0.025446250001561	0.224153938988321	0.763444541554458\\
0.8	0.0266034275149466	0.229126848436298	0.756961413910484\\
0.8	0.0277877186778607	0.234101003941464	0.750762400416763\\
0.8	0.0289991911930496	0.239076199769547	0.744859807805786\\
0.8	0.0302379108436654	0.244052228908366	0.73925931151808\\
0.8	0.0315039414701067	0.24902888308801	0.733960949044503\\
0.8	0.0327973449470747	0.254005952801449	0.728960082407262\\
0.8	0.0341181811608519	0.258983227325592	0.724248305111177\\
0.8	0.0354665079868145	0.263960494742775	0.719814276220681\\
0.8	0.0368423812671862	0.268937541962694	0.71564447061263\\
0.8	0.0382458547890421	0.273914154744766	0.711723839838041\\
0.8	0.0396769802625738	0.278890117720924	0.708036382389731\\
0.8	0.0411358072996216	0.283865214418836	0.704565625567165\\
0.8	0.0426223833924861	0.288839227285554	0.701295023640316\\
0.8	0.0441367538930258	0.293811937711588	0.698208278749353\\
0.8	0.0456789619920503	0.298783126055387	0.695289592055351\\
0.8	0.0472490486990192	0.303752571668251	0.692523853197512\\
0.8	0.0488470528220538	0.308720052919643	0.689896776228118\\
0.8	0.0504730109482718	0.313685347222913	0.687394989990281\\
0.8	0.0521269574244539	0.318648231061428	0.685006090465981\\
0.8	0.0538089243380495	0.323608480015097	0.682718662029127\\
0.8	0.0555189414985328	0.328565868787292	0.680522273853469\\
0.8	0.0572570364191156	0.333520171232163	0.67840745699785\\
0.8	0.0590232342988274	0.338471160382329	0.676365666960044\\
0.8	0.0608175580049697	0.34341860847696	0.674389235783157\\
0.8	0.0626400280559541	0.348362286990222	0.672471317134862\\
0.8	0.064490662604533	0.3533019666601	0.670605827172181\\
0.8	0.0663694774214291	0.358237417517579	0.668787383459804\\
0.8	0.0682764858793752	0.363168408916186	0.667011243730729\\
0.8	0.0702116989375697	0.368094709561873	0.665273245863649\\
0.8	0.0721751251265572	0.373016087543259	0.663569750098932\\
0.8	0.0741667705335422	0.377932310362199	0.661897584220061\\
0.8	0.0761866387881432	0.382843144964686	0.660253992184442\\
0.8	0.0782347310485954	0.38774835777208	0.658636586490997\\
0.8	0.0803110459884098	0.392647714712651	0.657043304415871\\
0.8	0.0824155797834956	0.397540981253432	0.655472368126509\\
0.8	0.084548326099754	0.402427922432375	0.653922248592745\\
0.8	0.0867092760811502	0.4073083028908	0.652391633146569\\
0.8	0.0888984183382709	0.412181886906127	0.650879396495578\\
0.8	0.0911157389373742	0.417048438424896	0.649384574964844\\
0.8	0.0933612213899392	0.421907721096044	0.647906343724685\\
0.8	0.0956348466427212	0.42675949830445	0.646443996754805\\
0.8	0.0979365930683194	0.431603533204733	0.644996929295961\\
0.8	0.100266436456264	0.436439588755293	0.643564622546847\\
0.8	0.102624350004627	0.441267427752584	0.64214663037443\\
0.8	0.105010304312169	0.446086812865616	0.640742567819433\\
0.8	0.107424267371012	0.450897506670668	0.639352101193731\\
0.8	0.109866204559871	0.455699271686218	0.637974939582442\\
0.8	0.112336078637817	0.460491870408058	0.636610827579862\\
0.8	0.114833849738607	0.465275065344603	0.635259539104384\\
0.8	0.117359475365564	0.470048619052377	0.633920872153154\\
0.8	0.119912910387023	0.474812294171664	0.632594644371963\\
0.8	0.122494107032347	0.479565853462319	0.631280689329641\\
0.8	0.125103014888515	0.484309059839721	0.62997885339906\\
0.8	0.127739580897283	0.489041676410868	0.628690460824558\\
0.805	0	0	0.710202759607373\\
0.805	1.11327767495586e-05	0.00471862581271275	0.713003157837853\\
0.805	4.46175251031912e-05	0.009446325184051	0.715846486660142\\
0.805	0.000100583311362513	0.0141829653360114	0.718788838379794\\
0.805	0.000179158434668431	0.018928411755669	0.721886092228356\\
0.805	0.000280470402701511	0.02368252819604	0.725200636220892\\
0.805	0.000404645907256436	0.0284451766772965	0.728799085274723\\
0.805	0.000551810799695644	0.0332162174883388	0.732749123983336\\
0.805	0.000722090066287311	0.037995509188729	0.737115575253327\\
0.805	0.000915607803432999	0.0427829086109896	0.741955893907011\\
0.805	0.0011324871927904	0.0475782708632729	0.747315373652289\\
0.805	0.00137285047629673	0.0523814493324038	0.75322242404581\\
0.805	0.00163681893109844	0.057192295687301	0.759684310525652\\
0.805	0.00192451284439304	0.0620106598827802	0.766683748102665\\
0.805	0.00223605148818899	0.0668363901637434	0.774176695511256\\
0.805	0.00257155309398959	0.0716693330697584	0.782091614494493\\
0.805	0.00293113482740716	0.0765093334400312	0.79033034642109\\
0.805	0.00331491276271365	0.0813562344187764	0.79877062752665\\
0.805	0.00372300185733413	0.0862098774609879	0.80727012893036\\
0.805	0.00415551592628969	0.0910701023386145	0.815671782700194\\
0.805	0.00461256761659621	0.0959367471471425	0.82381005350733\\
0.805	0.00509426838162598	0.100809648312589	0.831517746503354\\
0.805	0.00560072845543873	0.105688640598912	0.838632911419452\\
0.805	0.00613205682708918	0.11057355711583	0.845005411376291\\
0.805	0.00668836121491815	0.115464229327074	0.850502769049302\\
0.805	0.00726974804083431	0.120360487059049	0.855014975717657\\
0.805	0.00787632240459385	0.125262158509929	0.858458041096606\\
0.805	0.00850818805808555	0.13016907025918	0.860776163494182\\
0.805	0.00916544737962849	0.135081047277508	0.861942500814223\\
0.805	0.00984820134829017	0.139997912937243	0.86195861464842\\
0.805	0.0105565495182326	0.144919489023162	0.860852735617091\\
0.805	0.0112905899930939	0.149845595743738	0.858677054109651\\
0.805	0.012050419400414	0.154776051742839	0.855504274967458\\
0.805	0.0128361328661109	0.159710674111862	0.851423687967756\\
0.805	0.0136478239890177	0.164649278402306	0.846537000484341\\
0.805	0.0144855848154856	0.169591678638796	0.840954157904247\\
0.805	0.0153495058140643	0.174537687332543	0.834789345403934\\
0.805	0.0162396758502651	0.179487115495259	0.8281573258062\\
0.805	0.0171561821614173	0.184439772653511	0.821170226448695\\
0.805	0.0180991103316243	0.189395466863524	0.813934846725692\\
0.805	0.0190685442668299	0.194354004726437	0.806550519904212\\
0.805	0.0200645661700016	0.199315191403999	0.799107529846305\\
0.805	0.0210872565164405	0.204278830634725	0.791686056496194\\
0.805	0.0221366940292259	0.20924472475049	0.7843556038306\\
0.805	0.023212955654804	0.214212674693577	0.777174850281117\\
0.805	0.0243161165387281	0.219182480034174	0.770191853862645\\
0.805	0.025446250001561	0.224153938988321	0.763444541554459\\
0.805	0.0266034275149466	0.229126848436298	0.756961413910484\\
0.805	0.0277877186778607	0.234101003941464	0.750762400416765\\
0.805	0.0289991911930496	0.239076199769547	0.744859807805785\\
0.805	0.0302379108436654	0.244052228908366	0.739259311518078\\
0.805	0.0315039414701067	0.24902888308801	0.733960949044503\\
0.805	0.0327973449470747	0.254005952801449	0.72896008240726\\
0.805	0.0341181811608519	0.258983227325592	0.724248305111178\\
0.805	0.0354665079868145	0.263960494742775	0.719814276220681\\
0.805	0.0368423812671862	0.268937541962694	0.715644470612629\\
0.805	0.0382458547890421	0.273914154744766	0.711723839838041\\
0.805	0.0396769802625738	0.278890117720924	0.70803638238973\\
0.805	0.0411358072996216	0.283865214418836	0.704565625567164\\
0.805	0.0426223833924861	0.288839227285554	0.701295023640316\\
0.805	0.0441367538930258	0.293811937711588	0.698208278749354\\
0.805	0.0456789619920503	0.298783126055387	0.69528959205535\\
0.805	0.0472490486990192	0.303752571668251	0.692523853197512\\
0.805	0.0488470528220538	0.308720052919643	0.689896776228118\\
0.805	0.0504730109482718	0.313685347222913	0.687394989990281\\
0.805	0.0521269574244539	0.318648231061428	0.685006090465981\\
0.805	0.0538089243380495	0.323608480015096	0.682718662029128\\
0.805	0.0555189414985328	0.328565868787292	0.68052227385347\\
0.805	0.0572570364191156	0.333520171232163	0.67840745699785\\
0.805	0.0590232342988274	0.338471160382329	0.676365666960043\\
0.805	0.0608175580049697	0.34341860847696	0.674389235783156\\
0.805	0.0626400280559541	0.348362286990222	0.672471317134862\\
0.805	0.064490662604533	0.3533019666601	0.670605827172183\\
0.805	0.066369477421429	0.358237417517579	0.668787383459803\\
0.805	0.0682764858793752	0.363168408916186	0.667011243730727\\
0.805	0.0702116989375697	0.368094709561873	0.66527324586365\\
0.805	0.0721751251265572	0.373016087543259	0.663569750098933\\
0.805	0.0741667705335422	0.377932310362199	0.66189758422006\\
0.805	0.0761866387881432	0.382843144964686	0.660253992184442\\
0.805	0.0782347310485954	0.38774835777208	0.658636586490998\\
0.805	0.0803110459884098	0.392647714712651	0.657043304415871\\
0.805	0.0824155797834956	0.397540981253432	0.655472368126509\\
0.805	0.084548326099754	0.402427922432375	0.653922248592746\\
0.805	0.0867092760811502	0.4073083028908	0.65239163314657\\
0.805	0.0888984183382708	0.412181886906127	0.650879396495579\\
0.805	0.0911157389373742	0.417048438424897	0.649384574964843\\
0.805	0.0933612213899392	0.421907721096044	0.647906343724684\\
0.805	0.0956348466427212	0.42675949830445	0.646443996754806\\
0.805	0.0979365930683194	0.431603533204733	0.644996929295962\\
0.805	0.100266436456264	0.436439588755293	0.643564622546846\\
0.805	0.102624350004627	0.441267427752585	0.64214663037443\\
0.805	0.105010304312169	0.446086812865616	0.640742567819434\\
0.805	0.107424267371012	0.450897506670668	0.639352101193731\\
0.805	0.109866204559871	0.455699271686218	0.637974939582442\\
0.805	0.112336078637817	0.460491870408058	0.636610827579861\\
0.805	0.114833849738607	0.465275065344603	0.635259539104382\\
0.805	0.117359475365564	0.470048619052377	0.633920872153155\\
0.805	0.119912910387022	0.474812294171664	0.632594644371966\\
0.805	0.122494107032347	0.479565853462319	0.63128068932964\\
0.805	0.125103014888515	0.484309059839721	0.629978853399059\\
0.805	0.127739580897283	0.489041676410868	0.628690460824563\\
0.81	0	0	0.710202759607373\\
0.81	1.11327767495586e-05	0.00471862581271275	0.713003157837853\\
0.81	4.46175251031912e-05	0.009446325184051	0.715846486660142\\
0.81	0.000100583311362513	0.0141829653360114	0.718788838379794\\
0.81	0.000179158434668431	0.018928411755669	0.721886092228357\\
0.81	0.000280470402701511	0.02368252819604	0.725200636220892\\
0.81	0.000404645907256436	0.0284451766772965	0.728799085274723\\
0.81	0.000551810799695644	0.0332162174883389	0.732749123983336\\
0.81	0.000722090066287311	0.037995509188729	0.737115575253328\\
0.81	0.000915607803432999	0.0427829086109896	0.741955893907011\\
0.81	0.0011324871927904	0.047578270863273	0.747315373652289\\
0.81	0.00137285047629673	0.0523814493324038	0.753222424045811\\
0.81	0.00163681893109844	0.057192295687301	0.759684310525652\\
0.81	0.00192451284439304	0.0620106598827802	0.766683748102665\\
0.81	0.00223605148818898	0.0668363901637434	0.774176695511256\\
0.81	0.00257155309398959	0.0716693330697584	0.782091614494494\\
0.81	0.00293113482740716	0.0765093334400312	0.79033034642109\\
0.81	0.00331491276271365	0.0813562344187764	0.79877062752665\\
0.81	0.00372300185733413	0.0862098774609879	0.807270128930359\\
0.81	0.00415551592628969	0.0910701023386145	0.815671782700193\\
0.81	0.00461256761659621	0.0959367471471425	0.82381005350733\\
0.81	0.00509426838162598	0.100809648312589	0.831517746503353\\
0.81	0.00560072845543873	0.105688640598912	0.838632911419452\\
0.81	0.00613205682708918	0.11057355711583	0.845005411376292\\
0.81	0.00668836121491816	0.115464229327074	0.850502769049301\\
0.81	0.00726974804083431	0.120360487059049	0.855014975717657\\
0.81	0.00787632240459385	0.125262158509929	0.858458041096606\\
0.81	0.00850818805808555	0.13016907025918	0.860776163494182\\
0.81	0.00916544737962849	0.135081047277508	0.861942500814224\\
0.81	0.00984820134829017	0.139997912937243	0.86195861464842\\
0.81	0.0105565495182326	0.144919489023162	0.86085273561709\\
0.81	0.0112905899930939	0.149845595743738	0.858677054109651\\
0.81	0.012050419400414	0.154776051742839	0.855504274967458\\
0.81	0.0128361328661109	0.159710674111862	0.851423687967755\\
0.81	0.0136478239890177	0.164649278402306	0.846537000484339\\
0.81	0.0144855848154856	0.169591678638796	0.840954157904248\\
0.81	0.0153495058140643	0.174537687332543	0.834789345403934\\
0.81	0.0162396758502651	0.179487115495259	0.828157325806202\\
0.81	0.0171561821614173	0.184439772653511	0.821170226448694\\
0.81	0.0180991103316243	0.189395466863524	0.813934846725691\\
0.81	0.0190685442668299	0.194354004726437	0.806550519904211\\
0.81	0.0200645661700016	0.199315191403999	0.799107529846304\\
0.81	0.0210872565164405	0.204278830634725	0.791686056496195\\
0.81	0.0221366940292259	0.20924472475049	0.784355603830601\\
0.81	0.023212955654804	0.214212674693577	0.777174850281119\\
0.81	0.0243161165387281	0.219182480034174	0.770191853862644\\
0.81	0.025446250001561	0.224153938988321	0.763444541554458\\
0.81	0.0266034275149466	0.229126848436298	0.756961413910483\\
0.81	0.0277877186778607	0.234101003941464	0.750762400416763\\
0.81	0.0289991911930496	0.239076199769547	0.744859807805782\\
0.81	0.0302379108436654	0.244052228908366	0.73925931151808\\
0.81	0.0315039414701067	0.24902888308801	0.733960949044504\\
0.81	0.0327973449470747	0.254005952801449	0.72896008240726\\
0.81	0.0341181811608519	0.258983227325592	0.724248305111177\\
0.81	0.0354665079868145	0.263960494742775	0.719814276220681\\
0.81	0.0368423812671862	0.268937541962694	0.715644470612631\\
0.81	0.0382458547890421	0.273914154744766	0.711723839838041\\
0.81	0.0396769802625738	0.278890117720924	0.708036382389729\\
0.81	0.0411358072996216	0.283865214418836	0.704565625567164\\
0.81	0.0426223833924861	0.288839227285554	0.701295023640317\\
0.81	0.0441367538930258	0.293811937711588	0.698208278749354\\
0.81	0.0456789619920503	0.298783126055387	0.69528959205535\\
0.81	0.0472490486990192	0.303752571668251	0.692523853197512\\
0.81	0.0488470528220538	0.308720052919643	0.689896776228118\\
0.81	0.0504730109482718	0.313685347222913	0.687394989990281\\
0.81	0.0521269574244539	0.318648231061428	0.68500609046598\\
0.81	0.0538089243380495	0.323608480015096	0.682718662029127\\
0.81	0.0555189414985328	0.328565868787292	0.68052227385347\\
0.81	0.0572570364191156	0.333520171232163	0.67840745699785\\
0.81	0.0590232342988274	0.338471160382329	0.676365666960043\\
0.81	0.0608175580049697	0.34341860847696	0.674389235783156\\
0.81	0.0626400280559541	0.348362286990222	0.672471317134861\\
0.81	0.064490662604533	0.3533019666601	0.670605827172182\\
0.81	0.0663694774214291	0.358237417517579	0.668787383459805\\
0.81	0.0682764858793752	0.363168408916186	0.667011243730728\\
0.81	0.0702116989375697	0.368094709561873	0.665273245863649\\
0.81	0.0721751251265572	0.373016087543259	0.663569750098932\\
0.81	0.0741667705335422	0.377932310362199	0.66189758422006\\
0.81	0.0761866387881432	0.382843144964686	0.660253992184442\\
0.81	0.0782347310485954	0.38774835777208	0.658636586490997\\
0.81	0.0803110459884098	0.392647714712651	0.657043304415871\\
0.81	0.0824155797834956	0.397540981253432	0.655472368126508\\
0.81	0.084548326099754	0.402427922432375	0.653922248592746\\
0.81	0.0867092760811502	0.4073083028908	0.65239163314657\\
0.81	0.0888984183382709	0.412181886906127	0.650879396495579\\
0.81	0.0911157389373742	0.417048438424897	0.649384574964845\\
0.81	0.0933612213899392	0.421907721096044	0.647906343724684\\
0.81	0.0956348466427212	0.42675949830445	0.646443996754806\\
0.81	0.0979365930683194	0.431603533204733	0.644996929295962\\
0.81	0.100266436456264	0.436439588755293	0.643564622546846\\
0.81	0.102624350004627	0.441267427752584	0.64214663037443\\
0.81	0.105010304312169	0.446086812865616	0.640742567819434\\
0.81	0.107424267371012	0.450897506670668	0.639352101193731\\
0.81	0.109866204559871	0.455699271686218	0.637974939582443\\
0.81	0.112336078637817	0.460491870408058	0.636610827579863\\
0.81	0.114833849738607	0.465275065344603	0.635259539104383\\
0.81	0.117359475365564	0.470048619052377	0.633920872153154\\
0.81	0.119912910387023	0.474812294171664	0.632594644371964\\
0.81	0.122494107032347	0.479565853462319	0.631280689329641\\
0.81	0.125103014888515	0.484309059839721	0.629978853399057\\
0.81	0.127739580897283	0.489041676410868	0.628690460824558\\
0.815	0	0	0.710202759607373\\
0.815	1.11327767495586e-05	0.00471862581271275	0.713003157837853\\
0.815	4.46175251031912e-05	0.009446325184051	0.715846486660142\\
0.815	0.000100583311362513	0.0141829653360114	0.718788838379794\\
0.815	0.000179158434668431	0.018928411755669	0.721886092228356\\
0.815	0.000280470402701511	0.02368252819604	0.725200636220892\\
0.815	0.000404645907256436	0.0284451766772965	0.728799085274723\\
0.815	0.000551810799695644	0.0332162174883389	0.732749123983336\\
0.815	0.000722090066287311	0.037995509188729	0.737115575253327\\
0.815	0.000915607803432999	0.0427829086109896	0.741955893907011\\
0.815	0.0011324871927904	0.0475782708632729	0.747315373652289\\
0.815	0.00137285047629673	0.0523814493324038	0.753222424045811\\
0.815	0.00163681893109844	0.057192295687301	0.759684310525652\\
0.815	0.00192451284439304	0.0620106598827802	0.766683748102665\\
0.815	0.00223605148818898	0.0668363901637434	0.774176695511256\\
0.815	0.00257155309398959	0.0716693330697584	0.782091614494494\\
0.815	0.00293113482740716	0.0765093334400312	0.79033034642109\\
0.815	0.00331491276271365	0.0813562344187764	0.798770627526651\\
0.815	0.00372300185733413	0.0862098774609879	0.807270128930359\\
0.815	0.00415551592628969	0.0910701023386145	0.815671782700194\\
0.815	0.00461256761659621	0.0959367471471424	0.82381005350733\\
0.815	0.00509426838162598	0.100809648312589	0.831517746503354\\
0.815	0.00560072845543873	0.105688640598912	0.838632911419452\\
0.815	0.00613205682708918	0.11057355711583	0.845005411376293\\
0.815	0.00668836121491816	0.115464229327074	0.850502769049302\\
0.815	0.00726974804083431	0.120360487059049	0.855014975717656\\
0.815	0.00787632240459385	0.125262158509929	0.858458041096606\\
0.815	0.00850818805808555	0.13016907025918	0.860776163494182\\
0.815	0.00916544737962849	0.135081047277508	0.861942500814225\\
0.815	0.00984820134829017	0.139997912937243	0.861958614648419\\
0.815	0.0105565495182326	0.144919489023162	0.86085273561709\\
0.815	0.0112905899930939	0.149845595743738	0.858677054109649\\
0.815	0.012050419400414	0.154776051742839	0.855504274967455\\
0.815	0.0128361328661109	0.159710674111862	0.851423687967753\\
0.815	0.0136478239890177	0.164649278402306	0.84653700048434\\
0.815	0.0144855848154856	0.169591678638796	0.84095415790425\\
0.815	0.0153495058140643	0.174537687332543	0.834789345403935\\
0.815	0.0162396758502651	0.179487115495259	0.8281573258062\\
0.815	0.0171561821614173	0.184439772653511	0.821170226448693\\
0.815	0.0180991103316243	0.189395466863524	0.813934846725692\\
0.815	0.0190685442668299	0.194354004726437	0.806550519904212\\
0.815	0.0200645661700016	0.199315191403999	0.799107529846306\\
0.815	0.0210872565164405	0.204278830634725	0.791686056496195\\
0.815	0.0221366940292259	0.20924472475049	0.784355603830601\\
0.815	0.023212955654804	0.214212674693577	0.777174850281119\\
0.815	0.0243161165387281	0.219182480034174	0.770191853862642\\
0.815	0.025446250001561	0.224153938988321	0.763444541554458\\
0.815	0.0266034275149466	0.229126848436298	0.756961413910483\\
0.815	0.0277877186778607	0.234101003941464	0.750762400416762\\
0.815	0.0289991911930496	0.239076199769547	0.744859807805785\\
0.815	0.0302379108436654	0.244052228908366	0.739259311518081\\
0.815	0.0315039414701067	0.24902888308801	0.733960949044503\\
0.815	0.0327973449470747	0.254005952801449	0.72896008240726\\
0.815	0.0341181811608519	0.258983227325592	0.724248305111177\\
0.815	0.0354665079868145	0.263960494742775	0.719814276220682\\
0.815	0.0368423812671862	0.268937541962694	0.71564447061263\\
0.815	0.0382458547890421	0.273914154744766	0.711723839838042\\
0.815	0.0396769802625738	0.278890117720924	0.708036382389731\\
0.815	0.0411358072996216	0.283865214418836	0.704565625567163\\
0.815	0.0426223833924861	0.288839227285554	0.701295023640316\\
0.815	0.0441367538930258	0.293811937711588	0.698208278749354\\
0.815	0.0456789619920503	0.298783126055387	0.69528959205535\\
0.815	0.0472490486990192	0.303752571668251	0.692523853197512\\
0.815	0.0488470528220538	0.308720052919643	0.689896776228119\\
0.815	0.0504730109482718	0.313685347222913	0.687394989990281\\
0.815	0.0521269574244539	0.318648231061428	0.68500609046598\\
0.815	0.0538089243380495	0.323608480015096	0.682718662029127\\
0.815	0.0555189414985328	0.328565868787292	0.68052227385347\\
0.815	0.0572570364191156	0.333520171232163	0.678407456997851\\
0.815	0.0590232342988274	0.338471160382329	0.676365666960043\\
0.815	0.0608175580049697	0.34341860847696	0.674389235783156\\
0.815	0.0626400280559541	0.348362286990222	0.672471317134862\\
0.815	0.064490662604533	0.3533019666601	0.670605827172182\\
0.815	0.0663694774214291	0.358237417517579	0.668787383459804\\
0.815	0.0682764858793752	0.363168408916186	0.667011243730728\\
0.815	0.0702116989375697	0.368094709561873	0.665273245863648\\
0.815	0.0721751251265572	0.373016087543259	0.663569750098932\\
0.815	0.0741667705335422	0.377932310362199	0.66189758422006\\
0.815	0.0761866387881432	0.382843144964686	0.660253992184442\\
0.815	0.0782347310485954	0.38774835777208	0.658636586490997\\
0.815	0.0803110459884098	0.392647714712651	0.657043304415873\\
0.815	0.0824155797834956	0.397540981253432	0.655472368126509\\
0.815	0.084548326099754	0.402427922432375	0.653922248592744\\
0.815	0.0867092760811502	0.4073083028908	0.65239163314657\\
0.815	0.0888984183382709	0.412181886906127	0.650879396495579\\
0.815	0.0911157389373742	0.417048438424896	0.649384574964844\\
0.815	0.0933612213899392	0.421907721096044	0.647906343724685\\
0.815	0.0956348466427212	0.42675949830445	0.646443996754804\\
0.815	0.0979365930683194	0.431603533204733	0.644996929295962\\
0.815	0.100266436456264	0.436439588755293	0.643564622546846\\
0.815	0.102624350004627	0.441267427752584	0.64214663037443\\
0.815	0.105010304312169	0.446086812865616	0.640742567819434\\
0.815	0.107424267371012	0.450897506670668	0.639352101193731\\
0.815	0.109866204559871	0.455699271686218	0.637974939582442\\
0.815	0.112336078637817	0.460491870408058	0.636610827579862\\
0.815	0.114833849738607	0.465275065344603	0.635259539104384\\
0.815	0.117359475365564	0.470048619052377	0.633920872153155\\
0.815	0.119912910387022	0.474812294171664	0.632594644371964\\
0.815	0.122494107032347	0.479565853462319	0.631280689329641\\
0.815	0.125103014888515	0.484309059839721	0.629978853399058\\
0.815	0.127739580897283	0.489041676410868	0.628690460824554\\
0.82	0	0	0.710202759607373\\
0.82	1.11327767495586e-05	0.00471862581271275	0.713003157837853\\
0.82	4.46175251031912e-05	0.009446325184051	0.715846486660142\\
0.82	0.000100583311362513	0.0141829653360114	0.718788838379794\\
0.82	0.000179158434668431	0.018928411755669	0.721886092228356\\
0.82	0.000280470402701511	0.02368252819604	0.725200636220892\\
0.82	0.000404645907256436	0.0284451766772965	0.728799085274723\\
0.82	0.000551810799695644	0.0332162174883388	0.732749123983336\\
0.82	0.000722090066287311	0.037995509188729	0.737115575253327\\
0.82	0.000915607803432999	0.0427829086109896	0.741955893907011\\
0.82	0.0011324871927904	0.0475782708632729	0.747315373652289\\
0.82	0.00137285047629673	0.0523814493324038	0.753222424045811\\
0.82	0.00163681893109843	0.057192295687301	0.759684310525652\\
0.82	0.00192451284439304	0.0620106598827802	0.766683748102665\\
0.82	0.00223605148818898	0.0668363901637434	0.774176695511256\\
0.82	0.00257155309398959	0.0716693330697584	0.782091614494494\\
0.82	0.00293113482740716	0.0765093334400312	0.79033034642109\\
0.82	0.00331491276271365	0.0813562344187764	0.79877062752665\\
0.82	0.00372300185733413	0.0862098774609879	0.807270128930359\\
0.82	0.00415551592628969	0.0910701023386145	0.815671782700193\\
0.82	0.00461256761659621	0.0959367471471424	0.823810053507331\\
0.82	0.00509426838162598	0.100809648312589	0.831517746503354\\
0.82	0.00560072845543873	0.105688640598912	0.838632911419452\\
0.82	0.00613205682708918	0.11057355711583	0.845005411376293\\
0.82	0.00668836121491815	0.115464229327074	0.850502769049302\\
0.82	0.00726974804083431	0.120360487059049	0.855014975717655\\
0.82	0.00787632240459385	0.125262158509929	0.858458041096606\\
0.82	0.00850818805808554	0.13016907025918	0.860776163494182\\
0.82	0.00916544737962849	0.135081047277508	0.861942500814224\\
0.82	0.00984820134829017	0.139997912937243	0.86195861464842\\
0.82	0.0105565495182326	0.144919489023162	0.860852735617092\\
0.82	0.0112905899930939	0.149845595743738	0.85867705410965\\
0.82	0.012050419400414	0.154776051742839	0.855504274967457\\
0.82	0.0128361328661109	0.159710674111862	0.851423687967756\\
0.82	0.0136478239890177	0.164649278402306	0.846537000484341\\
0.82	0.0144855848154856	0.169591678638796	0.840954157904249\\
0.82	0.0153495058140643	0.174537687332543	0.834789345403935\\
0.82	0.0162396758502651	0.179487115495259	0.8281573258062\\
0.82	0.0171561821614173	0.184439772653511	0.821170226448695\\
0.82	0.0180991103316243	0.189395466863524	0.813934846725693\\
0.82	0.0190685442668299	0.194354004726437	0.806550519904212\\
0.82	0.0200645661700016	0.199315191403999	0.799107529846306\\
0.82	0.0210872565164405	0.204278830634725	0.791686056496193\\
0.82	0.0221366940292259	0.20924472475049	0.784355603830604\\
0.82	0.023212955654804	0.214212674693577	0.777174850281119\\
0.82	0.0243161165387281	0.219182480034174	0.770191853862642\\
0.82	0.025446250001561	0.224153938988321	0.763444541554457\\
0.82	0.0266034275149466	0.229126848436298	0.756961413910482\\
0.82	0.0277877186778607	0.234101003941464	0.750762400416765\\
0.82	0.0289991911930496	0.239076199769547	0.744859807805785\\
0.82	0.0302379108436654	0.244052228908366	0.739259311518081\\
0.82	0.0315039414701067	0.24902888308801	0.733960949044503\\
0.82	0.0327973449470747	0.254005952801449	0.72896008240726\\
0.82	0.0341181811608519	0.258983227325592	0.724248305111179\\
0.82	0.0354665079868145	0.263960494742775	0.719814276220681\\
0.82	0.0368423812671862	0.268937541962694	0.715644470612629\\
0.82	0.0382458547890421	0.273914154744766	0.711723839838043\\
0.82	0.0396769802625738	0.278890117720924	0.70803638238973\\
0.82	0.0411358072996216	0.283865214418836	0.704565625567163\\
0.82	0.0426223833924861	0.288839227285554	0.701295023640316\\
0.82	0.0441367538930258	0.293811937711588	0.698208278749353\\
0.82	0.0456789619920503	0.298783126055387	0.69528959205535\\
0.82	0.0472490486990192	0.303752571668251	0.692523853197512\\
0.82	0.0488470528220538	0.308720052919643	0.689896776228119\\
0.82	0.0504730109482718	0.313685347222913	0.687394989990282\\
0.82	0.0521269574244539	0.318648231061428	0.685006090465981\\
0.82	0.0538089243380495	0.323608480015096	0.682718662029128\\
0.82	0.0555189414985328	0.328565868787292	0.680522273853469\\
0.82	0.0572570364191156	0.333520171232163	0.67840745699785\\
0.82	0.0590232342988274	0.338471160382329	0.676365666960043\\
0.82	0.0608175580049697	0.34341860847696	0.674389235783156\\
0.82	0.0626400280559541	0.348362286990222	0.672471317134862\\
0.82	0.064490662604533	0.3533019666601	0.670605827172182\\
0.82	0.0663694774214291	0.358237417517579	0.668787383459804\\
0.82	0.0682764858793752	0.363168408916186	0.667011243730729\\
0.82	0.0702116989375697	0.368094709561873	0.665273245863649\\
0.82	0.0721751251265572	0.373016087543259	0.663569750098933\\
0.82	0.0741667705335422	0.377932310362199	0.661897584220061\\
0.82	0.0761866387881432	0.382843144964686	0.660253992184441\\
0.82	0.0782347310485954	0.38774835777208	0.658636586490996\\
0.82	0.0803110459884098	0.392647714712651	0.657043304415872\\
0.82	0.0824155797834956	0.397540981253432	0.65547236812651\\
0.82	0.084548326099754	0.402427922432375	0.653922248592746\\
0.82	0.0867092760811502	0.4073083028908	0.65239163314657\\
0.82	0.0888984183382709	0.412181886906127	0.650879396495579\\
0.82	0.0911157389373742	0.417048438424897	0.649384574964843\\
0.82	0.0933612213899392	0.421907721096044	0.647906343724685\\
0.82	0.0956348466427212	0.42675949830445	0.646443996754806\\
0.82	0.0979365930683194	0.431603533204733	0.644996929295961\\
0.82	0.100266436456264	0.436439588755293	0.643564622546847\\
0.82	0.102624350004627	0.441267427752585	0.642146630374429\\
0.82	0.105010304312169	0.446086812865616	0.640742567819433\\
0.82	0.107424267371012	0.450897506670668	0.639352101193732\\
0.82	0.109866204559871	0.455699271686218	0.637974939582443\\
0.82	0.112336078637817	0.460491870408058	0.63661082757986\\
0.82	0.114833849738607	0.465275065344603	0.635259539104383\\
0.82	0.117359475365564	0.470048619052377	0.633920872153156\\
0.82	0.119912910387023	0.474812294171664	0.632594644371965\\
0.82	0.122494107032347	0.479565853462319	0.631280689329641\\
0.82	0.125103014888515	0.484309059839721	0.629978853399058\\
0.82	0.127739580897283	0.489041676410868	0.628690460824556\\
0.825	0	0	0.710202759607373\\
0.825	1.11327767495586e-05	0.00471862581271275	0.713003157837853\\
0.825	4.46175251031912e-05	0.009446325184051	0.715846486660142\\
0.825	0.000100583311362513	0.0141829653360114	0.718788838379794\\
0.825	0.000179158434668431	0.018928411755669	0.721886092228356\\
0.825	0.000280470402701511	0.02368252819604	0.725200636220892\\
0.825	0.000404645907256436	0.0284451766772965	0.728799085274723\\
0.825	0.000551810799695644	0.0332162174883389	0.732749123983336\\
0.825	0.000722090066287311	0.037995509188729	0.737115575253328\\
0.825	0.000915607803432999	0.0427829086109896	0.741955893907011\\
0.825	0.0011324871927904	0.0475782708632729	0.747315373652289\\
0.825	0.00137285047629673	0.0523814493324038	0.753222424045811\\
0.825	0.00163681893109844	0.057192295687301	0.759684310525652\\
0.825	0.00192451284439304	0.0620106598827802	0.766683748102665\\
0.825	0.00223605148818899	0.0668363901637434	0.774176695511256\\
0.825	0.00257155309398959	0.0716693330697584	0.782091614494494\\
0.825	0.00293113482740716	0.0765093334400312	0.79033034642109\\
0.825	0.00331491276271365	0.0813562344187764	0.79877062752665\\
0.825	0.00372300185733413	0.0862098774609879	0.80727012893036\\
0.825	0.00415551592628969	0.0910701023386145	0.815671782700193\\
0.825	0.00461256761659621	0.0959367471471425	0.823810053507331\\
0.825	0.00509426838162598	0.100809648312589	0.831517746503354\\
0.825	0.00560072845543873	0.105688640598912	0.838632911419451\\
0.825	0.00613205682708918	0.11057355711583	0.845005411376292\\
0.825	0.00668836121491816	0.115464229327074	0.850502769049301\\
0.825	0.00726974804083431	0.120360487059049	0.855014975717657\\
0.825	0.00787632240459385	0.125262158509929	0.858458041096605\\
0.825	0.00850818805808555	0.13016907025918	0.860776163494183\\
0.825	0.00916544737962849	0.135081047277508	0.861942500814224\\
0.825	0.00984820134829017	0.139997912937243	0.861958614648421\\
0.825	0.0105565495182326	0.144919489023162	0.860852735617091\\
0.825	0.0112905899930939	0.149845595743738	0.85867705410965\\
0.825	0.012050419400414	0.154776051742839	0.855504274967458\\
0.825	0.0128361328661109	0.159710674111862	0.851423687967756\\
0.825	0.0136478239890177	0.164649278402306	0.846537000484339\\
0.825	0.0144855848154856	0.169591678638796	0.840954157904249\\
0.825	0.0153495058140643	0.174537687332543	0.834789345403936\\
0.825	0.0162396758502651	0.179487115495259	0.828157325806201\\
0.825	0.0171561821614173	0.184439772653511	0.821170226448693\\
0.825	0.0180991103316243	0.189395466863524	0.813934846725692\\
0.825	0.0190685442668299	0.194354004726437	0.806550519904214\\
0.825	0.0200645661700016	0.199315191403999	0.799107529846305\\
0.825	0.0210872565164405	0.204278830634725	0.791686056496196\\
0.825	0.0221366940292259	0.20924472475049	0.784355603830602\\
0.825	0.023212955654804	0.214212674693577	0.777174850281116\\
0.825	0.0243161165387281	0.219182480034174	0.770191853862641\\
0.825	0.025446250001561	0.224153938988321	0.763444541554458\\
0.825	0.0266034275149466	0.229126848436298	0.756961413910484\\
0.825	0.0277877186778607	0.234101003941464	0.750762400416765\\
0.825	0.0289991911930496	0.239076199769547	0.744859807805784\\
0.825	0.0302379108436654	0.244052228908366	0.73925931151808\\
0.825	0.0315039414701067	0.24902888308801	0.733960949044503\\
0.825	0.0327973449470747	0.254005952801449	0.728960082407262\\
0.825	0.0341181811608519	0.258983227325592	0.724248305111177\\
0.825	0.0354665079868145	0.263960494742775	0.719814276220681\\
0.825	0.0368423812671862	0.268937541962694	0.715644470612631\\
0.825	0.0382458547890421	0.273914154744766	0.711723839838041\\
0.825	0.0396769802625738	0.278890117720924	0.70803638238973\\
0.825	0.0411358072996216	0.283865214418836	0.704565625567165\\
0.825	0.0426223833924861	0.288839227285554	0.701295023640316\\
0.825	0.0441367538930258	0.293811937711588	0.698208278749354\\
0.825	0.0456789619920503	0.298783126055387	0.695289592055349\\
0.825	0.0472490486990192	0.303752571668251	0.692523853197512\\
0.825	0.0488470528220538	0.308720052919643	0.689896776228118\\
0.825	0.0504730109482718	0.313685347222913	0.687394989990281\\
0.825	0.0521269574244539	0.318648231061428	0.685006090465981\\
0.825	0.0538089243380495	0.323608480015096	0.682718662029127\\
0.825	0.0555189414985328	0.328565868787292	0.680522273853468\\
0.825	0.0572570364191156	0.333520171232163	0.67840745699785\\
0.825	0.0590232342988274	0.338471160382329	0.676365666960045\\
0.825	0.0608175580049697	0.34341860847696	0.674389235783156\\
0.825	0.0626400280559541	0.348362286990222	0.672471317134862\\
0.825	0.064490662604533	0.3533019666601	0.670605827172183\\
0.825	0.066369477421429	0.358237417517579	0.668787383459803\\
0.825	0.0682764858793752	0.363168408916186	0.667011243730728\\
0.825	0.0702116989375697	0.368094709561873	0.665273245863649\\
0.825	0.0721751251265572	0.373016087543259	0.663569750098932\\
0.825	0.0741667705335423	0.377932310362199	0.661897584220061\\
0.825	0.0761866387881432	0.382843144964686	0.660253992184442\\
0.825	0.0782347310485954	0.38774835777208	0.658636586490996\\
0.825	0.0803110459884098	0.392647714712651	0.657043304415871\\
0.825	0.0824155797834956	0.397540981253432	0.655472368126509\\
0.825	0.084548326099754	0.402427922432375	0.653922248592745\\
0.825	0.0867092760811502	0.4073083028908	0.65239163314657\\
0.825	0.0888984183382709	0.412181886906127	0.650879396495579\\
0.825	0.0911157389373742	0.417048438424896	0.649384574964845\\
0.825	0.0933612213899392	0.421907721096044	0.647906343724684\\
0.825	0.0956348466427212	0.42675949830445	0.646443996754806\\
0.825	0.0979365930683194	0.431603533204733	0.644996929295962\\
0.825	0.100266436456264	0.436439588755293	0.643564622546847\\
0.825	0.102624350004627	0.441267427752585	0.642146630374431\\
0.825	0.105010304312169	0.446086812865616	0.640742567819433\\
0.825	0.107424267371012	0.450897506670668	0.639352101193731\\
0.825	0.109866204559871	0.455699271686218	0.637974939582443\\
0.825	0.112336078637817	0.460491870408058	0.636610827579862\\
0.825	0.114833849738607	0.465275065344603	0.635259539104382\\
0.825	0.117359475365564	0.470048619052377	0.633920872153155\\
0.825	0.119912910387023	0.474812294171664	0.632594644371965\\
0.825	0.122494107032347	0.479565853462319	0.631280689329641\\
0.825	0.125103014888515	0.484309059839721	0.62997885339906\\
0.825	0.127739580897283	0.489041676410868	0.628690460824555\\
0.83	0	0	0.710202759607373\\
0.83	1.11327767495586e-05	0.00471862581271274	0.713003157837853\\
0.83	4.46175251031912e-05	0.009446325184051	0.715846486660142\\
0.83	0.000100583311362513	0.0141829653360114	0.718788838379794\\
0.83	0.000179158434668431	0.018928411755669	0.721886092228356\\
0.83	0.000280470402701511	0.02368252819604	0.725200636220892\\
0.83	0.000404645907256436	0.0284451766772965	0.728799085274723\\
0.83	0.000551810799695644	0.0332162174883388	0.732749123983336\\
0.83	0.000722090066287311	0.037995509188729	0.737115575253327\\
0.83	0.000915607803432999	0.0427829086109896	0.741955893907011\\
0.83	0.0011324871927904	0.047578270863273	0.747315373652289\\
0.83	0.00137285047629673	0.0523814493324038	0.753222424045811\\
0.83	0.00163681893109844	0.057192295687301	0.759684310525652\\
0.83	0.00192451284439304	0.0620106598827802	0.766683748102665\\
0.83	0.00223605148818898	0.0668363901637434	0.774176695511256\\
0.83	0.00257155309398959	0.0716693330697584	0.782091614494493\\
0.83	0.00293113482740716	0.0765093334400312	0.79033034642109\\
0.83	0.00331491276271365	0.0813562344187764	0.79877062752665\\
0.83	0.00372300185733413	0.0862098774609879	0.807270128930359\\
0.83	0.00415551592628969	0.0910701023386145	0.815671782700193\\
0.83	0.00461256761659621	0.0959367471471425	0.82381005350733\\
0.83	0.00509426838162598	0.100809648312589	0.831517746503354\\
0.83	0.00560072845543873	0.105688640598912	0.838632911419452\\
0.83	0.00613205682708918	0.11057355711583	0.845005411376293\\
0.83	0.00668836121491815	0.115464229327074	0.850502769049302\\
0.83	0.00726974804083431	0.120360487059049	0.855014975717657\\
0.83	0.00787632240459385	0.125262158509929	0.858458041096607\\
0.83	0.00850818805808555	0.13016907025918	0.860776163494182\\
0.83	0.00916544737962849	0.135081047277508	0.861942500814224\\
0.83	0.00984820134829017	0.139997912937243	0.86195861464842\\
0.83	0.0105565495182326	0.144919489023162	0.86085273561709\\
0.83	0.0112905899930939	0.149845595743738	0.85867705410965\\
0.83	0.012050419400414	0.154776051742839	0.855504274967458\\
0.83	0.0128361328661109	0.159710674111862	0.851423687967751\\
0.83	0.0136478239890177	0.164649278402306	0.846537000484341\\
0.83	0.0144855848154856	0.169591678638796	0.840954157904249\\
0.83	0.0153495058140643	0.174537687332543	0.834789345403935\\
0.83	0.0162396758502651	0.179487115495259	0.828157325806201\\
0.83	0.0171561821614173	0.184439772653511	0.821170226448695\\
0.83	0.0180991103316243	0.189395466863524	0.813934846725693\\
0.83	0.0190685442668299	0.194354004726437	0.806550519904212\\
0.83	0.0200645661700016	0.199315191403999	0.799107529846303\\
0.83	0.0210872565164405	0.204278830634725	0.791686056496193\\
0.83	0.0221366940292259	0.20924472475049	0.7843556038306\\
0.83	0.023212955654804	0.214212674693577	0.777174850281117\\
0.83	0.0243161165387281	0.219182480034174	0.770191853862642\\
0.83	0.025446250001561	0.224153938988321	0.763444541554458\\
0.83	0.0266034275149466	0.229126848436298	0.756961413910483\\
0.83	0.0277877186778607	0.234101003941464	0.750762400416763\\
0.83	0.0289991911930496	0.239076199769547	0.744859807805786\\
0.83	0.0302379108436654	0.244052228908366	0.739259311518081\\
0.83	0.0315039414701067	0.24902888308801	0.733960949044503\\
0.83	0.0327973449470747	0.254005952801449	0.728960082407259\\
0.83	0.0341181811608519	0.258983227325591	0.724248305111177\\
0.83	0.0354665079868145	0.263960494742775	0.719814276220683\\
0.83	0.0368423812671862	0.268937541962694	0.715644470612631\\
0.83	0.0382458547890421	0.273914154744766	0.711723839838043\\
0.83	0.0396769802625738	0.278890117720924	0.708036382389731\\
0.83	0.0411358072996216	0.283865214418836	0.704565625567162\\
0.83	0.0426223833924861	0.288839227285554	0.701295023640315\\
0.83	0.0441367538930258	0.293811937711588	0.698208278749355\\
0.83	0.0456789619920503	0.298783126055387	0.69528959205535\\
0.83	0.0472490486990192	0.303752571668251	0.692523853197512\\
0.83	0.0488470528220538	0.308720052919643	0.689896776228118\\
0.83	0.0504730109482718	0.313685347222913	0.687394989990282\\
0.83	0.0521269574244539	0.318648231061428	0.685006090465981\\
0.83	0.0538089243380495	0.323608480015097	0.682718662029128\\
0.83	0.0555189414985328	0.328565868787292	0.680522273853469\\
0.83	0.0572570364191156	0.333520171232163	0.678407456997849\\
0.83	0.0590232342988274	0.338471160382329	0.676365666960043\\
0.83	0.0608175580049697	0.34341860847696	0.674389235783156\\
0.83	0.0626400280559541	0.348362286990222	0.672471317134862\\
0.83	0.064490662604533	0.3533019666601	0.670605827172182\\
0.83	0.0663694774214291	0.358237417517579	0.668787383459804\\
0.83	0.0682764858793752	0.363168408916186	0.667011243730728\\
0.83	0.0702116989375697	0.368094709561873	0.66527324586365\\
0.83	0.0721751251265572	0.373016087543259	0.663569750098932\\
0.83	0.0741667705335423	0.377932310362199	0.661897584220061\\
0.83	0.0761866387881432	0.382843144964686	0.660253992184441\\
0.83	0.0782347310485954	0.38774835777208	0.658636586490997\\
0.83	0.0803110459884098	0.392647714712651	0.657043304415872\\
0.83	0.0824155797834956	0.397540981253432	0.65547236812651\\
0.83	0.084548326099754	0.402427922432375	0.653922248592744\\
0.83	0.0867092760811502	0.407308302890799	0.652391633146569\\
0.83	0.0888984183382709	0.412181886906127	0.650879396495579\\
0.83	0.0911157389373742	0.417048438424896	0.649384574964844\\
0.83	0.0933612213899392	0.421907721096044	0.647906343724685\\
0.83	0.0956348466427212	0.42675949830445	0.646443996754804\\
0.83	0.0979365930683194	0.431603533204733	0.644996929295961\\
0.83	0.100266436456264	0.436439588755293	0.643564622546847\\
0.83	0.102624350004627	0.441267427752584	0.64214663037443\\
0.83	0.105010304312169	0.446086812865616	0.640742567819434\\
0.83	0.107424267371012	0.450897506670668	0.639352101193731\\
0.83	0.109866204559871	0.455699271686218	0.637974939582442\\
0.83	0.112336078637817	0.460491870408058	0.636610827579861\\
0.83	0.114833849738607	0.465275065344603	0.635259539104384\\
0.83	0.117359475365564	0.470048619052377	0.633920872153155\\
0.83	0.119912910387023	0.474812294171664	0.632594644371964\\
0.83	0.122494107032347	0.479565853462319	0.63128068932964\\
0.83	0.125103014888515	0.484309059839721	0.629978853399059\\
0.83	0.127739580897283	0.489041676410868	0.62869046082456\\
0.835	0	0	0.710202759607373\\
0.835	1.11327767495586e-05	0.00471862581271274	0.713003157837853\\
0.835	4.46175251031912e-05	0.009446325184051	0.715846486660142\\
0.835	0.000100583311362513	0.0141829653360114	0.718788838379794\\
0.835	0.000179158434668431	0.018928411755669	0.721886092228357\\
0.835	0.000280470402701511	0.02368252819604	0.725200636220892\\
0.835	0.000404645907256436	0.0284451766772965	0.728799085274723\\
0.835	0.000551810799695644	0.0332162174883389	0.732749123983336\\
0.835	0.000722090066287311	0.037995509188729	0.737115575253328\\
0.835	0.000915607803432999	0.0427829086109896	0.741955893907011\\
0.835	0.0011324871927904	0.0475782708632729	0.747315373652289\\
0.835	0.00137285047629673	0.0523814493324038	0.75322242404581\\
0.835	0.00163681893109843	0.057192295687301	0.759684310525652\\
0.835	0.00192451284439304	0.0620106598827802	0.766683748102665\\
0.835	0.00223605148818899	0.0668363901637434	0.774176695511256\\
0.835	0.00257155309398959	0.0716693330697584	0.782091614494493\\
0.835	0.00293113482740716	0.0765093334400312	0.79033034642109\\
0.835	0.00331491276271365	0.0813562344187764	0.79877062752665\\
0.835	0.00372300185733414	0.0862098774609879	0.807270128930359\\
0.835	0.00415551592628969	0.0910701023386145	0.815671782700193\\
0.835	0.00461256761659621	0.0959367471471425	0.82381005350733\\
0.835	0.00509426838162598	0.100809648312589	0.831517746503354\\
0.835	0.00560072845543873	0.105688640598912	0.838632911419452\\
0.835	0.00613205682708918	0.11057355711583	0.845005411376292\\
0.835	0.00668836121491815	0.115464229327074	0.850502769049302\\
0.835	0.00726974804083431	0.120360487059049	0.855014975717657\\
0.835	0.00787632240459385	0.125262158509929	0.858458041096606\\
0.835	0.00850818805808554	0.13016907025918	0.860776163494182\\
0.835	0.00916544737962849	0.135081047277508	0.861942500814225\\
0.835	0.00984820134829017	0.139997912937243	0.86195861464842\\
0.835	0.0105565495182326	0.144919489023162	0.860852735617091\\
0.835	0.0112905899930939	0.149845595743738	0.858677054109652\\
0.835	0.012050419400414	0.154776051742839	0.855504274967457\\
0.835	0.0128361328661109	0.159710674111862	0.851423687967753\\
0.835	0.0136478239890177	0.164649278402306	0.846537000484342\\
0.835	0.0144855848154856	0.169591678638796	0.840954157904248\\
0.835	0.0153495058140643	0.174537687332543	0.834789345403936\\
0.835	0.0162396758502651	0.179487115495259	0.8281573258062\\
0.835	0.0171561821614173	0.184439772653511	0.821170226448694\\
0.835	0.0180991103316243	0.189395466863524	0.813934846725692\\
0.835	0.0190685442668299	0.194354004726437	0.806550519904211\\
0.835	0.0200645661700016	0.199315191403999	0.799107529846305\\
0.835	0.0210872565164405	0.204278830634725	0.791686056496193\\
0.835	0.0221366940292259	0.20924472475049	0.784355603830601\\
0.835	0.023212955654804	0.214212674693577	0.777174850281117\\
0.835	0.0243161165387281	0.219182480034174	0.770191853862644\\
0.835	0.025446250001561	0.224153938988321	0.763444541554459\\
0.835	0.0266034275149466	0.229126848436298	0.756961413910483\\
0.835	0.0277877186778607	0.234101003941464	0.750762400416764\\
0.835	0.0289991911930496	0.239076199769547	0.744859807805785\\
0.835	0.0302379108436654	0.244052228908366	0.73925931151808\\
0.835	0.0315039414701067	0.24902888308801	0.733960949044504\\
0.835	0.0327973449470747	0.254005952801449	0.72896008240726\\
0.835	0.0341181811608519	0.258983227325592	0.724248305111179\\
0.835	0.0354665079868145	0.263960494742775	0.719814276220682\\
0.835	0.0368423812671862	0.268937541962694	0.71564447061263\\
0.835	0.0382458547890421	0.273914154744766	0.711723839838042\\
0.835	0.0396769802625737	0.278890117720924	0.708036382389729\\
0.835	0.0411358072996216	0.283865214418836	0.704565625567163\\
0.835	0.0426223833924861	0.288839227285554	0.701295023640316\\
0.835	0.0441367538930258	0.293811937711588	0.698208278749354\\
0.835	0.0456789619920503	0.298783126055387	0.695289592055349\\
0.835	0.0472490486990192	0.303752571668251	0.692523853197511\\
0.835	0.0488470528220537	0.308720052919643	0.689896776228118\\
0.835	0.0504730109482718	0.313685347222913	0.687394989990282\\
0.835	0.0521269574244539	0.318648231061428	0.685006090465982\\
0.835	0.0538089243380495	0.323608480015096	0.682718662029128\\
0.835	0.0555189414985328	0.328565868787292	0.68052227385347\\
0.835	0.0572570364191156	0.333520171232163	0.67840745699785\\
0.835	0.0590232342988274	0.338471160382329	0.676365666960044\\
0.835	0.0608175580049697	0.34341860847696	0.674389235783156\\
0.835	0.0626400280559541	0.348362286990222	0.672471317134862\\
0.835	0.064490662604533	0.3533019666601	0.670605827172181\\
0.835	0.0663694774214291	0.358237417517579	0.668787383459804\\
0.835	0.0682764858793752	0.363168408916186	0.667011243730728\\
0.835	0.0702116989375697	0.368094709561873	0.665273245863649\\
0.835	0.0721751251265572	0.373016087543259	0.663569750098932\\
0.835	0.0741667705335422	0.377932310362199	0.661897584220061\\
0.835	0.0761866387881432	0.382843144964686	0.660253992184444\\
0.835	0.0782347310485954	0.38774835777208	0.658636586490997\\
0.835	0.0803110459884098	0.392647714712651	0.65704330441587\\
0.835	0.0824155797834956	0.397540981253432	0.655472368126509\\
0.835	0.084548326099754	0.402427922432375	0.653922248592747\\
0.835	0.0867092760811502	0.4073083028908	0.652391633146569\\
0.835	0.0888984183382709	0.412181886906127	0.65087939649558\\
0.835	0.0911157389373742	0.417048438424897	0.649384574964844\\
0.835	0.0933612213899393	0.421907721096044	0.647906343724684\\
0.835	0.0956348466427212	0.42675949830445	0.646443996754806\\
0.835	0.0979365930683194	0.431603533204733	0.64499692929596\\
0.835	0.100266436456264	0.436439588755293	0.643564622546846\\
0.835	0.102624350004627	0.441267427752584	0.64214663037443\\
0.835	0.105010304312169	0.446086812865616	0.640742567819433\\
0.835	0.107424267371012	0.450897506670668	0.639352101193732\\
0.835	0.109866204559871	0.455699271686218	0.637974939582443\\
0.835	0.112336078637817	0.460491870408058	0.63661082757986\\
0.835	0.114833849738607	0.465275065344603	0.635259539104383\\
0.835	0.117359475365564	0.470048619052377	0.633920872153156\\
0.835	0.119912910387022	0.474812294171664	0.632594644371964\\
0.835	0.122494107032347	0.479565853462319	0.631280689329641\\
0.835	0.125103014888515	0.484309059839721	0.629978853399059\\
0.835	0.127739580897283	0.489041676410868	0.628690460824561\\
0.84	0	0	0.710202759607373\\
0.84	1.11327767495586e-05	0.00471862581271275	0.713003157837853\\
0.84	4.46175251031912e-05	0.009446325184051	0.715846486660142\\
0.84	0.000100583311362513	0.0141829653360114	0.718788838379794\\
0.84	0.000179158434668431	0.018928411755669	0.721886092228356\\
0.84	0.000280470402701511	0.02368252819604	0.725200636220892\\
0.84	0.000404645907256436	0.0284451766772965	0.728799085274723\\
0.84	0.000551810799695644	0.0332162174883388	0.732749123983336\\
0.84	0.000722090066287311	0.037995509188729	0.737115575253327\\
0.84	0.000915607803432999	0.0427829086109896	0.741955893907011\\
0.84	0.0011324871927904	0.047578270863273	0.747315373652289\\
0.84	0.00137285047629673	0.0523814493324038	0.753222424045811\\
0.84	0.00163681893109844	0.057192295687301	0.759684310525652\\
0.84	0.00192451284439304	0.0620106598827802	0.766683748102665\\
0.84	0.00223605148818899	0.0668363901637434	0.774176695511256\\
0.84	0.00257155309398959	0.0716693330697584	0.782091614494493\\
0.84	0.00293113482740716	0.0765093334400312	0.79033034642109\\
0.84	0.00331491276271365	0.0813562344187764	0.79877062752665\\
0.84	0.00372300185733414	0.0862098774609879	0.80727012893036\\
0.84	0.00415551592628969	0.0910701023386145	0.815671782700194\\
0.84	0.00461256761659621	0.0959367471471425	0.82381005350733\\
0.84	0.00509426838162598	0.100809648312589	0.831517746503354\\
0.84	0.00560072845543873	0.105688640598912	0.838632911419453\\
0.84	0.00613205682708918	0.11057355711583	0.845005411376293\\
0.84	0.00668836121491816	0.115464229327074	0.8505027690493\\
0.84	0.00726974804083431	0.120360487059049	0.855014975717655\\
0.84	0.00787632240459385	0.125262158509929	0.858458041096605\\
0.84	0.00850818805808555	0.13016907025918	0.860776163494183\\
0.84	0.00916544737962849	0.135081047277508	0.861942500814226\\
0.84	0.00984820134829017	0.139997912937243	0.86195861464842\\
0.84	0.0105565495182326	0.144919489023162	0.860852735617092\\
0.84	0.0112905899930939	0.149845595743738	0.858677054109652\\
0.84	0.012050419400414	0.154776051742839	0.855504274967456\\
0.84	0.0128361328661109	0.159710674111862	0.851423687967757\\
0.84	0.0136478239890177	0.164649278402306	0.846537000484339\\
0.84	0.0144855848154856	0.169591678638796	0.840954157904248\\
0.84	0.0153495058140643	0.174537687332543	0.834789345403934\\
0.84	0.0162396758502651	0.179487115495259	0.8281573258062\\
0.84	0.0171561821614173	0.184439772653511	0.821170226448695\\
0.84	0.0180991103316243	0.189395466863524	0.813934846725693\\
0.84	0.0190685442668299	0.194354004726437	0.806550519904212\\
0.84	0.0200645661700016	0.199315191403999	0.799107529846306\\
0.84	0.0210872565164405	0.204278830634725	0.791686056496193\\
0.84	0.0221366940292259	0.20924472475049	0.784355603830601\\
0.84	0.023212955654804	0.214212674693577	0.777174850281119\\
0.84	0.0243161165387281	0.219182480034174	0.770191853862645\\
0.84	0.025446250001561	0.224153938988321	0.763444541554459\\
0.84	0.0266034275149466	0.229126848436298	0.756961413910482\\
0.84	0.0277877186778607	0.234101003941464	0.750762400416764\\
0.84	0.0289991911930496	0.239076199769547	0.744859807805785\\
0.84	0.0302379108436654	0.244052228908366	0.73925931151808\\
0.84	0.0315039414701067	0.24902888308801	0.733960949044503\\
0.84	0.0327973449470747	0.254005952801449	0.728960082407262\\
0.84	0.0341181811608519	0.258983227325592	0.724248305111178\\
0.84	0.0354665079868145	0.263960494742775	0.71981427622068\\
0.84	0.0368423812671862	0.268937541962694	0.715644470612629\\
0.84	0.0382458547890421	0.273914154744766	0.711723839838041\\
0.84	0.0396769802625738	0.278890117720924	0.708036382389731\\
0.84	0.0411358072996216	0.283865214418836	0.704565625567164\\
0.84	0.0426223833924861	0.288839227285554	0.701295023640316\\
0.84	0.0441367538930258	0.293811937711588	0.698208278749354\\
0.84	0.0456789619920503	0.298783126055387	0.695289592055349\\
0.84	0.0472490486990192	0.303752571668251	0.692523853197512\\
0.84	0.0488470528220537	0.308720052919643	0.689896776228119\\
0.84	0.0504730109482718	0.313685347222913	0.687394989990281\\
0.84	0.0521269574244539	0.318648231061428	0.685006090465981\\
0.84	0.0538089243380495	0.323608480015097	0.682718662029127\\
0.84	0.0555189414985328	0.328565868787292	0.680522273853469\\
0.84	0.0572570364191156	0.333520171232163	0.67840745699785\\
0.84	0.0590232342988274	0.338471160382329	0.676365666960044\\
0.84	0.0608175580049697	0.34341860847696	0.674389235783157\\
0.84	0.0626400280559541	0.348362286990222	0.672471317134862\\
0.84	0.064490662604533	0.3533019666601	0.670605827172182\\
0.84	0.0663694774214291	0.358237417517579	0.668787383459802\\
0.84	0.0682764858793752	0.363168408916186	0.667011243730727\\
0.84	0.0702116989375697	0.368094709561873	0.665273245863649\\
0.84	0.0721751251265572	0.373016087543259	0.663569750098933\\
0.84	0.0741667705335423	0.377932310362199	0.66189758422006\\
0.84	0.0761866387881432	0.382843144964686	0.660253992184442\\
0.84	0.0782347310485954	0.38774835777208	0.658636586490997\\
0.84	0.0803110459884098	0.392647714712651	0.657043304415872\\
0.84	0.0824155797834956	0.397540981253432	0.655472368126509\\
0.84	0.084548326099754	0.402427922432375	0.653922248592746\\
0.84	0.0867092760811502	0.4073083028908	0.65239163314657\\
0.84	0.0888984183382709	0.412181886906127	0.650879396495578\\
0.84	0.0911157389373742	0.417048438424896	0.649384574964846\\
0.84	0.0933612213899392	0.421907721096044	0.647906343724685\\
0.84	0.0956348466427212	0.42675949830445	0.646443996754806\\
0.84	0.0979365930683194	0.431603533204733	0.644996929295961\\
0.84	0.100266436456264	0.436439588755293	0.643564622546845\\
0.84	0.102624350004627	0.441267427752584	0.64214663037443\\
0.84	0.105010304312169	0.446086812865616	0.640742567819432\\
0.84	0.107424267371012	0.450897506670668	0.639352101193731\\
0.84	0.109866204559871	0.455699271686218	0.637974939582444\\
0.84	0.112336078637817	0.460491870408058	0.636610827579862\\
0.84	0.114833849738607	0.465275065344603	0.635259539104382\\
0.84	0.117359475365564	0.470048619052377	0.633920872153155\\
0.84	0.119912910387022	0.474812294171664	0.632594644371966\\
0.84	0.122494107032347	0.479565853462319	0.63128068932964\\
0.84	0.125103014888515	0.484309059839721	0.629978853399057\\
0.84	0.127739580897283	0.489041676410868	0.62869046082456\\
0.845	0	0	0.710202759607373\\
0.845	1.11327767495586e-05	0.00471862581271275	0.713003157837853\\
0.845	4.46175251031912e-05	0.009446325184051	0.715846486660142\\
0.845	0.000100583311362513	0.0141829653360114	0.718788838379794\\
0.845	0.000179158434668431	0.018928411755669	0.721886092228356\\
0.845	0.000280470402701511	0.02368252819604	0.725200636220892\\
0.845	0.000404645907256436	0.0284451766772965	0.728799085274723\\
0.845	0.000551810799695644	0.0332162174883388	0.732749123983336\\
0.845	0.000722090066287311	0.037995509188729	0.737115575253327\\
0.845	0.000915607803432999	0.0427829086109896	0.741955893907011\\
0.845	0.0011324871927904	0.0475782708632729	0.747315373652289\\
0.845	0.00137285047629673	0.0523814493324038	0.753222424045811\\
0.845	0.00163681893109844	0.057192295687301	0.759684310525652\\
0.845	0.00192451284439304	0.0620106598827802	0.766683748102665\\
0.845	0.00223605148818898	0.0668363901637434	0.774176695511256\\
0.845	0.00257155309398959	0.0716693330697584	0.782091614494494\\
0.845	0.00293113482740716	0.0765093334400312	0.79033034642109\\
0.845	0.00331491276271365	0.0813562344187764	0.79877062752665\\
0.845	0.00372300185733414	0.0862098774609879	0.807270128930359\\
0.845	0.00415551592628969	0.0910701023386146	0.815671782700194\\
0.845	0.00461256761659621	0.0959367471471424	0.82381005350733\\
0.845	0.00509426838162598	0.100809648312589	0.831517746503354\\
0.845	0.00560072845543873	0.105688640598912	0.838632911419451\\
0.845	0.00613205682708918	0.11057355711583	0.845005411376293\\
0.845	0.00668836121491816	0.115464229327074	0.850502769049302\\
0.845	0.00726974804083431	0.120360487059049	0.855014975717657\\
0.845	0.00787632240459385	0.125262158509929	0.858458041096605\\
0.845	0.00850818805808554	0.13016907025918	0.860776163494181\\
0.845	0.00916544737962849	0.135081047277508	0.861942500814223\\
0.845	0.00984820134829017	0.139997912937243	0.861958614648421\\
0.845	0.0105565495182326	0.144919489023162	0.86085273561709\\
0.845	0.0112905899930939	0.149845595743738	0.858677054109649\\
0.845	0.012050419400414	0.154776051742839	0.855504274967456\\
0.845	0.0128361328661109	0.159710674111862	0.851423687967754\\
0.845	0.0136478239890177	0.164649278402306	0.846537000484339\\
0.845	0.0144855848154856	0.169591678638796	0.84095415790425\\
0.845	0.0153495058140643	0.174537687332543	0.834789345403935\\
0.845	0.0162396758502651	0.179487115495259	0.828157325806202\\
0.845	0.0171561821614173	0.184439772653511	0.821170226448694\\
0.845	0.0180991103316243	0.189395466863524	0.813934846725691\\
0.845	0.0190685442668299	0.194354004726437	0.806550519904212\\
0.845	0.0200645661700016	0.199315191403999	0.799107529846304\\
0.845	0.0210872565164405	0.204278830634725	0.791686056496194\\
0.845	0.0221366940292259	0.20924472475049	0.784355603830603\\
0.845	0.023212955654804	0.214212674693577	0.777174850281119\\
0.845	0.0243161165387281	0.219182480034174	0.770191853862644\\
0.845	0.025446250001561	0.224153938988321	0.763444541554456\\
0.845	0.0266034275149466	0.229126848436298	0.756961413910483\\
0.845	0.0277877186778607	0.234101003941464	0.750762400416765\\
0.845	0.0289991911930496	0.239076199769547	0.744859807805785\\
0.845	0.0302379108436654	0.244052228908366	0.739259311518078\\
0.845	0.0315039414701067	0.24902888308801	0.733960949044504\\
0.845	0.0327973449470747	0.254005952801449	0.728960082407262\\
0.845	0.0341181811608519	0.258983227325592	0.724248305111178\\
0.845	0.0354665079868145	0.263960494742775	0.719814276220681\\
0.845	0.0368423812671862	0.268937541962694	0.715644470612628\\
0.845	0.0382458547890421	0.273914154744766	0.711723839838042\\
0.845	0.0396769802625738	0.278890117720924	0.70803638238973\\
0.845	0.0411358072996216	0.283865214418836	0.704565625567163\\
0.845	0.0426223833924861	0.288839227285554	0.701295023640316\\
0.845	0.0441367538930258	0.293811937711588	0.698208278749355\\
0.845	0.0456789619920503	0.298783126055387	0.695289592055348\\
0.845	0.0472490486990192	0.303752571668251	0.692523853197512\\
0.845	0.0488470528220538	0.308720052919643	0.689896776228119\\
0.845	0.0504730109482718	0.313685347222913	0.687394989990281\\
0.845	0.0521269574244539	0.318648231061428	0.685006090465981\\
0.845	0.0538089243380495	0.323608480015097	0.682718662029127\\
0.845	0.0555189414985328	0.328565868787292	0.680522273853469\\
0.845	0.0572570364191156	0.333520171232163	0.67840745699785\\
0.845	0.0590232342988274	0.338471160382329	0.676365666960043\\
0.845	0.0608175580049697	0.34341860847696	0.674389235783157\\
0.845	0.0626400280559541	0.348362286990222	0.672471317134862\\
0.845	0.064490662604533	0.3533019666601	0.670605827172182\\
0.845	0.0663694774214291	0.358237417517579	0.668787383459805\\
0.845	0.0682764858793752	0.363168408916186	0.667011243730728\\
0.845	0.0702116989375697	0.368094709561873	0.665273245863649\\
0.845	0.0721751251265572	0.373016087543259	0.663569750098932\\
0.845	0.0741667705335422	0.377932310362199	0.66189758422006\\
0.845	0.0761866387881432	0.382843144964686	0.660253992184442\\
0.845	0.0782347310485954	0.38774835777208	0.658636586490997\\
0.845	0.0803110459884098	0.392647714712651	0.657043304415871\\
0.845	0.0824155797834956	0.397540981253432	0.65547236812651\\
0.845	0.084548326099754	0.402427922432375	0.653922248592745\\
0.845	0.0867092760811502	0.407308302890799	0.652391633146569\\
0.845	0.0888984183382709	0.412181886906127	0.650879396495579\\
0.845	0.0911157389373742	0.417048438424896	0.649384574964843\\
0.845	0.0933612213899393	0.421907721096044	0.647906343724686\\
0.845	0.0956348466427212	0.42675949830445	0.646443996754806\\
0.845	0.0979365930683194	0.431603533204733	0.644996929295962\\
0.845	0.100266436456264	0.436439588755293	0.643564622546847\\
0.845	0.102624350004627	0.441267427752585	0.64214663037443\\
0.845	0.105010304312169	0.446086812865616	0.640742567819433\\
0.845	0.107424267371012	0.450897506670668	0.63935210119373\\
0.845	0.109866204559871	0.455699271686218	0.637974939582443\\
0.845	0.112336078637817	0.460491870408058	0.636610827579862\\
0.845	0.114833849738607	0.465275065344603	0.635259539104382\\
0.845	0.117359475365564	0.470048619052377	0.633920872153156\\
0.845	0.119912910387022	0.474812294171664	0.632594644371966\\
0.845	0.122494107032347	0.479565853462319	0.63128068932964\\
0.845	0.125103014888515	0.484309059839721	0.629978853399058\\
0.845	0.127739580897283	0.489041676410868	0.62869046082456\\
0.85	0	0	0.710202759607373\\
0.85	1.11327767495586e-05	0.00471862581271275	0.713003157837853\\
0.85	4.46175251031912e-05	0.009446325184051	0.715846486660142\\
0.85	0.000100583311362513	0.0141829653360114	0.718788838379794\\
0.85	0.000179158434668431	0.018928411755669	0.721886092228356\\
0.85	0.000280470402701511	0.02368252819604	0.725200636220892\\
0.85	0.000404645907256436	0.0284451766772965	0.728799085274723\\
0.85	0.000551810799695644	0.0332162174883389	0.732749123983336\\
0.85	0.000722090066287311	0.037995509188729	0.737115575253328\\
0.85	0.000915607803432999	0.0427829086109896	0.741955893907011\\
0.85	0.0011324871927904	0.047578270863273	0.747315373652289\\
0.85	0.00137285047629673	0.0523814493324038	0.753222424045811\\
0.85	0.00163681893109844	0.057192295687301	0.759684310525652\\
0.85	0.00192451284439304	0.0620106598827802	0.766683748102665\\
0.85	0.00223605148818898	0.0668363901637434	0.774176695511257\\
0.85	0.00257155309398959	0.0716693330697584	0.782091614494493\\
0.85	0.00293113482740716	0.0765093334400312	0.79033034642109\\
0.85	0.00331491276271365	0.0813562344187764	0.79877062752665\\
0.85	0.00372300185733413	0.0862098774609879	0.807270128930359\\
0.85	0.00415551592628969	0.0910701023386145	0.815671782700193\\
0.85	0.00461256761659621	0.0959367471471424	0.82381005350733\\
0.85	0.00509426838162598	0.100809648312589	0.831517746503354\\
0.85	0.00560072845543873	0.105688640598912	0.838632911419452\\
0.85	0.00613205682708918	0.11057355711583	0.845005411376292\\
0.85	0.00668836121491816	0.115464229327074	0.850502769049303\\
0.85	0.00726974804083431	0.120360487059049	0.855014975717659\\
0.85	0.00787632240459385	0.125262158509929	0.858458041096607\\
0.85	0.00850818805808554	0.13016907025918	0.860776163494182\\
0.85	0.00916544737962849	0.135081047277508	0.861942500814224\\
0.85	0.00984820134829017	0.139997912937243	0.86195861464842\\
0.85	0.0105565495182326	0.144919489023162	0.860852735617089\\
0.85	0.0112905899930939	0.149845595743738	0.858677054109649\\
0.85	0.012050419400414	0.154776051742839	0.855504274967458\\
0.85	0.0128361328661109	0.159710674111862	0.851423687967754\\
0.85	0.0136478239890177	0.164649278402306	0.846537000484341\\
0.85	0.0144855848154856	0.169591678638796	0.840954157904249\\
0.85	0.0153495058140643	0.174537687332543	0.834789345403935\\
0.85	0.0162396758502651	0.179487115495259	0.8281573258062\\
0.85	0.0171561821614173	0.184439772653511	0.821170226448692\\
0.85	0.0180991103316243	0.189395466863524	0.813934846725691\\
0.85	0.0190685442668299	0.194354004726437	0.806550519904213\\
0.85	0.0200645661700016	0.199315191403999	0.799107529846306\\
0.85	0.0210872565164405	0.204278830634725	0.791686056496194\\
0.85	0.0221366940292259	0.20924472475049	0.784355603830602\\
0.85	0.023212955654804	0.214212674693577	0.777174850281119\\
0.85	0.0243161165387281	0.219182480034174	0.770191853862642\\
0.85	0.025446250001561	0.224153938988321	0.763444541554458\\
0.85	0.0266034275149466	0.229126848436298	0.756961413910484\\
0.85	0.0277877186778607	0.234101003941464	0.750762400416764\\
0.85	0.0289991911930496	0.239076199769547	0.744859807805784\\
0.85	0.0302379108436654	0.244052228908366	0.739259311518081\\
0.85	0.0315039414701067	0.24902888308801	0.733960949044504\\
0.85	0.0327973449470747	0.254005952801449	0.728960082407261\\
0.85	0.0341181811608519	0.258983227325592	0.72424830511118\\
0.85	0.0354665079868145	0.263960494742775	0.71981427622068\\
0.85	0.0368423812671862	0.268937541962694	0.715644470612629\\
0.85	0.0382458547890421	0.273914154744766	0.711723839838043\\
0.85	0.0396769802625738	0.278890117720924	0.70803638238973\\
0.85	0.0411358072996216	0.283865214418836	0.704565625567164\\
0.85	0.0426223833924861	0.288839227285554	0.701295023640316\\
0.85	0.0441367538930258	0.293811937711588	0.698208278749354\\
0.85	0.0456789619920503	0.298783126055387	0.695289592055349\\
0.85	0.0472490486990192	0.303752571668251	0.692523853197512\\
0.85	0.0488470528220538	0.308720052919643	0.689896776228119\\
0.85	0.0504730109482718	0.313685347222913	0.687394989990281\\
0.85	0.0521269574244539	0.318648231061428	0.68500609046598\\
0.85	0.0538089243380495	0.323608480015096	0.682718662029128\\
0.85	0.0555189414985328	0.328565868787292	0.68052227385347\\
0.85	0.0572570364191156	0.333520171232163	0.67840745699785\\
0.85	0.0590232342988274	0.338471160382329	0.676365666960042\\
0.85	0.0608175580049697	0.34341860847696	0.674389235783157\\
0.85	0.0626400280559541	0.348362286990222	0.672471317134863\\
0.85	0.064490662604533	0.3533019666601	0.670605827172181\\
0.85	0.0663694774214291	0.358237417517579	0.668787383459803\\
0.85	0.0682764858793752	0.363168408916186	0.667011243730728\\
0.85	0.0702116989375697	0.368094709561873	0.66527324586365\\
0.85	0.0721751251265572	0.373016087543259	0.663569750098933\\
0.85	0.0741667705335422	0.377932310362199	0.66189758422006\\
0.85	0.0761866387881432	0.382843144964686	0.660253992184442\\
0.85	0.0782347310485954	0.38774835777208	0.658636586490996\\
0.85	0.0803110459884098	0.392647714712651	0.657043304415871\\
0.85	0.0824155797834956	0.397540981253432	0.655472368126509\\
0.85	0.084548326099754	0.402427922432375	0.653922248592746\\
0.85	0.0867092760811502	0.4073083028908	0.65239163314657\\
0.85	0.0888984183382709	0.412181886906127	0.650879396495578\\
0.85	0.0911157389373742	0.417048438424896	0.649384574964844\\
0.85	0.0933612213899392	0.421907721096044	0.647906343724684\\
0.85	0.0956348466427212	0.42675949830445	0.646443996754806\\
0.85	0.0979365930683194	0.431603533204733	0.644996929295962\\
0.85	0.100266436456264	0.436439588755293	0.643564622546847\\
0.85	0.102624350004627	0.441267427752584	0.64214663037443\\
0.85	0.105010304312169	0.446086812865616	0.640742567819433\\
0.85	0.107424267371012	0.450897506670668	0.639352101193731\\
0.85	0.109866204559871	0.455699271686218	0.637974939582443\\
0.85	0.112336078637817	0.460491870408058	0.636610827579862\\
0.85	0.114833849738607	0.465275065344603	0.635259539104382\\
0.85	0.117359475365564	0.470048619052377	0.633920872153155\\
0.85	0.119912910387023	0.474812294171664	0.632594644371967\\
0.85	0.122494107032347	0.479565853462319	0.631280689329641\\
0.85	0.125103014888515	0.484309059839721	0.629978853399059\\
0.85	0.127739580897283	0.489041676410868	0.628690460824562\\
0.855	0	0	0.710202759607373\\
0.855	1.11327767495586e-05	0.00471862581271274	0.713003157837853\\
0.855	4.46175251031912e-05	0.009446325184051	0.715846486660142\\
0.855	0.000100583311362513	0.0141829653360114	0.718788838379794\\
0.855	0.000179158434668431	0.018928411755669	0.721886092228356\\
0.855	0.000280470402701511	0.02368252819604	0.725200636220892\\
0.855	0.000404645907256436	0.0284451766772965	0.728799085274723\\
0.855	0.000551810799695644	0.0332162174883388	0.732749123983336\\
0.855	0.000722090066287311	0.037995509188729	0.737115575253327\\
0.855	0.000915607803432999	0.0427829086109896	0.741955893907011\\
0.855	0.0011324871927904	0.0475782708632729	0.747315373652289\\
0.855	0.00137285047629673	0.0523814493324038	0.753222424045811\\
0.855	0.00163681893109844	0.057192295687301	0.759684310525652\\
0.855	0.00192451284439304	0.0620106598827802	0.766683748102665\\
0.855	0.00223605148818899	0.0668363901637434	0.774176695511256\\
0.855	0.00257155309398959	0.0716693330697584	0.782091614494494\\
0.855	0.00293113482740716	0.0765093334400312	0.79033034642109\\
0.855	0.00331491276271365	0.0813562344187764	0.79877062752665\\
0.855	0.00372300185733414	0.0862098774609879	0.807270128930359\\
0.855	0.00415551592628969	0.0910701023386145	0.815671782700193\\
0.855	0.00461256761659621	0.0959367471471424	0.823810053507331\\
0.855	0.00509426838162598	0.100809648312589	0.831517746503353\\
0.855	0.00560072845543873	0.105688640598912	0.838632911419452\\
0.855	0.00613205682708918	0.11057355711583	0.845005411376293\\
0.855	0.00668836121491816	0.115464229327074	0.850502769049301\\
0.855	0.00726974804083431	0.120360487059049	0.855014975717657\\
0.855	0.00787632240459385	0.125262158509929	0.858458041096606\\
0.855	0.00850818805808554	0.13016907025918	0.860776163494183\\
0.855	0.00916544737962849	0.135081047277508	0.861942500814224\\
0.855	0.00984820134829017	0.139997912937243	0.861958614648419\\
0.855	0.0105565495182326	0.144919489023162	0.860852735617092\\
0.855	0.0112905899930939	0.149845595743738	0.858677054109652\\
0.855	0.012050419400414	0.154776051742839	0.855504274967458\\
0.855	0.0128361328661109	0.159710674111862	0.851423687967753\\
0.855	0.0136478239890177	0.164649278402306	0.846537000484341\\
0.855	0.0144855848154856	0.169591678638796	0.840954157904246\\
0.855	0.0153495058140643	0.174537687332543	0.834789345403935\\
0.855	0.0162396758502651	0.179487115495259	0.828157325806199\\
0.855	0.0171561821614173	0.184439772653511	0.821170226448694\\
0.855	0.0180991103316243	0.189395466863524	0.813934846725693\\
0.855	0.0190685442668299	0.194354004726437	0.806550519904213\\
0.855	0.0200645661700016	0.199315191403999	0.799107529846304\\
0.855	0.0210872565164405	0.204278830634725	0.791686056496194\\
0.855	0.0221366940292259	0.20924472475049	0.784355603830603\\
0.855	0.023212955654804	0.214212674693577	0.777174850281116\\
0.855	0.0243161165387281	0.219182480034174	0.770191853862643\\
0.855	0.025446250001561	0.224153938988321	0.76344454155446\\
0.855	0.0266034275149466	0.229126848436298	0.756961413910482\\
0.855	0.0277877186778607	0.234101003941464	0.750762400416763\\
0.855	0.0289991911930496	0.239076199769547	0.744859807805787\\
0.855	0.0302379108436654	0.244052228908366	0.739259311518081\\
0.855	0.0315039414701067	0.24902888308801	0.733960949044503\\
0.855	0.0327973449470747	0.254005952801449	0.728960082407262\\
0.855	0.0341181811608519	0.258983227325592	0.724248305111178\\
0.855	0.0354665079868145	0.263960494742775	0.71981427622068\\
0.855	0.0368423812671862	0.268937541962694	0.71564447061263\\
0.855	0.0382458547890421	0.273914154744766	0.711723839838041\\
0.855	0.0396769802625738	0.278890117720924	0.708036382389732\\
0.855	0.0411358072996216	0.283865214418836	0.704565625567164\\
0.855	0.0426223833924861	0.288839227285554	0.701295023640315\\
0.855	0.0441367538930258	0.293811937711588	0.698208278749353\\
0.855	0.0456789619920503	0.298783126055387	0.69528959205535\\
0.855	0.0472490486990192	0.303752571668251	0.692523853197512\\
0.855	0.0488470528220538	0.308720052919643	0.689896776228119\\
0.855	0.0504730109482718	0.313685347222913	0.687394989990283\\
0.855	0.0521269574244539	0.318648231061428	0.685006090465981\\
0.855	0.0538089243380495	0.323608480015096	0.682718662029128\\
0.855	0.0555189414985328	0.328565868787292	0.680522273853469\\
0.855	0.0572570364191156	0.333520171232163	0.67840745699785\\
0.855	0.0590232342988274	0.338471160382329	0.676365666960044\\
0.855	0.0608175580049697	0.34341860847696	0.674389235783156\\
0.855	0.0626400280559541	0.348362286990222	0.672471317134862\\
0.855	0.064490662604533	0.3533019666601	0.670605827172182\\
0.855	0.0663694774214291	0.358237417517579	0.668787383459804\\
0.855	0.0682764858793752	0.363168408916186	0.667011243730728\\
0.855	0.0702116989375697	0.368094709561873	0.665273245863649\\
0.855	0.0721751251265572	0.373016087543259	0.663569750098933\\
0.855	0.0741667705335423	0.377932310362199	0.661897584220061\\
0.855	0.0761866387881432	0.382843144964686	0.660253992184443\\
0.855	0.0782347310485954	0.38774835777208	0.658636586490997\\
0.855	0.0803110459884098	0.392647714712651	0.65704330441587\\
0.855	0.0824155797834956	0.397540981253432	0.655472368126509\\
0.855	0.084548326099754	0.402427922432375	0.653922248592745\\
0.855	0.0867092760811502	0.407308302890799	0.65239163314657\\
0.855	0.0888984183382709	0.412181886906127	0.650879396495579\\
0.855	0.0911157389373742	0.417048438424896	0.649384574964844\\
0.855	0.0933612213899392	0.421907721096044	0.647906343724685\\
0.855	0.0956348466427212	0.42675949830445	0.646443996754806\\
0.855	0.0979365930683194	0.431603533204733	0.644996929295961\\
0.855	0.100266436456264	0.436439588755293	0.643564622546847\\
0.855	0.102624350004627	0.441267427752585	0.642146630374431\\
0.855	0.105010304312169	0.446086812865616	0.640742567819432\\
0.855	0.107424267371012	0.450897506670668	0.639352101193731\\
0.855	0.109866204559871	0.455699271686218	0.637974939582444\\
0.855	0.112336078637817	0.460491870408058	0.636610827579861\\
0.855	0.114833849738607	0.465275065344603	0.635259539104382\\
0.855	0.117359475365564	0.470048619052377	0.633920872153155\\
0.855	0.119912910387022	0.474812294171664	0.632594644371966\\
0.855	0.122494107032347	0.479565853462319	0.631280689329641\\
0.855	0.125103014888515	0.484309059839721	0.629978853399058\\
0.855	0.127739580897283	0.489041676410868	0.62869046082456\\
0.86	0	0	0.710202759607373\\
0.86	1.11327767495586e-05	0.00471862581271275	0.713003157837853\\
0.86	4.46175251031912e-05	0.009446325184051	0.715846486660142\\
0.86	0.000100583311362513	0.0141829653360114	0.718788838379794\\
0.86	0.000179158434668431	0.018928411755669	0.721886092228356\\
0.86	0.000280470402701511	0.02368252819604	0.725200636220892\\
0.86	0.000404645907256436	0.0284451766772965	0.728799085274723\\
0.86	0.000551810799695644	0.0332162174883389	0.732749123983336\\
0.86	0.000722090066287311	0.037995509188729	0.737115575253328\\
0.86	0.000915607803432999	0.0427829086109896	0.741955893907011\\
0.86	0.0011324871927904	0.0475782708632729	0.747315373652289\\
0.86	0.00137285047629673	0.0523814493324038	0.753222424045811\\
0.86	0.00163681893109844	0.057192295687301	0.759684310525652\\
0.86	0.00192451284439304	0.0620106598827802	0.766683748102665\\
0.86	0.00223605148818899	0.0668363901637434	0.774176695511256\\
0.86	0.00257155309398959	0.0716693330697584	0.782091614494494\\
0.86	0.00293113482740716	0.0765093334400312	0.79033034642109\\
0.86	0.00331491276271365	0.0813562344187764	0.79877062752665\\
0.86	0.00372300185733414	0.0862098774609879	0.807270128930359\\
0.86	0.00415551592628969	0.0910701023386145	0.815671782700193\\
0.86	0.00461256761659621	0.0959367471471425	0.82381005350733\\
0.86	0.00509426838162598	0.100809648312589	0.831517746503354\\
0.86	0.00560072845543873	0.105688640598912	0.838632911419452\\
0.86	0.00613205682708918	0.11057355711583	0.845005411376293\\
0.86	0.00668836121491815	0.115464229327074	0.850502769049303\\
0.86	0.00726974804083431	0.120360487059049	0.855014975717656\\
0.86	0.00787632240459385	0.125262158509929	0.858458041096605\\
0.86	0.00850818805808554	0.13016907025918	0.860776163494181\\
0.86	0.00916544737962849	0.135081047277508	0.861942500814223\\
0.86	0.00984820134829017	0.139997912937243	0.86195861464842\\
0.86	0.0105565495182326	0.144919489023162	0.860852735617092\\
0.86	0.0112905899930939	0.149845595743738	0.858677054109649\\
0.86	0.012050419400414	0.154776051742839	0.855504274967457\\
0.86	0.0128361328661109	0.159710674111862	0.851423687967753\\
0.86	0.0136478239890177	0.164649278402306	0.84653700048434\\
0.86	0.0144855848154856	0.169591678638796	0.840954157904247\\
0.86	0.0153495058140643	0.174537687332543	0.834789345403936\\
0.86	0.0162396758502651	0.179487115495259	0.828157325806203\\
0.86	0.0171561821614173	0.184439772653511	0.821170226448694\\
0.86	0.0180991103316243	0.189395466863524	0.813934846725691\\
0.86	0.0190685442668299	0.194354004726437	0.806550519904212\\
0.86	0.0200645661700016	0.199315191403999	0.799107529846303\\
0.86	0.0210872565164405	0.204278830634725	0.791686056496194\\
0.86	0.0221366940292259	0.20924472475049	0.7843556038306\\
0.86	0.023212955654804	0.214212674693577	0.777174850281117\\
0.86	0.0243161165387281	0.219182480034174	0.770191853862644\\
0.86	0.025446250001561	0.224153938988321	0.763444541554457\\
0.86	0.0266034275149466	0.229126848436298	0.756961413910482\\
0.86	0.0277877186778607	0.234101003941464	0.750762400416766\\
0.86	0.0289991911930496	0.239076199769547	0.744859807805785\\
0.86	0.0302379108436654	0.244052228908366	0.739259311518078\\
0.86	0.0315039414701067	0.24902888308801	0.733960949044503\\
0.86	0.0327973449470747	0.254005952801449	0.728960082407261\\
0.86	0.0341181811608519	0.258983227325592	0.724248305111178\\
0.86	0.0354665079868145	0.263960494742775	0.719814276220681\\
0.86	0.0368423812671862	0.268937541962694	0.715644470612629\\
0.86	0.0382458547890421	0.273914154744766	0.711723839838041\\
0.86	0.0396769802625738	0.278890117720924	0.708036382389729\\
0.86	0.0411358072996216	0.283865214418836	0.704565625567165\\
0.86	0.0426223833924861	0.288839227285554	0.701295023640317\\
0.86	0.0441367538930258	0.293811937711588	0.698208278749353\\
0.86	0.0456789619920503	0.298783126055387	0.695289592055348\\
0.86	0.0472490486990192	0.303752571668251	0.692523853197511\\
0.86	0.0488470528220538	0.308720052919643	0.689896776228118\\
0.86	0.0504730109482718	0.313685347222913	0.687394989990282\\
0.86	0.0521269574244539	0.318648231061428	0.685006090465981\\
0.86	0.0538089243380495	0.323608480015096	0.682718662029129\\
0.86	0.0555189414985328	0.328565868787292	0.680522273853471\\
0.86	0.0572570364191156	0.333520171232163	0.67840745699785\\
0.86	0.0590232342988274	0.338471160382329	0.676365666960043\\
0.86	0.0608175580049697	0.34341860847696	0.674389235783155\\
0.86	0.0626400280559541	0.348362286990222	0.672471317134862\\
0.86	0.064490662604533	0.3533019666601	0.670605827172182\\
0.86	0.0663694774214291	0.358237417517579	0.668787383459803\\
0.86	0.0682764858793752	0.363168408916186	0.667011243730728\\
0.86	0.0702116989375697	0.368094709561873	0.66527324586365\\
0.86	0.0721751251265572	0.373016087543259	0.663569750098931\\
0.86	0.0741667705335423	0.377932310362199	0.661897584220061\\
0.86	0.0761866387881432	0.382843144964686	0.660253992184442\\
0.86	0.0782347310485954	0.38774835777208	0.658636586490998\\
0.86	0.0803110459884098	0.392647714712651	0.657043304415871\\
0.86	0.0824155797834956	0.397540981253432	0.655472368126509\\
0.86	0.084548326099754	0.402427922432375	0.653922248592746\\
0.86	0.0867092760811502	0.407308302890799	0.65239163314657\\
0.86	0.0888984183382709	0.412181886906127	0.650879396495578\\
0.86	0.0911157389373742	0.417048438424896	0.649384574964843\\
0.86	0.0933612213899392	0.421907721096044	0.647906343724685\\
0.86	0.0956348466427212	0.42675949830445	0.646443996754806\\
0.86	0.0979365930683194	0.431603533204733	0.644996929295962\\
0.86	0.100266436456264	0.436439588755293	0.643564622546846\\
0.86	0.102624350004627	0.441267427752584	0.642146630374431\\
0.86	0.105010304312169	0.446086812865616	0.640742567819434\\
0.86	0.107424267371012	0.450897506670668	0.639352101193731\\
0.86	0.109866204559871	0.455699271686218	0.637974939582443\\
0.86	0.112336078637817	0.460491870408058	0.636610827579861\\
0.86	0.114833849738607	0.465275065344603	0.635259539104382\\
0.86	0.117359475365564	0.470048619052377	0.633920872153155\\
0.86	0.119912910387023	0.474812294171664	0.632594644371965\\
0.86	0.122494107032347	0.479565853462319	0.631280689329641\\
0.86	0.125103014888515	0.484309059839721	0.62997885339906\\
0.86	0.127739580897283	0.489041676410868	0.628690460824558\\
0.865	0	0	0.710202759607373\\
0.865	1.11327767495586e-05	0.00471862581271274	0.713003157837853\\
0.865	4.46175251031912e-05	0.009446325184051	0.715846486660142\\
0.865	0.000100583311362513	0.0141829653360114	0.718788838379794\\
0.865	0.000179158434668431	0.018928411755669	0.721886092228356\\
0.865	0.000280470402701511	0.02368252819604	0.725200636220892\\
0.865	0.000404645907256436	0.0284451766772965	0.728799085274723\\
0.865	0.000551810799695644	0.0332162174883389	0.732749123983336\\
0.865	0.000722090066287312	0.037995509188729	0.737115575253328\\
0.865	0.000915607803432999	0.0427829086109896	0.741955893907011\\
0.865	0.0011324871927904	0.0475782708632729	0.747315373652289\\
0.865	0.00137285047629673	0.0523814493324038	0.753222424045811\\
0.865	0.00163681893109843	0.057192295687301	0.759684310525652\\
0.865	0.00192451284439304	0.0620106598827802	0.766683748102665\\
0.865	0.00223605148818899	0.0668363901637434	0.774176695511256\\
0.865	0.00257155309398959	0.0716693330697584	0.782091614494493\\
0.865	0.00293113482740716	0.0765093334400313	0.79033034642109\\
0.865	0.00331491276271365	0.0813562344187764	0.79877062752665\\
0.865	0.00372300185733414	0.0862098774609879	0.807270128930359\\
0.865	0.00415551592628969	0.0910701023386145	0.815671782700193\\
0.865	0.00461256761659621	0.0959367471471425	0.82381005350733\\
0.865	0.00509426838162598	0.100809648312589	0.831517746503354\\
0.865	0.00560072845543873	0.105688640598912	0.838632911419452\\
0.865	0.00613205682708918	0.11057355711583	0.845005411376293\\
0.865	0.00668836121491815	0.115464229327074	0.850502769049302\\
0.865	0.00726974804083431	0.120360487059049	0.855014975717657\\
0.865	0.00787632240459385	0.125262158509929	0.858458041096607\\
0.865	0.00850818805808555	0.13016907025918	0.860776163494182\\
0.865	0.00916544737962849	0.135081047277508	0.861942500814225\\
0.865	0.00984820134829017	0.139997912937243	0.861958614648421\\
0.865	0.0105565495182326	0.144919489023162	0.86085273561709\\
0.865	0.0112905899930939	0.149845595743738	0.858677054109649\\
0.865	0.012050419400414	0.154776051742839	0.855504274967457\\
0.865	0.0128361328661109	0.159710674111862	0.851423687967755\\
0.865	0.0136478239890177	0.164649278402306	0.84653700048434\\
0.865	0.0144855848154856	0.169591678638796	0.840954157904249\\
0.865	0.0153495058140643	0.174537687332543	0.834789345403937\\
0.865	0.0162396758502651	0.179487115495259	0.828157325806202\\
0.865	0.0171561821614173	0.184439772653511	0.821170226448692\\
0.865	0.0180991103316243	0.189395466863524	0.813934846725692\\
0.865	0.0190685442668299	0.194354004726437	0.806550519904212\\
0.865	0.0200645661700016	0.199315191403999	0.799107529846303\\
0.865	0.0210872565164405	0.204278830634725	0.791686056496194\\
0.865	0.0221366940292259	0.20924472475049	0.784355603830601\\
0.865	0.023212955654804	0.214212674693577	0.77717485028112\\
0.865	0.0243161165387281	0.219182480034174	0.770191853862644\\
0.865	0.025446250001561	0.224153938988321	0.76344454155446\\
0.865	0.0266034275149466	0.229126848436298	0.756961413910482\\
0.865	0.0277877186778607	0.234101003941464	0.750762400416763\\
0.865	0.0289991911930496	0.239076199769547	0.744859807805784\\
0.865	0.0302379108436654	0.244052228908366	0.73925931151808\\
0.865	0.0315039414701067	0.24902888308801	0.733960949044503\\
0.865	0.0327973449470747	0.254005952801449	0.72896008240726\\
0.865	0.0341181811608519	0.258983227325591	0.724248305111179\\
0.865	0.0354665079868145	0.263960494742775	0.719814276220681\\
0.865	0.0368423812671862	0.268937541962694	0.715644470612629\\
0.865	0.0382458547890421	0.273914154744766	0.711723839838042\\
0.865	0.0396769802625738	0.278890117720924	0.70803638238973\\
0.865	0.0411358072996216	0.283865214418836	0.704565625567164\\
0.865	0.0426223833924861	0.288839227285554	0.701295023640317\\
0.865	0.0441367538930258	0.293811937711588	0.698208278749354\\
0.865	0.0456789619920503	0.298783126055387	0.695289592055349\\
0.865	0.0472490486990192	0.303752571668251	0.692523853197512\\
0.865	0.0488470528220538	0.308720052919643	0.689896776228118\\
0.865	0.0504730109482718	0.313685347222913	0.687394989990281\\
0.865	0.0521269574244539	0.318648231061427	0.685006090465981\\
0.865	0.0538089243380495	0.323608480015096	0.682718662029128\\
0.865	0.0555189414985328	0.328565868787292	0.68052227385347\\
0.865	0.0572570364191156	0.333520171232163	0.67840745699785\\
0.865	0.0590232342988274	0.338471160382329	0.676365666960044\\
0.865	0.0608175580049697	0.34341860847696	0.674389235783156\\
0.865	0.0626400280559541	0.348362286990222	0.672471317134862\\
0.865	0.064490662604533	0.3533019666601	0.670605827172181\\
0.865	0.0663694774214291	0.358237417517579	0.668787383459804\\
0.865	0.0682764858793752	0.363168408916186	0.667011243730728\\
0.865	0.0702116989375697	0.368094709561873	0.665273245863649\\
0.865	0.0721751251265572	0.373016087543259	0.663569750098931\\
0.865	0.0741667705335422	0.377932310362199	0.66189758422006\\
0.865	0.0761866387881432	0.382843144964686	0.660253992184443\\
0.865	0.0782347310485954	0.38774835777208	0.658636586490997\\
0.865	0.0803110459884098	0.392647714712651	0.657043304415872\\
0.865	0.0824155797834956	0.397540981253432	0.655472368126508\\
0.865	0.084548326099754	0.402427922432375	0.653922248592744\\
0.865	0.0867092760811502	0.407308302890799	0.652391633146571\\
0.865	0.0888984183382709	0.412181886906127	0.65087939649558\\
0.865	0.0911157389373742	0.417048438424896	0.649384574964842\\
0.865	0.0933612213899392	0.421907721096044	0.647906343724685\\
0.865	0.0956348466427212	0.42675949830445	0.646443996754806\\
0.865	0.0979365930683194	0.431603533204733	0.644996929295962\\
0.865	0.100266436456264	0.436439588755293	0.643564622546847\\
0.865	0.102624350004627	0.441267427752584	0.64214663037443\\
0.865	0.105010304312169	0.446086812865616	0.640742567819433\\
0.865	0.107424267371012	0.450897506670668	0.63935210119373\\
0.865	0.109866204559871	0.455699271686218	0.637974939582444\\
0.865	0.112336078637817	0.460491870408058	0.636610827579863\\
0.865	0.114833849738607	0.465275065344603	0.635259539104382\\
0.865	0.117359475365564	0.470048619052377	0.633920872153154\\
0.865	0.119912910387022	0.474812294171664	0.632594644371965\\
0.865	0.122494107032347	0.479565853462319	0.63128068932964\\
0.865	0.125103014888515	0.484309059839721	0.629978853399059\\
0.865	0.127739580897283	0.489041676410868	0.62869046082456\\
0.87	0	0	0.710202759607373\\
0.87	1.11327767495586e-05	0.00471862581271275	0.713003157837853\\
0.87	4.46175251031912e-05	0.009446325184051	0.715846486660142\\
0.87	0.000100583311362513	0.0141829653360114	0.718788838379794\\
0.87	0.000179158434668431	0.018928411755669	0.721886092228356\\
0.87	0.000280470402701511	0.02368252819604	0.725200636220892\\
0.87	0.000404645907256436	0.0284451766772965	0.728799085274723\\
0.87	0.000551810799695644	0.0332162174883388	0.732749123983336\\
0.87	0.000722090066287311	0.037995509188729	0.737115575253327\\
0.87	0.000915607803432999	0.0427829086109896	0.741955893907011\\
0.87	0.0011324871927904	0.047578270863273	0.747315373652289\\
0.87	0.00137285047629673	0.0523814493324038	0.753222424045811\\
0.87	0.00163681893109844	0.057192295687301	0.759684310525652\\
0.87	0.00192451284439304	0.0620106598827802	0.766683748102665\\
0.87	0.00223605148818898	0.0668363901637434	0.774176695511256\\
0.87	0.00257155309398959	0.0716693330697584	0.782091614494493\\
0.87	0.00293113482740716	0.0765093334400312	0.79033034642109\\
0.87	0.00331491276271365	0.0813562344187764	0.79877062752665\\
0.87	0.00372300185733413	0.0862098774609879	0.807270128930359\\
0.87	0.00415551592628969	0.0910701023386145	0.815671782700193\\
0.87	0.00461256761659621	0.0959367471471424	0.82381005350733\\
0.87	0.00509426838162598	0.100809648312589	0.831517746503353\\
0.87	0.00560072845543873	0.105688640598912	0.838632911419451\\
0.87	0.00613205682708918	0.11057355711583	0.845005411376293\\
0.87	0.00668836121491816	0.115464229327074	0.850502769049301\\
0.87	0.00726974804083431	0.120360487059049	0.855014975717656\\
0.87	0.00787632240459385	0.125262158509929	0.858458041096606\\
0.87	0.00850818805808555	0.13016907025918	0.860776163494183\\
0.87	0.00916544737962849	0.135081047277508	0.861942500814226\\
0.87	0.00984820134829017	0.139997912937243	0.861958614648419\\
0.87	0.0105565495182326	0.144919489023162	0.86085273561709\\
0.87	0.0112905899930939	0.149845595743738	0.858677054109652\\
0.87	0.012050419400414	0.154776051742839	0.855504274967457\\
0.87	0.0128361328661109	0.159710674111862	0.851423687967755\\
0.87	0.0136478239890177	0.164649278402306	0.846537000484341\\
0.87	0.0144855848154856	0.169591678638796	0.840954157904249\\
0.87	0.0153495058140643	0.174537687332543	0.834789345403936\\
0.87	0.0162396758502651	0.179487115495259	0.828157325806199\\
0.87	0.0171561821614173	0.184439772653511	0.821170226448694\\
0.87	0.0180991103316243	0.189395466863524	0.813934846725693\\
0.87	0.0190685442668299	0.194354004726437	0.806550519904211\\
0.87	0.0200645661700016	0.199315191403999	0.799107529846306\\
0.87	0.0210872565164405	0.204278830634725	0.791686056496194\\
0.87	0.0221366940292259	0.20924472475049	0.784355603830603\\
0.87	0.023212955654804	0.214212674693577	0.777174850281118\\
0.87	0.0243161165387281	0.219182480034174	0.770191853862643\\
0.87	0.025446250001561	0.224153938988321	0.763444541554457\\
0.87	0.0266034275149466	0.229126848436298	0.756961413910482\\
0.87	0.0277877186778607	0.234101003941464	0.750762400416763\\
0.87	0.0289991911930496	0.239076199769547	0.744859807805786\\
0.87	0.0302379108436654	0.244052228908366	0.73925931151808\\
0.87	0.0315039414701067	0.24902888308801	0.733960949044503\\
0.87	0.0327973449470747	0.254005952801449	0.728960082407261\\
0.87	0.0341181811608519	0.258983227325592	0.724248305111177\\
0.87	0.0354665079868145	0.263960494742775	0.71981427622068\\
0.87	0.0368423812671862	0.268937541962694	0.715644470612631\\
0.87	0.0382458547890421	0.273914154744766	0.711723839838041\\
0.87	0.0396769802625738	0.278890117720924	0.70803638238973\\
0.87	0.0411358072996216	0.283865214418836	0.704565625567166\\
0.87	0.0426223833924861	0.288839227285554	0.701295023640317\\
0.87	0.0441367538930258	0.293811937711588	0.698208278749353\\
0.87	0.0456789619920503	0.298783126055387	0.695289592055349\\
0.87	0.0472490486990192	0.303752571668251	0.692523853197512\\
0.87	0.0488470528220538	0.308720052919643	0.689896776228119\\
0.87	0.0504730109482718	0.313685347222913	0.687394989990281\\
0.87	0.0521269574244539	0.318648231061428	0.685006090465981\\
0.87	0.0538089243380495	0.323608480015097	0.682718662029128\\
0.87	0.0555189414985328	0.328565868787292	0.680522273853469\\
0.87	0.0572570364191156	0.333520171232163	0.67840745699785\\
0.87	0.0590232342988274	0.338471160382329	0.676365666960044\\
0.87	0.0608175580049697	0.34341860847696	0.674389235783156\\
0.87	0.0626400280559541	0.348362286990222	0.672471317134862\\
0.87	0.064490662604533	0.3533019666601	0.670605827172182\\
0.87	0.0663694774214291	0.358237417517579	0.668787383459803\\
0.87	0.0682764858793752	0.363168408916186	0.667011243730728\\
0.87	0.0702116989375697	0.368094709561873	0.66527324586365\\
0.87	0.0721751251265572	0.373016087543259	0.663569750098933\\
0.87	0.0741667705335422	0.377932310362199	0.66189758422006\\
0.87	0.0761866387881432	0.382843144964686	0.660253992184443\\
0.87	0.0782347310485954	0.38774835777208	0.658636586490996\\
0.87	0.0803110459884098	0.392647714712651	0.657043304415872\\
0.87	0.0824155797834956	0.397540981253432	0.65547236812651\\
0.87	0.084548326099754	0.402427922432375	0.653922248592745\\
0.87	0.0867092760811502	0.407308302890799	0.652391633146569\\
0.87	0.0888984183382709	0.412181886906127	0.65087939649558\\
0.87	0.0911157389373742	0.417048438424896	0.649384574964843\\
0.87	0.0933612213899392	0.421907721096044	0.647906343724685\\
0.87	0.0956348466427212	0.42675949830445	0.646443996754805\\
0.87	0.0979365930683194	0.431603533204733	0.644996929295962\\
0.87	0.100266436456264	0.436439588755293	0.643564622546848\\
0.87	0.102624350004627	0.441267427752585	0.642146630374429\\
0.87	0.105010304312169	0.446086812865616	0.640742567819433\\
0.87	0.107424267371012	0.450897506670668	0.639352101193731\\
0.87	0.109866204559871	0.455699271686218	0.637974939582442\\
0.87	0.112336078637817	0.460491870408058	0.636610827579862\\
0.87	0.114833849738607	0.465275065344603	0.635259539104384\\
0.87	0.117359475365564	0.470048619052377	0.633920872153154\\
0.87	0.119912910387022	0.474812294171664	0.632594644371965\\
0.87	0.122494107032347	0.479565853462319	0.631280689329641\\
0.87	0.125103014888515	0.484309059839721	0.62997885339906\\
0.87	0.127739580897283	0.489041676410868	0.628690460824556\\
0.875	0	0	0.710202759607373\\
0.875	1.11327767495586e-05	0.00471862581271274	0.713003157837853\\
0.875	4.46175251031912e-05	0.009446325184051	0.715846486660142\\
0.875	0.000100583311362513	0.0141829653360114	0.718788838379794\\
0.875	0.000179158434668431	0.018928411755669	0.721886092228356\\
0.875	0.000280470402701511	0.02368252819604	0.725200636220892\\
0.875	0.000404645907256436	0.0284451766772965	0.728799085274723\\
0.875	0.000551810799695644	0.0332162174883388	0.732749123983336\\
0.875	0.000722090066287311	0.037995509188729	0.737115575253327\\
0.875	0.000915607803432999	0.0427829086109896	0.741955893907011\\
0.875	0.0011324871927904	0.0475782708632729	0.747315373652289\\
0.875	0.00137285047629673	0.0523814493324038	0.75322242404581\\
0.875	0.00163681893109844	0.057192295687301	0.759684310525652\\
0.875	0.00192451284439304	0.0620106598827802	0.766683748102665\\
0.875	0.00223605148818899	0.0668363901637434	0.774176695511256\\
0.875	0.00257155309398959	0.0716693330697584	0.782091614494493\\
0.875	0.00293113482740716	0.0765093334400312	0.79033034642109\\
0.875	0.00331491276271365	0.0813562344187764	0.79877062752665\\
0.875	0.00372300185733413	0.0862098774609879	0.807270128930359\\
0.875	0.00415551592628969	0.0910701023386145	0.815671782700193\\
0.875	0.00461256761659621	0.0959367471471425	0.82381005350733\\
0.875	0.00509426838162598	0.100809648312589	0.831517746503354\\
0.875	0.00560072845543873	0.105688640598912	0.838632911419452\\
0.875	0.00613205682708918	0.11057355711583	0.845005411376293\\
0.875	0.00668836121491816	0.115464229327074	0.850502769049302\\
0.875	0.00726974804083431	0.120360487059049	0.855014975717657\\
0.875	0.00787632240459385	0.125262158509929	0.858458041096605\\
0.875	0.00850818805808554	0.13016907025918	0.860776163494182\\
0.875	0.00916544737962849	0.135081047277508	0.861942500814223\\
0.875	0.00984820134829017	0.139997912937243	0.861958614648422\\
0.875	0.0105565495182326	0.144919489023162	0.860852735617092\\
0.875	0.0112905899930939	0.149845595743738	0.858677054109651\\
0.875	0.012050419400414	0.154776051742839	0.855504274967456\\
0.875	0.0128361328661109	0.159710674111862	0.851423687967756\\
0.875	0.0136478239890177	0.164649278402306	0.846537000484339\\
0.875	0.0144855848154856	0.169591678638796	0.840954157904248\\
0.875	0.0153495058140643	0.174537687332543	0.834789345403933\\
0.875	0.0162396758502651	0.179487115495259	0.828157325806199\\
0.875	0.0171561821614173	0.184439772653511	0.821170226448693\\
0.875	0.0180991103316243	0.189395466863524	0.813934846725691\\
0.875	0.0190685442668299	0.194354004726437	0.806550519904212\\
0.875	0.0200645661700016	0.199315191403999	0.799107529846304\\
0.875	0.0210872565164405	0.204278830634725	0.791686056496193\\
0.875	0.0221366940292259	0.20924472475049	0.784355603830601\\
0.875	0.023212955654804	0.214212674693577	0.777174850281117\\
0.875	0.0243161165387281	0.219182480034174	0.770191853862643\\
0.875	0.025446250001561	0.224153938988321	0.763444541554459\\
0.875	0.0266034275149466	0.229126848436298	0.756961413910482\\
0.875	0.0277877186778607	0.234101003941464	0.750762400416763\\
0.875	0.0289991911930496	0.239076199769547	0.744859807805786\\
0.875	0.0302379108436654	0.244052228908366	0.739259311518078\\
0.875	0.0315039414701067	0.24902888308801	0.733960949044504\\
0.875	0.0327973449470747	0.254005952801449	0.728960082407261\\
0.875	0.0341181811608519	0.258983227325592	0.724248305111177\\
0.875	0.0354665079868145	0.263960494742775	0.719814276220681\\
0.875	0.0368423812671862	0.268937541962694	0.715644470612629\\
0.875	0.0382458547890421	0.273914154744766	0.711723839838041\\
0.875	0.0396769802625737	0.278890117720924	0.708036382389731\\
0.875	0.0411358072996216	0.283865214418836	0.704565625567164\\
0.875	0.0426223833924861	0.288839227285554	0.701295023640315\\
0.875	0.0441367538930258	0.293811937711588	0.698208278749353\\
0.875	0.0456789619920503	0.298783126055387	0.69528959205535\\
0.875	0.0472490486990192	0.303752571668251	0.692523853197512\\
0.875	0.0488470528220538	0.308720052919643	0.689896776228119\\
0.875	0.0504730109482718	0.313685347222913	0.687394989990282\\
0.875	0.0521269574244539	0.318648231061427	0.68500609046598\\
0.875	0.0538089243380495	0.323608480015097	0.682718662029128\\
0.875	0.0555189414985328	0.328565868787292	0.680522273853469\\
0.875	0.0572570364191156	0.333520171232163	0.67840745699785\\
0.875	0.0590232342988274	0.338471160382329	0.676365666960043\\
0.875	0.0608175580049697	0.34341860847696	0.674389235783157\\
0.875	0.0626400280559541	0.348362286990222	0.672471317134863\\
0.875	0.064490662604533	0.3533019666601	0.670605827172182\\
0.875	0.0663694774214291	0.358237417517579	0.668787383459804\\
0.875	0.0682764858793752	0.363168408916186	0.667011243730727\\
0.875	0.0702116989375697	0.368094709561873	0.665273245863649\\
0.875	0.0721751251265572	0.373016087543259	0.663569750098933\\
0.875	0.0741667705335422	0.377932310362199	0.66189758422006\\
0.875	0.0761866387881432	0.382843144964686	0.660253992184443\\
0.875	0.0782347310485954	0.38774835777208	0.658636586490997\\
0.875	0.0803110459884098	0.392647714712651	0.657043304415871\\
0.875	0.0824155797834956	0.397540981253432	0.655472368126509\\
0.875	0.084548326099754	0.402427922432375	0.653922248592747\\
0.875	0.0867092760811502	0.4073083028908	0.652391633146571\\
0.875	0.0888984183382709	0.412181886906127	0.650879396495579\\
0.875	0.0911157389373742	0.417048438424897	0.649384574964842\\
0.875	0.0933612213899392	0.421907721096044	0.647906343724684\\
0.875	0.0956348466427212	0.42675949830445	0.646443996754806\\
0.875	0.0979365930683194	0.431603533204733	0.644996929295961\\
0.875	0.100266436456264	0.436439588755293	0.643564622546847\\
0.875	0.102624350004627	0.441267427752584	0.64214663037443\\
0.875	0.105010304312169	0.446086812865616	0.640742567819433\\
0.875	0.107424267371012	0.450897506670668	0.63935210119373\\
0.875	0.109866204559871	0.455699271686218	0.637974939582442\\
0.875	0.112336078637817	0.460491870408058	0.636610827579861\\
0.875	0.114833849738607	0.465275065344603	0.635259539104384\\
0.875	0.117359475365564	0.470048619052377	0.633920872153154\\
0.875	0.119912910387022	0.474812294171664	0.632594644371966\\
0.875	0.122494107032347	0.479565853462319	0.63128068932964\\
0.875	0.125103014888515	0.484309059839721	0.629978853399059\\
0.875	0.127739580897283	0.489041676410868	0.628690460824564\\
0.88	0	0	0.710202759607373\\
0.88	1.11327767495586e-05	0.00471862581271274	0.713003157837853\\
0.88	4.46175251031912e-05	0.009446325184051	0.715846486660142\\
0.88	0.000100583311362513	0.0141829653360114	0.718788838379794\\
0.88	0.000179158434668431	0.018928411755669	0.721886092228356\\
0.88	0.000280470402701511	0.02368252819604	0.725200636220892\\
0.88	0.000404645907256436	0.0284451766772965	0.728799085274723\\
0.88	0.000551810799695644	0.0332162174883388	0.732749123983336\\
0.88	0.000722090066287311	0.037995509188729	0.737115575253327\\
0.88	0.000915607803432999	0.0427829086109896	0.741955893907011\\
0.88	0.0011324871927904	0.047578270863273	0.747315373652289\\
0.88	0.00137285047629673	0.0523814493324038	0.753222424045811\\
0.88	0.00163681893109844	0.057192295687301	0.759684310525652\\
0.88	0.00192451284439304	0.0620106598827802	0.766683748102665\\
0.88	0.00223605148818898	0.0668363901637434	0.774176695511256\\
0.88	0.00257155309398959	0.0716693330697584	0.782091614494494\\
0.88	0.00293113482740716	0.0765093334400312	0.79033034642109\\
0.88	0.00331491276271365	0.0813562344187764	0.79877062752665\\
0.88	0.00372300185733413	0.0862098774609879	0.807270128930359\\
0.88	0.00415551592628969	0.0910701023386145	0.815671782700194\\
0.88	0.00461256761659621	0.0959367471471425	0.82381005350733\\
0.88	0.00509426838162598	0.100809648312589	0.831517746503354\\
0.88	0.00560072845543873	0.105688640598912	0.838632911419452\\
0.88	0.00613205682708918	0.11057355711583	0.845005411376293\\
0.88	0.00668836121491816	0.115464229327074	0.850502769049302\\
0.88	0.00726974804083431	0.120360487059049	0.855014975717657\\
0.88	0.00787632240459385	0.125262158509929	0.858458041096607\\
0.88	0.00850818805808555	0.13016907025918	0.860776163494181\\
0.88	0.00916544737962849	0.135081047277508	0.861942500814222\\
0.88	0.00984820134829017	0.139997912937243	0.86195861464842\\
0.88	0.0105565495182326	0.144919489023162	0.86085273561709\\
0.88	0.0112905899930939	0.149845595743738	0.85867705410965\\
0.88	0.012050419400414	0.154776051742839	0.855504274967459\\
0.88	0.0128361328661109	0.159710674111862	0.851423687967754\\
0.88	0.0136478239890177	0.164649278402306	0.846537000484339\\
0.88	0.0144855848154856	0.169591678638796	0.840954157904249\\
0.88	0.0153495058140643	0.174537687332543	0.834789345403935\\
0.88	0.0162396758502651	0.179487115495259	0.8281573258062\\
0.88	0.0171561821614173	0.184439772653511	0.821170226448693\\
0.88	0.0180991103316243	0.189395466863524	0.813934846725693\\
0.88	0.0190685442668299	0.194354004726437	0.806550519904212\\
0.88	0.0200645661700016	0.199315191403999	0.799107529846304\\
0.88	0.0210872565164405	0.204278830634725	0.791686056496195\\
0.88	0.0221366940292259	0.20924472475049	0.784355603830601\\
0.88	0.023212955654804	0.214212674693577	0.777174850281118\\
0.88	0.0243161165387281	0.219182480034174	0.770191853862643\\
0.88	0.025446250001561	0.224153938988321	0.763444541554459\\
0.88	0.0266034275149466	0.229126848436298	0.756961413910482\\
0.88	0.0277877186778607	0.234101003941464	0.750762400416766\\
0.88	0.0289991911930496	0.239076199769547	0.744859807805785\\
0.88	0.0302379108436654	0.244052228908366	0.739259311518079\\
0.88	0.0315039414701067	0.24902888308801	0.733960949044503\\
0.88	0.0327973449470747	0.254005952801449	0.728960082407259\\
0.88	0.0341181811608519	0.258983227325591	0.724248305111179\\
0.88	0.0354665079868146	0.263960494742775	0.719814276220682\\
0.88	0.0368423812671862	0.268937541962694	0.71564447061263\\
0.88	0.0382458547890421	0.273914154744766	0.711723839838043\\
0.88	0.0396769802625738	0.278890117720924	0.70803638238973\\
0.88	0.0411358072996216	0.283865214418836	0.704565625567163\\
0.88	0.0426223833924861	0.288839227285554	0.701295023640316\\
0.88	0.0441367538930258	0.293811937711588	0.698208278749353\\
0.88	0.0456789619920503	0.298783126055387	0.69528959205535\\
0.88	0.0472490486990192	0.303752571668251	0.692523853197512\\
0.88	0.0488470528220538	0.308720052919643	0.689896776228118\\
0.88	0.0504730109482718	0.313685347222913	0.687394989990283\\
0.88	0.0521269574244539	0.318648231061428	0.685006090465981\\
0.88	0.0538089243380495	0.323608480015096	0.682718662029127\\
0.88	0.0555189414985328	0.328565868787292	0.68052227385347\\
0.88	0.0572570364191156	0.333520171232163	0.678407456997851\\
0.88	0.0590232342988274	0.338471160382329	0.676365666960043\\
0.88	0.0608175580049697	0.34341860847696	0.674389235783156\\
0.88	0.0626400280559541	0.348362286990222	0.672471317134862\\
0.88	0.064490662604533	0.3533019666601	0.670605827172182\\
0.88	0.0663694774214291	0.358237417517579	0.668787383459805\\
0.88	0.0682764858793752	0.363168408916186	0.667011243730728\\
0.88	0.0702116989375697	0.368094709561873	0.665273245863649\\
0.88	0.0721751251265572	0.373016087543259	0.663569750098933\\
0.88	0.0741667705335422	0.377932310362199	0.66189758422006\\
0.88	0.0761866387881432	0.382843144964686	0.660253992184441\\
0.88	0.0782347310485954	0.38774835777208	0.658636586490997\\
0.88	0.0803110459884098	0.392647714712651	0.65704330441587\\
0.88	0.0824155797834956	0.397540981253432	0.655472368126509\\
0.88	0.084548326099754	0.402427922432375	0.653922248592746\\
0.88	0.0867092760811502	0.4073083028908	0.652391633146572\\
0.88	0.0888984183382709	0.412181886906127	0.650879396495578\\
0.88	0.0911157389373742	0.417048438424896	0.649384574964843\\
0.88	0.0933612213899392	0.421907721096044	0.647906343724684\\
0.88	0.0956348466427212	0.42675949830445	0.646443996754806\\
0.88	0.0979365930683194	0.431603533204733	0.644996929295962\\
0.88	0.100266436456264	0.436439588755293	0.643564622546848\\
0.88	0.102624350004627	0.441267427752585	0.642146630374429\\
0.88	0.105010304312169	0.446086812865616	0.640742567819434\\
0.88	0.107424267371012	0.450897506670668	0.639352101193732\\
0.88	0.109866204559871	0.455699271686218	0.637974939582442\\
0.88	0.112336078637817	0.460491870408058	0.636610827579861\\
0.88	0.114833849738607	0.465275065344603	0.635259539104382\\
0.88	0.117359475365564	0.470048619052377	0.633920872153154\\
0.88	0.119912910387022	0.474812294171664	0.632594644371966\\
0.88	0.122494107032347	0.479565853462319	0.631280689329642\\
0.88	0.125103014888515	0.484309059839721	0.629978853399057\\
0.88	0.127739580897283	0.489041676410868	0.628690460824562\\
0.885	0	0	0.710202759607373\\
0.885	1.11327767495586e-05	0.00471862581271275	0.713003157837853\\
0.885	4.46175251031912e-05	0.009446325184051	0.715846486660142\\
0.885	0.000100583311362513	0.0141829653360114	0.718788838379794\\
0.885	0.000179158434668431	0.018928411755669	0.721886092228356\\
0.885	0.000280470402701511	0.02368252819604	0.725200636220892\\
0.885	0.000404645907256436	0.0284451766772965	0.728799085274723\\
0.885	0.000551810799695644	0.0332162174883388	0.732749123983336\\
0.885	0.000722090066287311	0.037995509188729	0.737115575253327\\
0.885	0.000915607803432999	0.0427829086109896	0.741955893907011\\
0.885	0.0011324871927904	0.0475782708632729	0.747315373652289\\
0.885	0.00137285047629673	0.0523814493324038	0.753222424045811\\
0.885	0.00163681893109844	0.057192295687301	0.759684310525652\\
0.885	0.00192451284439304	0.0620106598827802	0.766683748102665\\
0.885	0.00223605148818898	0.0668363901637434	0.774176695511257\\
0.885	0.00257155309398959	0.0716693330697584	0.782091614494493\\
0.885	0.00293113482740716	0.0765093334400312	0.79033034642109\\
0.885	0.00331491276271365	0.0813562344187764	0.79877062752665\\
0.885	0.00372300185733413	0.0862098774609879	0.80727012893036\\
0.885	0.00415551592628969	0.0910701023386145	0.815671782700194\\
0.885	0.00461256761659621	0.0959367471471425	0.82381005350733\\
0.885	0.00509426838162598	0.100809648312589	0.831517746503354\\
0.885	0.00560072845543873	0.105688640598912	0.838632911419452\\
0.885	0.00613205682708918	0.11057355711583	0.845005411376293\\
0.885	0.00668836121491816	0.115464229327074	0.850502769049302\\
0.885	0.00726974804083431	0.120360487059049	0.855014975717657\\
0.885	0.00787632240459385	0.125262158509929	0.858458041096607\\
0.885	0.00850818805808555	0.13016907025918	0.860776163494182\\
0.885	0.00916544737962849	0.135081047277508	0.861942500814225\\
0.885	0.00984820134829017	0.139997912937243	0.861958614648419\\
0.885	0.0105565495182326	0.144919489023162	0.86085273561709\\
0.885	0.0112905899930939	0.149845595743738	0.858677054109652\\
0.885	0.012050419400414	0.154776051742839	0.855504274967457\\
0.885	0.0128361328661109	0.159710674111862	0.851423687967753\\
0.885	0.0136478239890177	0.164649278402306	0.846537000484341\\
0.885	0.0144855848154856	0.169591678638796	0.840954157904248\\
0.885	0.0153495058140643	0.174537687332543	0.834789345403936\\
0.885	0.0162396758502651	0.179487115495259	0.8281573258062\\
0.885	0.0171561821614173	0.184439772653511	0.821170226448695\\
0.885	0.0180991103316243	0.189395466863524	0.813934846725692\\
0.885	0.0190685442668299	0.194354004726437	0.806550519904213\\
0.885	0.0200645661700016	0.199315191403999	0.799107529846306\\
0.885	0.0210872565164405	0.204278830634725	0.791686056496193\\
0.885	0.0221366940292259	0.20924472475049	0.7843556038306\\
0.885	0.023212955654804	0.214212674693577	0.777174850281118\\
0.885	0.0243161165387281	0.219182480034174	0.770191853862644\\
0.885	0.025446250001561	0.224153938988321	0.763444541554457\\
0.885	0.0266034275149466	0.229126848436298	0.756961413910483\\
0.885	0.0277877186778607	0.234101003941464	0.750762400416765\\
0.885	0.0289991911930496	0.239076199769547	0.744859807805785\\
0.885	0.0302379108436654	0.244052228908366	0.739259311518079\\
0.885	0.0315039414701067	0.24902888308801	0.733960949044504\\
0.885	0.0327973449470747	0.254005952801449	0.728960082407262\\
0.885	0.0341181811608519	0.258983227325592	0.724248305111179\\
0.885	0.0354665079868145	0.263960494742775	0.71981427622068\\
0.885	0.0368423812671862	0.268937541962694	0.715644470612629\\
0.885	0.0382458547890421	0.273914154744766	0.711723839838041\\
0.885	0.0396769802625737	0.278890117720924	0.70803638238973\\
0.885	0.0411358072996216	0.283865214418836	0.704565625567164\\
0.885	0.0426223833924861	0.288839227285554	0.701295023640316\\
0.885	0.0441367538930258	0.293811937711588	0.698208278749354\\
0.885	0.0456789619920503	0.298783126055387	0.695289592055351\\
0.885	0.0472490486990192	0.303752571668251	0.692523853197511\\
0.885	0.0488470528220538	0.308720052919643	0.689896776228118\\
0.885	0.0504730109482718	0.313685347222913	0.687394989990282\\
0.885	0.0521269574244539	0.318648231061428	0.685006090465981\\
0.885	0.0538089243380495	0.323608480015096	0.682718662029128\\
0.885	0.0555189414985328	0.328565868787292	0.680522273853469\\
0.885	0.0572570364191156	0.333520171232163	0.67840745699785\\
0.885	0.0590232342988274	0.338471160382329	0.676365666960044\\
0.885	0.0608175580049697	0.34341860847696	0.674389235783156\\
0.885	0.0626400280559541	0.348362286990222	0.672471317134863\\
0.885	0.064490662604533	0.3533019666601	0.670605827172182\\
0.885	0.0663694774214291	0.358237417517579	0.668787383459803\\
0.885	0.0682764858793752	0.363168408916186	0.667011243730728\\
0.885	0.0702116989375697	0.368094709561873	0.665273245863648\\
0.885	0.0721751251265572	0.373016087543259	0.663569750098932\\
0.885	0.0741667705335422	0.377932310362199	0.661897584220061\\
0.885	0.0761866387881432	0.382843144964686	0.660253992184442\\
0.885	0.0782347310485954	0.38774835777208	0.658636586490997\\
0.885	0.0803110459884098	0.392647714712651	0.657043304415871\\
0.885	0.0824155797834956	0.397540981253432	0.655472368126508\\
0.885	0.084548326099754	0.402427922432375	0.653922248592745\\
0.885	0.0867092760811502	0.407308302890799	0.652391633146572\\
0.885	0.0888984183382709	0.412181886906127	0.65087939649558\\
0.885	0.0911157389373742	0.417048438424896	0.649384574964842\\
0.885	0.0933612213899392	0.421907721096044	0.647906343724685\\
0.885	0.0956348466427212	0.42675949830445	0.646443996754806\\
0.885	0.0979365930683194	0.431603533204733	0.644996929295961\\
0.885	0.100266436456264	0.436439588755293	0.643564622546847\\
0.885	0.102624350004627	0.441267427752584	0.642146630374431\\
0.885	0.105010304312169	0.446086812865616	0.640742567819433\\
0.885	0.107424267371012	0.450897506670668	0.639352101193732\\
0.885	0.109866204559871	0.455699271686218	0.637974939582444\\
0.885	0.112336078637817	0.460491870408058	0.636610827579862\\
0.885	0.114833849738607	0.465275065344603	0.635259539104383\\
0.885	0.117359475365564	0.470048619052377	0.633920872153154\\
0.885	0.119912910387022	0.474812294171664	0.632594644371964\\
0.885	0.122494107032347	0.479565853462319	0.631280689329642\\
0.885	0.125103014888515	0.484309059839721	0.629978853399057\\
0.885	0.127739580897283	0.489041676410868	0.628690460824555\\
0.89	0	0	0.710202759607373\\
0.89	1.11327767495586e-05	0.00471862581271274	0.713003157837853\\
0.89	4.46175251031912e-05	0.009446325184051	0.715846486660142\\
0.89	0.000100583311362513	0.0141829653360114	0.718788838379794\\
0.89	0.000179158434668431	0.018928411755669	0.721886092228356\\
0.89	0.000280470402701511	0.02368252819604	0.725200636220892\\
0.89	0.000404645907256436	0.0284451766772965	0.728799085274723\\
0.89	0.000551810799695644	0.0332162174883389	0.732749123983336\\
0.89	0.000722090066287312	0.037995509188729	0.737115575253327\\
0.89	0.000915607803432999	0.0427829086109896	0.741955893907011\\
0.89	0.0011324871927904	0.0475782708632729	0.747315373652289\\
0.89	0.00137285047629673	0.0523814493324038	0.753222424045811\\
0.89	0.00163681893109844	0.057192295687301	0.759684310525652\\
0.89	0.00192451284439304	0.0620106598827802	0.766683748102665\\
0.89	0.00223605148818899	0.0668363901637434	0.774176695511256\\
0.89	0.00257155309398959	0.0716693330697584	0.782091614494494\\
0.89	0.00293113482740716	0.0765093334400312	0.79033034642109\\
0.89	0.00331491276271365	0.0813562344187764	0.79877062752665\\
0.89	0.00372300185733414	0.0862098774609879	0.807270128930359\\
0.89	0.00415551592628969	0.0910701023386145	0.815671782700193\\
0.89	0.00461256761659621	0.0959367471471424	0.82381005350733\\
0.89	0.00509426838162598	0.100809648312589	0.831517746503354\\
0.89	0.00560072845543873	0.105688640598912	0.838632911419452\\
0.89	0.00613205682708918	0.11057355711583	0.845005411376293\\
0.89	0.00668836121491816	0.115464229327074	0.850502769049301\\
0.89	0.00726974804083431	0.120360487059049	0.855014975717657\\
0.89	0.00787632240459385	0.125262158509929	0.858458041096606\\
0.89	0.00850818805808555	0.13016907025918	0.860776163494182\\
0.89	0.00916544737962849	0.135081047277508	0.861942500814225\\
0.89	0.00984820134829017	0.139997912937243	0.86195861464842\\
0.89	0.0105565495182326	0.144919489023162	0.860852735617092\\
0.89	0.0112905899930939	0.149845595743738	0.858677054109651\\
0.89	0.012050419400414	0.154776051742839	0.855504274967457\\
0.89	0.0128361328661109	0.159710674111862	0.851423687967754\\
0.89	0.0136478239890177	0.164649278402306	0.846537000484341\\
0.89	0.0144855848154856	0.169591678638796	0.840954157904248\\
0.89	0.0153495058140643	0.174537687332543	0.834789345403935\\
0.89	0.0162396758502651	0.179487115495259	0.8281573258062\\
0.89	0.0171561821614173	0.184439772653511	0.821170226448695\\
0.89	0.0180991103316243	0.189395466863524	0.813934846725693\\
0.89	0.0190685442668299	0.194354004726437	0.806550519904214\\
0.89	0.0200645661700016	0.199315191403999	0.799107529846304\\
0.89	0.0210872565164405	0.204278830634725	0.791686056496192\\
0.89	0.0221366940292259	0.20924472475049	0.784355603830601\\
0.89	0.023212955654804	0.214212674693577	0.77717485028112\\
0.89	0.0243161165387281	0.219182480034174	0.770191853862644\\
0.89	0.025446250001561	0.224153938988321	0.76344454155446\\
0.89	0.0266034275149466	0.229126848436298	0.756961413910484\\
0.89	0.0277877186778607	0.234101003941464	0.750762400416765\\
0.89	0.0289991911930496	0.239076199769547	0.744859807805784\\
0.89	0.0302379108436654	0.244052228908366	0.739259311518079\\
0.89	0.0315039414701067	0.24902888308801	0.733960949044503\\
0.89	0.0327973449470747	0.254005952801449	0.728960082407261\\
0.89	0.0341181811608519	0.258983227325592	0.724248305111178\\
0.89	0.0354665079868145	0.263960494742775	0.719814276220682\\
0.89	0.0368423812671862	0.268937541962694	0.715644470612629\\
0.89	0.0382458547890421	0.273914154744766	0.711723839838041\\
0.89	0.0396769802625738	0.278890117720924	0.70803638238973\\
0.89	0.0411358072996216	0.283865214418836	0.704565625567163\\
0.89	0.0426223833924861	0.288839227285554	0.701295023640316\\
0.89	0.0441367538930258	0.293811937711588	0.698208278749354\\
0.89	0.0456789619920503	0.298783126055387	0.695289592055349\\
0.89	0.0472490486990192	0.303752571668251	0.692523853197511\\
0.89	0.0488470528220538	0.308720052919643	0.689896776228119\\
0.89	0.0504730109482718	0.313685347222913	0.687394989990281\\
0.89	0.0521269574244539	0.318648231061427	0.685006090465981\\
0.89	0.0538089243380495	0.323608480015096	0.682718662029128\\
0.89	0.0555189414985328	0.328565868787292	0.680522273853469\\
0.89	0.0572570364191156	0.333520171232163	0.678407456997851\\
0.89	0.0590232342988274	0.338471160382329	0.676365666960044\\
0.89	0.0608175580049697	0.34341860847696	0.674389235783155\\
0.89	0.0626400280559541	0.348362286990222	0.672471317134862\\
0.89	0.064490662604533	0.3533019666601	0.670605827172182\\
0.89	0.0663694774214291	0.358237417517579	0.668787383459804\\
0.89	0.0682764858793752	0.363168408916186	0.667011243730728\\
0.89	0.0702116989375697	0.368094709561873	0.665273245863649\\
0.89	0.0721751251265572	0.373016087543259	0.663569750098931\\
0.89	0.0741667705335422	0.377932310362199	0.661897584220061\\
0.89	0.0761866387881432	0.382843144964686	0.660253992184444\\
0.89	0.0782347310485954	0.38774835777208	0.658636586490996\\
0.89	0.0803110459884098	0.392647714712651	0.657043304415871\\
0.89	0.0824155797834956	0.397540981253432	0.655472368126509\\
0.89	0.084548326099754	0.402427922432375	0.653922248592745\\
0.89	0.0867092760811502	0.407308302890799	0.65239163314657\\
0.89	0.0888984183382709	0.412181886906127	0.650879396495581\\
0.89	0.0911157389373742	0.417048438424897	0.649384574964843\\
0.89	0.0933612213899392	0.421907721096044	0.647906343724683\\
0.89	0.0956348466427212	0.42675949830445	0.646443996754806\\
0.89	0.0979365930683194	0.431603533204733	0.644996929295961\\
0.89	0.100266436456264	0.436439588755293	0.643564622546845\\
0.89	0.102624350004627	0.441267427752584	0.64214663037443\\
0.89	0.105010304312169	0.446086812865616	0.640742567819433\\
0.89	0.107424267371012	0.450897506670668	0.63935210119373\\
0.89	0.109866204559871	0.455699271686218	0.637974939582443\\
0.89	0.112336078637817	0.460491870408058	0.636610827579862\\
0.89	0.114833849738607	0.465275065344603	0.635259539104381\\
0.89	0.117359475365564	0.470048619052377	0.633920872153155\\
0.89	0.119912910387022	0.474812294171664	0.632594644371966\\
0.89	0.122494107032347	0.479565853462319	0.631280689329641\\
0.89	0.125103014888515	0.484309059839721	0.629978853399058\\
0.89	0.127739580897283	0.489041676410868	0.628690460824559\\
0.895	0	0	0.710202759607373\\
0.895	1.11327767495586e-05	0.00471862581271275	0.713003157837853\\
0.895	4.46175251031912e-05	0.009446325184051	0.715846486660142\\
0.895	0.000100583311362513	0.0141829653360114	0.718788838379794\\
0.895	0.000179158434668431	0.018928411755669	0.721886092228356\\
0.895	0.000280470402701511	0.02368252819604	0.725200636220892\\
0.895	0.000404645907256436	0.0284451766772965	0.728799085274723\\
0.895	0.000551810799695644	0.0332162174883389	0.732749123983336\\
0.895	0.000722090066287311	0.037995509188729	0.737115575253328\\
0.895	0.000915607803432999	0.0427829086109896	0.741955893907011\\
0.895	0.0011324871927904	0.0475782708632729	0.747315373652289\\
0.895	0.00137285047629673	0.0523814493324038	0.75322242404581\\
0.895	0.00163681893109843	0.057192295687301	0.759684310525652\\
0.895	0.00192451284439304	0.0620106598827802	0.766683748102665\\
0.895	0.00223605148818899	0.0668363901637434	0.774176695511256\\
0.895	0.00257155309398959	0.0716693330697584	0.782091614494493\\
0.895	0.00293113482740716	0.0765093334400313	0.79033034642109\\
0.895	0.00331491276271365	0.0813562344187764	0.79877062752665\\
0.895	0.00372300185733414	0.0862098774609879	0.807270128930359\\
0.895	0.00415551592628969	0.0910701023386145	0.815671782700193\\
0.895	0.00461256761659621	0.0959367471471425	0.823810053507331\\
0.895	0.00509426838162598	0.100809648312589	0.831517746503354\\
0.895	0.00560072845543873	0.105688640598912	0.838632911419452\\
0.895	0.00613205682708918	0.11057355711583	0.845005411376292\\
0.895	0.00668836121491815	0.115464229327074	0.850502769049303\\
0.895	0.00726974804083431	0.120360487059049	0.855014975717656\\
0.895	0.00787632240459385	0.125262158509929	0.858458041096606\\
0.895	0.00850818805808555	0.13016907025918	0.860776163494182\\
0.895	0.00916544737962849	0.135081047277508	0.861942500814226\\
0.895	0.00984820134829017	0.139997912937243	0.861958614648421\\
0.895	0.0105565495182326	0.144919489023162	0.860852735617092\\
0.895	0.0112905899930939	0.149845595743738	0.85867705410965\\
0.895	0.012050419400414	0.154776051742839	0.855504274967456\\
0.895	0.0128361328661109	0.159710674111862	0.851423687967756\\
0.895	0.0136478239890177	0.164649278402306	0.84653700048434\\
0.895	0.0144855848154856	0.169591678638796	0.840954157904249\\
0.895	0.0153495058140643	0.174537687332543	0.834789345403936\\
0.895	0.0162396758502651	0.179487115495259	0.828157325806199\\
0.895	0.0171561821614173	0.184439772653511	0.821170226448695\\
0.895	0.0180991103316243	0.189395466863524	0.813934846725691\\
0.895	0.0190685442668299	0.194354004726437	0.806550519904211\\
0.895	0.0200645661700016	0.199315191403999	0.799107529846304\\
0.895	0.0210872565164405	0.204278830634725	0.791686056496193\\
0.895	0.0221366940292259	0.20924472475049	0.784355603830603\\
0.895	0.023212955654804	0.214212674693577	0.77717485028112\\
0.895	0.0243161165387281	0.219182480034174	0.770191853862644\\
0.895	0.025446250001561	0.224153938988321	0.763444541554459\\
0.895	0.0266034275149466	0.229126848436298	0.756961413910482\\
0.895	0.0277877186778607	0.234101003941464	0.750762400416766\\
0.895	0.0289991911930496	0.239076199769547	0.744859807805787\\
0.895	0.0302379108436654	0.244052228908366	0.73925931151808\\
0.895	0.0315039414701067	0.24902888308801	0.733960949044501\\
0.895	0.0327973449470747	0.254005952801449	0.728960082407259\\
0.895	0.0341181811608519	0.258983227325591	0.724248305111179\\
0.895	0.0354665079868145	0.263960494742775	0.719814276220681\\
0.895	0.0368423812671862	0.268937541962694	0.715644470612629\\
0.895	0.0382458547890421	0.273914154744766	0.711723839838042\\
0.895	0.0396769802625738	0.278890117720924	0.70803638238973\\
0.895	0.0411358072996216	0.283865214418836	0.704565625567164\\
0.895	0.0426223833924861	0.288839227285554	0.701295023640316\\
0.895	0.0441367538930258	0.293811937711588	0.698208278749354\\
0.895	0.0456789619920503	0.298783126055387	0.695289592055347\\
0.895	0.0472490486990192	0.303752571668251	0.692523853197512\\
0.895	0.0488470528220538	0.308720052919643	0.68989677622812\\
0.895	0.0504730109482718	0.313685347222913	0.687394989990281\\
0.895	0.0521269574244539	0.318648231061428	0.685006090465981\\
0.895	0.0538089243380495	0.323608480015096	0.682718662029128\\
0.895	0.0555189414985328	0.328565868787292	0.68052227385347\\
0.895	0.0572570364191156	0.333520171232163	0.67840745699785\\
0.895	0.0590232342988274	0.338471160382329	0.676365666960044\\
0.895	0.0608175580049697	0.34341860847696	0.674389235783156\\
0.895	0.0626400280559541	0.348362286990222	0.672471317134862\\
0.895	0.064490662604533	0.3533019666601	0.670605827172182\\
0.895	0.066369477421429	0.358237417517579	0.668787383459804\\
0.895	0.0682764858793752	0.363168408916186	0.667011243730729\\
0.895	0.0702116989375697	0.368094709561873	0.665273245863649\\
0.895	0.0721751251265572	0.373016087543259	0.663569750098932\\
0.895	0.0741667705335422	0.377932310362199	0.661897584220059\\
0.895	0.0761866387881432	0.382843144964686	0.660253992184444\\
0.895	0.0782347310485954	0.38774835777208	0.658636586490998\\
0.895	0.0803110459884098	0.392647714712651	0.65704330441587\\
0.895	0.0824155797834956	0.397540981253432	0.655472368126509\\
0.895	0.084548326099754	0.402427922432375	0.653922248592745\\
0.895	0.0867092760811502	0.407308302890799	0.652391633146569\\
0.895	0.0888984183382709	0.412181886906127	0.65087939649558\\
0.895	0.0911157389373742	0.417048438424897	0.649384574964845\\
0.895	0.0933612213899392	0.421907721096044	0.647906343724684\\
0.895	0.0956348466427212	0.42675949830445	0.646443996754806\\
0.895	0.0979365930683194	0.431603533204733	0.644996929295962\\
0.895	0.100266436456264	0.436439588755293	0.643564622546847\\
0.895	0.102624350004627	0.441267427752585	0.642146630374429\\
0.895	0.105010304312169	0.446086812865616	0.640742567819433\\
0.895	0.107424267371012	0.450897506670668	0.639352101193731\\
0.895	0.109866204559871	0.455699271686218	0.637974939582443\\
0.895	0.112336078637817	0.460491870408058	0.636610827579862\\
0.895	0.114833849738607	0.465275065344603	0.635259539104382\\
0.895	0.117359475365564	0.470048619052377	0.633920872153153\\
0.895	0.119912910387022	0.474812294171664	0.632594644371965\\
0.895	0.122494107032347	0.479565853462319	0.631280689329641\\
0.895	0.125103014888515	0.484309059839721	0.629978853399058\\
0.895	0.127739580897283	0.489041676410868	0.628690460824561\\
0.9	0	0	0.710202759607373\\
0.9	1.11327767495586e-05	0.00471862581271274	0.713003157837853\\
0.9	4.46175251031912e-05	0.009446325184051	0.715846486660142\\
0.9	0.000100583311362513	0.0141829653360114	0.718788838379794\\
0.9	0.000179158434668431	0.018928411755669	0.721886092228356\\
0.9	0.000280470402701511	0.02368252819604	0.725200636220892\\
0.9	0.000404645907256436	0.0284451766772965	0.728799085274723\\
0.9	0.000551810799695644	0.0332162174883388	0.732749123983336\\
0.9	0.000722090066287311	0.037995509188729	0.737115575253328\\
0.9	0.000915607803432999	0.0427829086109896	0.741955893907011\\
0.9	0.0011324871927904	0.047578270863273	0.747315373652289\\
0.9	0.00137285047629673	0.0523814493324038	0.75322242404581\\
0.9	0.00163681893109844	0.057192295687301	0.759684310525652\\
0.9	0.00192451284439304	0.0620106598827802	0.766683748102665\\
0.9	0.00223605148818899	0.0668363901637434	0.774176695511256\\
0.9	0.00257155309398959	0.0716693330697584	0.782091614494493\\
0.9	0.00293113482740716	0.0765093334400312	0.79033034642109\\
0.9	0.00331491276271365	0.0813562344187764	0.79877062752665\\
0.9	0.00372300185733414	0.0862098774609879	0.807270128930359\\
0.9	0.00415551592628969	0.0910701023386145	0.815671782700193\\
0.9	0.00461256761659621	0.0959367471471425	0.823810053507331\\
0.9	0.00509426838162598	0.100809648312589	0.831517746503354\\
0.9	0.00560072845543873	0.105688640598912	0.838632911419452\\
0.9	0.00613205682708918	0.11057355711583	0.845005411376293\\
0.9	0.00668836121491815	0.115464229327074	0.850502769049302\\
0.9	0.00726974804083431	0.120360487059049	0.855014975717657\\
0.9	0.00787632240459385	0.125262158509929	0.858458041096605\\
0.9	0.00850818805808555	0.13016907025918	0.860776163494182\\
0.9	0.00916544737962849	0.135081047277508	0.861942500814224\\
0.9	0.00984820134829017	0.139997912937243	0.861958614648422\\
0.9	0.0105565495182326	0.144919489023162	0.860852735617091\\
0.9	0.0112905899930939	0.149845595743738	0.85867705410965\\
0.9	0.012050419400414	0.154776051742839	0.855504274967458\\
0.9	0.0128361328661109	0.159710674111862	0.851423687967755\\
0.9	0.0136478239890177	0.164649278402306	0.846537000484341\\
0.9	0.0144855848154856	0.169591678638796	0.840954157904248\\
0.9	0.0153495058140643	0.174537687332543	0.834789345403935\\
0.9	0.0162396758502651	0.179487115495259	0.828157325806201\\
0.9	0.0171561821614173	0.184439772653511	0.821170226448695\\
0.9	0.0180991103316243	0.189395466863524	0.813934846725691\\
0.9	0.0190685442668299	0.194354004726437	0.806550519904212\\
0.9	0.0200645661700016	0.199315191403999	0.799107529846305\\
0.9	0.0210872565164405	0.204278830634725	0.791686056496193\\
0.9	0.0221366940292259	0.20924472475049	0.784355603830605\\
0.9	0.023212955654804	0.214212674693577	0.777174850281115\\
0.9	0.0243161165387281	0.219182480034174	0.770191853862644\\
0.9	0.025446250001561	0.224153938988321	0.763444541554459\\
0.9	0.0266034275149466	0.229126848436298	0.756961413910483\\
0.9	0.0277877186778607	0.234101003941464	0.750762400416766\\
0.9	0.0289991911930496	0.239076199769547	0.744859807805785\\
0.9	0.0302379108436654	0.244052228908366	0.739259311518077\\
0.9	0.0315039414701067	0.24902888308801	0.733960949044502\\
0.9	0.0327973449470747	0.254005952801449	0.72896008240726\\
0.9	0.0341181811608519	0.258983227325592	0.724248305111178\\
0.9	0.0354665079868145	0.263960494742775	0.71981427622068\\
0.9	0.0368423812671862	0.268937541962694	0.71564447061263\\
0.9	0.0382458547890421	0.273914154744766	0.711723839838042\\
0.9	0.0396769802625738	0.278890117720924	0.708036382389732\\
0.9	0.0411358072996216	0.283865214418836	0.704565625567164\\
0.9	0.0426223833924861	0.288839227285554	0.701295023640316\\
0.9	0.0441367538930258	0.293811937711588	0.698208278749355\\
0.9	0.0456789619920503	0.298783126055387	0.69528959205535\\
0.9	0.0472490486990192	0.303752571668251	0.692523853197512\\
0.9	0.0488470528220538	0.308720052919643	0.689896776228117\\
0.9	0.0504730109482718	0.313685347222913	0.687394989990281\\
0.9	0.0521269574244539	0.318648231061428	0.685006090465982\\
0.9	0.0538089243380495	0.323608480015097	0.682718662029128\\
0.9	0.0555189414985328	0.328565868787292	0.680522273853469\\
0.9	0.0572570364191156	0.333520171232163	0.678407456997849\\
0.9	0.0590232342988274	0.338471160382329	0.676365666960044\\
0.9	0.0608175580049697	0.34341860847696	0.674389235783157\\
0.9	0.0626400280559541	0.348362286990222	0.672471317134862\\
0.9	0.064490662604533	0.3533019666601	0.670605827172182\\
0.9	0.066369477421429	0.358237417517579	0.668787383459804\\
0.9	0.0682764858793752	0.363168408916186	0.667011243730728\\
0.9	0.0702116989375697	0.368094709561873	0.665273245863649\\
0.9	0.0721751251265572	0.373016087543259	0.663569750098932\\
0.9	0.0741667705335422	0.377932310362199	0.661897584220061\\
0.9	0.0761866387881432	0.382843144964686	0.660253992184441\\
0.9	0.0782347310485954	0.38774835777208	0.658636586490997\\
0.9	0.0803110459884098	0.392647714712651	0.657043304415872\\
0.9	0.0824155797834956	0.397540981253432	0.655472368126509\\
0.9	0.084548326099754	0.402427922432375	0.653922248592746\\
0.9	0.0867092760811502	0.4073083028908	0.65239163314657\\
0.9	0.0888984183382709	0.412181886906127	0.650879396495577\\
0.9	0.0911157389373742	0.417048438424896	0.649384574964844\\
0.9	0.0933612213899392	0.421907721096044	0.647906343724685\\
0.9	0.0956348466427212	0.42675949830445	0.646443996754806\\
0.9	0.0979365930683194	0.431603533204733	0.644996929295961\\
0.9	0.100266436456264	0.436439588755293	0.643564622546847\\
0.9	0.102624350004627	0.441267427752584	0.64214663037443\\
0.9	0.105010304312169	0.446086812865616	0.640742567819433\\
0.9	0.107424267371012	0.450897506670668	0.63935210119373\\
0.9	0.109866204559871	0.455699271686218	0.637974939582444\\
0.9	0.112336078637817	0.460491870408058	0.636610827579863\\
0.9	0.114833849738607	0.465275065344603	0.635259539104383\\
0.9	0.117359475365564	0.470048619052377	0.633920872153156\\
0.9	0.119912910387023	0.474812294171664	0.632594644371964\\
0.9	0.122494107032347	0.479565853462319	0.631280689329641\\
0.9	0.125103014888515	0.484309059839721	0.629978853399059\\
0.9	0.127739580897283	0.489041676410868	0.628690460824556\\
0.905	0	0	0.710202759607373\\
0.905	1.11327767495586e-05	0.00471862581271275	0.713003157837853\\
0.905	4.46175251031912e-05	0.009446325184051	0.715846486660142\\
0.905	0.000100583311362513	0.0141829653360114	0.718788838379794\\
0.905	0.000179158434668431	0.018928411755669	0.721886092228356\\
0.905	0.000280470402701511	0.02368252819604	0.725200636220892\\
0.905	0.000404645907256436	0.0284451766772965	0.728799085274723\\
0.905	0.000551810799695644	0.0332162174883389	0.732749123983336\\
0.905	0.000722090066287311	0.037995509188729	0.737115575253328\\
0.905	0.000915607803432999	0.0427829086109896	0.741955893907011\\
0.905	0.0011324871927904	0.0475782708632729	0.747315373652289\\
0.905	0.00137285047629673	0.0523814493324038	0.753222424045811\\
0.905	0.00163681893109844	0.057192295687301	0.759684310525652\\
0.905	0.00192451284439304	0.0620106598827802	0.766683748102665\\
0.905	0.00223605148818899	0.0668363901637434	0.774176695511257\\
0.905	0.00257155309398959	0.0716693330697584	0.782091614494493\\
0.905	0.00293113482740716	0.0765093334400312	0.79033034642109\\
0.905	0.00331491276271365	0.0813562344187764	0.79877062752665\\
0.905	0.00372300185733413	0.0862098774609879	0.80727012893036\\
0.905	0.00415551592628969	0.0910701023386145	0.815671782700193\\
0.905	0.00461256761659621	0.0959367471471425	0.823810053507331\\
0.905	0.00509426838162598	0.100809648312589	0.831517746503354\\
0.905	0.00560072845543873	0.105688640598912	0.838632911419451\\
0.905	0.00613205682708918	0.11057355711583	0.845005411376293\\
0.905	0.00668836121491815	0.115464229327074	0.850502769049301\\
0.905	0.00726974804083431	0.120360487059049	0.855014975717657\\
0.905	0.00787632240459385	0.125262158509929	0.858458041096605\\
0.905	0.00850818805808555	0.13016907025918	0.860776163494182\\
0.905	0.00916544737962849	0.135081047277508	0.861942500814224\\
0.905	0.00984820134829017	0.139997912937243	0.86195861464842\\
0.905	0.0105565495182326	0.144919489023162	0.86085273561709\\
0.905	0.0112905899930939	0.149845595743738	0.858677054109652\\
0.905	0.012050419400414	0.154776051742839	0.855504274967458\\
0.905	0.0128361328661109	0.159710674111862	0.851423687967753\\
0.905	0.0136478239890177	0.164649278402306	0.846537000484341\\
0.905	0.0144855848154856	0.169591678638796	0.840954157904245\\
0.905	0.0153495058140643	0.174537687332543	0.834789345403936\\
0.905	0.0162396758502651	0.179487115495259	0.828157325806203\\
0.905	0.0171561821614173	0.184439772653511	0.821170226448693\\
0.905	0.0180991103316243	0.189395466863524	0.813934846725692\\
0.905	0.0190685442668299	0.194354004726437	0.806550519904213\\
0.905	0.0200645661700016	0.199315191403999	0.799107529846305\\
0.905	0.0210872565164405	0.204278830634725	0.791686056496194\\
0.905	0.0221366940292259	0.20924472475049	0.784355603830601\\
0.905	0.023212955654804	0.214212674693577	0.777174850281115\\
0.905	0.0243161165387281	0.219182480034174	0.770191853862645\\
0.905	0.025446250001561	0.224153938988321	0.763444541554457\\
0.905	0.0266034275149466	0.229126848436298	0.756961413910483\\
0.905	0.0277877186778607	0.234101003941464	0.750762400416764\\
0.905	0.0289991911930496	0.239076199769547	0.744859807805783\\
0.905	0.0302379108436654	0.244052228908366	0.739259311518079\\
0.905	0.0315039414701067	0.24902888308801	0.733960949044502\\
0.905	0.0327973449470747	0.254005952801449	0.72896008240726\\
0.905	0.0341181811608519	0.258983227325592	0.724248305111179\\
0.905	0.0354665079868145	0.263960494742775	0.719814276220682\\
0.905	0.0368423812671862	0.268937541962694	0.715644470612629\\
0.905	0.0382458547890421	0.273914154744766	0.711723839838041\\
0.905	0.0396769802625738	0.278890117720924	0.70803638238973\\
0.905	0.0411358072996216	0.283865214418836	0.704565625567164\\
0.905	0.0426223833924861	0.288839227285554	0.701295023640317\\
0.905	0.0441367538930258	0.293811937711588	0.698208278749354\\
0.905	0.0456789619920503	0.298783126055387	0.695289592055346\\
0.905	0.0472490486990192	0.303752571668251	0.692523853197511\\
0.905	0.0488470528220538	0.308720052919643	0.689896776228121\\
0.905	0.0504730109482718	0.313685347222913	0.687394989990282\\
0.905	0.0521269574244539	0.318648231061428	0.68500609046598\\
0.905	0.0538089243380495	0.323608480015096	0.682718662029128\\
0.905	0.0555189414985328	0.328565868787292	0.68052227385347\\
0.905	0.0572570364191156	0.333520171232163	0.67840745699785\\
0.905	0.0590232342988274	0.338471160382329	0.676365666960044\\
0.905	0.0608175580049697	0.34341860847696	0.674389235783155\\
0.905	0.0626400280559541	0.348362286990221	0.672471317134862\\
0.905	0.064490662604533	0.3533019666601	0.670605827172183\\
0.905	0.0663694774214291	0.358237417517579	0.668787383459804\\
0.905	0.0682764858793752	0.363168408916186	0.667011243730728\\
0.905	0.0702116989375697	0.368094709561873	0.665273245863648\\
0.905	0.0721751251265572	0.373016087543259	0.663569750098931\\
0.905	0.0741667705335422	0.377932310362199	0.661897584220061\\
0.905	0.0761866387881432	0.382843144964686	0.660253992184444\\
0.905	0.0782347310485954	0.38774835777208	0.658636586490998\\
0.905	0.0803110459884098	0.392647714712651	0.657043304415871\\
0.905	0.0824155797834956	0.397540981253432	0.655472368126509\\
0.905	0.084548326099754	0.402427922432375	0.653922248592745\\
0.905	0.0867092760811502	0.4073083028908	0.652391633146571\\
0.905	0.0888984183382709	0.412181886906127	0.650879396495579\\
0.905	0.0911157389373742	0.417048438424897	0.649384574964843\\
0.905	0.0933612213899393	0.421907721096044	0.647906343724685\\
0.905	0.0956348466427212	0.42675949830445	0.646443996754805\\
0.905	0.0979365930683194	0.431603533204733	0.644996929295962\\
0.905	0.100266436456264	0.436439588755293	0.643564622546847\\
0.905	0.102624350004627	0.441267427752584	0.64214663037443\\
0.905	0.105010304312169	0.446086812865616	0.640742567819433\\
0.905	0.107424267371012	0.450897506670668	0.639352101193731\\
0.905	0.109866204559871	0.455699271686218	0.637974939582443\\
0.905	0.112336078637817	0.460491870408058	0.636610827579862\\
0.905	0.114833849738607	0.465275065344603	0.635259539104382\\
0.905	0.117359475365564	0.470048619052377	0.633920872153156\\
0.905	0.119912910387022	0.474812294171664	0.632594644371966\\
0.905	0.122494107032347	0.479565853462319	0.631280689329639\\
0.905	0.125103014888515	0.484309059839721	0.629978853399059\\
0.905	0.127739580897283	0.489041676410868	0.628690460824562\\
0.91	0	0	0.710202759607373\\
0.91	1.11327767495586e-05	0.00471862581271275	0.713003157837853\\
0.91	4.46175251031912e-05	0.009446325184051	0.715846486660142\\
0.91	0.000100583311362513	0.0141829653360114	0.718788838379794\\
0.91	0.000179158434668431	0.018928411755669	0.721886092228356\\
0.91	0.000280470402701511	0.02368252819604	0.725200636220892\\
0.91	0.000404645907256436	0.0284451766772965	0.728799085274723\\
0.91	0.000551810799695644	0.0332162174883389	0.732749123983336\\
0.91	0.000722090066287311	0.037995509188729	0.737115575253327\\
0.91	0.000915607803432999	0.0427829086109896	0.741955893907011\\
0.91	0.0011324871927904	0.0475782708632729	0.747315373652289\\
0.91	0.00137285047629673	0.0523814493324038	0.753222424045811\\
0.91	0.00163681893109844	0.057192295687301	0.759684310525652\\
0.91	0.00192451284439304	0.0620106598827802	0.766683748102665\\
0.91	0.00223605148818899	0.0668363901637434	0.774176695511257\\
0.91	0.00257155309398959	0.0716693330697584	0.782091614494493\\
0.91	0.00293113482740716	0.0765093334400312	0.79033034642109\\
0.91	0.00331491276271365	0.0813562344187764	0.79877062752665\\
0.91	0.00372300185733414	0.0862098774609879	0.80727012893036\\
0.91	0.00415551592628969	0.0910701023386145	0.815671782700193\\
0.91	0.00461256761659621	0.0959367471471425	0.82381005350733\\
0.91	0.00509426838162598	0.100809648312589	0.831517746503354\\
0.91	0.00560072845543873	0.105688640598912	0.838632911419452\\
0.91	0.00613205682708918	0.11057355711583	0.845005411376292\\
0.91	0.00668836121491816	0.115464229327074	0.850502769049302\\
0.91	0.00726974804083431	0.120360487059049	0.855014975717657\\
0.91	0.00787632240459385	0.125262158509929	0.858458041096607\\
0.91	0.00850818805808555	0.13016907025918	0.860776163494181\\
0.91	0.00916544737962849	0.135081047277508	0.861942500814224\\
0.91	0.00984820134829017	0.139997912937243	0.861958614648419\\
0.91	0.0105565495182326	0.144919489023162	0.860852735617091\\
0.91	0.0112905899930939	0.149845595743738	0.858677054109651\\
0.91	0.012050419400414	0.154776051742839	0.855504274967456\\
0.91	0.0128361328661109	0.159710674111862	0.851423687967755\\
0.91	0.0136478239890177	0.164649278402306	0.846537000484339\\
0.91	0.0144855848154856	0.169591678638796	0.84095415790425\\
0.91	0.0153495058140643	0.174537687332543	0.834789345403936\\
0.91	0.0162396758502651	0.179487115495259	0.8281573258062\\
0.91	0.0171561821614173	0.184439772653511	0.821170226448692\\
0.91	0.0180991103316243	0.189395466863524	0.813934846725692\\
0.91	0.0190685442668299	0.194354004726437	0.806550519904212\\
0.91	0.0200645661700016	0.199315191403999	0.799107529846304\\
0.91	0.0210872565164405	0.204278830634725	0.791686056496193\\
0.91	0.0221366940292259	0.20924472475049	0.7843556038306\\
0.91	0.023212955654804	0.214212674693577	0.777174850281118\\
0.91	0.0243161165387281	0.219182480034174	0.770191853862643\\
0.91	0.025446250001561	0.224153938988321	0.763444541554459\\
0.91	0.0266034275149466	0.229126848436298	0.756961413910483\\
0.91	0.0277877186778607	0.234101003941464	0.750762400416764\\
0.91	0.0289991911930496	0.239076199769547	0.744859807805786\\
0.91	0.0302379108436654	0.244052228908366	0.739259311518078\\
0.91	0.0315039414701067	0.24902888308801	0.733960949044502\\
0.91	0.0327973449470747	0.254005952801449	0.728960082407261\\
0.91	0.0341181811608519	0.258983227325592	0.724248305111179\\
0.91	0.0354665079868145	0.263960494742775	0.719814276220681\\
0.91	0.0368423812671862	0.268937541962694	0.715644470612629\\
0.91	0.0382458547890421	0.273914154744766	0.711723839838042\\
0.91	0.0396769802625738	0.278890117720924	0.70803638238973\\
0.91	0.0411358072996216	0.283865214418836	0.704565625567164\\
0.91	0.0426223833924861	0.288839227285554	0.701295023640317\\
0.91	0.0441367538930258	0.293811937711588	0.698208278749353\\
0.91	0.0456789619920503	0.298783126055387	0.695289592055347\\
0.91	0.0472490486990192	0.303752571668251	0.692523853197512\\
0.91	0.0488470528220537	0.308720052919643	0.68989677622812\\
0.91	0.0504730109482718	0.313685347222913	0.687394989990281\\
0.91	0.0521269574244539	0.318648231061428	0.685006090465981\\
0.91	0.0538089243380495	0.323608480015097	0.682718662029127\\
0.91	0.0555189414985328	0.328565868787292	0.68052227385347\\
0.91	0.0572570364191156	0.333520171232163	0.67840745699785\\
0.91	0.0590232342988274	0.338471160382329	0.676365666960043\\
0.91	0.0608175580049697	0.34341860847696	0.674389235783156\\
0.91	0.0626400280559541	0.348362286990222	0.672471317134862\\
0.91	0.0644906626045329	0.3533019666601	0.670605827172183\\
0.91	0.066369477421429	0.358237417517579	0.668787383459804\\
0.91	0.0682764858793752	0.363168408916186	0.667011243730728\\
0.91	0.0702116989375697	0.368094709561873	0.665273245863649\\
0.91	0.0721751251265572	0.373016087543259	0.663569750098931\\
0.91	0.0741667705335422	0.377932310362199	0.661897584220059\\
0.91	0.0761866387881432	0.382843144964686	0.660253992184443\\
0.91	0.0782347310485954	0.38774835777208	0.658636586490998\\
0.91	0.0803110459884098	0.392647714712651	0.657043304415871\\
0.91	0.0824155797834956	0.397540981253432	0.65547236812651\\
0.91	0.084548326099754	0.402427922432375	0.653922248592745\\
0.91	0.0867092760811502	0.407308302890799	0.65239163314657\\
0.91	0.0888984183382709	0.412181886906127	0.650879396495579\\
0.91	0.0911157389373742	0.417048438424896	0.649384574964844\\
0.91	0.0933612213899392	0.421907721096044	0.647906343724686\\
0.91	0.0956348466427212	0.42675949830445	0.646443996754805\\
0.91	0.0979365930683194	0.431603533204733	0.644996929295961\\
0.91	0.100266436456264	0.436439588755293	0.643564622546847\\
0.91	0.102624350004627	0.441267427752584	0.64214663037443\\
0.91	0.105010304312169	0.446086812865616	0.640742567819433\\
0.91	0.107424267371012	0.450897506670668	0.63935210119373\\
0.91	0.109866204559871	0.455699271686218	0.637974939582443\\
0.91	0.112336078637817	0.460491870408058	0.636610827579863\\
0.91	0.114833849738607	0.465275065344603	0.635259539104382\\
0.91	0.117359475365564	0.470048619052377	0.633920872153154\\
0.91	0.119912910387022	0.474812294171664	0.632594644371966\\
0.91	0.122494107032347	0.479565853462319	0.631280689329641\\
0.91	0.125103014888515	0.484309059839721	0.629978853399057\\
0.91	0.127739580897283	0.489041676410868	0.628690460824557\\
0.915	0	0	0.710202759607373\\
0.915	1.11327767495586e-05	0.00471862581271274	0.713003157837853\\
0.915	4.46175251031912e-05	0.009446325184051	0.715846486660142\\
0.915	0.000100583311362513	0.0141829653360114	0.718788838379794\\
0.915	0.000179158434668431	0.018928411755669	0.721886092228356\\
0.915	0.000280470402701511	0.02368252819604	0.725200636220892\\
0.915	0.000404645907256436	0.0284451766772965	0.728799085274723\\
0.915	0.000551810799695644	0.0332162174883389	0.732749123983336\\
0.915	0.000722090066287311	0.037995509188729	0.737115575253328\\
0.915	0.000915607803432999	0.0427829086109896	0.741955893907011\\
0.915	0.0011324871927904	0.0475782708632729	0.747315373652289\\
0.915	0.00137285047629673	0.0523814493324038	0.75322242404581\\
0.915	0.00163681893109843	0.057192295687301	0.759684310525652\\
0.915	0.00192451284439304	0.0620106598827802	0.766683748102665\\
0.915	0.00223605148818899	0.0668363901637434	0.774176695511256\\
0.915	0.00257155309398959	0.0716693330697584	0.782091614494494\\
0.915	0.00293113482740716	0.0765093334400312	0.79033034642109\\
0.915	0.00331491276271365	0.0813562344187764	0.79877062752665\\
0.915	0.00372300185733414	0.0862098774609879	0.807270128930359\\
0.915	0.00415551592628969	0.0910701023386145	0.815671782700194\\
0.915	0.00461256761659621	0.0959367471471425	0.82381005350733\\
0.915	0.00509426838162598	0.100809648312589	0.831517746503354\\
0.915	0.00560072845543873	0.105688640598912	0.838632911419452\\
0.915	0.00613205682708918	0.11057355711583	0.845005411376292\\
0.915	0.00668836121491815	0.115464229327074	0.850502769049302\\
0.915	0.00726974804083431	0.120360487059049	0.855014975717657\\
0.915	0.00787632240459385	0.125262158509929	0.858458041096608\\
0.915	0.00850818805808554	0.13016907025918	0.860776163494182\\
0.915	0.00916544737962849	0.135081047277508	0.861942500814224\\
0.915	0.00984820134829017	0.139997912937243	0.861958614648419\\
0.915	0.0105565495182326	0.144919489023162	0.860852735617091\\
0.915	0.0112905899930939	0.149845595743738	0.858677054109649\\
0.915	0.012050419400414	0.154776051742839	0.855504274967457\\
0.915	0.0128361328661109	0.159710674111862	0.851423687967754\\
0.915	0.0136478239890177	0.164649278402306	0.846537000484339\\
0.915	0.0144855848154856	0.169591678638796	0.84095415790425\\
0.915	0.0153495058140643	0.174537687332543	0.834789345403935\\
0.915	0.0162396758502651	0.179487115495259	0.8281573258062\\
0.915	0.0171561821614173	0.184439772653511	0.821170226448694\\
0.915	0.0180991103316243	0.189395466863524	0.813934846725693\\
0.915	0.0190685442668299	0.194354004726437	0.806550519904212\\
0.915	0.0200645661700016	0.199315191403999	0.799107529846304\\
0.915	0.0210872565164405	0.204278830634725	0.791686056496193\\
0.915	0.0221366940292259	0.20924472475049	0.784355603830602\\
0.915	0.023212955654804	0.214212674693577	0.777174850281119\\
0.915	0.0243161165387281	0.219182480034174	0.770191853862643\\
0.915	0.025446250001561	0.224153938988321	0.76344454155446\\
0.915	0.0266034275149466	0.229126848436298	0.756961413910482\\
0.915	0.0277877186778607	0.234101003941464	0.750762400416764\\
0.915	0.0289991911930496	0.239076199769547	0.744859807805785\\
0.915	0.0302379108436654	0.244052228908366	0.739259311518077\\
0.915	0.0315039414701067	0.24902888308801	0.733960949044504\\
0.915	0.0327973449470747	0.254005952801449	0.728960082407262\\
0.915	0.0341181811608519	0.258983227325592	0.724248305111178\\
0.915	0.0354665079868145	0.263960494742775	0.71981427622068\\
0.915	0.0368423812671862	0.268937541962694	0.715644470612629\\
0.915	0.0382458547890421	0.273914154744766	0.711723839838042\\
0.915	0.0396769802625738	0.278890117720924	0.70803638238973\\
0.915	0.0411358072996216	0.283865214418836	0.704565625567164\\
0.915	0.0426223833924861	0.288839227285554	0.701295023640316\\
0.915	0.0441367538930258	0.293811937711588	0.698208278749353\\
0.915	0.0456789619920503	0.298783126055387	0.695289592055349\\
0.915	0.0472490486990192	0.303752571668251	0.692523853197512\\
0.915	0.0488470528220538	0.308720052919643	0.689896776228119\\
0.915	0.0504730109482718	0.313685347222913	0.687394989990281\\
0.915	0.0521269574244539	0.318648231061427	0.685006090465981\\
0.915	0.0538089243380495	0.323608480015096	0.682718662029128\\
0.915	0.0555189414985328	0.328565868787292	0.680522273853469\\
0.915	0.0572570364191156	0.333520171232163	0.67840745699785\\
0.915	0.0590232342988274	0.338471160382329	0.676365666960044\\
0.915	0.0608175580049697	0.34341860847696	0.674389235783157\\
0.915	0.0626400280559541	0.348362286990222	0.672471317134861\\
0.915	0.064490662604533	0.3533019666601	0.670605827172181\\
0.915	0.0663694774214291	0.358237417517579	0.668787383459805\\
0.915	0.0682764858793752	0.363168408916186	0.667011243730728\\
0.915	0.0702116989375697	0.368094709561873	0.66527324586365\\
0.915	0.0721751251265572	0.373016087543259	0.663569750098933\\
0.915	0.0741667705335422	0.377932310362199	0.66189758422006\\
0.915	0.0761866387881432	0.382843144964686	0.660253992184442\\
0.915	0.0782347310485954	0.38774835777208	0.658636586490997\\
0.915	0.0803110459884098	0.392647714712651	0.65704330441587\\
0.915	0.0824155797834956	0.397540981253432	0.65547236812651\\
0.915	0.084548326099754	0.402427922432375	0.653922248592746\\
0.915	0.0867092760811502	0.4073083028908	0.65239163314657\\
0.915	0.0888984183382709	0.412181886906127	0.650879396495579\\
0.915	0.0911157389373742	0.417048438424897	0.649384574964843\\
0.915	0.0933612213899392	0.421907721096044	0.647906343724685\\
0.915	0.0956348466427212	0.42675949830445	0.646443996754806\\
0.915	0.0979365930683194	0.431603533204733	0.644996929295961\\
0.915	0.100266436456264	0.436439588755293	0.643564622546846\\
0.915	0.102624350004627	0.441267427752584	0.64214663037443\\
0.915	0.105010304312169	0.446086812865616	0.640742567819434\\
0.915	0.107424267371012	0.450897506670668	0.639352101193731\\
0.915	0.109866204559871	0.455699271686218	0.637974939582442\\
0.915	0.112336078637817	0.460491870408058	0.636610827579862\\
0.915	0.114833849738607	0.465275065344603	0.635259539104383\\
0.915	0.117359475365564	0.470048619052377	0.633920872153154\\
0.915	0.119912910387022	0.474812294171664	0.632594644371965\\
0.915	0.122494107032347	0.479565853462319	0.631280689329642\\
0.915	0.125103014888515	0.484309059839721	0.629978853399059\\
0.915	0.127739580897283	0.489041676410868	0.628690460824558\\
0.92	0	0	0.710202759607373\\
0.92	1.11327767495586e-05	0.00471862581271275	0.713003157837853\\
0.92	4.46175251031912e-05	0.009446325184051	0.715846486660142\\
0.92	0.000100583311362513	0.0141829653360114	0.718788838379794\\
0.92	0.000179158434668431	0.018928411755669	0.721886092228356\\
0.92	0.000280470402701511	0.02368252819604	0.725200636220892\\
0.92	0.000404645907256436	0.0284451766772965	0.728799085274723\\
0.92	0.000551810799695644	0.0332162174883388	0.732749123983336\\
0.92	0.000722090066287311	0.037995509188729	0.737115575253327\\
0.92	0.000915607803432999	0.0427829086109896	0.741955893907011\\
0.92	0.0011324871927904	0.0475782708632729	0.747315373652289\\
0.92	0.00137285047629673	0.0523814493324038	0.753222424045811\\
0.92	0.00163681893109844	0.057192295687301	0.759684310525652\\
0.92	0.00192451284439304	0.0620106598827802	0.766683748102665\\
0.92	0.00223605148818899	0.0668363901637434	0.774176695511256\\
0.92	0.00257155309398959	0.0716693330697584	0.782091614494494\\
0.92	0.00293113482740716	0.0765093334400313	0.79033034642109\\
0.92	0.00331491276271365	0.0813562344187764	0.79877062752665\\
0.92	0.00372300185733414	0.0862098774609879	0.807270128930359\\
0.92	0.00415551592628969	0.0910701023386145	0.815671782700193\\
0.92	0.00461256761659621	0.0959367471471425	0.823810053507331\\
0.92	0.00509426838162598	0.100809648312589	0.831517746503354\\
0.92	0.00560072845543873	0.105688640598912	0.838632911419452\\
0.92	0.00613205682708918	0.11057355711583	0.845005411376292\\
0.92	0.00668836121491816	0.115464229327074	0.850502769049302\\
0.92	0.00726974804083431	0.120360487059049	0.855014975717657\\
0.92	0.00787632240459385	0.125262158509929	0.858458041096605\\
0.92	0.00850818805808555	0.13016907025918	0.860776163494183\\
0.92	0.00916544737962849	0.135081047277508	0.861942500814225\\
0.92	0.00984820134829017	0.139997912937243	0.861958614648422\\
0.92	0.0105565495182326	0.144919489023162	0.860852735617091\\
0.92	0.0112905899930939	0.149845595743738	0.858677054109651\\
0.92	0.012050419400414	0.154776051742839	0.855504274967457\\
0.92	0.0128361328661109	0.159710674111862	0.851423687967754\\
0.92	0.0136478239890177	0.164649278402306	0.846537000484341\\
0.92	0.0144855848154856	0.169591678638796	0.840954157904248\\
0.92	0.0153495058140643	0.174537687332543	0.834789345403935\\
0.92	0.0162396758502651	0.179487115495259	0.828157325806199\\
0.92	0.0171561821614173	0.184439772653511	0.821170226448694\\
0.92	0.0180991103316243	0.189395466863524	0.81393484672569\\
0.92	0.0190685442668299	0.194354004726437	0.806550519904213\\
0.92	0.0200645661700016	0.199315191403999	0.799107529846304\\
0.92	0.0210872565164405	0.204278830634725	0.791686056496194\\
0.92	0.0221366940292259	0.20924472475049	0.784355603830602\\
0.92	0.023212955654804	0.214212674693577	0.777174850281118\\
0.92	0.0243161165387281	0.219182480034174	0.770191853862644\\
0.92	0.025446250001561	0.224153938988321	0.76344454155446\\
0.92	0.0266034275149466	0.229126848436298	0.756961413910481\\
0.92	0.0277877186778607	0.234101003941464	0.750762400416766\\
0.92	0.0289991911930496	0.239076199769547	0.744859807805785\\
0.92	0.0302379108436654	0.244052228908366	0.739259311518079\\
0.92	0.0315039414701067	0.24902888308801	0.733960949044505\\
0.92	0.0327973449470747	0.254005952801449	0.72896008240726\\
0.92	0.0341181811608519	0.258983227325591	0.724248305111178\\
0.92	0.0354665079868145	0.263960494742775	0.719814276220682\\
0.92	0.0368423812671862	0.268937541962694	0.71564447061263\\
0.92	0.0382458547890421	0.273914154744766	0.711723839838042\\
0.92	0.0396769802625738	0.278890117720924	0.70803638238973\\
0.92	0.0411358072996216	0.283865214418836	0.704565625567164\\
0.92	0.0426223833924861	0.288839227285554	0.701295023640316\\
0.92	0.0441367538930258	0.293811937711588	0.698208278749354\\
0.92	0.0456789619920503	0.298783126055387	0.695289592055349\\
0.92	0.0472490486990192	0.303752571668251	0.692523853197511\\
0.92	0.0488470528220538	0.308720052919643	0.689896776228119\\
0.92	0.0504730109482718	0.313685347222913	0.687394989990282\\
0.92	0.0521269574244539	0.318648231061428	0.685006090465981\\
0.92	0.0538089243380495	0.323608480015096	0.682718662029128\\
0.92	0.0555189414985328	0.328565868787292	0.680522273853469\\
0.92	0.0572570364191156	0.333520171232163	0.67840745699785\\
0.92	0.0590232342988274	0.338471160382329	0.676365666960043\\
0.92	0.0608175580049697	0.34341860847696	0.674389235783157\\
0.92	0.0626400280559541	0.348362286990222	0.672471317134863\\
0.92	0.064490662604533	0.3533019666601	0.670605827172181\\
0.92	0.0663694774214291	0.358237417517579	0.668787383459803\\
0.92	0.0682764858793752	0.363168408916186	0.667011243730729\\
0.92	0.0702116989375697	0.368094709561873	0.665273245863649\\
0.92	0.0721751251265572	0.373016087543259	0.663569750098933\\
0.92	0.0741667705335422	0.377932310362199	0.66189758422006\\
0.92	0.0761866387881432	0.382843144964686	0.660253992184443\\
0.92	0.0782347310485954	0.38774835777208	0.658636586490998\\
0.92	0.0803110459884098	0.392647714712651	0.65704330441587\\
0.92	0.0824155797834956	0.397540981253432	0.655472368126508\\
0.92	0.084548326099754	0.402427922432375	0.653922248592745\\
0.92	0.0867092760811502	0.407308302890799	0.652391633146571\\
0.92	0.0888984183382709	0.412181886906127	0.65087939649558\\
0.92	0.0911157389373742	0.417048438424896	0.649384574964843\\
0.92	0.0933612213899392	0.421907721096044	0.647906343724685\\
0.92	0.0956348466427212	0.42675949830445	0.646443996754806\\
0.92	0.0979365930683194	0.431603533204733	0.644996929295961\\
0.92	0.100266436456264	0.436439588755293	0.643564622546846\\
0.92	0.102624350004627	0.441267427752584	0.642146630374429\\
0.92	0.105010304312169	0.446086812865616	0.640742567819433\\
0.92	0.107424267371012	0.450897506670668	0.63935210119373\\
0.92	0.109866204559871	0.455699271686218	0.637974939582444\\
0.92	0.112336078637817	0.460491870408058	0.636610827579863\\
0.92	0.114833849738607	0.465275065344603	0.635259539104381\\
0.92	0.117359475365564	0.470048619052377	0.633920872153154\\
0.92	0.119912910387022	0.474812294171664	0.632594644371966\\
0.92	0.122494107032347	0.479565853462319	0.631280689329641\\
0.92	0.125103014888515	0.484309059839721	0.629978853399058\\
0.92	0.127739580897283	0.489041676410868	0.628690460824559\\
0.925	0	0	0.710202759607373\\
0.925	1.11327767495586e-05	0.00471862581271275	0.713003157837853\\
0.925	4.46175251031912e-05	0.009446325184051	0.715846486660142\\
0.925	0.000100583311362513	0.0141829653360114	0.718788838379794\\
0.925	0.000179158434668431	0.018928411755669	0.721886092228356\\
0.925	0.000280470402701511	0.02368252819604	0.725200636220892\\
0.925	0.000404645907256436	0.0284451766772965	0.728799085274723\\
0.925	0.000551810799695644	0.0332162174883389	0.732749123983336\\
0.925	0.000722090066287311	0.037995509188729	0.737115575253327\\
0.925	0.000915607803432999	0.0427829086109896	0.741955893907011\\
0.925	0.0011324871927904	0.0475782708632729	0.747315373652289\\
0.925	0.00137285047629673	0.0523814493324038	0.75322242404581\\
0.925	0.00163681893109844	0.057192295687301	0.759684310525652\\
0.925	0.00192451284439304	0.0620106598827802	0.766683748102665\\
0.925	0.00223605148818899	0.0668363901637434	0.774176695511256\\
0.925	0.00257155309398959	0.0716693330697584	0.782091614494493\\
0.925	0.00293113482740716	0.0765093334400312	0.79033034642109\\
0.925	0.00331491276271365	0.0813562344187764	0.79877062752665\\
0.925	0.00372300185733413	0.0862098774609879	0.807270128930359\\
0.925	0.00415551592628969	0.0910701023386145	0.815671782700194\\
0.925	0.00461256761659621	0.0959367471471425	0.823810053507331\\
0.925	0.00509426838162598	0.100809648312589	0.831517746503353\\
0.925	0.00560072845543873	0.105688640598912	0.838632911419452\\
0.925	0.00613205682708918	0.11057355711583	0.845005411376293\\
0.925	0.00668836121491815	0.115464229327074	0.850502769049302\\
0.925	0.00726974804083431	0.120360487059049	0.855014975717657\\
0.925	0.00787632240459385	0.125262158509929	0.858458041096605\\
0.925	0.00850818805808555	0.13016907025918	0.860776163494182\\
0.925	0.00916544737962849	0.135081047277508	0.861942500814224\\
0.925	0.00984820134829017	0.139997912937243	0.861958614648421\\
0.925	0.0105565495182326	0.144919489023162	0.860852735617091\\
0.925	0.0112905899930939	0.149845595743738	0.858677054109652\\
0.925	0.012050419400414	0.154776051742839	0.855504274967458\\
0.925	0.0128361328661109	0.159710674111862	0.851423687967754\\
0.925	0.0136478239890177	0.164649278402306	0.84653700048434\\
0.925	0.0144855848154856	0.169591678638796	0.840954157904248\\
0.925	0.0153495058140643	0.174537687332543	0.834789345403935\\
0.925	0.0162396758502651	0.179487115495259	0.8281573258062\\
0.925	0.0171561821614173	0.184439772653511	0.821170226448693\\
0.925	0.0180991103316243	0.189395466863524	0.813934846725691\\
0.925	0.0190685442668299	0.194354004726437	0.806550519904212\\
0.925	0.0200645661700016	0.199315191403999	0.799107529846304\\
0.925	0.0210872565164405	0.204278830634725	0.791686056496193\\
0.925	0.0221366940292259	0.20924472475049	0.784355603830603\\
0.925	0.023212955654804	0.214212674693577	0.777174850281119\\
0.925	0.0243161165387281	0.219182480034174	0.770191853862645\\
0.925	0.025446250001561	0.224153938988321	0.763444541554458\\
0.925	0.0266034275149466	0.229126848436298	0.756961413910482\\
0.925	0.0277877186778607	0.234101003941464	0.750762400416766\\
0.925	0.0289991911930496	0.239076199769547	0.744859807805786\\
0.925	0.0302379108436654	0.244052228908366	0.739259311518079\\
0.925	0.0315039414701067	0.24902888308801	0.733960949044503\\
0.925	0.0327973449470747	0.254005952801449	0.728960082407261\\
0.925	0.0341181811608519	0.258983227325592	0.724248305111178\\
0.925	0.0354665079868145	0.263960494742775	0.719814276220681\\
0.925	0.0368423812671862	0.268937541962694	0.71564447061263\\
0.925	0.0382458547890421	0.273914154744766	0.71172383983804\\
0.925	0.0396769802625738	0.278890117720924	0.70803638238973\\
0.925	0.0411358072996216	0.283865214418836	0.704565625567166\\
0.925	0.0426223833924861	0.288839227285554	0.701295023640316\\
0.925	0.0441367538930258	0.293811937711588	0.698208278749353\\
0.925	0.0456789619920503	0.298783126055387	0.695289592055347\\
0.925	0.0472490486990192	0.303752571668251	0.692523853197512\\
0.925	0.0488470528220538	0.308720052919643	0.68989677622812\\
0.925	0.0504730109482718	0.313685347222913	0.687394989990281\\
0.925	0.0521269574244539	0.318648231061427	0.685006090465982\\
0.925	0.0538089243380495	0.323608480015097	0.682718662029129\\
0.925	0.0555189414985328	0.328565868787292	0.680522273853468\\
0.925	0.0572570364191156	0.333520171232163	0.67840745699785\\
0.925	0.0590232342988274	0.338471160382329	0.676365666960043\\
0.925	0.0608175580049697	0.34341860847696	0.674389235783156\\
0.925	0.0626400280559541	0.348362286990222	0.672471317134862\\
0.925	0.064490662604533	0.3533019666601	0.670605827172183\\
0.925	0.0663694774214291	0.358237417517579	0.668787383459803\\
0.925	0.0682764858793752	0.363168408916186	0.667011243730728\\
0.925	0.0702116989375697	0.368094709561873	0.665273245863649\\
0.925	0.0721751251265572	0.373016087543259	0.663569750098932\\
0.925	0.0741667705335422	0.377932310362199	0.661897584220061\\
0.925	0.0761866387881432	0.382843144964686	0.660253992184443\\
0.925	0.0782347310485954	0.38774835777208	0.658636586490998\\
0.925	0.0803110459884098	0.392647714712651	0.657043304415871\\
0.925	0.0824155797834956	0.397540981253432	0.655472368126509\\
0.925	0.084548326099754	0.402427922432375	0.653922248592745\\
0.925	0.0867092760811502	0.407308302890799	0.652391633146569\\
0.925	0.0888984183382708	0.412181886906127	0.65087939649558\\
0.925	0.0911157389373742	0.417048438424896	0.649384574964844\\
0.925	0.0933612213899392	0.421907721096044	0.647906343724684\\
0.925	0.0956348466427212	0.42675949830445	0.646443996754806\\
0.925	0.0979365930683194	0.431603533204733	0.644996929295962\\
0.925	0.100266436456264	0.436439588755293	0.643564622546847\\
0.925	0.102624350004627	0.441267427752584	0.64214663037443\\
0.925	0.105010304312169	0.446086812865616	0.640742567819433\\
0.925	0.107424267371012	0.450897506670668	0.639352101193731\\
0.925	0.109866204559871	0.455699271686218	0.637974939582442\\
0.925	0.112336078637817	0.460491870408058	0.636610827579862\\
0.925	0.114833849738607	0.465275065344603	0.635259539104383\\
0.925	0.117359475365564	0.470048619052377	0.633920872153154\\
0.925	0.119912910387022	0.474812294171664	0.632594644371966\\
0.925	0.122494107032347	0.479565853462319	0.631280689329641\\
0.925	0.125103014888515	0.484309059839721	0.629978853399059\\
0.925	0.127739580897283	0.489041676410868	0.628690460824558\\
0.93	0	0	0.710202759607373\\
0.93	1.11327767495586e-05	0.00471862581271275	0.713003157837853\\
0.93	4.46175251031912e-05	0.009446325184051	0.715846486660142\\
0.93	0.000100583311362513	0.0141829653360114	0.718788838379794\\
0.93	0.000179158434668431	0.018928411755669	0.721886092228356\\
0.93	0.000280470402701511	0.02368252819604	0.725200636220892\\
0.93	0.000404645907256436	0.0284451766772965	0.728799085274723\\
0.93	0.000551810799695644	0.0332162174883388	0.732749123983336\\
0.93	0.000722090066287311	0.037995509188729	0.737115575253327\\
0.93	0.000915607803432999	0.0427829086109896	0.741955893907011\\
0.93	0.0011324871927904	0.0475782708632729	0.747315373652289\\
0.93	0.00137285047629673	0.0523814493324038	0.75322242404581\\
0.93	0.00163681893109844	0.057192295687301	0.759684310525653\\
0.93	0.00192451284439304	0.0620106598827802	0.766683748102665\\
0.93	0.00223605148818899	0.0668363901637434	0.774176695511256\\
0.93	0.00257155309398959	0.0716693330697584	0.782091614494493\\
0.93	0.00293113482740716	0.0765093334400313	0.79033034642109\\
0.93	0.00331491276271365	0.0813562344187764	0.798770627526651\\
0.93	0.00372300185733413	0.0862098774609879	0.807270128930359\\
0.93	0.00415551592628969	0.0910701023386145	0.815671782700193\\
0.93	0.00461256761659621	0.0959367471471425	0.82381005350733\\
0.93	0.00509426838162598	0.100809648312589	0.831517746503354\\
0.93	0.00560072845543873	0.105688640598912	0.838632911419451\\
0.93	0.00613205682708918	0.11057355711583	0.845005411376292\\
0.93	0.00668836121491815	0.115464229327074	0.850502769049302\\
0.93	0.00726974804083431	0.120360487059049	0.855014975717656\\
0.93	0.00787632240459385	0.125262158509929	0.858458041096605\\
0.93	0.00850818805808555	0.13016907025918	0.860776163494182\\
0.93	0.00916544737962849	0.135081047277508	0.861942500814224\\
0.93	0.00984820134829017	0.139997912937243	0.86195861464842\\
0.93	0.0105565495182326	0.144919489023162	0.86085273561709\\
0.93	0.0112905899930939	0.149845595743738	0.858677054109649\\
0.93	0.012050419400414	0.154776051742839	0.855504274967458\\
0.93	0.0128361328661109	0.159710674111862	0.851423687967756\\
0.93	0.0136478239890177	0.164649278402306	0.846537000484339\\
0.93	0.0144855848154856	0.169591678638796	0.84095415790425\\
0.93	0.0153495058140643	0.174537687332543	0.834789345403935\\
0.93	0.0162396758502651	0.179487115495259	0.8281573258062\\
0.93	0.0171561821614173	0.184439772653511	0.821170226448694\\
0.93	0.0180991103316243	0.189395466863524	0.813934846725693\\
0.93	0.0190685442668299	0.194354004726437	0.806550519904213\\
0.93	0.0200645661700016	0.199315191403999	0.799107529846304\\
0.93	0.0210872565164405	0.204278830634725	0.791686056496194\\
0.93	0.0221366940292259	0.20924472475049	0.784355603830601\\
0.93	0.023212955654804	0.214212674693577	0.777174850281118\\
0.93	0.0243161165387281	0.219182480034174	0.770191853862643\\
0.93	0.025446250001561	0.224153938988321	0.763444541554458\\
0.93	0.0266034275149466	0.229126848436298	0.756961413910484\\
0.93	0.0277877186778607	0.234101003941464	0.750762400416766\\
0.93	0.0289991911930496	0.239076199769547	0.744859807805784\\
0.93	0.0302379108436654	0.244052228908366	0.739259311518078\\
0.93	0.0315039414701067	0.24902888308801	0.733960949044504\\
0.93	0.0327973449470747	0.254005952801449	0.72896008240726\\
0.93	0.0341181811608519	0.258983227325592	0.724248305111178\\
0.93	0.0354665079868145	0.263960494742775	0.719814276220681\\
0.93	0.0368423812671862	0.268937541962694	0.715644470612629\\
0.93	0.0382458547890421	0.273914154744766	0.711723839838042\\
0.93	0.0396769802625738	0.278890117720924	0.708036382389731\\
0.93	0.0411358072996216	0.283865214418836	0.704565625567164\\
0.93	0.0426223833924861	0.288839227285554	0.701295023640317\\
0.93	0.0441367538930258	0.293811937711588	0.698208278749354\\
0.93	0.0456789619920503	0.298783126055387	0.695289592055349\\
0.93	0.0472490486990192	0.303752571668251	0.692523853197511\\
0.93	0.0488470528220538	0.308720052919643	0.689896776228119\\
0.93	0.0504730109482718	0.313685347222913	0.687394989990281\\
0.93	0.0521269574244539	0.318648231061427	0.685006090465979\\
0.93	0.0538089243380495	0.323608480015096	0.682718662029129\\
0.929999999999999	0.0555189414985328	0.328565868787292	0.680522273853471\\
0.93	0.0572570364191156	0.333520171232163	0.67840745699785\\
0.93	0.0590232342988274	0.338471160382329	0.676365666960043\\
0.93	0.0608175580049697	0.34341860847696	0.674389235783155\\
0.929999999999999	0.0626400280559541	0.348362286990221	0.672471317134862\\
0.93	0.064490662604533	0.3533019666601	0.670605827172184\\
0.93	0.0663694774214291	0.358237417517579	0.668787383459804\\
0.93	0.0682764858793752	0.363168408916186	0.667011243730728\\
0.93	0.0702116989375697	0.368094709561873	0.665273245863648\\
0.93	0.0721751251265572	0.373016087543259	0.663569750098931\\
0.93	0.0741667705335422	0.377932310362199	0.661897584220059\\
0.93	0.0761866387881432	0.382843144964686	0.660253992184443\\
0.93	0.0782347310485954	0.38774835777208	0.658636586490998\\
0.93	0.0803110459884098	0.392647714712651	0.657043304415871\\
0.93	0.0824155797834956	0.397540981253432	0.65547236812651\\
0.93	0.084548326099754	0.402427922432375	0.653922248592745\\
0.93	0.0867092760811502	0.407308302890799	0.652391633146568\\
0.93	0.0888984183382709	0.412181886906127	0.650879396495579\\
0.93	0.0911157389373742	0.417048438424896	0.649384574964843\\
0.93	0.0933612213899392	0.421907721096044	0.647906343724686\\
0.93	0.0956348466427212	0.42675949830445	0.646443996754806\\
0.93	0.0979365930683194	0.431603533204733	0.64499692929596\\
0.93	0.100266436456264	0.436439588755293	0.643564622546847\\
0.93	0.102624350004627	0.441267427752584	0.64214663037443\\
0.93	0.105010304312169	0.446086812865616	0.640742567819433\\
0.93	0.107424267371012	0.450897506670668	0.639352101193731\\
0.929999999999999	0.109866204559871	0.455699271686218	0.637974939582444\\
0.93	0.112336078637817	0.460491870408058	0.636610827579862\\
0.93	0.114833849738607	0.465275065344603	0.635259539104383\\
0.93	0.117359475365564	0.470048619052377	0.633920872153153\\
0.93	0.119912910387022	0.474812294171664	0.632594644371966\\
0.93	0.122494107032347	0.479565853462319	0.631280689329642\\
0.93	0.125103014888515	0.484309059839721	0.629978853399059\\
0.93	0.127739580897283	0.489041676410868	0.628690460824561\\
0.935	0	0	0.710202759607373\\
0.935	1.11327767495586e-05	0.00471862581271274	0.713003157837853\\
0.935	4.46175251031912e-05	0.009446325184051	0.715846486660142\\
0.935	0.000100583311362513	0.0141829653360114	0.718788838379794\\
0.935	0.000179158434668431	0.018928411755669	0.721886092228356\\
0.935	0.000280470402701511	0.02368252819604	0.725200636220892\\
0.935	0.000404645907256436	0.0284451766772965	0.728799085274723\\
0.935	0.000551810799695644	0.0332162174883388	0.732749123983336\\
0.935	0.000722090066287311	0.037995509188729	0.737115575253327\\
0.935	0.000915607803432999	0.0427829086109896	0.741955893907011\\
0.935	0.0011324871927904	0.0475782708632729	0.747315373652289\\
0.935	0.00137285047629673	0.0523814493324038	0.75322242404581\\
0.935	0.00163681893109844	0.057192295687301	0.759684310525652\\
0.935	0.00192451284439304	0.0620106598827802	0.766683748102665\\
0.935	0.00223605148818898	0.0668363901637434	0.774176695511256\\
0.935	0.00257155309398959	0.0716693330697584	0.782091614494493\\
0.935	0.00293113482740716	0.0765093334400312	0.790330346421089\\
0.935	0.00331491276271365	0.0813562344187764	0.79877062752665\\
0.935	0.00372300185733413	0.0862098774609879	0.80727012893036\\
0.935	0.00415551592628969	0.0910701023386145	0.815671782700193\\
0.935	0.00461256761659621	0.0959367471471425	0.82381005350733\\
0.935	0.00509426838162598	0.100809648312589	0.831517746503355\\
0.935	0.00560072845543873	0.105688640598912	0.838632911419452\\
0.935	0.00613205682708918	0.11057355711583	0.845005411376293\\
0.935	0.00668836121491815	0.115464229327074	0.850502769049302\\
0.935	0.00726974804083431	0.120360487059049	0.855014975717657\\
0.935	0.00787632240459385	0.125262158509929	0.858458041096605\\
0.935	0.00850818805808555	0.13016907025918	0.860776163494182\\
0.935	0.00916544737962849	0.135081047277508	0.861942500814224\\
0.935	0.00984820134829017	0.139997912937243	0.86195861464842\\
0.935	0.0105565495182326	0.144919489023162	0.860852735617091\\
0.935	0.0112905899930939	0.149845595743738	0.858677054109651\\
0.935	0.012050419400414	0.154776051742839	0.855504274967458\\
0.935	0.0128361328661109	0.159710674111862	0.851423687967755\\
0.935	0.0136478239890177	0.164649278402306	0.84653700048434\\
0.935	0.0144855848154856	0.169591678638796	0.840954157904249\\
0.935	0.0153495058140643	0.174537687332543	0.834789345403935\\
0.935	0.0162396758502651	0.179487115495259	0.8281573258062\\
0.935	0.0171561821614173	0.184439772653511	0.821170226448695\\
0.935	0.0180991103316243	0.189395466863524	0.813934846725692\\
0.935	0.0190685442668299	0.194354004726437	0.806550519904211\\
0.935	0.0200645661700016	0.199315191403999	0.799107529846305\\
0.935	0.0210872565164405	0.204278830634725	0.791686056496194\\
0.935	0.0221366940292259	0.20924472475049	0.784355603830601\\
0.935	0.023212955654804	0.214212674693577	0.777174850281119\\
0.935	0.0243161165387281	0.219182480034174	0.770191853862644\\
0.935	0.025446250001561	0.224153938988321	0.76344454155446\\
0.935	0.0266034275149466	0.229126848436298	0.756961413910483\\
0.935	0.0277877186778607	0.234101003941464	0.750762400416763\\
0.935	0.0289991911930496	0.239076199769547	0.744859807805784\\
0.935	0.0302379108436654	0.244052228908366	0.739259311518081\\
0.935	0.0315039414701067	0.24902888308801	0.733960949044503\\
0.935	0.0327973449470747	0.254005952801449	0.72896008240726\\
0.935	0.0341181811608519	0.258983227325592	0.72424830511118\\
0.935	0.0354665079868145	0.263960494742775	0.71981427622068\\
0.935	0.0368423812671862	0.268937541962694	0.715644470612628\\
0.935	0.0382458547890421	0.273914154744766	0.711723839838043\\
0.935	0.0396769802625738	0.278890117720924	0.708036382389731\\
0.935	0.0411358072996216	0.283865214418836	0.704565625567163\\
0.935	0.0426223833924861	0.288839227285554	0.701295023640316\\
0.935	0.0441367538930258	0.293811937711588	0.698208278749353\\
0.935	0.0456789619920503	0.298783126055387	0.695289592055349\\
0.935	0.0472490486990192	0.303752571668251	0.692523853197512\\
0.935	0.0488470528220538	0.308720052919643	0.689896776228119\\
0.935	0.0504730109482718	0.313685347222913	0.687394989990282\\
0.935	0.0521269574244539	0.318648231061428	0.685006090465981\\
0.935	0.0538089243380495	0.323608480015096	0.682718662029127\\
0.935	0.0555189414985328	0.328565868787292	0.68052227385347\\
0.935	0.0572570364191156	0.333520171232163	0.678407456997851\\
0.935	0.0590232342988274	0.338471160382329	0.676365666960043\\
0.935	0.0608175580049697	0.34341860847696	0.674389235783156\\
0.935	0.0626400280559541	0.348362286990222	0.672471317134861\\
0.935	0.064490662604533	0.3533019666601	0.670605827172182\\
0.935	0.0663694774214291	0.358237417517579	0.668787383459804\\
0.935	0.0682764858793752	0.363168408916186	0.667011243730728\\
0.935	0.0702116989375697	0.368094709561873	0.665273245863649\\
0.935	0.0721751251265572	0.373016087543259	0.663569750098934\\
0.935	0.0741667705335423	0.377932310362199	0.661897584220061\\
0.935	0.0761866387881432	0.382843144964686	0.66025399218444\\
0.935	0.0782347310485954	0.38774835777208	0.658636586490996\\
0.935	0.0803110459884098	0.392647714712651	0.657043304415871\\
0.935	0.0824155797834956	0.397540981253432	0.65547236812651\\
0.935	0.084548326099754	0.402427922432375	0.653922248592747\\
0.935	0.0867092760811502	0.4073083028908	0.652391633146569\\
0.935	0.0888984183382709	0.412181886906127	0.650879396495578\\
0.935	0.0911157389373742	0.417048438424897	0.649384574964843\\
0.935	0.0933612213899392	0.421907721096044	0.647906343724684\\
0.935	0.0956348466427212	0.42675949830445	0.646443996754807\\
0.935	0.0979365930683194	0.431603533204733	0.644996929295962\\
0.935	0.100266436456264	0.436439588755293	0.643564622546846\\
0.935	0.102624350004627	0.441267427752585	0.64214663037443\\
0.935	0.105010304312169	0.446086812865616	0.640742567819433\\
0.935	0.107424267371012	0.450897506670668	0.639352101193729\\
0.935	0.109866204559871	0.455699271686218	0.637974939582444\\
0.935	0.112336078637817	0.460491870408058	0.636610827579863\\
0.935	0.114833849738607	0.465275065344603	0.635259539104382\\
0.935	0.117359475365564	0.470048619052377	0.633920872153154\\
0.935	0.119912910387023	0.474812294171664	0.632594644371965\\
0.935	0.122494107032347	0.479565853462319	0.631280689329642\\
0.935	0.125103014888515	0.484309059839721	0.629978853399058\\
0.935	0.127739580897283	0.489041676410868	0.62869046082456\\
0.94	0	0	0.710202759607373\\
0.94	1.11327767495586e-05	0.00471862581271274	0.713003157837853\\
0.94	4.46175251031912e-05	0.009446325184051	0.715846486660142\\
0.94	0.000100583311362513	0.0141829653360114	0.718788838379794\\
0.94	0.000179158434668431	0.018928411755669	0.721886092228356\\
0.94	0.000280470402701511	0.02368252819604	0.725200636220892\\
0.94	0.000404645907256436	0.0284451766772965	0.728799085274723\\
0.94	0.000551810799695644	0.0332162174883389	0.732749123983336\\
0.94	0.000722090066287311	0.037995509188729	0.737115575253328\\
0.94	0.000915607803432999	0.0427829086109896	0.741955893907011\\
0.94	0.0011324871927904	0.0475782708632729	0.747315373652289\\
0.94	0.00137285047629673	0.0523814493324038	0.753222424045811\\
0.94	0.00163681893109844	0.057192295687301	0.759684310525652\\
0.94	0.00192451284439304	0.0620106598827802	0.766683748102665\\
0.94	0.00223605148818899	0.0668363901637434	0.774176695511257\\
0.94	0.00257155309398959	0.0716693330697584	0.782091614494494\\
0.94	0.00293113482740716	0.0765093334400312	0.79033034642109\\
0.94	0.00331491276271365	0.0813562344187764	0.79877062752665\\
0.94	0.00372300185733413	0.0862098774609879	0.807270128930359\\
0.94	0.00415551592628969	0.0910701023386145	0.815671782700194\\
0.94	0.00461256761659621	0.0959367471471425	0.823810053507331\\
0.94	0.00509426838162598	0.100809648312589	0.831517746503354\\
0.94	0.00560072845543873	0.105688640598912	0.838632911419452\\
0.94	0.00613205682708918	0.11057355711583	0.845005411376294\\
0.94	0.00668836121491815	0.115464229327074	0.850502769049302\\
0.94	0.00726974804083431	0.120360487059049	0.855014975717657\\
0.94	0.00787632240459385	0.125262158509929	0.858458041096607\\
0.94	0.00850818805808555	0.13016907025918	0.860776163494182\\
0.94	0.00916544737962849	0.135081047277508	0.861942500814224\\
0.94	0.00984820134829017	0.139997912937243	0.86195861464842\\
0.94	0.0105565495182326	0.144919489023162	0.860852735617091\\
0.94	0.0112905899930939	0.149845595743738	0.858677054109652\\
0.94	0.012050419400414	0.154776051742839	0.855504274967457\\
0.94	0.0128361328661109	0.159710674111862	0.851423687967752\\
0.94	0.0136478239890177	0.164649278402306	0.846537000484343\\
0.94	0.0144855848154856	0.169591678638796	0.840954157904246\\
0.94	0.0153495058140643	0.174537687332543	0.834789345403937\\
0.94	0.0162396758502651	0.179487115495259	0.828157325806198\\
0.94	0.0171561821614173	0.184439772653511	0.821170226448693\\
0.94	0.0180991103316243	0.189395466863524	0.81393484672569\\
0.94	0.0190685442668299	0.194354004726437	0.806550519904213\\
0.94	0.0200645661700016	0.199315191403999	0.799107529846305\\
0.94	0.0210872565164405	0.204278830634725	0.791686056496192\\
0.94	0.0221366940292259	0.20924472475049	0.784355603830601\\
0.94	0.023212955654804	0.214212674693577	0.777174850281117\\
0.94	0.0243161165387281	0.219182480034174	0.770191853862643\\
0.94	0.025446250001561	0.224153938988321	0.763444541554458\\
0.94	0.0266034275149466	0.229126848436298	0.756961413910482\\
0.94	0.0277877186778607	0.234101003941464	0.750762400416762\\
0.94	0.0289991911930496	0.239076199769547	0.744859807805786\\
0.94	0.0302379108436654	0.244052228908366	0.73925931151808\\
0.94	0.0315039414701067	0.24902888308801	0.733960949044502\\
0.94	0.0327973449470747	0.254005952801449	0.728960082407261\\
0.94	0.0341181811608519	0.258983227325592	0.724248305111178\\
0.94	0.0354665079868145	0.263960494742775	0.71981427622068\\
0.94	0.0368423812671862	0.268937541962694	0.715644470612629\\
0.94	0.0382458547890421	0.273914154744766	0.711723839838042\\
0.94	0.0396769802625738	0.278890117720924	0.70803638238973\\
0.94	0.0411358072996216	0.283865214418836	0.704565625567164\\
0.94	0.0426223833924861	0.288839227285554	0.701295023640316\\
0.94	0.0441367538930258	0.293811937711588	0.698208278749354\\
0.94	0.0456789619920503	0.298783126055387	0.695289592055349\\
0.94	0.0472490486990192	0.303752571668251	0.692523853197512\\
0.94	0.0488470528220538	0.308720052919643	0.689896776228119\\
0.94	0.0504730109482718	0.313685347222913	0.68739498999028\\
0.94	0.0521269574244539	0.318648231061427	0.685006090465981\\
0.94	0.0538089243380495	0.323608480015096	0.682718662029128\\
0.94	0.0555189414985328	0.328565868787292	0.680522273853469\\
0.94	0.0572570364191156	0.333520171232163	0.67840745699785\\
0.94	0.0590232342988274	0.338471160382329	0.676365666960043\\
0.94	0.0608175580049697	0.34341860847696	0.674389235783155\\
0.94	0.0626400280559541	0.348362286990222	0.672471317134862\\
0.94	0.064490662604533	0.3533019666601	0.670605827172183\\
0.94	0.0663694774214291	0.358237417517579	0.668787383459804\\
0.94	0.0682764858793752	0.363168408916186	0.667011243730728\\
0.94	0.0702116989375697	0.368094709561873	0.665273245863648\\
0.94	0.0721751251265572	0.373016087543259	0.663569750098931\\
0.94	0.0741667705335422	0.377932310362199	0.661897584220062\\
0.94	0.0761866387881432	0.382843144964686	0.660253992184443\\
0.94	0.0782347310485954	0.38774835777208	0.658636586490996\\
0.94	0.0803110459884098	0.392647714712651	0.657043304415872\\
0.94	0.0824155797834956	0.397540981253432	0.65547236812651\\
0.94	0.084548326099754	0.402427922432375	0.653922248592745\\
0.94	0.0867092760811502	0.407308302890799	0.652391633146571\\
0.94	0.0888984183382708	0.412181886906127	0.650879396495579\\
0.94	0.0911157389373742	0.417048438424896	0.649384574964844\\
0.94	0.0933612213899392	0.421907721096044	0.647906343724685\\
0.94	0.0956348466427212	0.42675949830445	0.646443996754806\\
0.94	0.0979365930683194	0.431603533204733	0.644996929295961\\
0.94	0.100266436456264	0.436439588755293	0.643564622546847\\
0.94	0.102624350004627	0.441267427752585	0.642146630374431\\
0.94	0.105010304312169	0.446086812865616	0.640742567819433\\
0.94	0.107424267371012	0.450897506670668	0.639352101193731\\
0.94	0.109866204559871	0.455699271686218	0.637974939582443\\
0.94	0.112336078637817	0.460491870408058	0.636610827579863\\
0.94	0.114833849738607	0.465275065344603	0.635259539104383\\
0.94	0.117359475365564	0.470048619052377	0.633920872153154\\
0.94	0.119912910387022	0.474812294171664	0.632594644371965\\
0.94	0.122494107032347	0.479565853462319	0.63128068932964\\
0.94	0.125103014888515	0.484309059839721	0.629978853399058\\
0.94	0.127739580897283	0.489041676410868	0.628690460824559\\
0.945	0	0	0.710202759607373\\
0.945	1.11327767495586e-05	0.00471862581271275	0.713003157837853\\
0.945	4.46175251031912e-05	0.009446325184051	0.715846486660142\\
0.945	0.000100583311362513	0.0141829653360114	0.718788838379794\\
0.945	0.000179158434668431	0.018928411755669	0.721886092228356\\
0.945	0.000280470402701511	0.02368252819604	0.725200636220892\\
0.945	0.000404645907256436	0.0284451766772965	0.728799085274723\\
0.945	0.000551810799695644	0.0332162174883388	0.732749123983336\\
0.945	0.000722090066287311	0.037995509188729	0.737115575253327\\
0.945	0.000915607803432999	0.0427829086109896	0.741955893907011\\
0.945	0.0011324871927904	0.0475782708632729	0.747315373652289\\
0.945	0.00137285047629673	0.0523814493324038	0.753222424045811\\
0.945	0.00163681893109844	0.057192295687301	0.759684310525652\\
0.945	0.00192451284439304	0.0620106598827802	0.766683748102665\\
0.945	0.00223605148818899	0.0668363901637434	0.774176695511256\\
0.945	0.00257155309398959	0.0716693330697584	0.782091614494494\\
0.945	0.00293113482740716	0.0765093334400312	0.79033034642109\\
0.945	0.00331491276271365	0.0813562344187764	0.79877062752665\\
0.945	0.00372300185733413	0.0862098774609879	0.807270128930359\\
0.945	0.00415551592628969	0.0910701023386145	0.815671782700193\\
0.945	0.00461256761659621	0.0959367471471425	0.823810053507331\\
0.945	0.00509426838162598	0.100809648312589	0.831517746503354\\
0.945	0.00560072845543873	0.105688640598912	0.838632911419452\\
0.945	0.00613205682708918	0.11057355711583	0.845005411376293\\
0.945	0.00668836121491816	0.115464229327074	0.850502769049302\\
0.945	0.00726974804083431	0.120360487059049	0.855014975717656\\
0.945	0.00787632240459385	0.125262158509929	0.858458041096606\\
0.945	0.00850818805808555	0.13016907025918	0.860776163494181\\
0.945	0.00916544737962849	0.135081047277508	0.861942500814225\\
0.945	0.00984820134829017	0.139997912937243	0.861958614648422\\
0.945	0.0105565495182326	0.144919489023162	0.860852735617092\\
0.945	0.0112905899930939	0.149845595743738	0.858677054109652\\
0.945	0.012050419400414	0.154776051742839	0.855504274967456\\
0.945	0.0128361328661109	0.159710674111862	0.851423687967755\\
0.945	0.0136478239890177	0.164649278402306	0.84653700048434\\
0.945	0.0144855848154856	0.169591678638796	0.840954157904247\\
0.945	0.0153495058140643	0.174537687332543	0.834789345403935\\
0.945	0.0162396758502651	0.179487115495259	0.8281573258062\\
0.945	0.0171561821614173	0.184439772653511	0.821170226448694\\
0.945	0.0180991103316243	0.189395466863524	0.813934846725692\\
0.945	0.0190685442668299	0.194354004726437	0.806550519904214\\
0.945	0.0200645661700016	0.199315191403999	0.799107529846302\\
0.945	0.0210872565164405	0.204278830634725	0.791686056496193\\
0.945	0.0221366940292259	0.20924472475049	0.784355603830601\\
0.945	0.023212955654804	0.214212674693577	0.777174850281118\\
0.945	0.0243161165387281	0.219182480034174	0.770191853862644\\
0.945	0.025446250001561	0.224153938988321	0.763444541554461\\
0.945	0.0266034275149466	0.229126848436298	0.756961413910482\\
0.945	0.0277877186778607	0.234101003941464	0.750762400416764\\
0.945	0.0289991911930496	0.239076199769547	0.744859807805786\\
0.945	0.0302379108436654	0.244052228908366	0.73925931151808\\
0.945	0.0315039414701067	0.24902888308801	0.733960949044503\\
0.945	0.0327973449470747	0.254005952801449	0.72896008240726\\
0.945	0.0341181811608519	0.258983227325591	0.724248305111179\\
0.945	0.0354665079868145	0.263960494742775	0.719814276220681\\
0.945	0.0368423812671862	0.268937541962694	0.715644470612629\\
0.945	0.0382458547890421	0.273914154744766	0.711723839838043\\
0.945	0.0396769802625738	0.278890117720924	0.70803638238973\\
0.945	0.0411358072996216	0.283865214418836	0.704565625567163\\
0.945	0.0426223833924861	0.288839227285554	0.701295023640316\\
0.945	0.0441367538930258	0.293811937711588	0.698208278749354\\
0.945	0.0456789619920503	0.298783126055387	0.69528959205535\\
0.945	0.0472490486990192	0.303752571668251	0.692523853197511\\
0.945	0.0488470528220538	0.308720052919643	0.689896776228118\\
0.945	0.0504730109482718	0.313685347222913	0.687394989990282\\
0.945	0.0521269574244539	0.318648231061428	0.685006090465981\\
0.945	0.0538089243380495	0.323608480015096	0.682718662029127\\
0.945	0.0555189414985328	0.328565868787292	0.68052227385347\\
0.945	0.0572570364191156	0.333520171232163	0.678407456997851\\
0.945	0.0590232342988274	0.338471160382329	0.676365666960043\\
0.945	0.0608175580049697	0.34341860847696	0.674389235783155\\
0.945	0.0626400280559541	0.348362286990222	0.672471317134862\\
0.945	0.064490662604533	0.3533019666601	0.670605827172183\\
0.945	0.066369477421429	0.358237417517579	0.668787383459804\\
0.945	0.0682764858793752	0.363168408916186	0.667011243730727\\
0.945	0.0702116989375697	0.368094709561873	0.665273245863649\\
0.945	0.0721751251265572	0.373016087543259	0.663569750098932\\
0.945	0.0741667705335422	0.377932310362199	0.66189758422006\\
0.945	0.0761866387881432	0.382843144964686	0.660253992184444\\
0.945	0.0782347310485954	0.38774835777208	0.658636586490997\\
0.945	0.0803110459884098	0.392647714712651	0.65704330441587\\
0.945	0.0824155797834956	0.397540981253432	0.655472368126509\\
0.945	0.084548326099754	0.402427922432375	0.653922248592745\\
0.945	0.0867092760811502	0.407308302890799	0.65239163314657\\
0.945	0.0888984183382709	0.412181886906127	0.650879396495579\\
0.945	0.0911157389373742	0.417048438424896	0.649384574964844\\
0.945	0.0933612213899392	0.421907721096044	0.647906343724685\\
0.945	0.0956348466427212	0.42675949830445	0.646443996754806\\
0.945	0.0979365930683194	0.431603533204733	0.644996929295961\\
0.945	0.100266436456264	0.436439588755293	0.643564622546846\\
0.945	0.102624350004627	0.441267427752584	0.642146630374431\\
0.945	0.105010304312169	0.446086812865616	0.640742567819433\\
0.945	0.107424267371012	0.450897506670668	0.63935210119373\\
0.945	0.109866204559871	0.455699271686218	0.637974939582443\\
0.945	0.112336078637817	0.460491870408058	0.636610827579862\\
0.945	0.114833849738607	0.465275065344603	0.635259539104383\\
0.945	0.117359475365564	0.470048619052377	0.633920872153154\\
0.945	0.119912910387022	0.474812294171664	0.632594644371966\\
0.945	0.122494107032347	0.479565853462319	0.631280689329643\\
0.945	0.125103014888515	0.484309059839721	0.629978853399058\\
0.945	0.127739580897283	0.489041676410868	0.628690460824556\\
0.95	0	0	0.710202759607373\\
0.95	1.11327767495586e-05	0.00471862581271274	0.713003157837853\\
0.95	4.46175251031912e-05	0.009446325184051	0.715846486660142\\
0.95	0.000100583311362513	0.0141829653360114	0.718788838379794\\
0.95	0.000179158434668431	0.018928411755669	0.721886092228356\\
0.95	0.000280470402701511	0.02368252819604	0.725200636220892\\
0.95	0.000404645907256436	0.0284451766772965	0.728799085274723\\
0.95	0.000551810799695644	0.0332162174883389	0.732749123983336\\
0.95	0.000722090066287311	0.037995509188729	0.737115575253328\\
0.95	0.000915607803432999	0.0427829086109896	0.741955893907011\\
0.95	0.0011324871927904	0.0475782708632729	0.747315373652289\\
0.95	0.00137285047629673	0.0523814493324038	0.753222424045811\\
0.95	0.00163681893109844	0.057192295687301	0.759684310525652\\
0.95	0.00192451284439304	0.0620106598827802	0.766683748102665\\
0.95	0.00223605148818898	0.0668363901637434	0.774176695511256\\
0.95	0.00257155309398959	0.0716693330697584	0.782091614494493\\
0.95	0.00293113482740716	0.0765093334400312	0.79033034642109\\
0.95	0.00331491276271365	0.0813562344187764	0.79877062752665\\
0.95	0.00372300185733413	0.0862098774609879	0.807270128930359\\
0.95	0.00415551592628969	0.0910701023386145	0.815671782700193\\
0.95	0.00461256761659621	0.0959367471471425	0.823810053507331\\
0.95	0.00509426838162598	0.100809648312589	0.831517746503353\\
0.95	0.00560072845543873	0.105688640598912	0.838632911419452\\
0.95	0.00613205682708918	0.11057355711583	0.845005411376293\\
0.95	0.00668836121491815	0.115464229327074	0.850502769049302\\
0.95	0.00726974804083431	0.120360487059049	0.855014975717657\\
0.95	0.00787632240459385	0.125262158509929	0.858458041096605\\
0.95	0.00850818805808555	0.13016907025918	0.860776163494182\\
0.95	0.00916544737962849	0.135081047277508	0.861942500814224\\
0.95	0.00984820134829017	0.139997912937243	0.86195861464842\\
0.95	0.0105565495182326	0.144919489023162	0.860852735617091\\
0.95	0.0112905899930939	0.149845595743738	0.85867705410965\\
0.95	0.012050419400414	0.154776051742839	0.855504274967458\\
0.95	0.0128361328661109	0.159710674111862	0.851423687967754\\
0.95	0.0136478239890177	0.164649278402306	0.846537000484339\\
0.95	0.0144855848154856	0.169591678638796	0.840954157904248\\
0.95	0.0153495058140643	0.174537687332543	0.834789345403935\\
0.95	0.0162396758502651	0.179487115495259	0.828157325806202\\
0.95	0.0171561821614173	0.184439772653511	0.821170226448693\\
0.95	0.0180991103316243	0.189395466863524	0.813934846725692\\
0.95	0.0190685442668299	0.194354004726437	0.806550519904211\\
0.95	0.0200645661700016	0.199315191403999	0.799107529846304\\
0.95	0.0210872565164405	0.204278830634725	0.791686056496195\\
0.95	0.0221366940292259	0.20924472475049	0.784355603830602\\
0.95	0.023212955654804	0.214212674693577	0.77717485028112\\
0.95	0.0243161165387281	0.219182480034174	0.770191853862646\\
0.95	0.025446250001561	0.224153938988321	0.763444541554458\\
0.95	0.0266034275149466	0.229126848436298	0.756961413910481\\
0.95	0.0277877186778607	0.234101003941464	0.750762400416766\\
0.95	0.0289991911930496	0.239076199769547	0.744859807805787\\
0.95	0.0302379108436654	0.244052228908366	0.73925931151808\\
0.95	0.0315039414701067	0.24902888308801	0.733960949044503\\
0.95	0.0327973449470747	0.254005952801449	0.728960082407261\\
0.95	0.0341181811608519	0.258983227325592	0.724248305111177\\
0.95	0.0354665079868145	0.263960494742775	0.719814276220679\\
0.95	0.0368423812671862	0.268937541962694	0.715644470612631\\
0.95	0.0382458547890421	0.273914154744766	0.711723839838042\\
0.95	0.0396769802625738	0.278890117720924	0.708036382389729\\
0.95	0.0411358072996216	0.283865214418836	0.704565625567164\\
0.95	0.0426223833924861	0.288839227285554	0.701295023640317\\
0.95	0.0441367538930258	0.293811937711588	0.698208278749353\\
0.95	0.0456789619920503	0.298783126055387	0.695289592055347\\
0.95	0.0472490486990192	0.303752571668251	0.692523853197513\\
0.95	0.0488470528220537	0.308720052919643	0.68989677622812\\
0.95	0.0504730109482718	0.313685347222913	0.687394989990281\\
0.95	0.0521269574244539	0.318648231061428	0.68500609046598\\
0.95	0.0538089243380495	0.323608480015096	0.682718662029128\\
0.95	0.0555189414985328	0.328565868787292	0.68052227385347\\
0.95	0.0572570364191156	0.333520171232163	0.67840745699785\\
0.95	0.0590232342988274	0.338471160382329	0.676365666960044\\
0.95	0.0608175580049697	0.34341860847696	0.674389235783156\\
0.95	0.0626400280559541	0.348362286990222	0.672471317134861\\
0.95	0.064490662604533	0.3533019666601	0.670605827172182\\
0.95	0.0663694774214291	0.358237417517579	0.668787383459805\\
0.95	0.0682764858793752	0.363168408916186	0.667011243730729\\
0.95	0.0702116989375697	0.368094709561873	0.665273245863648\\
0.95	0.0721751251265572	0.373016087543259	0.663569750098932\\
0.95	0.0741667705335422	0.377932310362199	0.66189758422006\\
0.95	0.0761866387881432	0.382843144964686	0.660253992184442\\
0.95	0.0782347310485954	0.38774835777208	0.658636586490997\\
0.95	0.0803110459884098	0.392647714712651	0.657043304415872\\
0.95	0.0824155797834956	0.397540981253432	0.655472368126509\\
0.95	0.084548326099754	0.402427922432375	0.653922248592745\\
0.95	0.0867092760811502	0.407308302890799	0.65239163314657\\
0.95	0.0888984183382709	0.412181886906127	0.650879396495579\\
0.95	0.0911157389373742	0.417048438424896	0.649384574964843\\
0.95	0.0933612213899392	0.421907721096044	0.647906343724686\\
0.95	0.0956348466427212	0.42675949830445	0.646443996754806\\
0.95	0.0979365930683194	0.431603533204733	0.644996929295961\\
0.95	0.100266436456264	0.436439588755293	0.643564622546847\\
0.95	0.102624350004627	0.441267427752585	0.64214663037443\\
0.95	0.105010304312169	0.446086812865616	0.640742567819433\\
0.95	0.107424267371012	0.450897506670668	0.639352101193731\\
0.95	0.109866204559871	0.455699271686218	0.637974939582442\\
0.95	0.112336078637817	0.460491870408058	0.636610827579862\\
0.95	0.114833849738607	0.465275065344603	0.635259539104382\\
0.95	0.117359475365564	0.470048619052377	0.633920872153155\\
0.95	0.119912910387023	0.474812294171664	0.632594644371964\\
0.95	0.122494107032347	0.479565853462319	0.631280689329641\\
0.95	0.125103014888515	0.484309059839721	0.629978853399061\\
0.95	0.127739580897283	0.489041676410868	0.628690460824557\\
0.955	0	0	0.710202759607373\\
0.955	1.11327767495586e-05	0.00471862581271275	0.713003157837853\\
0.955	4.46175251031912e-05	0.009446325184051	0.715846486660142\\
0.955	0.000100583311362513	0.0141829653360114	0.718788838379794\\
0.955	0.000179158434668431	0.018928411755669	0.721886092228356\\
0.955	0.000280470402701511	0.02368252819604	0.725200636220892\\
0.955	0.000404645907256436	0.0284451766772965	0.728799085274723\\
0.955	0.000551810799695644	0.0332162174883388	0.732749123983336\\
0.955	0.000722090066287311	0.037995509188729	0.737115575253327\\
0.955	0.000915607803432999	0.0427829086109896	0.741955893907011\\
0.955	0.0011324871927904	0.0475782708632729	0.747315373652289\\
0.955	0.00137285047629673	0.0523814493324038	0.75322242404581\\
0.955	0.00163681893109844	0.057192295687301	0.759684310525652\\
0.955	0.00192451284439304	0.0620106598827802	0.766683748102665\\
0.955	0.00223605148818899	0.0668363901637434	0.774176695511256\\
0.955	0.00257155309398959	0.0716693330697584	0.782091614494493\\
0.955	0.00293113482740716	0.0765093334400312	0.79033034642109\\
0.955	0.00331491276271365	0.0813562344187764	0.79877062752665\\
0.955	0.00372300185733413	0.0862098774609879	0.807270128930359\\
0.955	0.00415551592628969	0.0910701023386145	0.815671782700193\\
0.955	0.00461256761659621	0.0959367471471425	0.823810053507331\\
0.955	0.00509426838162598	0.100809648312589	0.831517746503354\\
0.955	0.00560072845543873	0.105688640598912	0.838632911419452\\
0.955	0.00613205682708918	0.11057355711583	0.845005411376293\\
0.955	0.00668836121491816	0.115464229327074	0.850502769049301\\
0.955	0.00726974804083431	0.120360487059049	0.855014975717657\\
0.955	0.00787632240459385	0.125262158509929	0.858458041096606\\
0.955	0.00850818805808555	0.13016907025918	0.860776163494183\\
0.955	0.00916544737962849	0.135081047277508	0.861942500814224\\
0.955	0.00984820134829017	0.139997912937243	0.861958614648418\\
0.955	0.0105565495182326	0.144919489023162	0.860852735617089\\
0.955	0.0112905899930939	0.149845595743738	0.858677054109649\\
0.955	0.012050419400414	0.154776051742839	0.855504274967457\\
0.955	0.0128361328661109	0.159710674111862	0.851423687967756\\
0.955	0.0136478239890177	0.164649278402306	0.84653700048434\\
0.955	0.0144855848154856	0.169591678638796	0.840954157904252\\
0.955	0.0153495058140643	0.174537687332543	0.834789345403936\\
0.955	0.0162396758502651	0.179487115495259	0.828157325806199\\
0.955	0.0171561821614173	0.184439772653511	0.821170226448695\\
0.955	0.0180991103316243	0.189395466863524	0.813934846725693\\
0.955	0.0190685442668299	0.194354004726437	0.806550519904213\\
0.955	0.0200645661700016	0.199315191403999	0.799107529846306\\
0.955	0.0210872565164405	0.204278830634725	0.791686056496194\\
0.955	0.0221366940292259	0.20924472475049	0.784355603830603\\
0.955	0.023212955654804	0.214212674693577	0.777174850281119\\
0.955	0.0243161165387281	0.219182480034174	0.770191853862641\\
0.955	0.025446250001561	0.224153938988321	0.763444541554458\\
0.955	0.0266034275149466	0.229126848436298	0.756961413910484\\
0.955	0.0277877186778607	0.234101003941464	0.750762400416764\\
0.955	0.0289991911930496	0.239076199769547	0.744859807805784\\
0.955	0.0302379108436654	0.244052228908366	0.739259311518078\\
0.955	0.0315039414701067	0.24902888308801	0.733960949044504\\
0.955	0.0327973449470747	0.254005952801449	0.728960082407259\\
0.955	0.0341181811608519	0.258983227325591	0.724248305111176\\
0.955	0.0354665079868145	0.263960494742774	0.719814276220681\\
0.955	0.0368423812671862	0.268937541962694	0.71564447061263\\
0.955	0.0382458547890421	0.273914154744766	0.711723839838042\\
0.955	0.0396769802625738	0.278890117720924	0.708036382389731\\
0.955	0.0411358072996216	0.283865214418836	0.704565625567164\\
0.955	0.0426223833924861	0.288839227285554	0.701295023640316\\
0.955	0.0441367538930258	0.293811937711588	0.698208278749354\\
0.955	0.0456789619920503	0.298783126055387	0.69528959205535\\
0.955	0.0472490486990192	0.303752571668251	0.692523853197511\\
0.955	0.0488470528220538	0.308720052919643	0.689896776228119\\
0.955	0.0504730109482718	0.313685347222913	0.687394989990282\\
0.955	0.0521269574244539	0.318648231061428	0.68500609046598\\
0.955	0.0538089243380495	0.323608480015096	0.682718662029128\\
0.955	0.0555189414985328	0.328565868787292	0.68052227385347\\
0.955	0.0572570364191156	0.333520171232163	0.67840745699785\\
0.955	0.0590232342988274	0.338471160382329	0.676365666960043\\
0.955	0.0608175580049697	0.34341860847696	0.674389235783155\\
0.955	0.0626400280559541	0.348362286990222	0.672471317134862\\
0.955	0.064490662604533	0.3533019666601	0.670605827172182\\
0.955	0.0663694774214291	0.358237417517579	0.668787383459804\\
0.955	0.0682764858793752	0.363168408916186	0.667011243730728\\
0.955	0.0702116989375697	0.368094709561873	0.665273245863649\\
0.955	0.0721751251265572	0.373016087543259	0.663569750098932\\
0.955	0.0741667705335422	0.377932310362199	0.661897584220059\\
0.955	0.0761866387881432	0.382843144964686	0.660253992184442\\
0.955	0.0782347310485954	0.38774835777208	0.658636586490997\\
0.955	0.0803110459884098	0.392647714712651	0.657043304415871\\
0.955	0.0824155797834956	0.397540981253432	0.655472368126509\\
0.955	0.084548326099754	0.402427922432375	0.653922248592746\\
0.955	0.0867092760811502	0.407308302890799	0.652391633146571\\
0.955	0.0888984183382709	0.412181886906127	0.65087939649558\\
0.955	0.0911157389373742	0.417048438424897	0.649384574964843\\
0.955	0.0933612213899392	0.421907721096044	0.647906343724684\\
0.955	0.0956348466427212	0.42675949830445	0.646443996754807\\
0.955	0.0979365930683194	0.431603533204733	0.644996929295962\\
0.955	0.100266436456264	0.436439588755293	0.643564622546846\\
0.955	0.102624350004627	0.441267427752584	0.642146630374431\\
0.955	0.105010304312169	0.446086812865616	0.640742567819433\\
0.955	0.107424267371012	0.450897506670668	0.639352101193731\\
0.955	0.109866204559871	0.455699271686218	0.637974939582443\\
0.955	0.112336078637817	0.460491870408058	0.636610827579862\\
0.955	0.114833849738607	0.465275065344603	0.635259539104381\\
0.955	0.117359475365564	0.470048619052377	0.633920872153154\\
0.955	0.119912910387022	0.474812294171664	0.632594644371967\\
0.955	0.122494107032347	0.479565853462319	0.631280689329639\\
0.955	0.125103014888515	0.484309059839721	0.629978853399059\\
0.955	0.127739580897283	0.489041676410868	0.628690460824564\\
0.96	0	0	0.710202759607373\\
0.96	1.11327767495586e-05	0.00471862581271275	0.713003157837853\\
0.96	4.46175251031912e-05	0.009446325184051	0.715846486660142\\
0.96	0.000100583311362513	0.0141829653360114	0.718788838379794\\
0.96	0.000179158434668431	0.018928411755669	0.721886092228356\\
0.96	0.000280470402701511	0.02368252819604	0.725200636220892\\
0.96	0.000404645907256436	0.0284451766772965	0.728799085274723\\
0.96	0.000551810799695644	0.0332162174883389	0.732749123983336\\
0.96	0.000722090066287311	0.037995509188729	0.737115575253328\\
0.96	0.000915607803432999	0.0427829086109896	0.741955893907011\\
0.96	0.0011324871927904	0.0475782708632729	0.747315373652289\\
0.96	0.00137285047629673	0.0523814493324038	0.753222424045811\\
0.96	0.00163681893109844	0.057192295687301	0.759684310525652\\
0.96	0.00192451284439304	0.0620106598827802	0.766683748102665\\
0.96	0.00223605148818899	0.0668363901637434	0.774176695511257\\
0.96	0.00257155309398959	0.0716693330697584	0.782091614494494\\
0.96	0.00293113482740716	0.0765093334400312	0.79033034642109\\
0.96	0.00331491276271365	0.0813562344187764	0.79877062752665\\
0.96	0.00372300185733413	0.0862098774609879	0.807270128930359\\
0.96	0.00415551592628969	0.0910701023386145	0.815671782700192\\
0.96	0.00461256761659621	0.0959367471471425	0.82381005350733\\
0.96	0.00509426838162598	0.100809648312589	0.831517746503354\\
0.96	0.00560072845543873	0.105688640598912	0.838632911419452\\
0.96	0.00613205682708918	0.11057355711583	0.845005411376292\\
0.96	0.00668836121491816	0.115464229327074	0.850502769049303\\
0.96	0.00726974804083431	0.120360487059049	0.855014975717657\\
0.96	0.00787632240459385	0.125262158509929	0.858458041096607\\
0.96	0.00850818805808555	0.13016907025918	0.860776163494181\\
0.96	0.00916544737962849	0.135081047277508	0.861942500814225\\
0.96	0.00984820134829017	0.139997912937243	0.861958614648422\\
0.96	0.0105565495182326	0.144919489023162	0.860852735617091\\
0.96	0.0112905899930939	0.149845595743738	0.858677054109651\\
0.96	0.012050419400414	0.154776051742839	0.855504274967457\\
0.96	0.0128361328661109	0.159710674111862	0.851423687967756\\
0.96	0.0136478239890177	0.164649278402306	0.846537000484341\\
0.96	0.0144855848154856	0.169591678638796	0.840954157904249\\
0.96	0.0153495058140643	0.174537687332543	0.834789345403934\\
0.96	0.0162396758502651	0.179487115495259	0.828157325806202\\
0.96	0.0171561821614173	0.184439772653511	0.821170226448696\\
0.96	0.0180991103316243	0.189395466863524	0.813934846725692\\
0.96	0.0190685442668299	0.194354004726437	0.806550519904213\\
0.96	0.0200645661700016	0.199315191403999	0.799107529846303\\
0.96	0.0210872565164405	0.204278830634725	0.791686056496194\\
0.96	0.0221366940292259	0.20924472475049	0.784355603830602\\
0.96	0.023212955654804	0.214212674693577	0.777174850281117\\
0.96	0.0243161165387281	0.219182480034174	0.770191853862644\\
0.96	0.025446250001561	0.224153938988321	0.76344454155446\\
0.96	0.0266034275149466	0.229126848436298	0.756961413910481\\
0.96	0.0277877186778607	0.234101003941464	0.750762400416764\\
0.96	0.0289991911930496	0.239076199769547	0.744859807805787\\
0.96	0.0302379108436654	0.244052228908366	0.73925931151808\\
0.96	0.0315039414701067	0.24902888308801	0.733960949044502\\
0.96	0.0327973449470747	0.254005952801449	0.728960082407259\\
0.96	0.0341181811608519	0.258983227325591	0.724248305111178\\
0.96	0.0354665079868145	0.263960494742775	0.719814276220682\\
0.96	0.0368423812671862	0.268937541962694	0.715644470612629\\
0.96	0.0382458547890421	0.273914154744766	0.711723839838043\\
0.96	0.0396769802625737	0.278890117720924	0.708036382389733\\
0.96	0.0411358072996216	0.283865214418836	0.704565625567164\\
0.96	0.0426223833924861	0.288839227285554	0.701295023640315\\
0.96	0.0441367538930258	0.293811937711588	0.698208278749354\\
0.96	0.0456789619920503	0.298783126055387	0.69528959205535\\
0.96	0.0472490486990192	0.303752571668251	0.692523853197511\\
0.96	0.0488470528220538	0.308720052919643	0.689896776228118\\
0.96	0.0504730109482718	0.313685347222913	0.687394989990282\\
0.96	0.0521269574244539	0.318648231061428	0.68500609046598\\
0.96	0.0538089243380495	0.323608480015096	0.682718662029127\\
0.96	0.0555189414985328	0.328565868787292	0.68052227385347\\
0.96	0.0572570364191156	0.333520171232163	0.678407456997851\\
0.96	0.0590232342988274	0.338471160382329	0.676365666960044\\
0.96	0.0608175580049697	0.34341860847696	0.674389235783155\\
0.96	0.0626400280559541	0.348362286990222	0.672471317134862\\
0.96	0.064490662604533	0.3533019666601	0.670605827172183\\
0.96	0.0663694774214291	0.35823741751758	0.668787383459803\\
0.96	0.0682764858793752	0.363168408916186	0.667011243730728\\
0.96	0.0702116989375697	0.368094709561873	0.66527324586365\\
0.96	0.0721751251265572	0.373016087543259	0.663569750098932\\
0.96	0.0741667705335423	0.377932310362199	0.66189758422006\\
0.96	0.0761866387881432	0.382843144964686	0.660253992184442\\
0.96	0.0782347310485954	0.38774835777208	0.658636586490997\\
0.96	0.0803110459884098	0.392647714712651	0.657043304415871\\
0.96	0.0824155797834956	0.397540981253432	0.655472368126508\\
0.96	0.084548326099754	0.402427922432375	0.653922248592745\\
0.96	0.0867092760811502	0.407308302890799	0.65239163314657\\
0.96	0.0888984183382709	0.412181886906127	0.650879396495579\\
0.96	0.0911157389373742	0.417048438424896	0.649384574964845\\
0.96	0.0933612213899392	0.421907721096044	0.647906343724685\\
0.96	0.0956348466427212	0.42675949830445	0.646443996754805\\
0.96	0.0979365930683194	0.431603533204733	0.644996929295962\\
0.96	0.100266436456264	0.436439588755293	0.643564622546846\\
0.96	0.102624350004627	0.441267427752584	0.642146630374429\\
0.96	0.105010304312169	0.446086812865616	0.640742567819434\\
0.96	0.107424267371012	0.450897506670668	0.639352101193731\\
0.96	0.109866204559871	0.455699271686218	0.637974939582443\\
0.96	0.112336078637817	0.460491870408058	0.636610827579862\\
0.96	0.114833849738607	0.465275065344603	0.635259539104382\\
0.96	0.117359475365564	0.470048619052377	0.633920872153153\\
0.96	0.119912910387022	0.474812294171664	0.632594644371966\\
0.96	0.122494107032347	0.479565853462319	0.631280689329642\\
0.96	0.125103014888515	0.484309059839721	0.62997885339906\\
0.96	0.127739580897283	0.489041676410868	0.628690460824559\\
0.965	0	0	0.710202759607373\\
0.965	1.11327767495586e-05	0.00471862581271275	0.713003157837853\\
0.965	4.46175251031912e-05	0.009446325184051	0.715846486660142\\
0.965	0.000100583311362513	0.0141829653360114	0.718788838379794\\
0.965	0.000179158434668431	0.018928411755669	0.721886092228356\\
0.965	0.000280470402701511	0.02368252819604	0.725200636220892\\
0.965	0.000404645907256436	0.0284451766772965	0.728799085274723\\
0.965	0.000551810799695644	0.0332162174883388	0.732749123983336\\
0.965	0.000722090066287311	0.037995509188729	0.737115575253327\\
0.965	0.000915607803432999	0.0427829086109896	0.741955893907011\\
0.965	0.0011324871927904	0.0475782708632729	0.747315373652289\\
0.965	0.00137285047629673	0.0523814493324038	0.753222424045811\\
0.965	0.00163681893109844	0.057192295687301	0.759684310525652\\
0.965	0.00192451284439304	0.0620106598827802	0.766683748102665\\
0.965	0.00223605148818898	0.0668363901637434	0.774176695511256\\
0.965	0.00257155309398959	0.0716693330697584	0.782091614494494\\
0.965	0.00293113482740716	0.0765093334400312	0.79033034642109\\
0.965	0.00331491276271365	0.0813562344187764	0.79877062752665\\
0.965	0.00372300185733413	0.0862098774609879	0.80727012893036\\
0.965	0.00415551592628969	0.0910701023386145	0.815671782700193\\
0.965	0.00461256761659621	0.0959367471471425	0.82381005350733\\
0.965	0.00509426838162598	0.100809648312589	0.831517746503353\\
0.965	0.00560072845543873	0.105688640598912	0.838632911419452\\
0.965	0.00613205682708918	0.11057355711583	0.845005411376293\\
0.965	0.00668836121491816	0.115464229327074	0.850502769049302\\
0.965	0.00726974804083431	0.120360487059049	0.855014975717656\\
0.965	0.00787632240459385	0.125262158509929	0.858458041096606\\
0.965	0.00850818805808555	0.13016907025918	0.860776163494183\\
0.965	0.00916544737962849	0.135081047277508	0.861942500814224\\
0.965	0.00984820134829017	0.139997912937243	0.861958614648421\\
0.965	0.0105565495182326	0.144919489023162	0.860852735617091\\
0.965	0.0112905899930939	0.149845595743738	0.858677054109651\\
0.965	0.012050419400414	0.154776051742839	0.855504274967458\\
0.965	0.0128361328661109	0.159710674111862	0.851423687967754\\
0.965	0.0136478239890177	0.164649278402306	0.84653700048434\\
0.965	0.0144855848154856	0.169591678638796	0.840954157904246\\
0.965	0.0153495058140643	0.174537687332543	0.834789345403936\\
0.965	0.0162396758502651	0.179487115495259	0.8281573258062\\
0.965	0.0171561821614173	0.184439772653511	0.821170226448694\\
0.965	0.0180991103316243	0.189395466863524	0.813934846725694\\
0.965	0.0190685442668299	0.194354004726437	0.806550519904211\\
0.965	0.0200645661700016	0.199315191403999	0.799107529846303\\
0.965	0.0210872565164405	0.204278830634725	0.791686056496195\\
0.965	0.0221366940292259	0.20924472475049	0.784355603830603\\
0.965	0.023212955654804	0.214212674693577	0.777174850281117\\
0.965	0.0243161165387281	0.219182480034174	0.770191853862644\\
0.965	0.025446250001561	0.224153938988321	0.763444541554459\\
0.965	0.0266034275149466	0.229126848436298	0.756961413910483\\
0.965	0.0277877186778607	0.234101003941464	0.750762400416765\\
0.965	0.0289991911930496	0.239076199769547	0.744859807805784\\
0.965	0.0302379108436654	0.244052228908366	0.739259311518079\\
0.965	0.0315039414701067	0.24902888308801	0.733960949044504\\
0.965	0.0327973449470747	0.254005952801449	0.728960082407258\\
0.965	0.0341181811608519	0.258983227325591	0.724248305111178\\
0.965	0.0354665079868146	0.263960494742775	0.719814276220683\\
0.965	0.0368423812671862	0.268937541962694	0.71564447061263\\
0.965	0.0382458547890421	0.273914154744766	0.711723839838043\\
0.965	0.0396769802625738	0.278890117720924	0.708036382389731\\
0.965	0.0411358072996216	0.283865214418836	0.704565625567163\\
0.965	0.0426223833924861	0.288839227285554	0.701295023640315\\
0.965	0.0441367538930258	0.293811937711588	0.698208278749353\\
0.965	0.0456789619920503	0.298783126055387	0.695289592055348\\
0.965	0.0472490486990192	0.303752571668251	0.692523853197513\\
0.965	0.0488470528220538	0.308720052919643	0.68989677622812\\
0.965	0.0504730109482718	0.313685347222913	0.68739498999028\\
0.965	0.0521269574244539	0.318648231061428	0.685006090465981\\
0.965	0.0538089243380495	0.323608480015097	0.682718662029127\\
0.965	0.0555189414985328	0.328565868787292	0.680522273853468\\
0.965	0.0572570364191156	0.333520171232163	0.67840745699785\\
0.965	0.0590232342988274	0.338471160382329	0.676365666960043\\
0.965	0.0608175580049697	0.34341860847696	0.674389235783156\\
0.965	0.0626400280559541	0.348362286990222	0.672471317134862\\
0.965	0.064490662604533	0.3533019666601	0.670605827172183\\
0.965	0.0663694774214291	0.358237417517579	0.668787383459804\\
0.965	0.0682764858793752	0.363168408916186	0.667011243730728\\
0.965	0.0702116989375697	0.368094709561873	0.665273245863649\\
0.965	0.0721751251265572	0.373016087543259	0.663569750098933\\
0.965	0.0741667705335422	0.377932310362199	0.66189758422006\\
0.965	0.0761866387881432	0.382843144964686	0.660253992184442\\
0.965	0.0782347310485954	0.38774835777208	0.658636586490997\\
0.965	0.0803110459884098	0.392647714712651	0.657043304415871\\
0.965	0.0824155797834956	0.397540981253432	0.655472368126509\\
0.965	0.084548326099754	0.402427922432375	0.653922248592745\\
0.965	0.0867092760811502	0.407308302890799	0.65239163314657\\
0.965	0.0888984183382709	0.412181886906127	0.650879396495579\\
0.965	0.0911157389373742	0.417048438424896	0.649384574964842\\
0.965	0.0933612213899392	0.421907721096044	0.647906343724685\\
0.965	0.0956348466427212	0.42675949830445	0.646443996754806\\
0.965	0.0979365930683194	0.431603533204733	0.644996929295961\\
0.965	0.100266436456264	0.436439588755293	0.643564622546847\\
0.965	0.102624350004627	0.441267427752584	0.64214663037443\\
0.965	0.105010304312169	0.446086812865616	0.640742567819433\\
0.965	0.107424267371012	0.450897506670668	0.639352101193731\\
0.965	0.109866204559871	0.455699271686218	0.637974939582443\\
0.965	0.112336078637817	0.460491870408058	0.636610827579861\\
0.965	0.114833849738607	0.465275065344603	0.635259539104383\\
0.965	0.117359475365564	0.470048619052377	0.633920872153155\\
0.965	0.119912910387023	0.474812294171664	0.632594644371965\\
0.965	0.122494107032347	0.479565853462319	0.631280689329639\\
0.965	0.125103014888515	0.484309059839721	0.629978853399059\\
0.965	0.127739580897283	0.489041676410868	0.628690460824564\\
0.97	0	0	0.710202759607373\\
0.97	1.11327767495586e-05	0.00471862581271275	0.713003157837853\\
0.97	4.46175251031912e-05	0.009446325184051	0.715846486660142\\
0.97	0.000100583311362513	0.0141829653360114	0.718788838379794\\
0.97	0.000179158434668431	0.018928411755669	0.721886092228356\\
0.97	0.000280470402701511	0.02368252819604	0.725200636220892\\
0.97	0.000404645907256436	0.0284451766772965	0.728799085274723\\
0.97	0.000551810799695644	0.0332162174883388	0.732749123983336\\
0.97	0.000722090066287311	0.037995509188729	0.737115575253327\\
0.97	0.000915607803432999	0.0427829086109896	0.741955893907011\\
0.97	0.0011324871927904	0.047578270863273	0.747315373652289\\
0.97	0.00137285047629673	0.0523814493324038	0.753222424045811\\
0.97	0.00163681893109844	0.057192295687301	0.759684310525652\\
0.97	0.00192451284439304	0.0620106598827802	0.766683748102665\\
0.97	0.00223605148818898	0.0668363901637434	0.774176695511257\\
0.97	0.00257155309398959	0.0716693330697584	0.782091614494494\\
0.97	0.00293113482740716	0.0765093334400312	0.79033034642109\\
0.97	0.00331491276271365	0.0813562344187764	0.79877062752665\\
0.97	0.00372300185733413	0.0862098774609879	0.807270128930359\\
0.97	0.00415551592628969	0.0910701023386145	0.815671782700194\\
0.97	0.00461256761659621	0.0959367471471424	0.823810053507331\\
0.97	0.00509426838162598	0.100809648312589	0.831517746503354\\
0.97	0.00560072845543873	0.105688640598912	0.838632911419451\\
0.97	0.00613205682708918	0.11057355711583	0.845005411376294\\
0.97	0.00668836121491816	0.115464229327074	0.850502769049301\\
0.97	0.00726974804083431	0.120360487059049	0.855014975717657\\
0.97	0.00787632240459385	0.125262158509929	0.858458041096605\\
0.97	0.00850818805808555	0.13016907025918	0.860776163494183\\
0.97	0.00916544737962849	0.135081047277508	0.861942500814224\\
0.97	0.00984820134829017	0.139997912937243	0.86195861464842\\
0.97	0.0105565495182326	0.144919489023162	0.860852735617091\\
0.97	0.0112905899930939	0.149845595743738	0.85867705410965\\
0.97	0.012050419400414	0.154776051742839	0.855504274967459\\
0.97	0.0128361328661109	0.159710674111862	0.851423687967752\\
0.97	0.0136478239890177	0.164649278402306	0.846537000484341\\
0.97	0.0144855848154856	0.169591678638796	0.840954157904248\\
0.97	0.0153495058140643	0.174537687332543	0.834789345403934\\
0.97	0.0162396758502651	0.179487115495259	0.828157325806202\\
0.97	0.0171561821614173	0.184439772653511	0.821170226448696\\
0.97	0.0180991103316243	0.189395466863524	0.81393484672569\\
0.97	0.0190685442668299	0.194354004726437	0.80655051990421\\
0.97	0.0200645661700016	0.199315191403999	0.799107529846306\\
0.97	0.0210872565164405	0.204278830634725	0.791686056496195\\
0.97	0.0221366940292259	0.20924472475049	0.784355603830601\\
0.97	0.023212955654804	0.214212674693577	0.777174850281118\\
0.97	0.0243161165387281	0.219182480034174	0.770191853862644\\
0.97	0.025446250001561	0.224153938988321	0.763444541554458\\
0.97	0.0266034275149466	0.229126848436298	0.756961413910483\\
0.97	0.0277877186778607	0.234101003941464	0.750762400416764\\
0.97	0.0289991911930496	0.239076199769547	0.744859807805784\\
0.97	0.0302379108436654	0.244052228908366	0.739259311518079\\
0.97	0.0315039414701067	0.24902888308801	0.733960949044503\\
0.97	0.0327973449470747	0.254005952801449	0.728960082407261\\
0.97	0.0341181811608519	0.258983227325592	0.724248305111178\\
0.97	0.0354665079868145	0.263960494742774	0.719814276220681\\
0.97	0.0368423812671862	0.268937541962694	0.715644470612631\\
0.97	0.0382458547890421	0.273914154744766	0.711723839838042\\
0.97	0.0396769802625738	0.278890117720924	0.70803638238973\\
0.97	0.0411358072996216	0.283865214418836	0.704565625567163\\
0.97	0.0426223833924861	0.288839227285554	0.701295023640316\\
0.97	0.0441367538930258	0.293811937711588	0.698208278749355\\
0.97	0.0456789619920503	0.298783126055387	0.695289592055349\\
0.97	0.0472490486990192	0.303752571668251	0.692523853197512\\
0.97	0.0488470528220538	0.308720052919643	0.689896776228119\\
0.97	0.0504730109482718	0.313685347222913	0.68739498999028\\
0.97	0.0521269574244539	0.318648231061428	0.685006090465981\\
0.97	0.0538089243380495	0.323608480015097	0.682718662029128\\
0.97	0.0555189414985328	0.328565868787292	0.680522273853468\\
0.97	0.0572570364191156	0.333520171232163	0.67840745699785\\
0.97	0.0590232342988274	0.338471160382329	0.676365666960044\\
0.97	0.0608175580049697	0.34341860847696	0.674389235783156\\
0.97	0.0626400280559541	0.348362286990222	0.672471317134863\\
0.97	0.064490662604533	0.3533019666601	0.670605827172182\\
0.97	0.0663694774214291	0.358237417517579	0.668787383459803\\
0.97	0.0682764858793752	0.363168408916186	0.667011243730728\\
0.97	0.0702116989375697	0.368094709561873	0.665273245863649\\
0.97	0.0721751251265572	0.373016087543259	0.663569750098933\\
0.97	0.0741667705335423	0.377932310362199	0.661897584220061\\
0.97	0.0761866387881432	0.382843144964686	0.660253992184443\\
0.97	0.0782347310485954	0.38774835777208	0.658636586490997\\
0.97	0.0803110459884098	0.392647714712651	0.657043304415871\\
0.97	0.0824155797834956	0.397540981253432	0.655472368126509\\
0.97	0.084548326099754	0.402427922432375	0.653922248592746\\
0.97	0.0867092760811502	0.4073083028908	0.652391633146569\\
0.97	0.0888984183382709	0.412181886906127	0.650879396495579\\
0.97	0.0911157389373742	0.417048438424897	0.649384574964844\\
0.97	0.0933612213899392	0.421907721096044	0.647906343724684\\
0.97	0.0956348466427212	0.42675949830445	0.646443996754807\\
0.97	0.0979365930683194	0.431603533204733	0.644996929295962\\
0.97	0.100266436456264	0.436439588755293	0.643564622546846\\
0.97	0.102624350004627	0.441267427752584	0.642146630374429\\
0.97	0.105010304312169	0.446086812865616	0.640742567819434\\
0.97	0.107424267371012	0.450897506670668	0.639352101193731\\
0.97	0.109866204559871	0.455699271686218	0.637974939582443\\
0.97	0.112336078637817	0.460491870408058	0.636610827579861\\
0.97	0.114833849738607	0.465275065344603	0.635259539104382\\
0.97	0.117359475365564	0.470048619052377	0.633920872153155\\
0.97	0.119912910387022	0.474812294171664	0.632594644371966\\
0.97	0.122494107032347	0.479565853462319	0.63128068932964\\
0.97	0.125103014888515	0.484309059839721	0.629978853399058\\
0.97	0.127739580897283	0.489041676410868	0.62869046082456\\
0.975	0	0	0.710202759607373\\
0.975	1.11327767495586e-05	0.00471862581271275	0.713003157837853\\
0.975	4.46175251031912e-05	0.009446325184051	0.715846486660142\\
0.975	0.000100583311362513	0.0141829653360114	0.718788838379794\\
0.975	0.000179158434668431	0.018928411755669	0.721886092228356\\
0.975	0.000280470402701511	0.02368252819604	0.725200636220892\\
0.975	0.000404645907256436	0.0284451766772965	0.728799085274723\\
0.975	0.000551810799695644	0.0332162174883389	0.732749123983336\\
0.975	0.000722090066287311	0.037995509188729	0.737115575253328\\
0.975	0.000915607803432999	0.0427829086109896	0.741955893907011\\
0.975	0.0011324871927904	0.0475782708632729	0.747315373652289\\
0.975	0.00137285047629673	0.0523814493324038	0.753222424045811\\
0.975	0.00163681893109843	0.057192295687301	0.759684310525652\\
0.975	0.00192451284439304	0.0620106598827802	0.766683748102665\\
0.975	0.00223605148818899	0.0668363901637434	0.774176695511256\\
0.975	0.00257155309398959	0.0716693330697584	0.782091614494494\\
0.975	0.00293113482740716	0.0765093334400312	0.79033034642109\\
0.975	0.00331491276271365	0.0813562344187764	0.79877062752665\\
0.975	0.00372300185733414	0.0862098774609879	0.807270128930359\\
0.975	0.00415551592628969	0.0910701023386145	0.815671782700193\\
0.975	0.00461256761659621	0.0959367471471425	0.823810053507331\\
0.975	0.00509426838162598	0.100809648312589	0.831517746503354\\
0.975	0.00560072845543873	0.105688640598912	0.838632911419452\\
0.975	0.00613205682708918	0.11057355711583	0.845005411376293\\
0.975	0.00668836121491815	0.115464229327074	0.850502769049302\\
0.975	0.00726974804083431	0.120360487059049	0.855014975717657\\
0.975	0.00787632240459385	0.125262158509929	0.858458041096606\\
0.975	0.00850818805808555	0.13016907025918	0.860776163494182\\
0.975	0.00916544737962849	0.135081047277508	0.861942500814224\\
0.975	0.00984820134829017	0.139997912937243	0.861958614648419\\
0.975	0.0105565495182326	0.144919489023162	0.860852735617091\\
0.975	0.0112905899930939	0.149845595743738	0.858677054109652\\
0.975	0.012050419400414	0.154776051742839	0.855504274967457\\
0.975	0.0128361328661109	0.159710674111862	0.851423687967754\\
0.975	0.0136478239890177	0.164649278402306	0.846537000484341\\
0.975	0.0144855848154856	0.169591678638796	0.840954157904248\\
0.975	0.0153495058140643	0.174537687332543	0.834789345403938\\
0.975	0.0162396758502651	0.179487115495259	0.828157325806201\\
0.975	0.0171561821614173	0.184439772653511	0.821170226448692\\
0.975	0.0180991103316243	0.189395466863524	0.81393484672569\\
0.975	0.0190685442668299	0.194354004726437	0.806550519904212\\
0.975	0.0200645661700016	0.199315191403999	0.799107529846305\\
0.975	0.0210872565164405	0.204278830634725	0.791686056496193\\
0.975	0.0221366940292259	0.20924472475049	0.784355603830602\\
0.975	0.023212955654804	0.214212674693577	0.777174850281118\\
0.975	0.0243161165387281	0.219182480034174	0.770191853862642\\
0.975	0.025446250001561	0.224153938988321	0.763444541554458\\
0.975	0.0266034275149466	0.229126848436298	0.756961413910482\\
0.975	0.0277877186778607	0.234101003941464	0.750762400416763\\
0.975	0.0289991911930496	0.239076199769547	0.744859807805785\\
0.975	0.0302379108436654	0.244052228908366	0.73925931151808\\
0.975	0.0315039414701067	0.24902888308801	0.733960949044503\\
0.975	0.0327973449470747	0.254005952801449	0.72896008240726\\
0.975	0.0341181811608519	0.258983227325591	0.724248305111179\\
0.975	0.0354665079868145	0.263960494742775	0.719814276220682\\
0.975	0.0368423812671862	0.268937541962694	0.715644470612629\\
0.975	0.0382458547890421	0.273914154744766	0.711723839838042\\
0.975	0.0396769802625738	0.278890117720924	0.70803638238973\\
0.975	0.0411358072996216	0.283865214418836	0.704565625567164\\
0.975	0.0426223833924861	0.288839227285554	0.701295023640317\\
0.975	0.0441367538930258	0.293811937711588	0.698208278749354\\
0.975	0.0456789619920503	0.298783126055387	0.695289592055349\\
0.975	0.0472490486990192	0.303752571668251	0.692523853197511\\
0.975	0.0488470528220538	0.308720052919643	0.689896776228119\\
0.975	0.0504730109482718	0.313685347222913	0.687394989990283\\
0.975	0.0521269574244539	0.318648231061428	0.685006090465981\\
0.975	0.0538089243380495	0.323608480015097	0.682718662029127\\
0.975	0.0555189414985328	0.328565868787292	0.680522273853469\\
0.975	0.0572570364191156	0.333520171232163	0.67840745699785\\
0.975	0.0590232342988274	0.338471160382329	0.676365666960044\\
0.975	0.0608175580049697	0.34341860847696	0.674389235783155\\
0.975	0.0626400280559541	0.348362286990221	0.672471317134863\\
0.975	0.064490662604533	0.3533019666601	0.670605827172183\\
0.975	0.0663694774214291	0.358237417517579	0.668787383459803\\
0.975	0.0682764858793752	0.363168408916186	0.667011243730728\\
0.975	0.0702116989375697	0.368094709561873	0.665273245863648\\
0.975	0.0721751251265572	0.373016087543259	0.663569750098933\\
0.975	0.0741667705335422	0.377932310362199	0.661897584220061\\
0.975	0.0761866387881432	0.382843144964686	0.660253992184442\\
0.975	0.0782347310485954	0.38774835777208	0.658636586490997\\
0.975	0.0803110459884098	0.392647714712651	0.657043304415871\\
0.975	0.0824155797834956	0.397540981253432	0.65547236812651\\
0.975	0.084548326099754	0.402427922432375	0.653922248592745\\
0.975	0.0867092760811502	0.407308302890799	0.65239163314657\\
0.975	0.0888984183382709	0.412181886906127	0.650879396495579\\
0.975	0.0911157389373742	0.417048438424896	0.649384574964844\\
0.975	0.0933612213899392	0.421907721096044	0.647906343724684\\
0.975	0.0956348466427212	0.42675949830445	0.646443996754807\\
0.975	0.0979365930683194	0.431603533204733	0.644996929295961\\
0.975	0.100266436456264	0.436439588755293	0.643564622546846\\
0.975	0.102624350004627	0.441267427752584	0.64214663037443\\
0.975	0.105010304312169	0.446086812865616	0.640742567819432\\
0.975	0.107424267371012	0.450897506670668	0.639352101193731\\
0.975	0.109866204559871	0.455699271686218	0.637974939582444\\
0.975	0.112336078637817	0.460491870408058	0.636610827579862\\
0.975	0.114833849738607	0.465275065344603	0.635259539104382\\
0.975	0.117359475365564	0.470048619052377	0.633920872153153\\
0.975	0.119912910387022	0.474812294171664	0.632594644371966\\
0.975	0.122494107032347	0.479565853462319	0.631280689329642\\
0.975	0.125103014888515	0.484309059839721	0.629978853399058\\
0.975	0.127739580897283	0.489041676410868	0.628690460824561\\
0.98	0	0	0.710202759607373\\
0.98	1.11327767495586e-05	0.00471862581271275	0.713003157837853\\
0.98	4.46175251031912e-05	0.009446325184051	0.715846486660142\\
0.98	0.000100583311362513	0.0141829653360114	0.718788838379794\\
0.98	0.000179158434668431	0.018928411755669	0.721886092228356\\
0.98	0.000280470402701511	0.02368252819604	0.725200636220892\\
0.98	0.000404645907256436	0.0284451766772965	0.728799085274723\\
0.98	0.000551810799695644	0.0332162174883388	0.732749123983336\\
0.98	0.000722090066287311	0.037995509188729	0.737115575253327\\
0.98	0.000915607803432999	0.0427829086109896	0.741955893907011\\
0.98	0.0011324871927904	0.0475782708632729	0.747315373652289\\
0.98	0.00137285047629673	0.0523814493324038	0.75322242404581\\
0.98	0.00163681893109844	0.057192295687301	0.759684310525653\\
0.98	0.00192451284439304	0.0620106598827802	0.766683748102665\\
0.98	0.00223605148818898	0.0668363901637434	0.774176695511256\\
0.98	0.00257155309398959	0.0716693330697584	0.782091614494493\\
0.98	0.00293113482740716	0.0765093334400312	0.79033034642109\\
0.98	0.00331491276271365	0.0813562344187764	0.798770627526651\\
0.98	0.00372300185733413	0.0862098774609879	0.807270128930359\\
0.98	0.00415551592628969	0.0910701023386145	0.815671782700193\\
0.98	0.00461256761659621	0.0959367471471425	0.82381005350733\\
0.98	0.00509426838162598	0.100809648312589	0.831517746503354\\
0.98	0.00560072845543873	0.105688640598912	0.838632911419451\\
0.98	0.00613205682708918	0.11057355711583	0.845005411376293\\
0.98	0.00668836121491815	0.115464229327074	0.850502769049303\\
0.98	0.00726974804083431	0.120360487059049	0.855014975717656\\
0.98	0.00787632240459385	0.125262158509929	0.858458041096607\\
0.98	0.00850818805808555	0.13016907025918	0.860776163494181\\
0.98	0.00916544737962849	0.135081047277508	0.861942500814224\\
0.98	0.00984820134829017	0.139997912937243	0.86195861464842\\
0.98	0.0105565495182326	0.144919489023162	0.86085273561709\\
0.98	0.0112905899930939	0.149845595743738	0.858677054109651\\
0.98	0.012050419400414	0.154776051742839	0.855504274967457\\
0.98	0.0128361328661109	0.159710674111862	0.851423687967756\\
0.98	0.0136478239890177	0.164649278402306	0.846537000484339\\
0.98	0.0144855848154856	0.169591678638796	0.84095415790425\\
0.98	0.0153495058140643	0.174537687332543	0.834789345403937\\
0.98	0.0162396758502651	0.179487115495259	0.828157325806199\\
0.98	0.0171561821614173	0.184439772653511	0.821170226448693\\
0.98	0.0180991103316243	0.189395466863524	0.813934846725692\\
0.98	0.0190685442668299	0.194354004726437	0.806550519904212\\
0.98	0.0200645661700016	0.199315191403999	0.799107529846304\\
0.98	0.0210872565164405	0.204278830634725	0.791686056496194\\
0.98	0.0221366940292259	0.20924472475049	0.784355603830601\\
0.98	0.023212955654804	0.214212674693577	0.777174850281118\\
0.98	0.0243161165387281	0.219182480034174	0.770191853862643\\
0.98	0.025446250001561	0.224153938988321	0.763444541554458\\
0.98	0.0266034275149466	0.229126848436298	0.756961413910482\\
0.98	0.0277877186778607	0.234101003941464	0.750762400416765\\
0.98	0.0289991911930496	0.239076199769547	0.744859807805784\\
0.98	0.0302379108436654	0.244052228908366	0.739259311518081\\
0.98	0.0315039414701067	0.24902888308801	0.733960949044503\\
0.98	0.0327973449470747	0.254005952801449	0.72896008240726\\
0.98	0.0341181811608519	0.258983227325591	0.72424830511118\\
0.98	0.0354665079868145	0.263960494742775	0.719814276220681\\
0.98	0.0368423812671862	0.268937541962694	0.715644470612629\\
0.98	0.0382458547890421	0.273914154744766	0.711723839838043\\
0.98	0.0396769802625738	0.278890117720924	0.70803638238973\\
0.98	0.0411358072996216	0.283865214418836	0.704565625567163\\
0.98	0.0426223833924861	0.288839227285554	0.701295023640316\\
0.98	0.0441367538930258	0.293811937711588	0.698208278749354\\
0.98	0.0456789619920503	0.298783126055387	0.69528959205535\\
0.98	0.0472490486990192	0.303752571668251	0.692523853197512\\
0.98	0.0488470528220538	0.308720052919643	0.689896776228118\\
0.98	0.0504730109482718	0.313685347222913	0.687394989990282\\
0.98	0.0521269574244539	0.318648231061428	0.685006090465981\\
0.98	0.0538089243380495	0.323608480015096	0.682718662029128\\
0.98	0.0555189414985328	0.328565868787292	0.680522273853469\\
0.98	0.0572570364191156	0.333520171232163	0.67840745699785\\
0.98	0.0590232342988274	0.338471160382329	0.676365666960043\\
0.98	0.0608175580049697	0.34341860847696	0.674389235783155\\
0.98	0.0626400280559541	0.348362286990221	0.672471317134862\\
0.98	0.064490662604533	0.3533019666601	0.670605827172183\\
0.98	0.0663694774214291	0.358237417517579	0.668787383459803\\
0.98	0.0682764858793752	0.363168408916186	0.667011243730728\\
0.98	0.0702116989375697	0.368094709561873	0.665273245863649\\
0.98	0.0721751251265572	0.373016087543259	0.663569750098932\\
0.98	0.0741667705335422	0.377932310362199	0.661897584220062\\
0.98	0.0761866387881432	0.382843144964686	0.660253992184443\\
0.98	0.0782347310485954	0.38774835777208	0.658636586490997\\
0.98	0.0803110459884098	0.392647714712651	0.65704330441587\\
0.98	0.0824155797834956	0.397540981253432	0.65547236812651\\
0.98	0.084548326099754	0.402427922432375	0.653922248592746\\
0.98	0.0867092760811502	0.407308302890799	0.65239163314657\\
0.98	0.0888984183382709	0.412181886906127	0.650879396495579\\
0.98	0.0911157389373742	0.417048438424896	0.649384574964844\\
0.98	0.0933612213899392	0.421907721096044	0.647906343724684\\
0.98	0.0956348466427212	0.42675949830445	0.646443996754806\\
0.98	0.0979365930683194	0.431603533204733	0.644996929295962\\
0.98	0.100266436456264	0.436439588755293	0.643564622546845\\
0.98	0.102624350004627	0.441267427752584	0.642146630374431\\
0.98	0.105010304312169	0.446086812865616	0.640742567819433\\
0.98	0.107424267371012	0.450897506670668	0.639352101193729\\
0.98	0.109866204559871	0.455699271686218	0.637974939582444\\
0.98	0.112336078637817	0.460491870408058	0.636610827579863\\
0.98	0.114833849738607	0.465275065344603	0.635259539104383\\
0.98	0.117359475365564	0.470048619052377	0.633920872153153\\
0.98	0.119912910387022	0.474812294171664	0.632594644371965\\
0.98	0.122494107032347	0.479565853462319	0.631280689329643\\
0.98	0.125103014888515	0.484309059839721	0.629978853399059\\
0.98	0.127739580897283	0.489041676410868	0.628690460824557\\
0.985	0	0	0.710202759607373\\
0.985	1.11327767495586e-05	0.00471862581271275	0.713003157837853\\
0.985	4.46175251031912e-05	0.009446325184051	0.715846486660142\\
0.985	0.000100583311362513	0.0141829653360114	0.718788838379794\\
0.985	0.000179158434668431	0.018928411755669	0.721886092228357\\
0.985	0.000280470402701511	0.02368252819604	0.725200636220892\\
0.985	0.000404645907256436	0.0284451766772965	0.728799085274723\\
0.985	0.000551810799695644	0.0332162174883388	0.732749123983336\\
0.985	0.000722090066287311	0.037995509188729	0.737115575253327\\
0.985	0.000915607803432999	0.0427829086109896	0.741955893907011\\
0.985	0.0011324871927904	0.047578270863273	0.747315373652289\\
0.985	0.00137285047629673	0.0523814493324038	0.75322242404581\\
0.985	0.00163681893109844	0.057192295687301	0.759684310525653\\
0.985	0.00192451284439304	0.0620106598827802	0.766683748102665\\
0.985	0.00223605148818898	0.0668363901637434	0.774176695511256\\
0.985	0.00257155309398959	0.0716693330697584	0.782091614494494\\
0.985	0.00293113482740716	0.0765093334400312	0.79033034642109\\
0.985	0.00331491276271365	0.0813562344187764	0.79877062752665\\
0.985	0.00372300185733413	0.0862098774609879	0.80727012893036\\
0.985	0.00415551592628969	0.0910701023386145	0.815671782700193\\
0.985	0.00461256761659621	0.0959367471471425	0.82381005350733\\
0.985	0.00509426838162598	0.100809648312589	0.831517746503354\\
0.985	0.00560072845543873	0.105688640598912	0.838632911419452\\
0.985	0.00613205682708918	0.11057355711583	0.845005411376292\\
0.985	0.00668836121491815	0.115464229327074	0.850502769049302\\
0.985	0.00726974804083431	0.120360487059049	0.855014975717657\\
0.985	0.00787632240459385	0.125262158509929	0.858458041096606\\
0.985	0.00850818805808555	0.13016907025918	0.860776163494182\\
0.985	0.00916544737962849	0.135081047277508	0.861942500814224\\
0.985	0.00984820134829017	0.139997912937243	0.861958614648422\\
0.985	0.0105565495182326	0.144919489023162	0.860852735617091\\
0.985	0.0112905899930939	0.149845595743738	0.85867705410965\\
0.985	0.012050419400414	0.154776051742839	0.855504274967457\\
0.985	0.0128361328661109	0.159710674111862	0.851423687967753\\
0.985	0.0136478239890177	0.164649278402306	0.846537000484341\\
0.985	0.0144855848154856	0.169591678638796	0.840954157904249\\
0.985	0.0153495058140643	0.174537687332543	0.834789345403934\\
0.985	0.0162396758502651	0.179487115495259	0.8281573258062\\
0.985	0.0171561821614173	0.184439772653511	0.821170226448694\\
0.985	0.0180991103316243	0.189395466863524	0.813934846725693\\
0.985	0.0190685442668299	0.194354004726437	0.806550519904213\\
0.985	0.0200645661700016	0.199315191403999	0.799107529846306\\
0.985	0.0210872565164405	0.204278830634725	0.791686056496194\\
0.985	0.0221366940292259	0.20924472475049	0.784355603830602\\
0.985	0.023212955654804	0.214212674693577	0.777174850281118\\
0.985	0.0243161165387281	0.219182480034174	0.770191853862643\\
0.985	0.025446250001561	0.224153938988321	0.763444541554458\\
0.985	0.0266034275149466	0.229126848436298	0.756961413910483\\
0.985	0.0277877186778607	0.234101003941464	0.750762400416763\\
0.985	0.0289991911930496	0.239076199769547	0.744859807805785\\
0.985	0.0302379108436654	0.244052228908366	0.739259311518081\\
0.985	0.0315039414701067	0.24902888308801	0.733960949044503\\
0.985	0.0327973449470747	0.254005952801449	0.728960082407261\\
0.985	0.0341181811608519	0.258983227325592	0.724248305111177\\
0.985	0.0354665079868145	0.263960494742775	0.71981427622068\\
0.985	0.0368423812671862	0.268937541962694	0.71564447061263\\
0.985	0.0382458547890421	0.273914154744766	0.71172383983804\\
0.985	0.0396769802625738	0.278890117720924	0.708036382389731\\
0.985	0.0411358072996216	0.283865214418836	0.704565625567165\\
0.985	0.0426223833924861	0.288839227285554	0.701295023640316\\
0.985	0.0441367538930258	0.293811937711588	0.698208278749354\\
0.985	0.0456789619920503	0.298783126055387	0.695289592055349\\
0.985	0.0472490486990192	0.303752571668251	0.692523853197512\\
0.985	0.0488470528220537	0.308720052919643	0.689896776228119\\
0.985	0.0504730109482718	0.313685347222913	0.687394989990282\\
0.985	0.0521269574244539	0.318648231061428	0.685006090465981\\
0.985	0.0538089243380495	0.323608480015097	0.682718662029127\\
0.985	0.0555189414985328	0.328565868787292	0.680522273853469\\
0.985	0.0572570364191156	0.333520171232163	0.67840745699785\\
0.985	0.0590232342988274	0.338471160382329	0.676365666960043\\
0.985	0.0608175580049697	0.34341860847696	0.674389235783156\\
0.985	0.0626400280559541	0.348362286990222	0.672471317134862\\
0.985	0.064490662604533	0.3533019666601	0.670605827172183\\
0.985	0.0663694774214291	0.358237417517579	0.668787383459804\\
0.985	0.0682764858793752	0.363168408916186	0.667011243730728\\
0.985	0.0702116989375697	0.368094709561873	0.665273245863648\\
0.985	0.0721751251265572	0.373016087543259	0.663569750098931\\
0.985	0.0741667705335422	0.377932310362199	0.66189758422006\\
0.985	0.0761866387881432	0.382843144964686	0.660253992184442\\
0.985	0.0782347310485954	0.38774835777208	0.658636586490997\\
0.985	0.0803110459884098	0.392647714712651	0.65704330441587\\
0.985	0.0824155797834956	0.397540981253432	0.655472368126509\\
0.985	0.084548326099754	0.402427922432375	0.653922248592746\\
0.985	0.0867092760811502	0.4073083028908	0.652391633146569\\
0.985	0.0888984183382709	0.412181886906127	0.650879396495579\\
0.985	0.0911157389373742	0.417048438424896	0.649384574964845\\
0.985	0.0933612213899392	0.421907721096044	0.647906343724685\\
0.985	0.0956348466427212	0.42675949830445	0.646443996754805\\
0.985	0.0979365930683194	0.431603533204733	0.644996929295963\\
0.985	0.100266436456264	0.436439588755293	0.643564622546847\\
0.985	0.102624350004627	0.441267427752585	0.642146630374429\\
0.985	0.105010304312169	0.446086812865616	0.640742567819434\\
0.985	0.107424267371012	0.450897506670668	0.639352101193731\\
0.985	0.109866204559871	0.455699271686218	0.637974939582442\\
0.985	0.112336078637817	0.460491870408058	0.636610827579861\\
0.985	0.114833849738607	0.465275065344603	0.635259539104384\\
0.985	0.117359475365564	0.470048619052377	0.633920872153156\\
0.985	0.119912910387023	0.474812294171664	0.632594644371963\\
0.985	0.122494107032347	0.479565853462319	0.63128068932964\\
0.985	0.125103014888515	0.484309059839721	0.629978853399058\\
0.985	0.127739580897283	0.489041676410868	0.628690460824556\\
0.99	0	0	0.710202759607373\\
0.99	1.11327767495586e-05	0.00471862581271275	0.713003157837853\\
0.99	4.46175251031912e-05	0.009446325184051	0.715846486660142\\
0.99	0.000100583311362513	0.0141829653360114	0.718788838379794\\
0.99	0.000179158434668431	0.018928411755669	0.721886092228356\\
0.99	0.000280470402701511	0.02368252819604	0.725200636220892\\
0.99	0.000404645907256436	0.0284451766772965	0.728799085274723\\
0.99	0.000551810799695644	0.0332162174883388	0.732749123983336\\
0.99	0.000722090066287311	0.037995509188729	0.737115575253327\\
0.99	0.000915607803432999	0.0427829086109896	0.741955893907011\\
0.99	0.0011324871927904	0.0475782708632729	0.747315373652289\\
0.99	0.00137285047629673	0.0523814493324038	0.753222424045811\\
0.99	0.00163681893109844	0.057192295687301	0.759684310525652\\
0.99	0.00192451284439304	0.0620106598827802	0.766683748102665\\
0.99	0.00223605148818899	0.0668363901637434	0.774176695511256\\
0.99	0.00257155309398959	0.0716693330697584	0.782091614494494\\
0.99	0.00293113482740716	0.0765093334400312	0.79033034642109\\
0.99	0.00331491276271365	0.0813562344187764	0.79877062752665\\
0.99	0.00372300185733414	0.0862098774609879	0.807270128930359\\
0.99	0.00415551592628969	0.0910701023386145	0.815671782700194\\
0.99	0.00461256761659621	0.0959367471471425	0.82381005350733\\
0.99	0.00509426838162598	0.100809648312589	0.831517746503353\\
0.99	0.00560072845543873	0.105688640598912	0.838632911419452\\
0.99	0.00613205682708918	0.11057355711583	0.845005411376293\\
0.99	0.00668836121491816	0.115464229327074	0.850502769049301\\
0.99	0.00726974804083431	0.120360487059049	0.855014975717657\\
0.99	0.00787632240459385	0.125262158509929	0.858458041096605\\
0.99	0.00850818805808555	0.13016907025918	0.860776163494183\\
0.99	0.00916544737962849	0.135081047277508	0.861942500814224\\
0.99	0.00984820134829017	0.139997912937243	0.861958614648421\\
0.99	0.0105565495182326	0.144919489023162	0.860852735617092\\
0.99	0.0112905899930939	0.149845595743738	0.85867705410965\\
0.99	0.012050419400414	0.154776051742839	0.855504274967457\\
0.99	0.0128361328661109	0.159710674111862	0.851423687967752\\
0.99	0.0136478239890177	0.164649278402306	0.846537000484342\\
0.99	0.0144855848154856	0.169591678638796	0.840954157904247\\
0.99	0.0153495058140643	0.174537687332543	0.834789345403934\\
0.99	0.0162396758502651	0.179487115495259	0.828157325806199\\
0.99	0.0171561821614173	0.184439772653511	0.821170226448694\\
0.99	0.0180991103316243	0.189395466863524	0.813934846725692\\
0.99	0.0190685442668299	0.194354004726437	0.806550519904213\\
0.99	0.0200645661700016	0.199315191403999	0.799107529846304\\
0.99	0.0210872565164405	0.204278830634725	0.791686056496192\\
0.99	0.0221366940292259	0.20924472475049	0.7843556038306\\
0.99	0.023212955654804	0.214212674693577	0.777174850281116\\
0.99	0.0243161165387281	0.219182480034174	0.770191853862643\\
0.99	0.025446250001561	0.224153938988321	0.763444541554458\\
0.99	0.0266034275149466	0.229126848436298	0.756961413910482\\
0.99	0.0277877186778607	0.234101003941464	0.750762400416764\\
0.99	0.0289991911930496	0.239076199769547	0.744859807805785\\
0.99	0.0302379108436654	0.244052228908366	0.739259311518079\\
0.99	0.0315039414701067	0.24902888308801	0.733960949044504\\
0.99	0.0327973449470747	0.254005952801449	0.72896008240726\\
0.99	0.0341181811608519	0.258983227325591	0.724248305111178\\
0.99	0.0354665079868145	0.263960494742775	0.719814276220682\\
0.99	0.0368423812671862	0.268937541962694	0.715644470612629\\
0.99	0.0382458547890421	0.273914154744766	0.711723839838042\\
0.99	0.0396769802625738	0.278890117720924	0.70803638238973\\
0.99	0.0411358072996216	0.283865214418836	0.704565625567164\\
0.99	0.0426223833924861	0.288839227285554	0.701295023640316\\
0.99	0.0441367538930258	0.293811937711588	0.698208278749355\\
0.99	0.0456789619920503	0.298783126055387	0.695289592055349\\
0.99	0.0472490486990192	0.303752571668251	0.692523853197512\\
0.99	0.0488470528220538	0.308720052919643	0.689896776228119\\
0.99	0.0504730109482718	0.313685347222913	0.68739498999028\\
0.99	0.0521269574244539	0.318648231061428	0.685006090465981\\
0.99	0.0538089243380495	0.323608480015096	0.682718662029128\\
0.99	0.0555189414985328	0.328565868787292	0.680522273853469\\
0.99	0.0572570364191156	0.333520171232163	0.678407456997851\\
0.99	0.0590232342988274	0.338471160382329	0.676365666960043\\
0.99	0.0608175580049697	0.34341860847696	0.674389235783155\\
0.99	0.0626400280559541	0.348362286990222	0.672471317134862\\
0.99	0.064490662604533	0.3533019666601	0.670605827172183\\
0.99	0.0663694774214291	0.358237417517579	0.668787383459805\\
0.99	0.0682764858793752	0.363168408916186	0.667011243730728\\
0.99	0.0702116989375697	0.368094709561873	0.66527324586365\\
0.99	0.0721751251265572	0.373016087543259	0.663569750098932\\
0.99	0.0741667705335423	0.377932310362199	0.661897584220059\\
0.99	0.0761866387881432	0.382843144964686	0.660253992184443\\
0.99	0.0782347310485954	0.38774835777208	0.658636586490997\\
0.99	0.0803110459884098	0.392647714712651	0.657043304415871\\
0.99	0.0824155797834956	0.397540981253432	0.655472368126508\\
0.99	0.084548326099754	0.402427922432375	0.653922248592746\\
0.99	0.0867092760811502	0.4073083028908	0.65239163314657\\
0.99	0.0888984183382708	0.412181886906127	0.650879396495579\\
0.99	0.0911157389373742	0.417048438424896	0.649384574964843\\
0.99	0.0933612213899392	0.421907721096044	0.647906343724685\\
0.99	0.0956348466427212	0.42675949830445	0.646443996754805\\
0.99	0.0979365930683194	0.431603533204733	0.644996929295961\\
0.99	0.100266436456264	0.436439588755293	0.643564622546847\\
0.99	0.102624350004627	0.441267427752584	0.642146630374431\\
0.99	0.105010304312169	0.446086812865616	0.640742567819433\\
0.99	0.107424267371012	0.450897506670668	0.639352101193731\\
0.99	0.109866204559871	0.455699271686218	0.637974939582444\\
0.99	0.112336078637817	0.460491870408058	0.636610827579862\\
0.99	0.114833849738607	0.465275065344603	0.635259539104381\\
0.99	0.117359475365564	0.470048619052377	0.633920872153154\\
0.99	0.119912910387022	0.474812294171664	0.632594644371967\\
0.99	0.122494107032347	0.479565853462319	0.631280689329642\\
0.99	0.125103014888515	0.484309059839721	0.629978853399059\\
0.99	0.127739580897283	0.489041676410868	0.628690460824557\\
0.995	0	0	0.710202759607373\\
0.995	1.11327767495586e-05	0.00471862581271274	0.713003157837853\\
0.995	4.46175251031912e-05	0.009446325184051	0.715846486660142\\
0.995	0.000100583311362513	0.0141829653360114	0.718788838379794\\
0.995	0.000179158434668431	0.018928411755669	0.721886092228356\\
0.995	0.000280470402701511	0.02368252819604	0.725200636220892\\
0.995	0.000404645907256436	0.0284451766772965	0.728799085274723\\
0.995	0.000551810799695644	0.0332162174883389	0.732749123983336\\
0.995	0.000722090066287311	0.037995509188729	0.737115575253328\\
0.995	0.000915607803432999	0.0427829086109896	0.741955893907011\\
0.995	0.0011324871927904	0.0475782708632729	0.747315373652289\\
0.995	0.00137285047629673	0.0523814493324038	0.753222424045811\\
0.995	0.00163681893109844	0.057192295687301	0.759684310525653\\
0.995	0.00192451284439304	0.0620106598827802	0.766683748102665\\
0.995	0.00223605148818899	0.0668363901637434	0.774176695511256\\
0.995	0.00257155309398959	0.0716693330697584	0.782091614494494\\
0.995	0.00293113482740716	0.0765093334400312	0.79033034642109\\
0.995	0.00331491276271365	0.0813562344187764	0.79877062752665\\
0.995	0.00372300185733414	0.0862098774609879	0.807270128930359\\
0.995	0.00415551592628969	0.0910701023386145	0.815671782700193\\
0.995	0.00461256761659621	0.0959367471471425	0.82381005350733\\
0.995	0.00509426838162598	0.100809648312589	0.831517746503354\\
0.995	0.00560072845543873	0.105688640598912	0.838632911419452\\
0.995	0.00613205682708918	0.11057355711583	0.845005411376293\\
0.995	0.00668836121491815	0.115464229327074	0.850502769049302\\
0.995	0.00726974804083431	0.120360487059049	0.855014975717657\\
0.995	0.00787632240459385	0.125262158509929	0.858458041096606\\
0.995	0.00850818805808555	0.13016907025918	0.860776163494183\\
0.995	0.00916544737962849	0.135081047277508	0.861942500814225\\
0.995	0.00984820134829017	0.139997912937243	0.861958614648421\\
0.995	0.0105565495182326	0.144919489023162	0.86085273561709\\
0.995	0.0112905899930939	0.149845595743738	0.858677054109649\\
0.995	0.012050419400414	0.154776051742839	0.855504274967456\\
0.995	0.0128361328661109	0.159710674111862	0.851423687967757\\
0.995	0.0136478239890177	0.164649278402306	0.846537000484341\\
0.995	0.0144855848154856	0.169591678638796	0.840954157904251\\
0.995	0.0153495058140643	0.174537687332543	0.834789345403936\\
0.995	0.0162396758502651	0.179487115495259	0.828157325806201\\
0.995	0.0171561821614173	0.184439772653511	0.821170226448695\\
0.995	0.0180991103316243	0.189395466863524	0.813934846725693\\
0.995	0.0190685442668299	0.194354004726437	0.806550519904213\\
0.995	0.0200645661700016	0.199315191403999	0.799107529846303\\
0.995	0.0210872565164405	0.204278830634725	0.791686056496192\\
0.995	0.0221366940292259	0.20924472475049	0.784355603830601\\
0.995	0.023212955654804	0.214212674693577	0.77717485028112\\
0.995	0.0243161165387281	0.219182480034174	0.770191853862644\\
0.995	0.025446250001561	0.224153938988321	0.76344454155446\\
0.995	0.0266034275149466	0.229126848436298	0.756961413910484\\
0.995	0.0277877186778607	0.234101003941464	0.750762400416764\\
0.995	0.0289991911930496	0.239076199769547	0.744859807805784\\
0.995	0.0302379108436654	0.244052228908366	0.739259311518081\\
0.995	0.0315039414701067	0.24902888308801	0.733960949044505\\
0.995	0.0327973449470747	0.254005952801449	0.728960082407261\\
0.995	0.0341181811608519	0.258983227325592	0.724248305111178\\
0.995	0.0354665079868145	0.263960494742775	0.71981427622068\\
0.995	0.0368423812671862	0.268937541962694	0.71564447061263\\
0.995	0.0382458547890421	0.273914154744766	0.711723839838042\\
0.995	0.0396769802625738	0.278890117720924	0.708036382389729\\
0.995	0.0411358072996216	0.283865214418836	0.704565625567164\\
0.995	0.0426223833924861	0.288839227285554	0.701295023640316\\
0.995	0.0441367538930258	0.293811937711588	0.698208278749354\\
0.995	0.0456789619920503	0.298783126055387	0.695289592055347\\
0.995	0.0472490486990192	0.303752571668251	0.692523853197511\\
0.995	0.0488470528220538	0.308720052919643	0.68989677622812\\
0.995	0.0504730109482718	0.313685347222913	0.687394989990282\\
0.995	0.0521269574244539	0.318648231061428	0.685006090465981\\
0.995	0.0538089243380495	0.323608480015096	0.682718662029128\\
0.995	0.0555189414985328	0.328565868787292	0.68052227385347\\
0.995	0.0572570364191156	0.333520171232163	0.67840745699785\\
0.995	0.0590232342988274	0.338471160382329	0.676365666960043\\
0.995	0.0608175580049697	0.34341860847696	0.674389235783155\\
0.995	0.0626400280559541	0.348362286990221	0.672471317134861\\
0.995	0.064490662604533	0.3533019666601	0.670605827172183\\
0.995	0.0663694774214291	0.358237417517579	0.668787383459805\\
0.995	0.0682764858793752	0.363168408916186	0.667011243730728\\
0.995	0.0702116989375697	0.368094709561873	0.66527324586365\\
0.995	0.0721751251265572	0.373016087543259	0.663569750098933\\
0.995	0.0741667705335422	0.377932310362199	0.66189758422006\\
0.995	0.0761866387881432	0.382843144964686	0.660253992184441\\
0.995	0.0782347310485954	0.38774835777208	0.658636586490998\\
0.995	0.0803110459884098	0.392647714712651	0.657043304415871\\
0.995	0.0824155797834956	0.397540981253432	0.655472368126509\\
0.995	0.084548326099754	0.402427922432375	0.653922248592745\\
0.995	0.0867092760811502	0.407308302890799	0.65239163314657\\
0.995	0.0888984183382708	0.412181886906127	0.650879396495579\\
0.995	0.0911157389373742	0.417048438424896	0.649384574964843\\
0.995	0.0933612213899392	0.421907721096044	0.647906343724685\\
0.995	0.0956348466427212	0.42675949830445	0.646443996754806\\
0.995	0.0979365930683194	0.431603533204733	0.644996929295961\\
0.995	0.100266436456264	0.436439588755293	0.643564622546846\\
0.995	0.102624350004627	0.441267427752584	0.64214663037443\\
0.995	0.105010304312169	0.446086812865616	0.640742567819433\\
0.995	0.107424267371012	0.450897506670668	0.63935210119373\\
0.995	0.109866204559871	0.455699271686218	0.637974939582444\\
0.995	0.112336078637817	0.460491870408058	0.636610827579863\\
0.995	0.114833849738607	0.465275065344603	0.635259539104382\\
0.995	0.117359475365564	0.470048619052377	0.633920872153153\\
0.995	0.119912910387022	0.474812294171664	0.632594644371964\\
0.995	0.122494107032347	0.479565853462319	0.631280689329642\\
0.995	0.125103014888515	0.484309059839721	0.62997885339906\\
0.995	0.127739580897283	0.489041676410868	0.628690460824559\\
1	0	0	0.710202759607373\\
1	1.11327767495586e-05	0.00471862581271275	0.713003157837853\\
1	4.46175251031912e-05	0.009446325184051	0.715846486660142\\
1	0.000100583311362513	0.0141829653360114	0.718788838379794\\
1	0.000179158434668431	0.018928411755669	0.721886092228356\\
1	0.000280470402701511	0.02368252819604	0.725200636220892\\
1	0.000404645907256436	0.0284451766772965	0.728799085274723\\
1	0.000551810799695644	0.0332162174883388	0.732749123983336\\
1	0.000722090066287311	0.037995509188729	0.737115575253327\\
1	0.000915607803432999	0.0427829086109896	0.741955893907011\\
1	0.0011324871927904	0.047578270863273	0.747315373652289\\
1	0.00137285047629673	0.0523814493324038	0.75322242404581\\
1	0.00163681893109844	0.057192295687301	0.759684310525652\\
1	0.00192451284439304	0.0620106598827802	0.766683748102665\\
1	0.00223605148818898	0.0668363901637434	0.774176695511256\\
1	0.00257155309398959	0.0716693330697584	0.782091614494494\\
1	0.00293113482740716	0.0765093334400312	0.79033034642109\\
1	0.00331491276271365	0.0813562344187764	0.79877062752665\\
1	0.00372300185733413	0.0862098774609879	0.807270128930359\\
1	0.00415551592628969	0.0910701023386145	0.815671782700192\\
1	0.00461256761659621	0.0959367471471425	0.823810053507328\\
1	0.00509426838162598	0.100809648312589	0.831517746503355\\
1	0.00560072845543873	0.105688640598912	0.838632911419453\\
1	0.00613205682708918	0.11057355711583	0.845005411376294\\
1	0.00668836121491816	0.115464229327074	0.850502769049302\\
1	0.00726974804083431	0.120360487059049	0.855014975717657\\
1	0.00787632240459385	0.125262158509929	0.858458041096606\\
1	0.00850818805808554	0.13016907025918	0.860776163494183\\
1	0.00916544737962849	0.135081047277508	0.861942500814227\\
1	0.00984820134829017	0.139997912937243	0.86195861464842\\
1	0.0105565495182326	0.144919489023162	0.86085273561709\\
1	0.0112905899930939	0.149845595743738	0.858677054109648\\
1	0.012050419400414	0.154776051742839	0.855504274967455\\
1	0.0128361328661109	0.159710674111862	0.851423687967763\\
1	0.0136478239890177	0.164649278402306	0.846537000484335\\
1	0.0144855848154856	0.169591678638796	0.840954157904254\\
1	0.0153495058140643	0.174537687332543	0.834789345403943\\
1	0.0162396758502651	0.179487115495259	0.828157325806207\\
1	0.0171561821614173	0.184439772653511	0.821170226448698\\
1	0.0180991103316243	0.189395466863524	0.813934846725698\\
1	0.0190685442668299	0.194354004726437	0.806550519904213\\
1	0.0200645661700016	0.199315191403999	0.7991075298463\\
1	0.0210872565164405	0.204278830634725	0.791686056496194\\
1	0.0221366940292259	0.20924472475049	0.784355603830603\\
1	0.023212955654804	0.214212674693577	0.77717485028112\\
1	0.0243161165387281	0.219182480034174	0.770191853862642\\
1	0.025446250001561	0.224153938988321	0.76344454155446\\
1	0.0266034275149466	0.229126848436298	0.756961413910482\\
1	0.0277877186778607	0.234101003941464	0.750762400416763\\
1	0.0289991911930496	0.239076199769547	0.744859807805785\\
1	0.0302379108436654	0.244052228908366	0.739259311518082\\
1	0.0315039414701067	0.24902888308801	0.733960949044504\\
1	0.0327973449470747	0.254005952801449	0.72896008240726\\
1	0.0341181811608519	0.258983227325592	0.724248305111177\\
1	0.0354665079868145	0.263960494742775	0.71981427622068\\
1	0.0368423812671862	0.268937541962694	0.71564447061263\\
1	0.0382458547890421	0.273914154744766	0.711723839838042\\
1	0.0396769802625738	0.278890117720924	0.708036382389732\\
1	0.0411358072996216	0.283865214418836	0.704565625567164\\
1	0.0426223833924861	0.288839227285554	0.701295023640316\\
1	0.0441367538930258	0.293811937711588	0.698208278749354\\
1	0.0456789619920503	0.298783126055387	0.69528959205535\\
1	0.0472490486990192	0.303752571668251	0.692523853197511\\
1	0.0488470528220538	0.308720052919643	0.689896776228118\\
1	0.0504730109482718	0.313685347222913	0.687394989990281\\
1	0.0521269574244539	0.318648231061428	0.685006090465981\\
1	0.0538089243380495	0.323608480015097	0.682718662029127\\
1	0.0555189414985328	0.328565868787292	0.680522273853469\\
1	0.0572570364191156	0.333520171232163	0.678407456997852\\
1	0.0590232342988274	0.338471160382329	0.676365666960042\\
1	0.0608175580049697	0.34341860847696	0.674389235783156\\
1	0.0626400280559541	0.348362286990222	0.672471317134859\\
1	0.064490662604533	0.3533019666601	0.670605827172181\\
1	0.0663694774214291	0.358237417517579	0.668787383459803\\
1	0.0682764858793752	0.363168408916186	0.667011243730729\\
1	0.0702116989375697	0.368094709561873	0.665273245863648\\
1	0.0721751251265572	0.373016087543259	0.663569750098933\\
1	0.0741667705335423	0.377932310362199	0.661897584220062\\
1	0.0761866387881432	0.382843144964686	0.660253992184441\\
1	0.0782347310485954	0.38774835777208	0.658636586490997\\
1	0.0803110459884098	0.392647714712651	0.657043304415871\\
1	0.0824155797834956	0.397540981253432	0.655472368126511\\
1	0.084548326099754	0.402427922432375	0.653922248592746\\
1	0.0867092760811502	0.4073083028908	0.652391633146568\\
1	0.0888984183382709	0.412181886906127	0.650879396495577\\
1	0.0911157389373742	0.417048438424897	0.649384574964844\\
1	0.0933612213899392	0.421907721096044	0.647906343724682\\
1	0.0956348466427212	0.42675949830445	0.646443996754806\\
1	0.0979365930683194	0.431603533204733	0.644996929295963\\
1	0.100266436456264	0.436439588755293	0.643564622546846\\
1	0.102624350004627	0.441267427752584	0.642146630374429\\
1	0.105010304312169	0.446086812865616	0.640742567819434\\
1	0.107424267371012	0.450897506670668	0.63935210119373\\
1	0.109866204559871	0.455699271686218	0.637974939582442\\
1	0.112336078637817	0.460491870408058	0.636610827579863\\
1	0.114833849738607	0.465275065344603	0.635259539104384\\
1	0.117359475365564	0.470048619052377	0.633920872153155\\
1	0.119912910387022	0.474812294171664	0.632594644371962\\
1	0.122494107032347	0.479565853462319	0.63128068932964\\
1	0.125103014888515	0.484309059839721	0.629978853399059\\
1	0.127739580897283	0.489041676410868	0.628690460824559\\
};
\addplot3 [color=mycolor1,solid,line width=1.5pt]
 table[row sep=crcr] {%
0	0	0\\
0.015	0	0\\
0.03	0	0\\
0.045	0	0\\
0.06	0	0\\
0.075	0	0\\
0.09	0	0\\
0.105	0	0\\
0.12	0	0\\
0.135	0	0\\
0.15	0	0\\
0.165	0	0\\
0.18	0	0\\
0.195	0	0\\
0.21	0	0\\
0.225	0	0\\
0.24	0	0\\
0.255	0	0\\
0.27	0	0\\
0.285	0	0\\
0.3	0	0\\
0.315	0	0\\
0.33	0	0\\
0.345	0	0\\
0.36	0	0\\
0.375	0	0\\
0.39	0	0\\
0.405	0	0\\
0.42	0	0\\
0.435	0	0\\
0.45	0	0\\
0.465	0	0\\
0.48	0	0\\
0.495	0	0\\
0.51	0	0\\
0.525	0	0\\
0.54	0	0\\
0.555	0	0\\
0.57	0	0\\
0.585	0	0\\
0.6	0	0\\
0.615	0	0\\
0.63	0	0\\
0.645	0	0\\
0.66	0	0\\
0.675	0	0\\
0.69	0	0\\
0.705	0	0\\
0.72	0	0\\
0.735	0	0\\
0.75	0	0\\
0.765	0	0\\
0.78	0	0\\
0.795	0	0\\
0.81	0	0\\
0.825	0	0\\
0.84	0	0\\
0.855	0	0\\
0.87	0	0\\
0.885	0	0\\
0.9	0	0\\
0.915	0	0\\
0.93	0	0\\
0.945	0	0\\
0.96	0	0\\
0.975	0	0\\
0.99	0	0\\
1.005	0	0\\
1.02	0	0\\
1.035	0	0\\
1.05	0	0\\
1.065	0	0\\
1.08	0	0\\
1.095	0	0\\
1.11	0	0\\
1.125	0	0\\
1.14	0	0\\
1.155	0	0\\
1.17	0	0\\
1.185	0	0\\
1.2	0	0\\
1.215	0	0\\
1.23	0	0\\
1.245	0	0\\
1.26	0	0\\
1.275	0	0\\
1.29	0	0\\
1.305	0	0\\
1.32	0	0\\
1.335	0	0\\
1.35	0	0\\
1.365	0	0\\
1.38	0	0\\
1.395	0	0\\
1.41	0	0\\
1.425	0	0\\
1.44	0	0\\
1.455	0	0\\
1.47	0	0\\
1.485	0	0\\
1.5	0	0\\
};
 \addplot3 [color=mycolor1,solid,line width=1.5pt]
 table[row sep=crcr] {%
0	0.127739580897283	0.489041676410868\\
0.015	0.127739580897283	0.489041676410868\\
0.03	0.127739580897283	0.489041676410868\\
0.045	0.127739580897283	0.489041676410868\\
0.06	0.127739580897283	0.489041676410868\\
0.075	0.127739580897283	0.489041676410868\\
0.09	0.127739580897283	0.489041676410868\\
0.105	0.127739580897283	0.489041676410868\\
0.12	0.127739580897283	0.489041676410868\\
0.135	0.127739580897283	0.489041676410868\\
0.15	0.127739580897283	0.489041676410868\\
0.165	0.127739580897283	0.489041676410868\\
0.18	0.127739580897283	0.489041676410868\\
0.195	0.127739580897283	0.489041676410868\\
0.21	0.127739580897283	0.489041676410868\\
0.225	0.127739580897283	0.489041676410868\\
0.24	0.127739580897283	0.489041676410868\\
0.255	0.127739580897283	0.489041676410868\\
0.27	0.127739580897283	0.489041676410868\\
0.285	0.127739580897283	0.489041676410868\\
0.3	0.127739580897283	0.489041676410868\\
0.315	0.127739580897283	0.489041676410868\\
0.33	0.127739580897283	0.489041676410868\\
0.345	0.127739580897283	0.489041676410868\\
0.36	0.127739580897283	0.489041676410868\\
0.375	0.127739580897283	0.489041676410868\\
0.39	0.127739580897283	0.489041676410868\\
0.405	0.127739580897283	0.489041676410868\\
0.42	0.127739580897283	0.489041676410868\\
0.435	0.127739580897283	0.489041676410868\\
0.45	0.127739580897283	0.489041676410868\\
0.465	0.127739580897283	0.489041676410868\\
0.48	0.127739580897283	0.489041676410868\\
0.495	0.127739580897283	0.489041676410868\\
0.51	0.127739580897283	0.489041676410868\\
0.525	0.127739580897283	0.489041676410868\\
0.54	0.127739580897283	0.489041676410868\\
0.555	0.127739580897283	0.489041676410868\\
0.57	0.127739580897283	0.489041676410868\\
0.585	0.127739580897283	0.489041676410868\\
0.6	0.127739580897283	0.489041676410868\\
0.615	0.127739580897283	0.489041676410868\\
0.63	0.127739580897283	0.489041676410868\\
0.645	0.127739580897283	0.489041676410868\\
0.66	0.127739580897283	0.489041676410868\\
0.675	0.127739580897283	0.489041676410868\\
0.69	0.127739580897283	0.489041676410868\\
0.705	0.127739580897283	0.489041676410868\\
0.72	0.127739580897283	0.489041676410868\\
0.735	0.127739580897283	0.489041676410868\\
0.75	0.127739580897283	0.489041676410868\\
0.765	0.127739580897283	0.489041676410868\\
0.78	0.127739580897283	0.489041676410868\\
0.795	0.127739580897283	0.489041676410868\\
0.81	0.127739580897283	0.489041676410868\\
0.825	0.127739580897283	0.489041676410868\\
0.84	0.127739580897283	0.489041676410868\\
0.855	0.127739580897283	0.489041676410868\\
0.87	0.127739580897283	0.489041676410868\\
0.885	0.127739580897283	0.489041676410868\\
0.9	0.127739580897283	0.489041676410868\\
0.915	0.127739580897283	0.489041676410868\\
0.93	0.127739580897283	0.489041676410868\\
0.945	0.127739580897283	0.489041676410868\\
0.96	0.127739580897283	0.489041676410868\\
0.975	0.127739580897283	0.489041676410868\\
0.99	0.127739580897283	0.489041676410868\\
1.005	0.127739580897283	0.489041676410868\\
1.02	0.127739580897283	0.489041676410868\\
1.035	0.127739580897283	0.489041676410868\\
1.05	0.127739580897283	0.489041676410868\\
1.065	0.127739580897283	0.489041676410868\\
1.08	0.127739580897283	0.489041676410868\\
1.095	0.127739580897283	0.489041676410868\\
1.11	0.127739580897283	0.489041676410868\\
1.125	0.127739580897283	0.489041676410868\\
1.14	0.127739580897283	0.489041676410868\\
1.155	0.127739580897283	0.489041676410868\\
1.17	0.127739580897283	0.489041676410868\\
1.185	0.127739580897283	0.489041676410868\\
1.2	0.127739580897283	0.489041676410868\\
1.215	0.127739580897283	0.489041676410868\\
1.23	0.127739580897283	0.489041676410868\\
1.245	0.127739580897283	0.489041676410868\\
1.26	0.127739580897283	0.489041676410868\\
1.275	0.127739580897283	0.489041676410868\\
1.29	0.127739580897283	0.489041676410868\\
1.305	0.127739580897283	0.489041676410868\\
1.32	0.127739580897283	0.489041676410868\\
1.335	0.127739580897283	0.489041676410868\\
1.35	0.127739580897283	0.489041676410868\\
1.365	0.127739580897283	0.489041676410868\\
1.38	0.127739580897283	0.489041676410868\\
1.395	0.127739580897283	0.489041676410868\\
1.41	0.127739580897283	0.489041676410868\\
1.425	0.127739580897283	0.489041676410868\\
1.44	0.127739580897283	0.489041676410868\\
1.455	0.127739580897283	0.489041676410868\\
1.47	0.127739580897283	0.489041676410868\\
1.485	0.127739580897283	0.489041676410868\\
1.5	0.127739580897283	0.489041676410868\\
};
 \addplot3 [color=mycolor1,solid,line width=1.5pt]
 table[row sep=crcr] {%
0	0.510958323589132	0.872260419102717\\
0.015	0.510958323589132	0.872260419102717\\
0.03	0.510958323589132	0.872260419102717\\
0.045	0.510958323589132	0.872260419102717\\
0.06	0.510958323589132	0.872260419102717\\
0.075	0.510958323589132	0.872260419102717\\
0.09	0.510958323589132	0.872260419102717\\
0.105	0.510958323589132	0.872260419102717\\
0.12	0.510958323589132	0.872260419102717\\
0.135	0.510958323589132	0.872260419102717\\
0.15	0.510958323589132	0.872260419102717\\
0.165	0.510958323589132	0.872260419102717\\
0.18	0.510958323589132	0.872260419102717\\
0.195	0.510958323589132	0.872260419102717\\
0.21	0.510958323589132	0.872260419102717\\
0.225	0.510958323589132	0.872260419102717\\
0.24	0.510958323589132	0.872260419102717\\
0.255	0.510958323589132	0.872260419102717\\
0.27	0.510958323589132	0.872260419102717\\
0.285	0.510958323589132	0.872260419102717\\
0.3	0.510958323589132	0.872260419102717\\
0.315	0.510958323589132	0.872260419102717\\
0.33	0.510958323589132	0.872260419102717\\
0.345	0.510958323589132	0.872260419102717\\
0.36	0.510958323589132	0.872260419102717\\
0.375	0.510958323589132	0.872260419102717\\
0.39	0.510958323589132	0.872260419102717\\
0.405	0.510958323589132	0.872260419102717\\
0.42	0.510958323589132	0.872260419102717\\
0.435	0.510958323589132	0.872260419102717\\
0.45	0.510958323589132	0.872260419102717\\
0.465	0.510958323589132	0.872260419102717\\
0.48	0.510958323589132	0.872260419102717\\
0.495	0.510958323589132	0.872260419102717\\
0.51	0.510958323589132	0.872260419102717\\
0.525	0.510958323589132	0.872260419102717\\
0.54	0.510958323589132	0.872260419102717\\
0.555	0.510958323589132	0.872260419102717\\
0.57	0.510958323589132	0.872260419102717\\
0.585	0.510958323589132	0.872260419102717\\
0.6	0.510958323589132	0.872260419102717\\
0.615	0.510958323589132	0.872260419102717\\
0.63	0.510958323589132	0.872260419102717\\
0.645	0.510958323589132	0.872260419102717\\
0.66	0.510958323589132	0.872260419102717\\
0.675	0.510958323589132	0.872260419102717\\
0.69	0.510958323589132	0.872260419102717\\
0.705	0.510958323589132	0.872260419102717\\
0.72	0.510958323589132	0.872260419102717\\
0.735	0.510958323589132	0.872260419102717\\
0.75	0.510958323589132	0.872260419102717\\
0.765	0.510958323589132	0.872260419102717\\
0.78	0.510958323589132	0.872260419102717\\
0.795	0.510958323589132	0.872260419102717\\
0.81	0.510958323589132	0.872260419102717\\
0.825	0.510958323589132	0.872260419102717\\
0.84	0.510958323589132	0.872260419102717\\
0.855	0.510958323589132	0.872260419102717\\
0.87	0.510958323589132	0.872260419102717\\
0.885	0.510958323589132	0.872260419102717\\
0.9	0.510958323589132	0.872260419102717\\
0.915	0.510958323589132	0.872260419102717\\
0.93	0.510958323589132	0.872260419102717\\
0.945	0.510958323589132	0.872260419102717\\
0.96	0.510958323589132	0.872260419102717\\
0.975	0.510958323589132	0.872260419102717\\
0.99	0.510958323589132	0.872260419102717\\
1.005	0.510958323589132	0.872260419102717\\
1.02	0.510958323589132	0.872260419102717\\
1.035	0.510958323589132	0.872260419102717\\
1.05	0.510958323589132	0.872260419102717\\
1.065	0.510958323589132	0.872260419102717\\
1.08	0.510958323589132	0.872260419102717\\
1.095	0.510958323589132	0.872260419102717\\
1.11	0.510958323589132	0.872260419102717\\
1.125	0.510958323589132	0.872260419102717\\
1.14	0.510958323589132	0.872260419102717\\
1.155	0.510958323589132	0.872260419102717\\
1.17	0.510958323589132	0.872260419102717\\
1.185	0.510958323589132	0.872260419102717\\
1.2	0.510958323589132	0.872260419102717\\
1.215	0.510958323589132	0.872260419102717\\
1.23	0.510958323589132	0.872260419102717\\
1.245	0.510958323589132	0.872260419102717\\
1.26	0.510958323589132	0.872260419102717\\
1.275	0.510958323589132	0.872260419102717\\
1.29	0.510958323589132	0.872260419102717\\
1.305	0.510958323589132	0.872260419102717\\
1.32	0.510958323589132	0.872260419102717\\
1.335	0.510958323589132	0.872260419102717\\
1.35	0.510958323589132	0.872260419102717\\
1.365	0.510958323589132	0.872260419102717\\
1.38	0.510958323589132	0.872260419102717\\
1.395	0.510958323589132	0.872260419102717\\
1.41	0.510958323589132	0.872260419102717\\
1.425	0.510958323589132	0.872260419102717\\
1.44	0.510958323589132	0.872260419102717\\
1.455	0.510958323589132	0.872260419102717\\
1.47	0.510958323589132	0.872260419102717\\
1.485	0.510958323589132	0.872260419102717\\
1.5	0.510958323589132	0.872260419102717\\
};
 \addplot3 [color=mycolor1,solid,line width=1.5pt]
 table[row sep=crcr] {%
0	1	1\\
0.015	1	1\\
0.03	1	1\\
0.045	1	1\\
0.06	1	1\\
0.075	1	1\\
0.09	1	1\\
0.105	1	1\\
0.12	1	1\\
0.135	1	1\\
0.15	1	1\\
0.165	1	1\\
0.18	1	1\\
0.195	1	1\\
0.21	1	1\\
0.225	1	1\\
0.24	1	1\\
0.255	1	1\\
0.27	1	1\\
0.285	1	1\\
0.3	1	1\\
0.315	1	1\\
0.33	1	1\\
0.345	1	1\\
0.36	1	1\\
0.375	1	1\\
0.39	1	1\\
0.405	1	1\\
0.42	1	1\\
0.435	1	1\\
0.45	1	1\\
0.465	1	1\\
0.48	1	1\\
0.495	1	1\\
0.51	1	1\\
0.525	1	1\\
0.54	1	1\\
0.555	1	1\\
0.57	1	1\\
0.585	1	1\\
0.6	1	1\\
0.615	1	1\\
0.63	1	1\\
0.645	1	1\\
0.66	1	1\\
0.675	1	1\\
0.69	1	1\\
0.705	1	1\\
0.72	1	1\\
0.735	1	1\\
0.75	1	1\\
0.765	1	1\\
0.78	1	1\\
0.795	1	1\\
0.81	1	1\\
0.825	1	1\\
0.84	1	1\\
0.855	1	1\\
0.87	1	1\\
0.885	1	1\\
0.9	1	1\\
0.915	1	1\\
0.93	1	1\\
0.945	1	1\\
0.96	1	1\\
0.975	1	1\\
0.99	1	1\\
1.005	1	1\\
1.02	1	1\\
1.035	1	1\\
1.05	1	1\\
1.065	1	1\\
1.08	1	1\\
1.095	1	1\\
1.11	1	1\\
1.125	1	1\\
1.14	1	1\\
1.155	1	1\\
1.17	1	1\\
1.185	1	1\\
1.2	1	1\\
1.215	1	1\\
1.23	1	1\\
1.245	1	1\\
1.26	1	1\\
1.275	1	1\\
1.29	1	1\\
1.305	1	1\\
1.32	1	1\\
1.335	1	1\\
1.35	1	1\\
1.365	1	1\\
1.38	1	1\\
1.395	1	1\\
1.41	1	1\\
1.425	1	1\\
1.44	1	1\\
1.455	1	1\\
1.47	1	1\\
1.485	1	1\\
1.5	1	1\\
};
 \addplot3 [color=mycolor1,solid,line width=1.5pt]
 table[row sep=crcr] {%
0	0	0\\
0	0.000100583311362513	0.0141829653360114\\
0	0.000404645907256436	0.0284451766772965\\
0	0.000915607803432999	0.0427829086109896\\
0	0.00163681893109844	0.057192295687301\\
0	0.00257155309398959	0.0716693330697585\\
0	0.00372300185733413	0.0862098774609879\\
0	0.00509426838162598	0.100809648312589\\
0	0.00668836121491815	0.115464229327074\\
0	0.00850818805808554	0.13016907025918\\
0	0.0105565495182326	0.144919489023162\\
0	0.0128361328661109	0.159710674111862\\
0	0.0153495058140643	0.174537687332543\\
0	0.0180991103316243	0.189395466863524\\
0	0.0210872565164405	0.204278830634725\\
0	0.0243161165387281	0.219182480034174\\
0	0.0277877186778607	0.234101003941464\\
0	0.0315039414701067	0.24902888308801\\
0	0.0354665079868145	0.263960494742775\\
0	0.0396769802625738	0.278890117720924\\
0	0.0441367538930258	0.293811937711588\\
0	0.0488470528220538	0.308720052919643\\
0	0.0538089243380495	0.323608480015096\\
0	0.0590232342988274	0.338471160382329\\
0	0.064490662604533	0.3533019666601\\
0	0.0702116989375697	0.368094709561873\\
0	0.0761866387881432	0.382843144964686\\
0	0.0824155797834956	0.397540981253432\\
0	0.0888984183382709	0.412181886906127\\
0	0.0956348466427212	0.42675949830445\\
0	0.102624350004627	0.441267427752584\\
0	0.109866204559871	0.455699271686218\\
0	0.117359475365564	0.470048619052377\\
0	0.125103014888515	0.484309059839721\\
0	0.133095461900593	0.498474193737904\\
0	0.14133524079124	0.512537638903683\\
0	0.149820561306021	0.526493040810599\\
0	0.158549418718625	0.540334081158348\\
0	0.167519594442213	0.554054486817265\\
0	0.176728657084455	0.567648038782867\\
0	0.186173963948925	0.581108581114919\\
0	0.195852662983903	0.594430029835236\\
0	0.205761695177907	0.607606381758231\\
0	0.215897797399558	0.620631723228143\\
0	0.226257505677663	0.633500238737009\\
0	0.236837158915675	0.646206219397571\\
0	0.247632903032949	0.658744071245709\\
0	0.258640695523527	0.671108323347402\\
0	0.269856310421543	0.683293635685828\\
0	0.28127534366066	0.695294806804925\\
0	0.292893218813452	0.707106781186547\\
0	0.304705193195075	0.71872465633934\\
0	0.316706364314172	0.730143689578457\\
0	0.328891676652598	0.741359304476472\\
0	0.341255928754291	0.752367096967051\\
0	0.353793780602429	0.763162841084325\\
0	0.366499761262991	0.773742494322337\\
0	0.379368276771857	0.784102202600442\\
0	0.392393618241769	0.794238304822092\\
0	0.405569970164763	0.804147337016097\\
0	0.418891418885081	0.813826036051075\\
0	0.432351961217132	0.823271342915544\\
0	0.445945513182735	0.832480405557786\\
0	0.459665918841652	0.841450581281375\\
0	0.473506959189401	0.850179438693979\\
0	0.487462361096317	0.85866475920876\\
0	0.501525806262096	0.866904538099407\\
0	0.515690940160279	0.874896985111485\\
0	0.529951380947623	0.882640524634436\\
0	0.544300728313782	0.890133795440129\\
0	0.558732572247415	0.897375649995373\\
0	0.57324050169555	0.904365153357279\\
0	0.587818113093873	0.911101581661729\\
0	0.602459018746568	0.917584420216504\\
0	0.617156855035314	0.923813361211857\\
0	0.631905290438127	0.92978830106243\\
0	0.6466980333399	0.935509337395467\\
0	0.661528839617671	0.940976765701173\\
0	0.676391519984904	0.94619107566195\\
0	0.691279947080357	0.951152947177946\\
0	0.706188062288412	0.955863246106974\\
0	0.721109882279076	0.960323019737426\\
0	0.736039505257226	0.964533492013186\\
0	0.75097111691199	0.968496058529893\\
0	0.765898996058536	0.972212281322139\\
0	0.780817519965826	0.975683883461272\\
0	0.795721169365275	0.97891274348356\\
0	0.810604533136476	0.981900889668376\\
0	0.825462312667457	0.984650494185936\\
0	0.840289325888138	0.987163867133889\\
0	0.855080510976839	0.989443450481767\\
0	0.86983092974082	0.991491811941914\\
0	0.884535770672926	0.993311638785082\\
0	0.899190351687411	0.994905731618374\\
0	0.913790122539012	0.996276998142666\\
0	0.928330666930242	0.997428446906011\\
0	0.942807704312699	0.998363181068902\\
0	0.95721709138901	0.999084392196567\\
0	0.971554823322703	0.999595354092743\\
0	0.985817034663989	0.999899416688637\\
0	1	1\\
};
 \addplot3 [color=mycolor1,solid,line width=1.5pt]
 table[row sep=crcr] {%
0.5	0	0\\
0.5	0.000100583311362513	0.0141829653360114\\
0.5	0.000404645907256436	0.0284451766772965\\
0.5	0.000915607803432999	0.0427829086109896\\
0.5	0.00163681893109844	0.057192295687301\\
0.5	0.00257155309398959	0.0716693330697584\\
0.5	0.00372300185733413	0.0862098774609879\\
0.5	0.00509426838162598	0.100809648312589\\
0.5	0.00668836121491816	0.115464229327074\\
0.5	0.00850818805808555	0.13016907025918\\
0.5	0.0105565495182326	0.144919489023162\\
0.5	0.0128361328661109	0.159710674111862\\
0.5	0.0153495058140643	0.174537687332543\\
0.5	0.0180991103316243	0.189395466863524\\
0.5	0.0210872565164405	0.204278830634725\\
0.5	0.0243161165387281	0.219182480034174\\
0.5	0.0277877186778607	0.234101003941464\\
0.5	0.0315039414701067	0.24902888308801\\
0.5	0.0354665079868145	0.263960494742775\\
0.5	0.0396769802625738	0.278890117720924\\
0.5	0.0441367538930258	0.293811937711588\\
0.5	0.0488470528220538	0.308720052919643\\
0.5	0.0538089243380495	0.323608480015096\\
0.5	0.0590232342988274	0.338471160382329\\
0.5	0.064490662604533	0.3533019666601\\
0.5	0.0702116989375697	0.368094709561873\\
0.5	0.0761866387881432	0.382843144964686\\
0.5	0.0824155797834956	0.397540981253432\\
0.5	0.0888984183382709	0.412181886906127\\
0.5	0.0956348466427212	0.42675949830445\\
0.5	0.102624350004627	0.441267427752584\\
0.5	0.109866204559871	0.455699271686218\\
0.5	0.117359475365564	0.470048619052377\\
0.5	0.125103014888515	0.484309059839721\\
0.5	0.133095461900593	0.498474193737904\\
0.5	0.14133524079124	0.512537638903683\\
0.5	0.149820561306021	0.526493040810599\\
0.5	0.158549418718625	0.540334081158348\\
0.5	0.167519594442214	0.554054486817265\\
0.5	0.176728657084455	0.567648038782867\\
0.5	0.186173963948925	0.581108581114919\\
0.5	0.195852662983903	0.594430029835236\\
0.5	0.205761695177907	0.607606381758231\\
0.5	0.215897797399558	0.620631723228143\\
0.5	0.226257505677663	0.633500238737009\\
0.5	0.236837158915675	0.646206219397571\\
0.5	0.247632903032949	0.658744071245709\\
0.5	0.258640695523527	0.671108323347402\\
0.5	0.269856310421543	0.683293635685828\\
0.5	0.28127534366066	0.695294806804924\\
0.5	0.292893218813452	0.707106781186547\\
0.5	0.304705193195075	0.71872465633934\\
0.5	0.316706364314172	0.730143689578457\\
0.5	0.328891676652598	0.741359304476472\\
0.5	0.341255928754291	0.752367096967051\\
0.5	0.353793780602429	0.763162841084325\\
0.5	0.366499761262991	0.773742494322337\\
0.5	0.379368276771857	0.784102202600442\\
0.5	0.392393618241769	0.794238304822092\\
0.5	0.405569970164763	0.804147337016097\\
0.5	0.418891418885081	0.813826036051075\\
0.5	0.432351961217132	0.823271342915544\\
0.5	0.445945513182735	0.832480405557786\\
0.5	0.459665918841652	0.841450581281375\\
0.5	0.473506959189401	0.850179438693979\\
0.5	0.487462361096317	0.85866475920876\\
0.5	0.501525806262096	0.866904538099407\\
0.5	0.515690940160279	0.874896985111485\\
0.5	0.529951380947623	0.882640524634436\\
0.5	0.544300728313782	0.890133795440129\\
0.5	0.558732572247415	0.897375649995373\\
0.5	0.57324050169555	0.904365153357279\\
0.5	0.587818113093873	0.911101581661729\\
0.5	0.602459018746568	0.917584420216504\\
0.5	0.617156855035314	0.923813361211857\\
0.5	0.631905290438127	0.92978830106243\\
0.5	0.6466980333399	0.935509337395467\\
0.5	0.661528839617671	0.940976765701173\\
0.5	0.676391519984904	0.94619107566195\\
0.5	0.691279947080357	0.951152947177946\\
0.5	0.706188062288412	0.955863246106974\\
0.5	0.721109882279076	0.960323019737426\\
0.5	0.736039505257226	0.964533492013186\\
0.5	0.75097111691199	0.968496058529893\\
0.5	0.765898996058536	0.972212281322139\\
0.5	0.780817519965826	0.975683883461272\\
0.5	0.795721169365275	0.97891274348356\\
0.5	0.810604533136476	0.981900889668376\\
0.5	0.825462312667457	0.984650494185936\\
0.5	0.840289325888138	0.987163867133889\\
0.5	0.855080510976839	0.989443450481767\\
0.5	0.86983092974082	0.991491811941914\\
0.5	0.884535770672926	0.993311638785082\\
0.5	0.899190351687411	0.994905731618374\\
0.5	0.913790122539012	0.996276998142666\\
0.5	0.928330666930242	0.997428446906011\\
0.5	0.942807704312699	0.998363181068902\\
0.5	0.95721709138901	0.999084392196567\\
0.5	0.971554823322703	0.999595354092743\\
0.5	0.985817034663989	0.999899416688637\\
0.5	1	1\\
};
 \addplot3 [color=mycolor1,solid,line width=1.5pt]
 table[row sep=crcr] {%
1	0	0\\
1	0.000100583311362513	0.0141829653360114\\
1	0.000404645907256436	0.0284451766772965\\
1	0.000915607803432999	0.0427829086109896\\
1	0.00163681893109844	0.057192295687301\\
1	0.00257155309398959	0.0716693330697584\\
1	0.00372300185733413	0.0862098774609879\\
1	0.00509426838162598	0.100809648312589\\
1	0.00668836121491816	0.115464229327074\\
1	0.00850818805808554	0.13016907025918\\
1	0.0105565495182326	0.144919489023162\\
1	0.0128361328661109	0.159710674111862\\
1	0.0153495058140643	0.174537687332543\\
1	0.0180991103316243	0.189395466863524\\
1	0.0210872565164405	0.204278830634725\\
1	0.0243161165387281	0.219182480034174\\
1	0.0277877186778607	0.234101003941464\\
1	0.0315039414701067	0.24902888308801\\
1	0.0354665079868145	0.263960494742775\\
1	0.0396769802625738	0.278890117720924\\
1	0.0441367538930258	0.293811937711588\\
1	0.0488470528220538	0.308720052919643\\
1	0.0538089243380495	0.323608480015097\\
1	0.0590232342988274	0.338471160382329\\
1	0.064490662604533	0.3533019666601\\
1	0.0702116989375697	0.368094709561873\\
1	0.0761866387881432	0.382843144964686\\
1	0.0824155797834956	0.397540981253432\\
1	0.0888984183382709	0.412181886906127\\
1	0.0956348466427212	0.42675949830445\\
1	0.102624350004627	0.441267427752584\\
1	0.109866204559871	0.455699271686218\\
1	0.117359475365564	0.470048619052377\\
1	0.125103014888515	0.484309059839721\\
1	0.133095461900593	0.498474193737904\\
1	0.14133524079124	0.512537638903683\\
1	0.149820561306021	0.526493040810599\\
1	0.158549418718625	0.540334081158348\\
1	0.167519594442213	0.554054486817265\\
1	0.176728657084455	0.567648038782867\\
1	0.186173963948925	0.581108581114919\\
1	0.195852662983903	0.594430029835237\\
1	0.205761695177907	0.607606381758231\\
1	0.215897797399558	0.620631723228144\\
1	0.226257505677663	0.633500238737009\\
1	0.236837158915675	0.646206219397571\\
1	0.247632903032949	0.658744071245709\\
1	0.258640695523527	0.671108323347402\\
1	0.269856310421543	0.683293635685828\\
1	0.28127534366066	0.695294806804924\\
1	0.292893218813452	0.707106781186547\\
1	0.304705193195075	0.71872465633934\\
1	0.316706364314172	0.730143689578457\\
1	0.328891676652598	0.741359304476472\\
1	0.341255928754291	0.752367096967051\\
1	0.353793780602429	0.763162841084325\\
1	0.366499761262991	0.773742494322337\\
1	0.379368276771857	0.784102202600443\\
1	0.392393618241769	0.794238304822092\\
1	0.405569970164763	0.804147337016097\\
1	0.418891418885081	0.813826036051075\\
1	0.432351961217133	0.823271342915545\\
1	0.445945513182735	0.832480405557786\\
1	0.459665918841652	0.841450581281375\\
1	0.473506959189401	0.850179438693979\\
1	0.487462361096317	0.85866475920876\\
1	0.501525806262096	0.866904538099407\\
1	0.515690940160279	0.874896985111485\\
1	0.529951380947623	0.882640524634436\\
1	0.544300728313782	0.890133795440129\\
1	0.558732572247415	0.897375649995373\\
1	0.57324050169555	0.904365153357279\\
1	0.587818113093873	0.911101581661729\\
1	0.602459018746568	0.917584420216504\\
1	0.617156855035314	0.923813361211857\\
1	0.631905290438127	0.92978830106243\\
1	0.6466980333399	0.935509337395467\\
1	0.661528839617671	0.940976765701173\\
1	0.676391519984903	0.94619107566195\\
1	0.691279947080357	0.951152947177946\\
1	0.706188062288412	0.955863246106974\\
1	0.721109882279076	0.960323019737426\\
1	0.736039505257226	0.964533492013186\\
1	0.75097111691199	0.968496058529893\\
1	0.765898996058536	0.972212281322139\\
1	0.780817519965826	0.975683883461272\\
1	0.795721169365275	0.978912743483559\\
1	0.810604533136476	0.981900889668376\\
1	0.825462312667457	0.984650494185936\\
1	0.840289325888138	0.987163867133889\\
1	0.855080510976838	0.989443450481767\\
1	0.86983092974082	0.991491811941914\\
1	0.884535770672926	0.993311638785082\\
1	0.899190351687411	0.994905731618374\\
1	0.913790122539012	0.996276998142666\\
1	0.928330666930242	0.997428446906011\\
1	0.942807704312699	0.998363181068902\\
1	0.95721709138901	0.999084392196567\\
1	0.971554823322703	0.999595354092743\\
1	0.985817034663989	0.999899416688637\\
1	1	1\\
};
 \addplot3 [color=mycolor1,solid,line width=1.5pt]
 table[row sep=crcr] {%
1.5	0	0\\
1.5	0.000100583311362513	0.0141829653360114\\
1.5	0.000404645907256436	0.0284451766772965\\
1.5	0.000915607803432999	0.0427829086109896\\
1.5	0.00163681893109844	0.057192295687301\\
1.5	0.00257155309398959	0.0716693330697585\\
1.5	0.00372300185733413	0.0862098774609879\\
1.5	0.00509426838162598	0.100809648312589\\
1.5	0.00668836121491815	0.115464229327074\\
1.5	0.00850818805808554	0.13016907025918\\
1.5	0.0105565495182326	0.144919489023162\\
1.5	0.0128361328661109	0.159710674111862\\
1.5	0.0153495058140643	0.174537687332543\\
1.5	0.0180991103316243	0.189395466863524\\
1.5	0.0210872565164405	0.204278830634725\\
1.5	0.0243161165387281	0.219182480034174\\
1.5	0.0277877186778607	0.234101003941464\\
1.5	0.0315039414701067	0.24902888308801\\
1.5	0.0354665079868145	0.263960494742775\\
1.5	0.0396769802625738	0.278890117720924\\
1.5	0.0441367538930258	0.293811937711588\\
1.5	0.0488470528220538	0.308720052919643\\
1.5	0.0538089243380495	0.323608480015096\\
1.5	0.0590232342988274	0.338471160382329\\
1.5	0.064490662604533	0.3533019666601\\
1.5	0.0702116989375697	0.368094709561873\\
1.5	0.0761866387881432	0.382843144964686\\
1.5	0.0824155797834956	0.397540981253432\\
1.5	0.0888984183382709	0.412181886906127\\
1.5	0.0956348466427212	0.42675949830445\\
1.5	0.102624350004627	0.441267427752584\\
1.5	0.109866204559871	0.455699271686218\\
1.5	0.117359475365564	0.470048619052377\\
1.5	0.125103014888515	0.484309059839721\\
1.5	0.133095461900593	0.498474193737904\\
1.5	0.14133524079124	0.512537638903683\\
1.5	0.149820561306021	0.526493040810599\\
1.5	0.158549418718625	0.540334081158348\\
1.5	0.167519594442213	0.554054486817265\\
1.5	0.176728657084455	0.567648038782867\\
1.5	0.186173963948925	0.581108581114919\\
1.5	0.195852662983903	0.594430029835236\\
1.5	0.205761695177907	0.607606381758231\\
1.5	0.215897797399558	0.620631723228143\\
1.5	0.226257505677663	0.633500238737009\\
1.5	0.236837158915675	0.646206219397571\\
1.5	0.247632903032949	0.658744071245709\\
1.5	0.258640695523527	0.671108323347402\\
1.5	0.269856310421543	0.683293635685828\\
1.5	0.28127534366066	0.695294806804925\\
1.5	0.292893218813452	0.707106781186547\\
1.5	0.304705193195075	0.71872465633934\\
1.5	0.316706364314172	0.730143689578457\\
1.5	0.328891676652598	0.741359304476472\\
1.5	0.341255928754291	0.752367096967051\\
1.5	0.353793780602429	0.763162841084325\\
1.5	0.366499761262991	0.773742494322337\\
1.5	0.379368276771857	0.784102202600442\\
1.5	0.392393618241769	0.794238304822092\\
1.5	0.405569970164763	0.804147337016097\\
1.5	0.418891418885081	0.813826036051075\\
1.5	0.432351961217132	0.823271342915544\\
1.5	0.445945513182735	0.832480405557786\\
1.5	0.459665918841652	0.841450581281375\\
1.5	0.473506959189401	0.850179438693979\\
1.5	0.487462361096317	0.85866475920876\\
1.5	0.501525806262096	0.866904538099407\\
1.5	0.515690940160279	0.874896985111485\\
1.5	0.529951380947623	0.882640524634436\\
1.5	0.544300728313782	0.890133795440129\\
1.5	0.558732572247415	0.897375649995373\\
1.5	0.57324050169555	0.904365153357279\\
1.5	0.587818113093873	0.911101581661729\\
1.5	0.602459018746568	0.917584420216504\\
1.5	0.617156855035314	0.923813361211857\\
1.5	0.631905290438127	0.92978830106243\\
1.5	0.6466980333399	0.935509337395467\\
1.5	0.661528839617671	0.940976765701173\\
1.5	0.676391519984904	0.94619107566195\\
1.5	0.691279947080357	0.951152947177946\\
1.5	0.706188062288412	0.955863246106974\\
1.5	0.721109882279076	0.960323019737426\\
1.5	0.736039505257226	0.964533492013186\\
1.5	0.75097111691199	0.968496058529893\\
1.5	0.765898996058536	0.972212281322139\\
1.5	0.780817519965826	0.975683883461272\\
1.5	0.795721169365275	0.97891274348356\\
1.5	0.810604533136476	0.981900889668376\\
1.5	0.825462312667457	0.984650494185936\\
1.5	0.840289325888138	0.987163867133889\\
1.5	0.855080510976839	0.989443450481767\\
1.5	0.86983092974082	0.991491811941914\\
1.5	0.884535770672926	0.993311638785082\\
1.5	0.899190351687411	0.994905731618374\\
1.5	0.913790122539012	0.996276998142666\\
1.5	0.928330666930242	0.997428446906011\\
1.5	0.942807704312699	0.998363181068902\\
1.5	0.95721709138901	0.999084392196567\\
1.5	0.971554823322703	0.999595354092743\\
1.5	0.985817034663989	0.999899416688637\\
1.5	1	1\\
};
 \addplot3 [draw=none, mark size=3.3pt, scatter,mark=ball,scatter/use mapped color={ball color=red},scatter src=rand,only marks,z buffer=sort]
 table[row sep=crcr] {%
0	0	0\\
0	0	0.261203874963741\\
0	0.265409196609864	0.734590803390136\\
0	0.738796125036259	1\\
0	1	1\\
0.25	0	0\\
0.25	0	0.261203874963741\\
0.25	0.265409196609864	0.734590803390136\\
0.25	0.738796125036259	1\\
0.25	1	1\\
0.75	0	0\\
0.75	0	0.261203874963741\\
0.75	0.265409196609864	0.734590803390136\\
0.75	0.738796125036259	1\\
0.75	1	1\\
1.25	0	0\\
1.25	0	0.261203874963741\\
1.25	0.265409196609864	0.734590803390136\\
1.25	0.738796125036259	1\\
1.25	1	1\\
1.5	0	0\\
1.5	0	0.261203874963741\\
1.5	0.265409196609864	0.734590803390136\\
1.5	0.738796125036259	1\\
1.5	1	1\\
};
 \addplot3 [color=black,dashed]
 table[row sep=crcr] {%
0	0	0\\
0.25	0	0\\
0.75	0	0\\
1.25	0	0\\
1.5	0	0\\
};
 \addplot3 [color=black,dashed]
 table[row sep=crcr] {%
0	0	0.261203874963741\\
0.25	0	0.261203874963741\\
0.75	0	0.261203874963741\\
1.25	0	0.261203874963741\\
1.5	0	0.261203874963741\\
};
 \addplot3 [color=black,dashed]
 table[row sep=crcr] {%
0	0.265409196609864	0.734590803390136\\
0.25	0.265409196609864	0.734590803390136\\
0.75	0.265409196609864	0.734590803390136\\
1.25	0.265409196609864	0.734590803390136\\
1.5	0.265409196609864	0.734590803390136\\
};
 \addplot3 [color=black,dashed]
 table[row sep=crcr] {%
0	0.738796125036259	1\\
0.25	0.738796125036259	1\\
0.75	0.738796125036259	1\\
1.25	0.738796125036259	1\\
1.5	0.738796125036259	1\\
};
 \addplot3 [color=black,dashed]
 table[row sep=crcr] {%
0	1	1\\
0.25	1	1\\
0.75	1	1\\
1.25	1	1\\
1.5	1	1\\
};
 \addplot3 [color=black,dashed]
 table[row sep=crcr] {%
0	0	0\\
0	0	0.261203874963741\\
0	0.265409196609864	0.734590803390136\\
0	0.738796125036259	1\\
0	1	1\\
};
 \addplot3 [color=black,dashed]
 table[row sep=crcr] {%
0.25	0	0\\
0.25	0	0.261203874963741\\
0.25	0.265409196609864	0.734590803390136\\
0.25	0.738796125036259	1\\
0.25	1	1\\
};
 \addplot3 [color=black,dashed]
 table[row sep=crcr] {%
0.75	0	0\\
0.75	0	0.261203874963741\\
0.75	0.265409196609864	0.734590803390136\\
0.75	0.738796125036259	1\\
0.75	1	1\\
};
 \addplot3 [color=black,dashed]
 table[row sep=crcr] {%
1.25	0	0\\
1.25	0	0.261203874963741\\
1.25	0.265409196609864	0.734590803390136\\
1.25	0.738796125036259	1\\
1.25	1	1\\
};
 \addplot3 [color=black,dashed]
 table[row sep=crcr] {%
1.5	0	0\\
1.5	0	0.261203874963741\\
1.5	0.265409196609864	0.734590803390136\\
1.5	0.738796125036259	1\\
1.5	1	1\\
};
\addplot3 [color=cyan, dashed, -stealth,line width=1.5pt, postaction={decorate,decoration={text along path,
              text={$\xi$} {--} direction, raise=1ex, text align={center}, text color={cyan},
          }}]
 table[row sep=crcr] {%
0	0	0\\
0	0.000100583311362513	0.0141829653360114\\
0	0.000404645907256436	0.0284451766772965\\
0	0.000915607803433	0.0427829086109896\\
0	0.00163681893109844	0.057192295687301\\
0	0.00257155309398959	0.0716693330697585\\
0	0.00372300185733414	0.086209877460988\\
0	0.00509426838162598	0.100809648312589\\
0	0.00668836121491816	0.115464229327074\\
0	0.00850818805808555	0.13016907025918\\
0	0.0105565495182326	0.144919489023162\\
0	0.0128361328661109	0.159710674111862\\
0	0.0153495058140643	0.174537687332543\\
0	0.0180991103316243	0.189395466863524\\
0	0.0210872565164405	0.204278830634725\\
0	0.0243161165387281	0.219182480034174\\
0	0.0277877186778607	0.234101003941464\\
0	0.0315039414701067	0.24902888308801\\
0	0.0354665079868145	0.263960494742775\\
0	0.0396769802625738	0.278890117720924\\
0	0.0441367538930258	0.293811937711588\\
0	0.0488470528220538	0.308720052919643\\
0	0.0538089243380495	0.323608480015096\\
0	0.0590232342988274	0.338471160382329\\
0	0.064490662604533	0.3533019666601\\
0	0.0702116989375697	0.368094709561873\\
0	0.0761866387881432	0.382843144964686\\
0	0.0824155797834956	0.397540981253432\\
0	0.0888984183382709	0.412181886906127\\
0	0.0956348466427212	0.42675949830445\\
0	0.102624350004627	0.441267427752584\\
0	0.109866204559871	0.455699271686219\\
0	0.117359475365564	0.470048619052377\\
0	0.125103014888515	0.484309059839721\\
0	0.133095461900593	0.498474193737904\\
0	0.14133524079124	0.512537638903683\\
0	0.149820561306021	0.526493040810599\\
0	0.158549418718625	0.540334081158348\\
0	0.167519594442214	0.554054486817265\\
0	0.176728657084455	0.567648038782867\\
0	0.186173963948925	0.581108581114919\\
0	0.195852662983903	0.594430029835237\\
0	0.205761695177907	0.607606381758231\\
0	0.215897797399558	0.620631723228143\\
0	0.226257505677663	0.633500238737009\\
0	0.236837158915675	0.646206219397571\\
0	0.247632903032949	0.658744071245709\\
0	0.258640695523528	0.671108323347402\\
0	0.269856310421543	0.683293635685828\\
0	0.28127534366066	0.695294806804925\\
0	0.292893218813452	0.707106781186547\\
0	0.304705193195075	0.71872465633934\\
};
\coordinate (A01) at (axis cs: 0, 0, 0);
\coordinate (A02) at (axis cs: 0, 0, 0.261203874963741);
\coordinate (A03) at (axis cs: 0, 0.265409196609864, 0.734590803390136);
\coordinate (A04) at (axis cs: 0, 0.738796125036259, 1);
\coordinate (A05) at (axis cs: 0, 1, 1);
\coordinate (A06) at (axis cs: 0.25, 0, 0);
\coordinate (A07) at (axis cs: 0.25, 0, 0.261203874963741);
\coordinate (A08) at (axis cs: 0.25, 0.265409196609864, 0.734590803390136);
\coordinate (A09) at (axis cs: 0.25, 0.738796125036259, 1);
\coordinate (A10) at (axis cs: 0.25, 1, 1);
\coordinate (A11) at (axis cs: 0.75, 0, 0);
\coordinate (A12) at (axis cs: 0.75, 0, 0.261203874963741);
\coordinate (A13) at (axis cs: 0.75, 0.265409196609864, 0.734590803390136);
\coordinate (A14) at (axis cs: 0.75, 0.738796125036259, 1);
\coordinate (A15) at (axis cs: 0.75, 1, 1);
\coordinate (A16) at (axis cs: 1.25, 0, 0);
\coordinate (A17) at (axis cs: 1.25, 0, 0.261203874963741);
\coordinate (A18) at (axis cs: 1.25, 0.265409196609864, 0.734590803390136);
\coordinate (A19) at (axis cs: 1.25, 0.738796125036259, 1);
\coordinate (A20) at (axis cs: 1.25, 1, 1);
\coordinate (A21) at (axis cs: 1.5, 0, 0);
\coordinate (A22) at (axis cs: 1.5, 0, 0.261203874963741);
\coordinate (A23) at (axis cs: 1.5, 0.265409196609864, 0.734590803390136);
\coordinate (A24) at (axis cs: 1.5, 0.738796125036259, 1);
\coordinate (A25) at (axis cs: 1.5, 1, 1);
\addplot3 [draw=none, mark size=2.0pt, scatter,mark=ball,scatter/use mapped color={ball color=blue},scatter src=rand,only marks,z buffer=sort]
 table[row sep=crcr] {%
0.556350832689629	0.00144183368737608	0.0536804293050094\\
};
\coordinate (gp) at (axis cs: 0.556350832689629, 0.00144183368737608, 0.0536804293050094);

\coordinate (A) at (axis cs: 0, 0, 0);
\coordinate (X) at (axis cs: 0.75, 0, 0);
\end{axis}
%\node[below] at (A01) {$\mathbf{P}_{1}$};
%\node[below] at (A02) {$\mathbf{P}_{2}$};
%\node[below] at (A03) {$\mathbf{P}_{3}$};
%\node[below] at (A04) {$\mathbf{P}_{4}$};
%\node[above] at (A05) {$\mathbf{P}_{5}$};
\node[right] at (A06) {$\mathbf{1}$};
\node[right] at (A07) {$\mathbf{2}$};
\node[right] at (A08) {$\mathbf{3}$};
%\node[below] at (A09) {$\mathbf{P}_{9}$};
%\node[above] at (A10) {$\mathbf{P}_{10}$};
\node[right] at (A11) {$\mathbf{4}$};
\node[right] at (A12) {$\mathbf{5}$};
\node[right] at (A13) {$\mathbf{6}$};
%\node[below] at (A14) {$\mathbf{P}_{14}$};
%\node[above] at (A15) {$\mathbf{P}_{15}$};
\node[right] at (A16) {$\mathbf{7}$};
\node[right] at (A17) {$\mathbf{8}$};
\node[right] at (A18) {$\mathbf{9}$};
%\node[below] at (A19) {$\mathbf{P}_{19}$};
%\node[above] at (A20) {$\mathbf{P}_{20}$};
%\node[below] at (A21) {$\mathbf{P}_{21}$};
%\node[below] at (A22) {$\mathbf{P}_{22}$};
%\node[below] at (A23) {$\mathbf{P}_{23}$};
%\node[below] at (A24) {$\mathbf{P}_{24}$};
%\node[above] at (A25) {$\mathbf{P}_{25}$};
\begin{scope}[shift = {(0.0cm, -1.5cm)}, scale=1, thin]
    \scriptsize
    \draw[fill=gray!10, fill opacity=0.70] (0, 0) rectangle (14, 0.5);
    {\fontencoding{T1}\selectfont \ttfamily
    \node [rectangle, anchor=west, fill=red!50, fill opacity=0.70, align=center, text width=0.5cm, left] at (0, 0.25) {$R$};

    \node [rectangle, anchor=west, fill=red!50, fill opacity=0.70, align=center, above] at (1, 0.5) {$1$};
    \node [rectangle, anchor=west, fill=red!50, fill opacity=0.70, align=center, above] at (2.5, 0.5) {$2$};
    \node [rectangle, anchor=west, fill=red!50, fill opacity=0.70, align=center, above] at (4.0, 0.5) {$3$};
    \node [rectangle, anchor=west, fill=red!50, fill opacity=0.70, align=center, above] at (5.5, 0.5) {$4$};
    \node [rectangle, anchor=west, fill=red!50, fill opacity=0.70, align=center, above] at (7.0, 0.5) {$5$};
    \node [rectangle, anchor=west, fill=red!50, fill opacity=0.70, align=center, above] at (8.5, 0.5) {$6$};
    \node [rectangle, anchor=west, fill=red!50, fill opacity=0.70, align=center, above] at (10.0, 0.5) {$7$};
    \node [rectangle, anchor=west, fill=red!50, fill opacity=0.70, align=center, above] at (11.5, 0.5) {$8$};
    \node [rectangle, anchor=west, fill=red!50, fill opacity=0.70, align=center, above] at (13.0, 0.5) {$9$};

    \node at (1, 0.25) {3.166e-1};
    \node at (2.5, 0.25) {7.489e-2};
    \node at (4.0, 0.25) {2.139e-3};
    \node at (5.5, 0.25) {4.826e-1};
    \node at (7.0, 0.25) {1.141e-1};
    \node at (8.5, 0.25) {3.256e-3};
    \node at (10, 0.25) {5.108e-3};
    \node at (11.5, 0.25) {1.208e-3};
    \node at (13, 0.25) {3.450e-5};
    %----------------------------------
    }
\end{scope}
\begin{scope}[shift = {(0.0cm, -3.0cm)}, scale=1, thin]
    \scriptsize
    \draw[fill=gray!10, fill opacity=0.70] (0, 0) rectangle (14, 1.0);
    {\fontencoding{T1}\selectfont \ttfamily
        \node [rectangle, anchor=west, fill=red!50, fill opacity=0.70, align=center, text width=1.0cm, left] at (0, 0.75) {$\dif R/ \dif \xi$};
        \node [rectangle, anchor=west, fill=red!50, fill opacity=0.70, align=center, text width=1.0cm, left] at (0, 0.25) {$\dif R/\dif\eta$};

        \node [rectangle, anchor=west, fill=red!50, fill opacity=0.70, align=center, above] at (1, 1.0) {$1$};
        \node [rectangle, anchor=west, fill=red!50, fill opacity=0.70, align=center, above] at (2.5, 1.0) {$2$};
        \node [rectangle, anchor=west, fill=red!50, fill opacity=0.70, align=center, above] at (4.0, 1.0) {$3$};
        \node [rectangle, anchor=west, fill=red!50, fill opacity=0.70, align=center, above] at (5.5, 1.0) {$4$};
        \node [rectangle, anchor=west, fill=red!50, fill opacity=0.70, align=center, above] at (7.0, 1.0) {$5$};
        \node [rectangle, anchor=west, fill=red!50, fill opacity=0.70, align=center, above] at (8.5, 1.0) {$6$};
        \node [rectangle, anchor=west, fill=red!50, fill opacity=0.70, align=center, above] at (10, 1.0) {$7$};
        \node [rectangle, anchor=west, fill=red!50, fill opacity=0.70, align=center, above] at (11.5, 1.0) {$8$};
        \node [rectangle, anchor=west, fill=red!50, fill opacity=0.70, align=center, above] at (13, 1.0) {$9$};

        \node at (1.0, 0.25) {-2.141e+0};
        \node at (2.5, 0.25) {-5.064e-1};
        \node at (4.0, 0.25) {-1.446e-2};
        \node at (5.5, 0.25) {1.869e0};
        \node at (7, 0.25) {4.421e-1};
        \node at (8.5, 0.25) {1.262e-2};
        \node at (10, 0.25) {2.720e-1};
        \node at (11.5, 0.25) {6.432e-2};
        \node at (13, 0.25) {1.837e-3};

        \node at (1.0, 0.75) {-1.966e+0};
        \node at (2.5, 0.75) {1.851e+0};
        \node at (4.0, 0.75) {1.150e-1};
        \node at (5.5, 0.75) {-2.996e+0};
        \node at (7, 0.75) {2.821e+0};
        \node at (8.5, 0.75) {1.753e-1};
        \node at (10, 0.75) {-3.172e-2};
        \node at (11.5, 0.75) {2.986e-2};
        \node at (13, 0.75) {1.856e-03};
        %----------------------------------
        }
\end{scope}
\begin{scope}[shift = {(9.0cm, 0.0cm)}, scale=1, thin]
    \fill[fill=mycolor2] (0, 1) -- (1, 1) -- (1, 2) -- (0, 2) -- cycle;
    \FPset{\xzero}{0.0};
    \FPneg{\yzero}{0.0};
    \FPset{\ab}{3};
    \FPeval{\yone}{yzero + ab};
    \FPeval{\xone}{xzero + ab};
    \draw[thick] (\xzero,\yzero) rectangle (\xone, \yone);

    \FPeval{\length}{ab/3};
    \FPeval{\xnextone}{xzero+length};
    \FPeval{\xnexttwo}{xzero+2*length};

    \FPeval{\ynextone}{yzero+length};
    \FPeval{\ynexttwo}{yzero+2*length};

    \pgfplotsinvokeforeach{\xnextone,\xnexttwo}
    {
        \draw[thick] (#1,\yzero) -- (#1,\yone);
    };
    \pgfplotsinvokeforeach{\ynextone,\ynexttwo}
    {
        \draw[thick] (\xzero,#1) -- (\xone,#1);
    };
    \fill[blue] (0.112701, 1.1127) circle (1.5pt);
%    \fill[blue] (0.112701, 1.5) circle (1.5pt);
%    \fill[blue] (0.112701, 1.8873) circle (1.5pt);
%
%    \fill[blue] (0.50001, 1.1127) circle (1.5pt);
%    \fill[blue] (0.50001, 1.5) circle (1.5pt);
%    \fill[blue] (0.50001, 1.8873) circle (1.5pt);
%
%    \fill[blue] (0.88731, 1.1127) circle (1.5pt);
%    \fill[blue] (0.88731, 1.5) circle (1.5pt);
%    \fill[blue] (0.88731, 1.8873) circle (1.5pt);

    \draw[->,>=stealth] (0.0, 0.0) -- (4.0, 0.0) node[at end,above] {$\xi$};
    \draw[->,>=stealth] (0.0, 0.0) -- (0.0, 4.0) node[at end,right] {$\eta$};

    \node[below] at (1.5, 0.0) {Parametric space};
\end{scope}
\begin{scope}[shift = {(11.0cm, 5.0cm)}, scale=1, thin]
    \draw[thick, fill = mycolor2] (0.0, 0.0) rectangle (1.5, 1.5);
    \draw[->,>=stealth] (0.75, 0.75) -- (2.0, 0.75) node[at end,above] {$\wtilde{\xi}$};
    \draw[->,>=stealth] (0.75, 0.75) -- (0.75, 2.0) node[at end,right] {$\wtilde{\eta}$};
    \node[left] at (0.0, 0.0) {$(-1,-1)$};
    \node[right,yshift=0.1cm] at (1.5, 1.5) {$(1,1)$};
    \node[below] at (0.75, 0.0) {Parent space};

    \fill[blue] (0.16905, 0.16905) circle (1.5pt);
    \fill[blue] (0.16905, 0.75) circle (1.5pt);
    \fill[blue] (0.16905, 1.33095) circle (1.5pt);

    \fill[blue] (0.75, 0.16905) circle (1.5pt);
    \fill[blue] (0.75, 0.75) circle (1.5pt);
    \fill[blue] (0.75, 1.33095) circle (1.5pt);

    \fill[blue] (1.33095, 0.16905) circle (1.5pt);
    \fill[blue] (1.33095, 0.75) circle (1.5pt);
    \fill[blue] (1.33095, 1.33095) circle (1.5pt);
\end{scope}
\draw[-stealth, thick] [postaction={decorate,decoration={text along path, raise=4pt, text align={left indent = 3.7cm}, reverse path,text={${g}{(\tilde{\boldsymbol{\xi}})}$ {} }}}] (11.16905, 5.16905) .. controls (12.0, 3.0) and (11.0, 1.0) .. (9.112701, 1.1127);

\draw[-stealth, thick] [postaction={decorate,decoration={text along path, raise=4pt, text align = {left indent = 4.0cm}, reverse path,text={${\mathbf{F}}{(\boldsymbol{\xi})}$ {} }}}] (9.112701, 1.1127) .. controls (8.0, 3.0) and (5.5, 2.0) .. (gp);

\draw[-stealth, dashed, line width=1.5pt, color=cyan] (A) -- node[below, midway, sloped] {$\eta$ -- direction} (X);

\draw[-latex, thick](7.0, 0.5)node[below]{evaluated point} to[out=45, in=180] (9.112701, 1.1127);
\end{tikzpicture} 
    \caption{Local numbering of the fourth element with basis function values and their first derivatives evaluated at the first Gaussian point of this element.}
    \label{fig:Ch3SurfAQuarterOfACylinderLocalEnum}
\end{figure}

The purpose of the element-to-control point connectivity array is to scatter local ordering of control points of the element to the global one. For the considering example, the array in question is presented in the right hand side of the Fig.~\ref{fig:Ch3SurfAQuarterOfACylinderEnum}, where the row indices represent the element numbers and the values stored in each row are global control point numbers pertain to each element. In SIMOPackage, the routine to generate this array is currently implemented in the files named \lstinline{Mesh1D} for a 1D patch, \lstinline{Mesh2D} for a 2D patch and \lstinline{Mesh3D} for a 3D patch, respectively. For example, to retrieve the connectivity array of the element 4, we invoke
\begin{lstlisting}
    Mesh = Mesh2D(Surf, 'VectorField');
    ElConn = Mesh.El(4, :);
\end{lstlisting}
Notice that, this array is solely a control point connectivity array, not a DOFs (Degrees Of Freedom) connectivity array. Naturally, it is a DOFs connectivity array for single field problems, but for multiple fields problems, we have to establish a mapping to relate the control point numbers and their control variables (component values). Let's clarify this matter by investigating a 2D elasticity problem where each control point of this problem has 2 degrees of freedom (2 component values). In this analysis, the final step of the processing stage is to solve the linear algebra system
\begin{equation}
    \mathbf{K} \mathbf{d} = \mathbf{F},
\end{equation}
which require some kind of ordering of component values of control points. There are two types of ordering used widely in practice. The first one is to alternate component values of each control point, the second one is to gather component values which have the same attribute ($x$ or $y$ displacement in this case) into adjacent fragments. These both types are applied for vector $\mathbf{d}$ as follow
%\begin{bmatrix}
%        K_{1,1}   & K_{1,2}   & K_{1,3}   & K_{1,4}    & \cdots & K_{1,n-1}   & K_{1,n} \\
%        K_{2,1}   & K_{2,2}   & K_{2,3}   & K_{2,4}    & \cdots & K_{2,n-1}     & K_{2,n} \\
%        K_{3,1}   & K_{3,2}   & K_{3,3}   & K_{3,4}    & \cdots & K_{3,n-1}     & K_{3,n} \\
%        K_{4,1}   & K_{4,2}   & K_{4,3}   & K_{4,4}    & \cdots & K_{4,n-1}     & K_{4,n} \\
%        \vdots    & \vdots    & \vdots    & \vdots     & \ddots & \vdots     & \vdots \\
%        K_{n-1,1} & K_{n-1,2} & K_{n-1,3} &  K_{n-1,4} & \cdots & K_{n-1,n-1}     & K_{n-1,n} \\
%        K_{n,1}   & K_{n,2}   & K_{n,3}   & K_{n,4}    & \cdots & K_{n,n-1} & K_{n,n}
%\end{bmatrix}
\begin{equation}
    \mathbf{d} = \LEFTRIGHT\lcbrace\rcbrace{
    \begin{matrix}
        u_{1,x} \\
        u_{1,y} \\
        u_{2,x} \\
        u_{2,y} \\
        \vdots \\
        u_{n,x} \\
        u_{n,y}
    \end{matrix}
    }, \qquad
    \mathbf{d} = \LEFTRIGHT\lcbrace\rcbrace{
    \begin{matrix}
        u_{1,x} \\
        u_{2,x} \\
        \vdots \\
        u_{n,x} \\
        u_{1,y} \\
        u_{2,y} \\
        \vdots \\
        u_{n,y}
    \end{matrix}
    },
\end{equation}
where the left-hand side and the right-hand side represent for the first and the second approach, respectively, the first index $i$ in $u_{i, j}$ identify the control point numbers and the second one $j$ identify the component of the field. Of course, we need to specify the ordering of entries in matrix $\mathbf{K}$ and vector $\mathbf{F}$ as well to adapt to vector $\mathbf{d}$. To do this, we can perform the ordering on the element level by arranging basis functions. We utilize the second approach in SIMOPackage, thus the basis functions matrix and the strain-displacement matrix have the following form
\begin{equation}
\begin{gathered}
        \mathbf{N} = %
    \undercbrace{%
        \begin{bmatrix}
          N_1 & N_2 & \ldots & N_{n_{en}} & 0   & 0   & 0      & 0 \\
          0   & 0  & 0      & 0      & N_1 & N_2 & \ldots & N_{n_{en}}
        \end{bmatrix}}_{2 \times 2n_{en}},\\
    \mathbf{B} =
        \begin{bmatrix}
          \dpd{N_1}{x} & \dpd{N_2}{x} & \ldots & \dpd{N_{n_{en}}}{x} & 0 & 0 & \ldots & 0\\[10pt]
          0 & 0 & \ldots & 0 & \dpd{N_1}{y} & \dpd{N_2}{y} & \ldots & \dpd{N_{n_{en}}}{y} \\[10pt]
          \dpd{N_1}{y} & \dpd{N_2}{y} & \ldots & \dpd{N_{n_{en}}}{y} & \dpd{N_1}{x} & \dpd{N_2}{x} & \ldots & \dpd{N_{n_{en}}}{x}
        \end{bmatrix},
\end{gathered}
\end{equation}
the elemental matrix is then scattered to the global one via a DOFs connectivity array adapted from the element-to-control points connectivity array by element-wise adding the total number of control points as
\begin{lstlisting}
    for el = 1 : Mesh.NEl
        ElConn = Mesh.El(el, :);
        DOFConn = [ElConn, ElConn + NURBS.NNP];
    end
\end{lstlisting}

For reasons of simplicity, local basis functions (one-to-one correspondence with control points) are numbered in the same manner as the global one (number in the $\xi$ direction first, followed by $\eta$) and this process is performed implicitly in SimoPackage. Let's discuss further about this, in order to compute elemental matrices, multivariate basis functions in company with their derivatives are calculated based on tensor product and stored in a 2D array as illustrated in the Fig.~\ref{fig:Ch3SurfAQuarterOfACylinderLocalEnum} for the evaluated basis function values and derivatives at the first Gaussian point of the element. Therefore, we can simply control local numbering by putting basis functions values into the array at the corresponding indices. Assuming 2D basis functions and derivatives are need to be calculated for the element \lstinline{e = sub2ind(Mesh.NElDir, ex, ey)} at the Gaussian point (\lstinline{qx, qy}), then this work can be accomplished by employing two nested loops as
\begin{lstlisting}
    N0 = zeros(1, Mesh.NEN);
    N1 = zeros(2, Mesh.NEN);
    for j = 1 : NURBS.Order(2) + 1 % loop over basis functions in eta direction
        for i = 1 : NURBS.Order(1) + 1 % loop over basis functions in xi direction
            N0(k) = Nx(ex, qx, i, 1)*Ny(ey, qy, j, 1);
            N1(1, k) = Nx(ex, qx, i, 2)*Ny(ey, qy, j, 1);
            N1(2, k) = Nx(ex, qx, i, 1)*Ny(ey, qy, j, 2);
            k = k + 1;
        end
    end
\end{lstlisting}
It is worthy to notice that from the Fig.~\ref{fig:Ch3SurfAQuarterOfACylinderLocalEnum}, we can know in advance that the determinants of Jacobian matrices of the mapping from parametric space to physical space is a negative number simply because the parametric space need to be swapped in order to compatible with the physical space. This issue is the consequence of the way we set up control points to build the NURBS patch. We almost do not see any affect of this issue in single-patch geometry problems, but it possibly causes strange results in multi-patches geometry cases when one of the patches has negative Jacobian determinants. To avoid this issue, we should always take the absolute value of the Jacobian determinant in the elemental matrices calculation process.
\begin{lstlisting}
    % Gradient of mapping from parameter space to physical space
    dxdxi = R1*CtrlPts(Mesh.El(e, :), :);

    J1 = abs(det(dxdxi));
\end{lstlisting}
\section{Elemental Matrices and Assembly Process}
\subsection{Evaluate values of basis functions and their derivatives at quadrature points}
In order to perform Gauss integration on the interested physical domain, the calculation to get the values of basis functions and their derivatives at quadrature points is an essential prerequisite. This work is accomplished by a routine which is already implemented in SIMO Package called ``calcDersBasisFunsAtGPs.m''. It should be noticed that this routine is implemented only for univariate case and the syntax to use it is
\begin{lstlisting}
    [Jx, Wx, Xi, Nx] = calcDersBasisFunsAtGPs(p, n, KntVect, d, NGPs, NEl);
\end{lstlisting}
where the explanations of the input parameters are represented as follows
\begin{itemize}
    \item \lstinline{p}: degree of basis functions
    \item \lstinline{n}: number of control points
    \item \lstinline{KntVect}: knot vector
    \item \lstinline{d}: maximum degree of derivatives need to be computed
    \item \lstinline{NGPs}: number of gauss points
    \item \lstinline{NEl}: number of elements (non-zero knot spans)
\end{itemize}
and the output parameters are
\begin{itemize}
    \item \lstinline{Jx}: jacobian of mapping from parent element to parametric space. These values are stored as a 1D array and the length of this array is equal to the number of the non-zero knot spans
    \item \lstinline{Wx}: gaussian quadrature weights stored as a 1D array, the length of it is equal to the number of the Gaussian points
    \item \lstinline{Xi}: coordinate of Gaussian points in parametric space
    \item \lstinline{Nx}: evaluated basis functions and their derivatives which are stored as an 4-D array, the forth dimension is organized according to orders of derivatives where each slice of this dimension returns a 3-D array which contains data of basis functions (or their derivatives) evaluated at all gauss points in all non-zero knot spans. Some code snippets are represented below as examples for this idea
        \begin{itemize}
            \item To retrieve $p + 1$ evaluated values of $p + 1$ basis functions in the \textbf{second} non-zero knot span (element) at the \textbf{third} Gaussian point, we invoke
            \begin{lstlisting}
                N0 = Nx(2, 3, :, 1)
            \end{lstlisting}
            \item To retrieve $p + 1$ evaluated first derivative values of $p + 1$ basis functions in the \textbf{first} non-zero knot span (element) at the \textbf{second} Gaussian point, we invoke
            \begin{lstlisting}
                N1 = Nx(1, 2, :, 2)
            \end{lstlisting}
        \end{itemize}
\end{itemize}

For the multivariate cases, basis functions and their derivatives are obtained by using tensor product as mentioned in Chapter 1. To do this in MATLAB, we can simply use for-loop as follows
\begin{itemize}
    \item For bivariate case
    \begin{lstlisting}
        N0 = zeros(1, Mesh.NEN);
        N1 = zeros(2, Mesh.NEN);
        for ey = 1 : Mesh.NElDir(2) % loop over elements in the second direction
            for ex = 1 : Mesh.NElDir(1) % loop over elements in the first direction
                for qy = 1 : NGPs(2) % loop over quadrature points of the second direction
                    for qx = 1 : NGPs(1) % loop over quadrature points of the first direction
                        k = 1;
                        for j = 1 : NURBS.Order(2) + 1 % loop over basis functions
                            for i = 1 : NURBS.Order(1) + 1 % loop over basis functions
                                N0(k) = Nx(ex, qx, i, 1)*Ny(ey, qy, j, 1);
                                N1(1, k) = Nx(ex, qx, i, 2)*Ny(ey, qy, j, 1);
                                N1(2, k) = Nx(ex, qx, i, 1)*Ny(ey, qy, j, 2);
                                k = k + 1;
                            end
                        end
                    end
                end
            end
        end
    \end{lstlisting}
    \item For trivariate case
    \begin{lstlisting}
        N0 = zeros(1, Mesh.NEN);
        N1 = zeros(3, Mesh.NEN);
        for ez = 1 : Mesh.NElDir(3) % loop over elements in the third direction
            for ey = 1 : Mesh.NElDir(2) % loop over elements in the second direction
                for ex = 1 : Mesh.NElDir(1) % loop over elements in the first direction
                    for qz = 1 : NGPs(3)
                        for qy = 1 : NGPs(2)
                            for qx = 1 : NGPs(1)
                                l = 1;
                                for k = 1 : r + 1
                                    for j = 1 : q + 1
                                        for i = 1 : p + 1
                                            N0(1, l) = Nx(ex, qx, i, 1)*Ny(ey, qy, j, 1)*Nz(ez, qz, k, 1);
                                            N1(1, l) = Nx(ex, qx, i, 2)*Ny(ey, qy, j, 1)*Nz(ez, qz, k, 1);
                                            N1(2, l) = Nx(ex, qx, i, 1)*Ny(ey, qy, j, 2)*Nz(ez, qz, k, 1);
                                            N1(3, l) = Nx(ex, qx, i, 1)*Ny(ey, qy, j, 1)*Nz(ez, qz, k, 2);
                                            l = l + 1;
                                        end
                                    end
                                end
                            end
                        end
                    end
                end
            end
        end
    \end{lstlisting}
\end{itemize}
Up to now we only consider B-spline basis functions which are non-rational basis functions. For visualizing conic geometry as well as its fields (heat distribution, displacement, stresses, etc.), NURBS basis functions are not required since we can readily generate a geometry data in homogenous space using B-splines and then project it to Cartesian space. But for computational aspect, we do need to convert these B-spline basis functions into the NURBS ones by employing the formulation provided in Chapter 1. In SIMO Package, the implementation to perform these procedures is responsible by the routine named ``Rationalize.m'', the syntax to use it is
\begin{itemize}
    \item For converting basis functions and their first derivatives
    \begin{lstlisting}
        [R0, R1] = Rationalize(Weights(Mesh.El(e, :)), N0, N1);
    \end{lstlisting}
    \item For converting basis functions, their first and second derivatives
    \begin{lstlisting}
        [R0, R1, R2] = Rationalize(Weights(Mesh.El(e, :)), N0, N1, N2);
    \end{lstlisting}
\end{itemize}
where the first input parameter is the weights of the control points associated with the input basis functions. This implementation is valid for \emph{univariate, bivariate and trivariate} cases.

\subsection{Matrix Assembly Process}
The usual way to assemble elemental matrices to global matrix is conducted by the following way
\begin{lstlisting}
    K = zeros(system_size, system_size);
    for el = 1 : NEl
        ElConn = El(el, :);
        . . .
        %compute local stiffness matrix ke
        . . .
        K(ElConn, ElConn) = K(ElConn, ElConn) + ke;
    end
\end{lstlisting}
where the command \lstinline{K = zeros(system_size, system_size);} is used to allocate a zero global matrix of \lstinline{system_size}-by-\lstinline{system_size}. Although this is not a obligatory requirement in MATLAB since MATLAB uses dynamic data structure to hold arrays, it is introduced to avoid reallocating memory for the global matrix during assembly process which can takes a large amount of time, especially if the size of global matrix is relatively large. This approach, however, is only appropriate for small problems (depends on the total RAM amount of your computer) simply because the global matrix is essentially a band matrix in which there are only a few non-zero elements distributed surrounding and along its main diagonal while the remains are zero values. Therefore if this method is applied for medium or large problems, an enormous amount of memory is occupied with useless zero data which predictably lead to an out of memory situation. To tackle this problem, MATLAB provide a data format called ``sparse array'', where only non-zero values of the array are stored. The above code snippet is need to modified a little bit to take the advantage of this build-in format as
\begin{lstlisting}
    K = sparse(system_size, system_size);
    for el = 1 : NEl
        ElConn = El(el, :);
        . . .
        %compute local stiffness matrix ke
        . . .
        K(ElConn, ElConn) = K(ElConn, ElConn) + ke;
    end
\end{lstlisting}
where only the command \lstinline{zeros} is replaced by the command \lstinline{sparse}. However, this approach can causes degraded performance of the assembly process, this is due to the fact that the sparsity structure of the matrix is internally unknown. Therefore, each time a elemental matrix is scattered to the global one, a reallocation of memory is occurred to hold the non-zero elements of the elemental matrix. This procedure is repeated constantly until the end of the assembly process, which consumes a huge amount of time. This problem can be overcome by pre-calculating the sparsity structure of the global matrix based on mesh structure of the computational domain and then assigning the zero values to the corresponding non-zero value positions of the matrix. Unfortunately, MATLAB does not provide any utility to do that.

For the relatively medium problems, the optimal solution to get rid of all these issues is to employ the so called ``'triplet sparse storage'' which is officially supported by MATLAB. Follow this method, all elemental matrices are computed in advance and assembled in one call to the build-in ``sparse'' command of MATLAB. To apply this method, three 1D arrays are required to be created (triplet), where the first array is used to store values of all elemental matrices, the second and the third ones are used to stored the corresponding row and column indices of these values in the global matrix. The \lstinline{sparse} command use these input parameters to assemble elemental matrices to the global matrix by summing values having the same row and column indices. The below code segment is an example to demonstrate this idea
\begin{lstlisting}
    Dof = 1; % degree of freedom per control point
    KVals = zeros((NEN * Dof) ^ 2, NEl);
    for el = 1 : NEl % Loop over elements (knot spans)
        Ke = zeros(NDofsEl, NDofsEl);
        . . .
        % evaluate local stiffness matrix ke
        . . .
        KVals(:, e) = Ke(:);
    end
    jtmp = repmat(1 : NEN * Dof, NEN * Dof, 1);
    itmp = jtmp';
    ii = El(:, itmp(:))';
    jj = El(:, jtmp(:))';

    rows = ii(:);
    cols = jj(:);
    vals = KVals(:);
    K = sparse(rows, cols, vals);
    clear rows cols vals
\end{lstlisting}
this approach, however, has possibly three drawbacks
\begin{itemize}
    \item Since it requires to compute all elemental matrices and store them in a 1D array, several duplicate elemental values (the exact number depends on the continuity of the basis function we choose) which will be summed up in the assembly process are stored separately in the memory $\rightarrow$ waste memory
    \item It also requires additional memory to stored the two associated indices arrays $\rightarrow$ waste memory
    \item Data of the sparse matrix are stored independently of the three input arrays $\rightarrow$ there is a peak of memory consumption when the assembly process finished, which leads to an out of memory situation.
\end{itemize}
Due to these drawbacks, this approach which is implemented in the current SIMO Package version is not suitable for large problems.

\section{Boundary Conditions}
\subsection{Natural boundary condition}
In NURBS-based IsoGeometric Analysis, the enforcement of the natural (Neumann) boundary conditions are performed naturally and in a similar way as in standard Finite Element Method. The implemented routine to impose this type of boundary condition is accomplished in the file named ``applyNewmannBdryVals.m''.The current implementation of this routine in SIMO Package only cover some standard problems including 2D and 3D heat conduction, 2D and 3D elasticity problems, however, users can readily extend it to any interested problems. The syntax to use it is
\begin{lstlisting}
    [Vals, GDofs] = applyNewmannBdryVals(NURBS, Mesh, g, Refs, LAB, varargin);
\end{lstlisting}
where the input and output parameters are explained as follows:

For the input parameters
\begin{itemize}
    \item NURBS: NURBS structure of the physical domain
    \item Mesh: Mesh structure of the physical domain
    \item g: the function defining the boundary which is provided as an anonymous function handle, e.g
    \begin{itemize}
        \item For a parabola distributed force: \lstinline{g = @(x, y) -P * (D ^ 2 / 4 - y ^ 2) / (2 * I);}
        \item For a constant pressure: \lstinline{g = @(x, y) 4;}
    \end{itemize}
    \item Refs: reference index (indices) of the corresponding boundary (boundaries), in case there are more than one boundary, the reference indices are stored in an array
    \item LAB: label to identify which type of the problem is investigating, valid labels are
    \begin{itemize}
        \item \lstinline{'HFLUX'}: Heat Fluxes in thermal problem
        \item \lstinline{'FX'}, \lstinline{'FY'} or \lstinline{'FZ'}: distributed force in elasticity problem
        \item \lstinline{'PRESS'}: pressure boundary condition in elasticity problem
    \end{itemize}
    \item varargin: a specific input parameter used to handle some particular kinds of problems, e.g
    \begin{itemize}
        \item In annular plane or hollow cylinder problems, geometry is modelled by a patch with an internal interface (Fig. \ref{fig:Ch3SurfAnnular}). This interface is a consequence of coincident control points at the beginning and the ending of the patch in circumferential direction. In order to solve the problem properly, we have to constraint control values of these overlapping control points such that each pair of control values of the corresponding coincident control points have the same value. One simple way to handle this issue is using Master--Slave method. In SIMO package, this method is implemented by global renumbering DOFs of physical domain in which coincident DOFs are numbered by the same indices. The global numbering is stored in an array and then passed as an extra input argument, i.e
            \begin{lstlisting}
                [FY, YDofs] = applyNewmannBdryVals(NURBS, Mesh, g, Refs, LAB, GNum)
            \end{lstlisting}
            \begin{figure}[H]
                \centering
                \tikzsetnextfilename{Ch3SurfAnnular}
                \normalsize
                % This file was created by matlab2tikz.
%
\definecolor{mycolor1}{rgb}{0.38000,0.54800,0.24000}%
%
\begin{tikzpicture}

\begin{axis}[%
width=7cm,
scale only axis,
point meta min=0,
point meta max=1,
xmin=-5.07157464212679,
xmax=5.07157464212679,
ymin=-4,
ymax=4,
hide axis,
unit vector ratio=1 1 1,%
]

\addplot[%
surf,
shader=interp,colormap={mymap}{[1pt] rgb(0pt)=(0.692,0.936,0.936); rgb(2pt)=(0.692,0.936,0.936)},mesh/rows=101]
table[row sep=crcr, point meta=\thisrow{c}] {%
%
x	y	c\\
2	0	0.636621979814147\\
1.9967263621378	0.114384591374602	0.636621979814147\\
1.98662327757016	0.230928458654148	0.636621979814147\\
1.96930098837187	0.349075374665086	0.636621979814147\\
1.94442456264428	0.468202007882929	0.636621979814147\\
1.91172649221395	0.587623875423176	0.636621979814147\\
1.87101867479093	0.706603933320199	0.636621979814147\\
1.82220316332346	0.824363773812255	0.636621979814147\\
1.76528104926887	0.940097238104754	0.636621979814147\\
1.70035887738796	1.0529860816212	0.636621979814147\\
1.62765207210215	1.16221716222984	0.636621979814147\\
1.54748498864467	1.26700047747402	0.636621979814147\\
1.46028737915691	1.36658727137166	0.636621979814147\\
1.36658727137166	1.46028737915691	0.636621979814147\\
1.26700047747402	1.54748498864467	0.636621979814147\\
1.16221716222984	1.62765207210215	0.636621979814147\\
1.0529860816212	1.70035887738796	0.636621979814147\\
0.940097238104754	1.76528104926887	0.636621979814147\\
0.824363773812255	1.82220316332346	0.636621979814147\\
0.7066039333202	1.87101867479093	0.636621979814147\\
0.587623875423176	1.91172649221395	0.636621979814147\\
0.468202007882929	1.94442456264428	0.636621979814147\\
0.349075374665087	1.96930098837187	0.636621979814147\\
0.230928458654148	1.98662327757016	0.636621979814147\\
0.114384591374603	1.9967263621378	0.636621979814147\\
4.44089209850063e-16	2	0.636621979814147\\
-0.114384591374602	1.9967263621378	0.636621979814147\\
-0.230928458654147	1.98662327757016	0.636621979814147\\
-0.349075374665086	1.96930098837187	0.636621979814147\\
-0.468202007882928	1.94442456264428	0.636621979814147\\
-0.587623875423175	1.91172649221395	0.636621979814147\\
-0.706603933320199	1.87101867479093	0.636621979814147\\
-0.824363773812254	1.82220316332346	0.636621979814147\\
-0.940097238104754	1.76528104926887	0.636621979814147\\
-1.0529860816212	1.70035887738796	0.636621979814147\\
-1.16221716222984	1.62765207210215	0.636621979814147\\
-1.26700047747402	1.54748498864467	0.636621979814147\\
-1.36658727137166	1.46028737915691	0.636621979814147\\
-1.46028737915691	1.36658727137166	0.636621979814147\\
-1.54748498864467	1.26700047747402	0.636621979814147\\
-1.62765207210215	1.16221716222984	0.636621979814147\\
-1.70035887738796	1.0529860816212	0.636621979814147\\
-1.76528104926887	0.940097238104755	0.636621979814147\\
-1.82220316332346	0.824363773812255	0.636621979814147\\
-1.87101867479093	0.7066039333202	0.636621979814147\\
-1.91172649221395	0.587623875423176	0.636621979814147\\
-1.94442456264428	0.468202007882929	0.636621979814147\\
-1.96930098837187	0.349075374665087	0.636621979814147\\
-1.98662327757016	0.230928458654148	0.636621979814147\\
-1.9967263621378	0.114384591374603	0.636621979814147\\
-2	5.66553889764798e-16	0.636621979814147\\
-1.9967263621378	-0.114384591374602	0.636621979814147\\
-1.98662327757016	-0.230928458654147	0.636621979814147\\
-1.96930098837187	-0.349075374665086	0.636621979814147\\
-1.94442456264428	-0.468202007882929	0.636621979814147\\
-1.91172649221395	-0.587623875423176	0.636621979814147\\
-1.87101867479093	-0.7066039333202	0.636621979814147\\
-1.82220316332346	-0.824363773812253	0.636621979814147\\
-1.76528104926887	-0.940097238104753	0.636621979814147\\
-1.70035887738796	-1.0529860816212	0.636621979814147\\
-1.62765207210215	-1.16221716222984	0.636621979814147\\
-1.54748498864468	-1.26700047747402	0.636621979814147\\
-1.46028737915691	-1.36658727137166	0.636621979814147\\
-1.36658727137166	-1.46028737915691	0.636621979814147\\
-1.26700047747402	-1.54748498864467	0.636621979814147\\
-1.16221716222984	-1.62765207210215	0.636621979814147\\
-1.0529860816212	-1.70035887738796	0.636621979814147\\
-0.940097238104754	-1.76528104926887	0.636621979814147\\
-0.824363773812254	-1.82220316332346	0.636621979814147\\
-0.706603933320201	-1.87101867479093	0.636621979814147\\
-0.587623875423177	-1.91172649221395	0.636621979814147\\
-0.46820200788293	-1.94442456264428	0.636621979814147\\
-0.349075374665087	-1.96930098837187	0.636621979814147\\
-0.230928458654148	-1.98662327757016	0.636621979814147\\
-0.114384591374603	-1.9967263621378	0.636621979814147\\
-6.89018569679533e-16	-2	0.636621979814147\\
0.114384591374601	-1.9967263621378	0.636621979814147\\
0.230928458654147	-1.98662327757016	0.636621979814147\\
0.349075374665086	-1.96930098837187	0.636621979814147\\
0.468202007882928	-1.94442456264428	0.636621979814147\\
0.587623875423176	-1.91172649221395	0.636621979814147\\
0.7066039333202	-1.87101867479093	0.636621979814147\\
0.824363773812253	-1.82220316332346	0.636621979814147\\
0.940097238104753	-1.76528104926887	0.636621979814147\\
1.0529860816212	-1.70035887738796	0.636621979814147\\
1.16221716222984	-1.62765207210215	0.636621979814147\\
1.26700047747402	-1.54748498864468	0.636621979814147\\
1.36658727137166	-1.46028737915691	0.636621979814147\\
1.46028737915691	-1.36658727137166	0.636621979814147\\
1.54748498864467	-1.26700047747402	0.636621979814147\\
1.62765207210215	-1.16221716222984	0.636621979814147\\
1.70035887738796	-1.0529860816212	0.636621979814147\\
1.76528104926887	-0.940097238104754	0.636621979814147\\
1.82220316332346	-0.824363773812254	0.636621979814147\\
1.87101867479093	-0.706603933320201	0.636621979814147\\
1.91172649221395	-0.587623875423177	0.636621979814147\\
1.94442456264428	-0.46820200788293	0.636621979814147\\
1.96930098837187	-0.349075374665087	0.636621979814147\\
1.98662327757016	-0.230928458654149	0.636621979814147\\
1.9967263621378	-0.114384591374603	0.636621979814147\\
2	-8.11483249594269e-16	0.636621979814147\\
2.02	0	0.636621979814147\\
2.01669362575918	0.115528437288348	0.636621979814147\\
2.00648951034587	0.233237743240689	0.636621979814147\\
1.98899399825559	0.352566128411737	0.636621979814147\\
1.96386880827072	0.472884027961758	0.636621979814147\\
1.93084375713609	0.593500114177407	0.636621979814147\\
1.88972886153884	0.713669972653401	0.636621979814147\\
1.84042519495669	0.832607411550377	0.636621979814147\\
1.78293385976156	0.949498210485802	0.636621979814147\\
1.71736246616184	1.06351594243741	0.636621979814147\\
1.64392859282317	1.17383933385214	0.636621979814147\\
1.56295983853112	1.27967048224876	0.636621979814147\\
1.47489025294848	1.38025314408537	0.636621979814147\\
1.38025314408537	1.47489025294848	0.636621979814147\\
1.27967048224876	1.56295983853112	0.636621979814147\\
1.17383933385214	1.64392859282317	0.636621979814147\\
1.06351594243741	1.71736246616184	0.636621979814147\\
0.949498210485802	1.78293385976156	0.636621979814147\\
0.832607411550377	1.84042519495669	0.636621979814147\\
0.713669972653402	1.88972886153884	0.636621979814147\\
0.593500114177408	1.93084375713609	0.636621979814147\\
0.472884027961758	1.96386880827072	0.636621979814147\\
0.352566128411737	1.98899399825559	0.636621979814147\\
0.233237743240689	2.00648951034586	0.636621979814147\\
0.115528437288349	2.01669362575918	0.636621979814147\\
4.48530101948563e-16	2.02	0.636621979814147\\
-0.115528437288348	2.01669362575918	0.636621979814147\\
-0.233237743240689	2.00648951034587	0.636621979814147\\
-0.352566128411737	1.98899399825559	0.636621979814147\\
-0.472884027961757	1.96386880827072	0.636621979814147\\
-0.593500114177407	1.93084375713609	0.636621979814147\\
-0.713669972653401	1.88972886153884	0.636621979814147\\
-0.832607411550377	1.84042519495669	0.636621979814147\\
-0.949498210485802	1.78293385976156	0.636621979814147\\
-1.06351594243741	1.71736246616184	0.636621979814147\\
-1.17383933385214	1.64392859282317	0.636621979814147\\
-1.27967048224876	1.56295983853112	0.636621979814147\\
-1.38025314408537	1.47489025294848	0.636621979814147\\
-1.47489025294848	1.38025314408537	0.636621979814147\\
-1.56295983853112	1.27967048224876	0.636621979814147\\
-1.64392859282317	1.17383933385214	0.636621979814147\\
-1.71736246616184	1.06351594243741	0.636621979814147\\
-1.78293385976156	0.949498210485802	0.636621979814147\\
-1.84042519495669	0.832607411550377	0.636621979814147\\
-1.88972886153884	0.713669972653402	0.636621979814147\\
-1.93084375713609	0.593500114177408	0.636621979814147\\
-1.96386880827072	0.472884027961758	0.636621979814147\\
-1.98899399825559	0.352566128411738	0.636621979814147\\
-2.00648951034586	0.23323774324069	0.636621979814147\\
-2.01669362575918	0.115528437288349	0.636621979814147\\
-2.02	5.72219428662446e-16	0.636621979814147\\
-2.01669362575918	-0.115528437288348	0.636621979814147\\
-2.00648951034587	-0.233237743240689	0.636621979814147\\
-1.98899399825559	-0.352566128411737	0.636621979814147\\
-1.96386880827072	-0.472884027961758	0.636621979814147\\
-1.93084375713609	-0.593500114177407	0.636621979814147\\
-1.88972886153884	-0.713669972653402	0.636621979814147\\
-1.84042519495669	-0.832607411550376	0.636621979814147\\
-1.78293385976156	-0.949498210485801	0.636621979814147\\
-1.71736246616184	-1.06351594243741	0.636621979814147\\
-1.64392859282317	-1.17383933385214	0.636621979814147\\
-1.56295983853112	-1.27967048224876	0.636621979814147\\
-1.47489025294848	-1.38025314408537	0.636621979814147\\
-1.38025314408537	-1.47489025294848	0.636621979814147\\
-1.27967048224876	-1.56295983853112	0.636621979814147\\
-1.17383933385214	-1.64392859282317	0.636621979814147\\
-1.06351594243741	-1.71736246616184	0.636621979814147\\
-0.949498210485802	-1.78293385976156	0.636621979814147\\
-0.832607411550377	-1.84042519495669	0.636621979814147\\
-0.713669972653403	-1.88972886153884	0.636621979814147\\
-0.593500114177409	-1.93084375713609	0.636621979814147\\
-0.472884027961759	-1.96386880827072	0.636621979814147\\
-0.352566128411738	-1.98899399825559	0.636621979814147\\
-0.23323774324069	-2.00648951034586	0.636621979814147\\
-0.115528437288349	-2.01669362575918	0.636621979814147\\
-6.95908755376329e-16	-2.02	0.636621979814147\\
0.115528437288347	-2.01669362575918	0.636621979814147\\
0.233237743240689	-2.00648951034587	0.636621979814147\\
0.352566128411737	-1.98899399825559	0.636621979814147\\
0.472884027961758	-1.96386880827072	0.636621979814147\\
0.593500114177407	-1.93084375713609	0.636621979814147\\
0.713669972653401	-1.88972886153884	0.636621979814147\\
0.832607411550376	-1.84042519495669	0.636621979814147\\
0.949498210485801	-1.78293385976156	0.636621979814147\\
1.06351594243741	-1.71736246616184	0.636621979814147\\
1.17383933385214	-1.64392859282317	0.636621979814147\\
1.27967048224876	-1.56295983853112	0.636621979814147\\
1.38025314408537	-1.47489025294848	0.636621979814147\\
1.47489025294848	-1.38025314408537	0.636621979814147\\
1.56295983853112	-1.27967048224876	0.636621979814147\\
1.64392859282317	-1.17383933385214	0.636621979814147\\
1.71736246616184	-1.06351594243741	0.636621979814147\\
1.78293385976156	-0.949498210485802	0.636621979814147\\
1.84042519495669	-0.832607411550377	0.636621979814147\\
1.88972886153884	-0.713669972653403	0.636621979814147\\
1.93084375713609	-0.593500114177409	0.636621979814147\\
1.96386880827072	-0.472884027961759	0.636621979814147\\
1.98899399825559	-0.352566128411738	0.636621979814147\\
2.00648951034586	-0.23323774324069	0.636621979814147\\
2.01669362575918	-0.115528437288349	0.636621979814147\\
2.02	-8.19598082090211e-16	0.636621979814147\\
2.04	0	0.636621979814147\\
2.03666088938056	0.116672283202094	0.636621979814147\\
2.02635574312157	0.235547027827231	0.636621979814147\\
2.00868700813931	0.356056882158388	0.636621979814147\\
1.98331305389716	0.477566048040587	0.636621979814147\\
1.94996102205823	0.599376352931639	0.636621979814147\\
1.90843904828675	0.720736011986604	0.636621979814147\\
1.85864722658993	0.8408510492885	0.636621979814147\\
1.80058667025425	0.958899182866849	0.636621979814147\\
1.73436605493572	1.07404580325362	0.636621979814147\\
1.66020511354419	1.18546150547443	0.636621979814147\\
1.57843468841757	1.2923404870235	0.636621979814147\\
1.48949312674005	1.39391901679909	0.636621979814147\\
1.39391901679909	1.48949312674005	0.636621979814147\\
1.2923404870235	1.57843468841757	0.636621979814147\\
1.18546150547443	1.66020511354419	0.636621979814147\\
1.07404580325362	1.73436605493572	0.636621979814147\\
0.95889918286685	1.80058667025425	0.636621979814147\\
0.8408510492885	1.85864722658993	0.636621979814147\\
0.720736011986604	1.90843904828675	0.636621979814147\\
0.599376352931639	1.94996102205823	0.636621979814147\\
0.477566048040588	1.98331305389716	0.636621979814147\\
0.356056882158388	2.00868700813931	0.636621979814147\\
0.235547027827231	2.02635574312157	0.636621979814147\\
0.116672283202095	2.03666088938056	0.636621979814147\\
4.52970994047064e-16	2.04	0.636621979814147\\
-0.116672283202094	2.03666088938056	0.636621979814147\\
-0.23554702782723	2.02635574312157	0.636621979814147\\
-0.356056882158388	2.00868700813931	0.636621979814147\\
-0.477566048040587	1.98331305389716	0.636621979814147\\
-0.599376352931639	1.94996102205823	0.636621979814147\\
-0.720736011986603	1.90843904828675	0.636621979814147\\
-0.8408510492885	1.85864722658993	0.636621979814147\\
-0.958899182866849	1.80058667025425	0.636621979814147\\
-1.07404580325362	1.73436605493572	0.636621979814147\\
-1.18546150547443	1.66020511354419	0.636621979814147\\
-1.2923404870235	1.57843468841757	0.636621979814147\\
-1.39391901679909	1.48949312674005	0.636621979814147\\
-1.48949312674005	1.39391901679909	0.636621979814147\\
-1.57843468841757	1.2923404870235	0.636621979814147\\
-1.66020511354419	1.18546150547443	0.636621979814147\\
-1.73436605493572	1.07404580325362	0.636621979814147\\
-1.80058667025425	0.95889918286685	0.636621979814147\\
-1.85864722658993	0.8408510492885	0.636621979814147\\
-1.90843904828675	0.720736011986604	0.636621979814147\\
-1.94996102205823	0.599376352931639	0.636621979814147\\
-1.98331305389716	0.477566048040587	0.636621979814147\\
-2.00868700813931	0.356056882158389	0.636621979814147\\
-2.02635574312157	0.235547027827231	0.636621979814147\\
-2.03666088938056	0.116672283202095	0.636621979814147\\
-2.04	5.77884967560094e-16	0.636621979814147\\
-2.03666088938056	-0.116672283202094	0.636621979814147\\
-2.02635574312157	-0.23554702782723	0.636621979814147\\
-2.00868700813931	-0.356056882158388	0.636621979814147\\
-1.98331305389716	-0.477566048040587	0.636621979814147\\
-1.94996102205823	-0.599376352931639	0.636621979814147\\
-1.90843904828675	-0.720736011986604	0.636621979814147\\
-1.85864722658993	-0.840851049288499	0.636621979814147\\
-1.80058667025425	-0.958899182866848	0.636621979814147\\
-1.73436605493572	-1.07404580325362	0.636621979814147\\
-1.66020511354419	-1.18546150547443	0.636621979814147\\
-1.57843468841757	-1.2923404870235	0.636621979814147\\
-1.48949312674005	-1.39391901679909	0.636621979814147\\
-1.39391901679909	-1.48949312674005	0.636621979814147\\
-1.2923404870235	-1.57843468841757	0.636621979814147\\
-1.18546150547443	-1.66020511354419	0.636621979814147\\
-1.07404580325362	-1.73436605493572	0.636621979814147\\
-0.958899182866849	-1.80058667025425	0.636621979814147\\
-0.8408510492885	-1.85864722658993	0.636621979814147\\
-0.720736011986605	-1.90843904828675	0.636621979814147\\
-0.59937635293164	-1.94996102205823	0.636621979814147\\
-0.477566048040588	-1.98331305389716	0.636621979814147\\
-0.356056882158389	-2.00868700813931	0.636621979814147\\
-0.235547027827231	-2.02635574312157	0.636621979814147\\
-0.116672283202095	-2.03666088938056	0.636621979814147\\
-7.02798941073124e-16	-2.04	0.636621979814147\\
0.116672283202093	-2.03666088938056	0.636621979814147\\
0.23554702782723	-2.02635574312157	0.636621979814147\\
0.356056882158388	-2.00868700813931	0.636621979814147\\
0.477566048040587	-1.98331305389716	0.636621979814147\\
0.599376352931639	-1.94996102205823	0.636621979814147\\
0.720736011986604	-1.90843904828675	0.636621979814147\\
0.840851049288499	-1.85864722658993	0.636621979814147\\
0.958899182866848	-1.80058667025425	0.636621979814147\\
1.07404580325362	-1.73436605493572	0.636621979814147\\
1.18546150547443	-1.66020511354419	0.636621979814147\\
1.2923404870235	-1.57843468841757	0.636621979814147\\
1.39391901679909	-1.48949312674005	0.636621979814147\\
1.48949312674005	-1.39391901679909	0.636621979814147\\
1.57843468841757	-1.2923404870235	0.636621979814147\\
1.66020511354419	-1.18546150547443	0.636621979814147\\
1.73436605493572	-1.07404580325362	0.636621979814147\\
1.80058667025425	-0.958899182866849	0.636621979814147\\
1.85864722658993	-0.8408510492885	0.636621979814147\\
1.90843904828675	-0.720736011986605	0.636621979814147\\
1.94996102205823	-0.59937635293164	0.636621979814147\\
1.98331305389716	-0.477566048040588	0.636621979814147\\
2.00868700813931	-0.356056882158389	0.636621979814147\\
2.02635574312157	-0.235547027827232	0.636621979814147\\
2.03666088938056	-0.116672283202095	0.636621979814147\\
2.04	-8.27712914586154e-16	0.636621979814147\\
2.06	0	0.636621979814147\\
2.05662815300194	0.11781612911584	0.636621979814147\\
2.04622197589727	0.237856312413772	0.636621979814147\\
2.02838001802303	0.359547635905039	0.636621979814147\\
2.00275729952361	0.482248068119416	0.636621979814147\\
1.96907828698037	0.605252591685871	0.636621979814147\\
1.92714923503466	0.727802051319806	0.636621979814147\\
1.87686925822316	0.849094687026622	0.636621979814147\\
1.81823948074694	0.968300155247897	0.636621979814147\\
1.7513696437096	1.08457566406983	0.636621979814147\\
1.67648163426521	1.19708367709673	0.636621979814147\\
1.59390953830402	1.30501049179824	0.636621979814147\\
1.50409600053162	1.40758488951281	0.636621979814147\\
1.40758488951281	1.50409600053162	0.636621979814147\\
1.30501049179824	1.59390953830402	0.636621979814147\\
1.19708367709673	1.67648163426521	0.636621979814147\\
1.08457566406983	1.7513696437096	0.636621979814147\\
0.968300155247897	1.81823948074694	0.636621979814147\\
0.849094687026623	1.87686925822316	0.636621979814147\\
0.727802051319806	1.92714923503466	0.636621979814147\\
0.605252591685871	1.96907828698037	0.636621979814147\\
0.482248068119417	2.00275729952361	0.636621979814147\\
0.359547635905039	2.02838001802303	0.636621979814147\\
0.237856312413772	2.04622197589727	0.636621979814147\\
0.117816129115841	2.05662815300194	0.636621979814147\\
4.57411886145565e-16	2.06	0.636621979814147\\
-0.11781612911584	2.05662815300194	0.636621979814147\\
-0.237856312413772	2.04622197589727	0.636621979814147\\
-0.359547635905039	2.02838001802303	0.636621979814147\\
-0.482248068119416	2.00275729952361	0.636621979814147\\
-0.605252591685871	1.96907828698037	0.636621979814147\\
-0.727802051319805	1.92714923503466	0.636621979814147\\
-0.849094687026622	1.87686925822316	0.636621979814147\\
-0.968300155247897	1.81823948074694	0.636621979814147\\
-1.08457566406983	1.7513696437096	0.636621979814147\\
-1.19708367709673	1.67648163426521	0.636621979814147\\
-1.30501049179824	1.59390953830402	0.636621979814147\\
-1.40758488951281	1.50409600053162	0.636621979814147\\
-1.50409600053162	1.40758488951281	0.636621979814147\\
-1.59390953830401	1.30501049179824	0.636621979814147\\
-1.67648163426521	1.19708367709673	0.636621979814147\\
-1.7513696437096	1.08457566406984	0.636621979814147\\
-1.81823948074694	0.968300155247897	0.636621979814147\\
-1.87686925822316	0.849094687026623	0.636621979814147\\
-1.92714923503466	0.727802051319806	0.636621979814147\\
-1.96907828698037	0.605252591685871	0.636621979814147\\
-2.00275729952361	0.482248068119417	0.636621979814147\\
-2.02838001802303	0.35954763590504	0.636621979814147\\
-2.04622197589727	0.237856312413773	0.636621979814147\\
-2.05662815300194	0.117816129115841	0.636621979814147\\
-2.06	5.83550506457742e-16	0.636621979814147\\
-2.05662815300194	-0.11781612911584	0.636621979814147\\
-2.04622197589727	-0.237856312413772	0.636621979814147\\
-2.02838001802303	-0.359547635905038	0.636621979814147\\
-2.00275729952361	-0.482248068119416	0.636621979814147\\
-1.96907828698037	-0.605252591685871	0.636621979814147\\
-1.92714923503466	-0.727802051319806	0.636621979814147\\
-1.87686925822316	-0.849094687026621	0.636621979814147\\
-1.81823948074694	-0.968300155247896	0.636621979814147\\
-1.7513696437096	-1.08457566406983	0.636621979814147\\
-1.67648163426521	-1.19708367709673	0.636621979814147\\
-1.59390953830402	-1.30501049179824	0.636621979814147\\
-1.50409600053162	-1.40758488951281	0.636621979814147\\
-1.40758488951281	-1.50409600053162	0.636621979814147\\
-1.30501049179824	-1.59390953830401	0.636621979814147\\
-1.19708367709673	-1.67648163426521	0.636621979814147\\
-1.08457566406983	-1.7513696437096	0.636621979814147\\
-0.968300155247897	-1.81823948074694	0.636621979814147\\
-0.849094687026622	-1.87686925822316	0.636621979814147\\
-0.727802051319807	-1.92714923503466	0.636621979814147\\
-0.605252591685872	-1.96907828698037	0.636621979814147\\
-0.482248068119417	-2.00275729952361	0.636621979814147\\
-0.35954763590504	-2.02838001802303	0.636621979814147\\
-0.237856312413773	-2.04622197589727	0.636621979814147\\
-0.117816129115841	-2.05662815300194	0.636621979814147\\
-7.09689126769919e-16	-2.06	0.636621979814147\\
0.117816129115839	-2.05662815300194	0.636621979814147\\
0.237856312413772	-2.04622197589727	0.636621979814147\\
0.359547635905038	-2.02838001802303	0.636621979814147\\
0.482248068119416	-2.00275729952361	0.636621979814147\\
0.605252591685871	-1.96907828698037	0.636621979814147\\
0.727802051319806	-1.92714923503466	0.636621979814147\\
0.849094687026621	-1.87686925822316	0.636621979814147\\
0.968300155247896	-1.81823948074694	0.636621979814147\\
1.08457566406983	-1.7513696437096	0.636621979814147\\
1.19708367709673	-1.67648163426522	0.636621979814147\\
1.30501049179824	-1.59390953830402	0.636621979814147\\
1.40758488951281	-1.50409600053162	0.636621979814147\\
1.50409600053162	-1.40758488951281	0.636621979814147\\
1.59390953830401	-1.30501049179824	0.636621979814147\\
1.67648163426521	-1.19708367709673	0.636621979814147\\
1.7513696437096	-1.08457566406983	0.636621979814147\\
1.81823948074694	-0.968300155247897	0.636621979814147\\
1.87686925822316	-0.849094687026622	0.636621979814147\\
1.92714923503466	-0.727802051319807	0.636621979814147\\
1.96907828698037	-0.605252591685872	0.636621979814147\\
2.00275729952361	-0.482248068119418	0.636621979814147\\
2.02838001802303	-0.35954763590504	0.636621979814147\\
2.04622197589727	-0.237856312413773	0.636621979814147\\
2.05662815300194	-0.117816129115841	0.636621979814147\\
2.06	-8.35827747082097e-16	0.636621979814147\\
2.08	0	0.636621979814147\\
2.07659541662332	0.118959975029586	0.636621979814147\\
2.06608820867297	0.240165597000313	0.636621979814147\\
2.04807302790675	0.36303838965169	0.636621979814147\\
2.02220154515005	0.486930088198246	0.636621979814147\\
1.98819555190251	0.611128830440103	0.636621979814147\\
1.94585942178257	0.734868090653007	0.636621979814147\\
1.8950912898564	0.857338324764745	0.636621979814147\\
1.83589229123963	0.977701127628944	0.636621979814147\\
1.76837323248348	1.09510552488605	0.636621979814147\\
1.69275815498624	1.20870584871903	0.636621979814147\\
1.60938438819046	1.31768049657298	0.636621979814147\\
1.51869887432319	1.42125076222652	0.636621979814147\\
1.42125076222652	1.51869887432319	0.636621979814147\\
1.31768049657298	1.60938438819046	0.636621979814147\\
1.20870584871903	1.69275815498624	0.636621979814147\\
1.09510552488605	1.76837323248348	0.636621979814147\\
0.977701127628944	1.83589229123963	0.636621979814147\\
0.857338324764745	1.8950912898564	0.636621979814147\\
0.734868090653008	1.94585942178257	0.636621979814147\\
0.611128830440103	1.98819555190251	0.636621979814147\\
0.486930088198246	2.02220154515005	0.636621979814147\\
0.36303838965169	2.04807302790675	0.636621979814147\\
0.240165597000314	2.06608820867297	0.636621979814147\\
0.118959975029587	2.07659541662332	0.636621979814147\\
4.61852778244065e-16	2.08	0.636621979814147\\
-0.118959975029586	2.07659541662332	0.636621979814147\\
-0.240165597000313	2.06608820867297	0.636621979814147\\
-0.36303838965169	2.04807302790675	0.636621979814147\\
-0.486930088198245	2.02220154515005	0.636621979814147\\
-0.611128830440102	1.98819555190251	0.636621979814147\\
-0.734868090653007	1.94585942178257	0.636621979814147\\
-0.857338324764745	1.8950912898564	0.636621979814147\\
-0.977701127628944	1.83589229123963	0.636621979814147\\
-1.09510552488605	1.76837323248348	0.636621979814147\\
-1.20870584871903	1.69275815498624	0.636621979814147\\
-1.31768049657298	1.60938438819046	0.636621979814147\\
-1.42125076222652	1.51869887432319	0.636621979814147\\
-1.51869887432319	1.42125076222652	0.636621979814147\\
-1.60938438819046	1.31768049657298	0.636621979814147\\
-1.69275815498624	1.20870584871903	0.636621979814147\\
-1.76837323248347	1.09510552488605	0.636621979814147\\
-1.83589229123963	0.977701127628945	0.636621979814147\\
-1.8950912898564	0.857338324764745	0.636621979814147\\
-1.94585942178257	0.734868090653008	0.636621979814147\\
-1.98819555190251	0.611128830440103	0.636621979814147\\
-2.02220154515005	0.486930088198246	0.636621979814147\\
-2.04807302790675	0.36303838965169	0.636621979814147\\
-2.06608820867297	0.240165597000314	0.636621979814147\\
-2.07659541662332	0.118959975029587	0.636621979814147\\
-2.08	5.8921604535539e-16	0.636621979814147\\
-2.07659541662332	-0.118959975029586	0.636621979814147\\
-2.06608820867297	-0.240165597000313	0.636621979814147\\
-2.04807302790675	-0.363038389651689	0.636621979814147\\
-2.02220154515005	-0.486930088198246	0.636621979814147\\
-1.98819555190251	-0.611128830440103	0.636621979814147\\
-1.94585942178257	-0.734868090653008	0.636621979814147\\
-1.8950912898564	-0.857338324764743	0.636621979814147\\
-1.83589229123963	-0.977701127628943	0.636621979814147\\
-1.76837323248348	-1.09510552488605	0.636621979814147\\
-1.69275815498624	-1.20870584871903	0.636621979814147\\
-1.60938438819046	-1.31768049657298	0.636621979814147\\
-1.51869887432319	-1.42125076222652	0.636621979814147\\
-1.42125076222652	-1.51869887432319	0.636621979814147\\
-1.31768049657298	-1.60938438819046	0.636621979814147\\
-1.20870584871903	-1.69275815498624	0.636621979814147\\
-1.09510552488605	-1.76837323248348	0.636621979814147\\
-0.977701127628944	-1.83589229123963	0.636621979814147\\
-0.857338324764745	-1.8950912898564	0.636621979814147\\
-0.734868090653009	-1.94585942178257	0.636621979814147\\
-0.611128830440104	-1.98819555190251	0.636621979814147\\
-0.486930088198247	-2.02220154515005	0.636621979814147\\
-0.363038389651691	-2.04807302790675	0.636621979814147\\
-0.240165597000314	-2.06608820867297	0.636621979814147\\
-0.118959975029587	-2.07659541662332	0.636621979814147\\
-7.16579312466715e-16	-2.08	0.636621979814147\\
0.118959975029586	-2.07659541662332	0.636621979814147\\
0.240165597000313	-2.06608820867297	0.636621979814147\\
0.363038389651689	-2.04807302790675	0.636621979814147\\
0.486930088198246	-2.02220154515005	0.636621979814147\\
0.611128830440103	-1.98819555190251	0.636621979814147\\
0.734868090653008	-1.94585942178257	0.636621979814147\\
0.857338324764743	-1.8950912898564	0.636621979814147\\
0.977701127628943	-1.83589229123963	0.636621979814147\\
1.09510552488605	-1.76837323248348	0.636621979814147\\
1.20870584871903	-1.69275815498624	0.636621979814147\\
1.31768049657298	-1.60938438819046	0.636621979814147\\
1.42125076222652	-1.51869887432319	0.636621979814147\\
1.51869887432319	-1.42125076222652	0.636621979814147\\
1.60938438819046	-1.31768049657298	0.636621979814147\\
1.69275815498624	-1.20870584871903	0.636621979814147\\
1.76837323248348	-1.09510552488605	0.636621979814147\\
1.83589229123963	-0.977701127628944	0.636621979814147\\
1.8950912898564	-0.857338324764745	0.636621979814147\\
1.94585942178257	-0.734868090653009	0.636621979814147\\
1.98819555190251	-0.611128830440104	0.636621979814147\\
2.02220154515005	-0.486930088198247	0.636621979814147\\
2.04807302790675	-0.363038389651691	0.636621979814147\\
2.06608820867297	-0.240165597000315	0.636621979814147\\
2.07659541662332	-0.118959975029587	0.636621979814147\\
2.08	-8.43942579578039e-16	0.636621979814147\\
2.1	0	0.636621979814147\\
2.09656268024469	0.120103820943332	0.636621979814147\\
2.08595444144867	0.242474881586855	0.636621979814147\\
2.06776603779047	0.36652914339834	0.636621979814147\\
2.04164579077649	0.491612108277075	0.636621979814147\\
2.00731281682465	0.617005069194334	0.636621979814147\\
1.96456960853048	0.741934129986209	0.636621979814147\\
1.91331332148963	0.865581962502867	0.636621979814147\\
1.85354510173232	0.987102100009992	0.636621979814147\\
1.78537682125736	1.10563538570226	0.636621979814147\\
1.70903467570726	1.22032802034133	0.636621979814147\\
1.62485923807691	1.33035050134772	0.636621979814147\\
1.53330174811476	1.43491663494024	0.636621979814147\\
1.43491663494024	1.53330174811476	0.636621979814147\\
1.33035050134772	1.62485923807691	0.636621979814147\\
1.22032802034133	1.70903467570726	0.636621979814147\\
1.10563538570226	1.78537682125736	0.636621979814147\\
0.987102100009992	1.85354510173232	0.636621979814147\\
0.865581962502868	1.91331332148963	0.636621979814147\\
0.74193412998621	1.96456960853048	0.636621979814147\\
0.617005069194335	2.00731281682465	0.636621979814147\\
0.491612108277076	2.04164579077649	0.636621979814147\\
0.366529143398341	2.06776603779046	0.636621979814147\\
0.242474881586855	2.08595444144867	0.636621979814147\\
0.120103820943333	2.09656268024469	0.636621979814147\\
4.66293670342566e-16	2.1	0.636621979814147\\
-0.120103820943332	2.09656268024469	0.636621979814147\\
-0.242474881586855	2.08595444144867	0.636621979814147\\
-0.36652914339834	2.06776603779046	0.636621979814147\\
-0.491612108277074	2.04164579077649	0.636621979814147\\
-0.617005069194334	2.00731281682465	0.636621979814147\\
-0.741934129986209	1.96456960853048	0.636621979814147\\
-0.865581962502867	1.91331332148963	0.636621979814147\\
-0.987102100009992	1.85354510173232	0.636621979814147\\
-1.10563538570226	1.78537682125736	0.636621979814147\\
-1.22032802034133	1.70903467570726	0.636621979814147\\
-1.33035050134772	1.62485923807691	0.636621979814147\\
-1.43491663494024	1.53330174811476	0.636621979814147\\
-1.53330174811476	1.43491663494024	0.636621979814147\\
-1.62485923807691	1.33035050134772	0.636621979814147\\
-1.70903467570726	1.22032802034133	0.636621979814147\\
-1.78537682125735	1.10563538570226	0.636621979814147\\
-1.85354510173232	0.987102100009993	0.636621979814147\\
-1.91331332148963	0.865581962502868	0.636621979814147\\
-1.96456960853048	0.74193412998621	0.636621979814147\\
-2.00731281682465	0.617005069194335	0.636621979814147\\
-2.04164579077649	0.491612108277075	0.636621979814147\\
-2.06776603779046	0.366529143398341	0.636621979814147\\
-2.08595444144867	0.242474881586856	0.636621979814147\\
-2.09656268024469	0.120103820943333	0.636621979814147\\
-2.1	5.94881584253038e-16	0.636621979814147\\
-2.09656268024469	-0.120103820943332	0.636621979814147\\
-2.08595444144867	-0.242474881586855	0.636621979814147\\
-2.06776603779046	-0.36652914339834	0.636621979814147\\
-2.04164579077649	-0.491612108277075	0.636621979814147\\
-2.00731281682465	-0.617005069194335	0.636621979814147\\
-1.96456960853048	-0.74193412998621	0.636621979814147\\
-1.91331332148963	-0.865581962502866	0.636621979814147\\
-1.85354510173232	-0.987102100009991	0.636621979814147\\
-1.78537682125736	-1.10563538570226	0.636621979814147\\
-1.70903467570726	-1.22032802034133	0.636621979814147\\
-1.62485923807691	-1.33035050134772	0.636621979814147\\
-1.53330174811476	-1.43491663494024	0.636621979814147\\
-1.43491663494024	-1.53330174811476	0.636621979814147\\
-1.33035050134772	-1.62485923807691	0.636621979814147\\
-1.22032802034133	-1.70903467570726	0.636621979814147\\
-1.10563538570226	-1.78537682125736	0.636621979814147\\
-0.987102100009992	-1.85354510173232	0.636621979814147\\
-0.865581962502867	-1.91331332148963	0.636621979814147\\
-0.741934129986211	-1.96456960853048	0.636621979814147\\
-0.617005069194336	-2.00731281682465	0.636621979814147\\
-0.491612108277076	-2.04164579077649	0.636621979814147\\
-0.366529143398341	-2.06776603779046	0.636621979814147\\
-0.242474881586856	-2.08595444144867	0.636621979814147\\
-0.120103820943333	-2.09656268024469	0.636621979814147\\
-7.2346949816351e-16	-2.1	0.636621979814147\\
0.120103820943332	-2.09656268024469	0.636621979814147\\
0.242474881586855	-2.08595444144867	0.636621979814147\\
0.36652914339834	-2.06776603779046	0.636621979814147\\
0.491612108277075	-2.04164579077649	0.636621979814147\\
0.617005069194335	-2.00731281682465	0.636621979814147\\
0.74193412998621	-1.96456960853048	0.636621979814147\\
0.865581962502866	-1.91331332148963	0.636621979814147\\
0.987102100009991	-1.85354510173232	0.636621979814147\\
1.10563538570226	-1.78537682125736	0.636621979814147\\
1.22032802034133	-1.70903467570726	0.636621979814147\\
1.33035050134772	-1.62485923807691	0.636621979814147\\
1.43491663494024	-1.53330174811476	0.636621979814147\\
1.53330174811476	-1.43491663494024	0.636621979814147\\
1.62485923807691	-1.33035050134772	0.636621979814147\\
1.70903467570726	-1.22032802034133	0.636621979814147\\
1.78537682125736	-1.10563538570226	0.636621979814147\\
1.85354510173232	-0.987102100009992	0.636621979814147\\
1.91331332148963	-0.865581962502867	0.636621979814147\\
1.96456960853048	-0.741934129986211	0.636621979814147\\
2.00731281682465	-0.617005069194336	0.636621979814147\\
2.04164579077649	-0.491612108277076	0.636621979814147\\
2.06776603779046	-0.366529143398341	0.636621979814147\\
2.08595444144867	-0.242474881586856	0.636621979814147\\
2.09656268024469	-0.120103820943333	0.636621979814147\\
2.1	-8.52057412073982e-16	0.636621979814147\\
2.12	0	0.636621979814147\\
2.11652994386607	0.121247666857078	0.636621979814147\\
2.10582067422437	0.244784166173396	0.636621979814147\\
2.08745904767418	0.370019897144991	0.636621979814147\\
2.06109003640294	0.496294128355904	0.636621979814147\\
2.02643008174679	0.622881307948566	0.636621979814147\\
1.98327979527839	0.749000169319411	0.636621979814147\\
1.93153535312287	0.87382560024099	0.636621979814147\\
1.87119791222501	0.996503072391039	0.636621979814147\\
1.80238041003123	1.11616524651847	0.636621979814147\\
1.72531119642828	1.23195019196363	0.636621979814147\\
1.64033408796336	1.34302050612246	0.636621979814147\\
1.54790462190633	1.44858250765396	0.636621979814147\\
1.44858250765396	1.54790462190633	0.636621979814147\\
1.34302050612246	1.64033408796336	0.636621979814147\\
1.23195019196363	1.72531119642828	0.636621979814147\\
1.11616524651847	1.80238041003123	0.636621979814147\\
0.99650307239104	1.87119791222501	0.636621979814147\\
0.87382560024099	1.93153535312287	0.636621979814147\\
0.749000169319412	1.98327979527839	0.636621979814147\\
0.622881307948567	2.02643008174679	0.636621979814147\\
0.496294128355905	2.06109003640294	0.636621979814147\\
0.370019897144992	2.08745904767418	0.636621979814147\\
0.244784166173397	2.10582067422437	0.636621979814147\\
0.121247666857079	2.11652994386607	0.636621979814147\\
4.70734562441066e-16	2.12	0.636621979814147\\
-0.121247666857078	2.11652994386607	0.636621979814147\\
-0.244784166173396	2.10582067422437	0.636621979814147\\
-0.370019897144991	2.08745904767418	0.636621979814147\\
-0.496294128355904	2.06109003640294	0.636621979814147\\
-0.622881307948566	2.02643008174679	0.636621979814147\\
-0.749000169319411	1.98327979527839	0.636621979814147\\
-0.87382560024099	1.93153535312287	0.636621979814147\\
-0.996503072391039	1.87119791222501	0.636621979814147\\
-1.11616524651847	1.80238041003123	0.636621979814147\\
-1.23195019196363	1.72531119642828	0.636621979814147\\
-1.34302050612246	1.64033408796336	0.636621979814147\\
-1.44858250765396	1.54790462190633	0.636621979814147\\
-1.54790462190633	1.44858250765396	0.636621979814147\\
-1.64033408796335	1.34302050612246	0.636621979814147\\
-1.72531119642828	1.23195019196363	0.636621979814147\\
-1.80238041003123	1.11616524651847	0.636621979814147\\
-1.87119791222501	0.99650307239104	0.636621979814147\\
-1.93153535312287	0.87382560024099	0.636621979814147\\
-1.98327979527839	0.749000169319412	0.636621979814147\\
-2.02643008174679	0.622881307948567	0.636621979814147\\
-2.06109003640294	0.496294128355905	0.636621979814147\\
-2.08745904767418	0.370019897144992	0.636621979814147\\
-2.10582067422437	0.244784166173397	0.636621979814147\\
-2.11652994386607	0.121247666857079	0.636621979814147\\
-2.12	6.00547123150686e-16	0.636621979814147\\
-2.11652994386607	-0.121247666857078	0.636621979814147\\
-2.10582067422437	-0.244784166173396	0.636621979814147\\
-2.08745904767418	-0.370019897144991	0.636621979814147\\
-2.06109003640294	-0.496294128355904	0.636621979814147\\
-2.02643008174679	-0.622881307948566	0.636621979814147\\
-1.98327979527839	-0.749000169319412	0.636621979814147\\
-1.93153535312287	-0.873825600240989	0.636621979814147\\
-1.87119791222501	-0.996503072391039	0.636621979814147\\
-1.80238041003123	-1.11616524651847	0.636621979814147\\
-1.72531119642828	-1.23195019196363	0.636621979814147\\
-1.64033408796336	-1.34302050612246	0.636621979814147\\
-1.54790462190633	-1.44858250765396	0.636621979814147\\
-1.44858250765396	-1.54790462190633	0.636621979814147\\
-1.34302050612246	-1.64033408796335	0.636621979814147\\
-1.23195019196363	-1.72531119642828	0.636621979814147\\
-1.11616524651847	-1.80238041003123	0.636621979814147\\
-0.996503072391039	-1.87119791222501	0.636621979814147\\
-0.87382560024099	-1.93153535312287	0.636621979814147\\
-0.749000169319413	-1.98327979527839	0.636621979814147\\
-0.622881307948568	-2.02643008174679	0.636621979814147\\
-0.496294128355905	-2.06109003640294	0.636621979814147\\
-0.370019897144992	-2.08745904767418	0.636621979814147\\
-0.244784166173397	-2.10582067422437	0.636621979814147\\
-0.121247666857079	-2.11652994386607	0.636621979814147\\
-7.30359683860305e-16	-2.12	0.636621979814147\\
0.121247666857078	-2.11652994386607	0.636621979814147\\
0.244784166173396	-2.10582067422437	0.636621979814147\\
0.370019897144991	-2.08745904767418	0.636621979814147\\
0.496294128355904	-2.06109003640294	0.636621979814147\\
0.622881307948566	-2.02643008174679	0.636621979814147\\
0.749000169319412	-1.98327979527839	0.636621979814147\\
0.873825600240989	-1.93153535312287	0.636621979814147\\
0.996503072391039	-1.87119791222501	0.636621979814147\\
1.11616524651847	-1.80238041003123	0.636621979814147\\
1.23195019196363	-1.72531119642828	0.636621979814147\\
1.34302050612246	-1.64033408796336	0.636621979814147\\
1.44858250765396	-1.54790462190633	0.636621979814147\\
1.54790462190633	-1.44858250765396	0.636621979814147\\
1.64033408796335	-1.34302050612246	0.636621979814147\\
1.72531119642828	-1.23195019196363	0.636621979814147\\
1.80238041003123	-1.11616524651847	0.636621979814147\\
1.87119791222501	-0.996503072391039	0.636621979814147\\
1.93153535312287	-0.87382560024099	0.636621979814147\\
1.98327979527839	-0.749000169319413	0.636621979814147\\
2.02643008174679	-0.622881307948568	0.636621979814147\\
2.06109003640294	-0.496294128355905	0.636621979814147\\
2.08745904767418	-0.370019897144992	0.636621979814147\\
2.10582067422437	-0.244784166173397	0.636621979814147\\
2.11652994386607	-0.121247666857079	0.636621979814147\\
2.12	-8.60172244569925e-16	0.636621979814147\\
2.14	0	0.636621979814147\\
2.13649720748745	0.122391512770824	0.636621979814147\\
2.12568690700008	0.247093450759938	0.636621979814147\\
2.1071520575579	0.373510650891642	0.636621979814147\\
2.08053428202938	0.500976148434734	0.636621979814147\\
2.04554734666892	0.628757546702798	0.636621979814147\\
2.0019899820263	0.756066208652613	0.636621979814147\\
1.9497573847561	0.882069237979113	0.636621979814147\\
1.88885072271769	1.00590404477209	0.636621979814147\\
1.81938399880511	1.12669510733468	0.636621979814147\\
1.7415877171493	1.24357236358593	0.636621979814147\\
1.6558089378498	1.3556905108972	0.636621979814147\\
1.5625074956979	1.46224838036767	0.636621979814147\\
1.46224838036767	1.5625074956979	0.636621979814147\\
1.3556905108972	1.6558089378498	0.636621979814147\\
1.24357236358593	1.7415877171493	0.636621979814147\\
1.12669510733468	1.81938399880511	0.636621979814147\\
1.00590404477209	1.88885072271769	0.636621979814147\\
0.882069237979113	1.9497573847561	0.636621979814147\\
0.756066208652614	2.0019899820263	0.636621979814147\\
0.628757546702798	2.04554734666893	0.636621979814147\\
0.500976148434734	2.08053428202938	0.636621979814147\\
0.373510650891643	2.1071520575579	0.636621979814147\\
0.247093450759938	2.12568690700008	0.636621979814147\\
0.122391512770825	2.13649720748745	0.636621979814147\\
4.75175454539567e-16	2.14	0.636621979814147\\
-0.122391512770824	2.13649720748745	0.636621979814147\\
-0.247093450759938	2.12568690700008	0.636621979814147\\
-0.373510650891642	2.1071520575579	0.636621979814147\\
-0.500976148434733	2.08053428202938	0.636621979814147\\
-0.628757546702798	2.04554734666893	0.636621979814147\\
-0.756066208652613	2.0019899820263	0.636621979814147\\
-0.882069237979112	1.9497573847561	0.636621979814147\\
-1.00590404477209	1.88885072271769	0.636621979814147\\
-1.12669510733468	1.81938399880511	0.636621979814147\\
-1.24357236358593	1.7415877171493	0.636621979814147\\
-1.3556905108972	1.6558089378498	0.636621979814147\\
-1.46224838036767	1.5625074956979	0.636621979814147\\
-1.5625074956979	1.46224838036767	0.636621979814147\\
-1.6558089378498	1.3556905108972	0.636621979814147\\
-1.7415877171493	1.24357236358593	0.636621979814147\\
-1.81938399880511	1.12669510733468	0.636621979814147\\
-1.88885072271769	1.00590404477209	0.636621979814147\\
-1.9497573847561	0.882069237979113	0.636621979814147\\
-2.0019899820263	0.756066208652614	0.636621979814147\\
-2.04554734666893	0.628757546702798	0.636621979814147\\
-2.08053428202938	0.500976148434734	0.636621979814147\\
-2.1071520575579	0.373510650891643	0.636621979814147\\
-2.12568690700008	0.247093450759939	0.636621979814147\\
-2.13649720748745	0.122391512770825	0.636621979814147\\
-2.14	6.06212662048334e-16	0.636621979814147\\
-2.13649720748745	-0.122391512770824	0.636621979814147\\
-2.12568690700008	-0.247093450759938	0.636621979814147\\
-2.1071520575579	-0.373510650891642	0.636621979814147\\
-2.08053428202938	-0.500976148434733	0.636621979814147\\
-2.04554734666892	-0.628757546702798	0.636621979814147\\
-2.0019899820263	-0.756066208652614	0.636621979814147\\
-1.9497573847561	-0.882069237979111	0.636621979814147\\
-1.88885072271769	-1.00590404477209	0.636621979814147\\
-1.81938399880511	-1.12669510733468	0.636621979814147\\
-1.7415877171493	-1.24357236358593	0.636621979814147\\
-1.6558089378498	-1.3556905108972	0.636621979814147\\
-1.5625074956979	-1.46224838036767	0.636621979814147\\
-1.46224838036767	-1.5625074956979	0.636621979814147\\
-1.3556905108972	-1.6558089378498	0.636621979814147\\
-1.24357236358593	-1.7415877171493	0.636621979814147\\
-1.12669510733468	-1.81938399880511	0.636621979814147\\
-1.00590404477209	-1.88885072271769	0.636621979814147\\
-0.882069237979112	-1.9497573847561	0.636621979814147\\
-0.756066208652615	-2.0019899820263	0.636621979814147\\
-0.628757546702799	-2.04554734666892	0.636621979814147\\
-0.500976148434735	-2.08053428202938	0.636621979814147\\
-0.373510650891643	-2.1071520575579	0.636621979814147\\
-0.247093450759939	-2.12568690700008	0.636621979814147\\
-0.122391512770825	-2.13649720748745	0.636621979814147\\
-7.37249869557101e-16	-2.14	0.636621979814147\\
0.122391512770824	-2.13649720748745	0.636621979814147\\
0.247093450759938	-2.12568690700008	0.636621979814147\\
0.373510650891642	-2.1071520575579	0.636621979814147\\
0.500976148434733	-2.08053428202938	0.636621979814147\\
0.628757546702798	-2.04554734666892	0.636621979814147\\
0.756066208652613	-2.0019899820263	0.636621979814147\\
0.882069237979111	-1.9497573847561	0.636621979814147\\
1.00590404477209	-1.88885072271769	0.636621979814147\\
1.12669510733468	-1.81938399880511	0.636621979814147\\
1.24357236358593	-1.7415877171493	0.636621979814147\\
1.3556905108972	-1.6558089378498	0.636621979814147\\
1.46224838036767	-1.5625074956979	0.636621979814147\\
1.5625074956979	-1.46224838036767	0.636621979814147\\
1.6558089378498	-1.3556905108972	0.636621979814147\\
1.7415877171493	-1.24357236358593	0.636621979814147\\
1.81938399880511	-1.12669510733468	0.636621979814147\\
1.88885072271769	-1.00590404477209	0.636621979814147\\
1.9497573847561	-0.882069237979112	0.636621979814147\\
2.0019899820263	-0.756066208652615	0.636621979814147\\
2.04554734666892	-0.628757546702799	0.636621979814147\\
2.08053428202938	-0.500976148434735	0.636621979814147\\
2.1071520575579	-0.373510650891643	0.636621979814147\\
2.12568690700008	-0.247093450759939	0.636621979814147\\
2.13649720748745	-0.122391512770825	0.636621979814147\\
2.14	-8.68287077065867e-16	0.636621979814147\\
2.16	0	0.636621979814147\\
2.15646447110883	0.12353535868457	0.636621979814147\\
2.14555313977578	0.249402735346479	0.636621979814147\\
2.12684506744162	0.377001404638293	0.636621979814147\\
2.09997852765582	0.505658168513563	0.636621979814147\\
2.06466461159106	0.63463378545703	0.636621979814147\\
2.02070016877421	0.763132247985815	0.636621979814147\\
1.96797941638933	0.890312875717235	0.636621979814147\\
1.90650353321038	1.01530501715313	0.636621979814147\\
1.83638758757899	1.13722496815089	0.636621979814147\\
1.75786423787032	1.25519453520822	0.636621979814147\\
1.67128378773625	1.36836051567194	0.636621979814147\\
1.57711036948947	1.47591425308139	0.636621979814147\\
1.47591425308139	1.57711036948947	0.636621979814147\\
1.36836051567194	1.67128378773625	0.636621979814147\\
1.25519453520823	1.75786423787032	0.636621979814147\\
1.13722496815089	1.83638758757899	0.636621979814147\\
1.01530501715313	1.90650353321038	0.636621979814147\\
0.890312875717235	1.96797941638933	0.636621979814147\\
0.763132247985816	2.02070016877421	0.636621979814147\\
0.63463378545703	2.06466461159106	0.636621979814147\\
0.505658168513563	2.09997852765582	0.636621979814147\\
0.377001404638293	2.12684506744162	0.636621979814147\\
0.24940273534648	2.14555313977578	0.636621979814147\\
0.123535358684571	2.15646447110883	0.636621979814147\\
4.79616346638068e-16	2.16	0.636621979814147\\
-0.12353535868457	2.15646447110883	0.636621979814147\\
-0.249402735346479	2.14555313977578	0.636621979814147\\
-0.377001404638293	2.12684506744162	0.636621979814147\\
-0.505658168513562	2.09997852765582	0.636621979814147\\
-0.634633785457029	2.06466461159106	0.636621979814147\\
-0.763132247985815	2.02070016877421	0.636621979814147\\
-0.890312875717235	1.96797941638933	0.636621979814147\\
-1.01530501715313	1.90650353321038	0.636621979814147\\
-1.13722496815089	1.83638758757899	0.636621979814147\\
-1.25519453520822	1.75786423787032	0.636621979814147\\
-1.36836051567194	1.67128378773625	0.636621979814147\\
-1.47591425308139	1.57711036948947	0.636621979814147\\
-1.57711036948947	1.47591425308139	0.636621979814147\\
-1.67128378773625	1.36836051567194	0.636621979814147\\
-1.75786423787032	1.25519453520823	0.636621979814147\\
-1.83638758757899	1.1372249681509	0.636621979814147\\
-1.90650353321038	1.01530501715314	0.636621979814147\\
-1.96797941638933	0.890312875717235	0.636621979814147\\
-2.02070016877421	0.763132247985816	0.636621979814147\\
-2.06466461159106	0.63463378545703	0.636621979814147\\
-2.09997852765582	0.505658168513563	0.636621979814147\\
-2.12684506744162	0.377001404638294	0.636621979814147\\
-2.14555313977578	0.24940273534648	0.636621979814147\\
-2.15646447110883	0.123535358684571	0.636621979814147\\
-2.16	6.11878200945982e-16	0.636621979814147\\
-2.15646447110883	-0.12353535868457	0.636621979814147\\
-2.14555313977578	-0.249402735346479	0.636621979814147\\
-2.12684506744162	-0.377001404638293	0.636621979814147\\
-2.09997852765582	-0.505658168513563	0.636621979814147\\
-2.06466461159106	-0.63463378545703	0.636621979814147\\
-2.02070016877421	-0.763132247985816	0.636621979814147\\
-1.96797941638934	-0.890312875717234	0.636621979814147\\
-1.90650353321038	-1.01530501715313	0.636621979814147\\
-1.83638758757899	-1.13722496815089	0.636621979814147\\
-1.75786423787032	-1.25519453520822	0.636621979814147\\
-1.67128378773625	-1.36836051567194	0.636621979814147\\
-1.57711036948947	-1.47591425308139	0.636621979814147\\
-1.47591425308139	-1.57711036948947	0.636621979814147\\
-1.36836051567194	-1.67128378773625	0.636621979814147\\
-1.25519453520823	-1.75786423787032	0.636621979814147\\
-1.13722496815089	-1.83638758757899	0.636621979814147\\
-1.01530501715313	-1.90650353321038	0.636621979814147\\
-0.890312875717235	-1.96797941638934	0.636621979814147\\
-0.763132247985817	-2.02070016877421	0.636621979814147\\
-0.634633785457031	-2.06466461159106	0.636621979814147\\
-0.505658168513564	-2.09997852765582	0.636621979814147\\
-0.377001404638294	-2.12684506744162	0.636621979814147\\
-0.24940273534648	-2.14555313977578	0.636621979814147\\
-0.123535358684571	-2.15646447110883	0.636621979814147\\
-7.44140055253896e-16	-2.16	0.636621979814147\\
0.12353535868457	-2.15646447110883	0.636621979814147\\
0.249402735346479	-2.14555313977578	0.636621979814147\\
0.377001404638293	-2.12684506744162	0.636621979814147\\
0.505658168513563	-2.09997852765582	0.636621979814147\\
0.63463378545703	-2.06466461159106	0.636621979814147\\
0.763132247985816	-2.02070016877421	0.636621979814147\\
0.890312875717234	-1.96797941638934	0.636621979814147\\
1.01530501715313	-1.90650353321038	0.636621979814147\\
1.13722496815089	-1.83638758757899	0.636621979814147\\
1.25519453520822	-1.75786423787032	0.636621979814147\\
1.36836051567194	-1.67128378773625	0.636621979814147\\
1.47591425308139	-1.57711036948947	0.636621979814147\\
1.57711036948947	-1.47591425308139	0.636621979814147\\
1.67128378773625	-1.36836051567194	0.636621979814147\\
1.75786423787032	-1.25519453520823	0.636621979814147\\
1.83638758757899	-1.13722496815089	0.636621979814147\\
1.90650353321038	-1.01530501715313	0.636621979814147\\
1.96797941638934	-0.890312875717235	0.636621979814147\\
2.02070016877421	-0.763132247985817	0.636621979814147\\
2.06466461159106	-0.634633785457031	0.636621979814147\\
2.09997852765582	-0.505658168513564	0.636621979814147\\
2.12684506744162	-0.377001404638294	0.636621979814147\\
2.14555313977578	-0.249402735346481	0.636621979814147\\
2.15646447110883	-0.123535358684571	0.636621979814147\\
2.16	-8.7640190956181e-16	0.636621979814147\\
2.18	0	0.636621979814147\\
2.17643173473021	0.124679204598316	0.636621979814147\\
2.16541937255148	0.251712019933021	0.636621979814147\\
2.14653807732534	0.380492158384944	0.636621979814147\\
2.11942277328226	0.510340188592392	0.636621979814147\\
2.0837818765132	0.640510024211261	0.636621979814147\\
2.03941035552212	0.770198287319017	0.636621979814147\\
1.98620144802257	0.898556513455358	0.636621979814147\\
1.92415634370307	1.02470598953418	0.636621979814147\\
1.85339117635287	1.14775482896711	0.636621979814147\\
1.77414075859134	1.26681670683052	0.636621979814147\\
1.6867586376227	1.38103052044668	0.636621979814147\\
1.59171324328104	1.48958012579511	0.636621979814147\\
1.48958012579511	1.59171324328104	0.636621979814147\\
1.38103052044668	1.6867586376227	0.636621979814147\\
1.26681670683052	1.77414075859134	0.636621979814147\\
1.14775482896711	1.85339117635287	0.636621979814147\\
1.02470598953418	1.92415634370307	0.636621979814147\\
0.898556513455358	1.98620144802257	0.636621979814147\\
0.770198287319018	2.03941035552212	0.636621979814147\\
0.640510024211262	2.0837818765132	0.636621979814147\\
0.510340188592393	2.11942277328226	0.636621979814147\\
0.380492158384944	2.14653807732534	0.636621979814147\\
0.251712019933021	2.16541937255148	0.636621979814147\\
0.124679204598317	2.17643173473021	0.636621979814147\\
4.84057238736568e-16	2.18	0.636621979814147\\
-0.124679204598316	2.1764317347302	0.636621979814147\\
-0.251712019933021	2.16541937255148	0.636621979814147\\
-0.380492158384944	2.14653807732534	0.636621979814147\\
-0.510340188592391	2.11942277328226	0.636621979814147\\
-0.640510024211261	2.0837818765132	0.636621979814147\\
-0.770198287319017	2.03941035552212	0.636621979814147\\
-0.898556513455357	1.98620144802257	0.636621979814147\\
-1.02470598953418	1.92415634370307	0.636621979814147\\
-1.14775482896711	1.85339117635287	0.636621979814147\\
-1.26681670683052	1.77414075859134	0.636621979814147\\
-1.38103052044668	1.6867586376227	0.636621979814147\\
-1.48958012579511	1.59171324328104	0.636621979814147\\
-1.59171324328104	1.48958012579511	0.636621979814147\\
-1.6867586376227	1.38103052044668	0.636621979814147\\
-1.77414075859134	1.26681670683052	0.636621979814147\\
-1.85339117635287	1.14775482896711	0.636621979814147\\
-1.92415634370307	1.02470598953418	0.636621979814147\\
-1.98620144802257	0.898556513455358	0.636621979814147\\
-2.03941035552212	0.770198287319018	0.636621979814147\\
-2.0837818765132	0.640510024211262	0.636621979814147\\
-2.11942277328226	0.510340188592392	0.636621979814147\\
-2.14653807732534	0.380492158384945	0.636621979814147\\
-2.16541937255148	0.251712019933022	0.636621979814147\\
-2.17643173473021	0.124679204598317	0.636621979814147\\
-2.18	6.1754373984363e-16	0.636621979814147\\
-2.1764317347302	-0.124679204598316	0.636621979814147\\
-2.16541937255148	-0.251712019933021	0.636621979814147\\
-2.14653807732534	-0.380492158384944	0.636621979814147\\
-2.11942277328226	-0.510340188592392	0.636621979814147\\
-2.0837818765132	-0.640510024211262	0.636621979814147\\
-2.03941035552212	-0.770198287319018	0.636621979814147\\
-1.98620144802257	-0.898556513455356	0.636621979814147\\
-1.92415634370307	-1.02470598953418	0.636621979814147\\
-1.85339117635287	-1.14775482896711	0.636621979814147\\
-1.77414075859134	-1.26681670683052	0.636621979814147\\
-1.6867586376227	-1.38103052044668	0.636621979814147\\
-1.59171324328104	-1.48958012579511	0.636621979814147\\
-1.48958012579511	-1.59171324328104	0.636621979814147\\
-1.38103052044668	-1.6867586376227	0.636621979814147\\
-1.26681670683052	-1.77414075859134	0.636621979814147\\
-1.14775482896711	-1.85339117635287	0.636621979814147\\
-1.02470598953418	-1.92415634370307	0.636621979814147\\
-0.898556513455357	-1.98620144802257	0.636621979814147\\
-0.770198287319019	-2.03941035552212	0.636621979814147\\
-0.640510024211263	-2.0837818765132	0.636621979814147\\
-0.510340188592393	-2.11942277328226	0.636621979814147\\
-0.380492158384945	-2.14653807732534	0.636621979814147\\
-0.251712019933022	-2.16541937255148	0.636621979814147\\
-0.124679204598317	-2.17643173473021	0.636621979814147\\
-7.51030240950691e-16	-2.18	0.636621979814147\\
0.124679204598316	-2.1764317347302	0.636621979814147\\
0.251712019933021	-2.16541937255148	0.636621979814147\\
0.380492158384944	-2.14653807732534	0.636621979814147\\
0.510340188592392	-2.11942277328226	0.636621979814147\\
0.640510024211262	-2.0837818765132	0.636621979814147\\
0.770198287319018	-2.03941035552212	0.636621979814147\\
0.898556513455356	-1.98620144802257	0.636621979814147\\
1.02470598953418	-1.92415634370307	0.636621979814147\\
1.14775482896711	-1.85339117635287	0.636621979814147\\
1.26681670683052	-1.77414075859134	0.636621979814147\\
1.38103052044668	-1.6867586376227	0.636621979814147\\
1.48958012579511	-1.59171324328104	0.636621979814147\\
1.59171324328104	-1.48958012579511	0.636621979814147\\
1.6867586376227	-1.38103052044668	0.636621979814147\\
1.77414075859134	-1.26681670683052	0.636621979814147\\
1.85339117635287	-1.14775482896711	0.636621979814147\\
1.92415634370307	-1.02470598953418	0.636621979814147\\
1.98620144802257	-0.898556513455357	0.636621979814147\\
2.03941035552212	-0.770198287319019	0.636621979814147\\
2.0837818765132	-0.640510024211263	0.636621979814147\\
2.11942277328226	-0.510340188592393	0.636621979814147\\
2.14653807732534	-0.380492158384945	0.636621979814147\\
2.16541937255148	-0.251712019933022	0.636621979814147\\
2.17643173473021	-0.124679204598317	0.636621979814147\\
2.18	-8.84516742057753e-16	0.636621979814147\\
2.2	0	0.636621979814147\\
2.19639899835158	0.125823050512062	0.636621979814147\\
2.18528560532718	0.254021304519562	0.636621979814147\\
2.16623108720906	0.383982912131595	0.636621979814147\\
2.13886701890871	0.515022208671222	0.636621979814147\\
2.10289914143534	0.646386262965493	0.636621979814147\\
2.05812054227003	0.777264326652219	0.636621979814147\\
2.0044234796558	0.90680015119348	0.636621979814147\\
1.94180915419576	1.03410696191523	0.636621979814147\\
1.87039476512675	1.15828468978332	0.636621979814147\\
1.79041727931237	1.27843887845282	0.636621979814147\\
1.70223348750914	1.39370052522142	0.636621979814147\\
1.60631611707261	1.50324599850882	0.636621979814147\\
1.50324599850882	1.60631611707261	0.636621979814147\\
1.39370052522142	1.70223348750914	0.636621979814147\\
1.27843887845282	1.79041727931237	0.636621979814147\\
1.15828468978332	1.87039476512675	0.636621979814147\\
1.03410696191523	1.94180915419576	0.636621979814147\\
0.90680015119348	2.0044234796558	0.636621979814147\\
0.77726432665222	2.05812054227003	0.636621979814147\\
0.646386262965494	2.10289914143534	0.636621979814147\\
0.515022208671222	2.13886701890871	0.636621979814147\\
0.383982912131595	2.16623108720906	0.636621979814147\\
0.254021304519563	2.18528560532718	0.636621979814147\\
0.125823050512063	2.19639899835158	0.636621979814147\\
4.88498130835069e-16	2.2	0.636621979814147\\
-0.125823050512062	2.19639899835158	0.636621979814147\\
-0.254021304519562	2.18528560532718	0.636621979814147\\
-0.383982912131595	2.16623108720906	0.636621979814147\\
-0.515022208671221	2.13886701890871	0.636621979814147\\
-0.646386262965493	2.10289914143534	0.636621979814147\\
-0.777264326652219	2.05812054227003	0.636621979814147\\
-0.90680015119348	2.0044234796558	0.636621979814147\\
-1.03410696191523	1.94180915419576	0.636621979814147\\
-1.15828468978332	1.87039476512675	0.636621979814147\\
-1.27843887845282	1.79041727931237	0.636621979814147\\
-1.39370052522142	1.70223348750914	0.636621979814147\\
-1.50324599850882	1.60631611707261	0.636621979814147\\
-1.60631611707261	1.50324599850882	0.636621979814147\\
-1.70223348750914	1.39370052522142	0.636621979814147\\
-1.79041727931237	1.27843887845282	0.636621979814147\\
-1.87039476512675	1.15828468978332	0.636621979814147\\
-1.94180915419576	1.03410696191523	0.636621979814147\\
-2.0044234796558	0.90680015119348	0.636621979814147\\
-2.05812054227003	0.77726432665222	0.636621979814147\\
-2.10289914143534	0.646386262965494	0.636621979814147\\
-2.13886701890871	0.515022208671222	0.636621979814147\\
-2.16623108720906	0.383982912131596	0.636621979814147\\
-2.18528560532718	0.254021304519563	0.636621979814147\\
-2.19639899835158	0.125823050512063	0.636621979814147\\
-2.2	6.23209278741278e-16	0.636621979814147\\
-2.19639899835158	-0.125823050512062	0.636621979814147\\
-2.18528560532718	-0.254021304519562	0.636621979814147\\
-2.16623108720906	-0.383982912131595	0.636621979814147\\
-2.13886701890871	-0.515022208671221	0.636621979814147\\
-2.10289914143534	-0.646386262965493	0.636621979814147\\
-2.05812054227003	-0.77726432665222	0.636621979814147\\
-2.0044234796558	-0.906800151193479	0.636621979814147\\
-1.94180915419576	-1.03410696191523	0.636621979814147\\
-1.87039476512675	-1.15828468978332	0.636621979814147\\
-1.79041727931237	-1.27843887845282	0.636621979814147\\
-1.70223348750914	-1.39370052522142	0.636621979814147\\
-1.60631611707261	-1.50324599850882	0.636621979814147\\
-1.50324599850882	-1.60631611707261	0.636621979814147\\
-1.39370052522142	-1.70223348750914	0.636621979814147\\
-1.27843887845282	-1.79041727931237	0.636621979814147\\
-1.15828468978332	-1.87039476512675	0.636621979814147\\
-1.03410696191523	-1.94180915419576	0.636621979814147\\
-0.90680015119348	-2.0044234796558	0.636621979814147\\
-0.777264326652221	-2.05812054227003	0.636621979814147\\
-0.646386262965495	-2.10289914143534	0.636621979814147\\
-0.515022208671223	-2.13886701890871	0.636621979814147\\
-0.383982912131596	-2.16623108720906	0.636621979814147\\
-0.254021304519563	-2.18528560532718	0.636621979814147\\
-0.125823050512063	-2.19639899835158	0.636621979814147\\
-7.57920426647487e-16	-2.2	0.636621979814147\\
0.125823050512062	-2.19639899835158	0.636621979814147\\
0.254021304519562	-2.18528560532718	0.636621979814147\\
0.383982912131594	-2.16623108720906	0.636621979814147\\
0.515022208671221	-2.13886701890871	0.636621979814147\\
0.646386262965493	-2.10289914143534	0.636621979814147\\
0.77726432665222	-2.05812054227003	0.636621979814147\\
0.906800151193479	-2.0044234796558	0.636621979814147\\
1.03410696191523	-1.94180915419576	0.636621979814147\\
1.15828468978332	-1.87039476512675	0.636621979814147\\
1.27843887845282	-1.79041727931237	0.636621979814147\\
1.39370052522142	-1.70223348750914	0.636621979814147\\
1.50324599850882	-1.60631611707261	0.636621979814147\\
1.60631611707261	-1.50324599850882	0.636621979814147\\
1.70223348750914	-1.39370052522142	0.636621979814147\\
1.79041727931237	-1.27843887845282	0.636621979814147\\
1.87039476512675	-1.15828468978332	0.636621979814147\\
1.94180915419576	-1.03410696191523	0.636621979814147\\
2.0044234796558	-0.90680015119348	0.636621979814147\\
2.05812054227003	-0.777264326652221	0.636621979814147\\
2.10289914143534	-0.646386262965495	0.636621979814147\\
2.13886701890871	-0.515022208671223	0.636621979814147\\
2.16623108720906	-0.383982912131596	0.636621979814147\\
2.18528560532718	-0.254021304519563	0.636621979814147\\
2.19639899835158	-0.125823050512063	0.636621979814147\\
2.2	-8.92631574553695e-16	0.636621979814147\\
2.22	0	0.636621979814147\\
2.21636626197296	0.126966896425808	0.636621979814147\\
2.20515183810288	0.256330589106104	0.636621979814147\\
2.18592409709278	0.387473665878246	0.636621979814147\\
2.15831126453515	0.519704228750051	0.636621979814147\\
2.12201640635748	0.652262501719725	0.636621979814147\\
2.07683072901794	0.784330365985421	0.636621979814147\\
2.02264551128904	0.915043788931603	0.636621979814147\\
1.95946196468845	1.04350793429628	0.636621979814147\\
1.88739835390063	1.16881455059953	0.636621979814147\\
1.80669380003339	1.29006105007512	0.636621979814147\\
1.71770833739559	1.40637052999616	0.636621979814147\\
1.62091899086418	1.51691187122254	0.636621979814147\\
1.51691187122254	1.62091899086418	0.636621979814147\\
1.40637052999616	1.71770833739559	0.636621979814147\\
1.29006105007512	1.80669380003339	0.636621979814147\\
1.16881455059953	1.88739835390063	0.636621979814147\\
1.04350793429628	1.95946196468845	0.636621979814147\\
0.915043788931603	2.02264551128904	0.636621979814147\\
0.784330365985422	2.07683072901794	0.636621979814147\\
0.652262501719725	2.12201640635748	0.636621979814147\\
0.519704228750051	2.15831126453515	0.636621979814147\\
0.387473665878246	2.18592409709278	0.636621979814147\\
0.256330589106104	2.20515183810288	0.636621979814147\\
0.126966896425809	2.21636626197296	0.636621979814147\\
4.9293902293357e-16	2.22	0.636621979814147\\
-0.126966896425808	2.21636626197296	0.636621979814147\\
-0.256330589106104	2.20515183810288	0.636621979814147\\
-0.387473665878246	2.18592409709278	0.636621979814147\\
-0.51970422875005	2.15831126453515	0.636621979814147\\
-0.652262501719725	2.12201640635748	0.636621979814147\\
-0.784330365985421	2.07683072901794	0.636621979814147\\
-0.915043788931602	2.02264551128904	0.636621979814147\\
-1.04350793429628	1.95946196468845	0.636621979814147\\
-1.16881455059953	1.88739835390063	0.636621979814147\\
-1.29006105007512	1.80669380003339	0.636621979814147\\
-1.40637052999616	1.71770833739559	0.636621979814147\\
-1.51691187122254	1.62091899086418	0.636621979814147\\
-1.62091899086417	1.51691187122254	0.636621979814147\\
-1.71770833739559	1.40637052999616	0.636621979814147\\
-1.80669380003339	1.29006105007512	0.636621979814147\\
-1.88739835390063	1.16881455059953	0.636621979814147\\
-1.95946196468845	1.04350793429628	0.636621979814147\\
-2.02264551128904	0.915043788931603	0.636621979814147\\
-2.07683072901794	0.784330365985422	0.636621979814147\\
-2.12201640635748	0.652262501719725	0.636621979814147\\
-2.15831126453515	0.519704228750051	0.636621979814147\\
-2.18592409709278	0.387473665878247	0.636621979814147\\
-2.20515183810288	0.256330589106105	0.636621979814147\\
-2.21636626197296	0.126966896425809	0.636621979814147\\
-2.22	6.28874817638926e-16	0.636621979814147\\
-2.21636626197296	-0.126966896425808	0.636621979814147\\
-2.20515183810288	-0.256330589106104	0.636621979814147\\
-2.18592409709278	-0.387473665878245	0.636621979814147\\
-2.15831126453515	-0.519704228750051	0.636621979814147\\
-2.12201640635748	-0.652262501719725	0.636621979814147\\
-2.07683072901794	-0.784330365985422	0.636621979814147\\
-2.02264551128904	-0.915043788931601	0.636621979814147\\
-1.95946196468845	-1.04350793429628	0.636621979814147\\
-1.88739835390063	-1.16881455059953	0.636621979814147\\
-1.80669380003339	-1.29006105007512	0.636621979814147\\
-1.71770833739559	-1.40637052999616	0.636621979814147\\
-1.62091899086418	-1.51691187122254	0.636621979814147\\
-1.51691187122254	-1.62091899086417	0.636621979814147\\
-1.40637052999616	-1.71770833739559	0.636621979814147\\
-1.29006105007512	-1.80669380003339	0.636621979814147\\
-1.16881455059953	-1.88739835390063	0.636621979814147\\
-1.04350793429628	-1.95946196468845	0.636621979814147\\
-0.915043788931602	-2.02264551128904	0.636621979814147\\
-0.784330365985423	-2.07683072901794	0.636621979814147\\
-0.652262501719726	-2.12201640635748	0.636621979814147\\
-0.519704228750052	-2.15831126453515	0.636621979814147\\
-0.387473665878247	-2.18592409709278	0.636621979814147\\
-0.256330589106105	-2.20515183810288	0.636621979814147\\
-0.126966896425809	-2.21636626197296	0.636621979814147\\
-7.64810612344282e-16	-2.22	0.636621979814147\\
0.126966896425808	-2.21636626197296	0.636621979814147\\
0.256330589106103	-2.20515183810288	0.636621979814147\\
0.387473665878245	-2.18592409709278	0.636621979814147\\
0.519704228750051	-2.15831126453515	0.636621979814147\\
0.652262501719725	-2.12201640635748	0.636621979814147\\
0.784330365985422	-2.07683072901794	0.636621979814147\\
0.915043788931601	-2.02264551128904	0.636621979814147\\
1.04350793429628	-1.95946196468845	0.636621979814147\\
1.16881455059953	-1.88739835390063	0.636621979814147\\
1.29006105007512	-1.80669380003339	0.636621979814147\\
1.40637052999616	-1.71770833739559	0.636621979814147\\
1.51691187122254	-1.62091899086418	0.636621979814147\\
1.62091899086417	-1.51691187122254	0.636621979814147\\
1.71770833739559	-1.40637052999616	0.636621979814147\\
1.80669380003339	-1.29006105007512	0.636621979814147\\
1.88739835390063	-1.16881455059953	0.636621979814147\\
1.95946196468845	-1.04350793429628	0.636621979814147\\
2.02264551128904	-0.915043788931602	0.636621979814147\\
2.07683072901794	-0.784330365985423	0.636621979814147\\
2.12201640635748	-0.652262501719726	0.636621979814147\\
2.15831126453515	-0.519704228750052	0.636621979814147\\
2.18592409709278	-0.387473665878247	0.636621979814147\\
2.20515183810288	-0.256330589106105	0.636621979814147\\
2.21636626197296	-0.126966896425809	0.636621979814147\\
2.22	-9.00746407049638e-16	0.636621979814147\\
2.24	0	0.636621979814147\\
2.23633352559434	0.128110742339554	0.636621979814147\\
2.22501807087858	0.258639873692645	0.636621979814147\\
2.2056171069765	0.390964419624897	0.636621979814147\\
2.17775551016159	0.52438624882888	0.636621979814147\\
2.14113367127962	0.658138740473957	0.636621979814147\\
2.09554091576585	0.791396405318623	0.636621979814147\\
2.04086754292227	0.923287426669725	0.636621979814147\\
1.97711477518114	1.05290890667733	0.636621979814147\\
1.90440194267451	1.17934441141574	0.636621979814147\\
1.82297032075441	1.30168322169742	0.636621979814147\\
1.73318318728204	1.4190405347709	0.636621979814147\\
1.63552186465574	1.53057774393626	0.636621979814147\\
1.53057774393626	1.63552186465574	0.636621979814147\\
1.4190405347709	1.73318318728204	0.636621979814147\\
1.30168322169742	1.82297032075441	0.636621979814147\\
1.17934441141574	1.90440194267451	0.636621979814147\\
1.05290890667732	1.97711477518114	0.636621979814147\\
0.923287426669726	2.04086754292227	0.636621979814147\\
0.791396405318624	2.09554091576585	0.636621979814147\\
0.658138740473957	2.14113367127962	0.636621979814147\\
0.524386248828881	2.17775551016159	0.636621979814147\\
0.390964419624897	2.2056171069765	0.636621979814147\\
0.258639873692646	2.22501807087858	0.636621979814147\\
0.128110742339555	2.23633352559434	0.636621979814147\\
4.9737991503207e-16	2.24	0.636621979814147\\
-0.128110742339554	2.23633352559434	0.636621979814147\\
-0.258639873692645	2.22501807087858	0.636621979814147\\
-0.390964419624896	2.2056171069765	0.636621979814147\\
-0.524386248828879	2.17775551016159	0.636621979814147\\
-0.658138740473956	2.14113367127962	0.636621979814147\\
-0.791396405318623	2.09554091576585	0.636621979814147\\
-0.923287426669725	2.04086754292227	0.636621979814147\\
-1.05290890667732	1.97711477518114	0.636621979814147\\
-1.17934441141574	1.90440194267451	0.636621979814147\\
-1.30168322169742	1.82297032075441	0.636621979814147\\
-1.4190405347709	1.73318318728204	0.636621979814147\\
-1.53057774393625	1.63552186465574	0.636621979814147\\
-1.63552186465574	1.53057774393626	0.636621979814147\\
-1.73318318728204	1.4190405347709	0.636621979814147\\
-1.82297032075441	1.30168322169742	0.636621979814147\\
-1.90440194267451	1.17934441141574	0.636621979814147\\
-1.97711477518114	1.05290890667733	0.636621979814147\\
-2.04086754292227	0.923287426669726	0.636621979814147\\
-2.09554091576585	0.791396405318624	0.636621979814147\\
-2.14113367127962	0.658138740473957	0.636621979814147\\
-2.17775551016159	0.52438624882888	0.636621979814147\\
-2.2056171069765	0.390964419624897	0.636621979814147\\
-2.22501807087858	0.258639873692646	0.636621979814147\\
-2.23633352559434	0.128110742339555	0.636621979814147\\
-2.24	6.34540356536574e-16	0.636621979814147\\
-2.23633352559434	-0.128110742339554	0.636621979814147\\
-2.22501807087858	-0.258639873692645	0.636621979814147\\
-2.2056171069765	-0.390964419624896	0.636621979814147\\
-2.17775551016159	-0.52438624882888	0.636621979814147\\
-2.14113367127962	-0.658138740473957	0.636621979814147\\
-2.09554091576585	-0.791396405318624	0.636621979814147\\
-2.04086754292227	-0.923287426669724	0.636621979814147\\
-1.97711477518114	-1.05290890667732	0.636621979814147\\
-1.90440194267451	-1.17934441141574	0.636621979814147\\
-1.82297032075441	-1.30168322169742	0.636621979814147\\
-1.73318318728204	-1.4190405347709	0.636621979814147\\
-1.63552186465574	-1.53057774393625	0.636621979814147\\
-1.53057774393626	-1.63552186465574	0.636621979814147\\
-1.4190405347709	-1.73318318728204	0.636621979814147\\
-1.30168322169742	-1.82297032075441	0.636621979814147\\
-1.17934441141574	-1.90440194267451	0.636621979814147\\
-1.05290890667733	-1.97711477518114	0.636621979814147\\
-0.923287426669725	-2.04086754292227	0.636621979814147\\
-0.791396405318625	-2.09554091576585	0.636621979814147\\
-0.658138740473958	-2.14113367127962	0.636621979814147\\
-0.524386248828881	-2.17775551016159	0.636621979814147\\
-0.390964419624898	-2.2056171069765	0.636621979814147\\
-0.258639873692646	-2.22501807087858	0.636621979814147\\
-0.128110742339555	-2.23633352559434	0.636621979814147\\
-7.71700798041077e-16	-2.24	0.636621979814147\\
0.128110742339554	-2.23633352559434	0.636621979814147\\
0.258639873692645	-2.22501807087858	0.636621979814147\\
0.390964419624896	-2.2056171069765	0.636621979814147\\
0.52438624882888	-2.17775551016159	0.636621979814147\\
0.658138740473957	-2.14113367127962	0.636621979814147\\
0.791396405318624	-2.09554091576585	0.636621979814147\\
0.923287426669724	-2.04086754292227	0.636621979814147\\
1.05290890667732	-1.97711477518114	0.636621979814147\\
1.17934441141574	-1.90440194267451	0.636621979814147\\
1.30168322169742	-1.82297032075441	0.636621979814147\\
1.4190405347709	-1.73318318728204	0.636621979814147\\
1.53057774393625	-1.63552186465574	0.636621979814147\\
1.63552186465574	-1.53057774393626	0.636621979814147\\
1.73318318728204	-1.4190405347709	0.636621979814147\\
1.82297032075441	-1.30168322169742	0.636621979814147\\
1.90440194267451	-1.17934441141574	0.636621979814147\\
1.97711477518114	-1.05290890667733	0.636621979814147\\
2.04086754292227	-0.923287426669725	0.636621979814147\\
2.09554091576585	-0.791396405318625	0.636621979814147\\
2.14113367127962	-0.658138740473958	0.636621979814147\\
2.17775551016159	-0.524386248828881	0.636621979814147\\
2.2056171069765	-0.390964419624898	0.636621979814147\\
2.22501807087858	-0.258639873692646	0.636621979814147\\
2.23633352559434	-0.128110742339555	0.636621979814147\\
2.24	-9.08861239545581e-16	0.636621979814147\\
2.26	0	0.636621979814147\\
2.25630078921572	0.1292545882533	0.636621979814147\\
2.24488430365429	0.260949158279187	0.636621979814147\\
2.22531011686021	0.394455173371547	0.636621979814147\\
2.19719975578804	0.529068268907709	0.636621979814147\\
2.16025093620176	0.664014979228189	0.636621979814147\\
2.11425110251376	0.798462444651825	0.636621979814147\\
2.05908957455551	0.931531064407848	0.636621979814147\\
1.99476758567383	1.06230987905837	0.636621979814147\\
1.92140553144839	1.18987427223195	0.636621979814147\\
1.83924684147543	1.31330539331972	0.636621979814147\\
1.74865803716848	1.43171053954564	0.636621979814147\\
1.65012473844731	1.54424361664997	0.636621979814147\\
1.54424361664997	1.65012473844731	0.636621979814147\\
1.43171053954564	1.74865803716848	0.636621979814147\\
1.31330539331972	1.83924684147543	0.636621979814147\\
1.18987427223195	1.92140553144839	0.636621979814147\\
1.06230987905837	1.99476758567383	0.636621979814147\\
0.931531064407848	2.05908957455551	0.636621979814147\\
0.798462444651826	2.11425110251376	0.636621979814147\\
0.664014979228189	2.16025093620176	0.636621979814147\\
0.52906826890771	2.19719975578804	0.636621979814147\\
0.394455173371548	2.22531011686021	0.636621979814147\\
0.260949158279187	2.24488430365429	0.636621979814147\\
0.129254588253301	2.25630078921572	0.636621979814147\\
5.01820807130571e-16	2.26	0.636621979814147\\
-0.1292545882533	2.25630078921572	0.636621979814147\\
-0.260949158279187	2.24488430365429	0.636621979814147\\
-0.394455173371547	2.22531011686021	0.636621979814147\\
-0.529068268907709	2.19719975578803	0.636621979814147\\
-0.664014979228188	2.16025093620176	0.636621979814147\\
-0.798462444651825	2.11425110251376	0.636621979814147\\
-0.931531064407848	2.05908957455551	0.636621979814147\\
-1.06230987905837	1.99476758567383	0.636621979814147\\
-1.18987427223195	1.92140553144839	0.636621979814147\\
-1.31330539331972	1.83924684147543	0.636621979814147\\
-1.43171053954564	1.74865803716848	0.636621979814147\\
-1.54424361664997	1.65012473844731	0.636621979814147\\
-1.65012473844731	1.54424361664997	0.636621979814147\\
-1.74865803716848	1.43171053954564	0.636621979814147\\
-1.83924684147543	1.31330539331972	0.636621979814147\\
-1.92140553144839	1.18987427223196	0.636621979814147\\
-1.99476758567383	1.06230987905837	0.636621979814147\\
-2.05908957455551	0.931531064407848	0.636621979814147\\
-2.11425110251376	0.798462444651826	0.636621979814147\\
-2.16025093620176	0.664014979228189	0.636621979814147\\
-2.19719975578803	0.52906826890771	0.636621979814147\\
-2.22531011686021	0.394455173371548	0.636621979814147\\
-2.24488430365428	0.260949158279188	0.636621979814147\\
-2.25630078921572	0.129254588253301	0.636621979814147\\
-2.26	6.40205895434222e-16	0.636621979814147\\
-2.25630078921572	-0.1292545882533	0.636621979814147\\
-2.24488430365429	-0.260949158279187	0.636621979814147\\
-2.22531011686021	-0.394455173371547	0.636621979814147\\
-2.19719975578803	-0.529068268907709	0.636621979814147\\
-2.16025093620176	-0.664014979228189	0.636621979814147\\
-2.11425110251376	-0.798462444651826	0.636621979814147\\
-2.05908957455551	-0.931531064407846	0.636621979814147\\
-1.99476758567383	-1.06230987905837	0.636621979814147\\
-1.92140553144839	-1.18987427223195	0.636621979814147\\
-1.83924684147543	-1.31330539331972	0.636621979814147\\
-1.74865803716848	-1.43171053954564	0.636621979814147\\
-1.65012473844731	-1.54424361664997	0.636621979814147\\
-1.54424361664997	-1.65012473844731	0.636621979814147\\
-1.43171053954564	-1.74865803716848	0.636621979814147\\
-1.31330539331972	-1.83924684147543	0.636621979814147\\
-1.18987427223195	-1.92140553144839	0.636621979814147\\
-1.06230987905837	-1.99476758567383	0.636621979814147\\
-0.931531064407848	-2.05908957455551	0.636621979814147\\
-0.798462444651827	-2.11425110251375	0.636621979814147\\
-0.66401497922819	-2.16025093620176	0.636621979814147\\
-0.52906826890771	-2.19719975578803	0.636621979814147\\
-0.394455173371548	-2.22531011686021	0.636621979814147\\
-0.260949158279188	-2.24488430365428	0.636621979814147\\
-0.129254588253301	-2.25630078921572	0.636621979814147\\
-7.78590983737873e-16	-2.26	0.636621979814147\\
0.1292545882533	-2.25630078921572	0.636621979814147\\
0.260949158279186	-2.24488430365429	0.636621979814147\\
0.394455173371547	-2.22531011686021	0.636621979814147\\
0.529068268907709	-2.19719975578803	0.636621979814147\\
0.664014979228189	-2.16025093620176	0.636621979814147\\
0.798462444651825	-2.11425110251376	0.636621979814147\\
0.931531064407846	-2.05908957455551	0.636621979814147\\
1.06230987905837	-1.99476758567383	0.636621979814147\\
1.18987427223195	-1.92140553144839	0.636621979814147\\
1.31330539331972	-1.83924684147543	0.636621979814147\\
1.43171053954564	-1.74865803716848	0.636621979814147\\
1.54424361664997	-1.65012473844731	0.636621979814147\\
1.65012473844731	-1.54424361664997	0.636621979814147\\
1.74865803716848	-1.43171053954564	0.636621979814147\\
1.83924684147543	-1.31330539331972	0.636621979814147\\
1.92140553144839	-1.18987427223195	0.636621979814147\\
1.99476758567383	-1.06230987905837	0.636621979814147\\
2.05908957455551	-0.931531064407848	0.636621979814147\\
2.11425110251375	-0.798462444651827	0.636621979814147\\
2.16025093620176	-0.66401497922819	0.636621979814147\\
2.19719975578803	-0.52906826890771	0.636621979814147\\
2.22531011686021	-0.394455173371548	0.636621979814147\\
2.24488430365428	-0.260949158279188	0.636621979814147\\
2.25630078921572	-0.129254588253301	0.636621979814147\\
2.26	-9.16976072041524e-16	0.636621979814147\\
2.28	0	0.636621979814147\\
2.2762680528371	0.130398434167046	0.636621979814147\\
2.26475053642999	0.263258442865728	0.636621979814147\\
2.24500312674393	0.397945927118198	0.636621979814147\\
2.21664400141448	0.533750288986539	0.636621979814147\\
2.1793682011239	0.66989121798242	0.636621979814147\\
2.13296128926166	0.805528483985027	0.636621979814147\\
2.07731160618874	0.93977470214597	0.636621979814147\\
2.01242039616651	1.07171085143942	0.636621979814147\\
1.93840912022227	1.20040413304817	0.636621979814147\\
1.85552336219645	1.32492756494202	0.636621979814147\\
1.76413288705493	1.44438054432038	0.636621979814147\\
1.66472761223888	1.55790948936369	0.636621979814147\\
1.55790948936369	1.66472761223888	0.636621979814147\\
1.44438054432038	1.76413288705493	0.636621979814147\\
1.32492756494202	1.85552336219645	0.636621979814147\\
1.20040413304817	1.93840912022227	0.636621979814147\\
1.07171085143942	2.01242039616652	0.636621979814147\\
0.939774702145971	2.07731160618874	0.636621979814147\\
0.805528483985028	2.13296128926166	0.636621979814147\\
0.669891217982421	2.1793682011239	0.636621979814147\\
0.533750288986539	2.21664400141448	0.636621979814147\\
0.397945927118199	2.24500312674393	0.636621979814147\\
0.263258442865729	2.26475053642999	0.636621979814147\\
0.130398434167047	2.2762680528371	0.636621979814147\\
5.06261699229071e-16	2.28	0.636621979814147\\
-0.130398434167046	2.2762680528371	0.636621979814147\\
-0.263258442865728	2.26475053642999	0.636621979814147\\
-0.397945927118198	2.24500312674393	0.636621979814147\\
-0.533750288986538	2.21664400141448	0.636621979814147\\
-0.66989121798242	2.1793682011239	0.636621979814147\\
-0.805528483985027	2.13296128926166	0.636621979814147\\
-0.93977470214597	2.07731160618874	0.636621979814147\\
-1.07171085143942	2.01242039616651	0.636621979814147\\
-1.20040413304817	1.93840912022227	0.636621979814147\\
-1.32492756494201	1.85552336219645	0.636621979814147\\
-1.44438054432038	1.76413288705493	0.636621979814147\\
-1.55790948936369	1.66472761223888	0.636621979814147\\
-1.66472761223888	1.55790948936369	0.636621979814147\\
-1.76413288705493	1.44438054432038	0.636621979814147\\
-1.85552336219645	1.32492756494202	0.636621979814147\\
-1.93840912022227	1.20040413304817	0.636621979814147\\
-2.01242039616651	1.07171085143942	0.636621979814147\\
-2.07731160618874	0.939774702145971	0.636621979814147\\
-2.13296128926166	0.805528483985028	0.636621979814147\\
-2.1793682011239	0.669891217982421	0.636621979814147\\
-2.21664400141448	0.533750288986539	0.636621979814147\\
-2.24500312674393	0.397945927118199	0.636621979814147\\
-2.26475053642999	0.263258442865729	0.636621979814147\\
-2.2762680528371	0.130398434167047	0.636621979814147\\
-2.28	6.4587143433187e-16	0.636621979814147\\
-2.2762680528371	-0.130398434167046	0.636621979814147\\
-2.26475053642999	-0.263258442865728	0.636621979814147\\
-2.24500312674393	-0.397945927118198	0.636621979814147\\
-2.21664400141448	-0.533750288986539	0.636621979814147\\
-2.1793682011239	-0.66989121798242	0.636621979814147\\
-2.13296128926166	-0.805528483985028	0.636621979814147\\
-2.07731160618874	-0.939774702145969	0.636621979814147\\
-2.01242039616652	-1.07171085143942	0.636621979814147\\
-1.93840912022227	-1.20040413304817	0.636621979814147\\
-1.85552336219645	-1.32492756494201	0.636621979814147\\
-1.76413288705493	-1.44438054432038	0.636621979814147\\
-1.66472761223888	-1.55790948936369	0.636621979814147\\
-1.55790948936369	-1.66472761223888	0.636621979814147\\
-1.44438054432038	-1.76413288705493	0.636621979814147\\
-1.32492756494202	-1.85552336219645	0.636621979814147\\
-1.20040413304817	-1.93840912022227	0.636621979814147\\
-1.07171085143942	-2.01242039616652	0.636621979814147\\
-0.93977470214597	-2.07731160618874	0.636621979814147\\
-0.805528483985029	-2.13296128926166	0.636621979814147\\
-0.669891217982422	-2.1793682011239	0.636621979814147\\
-0.53375028898654	-2.21664400141448	0.636621979814147\\
-0.397945927118199	-2.24500312674393	0.636621979814147\\
-0.263258442865729	-2.26475053642999	0.636621979814147\\
-0.130398434167047	-2.2762680528371	0.636621979814147\\
-7.85481169434668e-16	-2.28	0.636621979814147\\
0.130398434167046	-2.2762680528371	0.636621979814147\\
0.263258442865728	-2.26475053642999	0.636621979814147\\
0.397945927118198	-2.24500312674393	0.636621979814147\\
0.533750288986538	-2.21664400141448	0.636621979814147\\
0.66989121798242	-2.1793682011239	0.636621979814147\\
0.805528483985028	-2.13296128926166	0.636621979814147\\
0.939774702145969	-2.07731160618874	0.636621979814147\\
1.07171085143942	-2.01242039616652	0.636621979814147\\
1.20040413304817	-1.93840912022227	0.636621979814147\\
1.32492756494201	-1.85552336219645	0.636621979814147\\
1.44438054432038	-1.76413288705493	0.636621979814147\\
1.55790948936369	-1.66472761223888	0.636621979814147\\
1.66472761223888	-1.55790948936369	0.636621979814147\\
1.76413288705493	-1.44438054432038	0.636621979814147\\
1.85552336219645	-1.32492756494202	0.636621979814147\\
1.93840912022227	-1.20040413304817	0.636621979814147\\
2.01242039616652	-1.07171085143942	0.636621979814147\\
2.07731160618874	-0.93977470214597	0.636621979814147\\
2.13296128926166	-0.805528483985029	0.636621979814147\\
2.1793682011239	-0.669891217982422	0.636621979814147\\
2.21664400141448	-0.53375028898654	0.636621979814147\\
2.24500312674393	-0.397945927118199	0.636621979814147\\
2.26475053642999	-0.263258442865729	0.636621979814147\\
2.2762680528371	-0.130398434167047	0.636621979814147\\
2.28	-9.25090904537466e-16	0.636621979814147\\
2.3	0	0.636621979814147\\
2.29623531645847	0.131542280080792	0.636621979814147\\
2.28461676920569	0.26556772745227	0.636621979814147\\
2.26469613662765	0.401436680864849	0.636621979814147\\
2.23608824704092	0.538432309065368	0.636621979814147\\
2.19848546604604	0.675767456736652	0.636621979814147\\
2.15167147600957	0.812594523318229	0.636621979814147\\
2.09553363782198	0.948018339884093	0.636621979814147\\
2.0300732066592	1.08111182382047	0.636621979814147\\
1.95541270899615	1.21093399386438	0.636621979814147\\
1.87179988291747	1.33654973656431	0.636621979814147\\
1.77960773694138	1.45705054909512	0.636621979814147\\
1.67933048603045	1.5715753620774	0.636621979814147\\
1.57157536207741	1.67933048603045	0.636621979814147\\
1.45705054909512	1.77960773694138	0.636621979814147\\
1.33654973656431	1.87179988291747	0.636621979814147\\
1.21093399386438	1.95541270899615	0.636621979814147\\
1.08111182382047	2.0300732066592	0.636621979814147\\
0.948018339884093	2.09553363782198	0.636621979814147\\
0.81259452331823	2.15167147600957	0.636621979814147\\
0.675767456736652	2.19848546604604	0.636621979814147\\
0.538432309065368	2.23608824704092	0.636621979814147\\
0.40143668086485	2.26469613662765	0.636621979814147\\
0.26556772745227	2.28461676920569	0.636621979814147\\
0.131542280080793	2.29623531645847	0.636621979814147\\
5.10702591327572e-16	2.3	0.636621979814147\\
-0.131542280080792	2.29623531645847	0.636621979814147\\
-0.26556772745227	2.28461676920569	0.636621979814147\\
-0.401436680864849	2.26469613662765	0.636621979814147\\
-0.538432309065367	2.23608824704092	0.636621979814147\\
-0.675767456736652	2.19848546604604	0.636621979814147\\
-0.812594523318229	2.15167147600957	0.636621979814147\\
-0.948018339884093	2.09553363782198	0.636621979814147\\
-1.08111182382047	2.0300732066592	0.636621979814147\\
-1.21093399386438	1.95541270899615	0.636621979814147\\
-1.33654973656431	1.87179988291747	0.636621979814147\\
-1.45705054909512	1.77960773694138	0.636621979814147\\
-1.5715753620774	1.67933048603045	0.636621979814147\\
-1.67933048603045	1.57157536207741	0.636621979814147\\
-1.77960773694138	1.45705054909512	0.636621979814147\\
-1.87179988291747	1.33654973656431	0.636621979814147\\
-1.95541270899615	1.21093399386438	0.636621979814147\\
-2.0300732066592	1.08111182382047	0.636621979814147\\
-2.09553363782198	0.948018339884093	0.636621979814147\\
-2.15167147600957	0.81259452331823	0.636621979814147\\
-2.19848546604604	0.675767456736652	0.636621979814147\\
-2.23608824704092	0.538432309065368	0.636621979814147\\
-2.26469613662765	0.40143668086485	0.636621979814147\\
-2.28461676920569	0.265567727452271	0.636621979814147\\
-2.29623531645847	0.131542280080793	0.636621979814147\\
-2.3	6.51536973229518e-16	0.636621979814147\\
-2.29623531645847	-0.131542280080792	0.636621979814147\\
-2.28461676920569	-0.265567727452269	0.636621979814147\\
-2.26469613662765	-0.401436680864849	0.636621979814147\\
-2.23608824704092	-0.538432309065368	0.636621979814147\\
-2.19848546604604	-0.675767456736652	0.636621979814147\\
-2.15167147600957	-0.81259452331823	0.636621979814147\\
-2.09553363782198	-0.948018339884091	0.636621979814147\\
-2.0300732066592	-1.08111182382047	0.636621979814147\\
-1.95541270899615	-1.21093399386438	0.636621979814147\\
-1.87179988291747	-1.33654973656431	0.636621979814147\\
-1.77960773694138	-1.45705054909512	0.636621979814147\\
-1.67933048603045	-1.5715753620774	0.636621979814147\\
-1.57157536207741	-1.67933048603045	0.636621979814147\\
-1.45705054909512	-1.77960773694138	0.636621979814147\\
-1.33654973656431	-1.87179988291747	0.636621979814147\\
-1.21093399386438	-1.95541270899615	0.636621979814147\\
-1.08111182382047	-2.0300732066592	0.636621979814147\\
-0.948018339884093	-2.09553363782198	0.636621979814147\\
-0.812594523318231	-2.15167147600957	0.636621979814147\\
-0.675767456736653	-2.19848546604604	0.636621979814147\\
-0.538432309065369	-2.23608824704092	0.636621979814147\\
-0.40143668086485	-2.26469613662765	0.636621979814147\\
-0.265567727452271	-2.28461676920569	0.636621979814147\\
-0.131542280080793	-2.29623531645847	0.636621979814147\\
-7.92371355131463e-16	-2.3	0.636621979814147\\
0.131542280080792	-2.29623531645847	0.636621979814147\\
0.265567727452269	-2.28461676920569	0.636621979814147\\
0.401436680864849	-2.26469613662765	0.636621979814147\\
0.538432309065368	-2.23608824704092	0.636621979814147\\
0.675767456736652	-2.19848546604604	0.636621979814147\\
0.81259452331823	-2.15167147600957	0.636621979814147\\
0.948018339884091	-2.09553363782198	0.636621979814147\\
1.08111182382047	-2.0300732066592	0.636621979814147\\
1.21093399386438	-1.95541270899615	0.636621979814147\\
1.33654973656431	-1.87179988291747	0.636621979814147\\
1.45705054909512	-1.77960773694138	0.636621979814147\\
1.5715753620774	-1.67933048603045	0.636621979814147\\
1.67933048603045	-1.57157536207741	0.636621979814147\\
1.77960773694138	-1.45705054909512	0.636621979814147\\
1.87179988291747	-1.33654973656431	0.636621979814147\\
1.95541270899615	-1.21093399386438	0.636621979814147\\
2.0300732066592	-1.08111182382047	0.636621979814147\\
2.09553363782198	-0.948018339884093	0.636621979814147\\
2.15167147600957	-0.812594523318231	0.636621979814147\\
2.19848546604604	-0.675767456736653	0.636621979814147\\
2.23608824704092	-0.538432309065369	0.636621979814147\\
2.26469613662765	-0.40143668086485	0.636621979814147\\
2.28461676920569	-0.265567727452271	0.636621979814147\\
2.29623531645847	-0.131542280080793	0.636621979814147\\
2.3	-9.33205737033409e-16	0.636621979814147\\
2.32	0	0.636621979814147\\
2.31620258007985	0.132686125994538	0.636621979814147\\
2.30448300198139	0.267877012038811	0.636621979814147\\
2.28438914651137	0.4049274346115	0.636621979814147\\
2.25553249266736	0.543114329144197	0.636621979814147\\
2.21760273096818	0.681643695490884	0.636621979814147\\
2.17038166275748	0.819660562651431	0.636621979814147\\
2.11375566945521	0.956261977622215	0.636621979814147\\
2.04772601715189	1.09051279620151	0.636621979814147\\
1.97241629777003	1.22146385468059	0.636621979814147\\
1.88807640363849	1.34817190818661	0.636621979814147\\
1.79508258682782	1.46972055386986	0.636621979814147\\
1.69393335982202	1.58524123479112	0.636621979814147\\
1.58524123479112	1.69393335982202	0.636621979814147\\
1.46972055386986	1.79508258682782	0.636621979814147\\
1.34817190818661	1.88807640363849	0.636621979814147\\
1.22146385468059	1.97241629777003	0.636621979814147\\
1.09051279620152	2.04772601715189	0.636621979814147\\
0.956261977622216	2.11375566945521	0.636621979814147\\
0.819660562651432	2.17038166275748	0.636621979814147\\
0.681643695490884	2.21760273096818	0.636621979814147\\
0.543114329144198	2.25553249266736	0.636621979814147\\
0.4049274346115	2.28438914651137	0.636621979814147\\
0.267877012038812	2.30448300198139	0.636621979814147\\
0.132686125994539	2.31620258007985	0.636621979814147\\
5.15143483426073e-16	2.32	0.636621979814147\\
-0.132686125994538	2.31620258007985	0.636621979814147\\
-0.267877012038811	2.30448300198139	0.636621979814147\\
-0.4049274346115	2.28438914651137	0.636621979814147\\
-0.543114329144196	2.25553249266736	0.636621979814147\\
-0.681643695490883	2.21760273096818	0.636621979814147\\
-0.819660562651431	2.17038166275748	0.636621979814147\\
-0.956261977622215	2.11375566945521	0.636621979814147\\
-1.09051279620151	2.04772601715189	0.636621979814147\\
-1.22146385468059	1.97241629777003	0.636621979814147\\
-1.34817190818661	1.88807640363849	0.636621979814147\\
-1.46972055386986	1.79508258682782	0.636621979814147\\
-1.58524123479112	1.69393335982202	0.636621979814147\\
-1.69393335982202	1.58524123479112	0.636621979814147\\
-1.79508258682782	1.46972055386986	0.636621979814147\\
-1.88807640363849	1.34817190818661	0.636621979814147\\
-1.97241629777003	1.22146385468059	0.636621979814147\\
-2.04772601715189	1.09051279620152	0.636621979814147\\
-2.11375566945521	0.956261977622216	0.636621979814147\\
-2.17038166275748	0.819660562651432	0.636621979814147\\
-2.21760273096818	0.681643695490884	0.636621979814147\\
-2.25553249266736	0.543114329144197	0.636621979814147\\
-2.28438914651137	0.404927434611501	0.636621979814147\\
-2.30448300198139	0.267877012038812	0.636621979814147\\
-2.31620258007985	0.132686125994539	0.636621979814147\\
-2.32	6.57202512127166e-16	0.636621979814147\\
-2.31620258007985	-0.132686125994538	0.636621979814147\\
-2.30448300198139	-0.267877012038811	0.636621979814147\\
-2.28438914651137	-0.4049274346115	0.636621979814147\\
-2.25553249266736	-0.543114329144197	0.636621979814147\\
-2.21760273096818	-0.681643695490884	0.636621979814147\\
-2.17038166275748	-0.819660562651432	0.636621979814147\\
-2.11375566945521	-0.956261977622214	0.636621979814147\\
-2.04772601715189	-1.09051279620151	0.636621979814147\\
-1.97241629777003	-1.22146385468059	0.636621979814147\\
-1.88807640363849	-1.34817190818661	0.636621979814147\\
-1.79508258682782	-1.46972055386986	0.636621979814147\\
-1.69393335982202	-1.58524123479112	0.636621979814147\\
-1.58524123479112	-1.69393335982202	0.636621979814147\\
-1.46972055386986	-1.79508258682782	0.636621979814147\\
-1.34817190818661	-1.88807640363849	0.636621979814147\\
-1.22146385468059	-1.97241629777003	0.636621979814147\\
-1.09051279620151	-2.04772601715189	0.636621979814147\\
-0.956261977622215	-2.11375566945521	0.636621979814147\\
-0.819660562651433	-2.17038166275748	0.636621979814147\\
-0.681643695490885	-2.21760273096818	0.636621979814147\\
-0.543114329144198	-2.25553249266736	0.636621979814147\\
-0.404927434611501	-2.28438914651137	0.636621979814147\\
-0.267877012038812	-2.30448300198139	0.636621979814147\\
-0.132686125994539	-2.31620258007985	0.636621979814147\\
-7.99261540828259e-16	-2.32	0.636621979814147\\
0.132686125994538	-2.31620258007985	0.636621979814147\\
0.267877012038811	-2.30448300198139	0.636621979814147\\
0.4049274346115	-2.28438914651137	0.636621979814147\\
0.543114329144197	-2.25553249266736	0.636621979814147\\
0.681643695490884	-2.21760273096818	0.636621979814147\\
0.819660562651431	-2.17038166275748	0.636621979814147\\
0.956261977622214	-2.11375566945521	0.636621979814147\\
1.09051279620151	-2.04772601715189	0.636621979814147\\
1.22146385468059	-1.97241629777003	0.636621979814147\\
1.34817190818661	-1.88807640363849	0.636621979814147\\
1.46972055386986	-1.79508258682782	0.636621979814147\\
1.58524123479112	-1.69393335982202	0.636621979814147\\
1.69393335982202	-1.58524123479112	0.636621979814147\\
1.79508258682782	-1.46972055386986	0.636621979814147\\
1.88807640363849	-1.34817190818661	0.636621979814147\\
1.97241629777003	-1.22146385468059	0.636621979814147\\
2.04772601715189	-1.09051279620151	0.636621979814147\\
2.11375566945521	-0.956261977622215	0.636621979814147\\
2.17038166275748	-0.819660562651433	0.636621979814147\\
2.21760273096818	-0.681643695490885	0.636621979814147\\
2.25553249266736	-0.543114329144198	0.636621979814147\\
2.28438914651137	-0.404927434611501	0.636621979814147\\
2.30448300198139	-0.267877012038812	0.636621979814147\\
2.31620258007985	-0.132686125994539	0.636621979814147\\
2.32	-9.41320569529352e-16	0.636621979814147\\
2.34	0	0.636621979814147\\
2.33616984370123	0.133829971908284	0.636621979814147\\
2.32434923475709	0.270186296625353	0.636621979814147\\
2.30408215639509	0.408418188358151	0.636621979814147\\
2.27497673829381	0.547796349223027	0.636621979814147\\
2.23671999589032	0.687519934245115	0.636621979814147\\
2.18909184950539	0.826726601984633	0.636621979814147\\
2.13197770108845	0.964505615360338	0.636621979814147\\
2.06537882764458	1.09991376858256	0.636621979814147\\
1.98941988654391	1.2319937154968	0.636621979814147\\
1.90435292435952	1.35979407980891	0.636621979814147\\
1.81055743671427	1.4823905586446	0.636621979814147\\
1.70853623361359	1.59890710750484	0.636621979814147\\
1.59890710750484	1.70853623361359	0.636621979814147\\
1.4823905586446	1.81055743671427	0.636621979814147\\
1.35979407980891	1.90435292435952	0.636621979814147\\
1.2319937154968	1.98941988654391	0.636621979814147\\
1.09991376858256	2.06537882764458	0.636621979814147\\
0.964505615360338	2.13197770108845	0.636621979814147\\
0.826726601984634	2.18909184950539	0.636621979814147\\
0.687519934245116	2.23671999589032	0.636621979814147\\
0.547796349223027	2.27497673829381	0.636621979814147\\
0.408418188358151	2.30408215639509	0.636621979814147\\
0.270186296625353	2.32434923475709	0.636621979814147\\
0.133829971908285	2.33616984370123	0.636621979814147\\
5.19584375524573e-16	2.34	0.636621979814147\\
-0.133829971908284	2.33616984370123	0.636621979814147\\
-0.270186296625352	2.32434923475709	0.636621979814147\\
-0.408418188358151	2.30408215639509	0.636621979814147\\
-0.547796349223026	2.27497673829381	0.636621979814147\\
-0.687519934245115	2.23671999589032	0.636621979814147\\
-0.826726601984633	2.18909184950539	0.636621979814147\\
-0.964505615360338	2.13197770108845	0.636621979814147\\
-1.09991376858256	2.06537882764458	0.636621979814147\\
-1.2319937154968	1.98941988654391	0.636621979814147\\
-1.35979407980891	1.90435292435952	0.636621979814147\\
-1.4823905586446	1.81055743671427	0.636621979814147\\
-1.59890710750484	1.70853623361359	0.636621979814147\\
-1.70853623361359	1.59890710750484	0.636621979814147\\
-1.81055743671427	1.4823905586446	0.636621979814147\\
-1.90435292435952	1.35979407980891	0.636621979814147\\
-1.98941988654391	1.2319937154968	0.636621979814147\\
-2.06537882764458	1.09991376858256	0.636621979814147\\
-2.13197770108845	0.964505615360338	0.636621979814147\\
-2.18909184950539	0.826726601984634	0.636621979814147\\
-2.23671999589032	0.687519934245116	0.636621979814147\\
-2.27497673829381	0.547796349223027	0.636621979814147\\
-2.30408215639509	0.408418188358152	0.636621979814147\\
-2.32434923475709	0.270186296625354	0.636621979814147\\
-2.33616984370123	0.133829971908285	0.636621979814147\\
-2.34	6.62868051024814e-16	0.636621979814147\\
-2.33616984370123	-0.133829971908284	0.636621979814147\\
-2.32434923475709	-0.270186296625352	0.636621979814147\\
-2.30408215639509	-0.408418188358151	0.636621979814147\\
-2.27497673829381	-0.547796349223026	0.636621979814147\\
-2.23671999589032	-0.687519934245116	0.636621979814147\\
-2.18909184950539	-0.826726601984634	0.636621979814147\\
-2.13197770108845	-0.964505615360336	0.636621979814147\\
-2.06537882764458	-1.09991376858256	0.636621979814147\\
-1.98941988654391	-1.2319937154968	0.636621979814147\\
-1.90435292435952	-1.35979407980891	0.636621979814147\\
-1.81055743671427	-1.4823905586446	0.636621979814147\\
-1.70853623361359	-1.59890710750484	0.636621979814147\\
-1.59890710750484	-1.70853623361359	0.636621979814147\\
-1.4823905586446	-1.81055743671427	0.636621979814147\\
-1.35979407980891	-1.90435292435952	0.636621979814147\\
-1.2319937154968	-1.98941988654391	0.636621979814147\\
-1.09991376858256	-2.06537882764458	0.636621979814147\\
-0.964505615360338	-2.13197770108845	0.636621979814147\\
-0.826726601984635	-2.18909184950539	0.636621979814147\\
-0.687519934245117	-2.23671999589032	0.636621979814147\\
-0.547796349223028	-2.27497673829381	0.636621979814147\\
-0.408418188358152	-2.30408215639509	0.636621979814147\\
-0.270186296625354	-2.32434923475709	0.636621979814147\\
-0.133829971908285	-2.33616984370123	0.636621979814147\\
-8.06151726525054e-16	-2.34	0.636621979814147\\
0.133829971908284	-2.33616984370123	0.636621979814147\\
0.270186296625352	-2.32434923475709	0.636621979814147\\
0.40841818835815	-2.30408215639509	0.636621979814147\\
0.547796349223026	-2.27497673829381	0.636621979814147\\
0.687519934245116	-2.23671999589032	0.636621979814147\\
0.826726601984633	-2.18909184950539	0.636621979814147\\
0.964505615360336	-2.13197770108845	0.636621979814147\\
1.09991376858256	-2.06537882764458	0.636621979814147\\
1.2319937154968	-1.98941988654391	0.636621979814147\\
1.35979407980891	-1.90435292435952	0.636621979814147\\
1.4823905586446	-1.81055743671427	0.636621979814147\\
1.59890710750484	-1.70853623361359	0.636621979814147\\
1.70853623361359	-1.59890710750484	0.636621979814147\\
1.81055743671427	-1.4823905586446	0.636621979814147\\
1.90435292435952	-1.35979407980891	0.636621979814147\\
1.98941988654391	-1.2319937154968	0.636621979814147\\
2.06537882764458	-1.09991376858256	0.636621979814147\\
2.13197770108845	-0.964505615360338	0.636621979814147\\
2.18909184950539	-0.826726601984635	0.636621979814147\\
2.23671999589032	-0.687519934245117	0.636621979814147\\
2.27497673829381	-0.547796349223028	0.636621979814147\\
2.30408215639509	-0.408418188358152	0.636621979814147\\
2.32434923475709	-0.270186296625354	0.636621979814147\\
2.33616984370123	-0.133829971908285	0.636621979814147\\
2.34	-9.49435402025294e-16	0.636621979814147\\
2.36	0	0.636621979814147\\
2.35613710732261	0.13497381782203	0.636621979814147\\
2.34421546753279	0.272495581211894	0.636621979814147\\
2.32377516627881	0.411908942104802	0.636621979814147\\
2.29442098392025	0.552478369301856	0.636621979814147\\
2.25583726081246	0.693396172999347	0.636621979814147\\
2.2078020362533	0.833792641317835	0.636621979814147\\
2.15019973272168	0.97274925309846	0.636621979814147\\
2.08303163813727	1.10931474096361	0.636621979814147\\
2.00642347531779	1.24252357631301	0.636621979814147\\
1.92062944508054	1.37141625143121	0.636621979814147\\
1.82603228660072	1.49506056341934	0.636621979814147\\
1.72313910740516	1.61257298021855	0.636621979814147\\
1.61257298021856	1.72313910740516	0.636621979814147\\
1.49506056341934	1.82603228660072	0.636621979814147\\
1.37141625143121	1.92062944508054	0.636621979814147\\
1.24252357631301	2.00642347531779	0.636621979814147\\
1.10931474096361	2.08303163813727	0.636621979814147\\
0.972749253098461	2.15019973272168	0.636621979814147\\
0.833792641317836	2.2078020362533	0.636621979814147\\
0.693396172999348	2.25583726081246	0.636621979814147\\
0.552478369301856	2.29442098392025	0.636621979814147\\
0.411908942104802	2.32377516627881	0.636621979814147\\
0.272495581211895	2.34421546753279	0.636621979814147\\
0.134973817822031	2.35613710732261	0.636621979814147\\
5.24025267623074e-16	2.36	0.636621979814147\\
-0.13497381782203	2.35613710732261	0.636621979814147\\
-0.272495581211894	2.34421546753279	0.636621979814147\\
-0.411908942104802	2.32377516627881	0.636621979814147\\
-0.552478369301855	2.29442098392025	0.636621979814147\\
-0.693396172999347	2.25583726081246	0.636621979814147\\
-0.833792641317835	2.2078020362533	0.636621979814147\\
-0.97274925309846	2.15019973272168	0.636621979814147\\
-1.10931474096361	2.08303163813727	0.636621979814147\\
-1.24252357631301	2.00642347531779	0.636621979814147\\
-1.37141625143121	1.92062944508054	0.636621979814147\\
-1.49506056341934	1.82603228660072	0.636621979814147\\
-1.61257298021855	1.72313910740516	0.636621979814147\\
-1.72313910740516	1.61257298021856	0.636621979814147\\
-1.82603228660072	1.49506056341934	0.636621979814147\\
-1.92062944508054	1.37141625143121	0.636621979814147\\
-2.00642347531779	1.24252357631302	0.636621979814147\\
-2.08303163813727	1.10931474096361	0.636621979814147\\
-2.15019973272168	0.972749253098461	0.636621979814147\\
-2.2078020362533	0.833792641317836	0.636621979814147\\
-2.25583726081246	0.693396172999348	0.636621979814147\\
-2.29442098392025	0.552478369301856	0.636621979814147\\
-2.32377516627881	0.411908942104803	0.636621979814147\\
-2.34421546753279	0.272495581211895	0.636621979814147\\
-2.35613710732261	0.134973817822031	0.636621979814147\\
-2.36	6.68533589922462e-16	0.636621979814147\\
-2.35613710732261	-0.13497381782203	0.636621979814147\\
-2.34421546753279	-0.272495581211894	0.636621979814147\\
-2.32377516627881	-0.411908942104801	0.636621979814147\\
-2.29442098392025	-0.552478369301856	0.636621979814147\\
-2.25583726081246	-0.693396172999348	0.636621979814147\\
-2.2078020362533	-0.833792641317836	0.636621979814147\\
-2.15019973272168	-0.972749253098459	0.636621979814147\\
-2.08303163813727	-1.10931474096361	0.636621979814147\\
-2.00642347531779	-1.24252357631301	0.636621979814147\\
-1.92062944508054	-1.37141625143121	0.636621979814147\\
-1.82603228660072	-1.49506056341934	0.636621979814147\\
-1.72313910740516	-1.61257298021855	0.636621979814147\\
-1.61257298021856	-1.72313910740516	0.636621979814147\\
-1.49506056341934	-1.82603228660072	0.636621979814147\\
-1.37141625143121	-1.92062944508054	0.636621979814147\\
-1.24252357631301	-2.00642347531779	0.636621979814147\\
-1.10931474096361	-2.08303163813727	0.636621979814147\\
-0.97274925309846	-2.15019973272168	0.636621979814147\\
-0.833792641317837	-2.2078020362533	0.636621979814147\\
-0.693396172999349	-2.25583726081246	0.636621979814147\\
-0.552478369301857	-2.29442098392025	0.636621979814147\\
-0.411908942104803	-2.32377516627881	0.636621979814147\\
-0.272495581211895	-2.34421546753279	0.636621979814147\\
-0.134973817822031	-2.35613710732261	0.636621979814147\\
-8.13041912221849e-16	-2.36	0.636621979814147\\
0.13497381782203	-2.35613710732261	0.636621979814147\\
0.272495581211894	-2.34421546753279	0.636621979814147\\
0.411908942104801	-2.32377516627881	0.636621979814147\\
0.552478369301855	-2.29442098392025	0.636621979814147\\
0.693396172999348	-2.25583726081246	0.636621979814147\\
0.833792641317835	-2.2078020362533	0.636621979814147\\
0.972749253098459	-2.15019973272168	0.636621979814147\\
1.10931474096361	-2.08303163813727	0.636621979814147\\
1.24252357631301	-2.00642347531779	0.636621979814147\\
1.37141625143121	-1.92062944508054	0.636621979814147\\
1.49506056341934	-1.82603228660072	0.636621979814147\\
1.61257298021855	-1.72313910740516	0.636621979814147\\
1.72313910740516	-1.61257298021856	0.636621979814147\\
1.82603228660072	-1.49506056341934	0.636621979814147\\
1.92062944508054	-1.37141625143121	0.636621979814147\\
2.00642347531779	-1.24252357631301	0.636621979814147\\
2.08303163813727	-1.10931474096361	0.636621979814147\\
2.15019973272168	-0.97274925309846	0.636621979814147\\
2.2078020362533	-0.833792641317837	0.636621979814147\\
2.25583726081246	-0.693396172999349	0.636621979814147\\
2.29442098392025	-0.552478369301857	0.636621979814147\\
2.32377516627881	-0.411908942104803	0.636621979814147\\
2.34421546753279	-0.272495581211895	0.636621979814147\\
2.35613710732261	-0.134973817822031	0.636621979814147\\
2.36	-9.57550234521237e-16	0.636621979814147\\
2.38	0	0.636621979814147\\
2.37610437094399	0.136117663735776	0.636621979814147\\
2.36408170030849	0.274804865798436	0.636621979814147\\
2.34346817616253	0.415399695851453	0.636621979814147\\
2.31386522954669	0.557160389380685	0.636621979814147\\
2.2749545257346	0.699272411753579	0.636621979814147\\
2.22651222300121	0.840858680651037	0.636621979814147\\
2.16842176435492	0.980992890836583	0.636621979814147\\
2.10068444862996	1.11871571334466	0.636621979814147\\
2.02342706409167	1.25305343712923	0.636621979814147\\
1.93690596580156	1.38303842305351	0.636621979814147\\
1.84150713648716	1.50773056819408	0.636621979814147\\
1.73774198119673	1.62623885293227	0.636621979814147\\
1.62623885293227	1.73774198119673	0.636621979814147\\
1.50773056819408	1.84150713648716	0.636621979814147\\
1.38303842305351	1.93690596580156	0.636621979814147\\
1.25305343712923	2.02342706409167	0.636621979814147\\
1.11871571334466	2.10068444862996	0.636621979814147\\
0.980992890836583	2.16842176435492	0.636621979814147\\
0.840858680651038	2.22651222300121	0.636621979814147\\
0.699272411753579	2.2749545257346	0.636621979814147\\
0.557160389380686	2.31386522954669	0.636621979814147\\
0.415399695851453	2.34346817616253	0.636621979814147\\
0.274804865798436	2.36408170030849	0.636621979814147\\
0.136117663735777	2.37610437094399	0.636621979814147\\
5.28466159721574e-16	2.38	0.636621979814147\\
-0.136117663735776	2.37610437094399	0.636621979814147\\
-0.274804865798435	2.36408170030849	0.636621979814147\\
-0.415399695851452	2.34346817616253	0.636621979814147\\
-0.557160389380684	2.31386522954669	0.636621979814147\\
-0.699272411753579	2.2749545257346	0.636621979814147\\
-0.840858680651037	2.22651222300121	0.636621979814147\\
-0.980992890836583	2.16842176435492	0.636621979814147\\
-1.11871571334466	2.10068444862996	0.636621979814147\\
-1.25305343712923	2.02342706409167	0.636621979814147\\
-1.38303842305351	1.93690596580156	0.636621979814147\\
-1.50773056819408	1.84150713648716	0.636621979814147\\
-1.62623885293227	1.73774198119673	0.636621979814147\\
-1.73774198119673	1.62623885293227	0.636621979814147\\
-1.84150713648716	1.50773056819408	0.636621979814147\\
-1.93690596580156	1.38303842305351	0.636621979814147\\
-2.02342706409167	1.25305343712923	0.636621979814147\\
-2.10068444862996	1.11871571334466	0.636621979814147\\
-2.16842176435492	0.980992890836583	0.636621979814147\\
-2.22651222300121	0.840858680651038	0.636621979814147\\
-2.2749545257346	0.699272411753579	0.636621979814147\\
-2.31386522954669	0.557160389380685	0.636621979814147\\
-2.34346817616253	0.415399695851453	0.636621979814147\\
-2.36408170030849	0.274804865798436	0.636621979814147\\
-2.37610437094399	0.136117663735777	0.636621979814147\\
-2.38	6.7419912882011e-16	0.636621979814147\\
-2.37610437094399	-0.136117663735776	0.636621979814147\\
-2.36408170030849	-0.274804865798435	0.636621979814147\\
-2.34346817616253	-0.415399695851452	0.636621979814147\\
-2.31386522954669	-0.557160389380685	0.636621979814147\\
-2.2749545257346	-0.699272411753579	0.636621979814147\\
-2.22651222300121	-0.840858680651038	0.636621979814147\\
-2.16842176435492	-0.980992890836582	0.636621979814147\\
-2.10068444862996	-1.11871571334466	0.636621979814147\\
-2.02342706409167	-1.25305343712923	0.636621979814147\\
-1.93690596580156	-1.38303842305351	0.636621979814147\\
-1.84150713648716	-1.50773056819408	0.636621979814147\\
-1.73774198119673	-1.62623885293227	0.636621979814147\\
-1.62623885293227	-1.73774198119673	0.636621979814147\\
-1.50773056819408	-1.84150713648716	0.636621979814147\\
-1.38303842305351	-1.93690596580156	0.636621979814147\\
-1.25305343712923	-2.02342706409167	0.636621979814147\\
-1.11871571334466	-2.10068444862996	0.636621979814147\\
-0.980992890836583	-2.16842176435492	0.636621979814147\\
-0.840858680651039	-2.22651222300121	0.636621979814147\\
-0.69927241175358	-2.2749545257346	0.636621979814147\\
-0.557160389380686	-2.31386522954669	0.636621979814147\\
-0.415399695851454	-2.34346817616253	0.636621979814147\\
-0.274804865798437	-2.36408170030849	0.636621979814147\\
-0.136117663735777	-2.37610437094399	0.636621979814147\\
-8.19932097918645e-16	-2.38	0.636621979814147\\
0.136117663735776	-2.37610437094399	0.636621979814147\\
0.274804865798435	-2.36408170030849	0.636621979814147\\
0.415399695851452	-2.34346817616253	0.636621979814147\\
0.557160389380685	-2.31386522954669	0.636621979814147\\
0.699272411753579	-2.2749545257346	0.636621979814147\\
0.840858680651038	-2.22651222300121	0.636621979814147\\
0.980992890836582	-2.16842176435492	0.636621979814147\\
1.11871571334466	-2.10068444862996	0.636621979814147\\
1.25305343712923	-2.02342706409167	0.636621979814147\\
1.38303842305351	-1.93690596580156	0.636621979814147\\
1.50773056819408	-1.84150713648716	0.636621979814147\\
1.62623885293227	-1.73774198119673	0.636621979814147\\
1.73774198119673	-1.62623885293227	0.636621979814147\\
1.84150713648716	-1.50773056819408	0.636621979814147\\
1.93690596580156	-1.38303842305351	0.636621979814147\\
2.02342706409167	-1.25305343712923	0.636621979814147\\
2.10068444862996	-1.11871571334466	0.636621979814147\\
2.16842176435492	-0.980992890836583	0.636621979814147\\
2.22651222300121	-0.840858680651039	0.636621979814147\\
2.2749545257346	-0.699272411753581	0.636621979814147\\
2.31386522954669	-0.557160389380686	0.636621979814147\\
2.34346817616253	-0.415399695851454	0.636621979814147\\
2.36408170030849	-0.274804865798437	0.636621979814147\\
2.37610437094399	-0.136117663735777	0.636621979814147\\
2.38	-9.6566506701718e-16	0.636621979814147\\
2.4	0	0.636621979814147\\
2.39607163456536	0.137261509649522	0.636621979814147\\
2.3839479330842	0.277114150384977	0.636621979814147\\
2.36316118604625	0.418890449598103	0.636621979814147\\
2.33330947517313	0.561842409459514	0.636621979814147\\
2.29407179065674	0.705148650507811	0.636621979814147\\
2.24522240974912	0.847924719984239	0.636621979814147\\
2.18664379598815	0.989236528574706	0.636621979814147\\
2.11833725912265	1.12811668572571	0.636621979814147\\
2.04043065286555	1.26358329794544	0.636621979814147\\
1.95318248652258	1.39466059467581	0.636621979814147\\
1.85698198637361	1.52040057296882	0.636621979814147\\
1.7523448549883	1.63990472564599	0.636621979814147\\
1.63990472564599	1.7523448549883	0.636621979814147\\
1.52040057296882	1.85698198637361	0.636621979814147\\
1.39466059467581	1.95318248652258	0.636621979814147\\
1.26358329794544	2.04043065286555	0.636621979814147\\
1.12811668572571	2.11833725912265	0.636621979814147\\
0.989236528574706	2.18664379598815	0.636621979814147\\
0.84792471998424	2.24522240974912	0.636621979814147\\
0.705148650507811	2.29407179065674	0.636621979814147\\
0.561842409459515	2.33330947517313	0.636621979814147\\
0.418890449598104	2.36316118604625	0.636621979814147\\
0.277114150384978	2.3839479330842	0.636621979814147\\
0.137261509649523	2.39607163456536	0.636621979814147\\
5.32907051820075e-16	2.4	0.636621979814147\\
-0.137261509649522	2.39607163456536	0.636621979814147\\
-0.277114150384977	2.3839479330842	0.636621979814147\\
-0.418890449598103	2.36316118604625	0.636621979814147\\
-0.561842409459514	2.33330947517313	0.636621979814147\\
-0.70514865050781	2.29407179065674	0.636621979814147\\
-0.847924719984239	2.24522240974912	0.636621979814147\\
-0.989236528574705	2.18664379598815	0.636621979814147\\
-1.12811668572571	2.11833725912265	0.636621979814147\\
-1.26358329794544	2.04043065286555	0.636621979814147\\
-1.3946605946758	1.95318248652258	0.636621979814147\\
-1.52040057296882	1.85698198637361	0.636621979814147\\
-1.63990472564599	1.7523448549883	0.636621979814147\\
-1.7523448549883	1.63990472564599	0.636621979814147\\
-1.85698198637361	1.52040057296882	0.636621979814147\\
-1.95318248652258	1.39466059467581	0.636621979814147\\
-2.04043065286555	1.26358329794544	0.636621979814147\\
-2.11833725912265	1.12811668572571	0.636621979814147\\
-2.18664379598815	0.989236528574706	0.636621979814147\\
-2.24522240974912	0.84792471998424	0.636621979814147\\
-2.29407179065674	0.705148650507811	0.636621979814147\\
-2.33330947517313	0.561842409459515	0.636621979814147\\
-2.36316118604625	0.418890449598104	0.636621979814147\\
-2.3839479330842	0.277114150384978	0.636621979814147\\
-2.39607163456536	0.137261509649523	0.636621979814147\\
-2.4	6.79864667717757e-16	0.636621979814147\\
-2.39607163456536	-0.137261509649522	0.636621979814147\\
-2.3839479330842	-0.277114150384977	0.636621979814147\\
-2.36316118604625	-0.418890449598103	0.636621979814147\\
-2.33330947517313	-0.561842409459514	0.636621979814147\\
-2.29407179065674	-0.705148650507811	0.636621979814147\\
-2.24522240974912	-0.84792471998424	0.636621979814147\\
-2.18664379598815	-0.989236528574704	0.636621979814147\\
-2.11833725912265	-1.1281166857257	0.636621979814147\\
-2.04043065286555	-1.26358329794544	0.636621979814147\\
-1.95318248652258	-1.3946605946758	0.636621979814147\\
-1.85698198637361	-1.52040057296882	0.636621979814147\\
-1.7523448549883	-1.63990472564599	0.636621979814147\\
-1.63990472564599	-1.7523448549883	0.636621979814147\\
-1.52040057296882	-1.85698198637361	0.636621979814147\\
-1.39466059467581	-1.95318248652258	0.636621979814147\\
-1.26358329794544	-2.04043065286555	0.636621979814147\\
-1.12811668572571	-2.11833725912265	0.636621979814147\\
-0.989236528574705	-2.18664379598815	0.636621979814147\\
-0.847924719984241	-2.24522240974912	0.636621979814147\\
-0.705148650507812	-2.29407179065674	0.636621979814147\\
-0.561842409459516	-2.33330947517313	0.636621979814147\\
-0.418890449598104	-2.36316118604625	0.636621979814147\\
-0.277114150384978	-2.3839479330842	0.636621979814147\\
-0.137261509649523	-2.39607163456536	0.636621979814147\\
-8.2682228361544e-16	-2.4	0.636621979814147\\
0.137261509649522	-2.39607163456536	0.636621979814147\\
0.277114150384977	-2.3839479330842	0.636621979814147\\
0.418890449598103	-2.36316118604625	0.636621979814147\\
0.561842409459514	-2.33330947517313	0.636621979814147\\
0.705148650507811	-2.29407179065674	0.636621979814147\\
0.84792471998424	-2.24522240974912	0.636621979814147\\
0.989236528574704	-2.18664379598815	0.636621979814147\\
1.1281166857257	-2.11833725912265	0.636621979814147\\
1.26358329794544	-2.04043065286555	0.636621979814147\\
1.3946605946758	-1.95318248652258	0.636621979814147\\
1.52040057296882	-1.85698198637361	0.636621979814147\\
1.63990472564599	-1.7523448549883	0.636621979814147\\
1.7523448549883	-1.63990472564599	0.636621979814147\\
1.85698198637361	-1.52040057296882	0.636621979814147\\
1.95318248652258	-1.39466059467581	0.636621979814147\\
2.04043065286555	-1.26358329794544	0.636621979814147\\
2.11833725912265	-1.12811668572571	0.636621979814147\\
2.18664379598815	-0.989236528574705	0.636621979814147\\
2.24522240974912	-0.847924719984241	0.636621979814147\\
2.29407179065674	-0.705148650507812	0.636621979814147\\
2.33330947517313	-0.561842409459516	0.636621979814147\\
2.36316118604625	-0.418890449598104	0.636621979814147\\
2.3839479330842	-0.277114150384978	0.636621979814147\\
2.39607163456536	-0.137261509649523	0.636621979814147\\
2.4	-9.73779899513122e-16	0.636621979814147\\
2.42	0	0.636621979814147\\
2.41603889818674	0.138405355563268	0.636621979814147\\
2.4038141658599	0.279423434971519	0.636621979814147\\
2.38285419592996	0.422381203344754	0.636621979814147\\
2.35275372079958	0.566524429538344	0.636621979814147\\
2.31318905557888	0.711024889262042	0.636621979814147\\
2.26393259649703	0.854990759317441	0.636621979814147\\
2.20486582762138	0.997480166312828	0.636621979814147\\
2.13599006961534	1.13751765810675	0.636621979814147\\
2.05743424163943	1.27411315876165	0.636621979814147\\
1.9694590072436	1.4062827662981	0.636621979814147\\
1.87245683626006	1.53307057774356	0.636621979814147\\
1.76694772877987	1.6535705983597	0.636621979814147\\
1.6535705983597	1.76694772877987	0.636621979814147\\
1.53307057774356	1.87245683626006	0.636621979814147\\
1.4062827662981	1.9694590072436	0.636621979814147\\
1.27411315876165	2.05743424163943	0.636621979814147\\
1.13751765810675	2.13599006961534	0.636621979814147\\
0.997480166312829	2.20486582762138	0.636621979814147\\
0.854990759317442	2.26393259649703	0.636621979814147\\
0.711024889262043	2.31318905557888	0.636621979814147\\
0.566524429538344	2.35275372079958	0.636621979814147\\
0.422381203344755	2.38285419592996	0.636621979814147\\
0.279423434971519	2.4038141658599	0.636621979814147\\
0.138405355563269	2.41603889818674	0.636621979814147\\
5.37347943918576e-16	2.42	0.636621979814147\\
-0.138405355563268	2.41603889818674	0.636621979814147\\
-0.279423434971518	2.4038141658599	0.636621979814147\\
-0.422381203344754	2.38285419592996	0.636621979814147\\
-0.566524429538343	2.35275372079958	0.636621979814147\\
-0.711024889262042	2.31318905557888	0.636621979814147\\
-0.854990759317441	2.26393259649703	0.636621979814147\\
-0.997480166312828	2.20486582762138	0.636621979814147\\
-1.13751765810675	2.13599006961534	0.636621979814147\\
-1.27411315876165	2.05743424163943	0.636621979814147\\
-1.4062827662981	1.9694590072436	0.636621979814147\\
-1.53307057774356	1.87245683626006	0.636621979814147\\
-1.6535705983597	1.76694772877987	0.636621979814147\\
-1.76694772877987	1.6535705983597	0.636621979814147\\
-1.87245683626006	1.53307057774356	0.636621979814147\\
-1.9694590072436	1.4062827662981	0.636621979814147\\
-2.05743424163943	1.27411315876165	0.636621979814147\\
-2.13599006961534	1.13751765810675	0.636621979814147\\
-2.20486582762138	0.997480166312829	0.636621979814147\\
-2.26393259649703	0.854990759317442	0.636621979814147\\
-2.31318905557888	0.711024889262043	0.636621979814147\\
-2.35275372079958	0.566524429538344	0.636621979814147\\
-2.38285419592996	0.422381203344755	0.636621979814147\\
-2.4038141658599	0.27942343497152	0.636621979814147\\
-2.41603889818674	0.138405355563269	0.636621979814147\\
-2.42	6.85530206615406e-16	0.636621979814147\\
-2.41603889818674	-0.138405355563268	0.636621979814147\\
-2.4038141658599	-0.279423434971518	0.636621979814147\\
-2.38285419592996	-0.422381203344754	0.636621979814147\\
-2.35275372079958	-0.566524429538343	0.636621979814147\\
-2.31318905557888	-0.711024889262043	0.636621979814147\\
-2.26393259649703	-0.854990759317442	0.636621979814147\\
-2.20486582762138	-0.997480166312826	0.636621979814147\\
-2.13599006961534	-1.13751765810675	0.636621979814147\\
-2.05743424163943	-1.27411315876165	0.636621979814147\\
-1.9694590072436	-1.4062827662981	0.636621979814147\\
-1.87245683626006	-1.53307057774356	0.636621979814147\\
-1.76694772877987	-1.6535705983597	0.636621979814147\\
-1.6535705983597	-1.76694772877987	0.636621979814147\\
-1.53307057774356	-1.87245683626006	0.636621979814147\\
-1.4062827662981	-1.9694590072436	0.636621979814147\\
-1.27411315876165	-2.05743424163943	0.636621979814147\\
-1.13751765810675	-2.13599006961534	0.636621979814147\\
-0.997480166312828	-2.20486582762138	0.636621979814147\\
-0.854990759317443	-2.26393259649703	0.636621979814147\\
-0.711024889262044	-2.31318905557888	0.636621979814147\\
-0.566524429538345	-2.35275372079958	0.636621979814147\\
-0.422381203344755	-2.38285419592996	0.636621979814147\\
-0.27942343497152	-2.4038141658599	0.636621979814147\\
-0.138405355563269	-2.41603889818674	0.636621979814147\\
-8.33712469312235e-16	-2.42	0.636621979814147\\
0.138405355563268	-2.41603889818674	0.636621979814147\\
0.279423434971518	-2.4038141658599	0.636621979814147\\
0.422381203344754	-2.38285419592996	0.636621979814147\\
0.566524429538343	-2.35275372079958	0.636621979814147\\
0.711024889262043	-2.31318905557888	0.636621979814147\\
0.854990759317442	-2.26393259649703	0.636621979814147\\
0.997480166312826	-2.20486582762138	0.636621979814147\\
1.13751765810675	-2.13599006961534	0.636621979814147\\
1.27411315876165	-2.05743424163943	0.636621979814147\\
1.4062827662981	-1.9694590072436	0.636621979814147\\
1.53307057774356	-1.87245683626006	0.636621979814147\\
1.6535705983597	-1.76694772877987	0.636621979814147\\
1.76694772877987	-1.6535705983597	0.636621979814147\\
1.87245683626006	-1.53307057774356	0.636621979814147\\
1.9694590072436	-1.4062827662981	0.636621979814147\\
2.05743424163943	-1.27411315876165	0.636621979814147\\
2.13599006961534	-1.13751765810675	0.636621979814147\\
2.20486582762138	-0.997480166312828	0.636621979814147\\
2.26393259649703	-0.854990759317443	0.636621979814147\\
2.31318905557888	-0.711024889262044	0.636621979814147\\
2.35275372079958	-0.566524429538345	0.636621979814147\\
2.38285419592996	-0.422381203344755	0.636621979814147\\
2.4038141658599	-0.27942343497152	0.636621979814147\\
2.41603889818674	-0.138405355563269	0.636621979814147\\
2.42	-9.81894732009065e-16	0.636621979814147\\
2.44	0	0.636621979814147\\
2.43600616180812	0.139549201477014	0.636621979814147\\
2.4236803986356	0.28173271955806	0.636621979814147\\
2.40254720581368	0.425871957091405	0.636621979814147\\
2.37219796642602	0.571206449617173	0.636621979814147\\
2.33230632050102	0.716901128016274	0.636621979814147\\
2.28264278324494	0.862056798650643	0.636621979814147\\
2.22308785925462	1.00572380405095	0.636621979814147\\
2.15364288010802	1.1469186304878	0.636621979814147\\
2.07443783041331	1.28464301957786	0.636621979814147\\
1.98573552796462	1.4179049379204	0.636621979814147\\
1.8879316861465	1.5457405825183	0.636621979814147\\
1.78155060257144	1.66723647107342	0.636621979814147\\
1.66723647107342	1.78155060257144	0.636621979814147\\
1.5457405825183	1.8879316861465	0.636621979814147\\
1.4179049379204	1.98573552796462	0.636621979814147\\
1.28464301957786	2.07443783041331	0.636621979814147\\
1.1469186304878	2.15364288010803	0.636621979814147\\
1.00572380405095	2.22308785925462	0.636621979814147\\
0.862056798650644	2.28264278324494	0.636621979814147\\
0.716901128016275	2.33230632050102	0.636621979814147\\
0.571206449617174	2.37219796642602	0.636621979814147\\
0.425871957091406	2.40254720581368	0.636621979814147\\
0.28173271955806	2.4236803986356	0.636621979814147\\
0.139549201477015	2.43600616180812	0.636621979814147\\
5.41788836017076e-16	2.44	0.636621979814147\\
-0.139549201477014	2.43600616180812	0.636621979814147\\
-0.28173271955806	2.4236803986356	0.636621979814147\\
-0.425871957091405	2.40254720581368	0.636621979814147\\
-0.571206449617172	2.37219796642602	0.636621979814147\\
-0.716901128016274	2.33230632050102	0.636621979814147\\
-0.862056798650643	2.28264278324494	0.636621979814147\\
-1.00572380405095	2.22308785925462	0.636621979814147\\
-1.1469186304878	2.15364288010802	0.636621979814147\\
-1.28464301957786	2.07443783041331	0.636621979814147\\
-1.4179049379204	1.98573552796462	0.636621979814147\\
-1.5457405825183	1.8879316861465	0.636621979814147\\
-1.66723647107342	1.78155060257144	0.636621979814147\\
-1.78155060257144	1.66723647107342	0.636621979814147\\
-1.8879316861465	1.5457405825183	0.636621979814147\\
-1.98573552796462	1.4179049379204	0.636621979814147\\
-2.07443783041331	1.28464301957786	0.636621979814147\\
-2.15364288010802	1.1469186304878	0.636621979814147\\
-2.22308785925462	1.00572380405095	0.636621979814147\\
-2.28264278324494	0.862056798650644	0.636621979814147\\
-2.33230632050102	0.716901128016275	0.636621979814147\\
-2.37219796642602	0.571206449617173	0.636621979814147\\
-2.40254720581368	0.425871957091406	0.636621979814147\\
-2.4236803986356	0.281732719558061	0.636621979814147\\
-2.43600616180812	0.139549201477015	0.636621979814147\\
-2.44	6.91195745513053e-16	0.636621979814147\\
-2.43600616180812	-0.139549201477014	0.636621979814147\\
-2.4236803986356	-0.28173271955806	0.636621979814147\\
-2.40254720581368	-0.425871957091405	0.636621979814147\\
-2.37219796642602	-0.571206449617173	0.636621979814147\\
-2.33230632050102	-0.716901128016274	0.636621979814147\\
-2.28264278324494	-0.862056798650644	0.636621979814147\\
-2.22308785925462	-1.00572380405095	0.636621979814147\\
-2.15364288010803	-1.1469186304878	0.636621979814147\\
-2.07443783041331	-1.28464301957786	0.636621979814147\\
-1.98573552796462	-1.4179049379204	0.636621979814147\\
-1.8879316861465	-1.5457405825183	0.636621979814147\\
-1.78155060257144	-1.66723647107342	0.636621979814147\\
-1.66723647107342	-1.78155060257144	0.636621979814147\\
-1.5457405825183	-1.8879316861465	0.636621979814147\\
-1.4179049379204	-1.98573552796462	0.636621979814147\\
-1.28464301957786	-2.07443783041331	0.636621979814147\\
-1.1469186304878	-2.15364288010803	0.636621979814147\\
-1.00572380405095	-2.22308785925462	0.636621979814147\\
-0.862056798650645	-2.28264278324494	0.636621979814147\\
-0.716901128016276	-2.33230632050102	0.636621979814147\\
-0.571206449617174	-2.37219796642602	0.636621979814147\\
-0.425871957091406	-2.40254720581368	0.636621979814147\\
-0.281732719558061	-2.4236803986356	0.636621979814147\\
-0.139549201477015	-2.43600616180812	0.636621979814147\\
-8.40602655009031e-16	-2.44	0.636621979814147\\
0.139549201477014	-2.43600616180812	0.636621979814147\\
0.28173271955806	-2.4236803986356	0.636621979814147\\
0.425871957091405	-2.40254720581368	0.636621979814147\\
0.571206449617173	-2.37219796642602	0.636621979814147\\
0.716901128016274	-2.33230632050102	0.636621979814147\\
0.862056798650643	-2.28264278324494	0.636621979814147\\
1.00572380405095	-2.22308785925462	0.636621979814147\\
1.1469186304878	-2.15364288010803	0.636621979814147\\
1.28464301957786	-2.07443783041331	0.636621979814147\\
1.4179049379204	-1.98573552796462	0.636621979814147\\
1.5457405825183	-1.8879316861465	0.636621979814147\\
1.66723647107342	-1.78155060257144	0.636621979814147\\
1.78155060257144	-1.66723647107342	0.636621979814147\\
1.8879316861465	-1.5457405825183	0.636621979814147\\
1.98573552796462	-1.4179049379204	0.636621979814147\\
2.07443783041331	-1.28464301957786	0.636621979814147\\
2.15364288010803	-1.1469186304878	0.636621979814147\\
2.22308785925462	-1.00572380405095	0.636621979814147\\
2.28264278324494	-0.862056798650645	0.636621979814147\\
2.33230632050102	-0.716901128016276	0.636621979814147\\
2.37219796642602	-0.571206449617174	0.636621979814147\\
2.40254720581368	-0.425871957091406	0.636621979814147\\
2.4236803986356	-0.281732719558061	0.636621979814147\\
2.43600616180812	-0.139549201477016	0.636621979814147\\
2.44	-9.90009564505008e-16	0.636621979814147\\
2.46	0	0.636621979814147\\
2.4559734254295	0.14069304739076	0.636621979814147\\
2.4435466314113	0.284042004144602	0.636621979814147\\
2.4222402156974	0.429362710838056	0.636621979814147\\
2.39164221205246	0.575888469696002	0.636621979814147\\
2.35142358542316	0.722777366770506	0.636621979814147\\
2.30135296999285	0.869122837983845	0.636621979814147\\
2.24130989088785	1.01396744178907	0.636621979814147\\
2.17129569060071	1.15631960286885	0.636621979814147\\
2.09144141918719	1.29517288039407	0.636621979814147\\
2.00201204868564	1.4295271095427	0.636621979814147\\
1.90340653603295	1.55841058729304	0.636621979814147\\
1.796153476363	1.68090234378714	0.636621979814147\\
1.68090234378714	1.796153476363	0.636621979814147\\
1.55841058729304	1.90340653603295	0.636621979814147\\
1.4295271095427	2.00201204868564	0.636621979814147\\
1.29517288039407	2.09144141918719	0.636621979814147\\
1.15631960286885	2.17129569060071	0.636621979814147\\
1.01396744178907	2.24130989088785	0.636621979814147\\
0.869122837983846	2.30135296999285	0.636621979814147\\
0.722777366770506	2.35142358542316	0.636621979814147\\
0.575888469696003	2.39164221205246	0.636621979814147\\
0.429362710838056	2.4222402156974	0.636621979814147\\
0.284042004144602	2.4435466314113	0.636621979814147\\
0.140693047390761	2.4559734254295	0.636621979814147\\
5.46229728115577e-16	2.46	0.636621979814147\\
-0.14069304739076	2.4559734254295	0.636621979814147\\
-0.284042004144601	2.4435466314113	0.636621979814147\\
-0.429362710838056	2.4222402156974	0.636621979814147\\
-0.575888469696001	2.39164221205246	0.636621979814147\\
-0.722777366770506	2.35142358542316	0.636621979814147\\
-0.869122837983845	2.30135296999285	0.636621979814147\\
-1.01396744178907	2.24130989088785	0.636621979814147\\
-1.15631960286885	2.17129569060071	0.636621979814147\\
-1.29517288039407	2.09144141918719	0.636621979814147\\
-1.4295271095427	2.00201204868564	0.636621979814147\\
-1.55841058729304	1.90340653603295	0.636621979814147\\
-1.68090234378714	1.79615347636301	0.636621979814147\\
-1.796153476363	1.68090234378714	0.636621979814147\\
-1.90340653603295	1.55841058729304	0.636621979814147\\
-2.00201204868564	1.4295271095427	0.636621979814147\\
-2.09144141918719	1.29517288039408	0.636621979814147\\
-2.17129569060071	1.15631960286885	0.636621979814147\\
-2.24130989088785	1.01396744178907	0.636621979814147\\
-2.30135296999285	0.869122837983846	0.636621979814147\\
-2.35142358542316	0.722777366770506	0.636621979814147\\
-2.39164221205246	0.575888469696002	0.636621979814147\\
-2.4222402156974	0.429362710838057	0.636621979814147\\
-2.4435466314113	0.284042004144602	0.636621979814147\\
-2.4559734254295	0.140693047390761	0.636621979814147\\
-2.46	6.96861284410701e-16	0.636621979814147\\
-2.4559734254295	-0.14069304739076	0.636621979814147\\
-2.4435466314113	-0.284042004144601	0.636621979814147\\
-2.4222402156974	-0.429362710838056	0.636621979814147\\
-2.39164221205246	-0.575888469696002	0.636621979814147\\
-2.35142358542316	-0.722777366770506	0.636621979814147\\
-2.30135296999285	-0.869122837983846	0.636621979814147\\
-2.24130989088785	-1.01396744178907	0.636621979814147\\
-2.17129569060071	-1.15631960286885	0.636621979814147\\
-2.09144141918719	-1.29517288039407	0.636621979814147\\
-2.00201204868564	-1.4295271095427	0.636621979814147\\
-1.90340653603295	-1.55841058729304	0.636621979814147\\
-1.79615347636301	-1.68090234378714	0.636621979814147\\
-1.68090234378714	-1.796153476363	0.636621979814147\\
-1.55841058729304	-1.90340653603295	0.636621979814147\\
-1.4295271095427	-2.00201204868564	0.636621979814147\\
-1.29517288039407	-2.09144141918719	0.636621979814147\\
-1.15631960286885	-2.17129569060071	0.636621979814147\\
-1.01396744178907	-2.24130989088785	0.636621979814147\\
-0.869122837983847	-2.30135296999285	0.636621979814147\\
-0.722777366770508	-2.35142358542316	0.636621979814147\\
-0.575888469696003	-2.39164221205246	0.636621979814147\\
-0.429362710838057	-2.4222402156974	0.636621979814147\\
-0.284042004144603	-2.4435466314113	0.636621979814147\\
-0.140693047390761	-2.4559734254295	0.636621979814147\\
-8.47492840705826e-16	-2.46	0.636621979814147\\
0.14069304739076	-2.4559734254295	0.636621979814147\\
0.284042004144601	-2.4435466314113	0.636621979814147\\
0.429362710838056	-2.4222402156974	0.636621979814147\\
0.575888469696002	-2.39164221205246	0.636621979814147\\
0.722777366770506	-2.35142358542316	0.636621979814147\\
0.869122837983845	-2.30135296999285	0.636621979814147\\
1.01396744178907	-2.24130989088785	0.636621979814147\\
1.15631960286885	-2.17129569060071	0.636621979814147\\
1.29517288039407	-2.09144141918719	0.636621979814147\\
1.4295271095427	-2.00201204868565	0.636621979814147\\
1.55841058729304	-1.90340653603295	0.636621979814147\\
1.68090234378714	-1.79615347636301	0.636621979814147\\
1.796153476363	-1.68090234378714	0.636621979814147\\
1.90340653603295	-1.55841058729304	0.636621979814147\\
2.00201204868564	-1.4295271095427	0.636621979814147\\
2.09144141918719	-1.29517288039407	0.636621979814147\\
2.17129569060071	-1.15631960286885	0.636621979814147\\
2.24130989088785	-1.01396744178907	0.636621979814147\\
2.30135296999285	-0.869122837983847	0.636621979814147\\
2.35142358542316	-0.722777366770508	0.636621979814147\\
2.39164221205246	-0.575888469696003	0.636621979814147\\
2.4222402156974	-0.429362710838057	0.636621979814147\\
2.4435466314113	-0.284042004144603	0.636621979814147\\
2.4559734254295	-0.140693047390762	0.636621979814147\\
2.46	-9.98124397000951e-16	0.636621979814147\\
2.48	0	0.636621979814147\\
2.47594068905088	0.141836893304506	0.636621979814147\\
2.463412864187	0.286351288731143	0.636621979814147\\
2.44193322558112	0.432853464584707	0.636621979814147\\
2.41108645767891	0.580570489774832	0.636621979814147\\
2.3705408503453	0.728653605524738	0.636621979814147\\
2.32006315674076	0.876188877317047	0.636621979814147\\
2.25953192252109	1.0222110795272	0.636621979814147\\
2.1889485010934	1.1657205752499	0.636621979814147\\
2.10844500796107	1.30570274121029	0.636621979814147\\
2.01828856940667	1.441149281165	0.636621979814147\\
1.9188813859194	1.57108059206778	0.636621979814147\\
1.81075635015457	1.69456821650085	0.636621979814147\\
1.69456821650085	1.81075635015457	0.636621979814147\\
1.57108059206778	1.9188813859194	0.636621979814147\\
1.441149281165	2.01828856940667	0.636621979814147\\
1.30570274121029	2.10844500796107	0.636621979814147\\
1.1657205752499	2.1889485010934	0.636621979814147\\
1.0222110795272	2.25953192252109	0.636621979814147\\
0.876188877317048	2.32006315674076	0.636621979814147\\
0.728653605524738	2.3705408503453	0.636621979814147\\
0.580570489774832	2.41108645767891	0.636621979814147\\
0.432853464584707	2.44193322558112	0.636621979814147\\
0.286351288731143	2.463412864187	0.636621979814147\\
0.141836893304507	2.47594068905088	0.636621979814147\\
5.50670620214078e-16	2.48	0.636621979814147\\
-0.141836893304506	2.47594068905088	0.636621979814147\\
-0.286351288731143	2.463412864187	0.636621979814147\\
-0.432853464584707	2.44193322558112	0.636621979814147\\
-0.580570489774831	2.41108645767891	0.636621979814147\\
-0.728653605524737	2.3705408503453	0.636621979814147\\
-0.876188877317047	2.32006315674076	0.636621979814147\\
-1.0222110795272	2.25953192252109	0.636621979814147\\
-1.1657205752499	2.1889485010934	0.636621979814147\\
-1.30570274121029	2.10844500796107	0.636621979814147\\
-1.441149281165	2.01828856940667	0.636621979814147\\
-1.57108059206778	1.9188813859194	0.636621979814147\\
-1.69456821650085	1.81075635015457	0.636621979814147\\
-1.81075635015457	1.69456821650085	0.636621979814147\\
-1.9188813859194	1.57108059206778	0.636621979814147\\
-2.01828856940667	1.441149281165	0.636621979814147\\
-2.10844500796107	1.30570274121029	0.636621979814147\\
-2.1889485010934	1.1657205752499	0.636621979814147\\
-2.25953192252109	1.0222110795272	0.636621979814147\\
-2.32006315674076	0.876188877317048	0.636621979814147\\
-2.3705408503453	0.728653605524738	0.636621979814147\\
-2.41108645767891	0.580570489774832	0.636621979814147\\
-2.44193322558112	0.432853464584708	0.636621979814147\\
-2.463412864187	0.286351288731144	0.636621979814147\\
-2.47594068905088	0.141836893304507	0.636621979814147\\
-2.48	7.0252682330835e-16	0.636621979814147\\
-2.47594068905088	-0.141836893304506	0.636621979814147\\
-2.463412864187	-0.286351288731143	0.636621979814147\\
-2.44193322558112	-0.432853464584707	0.636621979814147\\
-2.41108645767891	-0.580570489774831	0.636621979814147\\
-2.3705408503453	-0.728653605524738	0.636621979814147\\
-2.32006315674076	-0.876188877317048	0.636621979814147\\
-2.25953192252109	-1.02221107952719	0.636621979814147\\
-2.1889485010934	-1.16572057524989	0.636621979814147\\
-2.10844500796107	-1.30570274121029	0.636621979814147\\
-2.01828856940667	-1.441149281165	0.636621979814147\\
-1.9188813859194	-1.57108059206778	0.636621979814147\\
-1.81075635015457	-1.69456821650085	0.636621979814147\\
-1.69456821650085	-1.81075635015457	0.636621979814147\\
-1.57108059206778	-1.9188813859194	0.636621979814147\\
-1.441149281165	-2.01828856940667	0.636621979814147\\
-1.30570274121029	-2.10844500796107	0.636621979814147\\
-1.1657205752499	-2.1889485010934	0.636621979814147\\
-1.0222110795272	-2.25953192252109	0.636621979814147\\
-0.876188877317049	-2.32006315674076	0.636621979814147\\
-0.728653605524739	-2.3705408503453	0.636621979814147\\
-0.580570489774833	-2.41108645767891	0.636621979814147\\
-0.432853464584708	-2.44193322558112	0.636621979814147\\
-0.286351288731144	-2.463412864187	0.636621979814147\\
-0.141836893304507	-2.47594068905088	0.636621979814147\\
-8.54383026402621e-16	-2.48	0.636621979814147\\
0.141836893304506	-2.47594068905088	0.636621979814147\\
0.286351288731143	-2.463412864187	0.636621979814147\\
0.432853464584706	-2.44193322558112	0.636621979814147\\
0.580570489774831	-2.41108645767891	0.636621979814147\\
0.728653605524738	-2.3705408503453	0.636621979814147\\
0.876188877317048	-2.32006315674076	0.636621979814147\\
1.02221107952719	-2.25953192252109	0.636621979814147\\
1.16572057524989	-2.1889485010934	0.636621979814147\\
1.30570274121029	-2.10844500796107	0.636621979814147\\
1.441149281165	-2.01828856940667	0.636621979814147\\
1.57108059206778	-1.9188813859194	0.636621979814147\\
1.69456821650085	-1.81075635015457	0.636621979814147\\
1.81075635015457	-1.69456821650085	0.636621979814147\\
1.9188813859194	-1.57108059206778	0.636621979814147\\
2.01828856940667	-1.441149281165	0.636621979814147\\
2.10844500796107	-1.30570274121029	0.636621979814147\\
2.1889485010934	-1.1657205752499	0.636621979814147\\
2.25953192252109	-1.0222110795272	0.636621979814147\\
2.32006315674076	-0.876188877317049	0.636621979814147\\
2.3705408503453	-0.728653605524739	0.636621979814147\\
2.41108645767891	-0.580570489774833	0.636621979814147\\
2.44193322558112	-0.432853464584708	0.636621979814147\\
2.463412864187	-0.286351288731144	0.636621979814147\\
2.47594068905088	-0.141836893304508	0.636621979814147\\
2.48	-1.00623922949689e-15	0.636621979814147\\
2.5	0	0.636621979814147\\
2.49590795267225	0.142980739218252	0.636621979814147\\
2.4832790969627	0.288660573317684	0.636621979814147\\
2.46162623546484	0.436344218331358	0.636621979814147\\
2.43053070330535	0.585252509853661	0.636621979814147\\
2.38965811526744	0.73452984427897	0.636621979814147\\
2.33877334348867	0.883254916650249	0.636621979814147\\
2.27775395415432	1.03045471726532	0.636621979814147\\
2.20660131158609	1.17512154763094	0.636621979814147\\
2.12544859673495	1.3162326020265	0.636621979814147\\
2.03456509012769	1.4527714527873	0.636621979814147\\
1.93435623580584	1.58375059684252	0.636621979814147\\
1.82535922394614	1.70823408921457	0.636621979814147\\
1.70823408921457	1.82535922394614	0.636621979814147\\
1.58375059684252	1.93435623580584	0.636621979814147\\
1.4527714527873	2.03456509012769	0.636621979814147\\
1.3162326020265	2.12544859673495	0.636621979814147\\
1.17512154763094	2.20660131158609	0.636621979814147\\
1.03045471726532	2.27775395415432	0.636621979814147\\
0.88325491665025	2.33877334348867	0.636621979814147\\
0.73452984427897	2.38965811526744	0.636621979814147\\
0.585252509853662	2.43053070330535	0.636621979814147\\
0.436344218331358	2.46162623546484	0.636621979814147\\
0.288660573317685	2.4832790969627	0.636621979814147\\
0.142980739218253	2.49590795267225	0.636621979814147\\
5.55111512312578e-16	2.5	0.636621979814147\\
-0.142980739218252	2.49590795267225	0.636621979814147\\
-0.288660573317684	2.4832790969627	0.636621979814147\\
-0.436344218331358	2.46162623546484	0.636621979814147\\
-0.58525250985366	2.43053070330535	0.636621979814147\\
-0.734529844278969	2.38965811526744	0.636621979814147\\
-0.883254916650249	2.33877334348867	0.636621979814147\\
-1.03045471726532	2.27775395415432	0.636621979814147\\
-1.17512154763094	2.20660131158609	0.636621979814147\\
-1.3162326020265	2.12544859673495	0.636621979814147\\
-1.4527714527873	2.03456509012769	0.636621979814147\\
-1.58375059684252	1.93435623580584	0.636621979814147\\
-1.70823408921457	1.82535922394614	0.636621979814147\\
-1.82535922394614	1.70823408921457	0.636621979814147\\
-1.93435623580584	1.58375059684252	0.636621979814147\\
-2.03456509012769	1.4527714527873	0.636621979814147\\
-2.12544859673495	1.3162326020265	0.636621979814147\\
-2.20660131158609	1.17512154763094	0.636621979814147\\
-2.27775395415432	1.03045471726532	0.636621979814147\\
-2.33877334348867	0.88325491665025	0.636621979814147\\
-2.38965811526744	0.73452984427897	0.636621979814147\\
-2.43053070330535	0.585252509853661	0.636621979814147\\
-2.46162623546484	0.436344218331359	0.636621979814147\\
-2.4832790969627	0.288660573317685	0.636621979814147\\
-2.49590795267225	0.142980739218253	0.636621979814147\\
-2.5	7.08192362205997e-16	0.636621979814147\\
-2.49590795267225	-0.142980739218252	0.636621979814147\\
-2.4832790969627	-0.288660573317684	0.636621979814147\\
-2.46162623546484	-0.436344218331357	0.636621979814147\\
-2.43053070330535	-0.585252509853661	0.636621979814147\\
-2.38965811526744	-0.73452984427897	0.636621979814147\\
-2.33877334348867	-0.88325491665025	0.636621979814147\\
-2.27775395415432	-1.03045471726532	0.636621979814147\\
-2.20660131158609	-1.17512154763094	0.636621979814147\\
-2.12544859673495	-1.3162326020265	0.636621979814147\\
-2.03456509012769	-1.4527714527873	0.636621979814147\\
-1.93435623580584	-1.58375059684252	0.636621979814147\\
-1.82535922394614	-1.70823408921457	0.636621979814147\\
-1.70823408921457	-1.82535922394614	0.636621979814147\\
-1.58375059684252	-1.93435623580584	0.636621979814147\\
-1.4527714527873	-2.03456509012769	0.636621979814147\\
-1.3162326020265	-2.12544859673495	0.636621979814147\\
-1.17512154763094	-2.20660131158609	0.636621979814147\\
-1.03045471726532	-2.27775395415432	0.636621979814147\\
-0.883254916650251	-2.33877334348867	0.636621979814147\\
-0.734529844278971	-2.38965811526744	0.636621979814147\\
-0.585252509853662	-2.43053070330535	0.636621979814147\\
-0.436344218331359	-2.46162623546484	0.636621979814147\\
-0.288660573317686	-2.4832790969627	0.636621979814147\\
-0.142980739218253	-2.49590795267225	0.636621979814147\\
-8.61273212099417e-16	-2.5	0.636621979814147\\
0.142980739218252	-2.49590795267225	0.636621979814147\\
0.288660573317684	-2.4832790969627	0.636621979814147\\
0.436344218331357	-2.46162623546484	0.636621979814147\\
0.585252509853661	-2.43053070330535	0.636621979814147\\
0.73452984427897	-2.38965811526744	0.636621979814147\\
0.88325491665025	-2.33877334348867	0.636621979814147\\
1.03045471726532	-2.27775395415432	0.636621979814147\\
1.17512154763094	-2.20660131158609	0.636621979814147\\
1.3162326020265	-2.12544859673495	0.636621979814147\\
1.4527714527873	-2.03456509012769	0.636621979814147\\
1.58375059684252	-1.93435623580584	0.636621979814147\\
1.70823408921457	-1.82535922394614	0.636621979814147\\
1.82535922394614	-1.70823408921457	0.636621979814147\\
1.93435623580584	-1.58375059684252	0.636621979814147\\
2.03456509012769	-1.4527714527873	0.636621979814147\\
2.12544859673495	-1.3162326020265	0.636621979814147\\
2.20660131158609	-1.17512154763094	0.636621979814147\\
2.27775395415432	-1.03045471726532	0.636621979814147\\
2.33877334348867	-0.883254916650251	0.636621979814147\\
2.38965811526744	-0.734529844278971	0.636621979814147\\
2.43053070330535	-0.585252509853662	0.636621979814147\\
2.46162623546484	-0.436344218331359	0.636621979814147\\
2.4832790969627	-0.288660573317686	0.636621979814147\\
2.49590795267225	-0.142980739218254	0.636621979814147\\
2.5	-1.01435406199284e-15	0.636621979814147\\
2.52	0	0.636621979814147\\
2.51587521629363	0.144124585131998	0.636621979814147\\
2.50314532973841	0.290969857904226	0.636621979814147\\
2.48131924534856	0.439834972078008	0.636621979814147\\
2.44997494893179	0.58993452993249	0.636621979814147\\
2.40877538018958	0.740406083033201	0.636621979814147\\
2.35748353023658	0.890320955983451	0.636621979814147\\
2.29597598578756	1.03869835500344	0.636621979814147\\
2.22425412207878	1.18452252001199	0.636621979814147\\
2.14245218550883	1.32676246284271	0.636621979814147\\
2.05084161084871	1.4643936244096	0.636621979814147\\
1.94983108569229	1.59642060161726	0.636621979814147\\
1.83996209773771	1.72189996192829	0.636621979814147\\
1.72189996192829	1.83996209773771	0.636621979814147\\
1.59642060161726	1.94983108569229	0.636621979814147\\
1.4643936244096	2.05084161084871	0.636621979814147\\
1.32676246284271	2.14245218550883	0.636621979814147\\
1.18452252001199	2.22425412207878	0.636621979814147\\
1.03869835500344	2.29597598578756	0.636621979814147\\
0.890320955983452	2.35748353023658	0.636621979814147\\
0.740406083033202	2.40877538018958	0.636621979814147\\
0.589934529932491	2.44997494893179	0.636621979814147\\
0.439834972078009	2.48131924534856	0.636621979814147\\
0.290969857904226	2.50314532973841	0.636621979814147\\
0.144124585131999	2.51587521629363	0.636621979814147\\
5.59552404411079e-16	2.52	0.636621979814147\\
-0.144124585131998	2.51587521629363	0.636621979814147\\
-0.290969857904226	2.50314532973841	0.636621979814147\\
-0.439834972078008	2.48131924534856	0.636621979814147\\
-0.589934529932489	2.44997494893179	0.636621979814147\\
-0.740406083033201	2.40877538018958	0.636621979814147\\
-0.890320955983451	2.35748353023658	0.636621979814147\\
-1.03869835500344	2.29597598578756	0.636621979814147\\
-1.18452252001199	2.22425412207878	0.636621979814147\\
-1.32676246284271	2.14245218550883	0.636621979814147\\
-1.4643936244096	2.05084161084871	0.636621979814147\\
-1.59642060161726	1.94983108569229	0.636621979814147\\
-1.72189996192829	1.83996209773771	0.636621979814147\\
-1.83996209773771	1.72189996192829	0.636621979814147\\
-1.94983108569229	1.59642060161726	0.636621979814147\\
-2.05084161084871	1.4643936244096	0.636621979814147\\
-2.14245218550883	1.32676246284271	0.636621979814147\\
-2.22425412207878	1.18452252001199	0.636621979814147\\
-2.29597598578756	1.03869835500344	0.636621979814147\\
-2.35748353023658	0.890320955983452	0.636621979814147\\
-2.40877538018958	0.740406083033202	0.636621979814147\\
-2.44997494893179	0.58993452993249	0.636621979814147\\
-2.48131924534856	0.43983497207801	0.636621979814147\\
-2.50314532973841	0.290969857904227	0.636621979814147\\
-2.51587521629363	0.144124585131999	0.636621979814147\\
-2.52	7.13857901103645e-16	0.636621979814147\\
-2.51587521629363	-0.144124585131998	0.636621979814147\\
-2.50314532973841	-0.290969857904226	0.636621979814147\\
-2.48131924534856	-0.439834972078008	0.636621979814147\\
-2.44997494893179	-0.58993452993249	0.636621979814147\\
-2.40877538018958	-0.740406083033202	0.636621979814147\\
-2.35748353023658	-0.890320955983452	0.636621979814147\\
-2.29597598578756	-1.03869835500344	0.636621979814147\\
-2.22425412207878	-1.18452252001199	0.636621979814147\\
-2.14245218550883	-1.32676246284271	0.636621979814147\\
-2.05084161084871	-1.4643936244096	0.636621979814147\\
-1.94983108569229	-1.59642060161726	0.636621979814147\\
-1.83996209773771	-1.72189996192829	0.636621979814147\\
-1.72189996192829	-1.83996209773771	0.636621979814147\\
-1.59642060161726	-1.94983108569229	0.636621979814147\\
-1.4643936244096	-2.05084161084871	0.636621979814147\\
-1.32676246284271	-2.14245218550883	0.636621979814147\\
-1.18452252001199	-2.22425412207878	0.636621979814147\\
-1.03869835500344	-2.29597598578756	0.636621979814147\\
-0.890320955983453	-2.35748353023658	0.636621979814147\\
-0.740406083033203	-2.40877538018957	0.636621979814147\\
-0.589934529932491	-2.44997494893179	0.636621979814147\\
-0.43983497207801	-2.48131924534856	0.636621979814147\\
-0.290969857904227	-2.50314532973841	0.636621979814147\\
-0.144124585131999	-2.51587521629363	0.636621979814147\\
-8.68163397796212e-16	-2.52	0.636621979814147\\
0.144124585131998	-2.51587521629363	0.636621979814147\\
0.290969857904226	-2.50314532973841	0.636621979814147\\
0.439834972078008	-2.48131924534856	0.636621979814147\\
0.58993452993249	-2.44997494893179	0.636621979814147\\
0.740406083033202	-2.40877538018958	0.636621979814147\\
0.890320955983452	-2.35748353023658	0.636621979814147\\
1.03869835500344	-2.29597598578756	0.636621979814147\\
1.18452252001199	-2.22425412207878	0.636621979814147\\
1.32676246284271	-2.14245218550883	0.636621979814147\\
1.4643936244096	-2.05084161084871	0.636621979814147\\
1.59642060161726	-1.94983108569229	0.636621979814147\\
1.72189996192829	-1.83996209773771	0.636621979814147\\
1.83996209773771	-1.72189996192829	0.636621979814147\\
1.94983108569229	-1.59642060161726	0.636621979814147\\
2.05084161084871	-1.4643936244096	0.636621979814147\\
2.14245218550883	-1.32676246284271	0.636621979814147\\
2.22425412207878	-1.18452252001199	0.636621979814147\\
2.29597598578756	-1.03869835500344	0.636621979814147\\
2.35748353023658	-0.890320955983453	0.636621979814147\\
2.40877538018957	-0.740406083033203	0.636621979814147\\
2.44997494893179	-0.589934529932491	0.636621979814147\\
2.48131924534856	-0.43983497207801	0.636621979814147\\
2.50314532973841	-0.290969857904227	0.636621979814147\\
2.51587521629363	-0.144124585132	0.636621979814147\\
2.52	-1.02246889448878e-15	0.636621979814147\\
2.54	0	0.636621979814147\\
2.53584247991501	0.145268431045745	0.636621979814147\\
2.52301156251411	0.293279142490767	0.636621979814147\\
2.50101225523228	0.443325725824659	0.636621979814147\\
2.46941919455823	0.594616550011319	0.636621979814147\\
2.42789264511171	0.746282321787433	0.636621979814147\\
2.37619371698449	0.897386995316653	0.636621979814147\\
2.31419801742079	1.04694199274156	0.636621979814147\\
2.24190693257147	1.19392349239304	0.636621979814147\\
2.15945577428271	1.33729232365892	0.636621979814147\\
2.06711813156973	1.47601579603189	0.636621979814147\\
1.96530593557874	1.609090606392	0.636621979814147\\
1.85456497152928	1.735565834642	0.636621979814147\\
1.735565834642	1.85456497152928	0.636621979814147\\
1.609090606392	1.96530593557874	0.636621979814147\\
1.47601579603189	2.06711813156973	0.636621979814147\\
1.33729232365892	2.15945577428271	0.636621979814147\\
1.19392349239304	2.24190693257147	0.636621979814147\\
1.04694199274156	2.31419801742079	0.636621979814147\\
0.897386995316654	2.37619371698449	0.636621979814147\\
0.746282321787434	2.42789264511171	0.636621979814147\\
0.59461655001132	2.46941919455823	0.636621979814147\\
0.44332572582466	2.50101225523228	0.636621979814147\\
0.293279142490768	2.52301156251411	0.636621979814147\\
0.145268431045745	2.53584247991501	0.636621979814147\\
5.6399329650958e-16	2.54	0.636621979814147\\
-0.145268431045744	2.53584247991501	0.636621979814147\\
-0.293279142490767	2.52301156251411	0.636621979814147\\
-0.443325725824659	2.50101225523228	0.636621979814147\\
-0.594616550011319	2.46941919455823	0.636621979814147\\
-0.746282321787433	2.42789264511171	0.636621979814147\\
-0.897386995316653	2.37619371698449	0.636621979814147\\
-1.04694199274156	2.31419801742079	0.636621979814147\\
-1.19392349239304	2.24190693257147	0.636621979814147\\
-1.33729232365892	2.15945577428271	0.636621979814147\\
-1.47601579603189	2.06711813156973	0.636621979814147\\
-1.609090606392	1.96530593557874	0.636621979814147\\
-1.735565834642	1.85456497152928	0.636621979814147\\
-1.85456497152928	1.735565834642	0.636621979814147\\
-1.96530593557874	1.609090606392	0.636621979814147\\
-2.06711813156973	1.47601579603189	0.636621979814147\\
-2.15945577428271	1.33729232365892	0.636621979814147\\
-2.24190693257147	1.19392349239304	0.636621979814147\\
-2.31419801742079	1.04694199274156	0.636621979814147\\
-2.37619371698449	0.897386995316654	0.636621979814147\\
-2.42789264511171	0.746282321787434	0.636621979814147\\
-2.46941919455823	0.59461655001132	0.636621979814147\\
-2.50101225523228	0.44332572582466	0.636621979814147\\
-2.52301156251411	0.293279142490768	0.636621979814147\\
-2.53584247991501	0.145268431045745	0.636621979814147\\
-2.54	7.19523440001293e-16	0.636621979814147\\
-2.53584247991501	-0.145268431045744	0.636621979814147\\
-2.52301156251411	-0.293279142490767	0.636621979814147\\
-2.50101225523228	-0.443325725824659	0.636621979814147\\
-2.46941919455823	-0.594616550011319	0.636621979814147\\
-2.42789264511171	-0.746282321787433	0.636621979814147\\
-2.37619371698449	-0.897386995316654	0.636621979814147\\
-2.31419801742079	-1.04694199274156	0.636621979814147\\
-2.24190693257147	-1.19392349239304	0.636621979814147\\
-2.15945577428271	-1.33729232365892	0.636621979814147\\
-2.06711813156973	-1.47601579603189	0.636621979814147\\
-1.96530593557874	-1.609090606392	0.636621979814147\\
-1.85456497152928	-1.735565834642	0.636621979814147\\
-1.735565834642	-1.85456497152928	0.636621979814147\\
-1.609090606392	-1.96530593557874	0.636621979814147\\
-1.47601579603189	-2.06711813156973	0.636621979814147\\
-1.33729232365892	-2.15945577428271	0.636621979814147\\
-1.19392349239304	-2.24190693257147	0.636621979814147\\
-1.04694199274156	-2.31419801742079	0.636621979814147\\
-0.897386995316655	-2.37619371698449	0.636621979814147\\
-0.746282321787435	-2.42789264511171	0.636621979814147\\
-0.594616550011321	-2.46941919455823	0.636621979814147\\
-0.443325725824661	-2.50101225523228	0.636621979814147\\
-0.293279142490769	-2.52301156251411	0.636621979814147\\
-0.145268431045746	-2.53584247991501	0.636621979814147\\
-8.75053583493007e-16	-2.54	0.636621979814147\\
0.145268431045744	-2.53584247991501	0.636621979814147\\
0.293279142490767	-2.52301156251411	0.636621979814147\\
0.443325725824659	-2.50101225523228	0.636621979814147\\
0.594616550011319	-2.46941919455823	0.636621979814147\\
0.746282321787433	-2.42789264511171	0.636621979814147\\
0.897386995316653	-2.37619371698449	0.636621979814147\\
1.04694199274156	-2.31419801742079	0.636621979814147\\
1.19392349239304	-2.24190693257147	0.636621979814147\\
1.33729232365892	-2.15945577428271	0.636621979814147\\
1.47601579603189	-2.06711813156973	0.636621979814147\\
1.609090606392	-1.96530593557874	0.636621979814147\\
1.735565834642	-1.85456497152928	0.636621979814147\\
1.85456497152928	-1.735565834642	0.636621979814147\\
1.96530593557874	-1.609090606392	0.636621979814147\\
2.06711813156973	-1.47601579603189	0.636621979814147\\
2.15945577428271	-1.33729232365892	0.636621979814147\\
2.24190693257147	-1.19392349239304	0.636621979814147\\
2.31419801742079	-1.04694199274156	0.636621979814147\\
2.37619371698449	-0.897386995316655	0.636621979814147\\
2.42789264511171	-0.746282321787435	0.636621979814147\\
2.46941919455823	-0.594616550011321	0.636621979814147\\
2.50101225523228	-0.443325725824661	0.636621979814147\\
2.52301156251411	-0.293279142490769	0.636621979814147\\
2.53584247991501	-0.145268431045746	0.636621979814147\\
2.54	-1.03058372698472e-15	0.636621979814147\\
2.56	0	0.636621979814147\\
2.55580974353639	0.146412276959491	0.636621979814147\\
2.54287779528981	0.295588427077309	0.636621979814147\\
2.520705265116	0.44681647957131	0.636621979814147\\
2.48886344018468	0.599298570090149	0.636621979814147\\
2.44700991003385	0.752158560541665	0.636621979814147\\
2.3949039037324	0.904453034649855	0.636621979814147\\
2.33242004905403	1.05518563047969	0.636621979814147\\
2.25955974306416	1.20332446477409	0.636621979814147\\
2.17645936305659	1.34782218447513	0.636621979814147\\
2.08339465229075	1.48763796765419	0.636621979814147\\
1.98078078546518	1.62176061116674	0.636621979814147\\
1.86916784532085	1.74923170735572	0.636621979814147\\
1.74923170735572	1.86916784532085	0.636621979814147\\
1.62176061116674	1.98078078546518	0.636621979814147\\
1.48763796765419	2.08339465229075	0.636621979814147\\
1.34782218447513	2.17645936305659	0.636621979814147\\
1.20332446477409	2.25955974306416	0.636621979814147\\
1.05518563047969	2.33242004905403	0.636621979814147\\
0.904453034649856	2.3949039037324	0.636621979814147\\
0.752158560541665	2.44700991003385	0.636621979814147\\
0.599298570090149	2.48886344018468	0.636621979814147\\
0.446816479571311	2.520705265116	0.636621979814147\\
0.295588427077309	2.54287779528981	0.636621979814147\\
0.146412276959491	2.55580974353639	0.636621979814147\\
5.6843418860808e-16	2.56	0.636621979814147\\
-0.14641227695949	2.55580974353639	0.636621979814147\\
-0.295588427077309	2.54287779528981	0.636621979814147\\
-0.44681647957131	2.520705265116	0.636621979814147\\
-0.599298570090148	2.48886344018468	0.636621979814147\\
-0.752158560541664	2.44700991003385	0.636621979814147\\
-0.904453034649855	2.3949039037324	0.636621979814147\\
-1.05518563047969	2.33242004905403	0.636621979814147\\
-1.20332446477409	2.25955974306416	0.636621979814147\\
-1.34782218447513	2.17645936305659	0.636621979814147\\
-1.48763796765419	2.08339465229075	0.636621979814147\\
-1.62176061116674	1.98078078546518	0.636621979814147\\
-1.74923170735572	1.86916784532085	0.636621979814147\\
-1.86916784532085	1.74923170735572	0.636621979814147\\
-1.98078078546518	1.62176061116674	0.636621979814147\\
-2.08339465229075	1.48763796765419	0.636621979814147\\
-2.17645936305659	1.34782218447514	0.636621979814147\\
-2.25955974306416	1.20332446477409	0.636621979814147\\
-2.33242004905403	1.05518563047969	0.636621979814147\\
-2.3949039037324	0.904453034649856	0.636621979814147\\
-2.44700991003385	0.752158560541665	0.636621979814147\\
-2.48886344018468	0.599298570090149	0.636621979814147\\
-2.520705265116	0.446816479571311	0.636621979814147\\
-2.54287779528981	0.29558842707731	0.636621979814147\\
-2.55580974353639	0.146412276959491	0.636621979814147\\
-2.56	7.25188978898941e-16	0.636621979814147\\
-2.55580974353639	-0.14641227695949	0.636621979814147\\
-2.54287779528981	-0.295588427077309	0.636621979814147\\
-2.520705265116	-0.44681647957131	0.636621979814147\\
-2.48886344018468	-0.599298570090149	0.636621979814147\\
-2.44700991003385	-0.752158560541665	0.636621979814147\\
-2.3949039037324	-0.904453034649856	0.636621979814147\\
-2.33242004905403	-1.05518563047968	0.636621979814147\\
-2.25955974306416	-1.20332446477408	0.636621979814147\\
-2.17645936305659	-1.34782218447513	0.636621979814147\\
-2.08339465229075	-1.48763796765419	0.636621979814147\\
-1.98078078546518	-1.62176061116674	0.636621979814147\\
-1.86916784532085	-1.74923170735572	0.636621979814147\\
-1.74923170735572	-1.86916784532085	0.636621979814147\\
-1.62176061116674	-1.98078078546518	0.636621979814147\\
-1.48763796765419	-2.08339465229075	0.636621979814147\\
-1.34782218447513	-2.17645936305659	0.636621979814147\\
-1.20332446477409	-2.25955974306416	0.636621979814147\\
-1.05518563047969	-2.33242004905403	0.636621979814147\\
-0.904453034649857	-2.39490390373239	0.636621979814147\\
-0.752158560541666	-2.44700991003385	0.636621979814147\\
-0.59929857009015	-2.48886344018468	0.636621979814147\\
-0.446816479571311	-2.520705265116	0.636621979814147\\
-0.29558842707731	-2.54287779528981	0.636621979814147\\
-0.146412276959492	-2.55580974353639	0.636621979814147\\
-8.81943769189803e-16	-2.56	0.636621979814147\\
0.14641227695949	-2.55580974353639	0.636621979814147\\
0.295588427077308	-2.54287779528981	0.636621979814147\\
0.44681647957131	-2.520705265116	0.636621979814147\\
0.599298570090148	-2.48886344018468	0.636621979814147\\
0.752158560541665	-2.44700991003385	0.636621979814147\\
0.904453034649855	-2.3949039037324	0.636621979814147\\
1.05518563047968	-2.33242004905403	0.636621979814147\\
1.20332446477408	-2.25955974306416	0.636621979814147\\
1.34782218447513	-2.17645936305659	0.636621979814147\\
1.48763796765419	-2.08339465229075	0.636621979814147\\
1.62176061116674	-1.98078078546518	0.636621979814147\\
1.74923170735572	-1.86916784532085	0.636621979814147\\
1.86916784532085	-1.74923170735572	0.636621979814147\\
1.98078078546518	-1.62176061116674	0.636621979814147\\
2.08339465229075	-1.48763796765419	0.636621979814147\\
2.17645936305659	-1.34782218447513	0.636621979814147\\
2.25955974306416	-1.20332446477409	0.636621979814147\\
2.33242004905403	-1.05518563047969	0.636621979814147\\
2.39490390373239	-0.904453034649857	0.636621979814147\\
2.44700991003385	-0.752158560541666	0.636621979814147\\
2.48886344018468	-0.59929857009015	0.636621979814147\\
2.520705265116	-0.446816479571312	0.636621979814147\\
2.54287779528981	-0.29558842707731	0.636621979814147\\
2.55580974353639	-0.146412276959492	0.636621979814147\\
2.56	-1.03869855948066e-15	0.636621979814147\\
2.58	0	0.636621979814147\\
2.57577700715777	0.147556122873237	0.636621979814147\\
2.56274402806551	0.29789771166385	0.636621979814147\\
2.54039827499971	0.450307233317961	0.636621979814147\\
2.50830768581112	0.603980590168978	0.636621979814147\\
2.46612717495599	0.758034799295896	0.636621979814147\\
2.41361409048031	0.911519073983057	0.636621979814147\\
2.35064208068726	1.06342926821781	0.636621979814147\\
2.27721255355685	1.21272543715513	0.636621979814147\\
2.19346295183047	1.35835204529135	0.636621979814147\\
2.09967117301177	1.49926013927649	0.636621979814147\\
1.99625563535163	1.63443061594148	0.636621979814147\\
1.88377071911242	1.76289758006944	0.636621979814147\\
1.76289758006944	1.88377071911242	0.636621979814147\\
1.63443061594148	1.99625563535163	0.636621979814147\\
1.49926013927649	2.09967117301177	0.636621979814147\\
1.35835204529135	2.19346295183047	0.636621979814147\\
1.21272543715513	2.27721255355685	0.636621979814147\\
1.06342926821781	2.35064208068726	0.636621979814147\\
0.911519073983058	2.4136140904803	0.636621979814147\\
0.758034799295897	2.46612717495599	0.636621979814147\\
0.603980590168979	2.50830768581112	0.636621979814147\\
0.450307233317962	2.54039827499971	0.636621979814147\\
0.297897711663851	2.56274402806551	0.636621979814147\\
0.147556122873237	2.57577700715777	0.636621979814147\\
5.72875080706581e-16	2.58	0.636621979814147\\
-0.147556122873236	2.57577700715777	0.636621979814147\\
-0.29789771166385	2.56274402806551	0.636621979814147\\
-0.450307233317961	2.54039827499971	0.636621979814147\\
-0.603980590168977	2.50830768581112	0.636621979814147\\
-0.758034799295896	2.46612717495599	0.636621979814147\\
-0.911519073983057	2.41361409048031	0.636621979814147\\
-1.06342926821781	2.35064208068726	0.636621979814147\\
-1.21272543715513	2.27721255355685	0.636621979814147\\
-1.35835204529135	2.19346295183047	0.636621979814147\\
-1.49926013927649	2.09967117301177	0.636621979814147\\
-1.63443061594148	1.99625563535163	0.636621979814147\\
-1.76289758006944	1.88377071911242	0.636621979814147\\
-1.88377071911242	1.76289758006944	0.636621979814147\\
-1.99625563535163	1.63443061594148	0.636621979814147\\
-2.09967117301177	1.49926013927649	0.636621979814147\\
-2.19346295183046	1.35835204529135	0.636621979814147\\
-2.27721255355685	1.21272543715513	0.636621979814147\\
-2.35064208068726	1.06342926821781	0.636621979814147\\
-2.4136140904803	0.911519073983058	0.636621979814147\\
-2.46612717495599	0.758034799295897	0.636621979814147\\
-2.50830768581112	0.603980590168978	0.636621979814147\\
-2.54039827499971	0.450307233317962	0.636621979814147\\
-2.56274402806551	0.297897711663851	0.636621979814147\\
-2.57577700715777	0.147556122873237	0.636621979814147\\
-2.58	7.30854517796589e-16	0.636621979814147\\
-2.57577700715777	-0.147556122873236	0.636621979814147\\
-2.56274402806551	-0.29789771166385	0.636621979814147\\
-2.54039827499971	-0.450307233317961	0.636621979814147\\
-2.50830768581112	-0.603980590168978	0.636621979814147\\
-2.46612717495599	-0.758034799295897	0.636621979814147\\
-2.4136140904803	-0.911519073983058	0.636621979814147\\
-2.35064208068726	-1.06342926821781	0.636621979814147\\
-2.27721255355685	-1.21272543715513	0.636621979814147\\
-2.19346295183046	-1.35835204529134	0.636621979814147\\
-2.09967117301177	-1.49926013927649	0.636621979814147\\
-1.99625563535163	-1.63443061594148	0.636621979814147\\
-1.88377071911242	-1.76289758006944	0.636621979814147\\
-1.76289758006944	-1.88377071911242	0.636621979814147\\
-1.63443061594148	-1.99625563535163	0.636621979814147\\
-1.49926013927649	-2.09967117301177	0.636621979814147\\
-1.35835204529135	-2.19346295183046	0.636621979814147\\
-1.21272543715513	-2.27721255355685	0.636621979814147\\
-1.06342926821781	-2.35064208068726	0.636621979814147\\
-0.911519073983059	-2.4136140904803	0.636621979814147\\
-0.758034799295898	-2.46612717495599	0.636621979814147\\
-0.603980590168979	-2.50830768581112	0.636621979814147\\
-0.450307233317962	-2.54039827499971	0.636621979814147\\
-0.297897711663851	-2.56274402806551	0.636621979814147\\
-0.147556122873238	-2.57577700715777	0.636621979814147\\
-8.88833954886598e-16	-2.58	0.636621979814147\\
0.147556122873236	-2.57577700715777	0.636621979814147\\
0.29789771166385	-2.56274402806551	0.636621979814147\\
0.450307233317961	-2.54039827499971	0.636621979814147\\
0.603980590168978	-2.50830768581112	0.636621979814147\\
0.758034799295897	-2.46612717495599	0.636621979814147\\
0.911519073983057	-2.4136140904803	0.636621979814147\\
1.06342926821781	-2.35064208068726	0.636621979814147\\
1.21272543715513	-2.27721255355685	0.636621979814147\\
1.35835204529134	-2.19346295183046	0.636621979814147\\
1.49926013927649	-2.09967117301177	0.636621979814147\\
1.63443061594148	-1.99625563535163	0.636621979814147\\
1.76289758006944	-1.88377071911242	0.636621979814147\\
1.88377071911242	-1.76289758006944	0.636621979814147\\
1.99625563535163	-1.63443061594148	0.636621979814147\\
2.09967117301177	-1.49926013927649	0.636621979814147\\
2.19346295183046	-1.35835204529135	0.636621979814147\\
2.27721255355685	-1.21272543715513	0.636621979814147\\
2.35064208068726	-1.06342926821781	0.636621979814147\\
2.4136140904803	-0.911519073983059	0.636621979814147\\
2.46612717495599	-0.758034799295898	0.636621979814147\\
2.50830768581112	-0.603980590168979	0.636621979814147\\
2.54039827499971	-0.450307233317962	0.636621979814147\\
2.56274402806551	-0.297897711663852	0.636621979814147\\
2.57577700715777	-0.147556122873238	0.636621979814147\\
2.58	-1.04681339197661e-15	0.636621979814147\\
2.6	0	0.636621979814147\\
2.59574427077914	0.148699968786983	0.636621979814147\\
2.58261026084121	0.300206996250392	0.636621979814147\\
2.56009128488343	0.453797987064612	0.636621979814147\\
2.52775193143756	0.608662610247807	0.636621979814147\\
2.48524443987813	0.763911038050128	0.636621979814147\\
2.43232427722821	0.918585113316259	0.636621979814147\\
2.3688641123205	1.07167290595593	0.636621979814147\\
2.29486536404953	1.22212640953618	0.636621979814147\\
2.21046654060434	1.36888190610756	0.636621979814147\\
2.1159476937328	1.51088231089879	0.636621979814147\\
2.01173048523808	1.64710062071622	0.636621979814147\\
1.89837359290399	1.77656345278315	0.636621979814147\\
1.77656345278315	1.89837359290399	0.636621979814147\\
1.64710062071622	2.01173048523808	0.636621979814147\\
1.51088231089879	2.1159476937328	0.636621979814147\\
1.36888190610756	2.21046654060434	0.636621979814147\\
1.22212640953618	2.29486536404953	0.636621979814147\\
1.07167290595593	2.3688641123205	0.636621979814147\\
0.91858511331626	2.43232427722821	0.636621979814147\\
0.763911038050129	2.48524443987813	0.636621979814147\\
0.608662610247808	2.52775193143756	0.636621979814147\\
0.453797987064613	2.56009128488343	0.636621979814147\\
0.300206996250392	2.58261026084121	0.636621979814147\\
0.148699968786983	2.59574427077914	0.636621979814147\\
5.77315972805081e-16	2.6	0.636621979814147\\
-0.148699968786982	2.59574427077914	0.636621979814147\\
-0.300206996250392	2.58261026084121	0.636621979814147\\
-0.453797987064612	2.56009128488343	0.636621979814147\\
-0.608662610247806	2.52775193143756	0.636621979814147\\
-0.763911038050128	2.48524443987813	0.636621979814147\\
-0.918585113316259	2.43232427722821	0.636621979814147\\
-1.07167290595593	2.3688641123205	0.636621979814147\\
-1.22212640953618	2.29486536404953	0.636621979814147\\
-1.36888190610756	2.21046654060434	0.636621979814147\\
-1.51088231089879	2.1159476937328	0.636621979814147\\
-1.64710062071622	2.01173048523808	0.636621979814147\\
-1.77656345278315	1.89837359290399	0.636621979814147\\
-1.89837359290399	1.77656345278315	0.636621979814147\\
-2.01173048523808	1.64710062071622	0.636621979814147\\
-2.1159476937328	1.51088231089879	0.636621979814147\\
-2.21046654060434	1.36888190610756	0.636621979814147\\
-2.29486536404953	1.22212640953618	0.636621979814147\\
-2.3688641123205	1.07167290595593	0.636621979814147\\
-2.43232427722821	0.91858511331626	0.636621979814147\\
-2.48524443987813	0.763911038050129	0.636621979814147\\
-2.52775193143756	0.608662610247807	0.636621979814147\\
-2.56009128488343	0.453797987064613	0.636621979814147\\
-2.58261026084121	0.300206996250393	0.636621979814147\\
-2.59574427077914	0.148699968786983	0.636621979814147\\
-2.6	7.36520056694237e-16	0.636621979814147\\
-2.59574427077914	-0.148699968786982	0.636621979814147\\
-2.58261026084121	-0.300206996250392	0.636621979814147\\
-2.56009128488343	-0.453797987064612	0.636621979814147\\
-2.52775193143756	-0.608662610247807	0.636621979814147\\
-2.48524443987813	-0.763911038050129	0.636621979814147\\
-2.43232427722821	-0.918585113316259	0.636621979814147\\
-2.3688641123205	-1.07167290595593	0.636621979814147\\
-2.29486536404954	-1.22212640953618	0.636621979814147\\
-2.21046654060434	-1.36888190610756	0.636621979814147\\
-2.1159476937328	-1.51088231089879	0.636621979814147\\
-2.01173048523808	-1.64710062071622	0.636621979814147\\
-1.89837359290399	-1.77656345278315	0.636621979814147\\
-1.77656345278315	-1.89837359290399	0.636621979814147\\
-1.64710062071622	-2.01173048523808	0.636621979814147\\
-1.51088231089879	-2.1159476937328	0.636621979814147\\
-1.36888190610756	-2.21046654060434	0.636621979814147\\
-1.22212640953618	-2.29486536404953	0.636621979814147\\
-1.07167290595593	-2.3688641123205	0.636621979814147\\
-0.918585113316261	-2.43232427722821	0.636621979814147\\
-0.76391103805013	-2.48524443987813	0.636621979814147\\
-0.608662610247808	-2.52775193143756	0.636621979814147\\
-0.453797987064613	-2.56009128488343	0.636621979814147\\
-0.300206996250393	-2.58261026084121	0.636621979814147\\
-0.148699968786984	-2.59574427077914	0.636621979814147\\
-8.95724140583393e-16	-2.6	0.636621979814147\\
0.148699968786982	-2.59574427077914	0.636621979814147\\
0.300206996250391	-2.58261026084121	0.636621979814147\\
0.453797987064612	-2.56009128488343	0.636621979814147\\
0.608662610247807	-2.52775193143756	0.636621979814147\\
0.763911038050129	-2.48524443987813	0.636621979814147\\
0.918585113316259	-2.43232427722821	0.636621979814147\\
1.07167290595593	-2.3688641123205	0.636621979814147\\
1.22212640953618	-2.29486536404954	0.636621979814147\\
1.36888190610756	-2.21046654060434	0.636621979814147\\
1.51088231089879	-2.1159476937328	0.636621979814147\\
1.64710062071622	-2.01173048523808	0.636621979814147\\
1.77656345278315	-1.89837359290399	0.636621979814147\\
1.89837359290399	-1.77656345278315	0.636621979814147\\
2.01173048523808	-1.64710062071622	0.636621979814147\\
2.1159476937328	-1.51088231089879	0.636621979814147\\
2.21046654060434	-1.36888190610756	0.636621979814147\\
2.29486536404953	-1.22212640953618	0.636621979814147\\
2.3688641123205	-1.07167290595593	0.636621979814147\\
2.43232427722821	-0.918585113316261	0.636621979814147\\
2.48524443987813	-0.76391103805013	0.636621979814147\\
2.52775193143756	-0.608662610247809	0.636621979814147\\
2.56009128488343	-0.453797987064613	0.636621979814147\\
2.58261026084121	-0.300206996250393	0.636621979814147\\
2.59574427077914	-0.148699968786984	0.636621979814147\\
2.6	-1.05492822447255e-15	0.636621979814147\\
2.62	0	0.636621979814147\\
2.61571153440052	0.149843814700729	0.636621979814147\\
2.60247649361691	0.302516280836933	0.636621979814147\\
2.57978429476715	0.457288740811263	0.636621979814147\\
2.54719617706401	0.613344630326637	0.636621979814147\\
2.50436170480027	0.76978727680436	0.636621979814147\\
2.45103446397612	0.925651152649461	0.636621979814147\\
2.38708614395373	1.07991654369405	0.636621979814147\\
2.31251817454222	1.23152738191723	0.636621979814147\\
2.22747012937822	1.37941176692377	0.636621979814147\\
2.13222421445382	1.52250448252109	0.636621979814147\\
2.02720533512452	1.65977062549096	0.636621979814147\\
1.91297646669556	1.79022932549687	0.636621979814147\\
1.79022932549687	1.91297646669556	0.636621979814147\\
1.65977062549096	2.02720533512452	0.636621979814147\\
1.52250448252109	2.13222421445382	0.636621979814147\\
1.37941176692377	2.22747012937822	0.636621979814147\\
1.23152738191723	2.31251817454222	0.636621979814147\\
1.07991654369405	2.38708614395373	0.636621979814147\\
0.925651152649462	2.45103446397612	0.636621979814147\\
0.769787276804361	2.50436170480027	0.636621979814147\\
0.613344630326637	2.54719617706401	0.636621979814147\\
0.457288740811263	2.57978429476715	0.636621979814147\\
0.302516280836934	2.60247649361691	0.636621979814147\\
0.149843814700729	2.61571153440052	0.636621979814147\\
5.81756864903582e-16	2.62	0.636621979814147\\
-0.149843814700728	2.61571153440052	0.636621979814147\\
-0.302516280836933	2.60247649361691	0.636621979814147\\
-0.457288740811263	2.57978429476715	0.636621979814147\\
-0.613344630326636	2.54719617706401	0.636621979814147\\
-0.76978727680436	2.50436170480027	0.636621979814147\\
-0.925651152649461	2.45103446397612	0.636621979814147\\
-1.07991654369405	2.38708614395373	0.636621979814147\\
-1.23152738191723	2.31251817454222	0.636621979814147\\
-1.37941176692377	2.22747012937822	0.636621979814147\\
-1.52250448252109	2.13222421445382	0.636621979814147\\
-1.65977062549096	2.02720533512452	0.636621979814147\\
-1.79022932549687	1.91297646669556	0.636621979814147\\
-1.91297646669556	1.79022932549687	0.636621979814147\\
-2.02720533512452	1.65977062549096	0.636621979814147\\
-2.13222421445382	1.52250448252109	0.636621979814147\\
-2.22747012937822	1.37941176692377	0.636621979814147\\
-2.31251817454222	1.23152738191723	0.636621979814147\\
-2.38708614395373	1.07991654369405	0.636621979814147\\
-2.45103446397612	0.925651152649462	0.636621979814147\\
-2.50436170480027	0.769787276804361	0.636621979814147\\
-2.54719617706401	0.613344630326637	0.636621979814147\\
-2.57978429476715	0.457288740811264	0.636621979814147\\
-2.60247649361691	0.302516280836934	0.636621979814147\\
-2.61571153440052	0.149843814700729	0.636621979814147\\
-2.62	7.42185595591885e-16	0.636621979814147\\
-2.61571153440052	-0.149843814700728	0.636621979814147\\
-2.60247649361691	-0.302516280836933	0.636621979814147\\
-2.57978429476715	-0.457288740811263	0.636621979814147\\
-2.54719617706401	-0.613344630326637	0.636621979814147\\
-2.50436170480027	-0.76978727680436	0.636621979814147\\
-2.45103446397612	-0.925651152649462	0.636621979814147\\
-2.38708614395373	-1.07991654369405	0.636621979814147\\
-2.31251817454222	-1.23152738191723	0.636621979814147\\
-2.22747012937822	-1.37941176692377	0.636621979814147\\
-2.13222421445382	-1.52250448252109	0.636621979814147\\
-2.02720533512452	-1.65977062549096	0.636621979814147\\
-1.91297646669556	-1.79022932549687	0.636621979814147\\
-1.79022932549687	-1.91297646669556	0.636621979814147\\
-1.65977062549096	-2.02720533512452	0.636621979814147\\
-1.52250448252109	-2.13222421445382	0.636621979814147\\
-1.37941176692377	-2.22747012937822	0.636621979814147\\
-1.23152738191723	-2.31251817454222	0.636621979814147\\
-1.07991654369405	-2.38708614395373	0.636621979814147\\
-0.925651152649463	-2.45103446397612	0.636621979814147\\
-0.769787276804362	-2.50436170480027	0.636621979814147\\
-0.613344630326638	-2.54719617706401	0.636621979814147\\
-0.457288740811264	-2.57978429476715	0.636621979814147\\
-0.302516280836935	-2.60247649361691	0.636621979814147\\
-0.14984381470073	-2.61571153440052	0.636621979814147\\
-9.02614326280189e-16	-2.62	0.636621979814147\\
0.149843814700728	-2.61571153440052	0.636621979814147\\
0.302516280836933	-2.60247649361691	0.636621979814147\\
0.457288740811262	-2.57978429476715	0.636621979814147\\
0.613344630326636	-2.54719617706401	0.636621979814147\\
0.76978727680436	-2.50436170480027	0.636621979814147\\
0.925651152649461	-2.45103446397612	0.636621979814147\\
1.07991654369405	-2.38708614395373	0.636621979814147\\
1.23152738191723	-2.31251817454222	0.636621979814147\\
1.37941176692377	-2.22747012937822	0.636621979814147\\
1.52250448252109	-2.13222421445382	0.636621979814147\\
1.65977062549096	-2.02720533512452	0.636621979814147\\
1.79022932549687	-1.91297646669556	0.636621979814147\\
1.91297646669556	-1.79022932549687	0.636621979814147\\
2.02720533512452	-1.65977062549096	0.636621979814147\\
2.13222421445382	-1.52250448252109	0.636621979814147\\
2.22747012937822	-1.37941176692377	0.636621979814147\\
2.31251817454222	-1.23152738191723	0.636621979814147\\
2.38708614395373	-1.07991654369405	0.636621979814147\\
2.45103446397612	-0.925651152649463	0.636621979814147\\
2.50436170480027	-0.769787276804362	0.636621979814147\\
2.54719617706401	-0.613344630326638	0.636621979814147\\
2.57978429476715	-0.457288740811264	0.636621979814147\\
2.60247649361691	-0.302516280836935	0.636621979814147\\
2.61571153440052	-0.14984381470073	0.636621979814147\\
2.62	-1.06304305696849e-15	0.636621979814147\\
2.64	0	0.636621979814147\\
2.6356787980219	0.150987660614475	0.636621979814147\\
2.62234272639262	0.304825565423475	0.636621979814147\\
2.59947730465087	0.460779494557914	0.636621979814147\\
2.56664042269045	0.618026650405466	0.636621979814147\\
2.52347896972241	0.775663515558592	0.636621979814147\\
2.46974465072403	0.932717191982663	0.636621979814147\\
2.40530817558696	1.08816018143218	0.636621979814147\\
2.33017098503491	1.24092835429828	0.636621979814147\\
2.2444737181521	1.38994162773998	0.636621979814147\\
2.14850073517484	1.53412665414339	0.636621979814147\\
2.04268018501097	1.6724406302657	0.636621979814147\\
1.92757934048713	1.80389519821059	0.636621979814147\\
1.80389519821059	1.92757934048713	0.636621979814147\\
1.6724406302657	2.04268018501097	0.636621979814147\\
1.53412665414339	2.14850073517484	0.636621979814147\\
1.38994162773998	2.2444737181521	0.636621979814147\\
1.24092835429828	2.33017098503491	0.636621979814147\\
1.08816018143218	2.40530817558696	0.636621979814147\\
0.932717191982664	2.46974465072403	0.636621979814147\\
0.775663515558592	2.52347896972241	0.636621979814147\\
0.618026650405467	2.56664042269045	0.636621979814147\\
0.460779494557914	2.59947730465087	0.636621979814147\\
0.304825565423475	2.62234272639262	0.636621979814147\\
0.150987660614475	2.6356787980219	0.636621979814147\\
5.86197757002083e-16	2.64	0.636621979814147\\
-0.150987660614474	2.6356787980219	0.636621979814147\\
-0.304825565423475	2.62234272639262	0.636621979814147\\
-0.460779494557914	2.59947730465087	0.636621979814147\\
-0.618026650405465	2.56664042269045	0.636621979814147\\
-0.775663515558591	2.52347896972241	0.636621979814147\\
-0.932717191982663	2.46974465072403	0.636621979814147\\
-1.08816018143218	2.40530817558696	0.636621979814147\\
-1.24092835429828	2.33017098503491	0.636621979814147\\
-1.38994162773998	2.2444737181521	0.636621979814147\\
-1.53412665414339	2.14850073517484	0.636621979814147\\
-1.6724406302657	2.04268018501097	0.636621979814147\\
-1.80389519821059	1.92757934048713	0.636621979814147\\
-1.92757934048713	1.80389519821059	0.636621979814147\\
-2.04268018501097	1.6724406302657	0.636621979814147\\
-2.14850073517484	1.53412665414339	0.636621979814147\\
-2.2444737181521	1.38994162773998	0.636621979814147\\
-2.33017098503491	1.24092835429828	0.636621979814147\\
-2.40530817558696	1.08816018143218	0.636621979814147\\
-2.46974465072403	0.932717191982664	0.636621979814147\\
-2.52347896972241	0.775663515558592	0.636621979814147\\
-2.56664042269045	0.618026650405466	0.636621979814147\\
-2.59947730465087	0.460779494557915	0.636621979814147\\
-2.62234272639262	0.304825565423476	0.636621979814147\\
-2.6356787980219	0.150987660614476	0.636621979814147\\
-2.64	7.47851134489533e-16	0.636621979814147\\
-2.6356787980219	-0.150987660614474	0.636621979814147\\
-2.62234272639262	-0.304825565423475	0.636621979814147\\
-2.59947730465087	-0.460779494557913	0.636621979814147\\
-2.56664042269045	-0.618026650405466	0.636621979814147\\
-2.52347896972241	-0.775663515558592	0.636621979814147\\
-2.46974465072403	-0.932717191982664	0.636621979814147\\
-2.40530817558697	-1.08816018143217	0.636621979814147\\
-2.33017098503491	-1.24092835429827	0.636621979814147\\
-2.2444737181521	-1.38994162773998	0.636621979814147\\
-2.14850073517484	-1.53412665414339	0.636621979814147\\
-2.04268018501097	-1.6724406302657	0.636621979814147\\
-1.92757934048713	-1.80389519821059	0.636621979814147\\
-1.80389519821059	-1.92757934048713	0.636621979814147\\
-1.6724406302657	-2.04268018501097	0.636621979814147\\
-1.53412665414339	-2.14850073517484	0.636621979814147\\
-1.38994162773998	-2.2444737181521	0.636621979814147\\
-1.24092835429828	-2.33017098503491	0.636621979814147\\
-1.08816018143218	-2.40530817558697	0.636621979814147\\
-0.932717191982665	-2.46974465072403	0.636621979814147\\
-0.775663515558593	-2.52347896972241	0.636621979814147\\
-0.618026650405467	-2.56664042269045	0.636621979814147\\
-0.460779494557915	-2.59947730465087	0.636621979814147\\
-0.304825565423476	-2.62234272639262	0.636621979814147\\
-0.150987660614476	-2.6356787980219	0.636621979814147\\
-9.09504511976984e-16	-2.64	0.636621979814147\\
0.150987660614474	-2.6356787980219	0.636621979814147\\
0.304825565423474	-2.62234272639262	0.636621979814147\\
0.460779494557913	-2.59947730465087	0.636621979814147\\
0.618026650405466	-2.56664042269045	0.636621979814147\\
0.775663515558592	-2.52347896972241	0.636621979814147\\
0.932717191982663	-2.46974465072403	0.636621979814147\\
1.08816018143217	-2.40530817558697	0.636621979814147\\
1.24092835429827	-2.33017098503491	0.636621979814147\\
1.38994162773998	-2.2444737181521	0.636621979814147\\
1.53412665414339	-2.14850073517484	0.636621979814147\\
1.6724406302657	-2.04268018501097	0.636621979814147\\
1.80389519821059	-1.92757934048713	0.636621979814147\\
1.92757934048713	-1.80389519821059	0.636621979814147\\
2.04268018501097	-1.6724406302657	0.636621979814147\\
2.14850073517484	-1.53412665414339	0.636621979814147\\
2.2444737181521	-1.38994162773998	0.636621979814147\\
2.33017098503491	-1.24092835429828	0.636621979814147\\
2.40530817558697	-1.08816018143218	0.636621979814147\\
2.46974465072403	-0.932717191982665	0.636621979814147\\
2.52347896972241	-0.775663515558593	0.636621979814147\\
2.56664042269045	-0.618026650405467	0.636621979814147\\
2.59947730465087	-0.460779494557915	0.636621979814147\\
2.62234272639262	-0.304825565423476	0.636621979814147\\
2.6356787980219	-0.150987660614476	0.636621979814147\\
2.64	-1.07115788946443e-15	0.636621979814147\\
2.66	0	0.636621979814147\\
2.65564606164328	0.152131506528221	0.636621979814147\\
2.64220895916832	0.307134850010016	0.636621979814147\\
2.61917031453459	0.464270248304565	0.636621979814147\\
2.58608466831689	0.622708670484295	0.636621979814147\\
2.54259623464455	0.781539754312824	0.636621979814147\\
2.48845483747194	0.939783231315865	0.636621979814147\\
2.4235302072202	1.0964038191703	0.636621979814147\\
2.3478237955276	1.25032932667932	0.636621979814147\\
2.26147730692598	1.40047148855619	0.636621979814147\\
2.16477725589586	1.54574882576568	0.636621979814147\\
2.05815503489742	1.68511063504044	0.636621979814147\\
1.9421822142787	1.8175610709243	0.636621979814147\\
1.8175610709243	1.9421822142787	0.636621979814147\\
1.68511063504044	2.05815503489742	0.636621979814147\\
1.54574882576568	2.16477725589586	0.636621979814147\\
1.40047148855619	2.26147730692598	0.636621979814147\\
1.25032932667932	2.3478237955276	0.636621979814147\\
1.0964038191703	2.4235302072202	0.636621979814147\\
0.939783231315866	2.48845483747194	0.636621979814147\\
0.781539754312824	2.54259623464455	0.636621979814147\\
0.622708670484296	2.58608466831689	0.636621979814147\\
0.464270248304565	2.61917031453459	0.636621979814147\\
0.307134850010017	2.64220895916832	0.636621979814147\\
0.152131506528221	2.65564606164328	0.636621979814147\\
5.90638649100583e-16	2.66	0.636621979814147\\
-0.15213150652822	2.65564606164328	0.636621979814147\\
-0.307134850010016	2.64220895916832	0.636621979814147\\
-0.464270248304564	2.61917031453459	0.636621979814147\\
-0.622708670484294	2.58608466831689	0.636621979814147\\
-0.781539754312823	2.54259623464455	0.636621979814147\\
-0.939783231315865	2.48845483747194	0.636621979814147\\
-1.0964038191703	2.4235302072202	0.636621979814147\\
-1.25032932667932	2.3478237955276	0.636621979814147\\
-1.40047148855619	2.26147730692598	0.636621979814147\\
-1.54574882576568	2.16477725589586	0.636621979814147\\
-1.68511063504044	2.05815503489742	0.636621979814147\\
-1.8175610709243	1.9421822142787	0.636621979814147\\
-1.9421822142787	1.8175610709243	0.636621979814147\\
-2.05815503489742	1.68511063504044	0.636621979814147\\
-2.16477725589586	1.54574882576568	0.636621979814147\\
-2.26147730692598	1.4004714885562	0.636621979814147\\
-2.3478237955276	1.25032932667932	0.636621979814147\\
-2.4235302072202	1.0964038191703	0.636621979814147\\
-2.48845483747194	0.939783231315866	0.636621979814147\\
-2.54259623464455	0.781539754312824	0.636621979814147\\
-2.58608466831689	0.622708670484295	0.636621979814147\\
-2.61917031453459	0.464270248304566	0.636621979814147\\
-2.64220895916832	0.307134850010017	0.636621979814147\\
-2.65564606164328	0.152131506528222	0.636621979814147\\
-2.66	7.53516673387181e-16	0.636621979814147\\
-2.65564606164328	-0.15213150652822	0.636621979814147\\
-2.64220895916832	-0.307134850010016	0.636621979814147\\
-2.61917031453459	-0.464270248304564	0.636621979814147\\
-2.58608466831689	-0.622708670484295	0.636621979814147\\
-2.54259623464455	-0.781539754312824	0.636621979814147\\
-2.48845483747194	-0.939783231315866	0.636621979814147\\
-2.4235302072202	-1.0964038191703	0.636621979814147\\
-2.3478237955276	-1.25032932667932	0.636621979814147\\
-2.26147730692598	-1.40047148855619	0.636621979814147\\
-2.16477725589586	-1.54574882576568	0.636621979814147\\
-2.05815503489742	-1.68511063504044	0.636621979814147\\
-1.9421822142787	-1.8175610709243	0.636621979814147\\
-1.8175610709243	-1.9421822142787	0.636621979814147\\
-1.68511063504044	-2.05815503489742	0.636621979814147\\
-1.54574882576568	-2.16477725589586	0.636621979814147\\
-1.40047148855619	-2.26147730692598	0.636621979814147\\
-1.25032932667932	-2.3478237955276	0.636621979814147\\
-1.0964038191703	-2.4235302072202	0.636621979814147\\
-0.939783231315867	-2.48845483747194	0.636621979814147\\
-0.781539754312825	-2.54259623464455	0.636621979814147\\
-0.622708670484296	-2.58608466831689	0.636621979814147\\
-0.464270248304566	-2.61917031453459	0.636621979814147\\
-0.307134850010017	-2.64220895916832	0.636621979814147\\
-0.152131506528222	-2.65564606164328	0.636621979814147\\
-9.16394697673779e-16	-2.66	0.636621979814147\\
0.15213150652822	-2.65564606164328	0.636621979814147\\
0.307134850010016	-2.64220895916832	0.636621979814147\\
0.464270248304564	-2.61917031453459	0.636621979814147\\
0.622708670484295	-2.58608466831689	0.636621979814147\\
0.781539754312824	-2.54259623464455	0.636621979814147\\
0.939783231315865	-2.48845483747194	0.636621979814147\\
1.0964038191703	-2.4235302072202	0.636621979814147\\
1.25032932667932	-2.3478237955276	0.636621979814147\\
1.40047148855619	-2.26147730692598	0.636621979814147\\
1.54574882576568	-2.16477725589586	0.636621979814147\\
1.68511063504044	-2.05815503489742	0.636621979814147\\
1.8175610709243	-1.9421822142787	0.636621979814147\\
1.9421822142787	-1.8175610709243	0.636621979814147\\
2.05815503489742	-1.68511063504044	0.636621979814147\\
2.16477725589586	-1.54574882576568	0.636621979814147\\
2.26147730692598	-1.40047148855619	0.636621979814147\\
2.3478237955276	-1.25032932667932	0.636621979814147\\
2.4235302072202	-1.0964038191703	0.636621979814147\\
2.48845483747194	-0.939783231315867	0.636621979814147\\
2.54259623464455	-0.781539754312826	0.636621979814147\\
2.58608466831689	-0.622708670484297	0.636621979814147\\
2.61917031453459	-0.464270248304566	0.636621979814147\\
2.64220895916832	-0.307134850010018	0.636621979814147\\
2.65564606164328	-0.152131506528222	0.636621979814147\\
2.66	-1.07927272196038e-15	0.636621979814147\\
2.68	0	0.636621979814147\\
2.67561332526466	0.153275352441967	0.636621979814147\\
2.66207519194402	0.309444134596558	0.636621979814147\\
2.63886332441831	0.467761002051215	0.636621979814147\\
2.60552891394333	0.627390690563124	0.636621979814147\\
2.56171349956669	0.787415993067055	0.636621979814147\\
2.50716502421985	0.946849270649067	0.636621979814147\\
2.44175223885343	1.10464745690842	0.636621979814147\\
2.36547660602029	1.25973029906037	0.636621979814147\\
2.27848089569986	1.41100134937241	0.636621979814147\\
2.18105377661688	1.55737099738798	0.636621979814147\\
2.07362988478386	1.69778063981518	0.636621979814147\\
1.95678508807027	1.83122694363802	0.636621979814147\\
1.83122694363802	1.95678508807027	0.636621979814147\\
1.69778063981518	2.07362988478386	0.636621979814147\\
1.55737099738798	2.18105377661688	0.636621979814147\\
1.41100134937241	2.27848089569986	0.636621979814147\\
1.25973029906037	2.36547660602029	0.636621979814147\\
1.10464745690842	2.44175223885343	0.636621979814147\\
0.946849270649068	2.50716502421985	0.636621979814147\\
0.787415993067056	2.56171349956669	0.636621979814147\\
0.627390690563125	2.60552891394333	0.636621979814147\\
0.467761002051216	2.63886332441831	0.636621979814147\\
0.309444134596558	2.66207519194402	0.636621979814147\\
0.153275352441967	2.67561332526466	0.636621979814147\\
5.95079541199084e-16	2.68	0.636621979814147\\
-0.153275352441966	2.67561332526466	0.636621979814147\\
-0.309444134596558	2.66207519194402	0.636621979814147\\
-0.467761002051215	2.63886332441831	0.636621979814147\\
-0.627390690563124	2.60552891394333	0.636621979814147\\
-0.787415993067055	2.56171349956669	0.636621979814147\\
-0.946849270649067	2.50716502421985	0.636621979814147\\
-1.10464745690842	2.44175223885343	0.636621979814147\\
-1.25973029906037	2.36547660602029	0.636621979814147\\
-1.41100134937241	2.27848089569986	0.636621979814147\\
-1.55737099738798	2.18105377661688	0.636621979814147\\
-1.69778063981518	2.07362988478386	0.636621979814147\\
-1.83122694363802	1.95678508807027	0.636621979814147\\
-1.95678508807027	1.83122694363802	0.636621979814147\\
-2.07362988478386	1.69778063981518	0.636621979814147\\
-2.18105377661688	1.55737099738798	0.636621979814147\\
-2.27848089569986	1.41100134937241	0.636621979814147\\
-2.36547660602029	1.25973029906037	0.636621979814147\\
-2.44175223885343	1.10464745690842	0.636621979814147\\
-2.50716502421985	0.946849270649068	0.636621979814147\\
-2.56171349956669	0.787415993067056	0.636621979814147\\
-2.60552891394333	0.627390690563125	0.636621979814147\\
-2.63886332441831	0.467761002051217	0.636621979814147\\
-2.66207519194402	0.309444134596559	0.636621979814147\\
-2.67561332526466	0.153275352441968	0.636621979814147\\
-2.68	7.59182212284829e-16	0.636621979814147\\
-2.67561332526466	-0.153275352441966	0.636621979814147\\
-2.66207519194402	-0.309444134596557	0.636621979814147\\
-2.63886332441831	-0.467761002051215	0.636621979814147\\
-2.60552891394333	-0.627390690563124	0.636621979814147\\
-2.56171349956669	-0.787415993067056	0.636621979814147\\
-2.50716502421985	-0.946849270649068	0.636621979814147\\
-2.44175223885343	-1.10464745690842	0.636621979814147\\
-2.36547660602029	-1.25973029906037	0.636621979814147\\
-2.27848089569986	-1.41100134937241	0.636621979814147\\
-2.18105377661688	-1.55737099738798	0.636621979814147\\
-2.07362988478386	-1.69778063981518	0.636621979814147\\
-1.95678508807027	-1.83122694363802	0.636621979814147\\
-1.83122694363802	-1.95678508807027	0.636621979814147\\
-1.69778063981518	-2.07362988478386	0.636621979814147\\
-1.55737099738798	-2.18105377661688	0.636621979814147\\
-1.41100134937241	-2.27848089569986	0.636621979814147\\
-1.25973029906037	-2.36547660602029	0.636621979814147\\
-1.10464745690842	-2.44175223885343	0.636621979814147\\
-0.946849270649069	-2.50716502421985	0.636621979814147\\
-0.787415993067057	-2.56171349956669	0.636621979814147\\
-0.627390690563126	-2.60552891394333	0.636621979814147\\
-0.467761002051217	-2.63886332441831	0.636621979814147\\
-0.309444134596559	-2.66207519194402	0.636621979814147\\
-0.153275352441968	-2.67561332526466	0.636621979814147\\
-9.23284883370575e-16	-2.68	0.636621979814147\\
0.153275352441966	-2.67561332526466	0.636621979814147\\
0.309444134596557	-2.66207519194402	0.636621979814147\\
0.467761002051215	-2.63886332441831	0.636621979814147\\
0.627390690563124	-2.60552891394333	0.636621979814147\\
0.787415993067056	-2.56171349956669	0.636621979814147\\
0.946849270649067	-2.50716502421985	0.636621979814147\\
1.10464745690842	-2.44175223885343	0.636621979814147\\
1.25973029906037	-2.36547660602029	0.636621979814147\\
1.41100134937241	-2.27848089569986	0.636621979814147\\
1.55737099738798	-2.18105377661688	0.636621979814147\\
1.69778063981518	-2.07362988478386	0.636621979814147\\
1.83122694363802	-1.95678508807027	0.636621979814147\\
1.95678508807027	-1.83122694363802	0.636621979814147\\
2.07362988478386	-1.69778063981518	0.636621979814147\\
2.18105377661688	-1.55737099738798	0.636621979814147\\
2.27848089569986	-1.41100134937241	0.636621979814147\\
2.36547660602029	-1.25973029906037	0.636621979814147\\
2.44175223885343	-1.10464745690842	0.636621979814147\\
2.50716502421985	-0.946849270649069	0.636621979814147\\
2.56171349956669	-0.787415993067057	0.636621979814147\\
2.60552891394333	-0.627390690563126	0.636621979814147\\
2.63886332441831	-0.467761002051217	0.636621979814147\\
2.66207519194402	-0.309444134596559	0.636621979814147\\
2.67561332526466	-0.153275352441968	0.636621979814147\\
2.68	-1.08738755445632e-15	0.636621979814147\\
2.7	0	0.636621979814147\\
2.69558058888603	0.154419198355713	0.636621979814147\\
2.68194142471972	0.311753419183099	0.636621979814147\\
2.65855633430203	0.471251755797866	0.636621979814147\\
2.62497315956978	0.632072710641954	0.636621979814147\\
2.58083076448883	0.793292231821287	0.636621979814147\\
2.52587521096776	0.953915309982269	0.636621979814147\\
2.45997427048667	1.11289109464654	0.636621979814147\\
2.38312941651298	1.26913127144142	0.636621979814147\\
2.29548448447374	1.42153121018862	0.636621979814147\\
2.1973302973379	1.56899316901028	0.636621979814147\\
2.08910473467031	1.71045064458992	0.636621979814147\\
1.97138796186183	1.84489281635174	0.636621979814147\\
1.84489281635174	1.97138796186183	0.636621979814147\\
1.71045064458993	2.08910473467031	0.636621979814147\\
1.56899316901028	2.1973302973379	0.636621979814147\\
1.42153121018862	2.29548448447374	0.636621979814147\\
1.26913127144142	2.38312941651298	0.636621979814147\\
1.11289109464654	2.45997427048667	0.636621979814147\\
0.95391530998227	2.52587521096776	0.636621979814147\\
0.793292231821288	2.58083076448883	0.636621979814147\\
0.632072710641954	2.62497315956978	0.636621979814147\\
0.471251755797867	2.65855633430203	0.636621979814147\\
0.3117534191831	2.68194142471972	0.636621979814147\\
0.154419198355713	2.69558058888603	0.636621979814147\\
5.99520433297585e-16	2.7	0.636621979814147\\
-0.154419198355712	2.69558058888603	0.636621979814147\\
-0.311753419183099	2.68194142471972	0.636621979814147\\
-0.471251755797866	2.65855633430203	0.636621979814147\\
-0.632072710641953	2.62497315956978	0.636621979814147\\
-0.793292231821287	2.58083076448883	0.636621979814147\\
-0.953915309982269	2.52587521096776	0.636621979814147\\
-1.11289109464654	2.45997427048667	0.636621979814147\\
-1.26913127144142	2.38312941651298	0.636621979814147\\
-1.42153121018862	2.29548448447374	0.636621979814147\\
-1.56899316901028	2.1973302973379	0.636621979814147\\
-1.71045064458992	2.08910473467031	0.636621979814147\\
-1.84489281635174	1.97138796186184	0.636621979814147\\
-1.97138796186183	1.84489281635174	0.636621979814147\\
-2.08910473467031	1.71045064458993	0.636621979814147\\
-2.1973302973379	1.56899316901028	0.636621979814147\\
-2.29548448447374	1.42153121018862	0.636621979814147\\
-2.38312941651298	1.26913127144142	0.636621979814147\\
-2.45997427048667	1.11289109464654	0.636621979814147\\
-2.52587521096776	0.95391530998227	0.636621979814147\\
-2.58083076448883	0.793292231821288	0.636621979814147\\
-2.62497315956978	0.632072710641954	0.636621979814147\\
-2.65855633430203	0.471251755797867	0.636621979814147\\
-2.68194142471972	0.3117534191831	0.636621979814147\\
-2.69558058888603	0.154419198355714	0.636621979814147\\
-2.7	7.64847751182477e-16	0.636621979814147\\
-2.69558058888603	-0.154419198355712	0.636621979814147\\
-2.68194142471972	-0.311753419183099	0.636621979814147\\
-2.65855633430203	-0.471251755797866	0.636621979814147\\
-2.62497315956978	-0.632072710641954	0.636621979814147\\
-2.58083076448883	-0.793292231821287	0.636621979814147\\
-2.52587521096776	-0.95391530998227	0.636621979814147\\
-2.45997427048667	-1.11289109464654	0.636621979814147\\
-2.38312941651298	-1.26913127144142	0.636621979814147\\
-2.29548448447374	-1.42153121018862	0.636621979814147\\
-2.1973302973379	-1.56899316901028	0.636621979814147\\
-2.08910473467031	-1.71045064458992	0.636621979814147\\
-1.97138796186184	-1.84489281635174	0.636621979814147\\
-1.84489281635174	-1.97138796186183	0.636621979814147\\
-1.71045064458993	-2.08910473467031	0.636621979814147\\
-1.56899316901028	-2.1973302973379	0.636621979814147\\
-1.42153121018862	-2.29548448447374	0.636621979814147\\
-1.26913127144142	-2.38312941651298	0.636621979814147\\
-1.11289109464654	-2.45997427048667	0.636621979814147\\
-0.953915309982271	-2.52587521096776	0.636621979814147\\
-0.793292231821289	-2.58083076448883	0.636621979814147\\
-0.632072710641955	-2.62497315956978	0.636621979814147\\
-0.471251755797867	-2.65855633430203	0.636621979814147\\
-0.3117534191831	-2.68194142471972	0.636621979814147\\
-0.154419198355714	-2.69558058888603	0.636621979814147\\
-9.3017506906737e-16	-2.7	0.636621979814147\\
0.154419198355712	-2.69558058888603	0.636621979814147\\
0.311753419183099	-2.68194142471972	0.636621979814147\\
0.471251755797866	-2.65855633430203	0.636621979814147\\
0.632072710641953	-2.62497315956978	0.636621979814147\\
0.793292231821287	-2.58083076448883	0.636621979814147\\
0.953915309982269	-2.52587521096776	0.636621979814147\\
1.11289109464654	-2.45997427048667	0.636621979814147\\
1.26913127144142	-2.38312941651298	0.636621979814147\\
1.42153121018862	-2.29548448447374	0.636621979814147\\
1.56899316901028	-2.1973302973379	0.636621979814147\\
1.71045064458992	-2.08910473467031	0.636621979814147\\
1.84489281635174	-1.97138796186184	0.636621979814147\\
1.97138796186183	-1.84489281635174	0.636621979814147\\
2.08910473467031	-1.71045064458993	0.636621979814147\\
2.1973302973379	-1.56899316901028	0.636621979814147\\
2.29548448447374	-1.42153121018862	0.636621979814147\\
2.38312941651298	-1.26913127144142	0.636621979814147\\
2.45997427048667	-1.11289109464654	0.636621979814147\\
2.52587521096776	-0.953915309982271	0.636621979814147\\
2.58083076448883	-0.793292231821289	0.636621979814147\\
2.62497315956978	-0.632072710641955	0.636621979814147\\
2.65855633430203	-0.471251755797868	0.636621979814147\\
2.68194142471972	-0.311753419183101	0.636621979814147\\
2.69558058888603	-0.154419198355714	0.636621979814147\\
2.7	-1.09550238695226e-15	0.636621979814147\\
2.72	0	0.636621979814147\\
2.71554785250741	0.155563044269459	0.636621979814147\\
2.70180765749542	0.314062703769641	0.636621979814147\\
2.67824934418575	0.474742509544517	0.636621979814147\\
2.64441740519622	0.636754730720783	0.636621979814147\\
2.59994802941097	0.799168470575519	0.636621979814147\\
2.54458539771567	0.960981349315471	0.636621979814147\\
2.4781963021199	1.12113473238467	0.636621979814147\\
2.40078222700567	1.27853224382247	0.636621979814147\\
2.31248807324762	1.43206107100483	0.636621979814147\\
2.21360681805892	1.58061534063258	0.636621979814147\\
2.10457958455676	1.72312064936467	0.636621979814147\\
1.9859908356534	1.85855868906545	0.636621979814147\\
1.85855868906545	1.9859908356534	0.636621979814147\\
1.72312064936467	2.10457958455676	0.636621979814147\\
1.58061534063258	2.21360681805892	0.636621979814147\\
1.43206107100483	2.31248807324762	0.636621979814147\\
1.27853224382247	2.40078222700567	0.636621979814147\\
1.12113473238467	2.4781963021199	0.636621979814147\\
0.960981349315472	2.54458539771567	0.636621979814147\\
0.799168470575519	2.59994802941097	0.636621979814147\\
0.636754730720784	2.64441740519622	0.636621979814147\\
0.474742509544518	2.67824934418575	0.636621979814147\\
0.314062703769641	2.70180765749542	0.636621979814147\\
0.155563044269459	2.71554785250741	0.636621979814147\\
6.03961325396085e-16	2.72	0.636621979814147\\
-0.155563044269458	2.71554785250741	0.636621979814147\\
-0.314062703769641	2.70180765749542	0.636621979814147\\
-0.474742509544517	2.67824934418575	0.636621979814147\\
-0.636754730720782	2.64441740519622	0.636621979814147\\
-0.799168470575518	2.59994802941097	0.636621979814147\\
-0.960981349315471	2.54458539771567	0.636621979814147\\
-1.12113473238467	2.4781963021199	0.636621979814147\\
-1.27853224382247	2.40078222700567	0.636621979814147\\
-1.43206107100483	2.31248807324762	0.636621979814147\\
-1.58061534063258	2.21360681805892	0.636621979814147\\
-1.72312064936466	2.10457958455676	0.636621979814147\\
-1.85855868906545	1.9859908356534	0.636621979814147\\
-1.9859908356534	1.85855868906545	0.636621979814147\\
-2.10457958455676	1.72312064936467	0.636621979814147\\
-2.21360681805892	1.58061534063258	0.636621979814147\\
-2.31248807324762	1.43206107100483	0.636621979814147\\
-2.40078222700567	1.27853224382247	0.636621979814147\\
-2.4781963021199	1.12113473238467	0.636621979814147\\
-2.54458539771567	0.960981349315472	0.636621979814147\\
-2.59994802941097	0.799168470575519	0.636621979814147\\
-2.64441740519622	0.636754730720783	0.636621979814147\\
-2.67824934418575	0.474742509544518	0.636621979814147\\
-2.70180765749542	0.314062703769642	0.636621979814147\\
-2.71554785250741	0.15556304426946	0.636621979814147\\
-2.72	7.70513290080125e-16	0.636621979814147\\
-2.71554785250741	-0.155563044269458	0.636621979814147\\
-2.70180765749542	-0.31406270376964	0.636621979814147\\
-2.67824934418575	-0.474742509544517	0.636621979814147\\
-2.64441740519622	-0.636754730720783	0.636621979814147\\
-2.59994802941097	-0.799168470575519	0.636621979814147\\
-2.54458539771567	-0.960981349315472	0.636621979814147\\
-2.4781963021199	-1.12113473238466	0.636621979814147\\
-2.40078222700567	-1.27853224382246	0.636621979814147\\
-2.31248807324762	-1.43206107100483	0.636621979814147\\
-2.21360681805892	-1.58061534063258	0.636621979814147\\
-2.10457958455676	-1.72312064936466	0.636621979814147\\
-1.9859908356534	-1.85855868906545	0.636621979814147\\
-1.85855868906545	-1.9859908356534	0.636621979814147\\
-1.72312064936467	-2.10457958455676	0.636621979814147\\
-1.58061534063258	-2.21360681805892	0.636621979814147\\
-1.43206107100483	-2.31248807324762	0.636621979814147\\
-1.27853224382247	-2.40078222700567	0.636621979814147\\
-1.12113473238467	-2.4781963021199	0.636621979814147\\
-0.960981349315473	-2.54458539771567	0.636621979814147\\
-0.79916847057552	-2.59994802941097	0.636621979814147\\
-0.636754730720784	-2.64441740519622	0.636621979814147\\
-0.474742509544518	-2.67824934418575	0.636621979814147\\
-0.314062703769642	-2.70180765749542	0.636621979814147\\
-0.15556304426946	-2.71554785250741	0.636621979814147\\
-9.37065254764165e-16	-2.72	0.636621979814147\\
0.155563044269458	-2.71554785250741	0.636621979814147\\
0.31406270376964	-2.70180765749542	0.636621979814147\\
0.474742509544517	-2.67824934418575	0.636621979814147\\
0.636754730720783	-2.64441740519622	0.636621979814147\\
0.799168470575519	-2.59994802941097	0.636621979814147\\
0.960981349315471	-2.54458539771567	0.636621979814147\\
1.12113473238466	-2.4781963021199	0.636621979814147\\
1.27853224382246	-2.40078222700567	0.636621979814147\\
1.43206107100483	-2.31248807324762	0.636621979814147\\
1.58061534063258	-2.21360681805892	0.636621979814147\\
1.72312064936466	-2.10457958455676	0.636621979814147\\
1.85855868906545	-1.9859908356534	0.636621979814147\\
1.9859908356534	-1.85855868906545	0.636621979814147\\
2.10457958455676	-1.72312064936467	0.636621979814147\\
2.21360681805892	-1.58061534063258	0.636621979814147\\
2.31248807324762	-1.43206107100483	0.636621979814147\\
2.40078222700567	-1.27853224382247	0.636621979814147\\
2.4781963021199	-1.12113473238467	0.636621979814147\\
2.54458539771567	-0.960981349315473	0.636621979814147\\
2.59994802941097	-0.79916847057552	0.636621979814147\\
2.64441740519622	-0.636754730720784	0.636621979814147\\
2.67824934418575	-0.474742509544518	0.636621979814147\\
2.70180765749542	-0.314062703769642	0.636621979814147\\
2.71554785250741	-0.15556304426946	0.636621979814147\\
2.72	-1.10361721944821e-15	0.636621979814147\\
2.74	0	0.636621979814147\\
2.73551511612879	0.156706890183205	0.636621979814147\\
2.72167389027112	0.316371988356182	0.636621979814147\\
2.69794235406946	0.478233263291168	0.636621979814147\\
2.66386165082266	0.641436750799612	0.636621979814147\\
2.61906529433311	0.805044709329751	0.636621979814147\\
2.56329558446358	0.968047388648673	0.636621979814147\\
2.49641833375314	1.12937837012279	0.636621979814147\\
2.41843503749836	1.28793321620351	0.636621979814147\\
2.3294916620215	1.44259093182104	0.636621979814147\\
2.22988333877995	1.59223751225488	0.636621979814147\\
2.1200544344432	1.73579065413941	0.636621979814147\\
2.00059370944497	1.87222456177917	0.636621979814147\\
1.87222456177917	2.00059370944497	0.636621979814147\\
1.73579065413941	2.1200544344432	0.636621979814147\\
1.59223751225488	2.22988333877995	0.636621979814147\\
1.44259093182104	2.3294916620215	0.636621979814147\\
1.28793321620351	2.41843503749836	0.636621979814147\\
1.12937837012279	2.49641833375314	0.636621979814147\\
0.968047388648674	2.56329558446358	0.636621979814147\\
0.805044709329751	2.61906529433311	0.636621979814147\\
0.641436750799613	2.66386165082266	0.636621979814147\\
0.478233263291169	2.69794235406946	0.636621979814147\\
0.316371988356183	2.72167389027112	0.636621979814147\\
0.156706890183205	2.73551511612879	0.636621979814147\\
6.08402217494586e-16	2.74	0.636621979814147\\
-0.156706890183204	2.73551511612879	0.636621979814147\\
-0.316371988356182	2.72167389027112	0.636621979814147\\
-0.478233263291168	2.69794235406946	0.636621979814147\\
-0.641436750799611	2.66386165082266	0.636621979814147\\
-0.80504470932975	2.61906529433311	0.636621979814147\\
-0.968047388648673	2.56329558446358	0.636621979814147\\
-1.12937837012279	2.49641833375314	0.636621979814147\\
-1.28793321620351	2.41843503749836	0.636621979814147\\
-1.44259093182104	2.3294916620215	0.636621979814147\\
-1.59223751225488	2.22988333877995	0.636621979814147\\
-1.7357906541394	2.1200544344432	0.636621979814147\\
-1.87222456177917	2.00059370944497	0.636621979814147\\
-2.00059370944497	1.87222456177917	0.636621979814147\\
-2.1200544344432	1.73579065413941	0.636621979814147\\
-2.22988333877995	1.59223751225488	0.636621979814147\\
-2.3294916620215	1.44259093182104	0.636621979814147\\
-2.41843503749836	1.28793321620351	0.636621979814147\\
-2.49641833375314	1.12937837012279	0.636621979814147\\
-2.56329558446358	0.968047388648674	0.636621979814147\\
-2.61906529433311	0.805044709329751	0.636621979814147\\
-2.66386165082266	0.641436750799612	0.636621979814147\\
-2.69794235406946	0.478233263291169	0.636621979814147\\
-2.72167389027112	0.316371988356183	0.636621979814147\\
-2.73551511612879	0.156706890183206	0.636621979814147\\
-2.74	7.76178828977773e-16	0.636621979814147\\
-2.73551511612879	-0.156706890183204	0.636621979814147\\
-2.72167389027112	-0.316371988356182	0.636621979814147\\
-2.69794235406946	-0.478233263291168	0.636621979814147\\
-2.66386165082266	-0.641436750799612	0.636621979814147\\
-2.61906529433311	-0.805044709329751	0.636621979814147\\
-2.56329558446358	-0.968047388648673	0.636621979814147\\
-2.49641833375314	-1.12937837012279	0.636621979814147\\
-2.41843503749836	-1.28793321620351	0.636621979814147\\
-2.3294916620215	-1.44259093182104	0.636621979814147\\
-2.22988333877995	-1.59223751225488	0.636621979814147\\
-2.12005443444321	-1.7357906541394	0.636621979814147\\
-2.00059370944497	-1.87222456177917	0.636621979814147\\
-1.87222456177917	-2.00059370944497	0.636621979814147\\
-1.73579065413941	-2.1200544344432	0.636621979814147\\
-1.59223751225488	-2.22988333877995	0.636621979814147\\
-1.44259093182104	-2.3294916620215	0.636621979814147\\
-1.28793321620351	-2.41843503749836	0.636621979814147\\
-1.12937837012279	-2.49641833375314	0.636621979814147\\
-0.968047388648675	-2.56329558446358	0.636621979814147\\
-0.805044709329752	-2.61906529433311	0.636621979814147\\
-0.641436750799613	-2.66386165082266	0.636621979814147\\
-0.478233263291169	-2.69794235406946	0.636621979814147\\
-0.316371988356183	-2.72167389027112	0.636621979814147\\
-0.156706890183206	-2.73551511612879	0.636621979814147\\
-9.43955440460961e-16	-2.74	0.636621979814147\\
0.156706890183204	-2.73551511612879	0.636621979814147\\
0.316371988356182	-2.72167389027112	0.636621979814147\\
0.478233263291168	-2.69794235406946	0.636621979814147\\
0.641436750799612	-2.66386165082266	0.636621979814147\\
0.805044709329751	-2.61906529433311	0.636621979814147\\
0.968047388648673	-2.56329558446358	0.636621979814147\\
1.12937837012279	-2.49641833375314	0.636621979814147\\
1.28793321620351	-2.41843503749836	0.636621979814147\\
1.44259093182104	-2.3294916620215	0.636621979814147\\
1.59223751225488	-2.22988333877995	0.636621979814147\\
1.7357906541394	-2.12005443444321	0.636621979814147\\
1.87222456177917	-2.00059370944497	0.636621979814147\\
2.00059370944497	-1.87222456177917	0.636621979814147\\
2.1200544344432	-1.73579065413941	0.636621979814147\\
2.22988333877995	-1.59223751225488	0.636621979814147\\
2.3294916620215	-1.44259093182104	0.636621979814147\\
2.41843503749836	-1.28793321620351	0.636621979814147\\
2.49641833375314	-1.12937837012279	0.636621979814147\\
2.56329558446358	-0.968047388648675	0.636621979814147\\
2.61906529433311	-0.805044709329752	0.636621979814147\\
2.66386165082266	-0.641436750799614	0.636621979814147\\
2.69794235406946	-0.478233263291169	0.636621979814147\\
2.72167389027112	-0.316371988356184	0.636621979814147\\
2.73551511612879	-0.156706890183206	0.636621979814147\\
2.74	-1.11173205194415e-15	0.636621979814147\\
2.76	0	0.636621979814147\\
2.75548237975017	0.157850736096951	0.636621979814147\\
2.74154012304683	0.318681272942724	0.636621979814147\\
2.71763536395318	0.481724017037819	0.636621979814147\\
2.6833058964491	0.646118770878442	0.636621979814147\\
2.63818255925525	0.810920948083982	0.636621979814147\\
2.58200577121149	0.975113427981875	0.636621979814147\\
2.51464036538637	1.13762200786091	0.636621979814147\\
2.43608784799104	1.29733418858456	0.636621979814147\\
2.34649525079538	1.45312079263725	0.636621979814147\\
2.24615985950097	1.60385968387718	0.636621979814147\\
2.13552928432965	1.74846065891415	0.636621979814147\\
2.01519658323654	1.88589043449289	0.636621979814147\\
1.88589043449289	2.01519658323654	0.636621979814147\\
1.74846065891415	2.13552928432965	0.636621979814147\\
1.60385968387718	2.24615985950097	0.636621979814147\\
1.45312079263725	2.34649525079538	0.636621979814147\\
1.29733418858456	2.43608784799104	0.636621979814147\\
1.13762200786091	2.51464036538637	0.636621979814147\\
0.975113427981876	2.58200577121149	0.636621979814147\\
0.810920948083983	2.63818255925525	0.636621979814147\\
0.646118770878442	2.6833058964491	0.636621979814147\\
0.481724017037819	2.71763536395318	0.636621979814147\\
0.318681272942724	2.74154012304683	0.636621979814147\\
0.157850736096951	2.75548237975017	0.636621979814147\\
6.12843109593086e-16	2.76	0.636621979814147\\
-0.15785073609695	2.75548237975017	0.636621979814147\\
-0.318681272942723	2.74154012304683	0.636621979814147\\
-0.481724017037819	2.71763536395318	0.636621979814147\\
-0.646118770878441	2.68330589644911	0.636621979814147\\
-0.810920948083982	2.63818255925525	0.636621979814147\\
-0.975113427981875	2.58200577121149	0.636621979814147\\
-1.13762200786091	2.51464036538637	0.636621979814147\\
-1.29733418858456	2.43608784799104	0.636621979814147\\
-1.45312079263725	2.34649525079538	0.636621979814147\\
-1.60385968387718	2.24615985950097	0.636621979814147\\
-1.74846065891414	2.13552928432965	0.636621979814147\\
-1.88589043449289	2.01519658323654	0.636621979814147\\
-2.01519658323654	1.88589043449289	0.636621979814147\\
-2.13552928432965	1.74846065891415	0.636621979814147\\
-2.24615985950097	1.60385968387718	0.636621979814147\\
-2.34649525079538	1.45312079263725	0.636621979814147\\
-2.43608784799104	1.29733418858456	0.636621979814147\\
-2.51464036538637	1.13762200786091	0.636621979814147\\
-2.58200577121149	0.975113427981876	0.636621979814147\\
-2.63818255925525	0.810920948083983	0.636621979814147\\
-2.68330589644911	0.646118770878442	0.636621979814147\\
-2.71763536395318	0.48172401703782	0.636621979814147\\
-2.74154012304683	0.318681272942725	0.636621979814147\\
-2.75548237975017	0.157850736096952	0.636621979814147\\
-2.76	7.81844367875421e-16	0.636621979814147\\
-2.75548237975017	-0.15785073609695	0.636621979814147\\
-2.74154012304683	-0.318681272942723	0.636621979814147\\
-2.71763536395318	-0.481724017037819	0.636621979814147\\
-2.6833058964491	-0.646118770878441	0.636621979814147\\
-2.63818255925525	-0.810920948083983	0.636621979814147\\
-2.58200577121149	-0.975113427981876	0.636621979814147\\
-2.51464036538637	-1.13762200786091	0.636621979814147\\
-2.43608784799105	-1.29733418858456	0.636621979814147\\
-2.34649525079538	-1.45312079263725	0.636621979814147\\
-2.24615985950097	-1.60385968387718	0.636621979814147\\
-2.13552928432965	-1.74846065891414	0.636621979814147\\
-2.01519658323654	-1.88589043449289	0.636621979814147\\
-1.88589043449289	-2.01519658323654	0.636621979814147\\
-1.74846065891415	-2.13552928432965	0.636621979814147\\
-1.60385968387718	-2.24615985950097	0.636621979814147\\
-1.45312079263725	-2.34649525079538	0.636621979814147\\
-1.29733418858456	-2.43608784799104	0.636621979814147\\
-1.13762200786091	-2.51464036538637	0.636621979814147\\
-0.975113427981877	-2.58200577121149	0.636621979814147\\
-0.810920948083984	-2.63818255925525	0.636621979814147\\
-0.646118770878443	-2.6833058964491	0.636621979814147\\
-0.48172401703782	-2.71763536395318	0.636621979814147\\
-0.318681272942725	-2.74154012304683	0.636621979814147\\
-0.157850736096952	-2.75548237975017	0.636621979814147\\
-9.50845626157756e-16	-2.76	0.636621979814147\\
0.15785073609695	-2.75548237975017	0.636621979814147\\
0.318681272942723	-2.74154012304683	0.636621979814147\\
0.481724017037818	-2.71763536395318	0.636621979814147\\
0.646118770878441	-2.6833058964491	0.636621979814147\\
0.810920948083983	-2.63818255925525	0.636621979814147\\
0.975113427981875	-2.58200577121149	0.636621979814147\\
1.13762200786091	-2.51464036538637	0.636621979814147\\
1.29733418858456	-2.43608784799105	0.636621979814147\\
1.45312079263725	-2.34649525079538	0.636621979814147\\
1.60385968387718	-2.24615985950097	0.636621979814147\\
1.74846065891414	-2.13552928432965	0.636621979814147\\
1.88589043449289	-2.01519658323654	0.636621979814147\\
2.01519658323654	-1.88589043449289	0.636621979814147\\
2.13552928432965	-1.74846065891415	0.636621979814147\\
2.24615985950097	-1.60385968387718	0.636621979814147\\
2.34649525079538	-1.45312079263725	0.636621979814147\\
2.43608784799104	-1.29733418858456	0.636621979814147\\
2.51464036538637	-1.13762200786091	0.636621979814147\\
2.58200577121149	-0.975113427981877	0.636621979814147\\
2.63818255925525	-0.810920948083984	0.636621979814147\\
2.6833058964491	-0.646118770878443	0.636621979814147\\
2.71763536395318	-0.48172401703782	0.636621979814147\\
2.74154012304683	-0.318681272942725	0.636621979814147\\
2.75548237975017	-0.157850736096952	0.636621979814147\\
2.76	-1.11984688444009e-15	0.636621979814147\\
2.78	0	0.636621979814147\\
2.77544964337155	0.158994582010697	0.636621979814147\\
2.76140635582253	0.320990557529265	0.636621979814147\\
2.7373283738369	0.48521477078447	0.636621979814147\\
2.70275014207555	0.650800790957271	0.636621979814147\\
2.65729982417739	0.816797186838214	0.636621979814147\\
2.6007159579594	0.982179467315077	0.636621979814147\\
2.53286239701961	1.14586564559903	0.636621979814147\\
2.45374065848373	1.30673516096561	0.636621979814147\\
2.36349883956926	1.46365065345347	0.636621979814147\\
2.26243638022199	1.61548185549947	0.636621979814147\\
2.1510041342161	1.76113066368889	0.636621979814147\\
2.02979945702811	1.8995563072066	0.636621979814147\\
1.8995563072066	2.02979945702811	0.636621979814147\\
1.76113066368889	2.1510041342161	0.636621979814147\\
1.61548185549948	2.26243638022199	0.636621979814147\\
1.46365065345347	2.36349883956926	0.636621979814147\\
1.30673516096561	2.45374065848373	0.636621979814147\\
1.14586564559903	2.53286239701961	0.636621979814147\\
0.982179467315078	2.6007159579594	0.636621979814147\\
0.816797186838215	2.65729982417739	0.636621979814147\\
0.650800790957272	2.70275014207555	0.636621979814147\\
0.48521477078447	2.7373283738369	0.636621979814147\\
0.320990557529266	2.76140635582253	0.636621979814147\\
0.158994582010698	2.77544964337155	0.636621979814147\\
6.17284001691587e-16	2.78	0.636621979814147\\
-0.158994582010696	2.77544964337155	0.636621979814147\\
-0.320990557529265	2.76140635582253	0.636621979814147\\
-0.48521477078447	2.7373283738369	0.636621979814147\\
-0.65080079095727	2.70275014207555	0.636621979814147\\
-0.816797186838214	2.65729982417739	0.636621979814147\\
-0.982179467315077	2.6007159579594	0.636621979814147\\
-1.14586564559903	2.53286239701961	0.636621979814147\\
-1.30673516096561	2.45374065848373	0.636621979814147\\
-1.46365065345347	2.36349883956926	0.636621979814147\\
-1.61548185549947	2.26243638022199	0.636621979814147\\
-1.76113066368889	2.1510041342161	0.636621979814147\\
-1.8995563072066	2.02979945702811	0.636621979814147\\
-2.02979945702811	1.8995563072066	0.636621979814147\\
-2.1510041342161	1.76113066368889	0.636621979814147\\
-2.26243638022199	1.61548185549947	0.636621979814147\\
-2.36349883956926	1.46365065345347	0.636621979814147\\
-2.45374065848373	1.30673516096561	0.636621979814147\\
-2.53286239701961	1.14586564559903	0.636621979814147\\
-2.6007159579594	0.982179467315078	0.636621979814147\\
-2.65729982417739	0.816797186838215	0.636621979814147\\
-2.70275014207555	0.650800790957271	0.636621979814147\\
-2.7373283738369	0.485214770784471	0.636621979814147\\
-2.76140635582253	0.320990557529266	0.636621979814147\\
-2.77544964337155	0.158994582010698	0.636621979814147\\
-2.78	7.87509906773069e-16	0.636621979814147\\
-2.77544964337155	-0.158994582010696	0.636621979814147\\
-2.76140635582253	-0.320990557529265	0.636621979814147\\
-2.7373283738369	-0.485214770784469	0.636621979814147\\
-2.70275014207555	-0.650800790957271	0.636621979814147\\
-2.65729982417739	-0.816797186838214	0.636621979814147\\
-2.6007159579594	-0.982179467315078	0.636621979814147\\
-2.53286239701961	-1.14586564559903	0.636621979814147\\
-2.45374065848373	-1.30673516096561	0.636621979814147\\
-2.36349883956926	-1.46365065345347	0.636621979814147\\
-2.26243638022199	-1.61548185549947	0.636621979814147\\
-2.1510041342161	-1.76113066368889	0.636621979814147\\
-2.02979945702811	-1.8995563072066	0.636621979814147\\
-1.8995563072066	-2.02979945702811	0.636621979814147\\
-1.76113066368889	-2.1510041342161	0.636621979814147\\
-1.61548185549947	-2.26243638022199	0.636621979814147\\
-1.46365065345347	-2.36349883956926	0.636621979814147\\
-1.30673516096561	-2.45374065848373	0.636621979814147\\
-1.14586564559903	-2.53286239701961	0.636621979814147\\
-0.982179467315079	-2.6007159579594	0.636621979814147\\
-0.816797186838216	-2.65729982417739	0.636621979814147\\
-0.650800790957272	-2.70275014207555	0.636621979814147\\
-0.485214770784471	-2.7373283738369	0.636621979814147\\
-0.320990557529266	-2.76140635582253	0.636621979814147\\
-0.158994582010698	-2.77544964337155	0.636621979814147\\
-9.57735811854551e-16	-2.78	0.636621979814147\\
0.158994582010696	-2.77544964337155	0.636621979814147\\
0.320990557529265	-2.76140635582253	0.636621979814147\\
0.485214770784469	-2.7373283738369	0.636621979814147\\
0.650800790957271	-2.70275014207555	0.636621979814147\\
0.816797186838214	-2.65729982417739	0.636621979814147\\
0.982179467315077	-2.6007159579594	0.636621979814147\\
1.14586564559903	-2.53286239701961	0.636621979814147\\
1.30673516096561	-2.45374065848373	0.636621979814147\\
1.46365065345347	-2.36349883956926	0.636621979814147\\
1.61548185549947	-2.26243638022199	0.636621979814147\\
1.76113066368889	-2.1510041342161	0.636621979814147\\
1.8995563072066	-2.02979945702811	0.636621979814147\\
2.02979945702811	-1.8995563072066	0.636621979814147\\
2.1510041342161	-1.76113066368889	0.636621979814147\\
2.26243638022199	-1.61548185549948	0.636621979814147\\
2.36349883956926	-1.46365065345347	0.636621979814147\\
2.45374065848373	-1.30673516096561	0.636621979814147\\
2.53286239701961	-1.14586564559903	0.636621979814147\\
2.6007159579594	-0.982179467315079	0.636621979814147\\
2.65729982417739	-0.816797186838216	0.636621979814147\\
2.70275014207555	-0.650800790957272	0.636621979814147\\
2.7373283738369	-0.485214770784471	0.636621979814147\\
2.76140635582253	-0.320990557529267	0.636621979814147\\
2.77544964337155	-0.158994582010698	0.636621979814147\\
2.78	-1.12796171693603e-15	0.636621979814147\\
2.8	0	0.636621979814147\\
2.79541690699292	0.160138427924443	0.636621979814147\\
2.78127258859823	0.323299842115807	0.636621979814147\\
2.75702138372062	0.488705524531121	0.636621979814147\\
2.72219438770199	0.6554828110361	0.636621979814147\\
2.67641708909953	0.822673425592446	0.636621979814147\\
2.61942614470731	0.989245506648279	0.636621979814147\\
2.55108442865284	1.15410928333716	0.636621979814147\\
2.47139346897642	1.31613613334666	0.636621979814147\\
2.38050242834314	1.47418051426968	0.636621979814147\\
2.27871290094301	1.62710402712177	0.636621979814147\\
2.16647898410254	1.77380066846363	0.636621979814147\\
2.04440233081968	1.91322217992032	0.636621979814147\\
1.91322217992032	2.04440233081968	0.636621979814147\\
1.77380066846363	2.16647898410254	0.636621979814147\\
1.62710402712177	2.27871290094301	0.636621979814147\\
1.47418051426968	2.38050242834314	0.636621979814147\\
1.31613613334666	2.47139346897642	0.636621979814147\\
1.15410928333716	2.55108442865284	0.636621979814147\\
0.98924550664828	2.61942614470731	0.636621979814147\\
0.822673425592446	2.67641708909953	0.636621979814147\\
0.655482811036101	2.72219438770199	0.636621979814147\\
0.488705524531121	2.75702138372062	0.636621979814147\\
0.323299842115807	2.78127258859823	0.636621979814147\\
0.160138427924444	2.79541690699292	0.636621979814147\\
6.21724893790088e-16	2.8	0.636621979814147\\
-0.160138427924442	2.79541690699292	0.636621979814147\\
-0.323299842115806	2.78127258859823	0.636621979814147\\
-0.48870552453112	2.75702138372062	0.636621979814147\\
-0.655482811036099	2.72219438770199	0.636621979814147\\
-0.822673425592445	2.67641708909953	0.636621979814147\\
-0.989245506648279	2.61942614470731	0.636621979814147\\
-1.15410928333716	2.55108442865284	0.636621979814147\\
-1.31613613334666	2.47139346897642	0.636621979814147\\
-1.47418051426968	2.38050242834314	0.636621979814147\\
-1.62710402712177	2.27871290094301	0.636621979814147\\
-1.77380066846363	2.16647898410254	0.636621979814147\\
-1.91322217992032	2.04440233081968	0.636621979814147\\
-2.04440233081968	1.91322217992032	0.636621979814147\\
-2.16647898410254	1.77380066846363	0.636621979814147\\
-2.27871290094301	1.62710402712177	0.636621979814147\\
-2.38050242834314	1.47418051426968	0.636621979814147\\
-2.47139346897642	1.31613613334666	0.636621979814147\\
-2.55108442865284	1.15410928333716	0.636621979814147\\
-2.61942614470731	0.98924550664828	0.636621979814147\\
-2.67641708909953	0.822673425592446	0.636621979814147\\
-2.72219438770199	0.6554828110361	0.636621979814147\\
-2.75702138372062	0.488705524531122	0.636621979814147\\
-2.78127258859823	0.323299842115808	0.636621979814147\\
-2.79541690699292	0.160138427924444	0.636621979814147\\
-2.8	7.93175445670717e-16	0.636621979814147\\
-2.79541690699292	-0.160138427924442	0.636621979814147\\
-2.78127258859823	-0.323299842115806	0.636621979814147\\
-2.75702138372062	-0.48870552453112	0.636621979814147\\
-2.72219438770199	-0.6554828110361	0.636621979814147\\
-2.67641708909953	-0.822673425592446	0.636621979814147\\
-2.61942614470731	-0.98924550664828	0.636621979814147\\
-2.55108442865284	-1.15410928333715	0.636621979814147\\
-2.47139346897642	-1.31613613334665	0.636621979814147\\
-2.38050242834314	-1.47418051426968	0.636621979814147\\
-2.27871290094301	-1.62710402712177	0.636621979814147\\
-2.16647898410254	-1.77380066846363	0.636621979814147\\
-2.04440233081968	-1.91322217992032	0.636621979814147\\
-1.91322217992032	-2.04440233081968	0.636621979814147\\
-1.77380066846363	-2.16647898410254	0.636621979814147\\
-1.62710402712177	-2.27871290094301	0.636621979814147\\
-1.47418051426968	-2.38050242834314	0.636621979814147\\
-1.31613613334666	-2.47139346897642	0.636621979814147\\
-1.15410928333716	-2.55108442865284	0.636621979814147\\
-0.989245506648281	-2.61942614470731	0.636621979814147\\
-0.822673425592448	-2.67641708909953	0.636621979814147\\
-0.655482811036102	-2.72219438770199	0.636621979814147\\
-0.488705524531122	-2.75702138372062	0.636621979814147\\
-0.323299842115808	-2.78127258859823	0.636621979814147\\
-0.160138427924444	-2.79541690699292	0.636621979814147\\
-9.64625997551347e-16	-2.8	0.636621979814147\\
0.160138427924442	-2.79541690699292	0.636621979814147\\
0.323299842115806	-2.78127258859823	0.636621979814147\\
0.48870552453112	-2.75702138372062	0.636621979814147\\
0.6554828110361	-2.72219438770199	0.636621979814147\\
0.822673425592446	-2.67641708909953	0.636621979814147\\
0.989245506648279	-2.61942614470731	0.636621979814147\\
1.15410928333715	-2.55108442865284	0.636621979814147\\
1.31613613334665	-2.47139346897642	0.636621979814147\\
1.47418051426968	-2.38050242834314	0.636621979814147\\
1.62710402712177	-2.27871290094301	0.636621979814147\\
1.77380066846363	-2.16647898410254	0.636621979814147\\
1.91322217992032	-2.04440233081968	0.636621979814147\\
2.04440233081968	-1.91322217992032	0.636621979814147\\
2.16647898410254	-1.77380066846363	0.636621979814147\\
2.27871290094301	-1.62710402712177	0.636621979814147\\
2.38050242834314	-1.47418051426968	0.636621979814147\\
2.47139346897642	-1.31613613334666	0.636621979814147\\
2.55108442865284	-1.15410928333716	0.636621979814147\\
2.61942614470731	-0.989245506648281	0.636621979814147\\
2.67641708909953	-0.822673425592448	0.636621979814147\\
2.72219438770199	-0.655482811036102	0.636621979814147\\
2.75702138372062	-0.488705524531122	0.636621979814147\\
2.78127258859823	-0.323299842115808	0.636621979814147\\
2.79541690699292	-0.160138427924444	0.636621979814147\\
2.8	-1.13607654943198e-15	0.636621979814147\\
2.82	0	0.636621979814147\\
2.8153841706143	0.161282273838189	0.636621979814147\\
2.80113882137393	0.325609126702348	0.636621979814147\\
2.77671439360434	0.492196278277771	0.636621979814147\\
2.74163863332843	0.660164831114929	0.636621979814147\\
2.69553435402167	0.828549664346678	0.636621979814147\\
2.63813633145522	0.996311545981481	0.636621979814147\\
2.56930646028608	1.16235292107528	0.636621979814147\\
2.48904627946911	1.3255371057277	0.636621979814147\\
2.39750601711702	1.48471037508589	0.636621979814147\\
2.29498942166403	1.63872619874407	0.636621979814147\\
2.18195383398899	1.78647067323837	0.636621979814147\\
2.05900520461125	1.92688805263404	0.636621979814147\\
1.92688805263404	2.05900520461125	0.636621979814147\\
1.78647067323837	2.18195383398899	0.636621979814147\\
1.63872619874407	2.29498942166403	0.636621979814147\\
1.48471037508589	2.39750601711702	0.636621979814147\\
1.3255371057277	2.48904627946911	0.636621979814147\\
1.16235292107528	2.56930646028608	0.636621979814147\\
0.996311545981482	2.63813633145522	0.636621979814147\\
0.828549664346678	2.69553435402167	0.636621979814147\\
0.66016483111493	2.74163863332843	0.636621979814147\\
0.492196278277772	2.77671439360434	0.636621979814147\\
0.325609126702349	2.80113882137393	0.636621979814147\\
0.16128227383819	2.8153841706143	0.636621979814147\\
6.26165785888588e-16	2.82	0.636621979814147\\
-0.161282273838188	2.8153841706143	0.636621979814147\\
-0.325609126702348	2.80113882137393	0.636621979814147\\
-0.492196278277771	2.77671439360434	0.636621979814147\\
-0.660164831114928	2.74163863332843	0.636621979814147\\
-0.828549664346677	2.69553435402167	0.636621979814147\\
-0.996311545981481	2.63813633145522	0.636621979814147\\
-1.16235292107528	2.56930646028608	0.636621979814147\\
-1.3255371057277	2.48904627946911	0.636621979814147\\
-1.48471037508589	2.39750601711702	0.636621979814147\\
-1.63872619874407	2.29498942166403	0.636621979814147\\
-1.78647067323837	2.18195383398899	0.636621979814147\\
-1.92688805263404	2.05900520461125	0.636621979814147\\
-2.05900520461125	1.92688805263404	0.636621979814147\\
-2.18195383398899	1.78647067323837	0.636621979814147\\
-2.29498942166403	1.63872619874407	0.636621979814147\\
-2.39750601711702	1.48471037508589	0.636621979814147\\
-2.48904627946911	1.3255371057277	0.636621979814147\\
-2.56930646028608	1.16235292107528	0.636621979814147\\
-2.63813633145522	0.996311545981482	0.636621979814147\\
-2.69553435402167	0.828549664346678	0.636621979814147\\
-2.74163863332843	0.66016483111493	0.636621979814147\\
-2.77671439360434	0.492196278277773	0.636621979814147\\
-2.80113882137393	0.325609126702349	0.636621979814147\\
-2.8153841706143	0.16128227383819	0.636621979814147\\
-2.82	7.98840984568365e-16	0.636621979814147\\
-2.8153841706143	-0.161282273838188	0.636621979814147\\
-2.80113882137393	-0.325609126702348	0.636621979814147\\
-2.77671439360434	-0.492196278277771	0.636621979814147\\
-2.74163863332843	-0.660164831114929	0.636621979814147\\
-2.69553435402167	-0.828549664346678	0.636621979814147\\
-2.63813633145522	-0.996311545981482	0.636621979814147\\
-2.56930646028608	-1.16235292107528	0.636621979814147\\
-2.48904627946911	-1.3255371057277	0.636621979814147\\
-2.39750601711702	-1.48471037508589	0.636621979814147\\
-2.29498942166403	-1.63872619874407	0.636621979814147\\
-2.18195383398899	-1.78647067323837	0.636621979814147\\
-2.05900520461125	-1.92688805263404	0.636621979814147\\
-1.92688805263404	-2.05900520461125	0.636621979814147\\
-1.78647067323837	-2.18195383398899	0.636621979814147\\
-1.63872619874407	-2.29498942166403	0.636621979814147\\
-1.48471037508589	-2.39750601711702	0.636621979814147\\
-1.3255371057277	-2.48904627946911	0.636621979814147\\
-1.16235292107528	-2.56930646028608	0.636621979814147\\
-0.996311545981483	-2.63813633145522	0.636621979814147\\
-0.828549664346679	-2.69553435402167	0.636621979814147\\
-0.660164831114931	-2.74163863332843	0.636621979814147\\
-0.492196278277773	-2.77671439360434	0.636621979814147\\
-0.325609126702349	-2.80113882137393	0.636621979814147\\
-0.16128227383819	-2.8153841706143	0.636621979814147\\
-9.71516183248142e-16	-2.82	0.636621979814147\\
0.161282273838188	-2.8153841706143	0.636621979814147\\
0.325609126702348	-2.80113882137393	0.636621979814147\\
0.492196278277771	-2.77671439360434	0.636621979814147\\
0.660164831114929	-2.74163863332843	0.636621979814147\\
0.828549664346678	-2.69553435402167	0.636621979814147\\
0.996311545981481	-2.63813633145522	0.636621979814147\\
1.16235292107528	-2.56930646028608	0.636621979814147\\
1.3255371057277	-2.48904627946911	0.636621979814147\\
1.48471037508589	-2.39750601711702	0.636621979814147\\
1.63872619874407	-2.29498942166403	0.636621979814147\\
1.78647067323837	-2.18195383398899	0.636621979814147\\
1.92688805263404	-2.05900520461125	0.636621979814147\\
2.05900520461125	-1.92688805263404	0.636621979814147\\
2.18195383398899	-1.78647067323837	0.636621979814147\\
2.29498942166403	-1.63872619874407	0.636621979814147\\
2.39750601711702	-1.48471037508589	0.636621979814147\\
2.48904627946911	-1.3255371057277	0.636621979814147\\
2.56930646028608	-1.16235292107528	0.636621979814147\\
2.63813633145522	-0.996311545981483	0.636621979814147\\
2.69553435402167	-0.828549664346679	0.636621979814147\\
2.74163863332843	-0.660164831114931	0.636621979814147\\
2.77671439360434	-0.492196278277773	0.636621979814147\\
2.80113882137393	-0.325609126702349	0.636621979814147\\
2.8153841706143	-0.16128227383819	0.636621979814147\\
2.82	-1.14419138192792e-15	0.636621979814147\\
2.84	0	0.636621979814147\\
2.83535143423568	0.162426119751935	0.636621979814147\\
2.82100505414963	0.32791841128889	0.636621979814147\\
2.79640740348806	0.495687032024422	0.636621979814147\\
2.76108287895488	0.664846851193759	0.636621979814147\\
2.71465161894381	0.834425903100909	0.636621979814147\\
2.65684651820313	1.00337758531468	0.636621979814147\\
2.58752849191931	1.1705965588134	0.636621979814147\\
2.5066990899618	1.33493807810875	0.636621979814147\\
2.4145096058909	1.4952402359021	0.636621979814147\\
2.31126594238505	1.65034837036637	0.636621979814147\\
2.19742868387544	1.79914067801311	0.636621979814147\\
2.07360807840282	1.94055392534775	0.636621979814147\\
1.94055392534775	2.07360807840282	0.636621979814147\\
1.79914067801311	2.19742868387544	0.636621979814147\\
1.65034837036637	2.31126594238505	0.636621979814147\\
1.4952402359021	2.4145096058909	0.636621979814147\\
1.33493807810875	2.5066990899618	0.636621979814147\\
1.1705965588134	2.58752849191931	0.636621979814147\\
1.00337758531468	2.65684651820313	0.636621979814147\\
0.83442590310091	2.71465161894381	0.636621979814147\\
0.664846851193759	2.76108287895488	0.636621979814147\\
0.495687032024423	2.79640740348806	0.636621979814147\\
0.32791841128889	2.82100505414963	0.636621979814147\\
0.162426119751936	2.83535143423568	0.636621979814147\\
6.30606677987089e-16	2.84	0.636621979814147\\
-0.162426119751934	2.83535143423568	0.636621979814147\\
-0.327918411288889	2.82100505414963	0.636621979814147\\
-0.495687032024422	2.79640740348806	0.636621979814147\\
-0.664846851193758	2.76108287895488	0.636621979814147\\
-0.834425903100909	2.71465161894381	0.636621979814147\\
-1.00337758531468	2.65684651820313	0.636621979814147\\
-1.1705965588134	2.58752849191931	0.636621979814147\\
-1.33493807810875	2.5066990899618	0.636621979814147\\
-1.4952402359021	2.4145096058909	0.636621979814147\\
-1.65034837036637	2.31126594238505	0.636621979814147\\
-1.79914067801311	2.19742868387544	0.636621979814147\\
-1.94055392534775	2.07360807840282	0.636621979814147\\
-2.07360807840282	1.94055392534775	0.636621979814147\\
-2.19742868387544	1.79914067801311	0.636621979814147\\
-2.31126594238505	1.65034837036637	0.636621979814147\\
-2.4145096058909	1.4952402359021	0.636621979814147\\
-2.5066990899618	1.33493807810875	0.636621979814147\\
-2.58752849191931	1.1705965588134	0.636621979814147\\
-2.65684651820313	1.00337758531468	0.636621979814147\\
-2.71465161894381	0.83442590310091	0.636621979814147\\
-2.76108287895488	0.664846851193759	0.636621979814147\\
-2.79640740348806	0.495687032024423	0.636621979814147\\
-2.82100505414963	0.327918411288891	0.636621979814147\\
-2.83535143423568	0.162426119751936	0.636621979814147\\
-2.84	8.04506523466013e-16	0.636621979814147\\
-2.83535143423568	-0.162426119751934	0.636621979814147\\
-2.82100505414963	-0.327918411288889	0.636621979814147\\
-2.79640740348806	-0.495687032024422	0.636621979814147\\
-2.76108287895488	-0.664846851193759	0.636621979814147\\
-2.71465161894381	-0.83442590310091	0.636621979814147\\
-2.65684651820313	-1.00337758531468	0.636621979814147\\
-2.58752849191931	-1.1705965588134	0.636621979814147\\
-2.5066990899618	-1.33493807810875	0.636621979814147\\
-2.4145096058909	-1.4952402359021	0.636621979814147\\
-2.31126594238505	-1.65034837036637	0.636621979814147\\
-2.19742868387544	-1.79914067801311	0.636621979814147\\
-2.07360807840282	-1.94055392534775	0.636621979814147\\
-1.94055392534775	-2.07360807840282	0.636621979814147\\
-1.79914067801311	-2.19742868387544	0.636621979814147\\
-1.65034837036637	-2.31126594238505	0.636621979814147\\
-1.4952402359021	-2.4145096058909	0.636621979814147\\
-1.33493807810875	-2.5066990899618	0.636621979814147\\
-1.1705965588134	-2.58752849191931	0.636621979814147\\
-1.00337758531469	-2.65684651820313	0.636621979814147\\
-0.834425903100911	-2.71465161894381	0.636621979814147\\
-0.66484685119376	-2.76108287895488	0.636621979814147\\
-0.495687032024424	-2.79640740348806	0.636621979814147\\
-0.327918411288891	-2.82100505414963	0.636621979814147\\
-0.162426119751936	-2.83535143423568	0.636621979814147\\
-9.78406368944937e-16	-2.84	0.636621979814147\\
0.162426119751934	-2.83535143423568	0.636621979814147\\
0.327918411288889	-2.82100505414963	0.636621979814147\\
0.495687032024422	-2.79640740348806	0.636621979814147\\
0.664846851193758	-2.76108287895488	0.636621979814147\\
0.83442590310091	-2.71465161894381	0.636621979814147\\
1.00337758531468	-2.65684651820313	0.636621979814147\\
1.1705965588134	-2.58752849191931	0.636621979814147\\
1.33493807810875	-2.5066990899618	0.636621979814147\\
1.4952402359021	-2.4145096058909	0.636621979814147\\
1.65034837036637	-2.31126594238505	0.636621979814147\\
1.79914067801311	-2.19742868387544	0.636621979814147\\
1.94055392534775	-2.07360807840282	0.636621979814147\\
2.07360807840282	-1.94055392534775	0.636621979814147\\
2.19742868387544	-1.79914067801311	0.636621979814147\\
2.31126594238505	-1.65034837036637	0.636621979814147\\
2.4145096058909	-1.4952402359021	0.636621979814147\\
2.5066990899618	-1.33493807810875	0.636621979814147\\
2.58752849191931	-1.1705965588134	0.636621979814147\\
2.65684651820313	-1.00337758531469	0.636621979814147\\
2.71465161894381	-0.834425903100911	0.636621979814147\\
2.76108287895488	-0.66484685119376	0.636621979814147\\
2.79640740348806	-0.495687032024424	0.636621979814147\\
2.82100505414963	-0.327918411288891	0.636621979814147\\
2.83535143423568	-0.162426119751936	0.636621979814147\\
2.84	-1.15230621442386e-15	0.636621979814147\\
2.86	0	0.636621979814147\\
2.85531869785706	0.163569965665681	0.636621979814147\\
2.84087128692533	0.330227695875431	0.636621979814147\\
2.81610041337178	0.499177785771073	0.636621979814147\\
2.78052712458132	0.669528871272588	0.636621979814147\\
2.73376888386595	0.840302141855141	0.636621979814147\\
2.67555670495104	1.01044362464789	0.636621979814147\\
2.60575052355255	1.17884019655152	0.636621979814147\\
2.52435190045449	1.3443390504898	0.636621979814147\\
2.43151319466478	1.50577009671831	0.636621979814147\\
2.32754246310607	1.66197054198867	0.636621979814147\\
2.21290353376188	1.81181068278785	0.636621979814147\\
2.08821095219439	1.95421979806147	0.636621979814147\\
1.95421979806147	2.08821095219439	0.636621979814147\\
1.81181068278785	2.21290353376188	0.636621979814147\\
1.66197054198867	2.32754246310607	0.636621979814147\\
1.50577009671831	2.43151319466478	0.636621979814147\\
1.3443390504898	2.52435190045449	0.636621979814147\\
1.17884019655152	2.60575052355255	0.636621979814147\\
1.01044362464789	2.67555670495104	0.636621979814147\\
0.840302141855142	2.73376888386595	0.636621979814147\\
0.669528871272589	2.78052712458132	0.636621979814147\\
0.499177785771074	2.81610041337178	0.636621979814147\\
0.330227695875432	2.84087128692533	0.636621979814147\\
0.163569965665682	2.85531869785706	0.636621979814147\\
6.3504757008559e-16	2.86	0.636621979814147\\
-0.16356996566568	2.85531869785706	0.636621979814147\\
-0.330227695875431	2.84087128692533	0.636621979814147\\
-0.499177785771073	2.81610041337178	0.636621979814147\\
-0.669528871272587	2.78052712458132	0.636621979814147\\
-0.840302141855141	2.73376888386595	0.636621979814147\\
-1.01044362464789	2.67555670495104	0.636621979814147\\
-1.17884019655152	2.60575052355255	0.636621979814147\\
-1.3443390504898	2.52435190045449	0.636621979814147\\
-1.50577009671831	2.43151319466478	0.636621979814147\\
-1.66197054198867	2.32754246310608	0.636621979814147\\
-1.81181068278785	2.21290353376188	0.636621979814147\\
-1.95421979806147	2.08821095219439	0.636621979814147\\
-2.08821095219439	1.95421979806147	0.636621979814147\\
-2.21290353376188	1.81181068278785	0.636621979814147\\
-2.32754246310607	1.66197054198867	0.636621979814147\\
-2.43151319466478	1.50577009671831	0.636621979814147\\
-2.52435190045449	1.3443390504898	0.636621979814147\\
-2.60575052355255	1.17884019655152	0.636621979814147\\
-2.67555670495104	1.01044362464789	0.636621979814147\\
-2.73376888386595	0.840302141855142	0.636621979814147\\
-2.78052712458132	0.669528871272588	0.636621979814147\\
-2.81610041337178	0.499177785771074	0.636621979814147\\
-2.84087128692533	0.330227695875432	0.636621979814147\\
-2.85531869785706	0.163569965665682	0.636621979814147\\
-2.86	8.10172062363661e-16	0.636621979814147\\
-2.85531869785706	-0.16356996566568	0.636621979814147\\
-2.84087128692533	-0.330227695875431	0.636621979814147\\
-2.81610041337178	-0.499177785771073	0.636621979814147\\
-2.78052712458132	-0.669528871272588	0.636621979814147\\
-2.73376888386595	-0.840302141855141	0.636621979814147\\
-2.67555670495104	-1.01044362464789	0.636621979814147\\
-2.60575052355255	-1.17884019655152	0.636621979814147\\
-2.52435190045449	-1.3443390504898	0.636621979814147\\
-2.43151319466478	-1.50577009671831	0.636621979814147\\
-2.32754246310608	-1.66197054198867	0.636621979814147\\
-2.21290353376189	-1.81181068278785	0.636621979814147\\
-2.08821095219439	-1.95421979806147	0.636621979814147\\
-1.95421979806147	-2.08821095219439	0.636621979814147\\
-1.81181068278785	-2.21290353376188	0.636621979814147\\
-1.66197054198867	-2.32754246310607	0.636621979814147\\
-1.50577009671831	-2.43151319466478	0.636621979814147\\
-1.3443390504898	-2.52435190045449	0.636621979814147\\
-1.17884019655152	-2.60575052355255	0.636621979814147\\
-1.01044362464789	-2.67555670495104	0.636621979814147\\
-0.840302141855143	-2.73376888386595	0.636621979814147\\
-0.669528871272589	-2.78052712458132	0.636621979814147\\
-0.499177785771074	-2.81610041337178	0.636621979814147\\
-0.330227695875432	-2.84087128692533	0.636621979814147\\
-0.163569965665682	-2.85531869785706	0.636621979814147\\
-9.85296554641733e-16	-2.86	0.636621979814147\\
0.16356996566568	-2.85531869785706	0.636621979814147\\
0.330227695875431	-2.84087128692533	0.636621979814147\\
0.499177785771073	-2.81610041337178	0.636621979814147\\
0.669528871272588	-2.78052712458132	0.636621979814147\\
0.840302141855141	-2.73376888386595	0.636621979814147\\
1.01044362464789	-2.67555670495104	0.636621979814147\\
1.17884019655152	-2.60575052355255	0.636621979814147\\
1.3443390504898	-2.52435190045449	0.636621979814147\\
1.50577009671831	-2.43151319466478	0.636621979814147\\
1.66197054198867	-2.32754246310608	0.636621979814147\\
1.81181068278785	-2.21290353376189	0.636621979814147\\
1.95421979806147	-2.08821095219439	0.636621979814147\\
2.08821095219439	-1.95421979806147	0.636621979814147\\
2.21290353376188	-1.81181068278785	0.636621979814147\\
2.32754246310607	-1.66197054198867	0.636621979814147\\
2.43151319466478	-1.50577009671831	0.636621979814147\\
2.52435190045449	-1.3443390504898	0.636621979814147\\
2.60575052355255	-1.17884019655152	0.636621979814147\\
2.67555670495104	-1.01044362464789	0.636621979814147\\
2.73376888386595	-0.840302141855143	0.636621979814147\\
2.78052712458132	-0.669528871272589	0.636621979814147\\
2.81610041337178	-0.499177785771075	0.636621979814147\\
2.84087128692533	-0.330227695875432	0.636621979814147\\
2.85531869785706	-0.163569965665682	0.636621979814147\\
2.86	-1.1604210469198e-15	0.636621979814147\\
2.88	0	0.636621979814147\\
2.87528596147844	0.164713811579427	0.636621979814147\\
2.86073751970104	0.332536980461973	0.636621979814147\\
2.83579342325549	0.502668539517724	0.636621979814147\\
2.79997137020776	0.674210891351417	0.636621979814147\\
2.75288614878809	0.846178380609373	0.636621979814147\\
2.69426689169895	1.01750966398109	0.636621979814147\\
2.62397255518578	1.18708383428965	0.636621979814147\\
2.54200471094718	1.35374002287085	0.636621979814147\\
2.44851678343866	1.51629995753453	0.636621979814147\\
2.3438189838271	1.67359271361097	0.636621979814147\\
2.22837838364833	1.82448068756259	0.636621979814147\\
2.10281382598596	1.96788567077519	0.636621979814147\\
1.96788567077519	2.10281382598596	0.636621979814147\\
1.82448068756259	2.22837838364833	0.636621979814147\\
1.67359271361097	2.3438189838271	0.636621979814147\\
1.51629995753453	2.44851678343866	0.636621979814147\\
1.35374002287085	2.54200471094718	0.636621979814147\\
1.18708383428965	2.62397255518578	0.636621979814147\\
1.01750966398109	2.69426689169894	0.636621979814147\\
0.846178380609374	2.75288614878809	0.636621979814147\\
0.674210891351418	2.79997137020776	0.636621979814147\\
0.502668539517725	2.83579342325549	0.636621979814147\\
0.332536980461973	2.86073751970104	0.636621979814147\\
0.164713811579428	2.87528596147844	0.636621979814147\\
6.3948846218409e-16	2.88	0.636621979814147\\
-0.164713811579426	2.87528596147844	0.636621979814147\\
-0.332536980461972	2.86073751970104	0.636621979814147\\
-0.502668539517724	2.83579342325549	0.636621979814147\\
-0.674210891351416	2.79997137020776	0.636621979814147\\
-0.846178380609372	2.75288614878809	0.636621979814147\\
-1.01750966398109	2.69426689169895	0.636621979814147\\
-1.18708383428965	2.62397255518578	0.636621979814147\\
-1.35374002287085	2.54200471094718	0.636621979814147\\
-1.51629995753453	2.44851678343866	0.636621979814147\\
-1.67359271361097	2.3438189838271	0.636621979814147\\
-1.82448068756259	2.22837838364833	0.636621979814147\\
-1.96788567077519	2.10281382598596	0.636621979814147\\
-2.10281382598596	1.96788567077519	0.636621979814147\\
-2.22837838364833	1.82448068756259	0.636621979814147\\
-2.3438189838271	1.67359271361097	0.636621979814147\\
-2.44851678343866	1.51629995753453	0.636621979814147\\
-2.54200471094718	1.35374002287085	0.636621979814147\\
-2.62397255518578	1.18708383428965	0.636621979814147\\
-2.69426689169894	1.01750966398109	0.636621979814147\\
-2.75288614878809	0.846178380609374	0.636621979814147\\
-2.79997137020776	0.674210891351417	0.636621979814147\\
-2.83579342325549	0.502668539517725	0.636621979814147\\
-2.86073751970104	0.332536980461974	0.636621979814147\\
-2.87528596147844	0.164713811579428	0.636621979814147\\
-2.88	8.15837601261309e-16	0.636621979814147\\
-2.87528596147844	-0.164713811579426	0.636621979814147\\
-2.86073751970104	-0.332536980461972	0.636621979814147\\
-2.83579342325549	-0.502668539517724	0.636621979814147\\
-2.79997137020776	-0.674210891351417	0.636621979814147\\
-2.75288614878809	-0.846178380609373	0.636621979814147\\
-2.69426689169894	-1.01750966398109	0.636621979814147\\
-2.62397255518578	-1.18708383428964	0.636621979814147\\
-2.54200471094718	-1.35374002287084	0.636621979814147\\
-2.44851678343866	-1.51629995753452	0.636621979814147\\
-2.3438189838271	-1.67359271361097	0.636621979814147\\
-2.22837838364833	-1.82448068756259	0.636621979814147\\
-2.10281382598596	-1.96788567077519	0.636621979814147\\
-1.96788567077519	-2.10281382598596	0.636621979814147\\
-1.82448068756259	-2.22837838364833	0.636621979814147\\
-1.67359271361097	-2.3438189838271	0.636621979814147\\
-1.51629995753453	-2.44851678343866	0.636621979814147\\
-1.35374002287085	-2.54200471094718	0.636621979814147\\
-1.18708383428965	-2.62397255518578	0.636621979814147\\
-1.01750966398109	-2.69426689169894	0.636621979814147\\
-0.846178380609374	-2.75288614878809	0.636621979814147\\
-0.674210891351419	-2.79997137020776	0.636621979814147\\
-0.502668539517725	-2.83579342325549	0.636621979814147\\
-0.332536980461974	-2.86073751970104	0.636621979814147\\
-0.164713811579428	-2.87528596147844	0.636621979814147\\
-9.92186740338528e-16	-2.88	0.636621979814147\\
0.164713811579426	-2.87528596147844	0.636621979814147\\
0.332536980461972	-2.86073751970104	0.636621979814147\\
0.502668539517724	-2.83579342325549	0.636621979814147\\
0.674210891351417	-2.79997137020776	0.636621979814147\\
0.846178380609373	-2.75288614878809	0.636621979814147\\
1.01750966398109	-2.69426689169894	0.636621979814147\\
1.18708383428964	-2.62397255518578	0.636621979814147\\
1.35374002287084	-2.54200471094718	0.636621979814147\\
1.51629995753452	-2.44851678343866	0.636621979814147\\
1.67359271361097	-2.3438189838271	0.636621979814147\\
1.82448068756259	-2.22837838364833	0.636621979814147\\
1.96788567077518	-2.10281382598596	0.636621979814147\\
2.10281382598596	-1.96788567077519	0.636621979814147\\
2.22837838364833	-1.82448068756259	0.636621979814147\\
2.3438189838271	-1.67359271361097	0.636621979814147\\
2.44851678343866	-1.51629995753453	0.636621979814147\\
2.54200471094718	-1.35374002287085	0.636621979814147\\
2.62397255518578	-1.18708383428965	0.636621979814147\\
2.69426689169894	-1.01750966398109	0.636621979814147\\
2.75288614878809	-0.846178380609375	0.636621979814147\\
2.79997137020776	-0.674210891351419	0.636621979814147\\
2.83579342325549	-0.502668539517726	0.636621979814147\\
2.86073751970104	-0.332536980461974	0.636621979814147\\
2.87528596147844	-0.164713811579428	0.636621979814147\\
2.88	-1.16853587941575e-15	0.636621979814147\\
2.9	0	0.636621979814147\\
2.89525322509981	0.165857657493173	0.636621979814147\\
2.88060375247674	0.334846265048514	0.636621979814147\\
2.85548643313921	0.506159293264375	0.636621979814147\\
2.8194156158342	0.678892911430246	0.636621979814147\\
2.77200341371023	0.852054619363605	0.636621979814147\\
2.71297707844685	1.02457570331429	0.636621979814147\\
2.64219458681901	1.19532747202777	0.636621979814147\\
2.55965752143987	1.36314099525189	0.636621979814147\\
2.46552037221254	1.52682981835074	0.636621979814147\\
2.36009550454812	1.68521488523326	0.636621979814147\\
2.24385323353478	1.83715069233733	0.636621979814147\\
2.11741669977753	1.9815515434889	0.636621979814147\\
1.9815515434889	2.11741669977753	0.636621979814147\\
1.83715069233733	2.24385323353478	0.636621979814147\\
1.68521488523327	2.36009550454812	0.636621979814147\\
1.52682981835074	2.46552037221254	0.636621979814147\\
1.36314099525189	2.55965752143987	0.636621979814147\\
1.19532747202777	2.64219458681901	0.636621979814147\\
1.02457570331429	2.71297707844685	0.636621979814147\\
0.852054619363605	2.77200341371023	0.636621979814147\\
0.678892911430247	2.8194156158342	0.636621979814147\\
0.506159293264375	2.85548643313921	0.636621979814147\\
0.334846265048515	2.88060375247674	0.636621979814147\\
0.165857657493174	2.89525322509981	0.636621979814147\\
6.43929354282591e-16	2.9	0.636621979814147\\
-0.165857657493172	2.89525322509981	0.636621979814147\\
-0.334846265048514	2.88060375247674	0.636621979814147\\
-0.506159293264375	2.85548643313921	0.636621979814147\\
-0.678892911430245	2.8194156158342	0.636621979814147\\
-0.852054619363604	2.77200341371023	0.636621979814147\\
-1.02457570331429	2.71297707844685	0.636621979814147\\
-1.19532747202777	2.64219458681901	0.636621979814147\\
-1.36314099525189	2.55965752143987	0.636621979814147\\
-1.52682981835074	2.46552037221254	0.636621979814147\\
-1.68521488523326	2.36009550454812	0.636621979814147\\
-1.83715069233733	2.24385323353478	0.636621979814147\\
-1.9815515434889	2.11741669977753	0.636621979814147\\
-2.11741669977753	1.9815515434889	0.636621979814147\\
-2.24385323353478	1.83715069233733	0.636621979814147\\
-2.36009550454812	1.68521488523326	0.636621979814147\\
-2.46552037221254	1.52682981835074	0.636621979814147\\
-2.55965752143987	1.36314099525189	0.636621979814147\\
-2.64219458681901	1.19532747202777	0.636621979814147\\
-2.71297707844685	1.02457570331429	0.636621979814147\\
-2.77200341371023	0.852054619363605	0.636621979814147\\
-2.8194156158342	0.678892911430247	0.636621979814147\\
-2.85548643313921	0.506159293264376	0.636621979814147\\
-2.88060375247674	0.334846265048515	0.636621979814147\\
-2.89525322509981	0.165857657493174	0.636621979814147\\
-2.9	8.21503140158957e-16	0.636621979814147\\
-2.89525322509981	-0.165857657493172	0.636621979814147\\
-2.88060375247674	-0.334846265048514	0.636621979814147\\
-2.85548643313921	-0.506159293264375	0.636621979814147\\
-2.8194156158342	-0.678892911430246	0.636621979814147\\
-2.77200341371023	-0.852054619363605	0.636621979814147\\
-2.71297707844685	-1.02457570331429	0.636621979814147\\
-2.64219458681901	-1.19532747202777	0.636621979814147\\
-2.55965752143987	-1.36314099525189	0.636621979814147\\
-2.46552037221254	-1.52682981835074	0.636621979814147\\
-2.36009550454812	-1.68521488523326	0.636621979814147\\
-2.24385323353478	-1.83715069233733	0.636621979814147\\
-2.11741669977753	-1.9815515434889	0.636621979814147\\
-1.9815515434889	-2.11741669977753	0.636621979814147\\
-1.83715069233733	-2.24385323353478	0.636621979814147\\
-1.68521488523326	-2.36009550454812	0.636621979814147\\
-1.52682981835074	-2.46552037221254	0.636621979814147\\
-1.36314099525189	-2.55965752143987	0.636621979814147\\
-1.19532747202777	-2.64219458681902	0.636621979814147\\
-1.02457570331429	-2.71297707844685	0.636621979814147\\
-0.852054619363606	-2.77200341371022	0.636621979814147\\
-0.678892911430248	-2.8194156158342	0.636621979814147\\
-0.506159293264376	-2.85548643313921	0.636621979814147\\
-0.334846265048515	-2.88060375247674	0.636621979814147\\
-0.165857657493174	-2.89525322509981	0.636621979814147\\
-9.99076926035323e-16	-2.9	0.636621979814147\\
0.165857657493172	-2.89525322509981	0.636621979814147\\
0.334846265048514	-2.88060375247674	0.636621979814147\\
0.506159293264374	-2.85548643313921	0.636621979814147\\
0.678892911430246	-2.8194156158342	0.636621979814147\\
0.852054619363605	-2.77200341371023	0.636621979814147\\
1.02457570331429	-2.71297707844685	0.636621979814147\\
1.19532747202777	-2.64219458681901	0.636621979814147\\
1.36314099525189	-2.55965752143987	0.636621979814147\\
1.52682981835074	-2.46552037221254	0.636621979814147\\
1.68521488523326	-2.36009550454812	0.636621979814147\\
1.83715069233733	-2.24385323353478	0.636621979814147\\
1.9815515434889	-2.11741669977753	0.636621979814147\\
2.11741669977753	-1.9815515434889	0.636621979814147\\
2.24385323353478	-1.83715069233733	0.636621979814147\\
2.36009550454812	-1.68521488523327	0.636621979814147\\
2.46552037221254	-1.52682981835074	0.636621979814147\\
2.55965752143987	-1.36314099525189	0.636621979814147\\
2.64219458681902	-1.19532747202777	0.636621979814147\\
2.71297707844685	-1.02457570331429	0.636621979814147\\
2.77200341371022	-0.852054619363606	0.636621979814147\\
2.8194156158342	-0.678892911430248	0.636621979814147\\
2.85548643313921	-0.506159293264376	0.636621979814147\\
2.88060375247674	-0.334846265048515	0.636621979814147\\
2.89525322509981	-0.165857657493174	0.636621979814147\\
2.9	-1.17665071191169e-15	0.636621979814147\\
2.92	0	0.636621979814147\\
2.91522048872119	0.167001503406919	0.636621979814147\\
2.90046998525244	0.337155549635055	0.636621979814147\\
2.87517944302293	0.509650047011026	0.636621979814147\\
2.83885986146065	0.683574931509076	0.636621979814147\\
2.79112067863237	0.857930858117836	0.636621979814147\\
2.73168726519476	1.03164174264749	0.636621979814147\\
2.66041661845225	1.20357110976589	0.636621979814147\\
2.57731033193255	1.37254196763294	0.636621979814147\\
2.48252396098642	1.53735967916695	0.636621979814147\\
2.37637202526914	1.69683705685556	0.636621979814147\\
2.25932808342123	1.84982069711207	0.636621979814147\\
2.1320195735691	1.99521741620262	0.636621979814147\\
1.99521741620262	2.1320195735691	0.636621979814147\\
1.84982069711207	2.25932808342123	0.636621979814147\\
1.69683705685556	2.37637202526914	0.636621979814147\\
1.53735967916695	2.48252396098642	0.636621979814147\\
1.37254196763294	2.57731033193255	0.636621979814147\\
1.20357110976589	2.66041661845225	0.636621979814147\\
1.03164174264749	2.73168726519476	0.636621979814147\\
0.857930858117837	2.79112067863237	0.636621979814147\\
0.683574931509077	2.83885986146065	0.636621979814147\\
0.509650047011026	2.87517944302293	0.636621979814147\\
0.337155549635056	2.90046998525244	0.636621979814147\\
0.16700150340692	2.91522048872119	0.636621979814147\\
6.48370246381091e-16	2.92	0.636621979814147\\
-0.167001503406918	2.91522048872119	0.636621979814147\\
-0.337155549635055	2.90046998525244	0.636621979814147\\
-0.509650047011026	2.87517944302293	0.636621979814147\\
-0.683574931509075	2.83885986146065	0.636621979814147\\
-0.857930858117836	2.79112067863237	0.636621979814147\\
-1.03164174264749	2.73168726519476	0.636621979814147\\
-1.20357110976589	2.66041661845225	0.636621979814147\\
-1.37254196763294	2.57731033193255	0.636621979814147\\
-1.53735967916695	2.48252396098642	0.636621979814147\\
-1.69683705685556	2.37637202526914	0.636621979814147\\
-1.84982069711207	2.25932808342123	0.636621979814147\\
-1.99521741620262	2.1320195735691	0.636621979814147\\
-2.1320195735691	1.99521741620262	0.636621979814147\\
-2.25932808342123	1.84982069711207	0.636621979814147\\
-2.37637202526914	1.69683705685556	0.636621979814147\\
-2.48252396098642	1.53735967916695	0.636621979814147\\
-2.57731033193255	1.37254196763294	0.636621979814147\\
-2.66041661845225	1.20357110976589	0.636621979814147\\
-2.73168726519476	1.03164174264749	0.636621979814147\\
-2.79112067863237	0.857930858117837	0.636621979814147\\
-2.83885986146065	0.683574931509076	0.636621979814147\\
-2.87517944302293	0.509650047011027	0.636621979814147\\
-2.90046998525244	0.337155549635057	0.636621979814147\\
-2.91522048872119	0.16700150340692	0.636621979814147\\
-2.92	8.27168679056605e-16	0.636621979814147\\
-2.91522048872119	-0.167001503406918	0.636621979814147\\
-2.90046998525244	-0.337155549635055	0.636621979814147\\
-2.87517944302293	-0.509650047011025	0.636621979814147\\
-2.83885986146065	-0.683574931509076	0.636621979814147\\
-2.79112067863236	-0.857930858117837	0.636621979814147\\
-2.73168726519476	-1.03164174264749	0.636621979814147\\
-2.66041661845225	-1.20357110976589	0.636621979814147\\
-2.57731033193255	-1.37254196763294	0.636621979814147\\
-2.48252396098642	-1.53735967916695	0.636621979814147\\
-2.37637202526914	-1.69683705685556	0.636621979814147\\
-2.25932808342123	-1.84982069711207	0.636621979814147\\
-2.1320195735691	-1.99521741620262	0.636621979814147\\
-1.99521741620262	-2.1320195735691	0.636621979814147\\
-1.84982069711207	-2.25932808342123	0.636621979814147\\
-1.69683705685556	-2.37637202526914	0.636621979814147\\
-1.53735967916695	-2.48252396098642	0.636621979814147\\
-1.37254196763294	-2.57731033193256	0.636621979814147\\
-1.20357110976589	-2.66041661845225	0.636621979814147\\
-1.03164174264749	-2.73168726519476	0.636621979814147\\
-0.857930858117838	-2.79112067863236	0.636621979814147\\
-0.683574931509077	-2.83885986146065	0.636621979814147\\
-0.509650047011027	-2.87517944302293	0.636621979814147\\
-0.337155549635057	-2.90046998525244	0.636621979814147\\
-0.16700150340692	-2.91522048872119	0.636621979814147\\
-1.00596711173212e-15	-2.92	0.636621979814147\\
0.167001503406918	-2.91522048872119	0.636621979814147\\
0.337155549635055	-2.90046998525244	0.636621979814147\\
0.509650047011025	-2.87517944302293	0.636621979814147\\
0.683574931509075	-2.83885986146065	0.636621979814147\\
0.857930858117837	-2.79112067863236	0.636621979814147\\
1.03164174264749	-2.73168726519476	0.636621979814147\\
1.20357110976589	-2.66041661845225	0.636621979814147\\
1.37254196763294	-2.57731033193255	0.636621979814147\\
1.53735967916695	-2.48252396098642	0.636621979814147\\
1.69683705685556	-2.37637202526914	0.636621979814147\\
1.84982069711207	-2.25932808342123	0.636621979814147\\
1.99521741620262	-2.1320195735691	0.636621979814147\\
2.1320195735691	-1.99521741620262	0.636621979814147\\
2.25932808342123	-1.84982069711207	0.636621979814147\\
2.37637202526914	-1.69683705685556	0.636621979814147\\
2.48252396098642	-1.53735967916695	0.636621979814147\\
2.57731033193256	-1.37254196763294	0.636621979814147\\
2.66041661845225	-1.20357110976589	0.636621979814147\\
2.73168726519476	-1.03164174264749	0.636621979814147\\
2.79112067863236	-0.857930858117838	0.636621979814147\\
2.83885986146065	-0.683574931509077	0.636621979814147\\
2.87517944302293	-0.509650047011027	0.636621979814147\\
2.90046998525244	-0.337155549635057	0.636621979814147\\
2.91522048872119	-0.16700150340692	0.636621979814147\\
2.92	-1.18476554440763e-15	0.636621979814147\\
2.94	0	0.636621979814147\\
2.93518775234257	0.168145349320665	0.636621979814147\\
2.92033621802814	0.339464834221597	0.636621979814147\\
2.89487245290665	0.513140800757677	0.636621979814147\\
2.85830410708709	0.688256951587905	0.636621979814147\\
2.8102379435545	0.863807096872068	0.636621979814147\\
2.75039745194267	1.03870778198069	0.636621979814147\\
2.67863865008548	1.21181474750401	0.636621979814147\\
2.59496314242524	1.38194294001399	0.636621979814147\\
2.4995275497603	1.54788953998316	0.636621979814147\\
2.39264854599016	1.70845922847786	0.636621979814147\\
2.27480293330767	1.86249070188681	0.636621979814147\\
2.14662244736066	2.00888328891634	0.636621979814147\\
2.00888328891634	2.14662244736066	0.636621979814147\\
1.86249070188681	2.27480293330767	0.636621979814147\\
1.70845922847786	2.39264854599016	0.636621979814147\\
1.54788953998316	2.4995275497603	0.636621979814147\\
1.38194294001399	2.59496314242524	0.636621979814147\\
1.21181474750401	2.67863865008548	0.636621979814147\\
1.03870778198069	2.75039745194267	0.636621979814147\\
0.863807096872069	2.8102379435545	0.636621979814147\\
0.688256951587906	2.85830410708709	0.636621979814147\\
0.513140800757677	2.89487245290665	0.636621979814147\\
0.339464834221597	2.92033621802814	0.636621979814147\\
0.168145349320666	2.93518775234257	0.636621979814147\\
6.52811138479592e-16	2.94	0.636621979814147\\
-0.168145349320664	2.93518775234257	0.636621979814147\\
-0.339464834221597	2.92033621802814	0.636621979814147\\
-0.513140800757676	2.89487245290665	0.636621979814147\\
-0.688256951587904	2.85830410708709	0.636621979814147\\
-0.863807096872068	2.8102379435545	0.636621979814147\\
-1.03870778198069	2.75039745194267	0.636621979814147\\
-1.21181474750401	2.67863865008548	0.636621979814147\\
-1.38194294001399	2.59496314242524	0.636621979814147\\
-1.54788953998316	2.4995275497603	0.636621979814147\\
-1.70845922847786	2.39264854599016	0.636621979814147\\
-1.86249070188681	2.27480293330767	0.636621979814147\\
-2.00888328891633	2.14662244736066	0.636621979814147\\
-2.14662244736066	2.00888328891634	0.636621979814147\\
-2.27480293330767	1.86249070188681	0.636621979814147\\
-2.39264854599016	1.70845922847786	0.636621979814147\\
-2.4995275497603	1.54788953998316	0.636621979814147\\
-2.59496314242524	1.38194294001399	0.636621979814147\\
-2.67863865008548	1.21181474750401	0.636621979814147\\
-2.75039745194267	1.03870778198069	0.636621979814147\\
-2.8102379435545	0.863807096872069	0.636621979814147\\
-2.85830410708709	0.688256951587905	0.636621979814147\\
-2.89487245290665	0.513140800757678	0.636621979814147\\
-2.92033621802814	0.339464834221598	0.636621979814147\\
-2.93518775234257	0.168145349320666	0.636621979814147\\
-2.94	8.32834217954253e-16	0.636621979814147\\
-2.93518775234257	-0.168145349320664	0.636621979814147\\
-2.92033621802814	-0.339464834221597	0.636621979814147\\
-2.89487245290665	-0.513140800757676	0.636621979814147\\
-2.85830410708709	-0.688256951587905	0.636621979814147\\
-2.8102379435545	-0.863807096872068	0.636621979814147\\
-2.75039745194267	-1.03870778198069	0.636621979814147\\
-2.67863865008548	-1.21181474750401	0.636621979814147\\
-2.59496314242524	-1.38194294001399	0.636621979814147\\
-2.4995275497603	-1.54788953998316	0.636621979814147\\
-2.39264854599016	-1.70845922847786	0.636621979814147\\
-2.27480293330767	-1.86249070188681	0.636621979814147\\
-2.14662244736066	-2.00888328891633	0.636621979814147\\
-2.00888328891634	-2.14662244736066	0.636621979814147\\
-1.86249070188681	-2.27480293330767	0.636621979814147\\
-1.70845922847786	-2.39264854599016	0.636621979814147\\
-1.54788953998316	-2.4995275497603	0.636621979814147\\
-1.38194294001399	-2.59496314242524	0.636621979814147\\
-1.21181474750401	-2.67863865008548	0.636621979814147\\
-1.03870778198069	-2.75039745194267	0.636621979814147\\
-0.86380709687207	-2.8102379435545	0.636621979814147\\
-0.688256951587906	-2.85830410708709	0.636621979814147\\
-0.513140800757678	-2.89487245290665	0.636621979814147\\
-0.339464834221598	-2.92033621802814	0.636621979814147\\
-0.168145349320666	-2.93518775234257	0.636621979814147\\
-1.01285729742891e-15	-2.94	0.636621979814147\\
0.168145349320664	-2.93518775234257	0.636621979814147\\
0.339464834221596	-2.92033621802814	0.636621979814147\\
0.513140800757676	-2.89487245290665	0.636621979814147\\
0.688256951587905	-2.85830410708709	0.636621979814147\\
0.863807096872068	-2.8102379435545	0.636621979814147\\
1.03870778198069	-2.75039745194267	0.636621979814147\\
1.21181474750401	-2.67863865008548	0.636621979814147\\
1.38194294001399	-2.59496314242524	0.636621979814147\\
1.54788953998316	-2.4995275497603	0.636621979814147\\
1.70845922847786	-2.39264854599016	0.636621979814147\\
1.86249070188681	-2.27480293330767	0.636621979814147\\
2.00888328891633	-2.14662244736066	0.636621979814147\\
2.14662244736066	-2.00888328891634	0.636621979814147\\
2.27480293330767	-1.86249070188681	0.636621979814147\\
2.39264854599016	-1.70845922847786	0.636621979814147\\
2.4995275497603	-1.54788953998316	0.636621979814147\\
2.59496314242524	-1.38194294001399	0.636621979814147\\
2.67863865008548	-1.21181474750401	0.636621979814147\\
2.75039745194267	-1.03870778198069	0.636621979814147\\
2.8102379435545	-0.86380709687207	0.636621979814147\\
2.85830410708709	-0.688256951587907	0.636621979814147\\
2.89487245290665	-0.513140800757678	0.636621979814147\\
2.92033621802814	-0.339464834221598	0.636621979814147\\
2.93518775234257	-0.168145349320666	0.636621979814147\\
2.94	-1.19288037690357e-15	0.636621979814147\\
2.96	0	0.636621979814147\\
2.95515501596395	0.169289195234411	0.636621979814147\\
2.94020245080384	0.341774118808138	0.636621979814147\\
2.91456546279037	0.516631554504328	0.636621979814147\\
2.87774835271353	0.692938971666734	0.636621979814147\\
2.82935520847664	0.8696833356263	0.636621979814147\\
2.76910763869058	1.0457738213139	0.636621979814147\\
2.69686068171872	1.22005838524214	0.636621979814147\\
2.61261595291793	1.39134391239504	0.636621979814147\\
2.51653113853418	1.55841940079937	0.636621979814147\\
2.40892506671118	1.72008140010016	0.636621979814147\\
2.29027778319412	1.87516070666155	0.636621979814147\\
2.16122532115223	2.02254916163005	0.636621979814147\\
2.02254916163005	2.16122532115223	0.636621979814147\\
1.87516070666155	2.29027778319412	0.636621979814147\\
1.72008140010016	2.40892506671118	0.636621979814147\\
1.55841940079937	2.51653113853418	0.636621979814147\\
1.39134391239504	2.61261595291793	0.636621979814147\\
1.22005838524214	2.69686068171872	0.636621979814147\\
1.0457738213139	2.76910763869058	0.636621979814147\\
0.8696833356263	2.82935520847664	0.636621979814147\\
0.692938971666735	2.87774835271353	0.636621979814147\\
0.516631554504328	2.91456546279037	0.636621979814147\\
0.341774118808139	2.94020245080384	0.636621979814147\\
0.169289195234412	2.95515501596395	0.636621979814147\\
6.57252030578093e-16	2.96	0.636621979814147\\
-0.16928919523441	2.95515501596395	0.636621979814147\\
-0.341774118808138	2.94020245080384	0.636621979814147\\
-0.516631554504327	2.91456546279037	0.636621979814147\\
-0.692938971666733	2.87774835271353	0.636621979814147\\
-0.869683335626299	2.82935520847664	0.636621979814147\\
-1.04577382131389	2.76910763869058	0.636621979814147\\
-1.22005838524214	2.69686068171872	0.636621979814147\\
-1.39134391239504	2.61261595291793	0.636621979814147\\
-1.55841940079937	2.51653113853418	0.636621979814147\\
-1.72008140010016	2.40892506671118	0.636621979814147\\
-1.87516070666155	2.29027778319412	0.636621979814147\\
-2.02254916163005	2.16122532115223	0.636621979814147\\
-2.16122532115223	2.02254916163005	0.636621979814147\\
-2.29027778319412	1.87516070666155	0.636621979814147\\
-2.40892506671118	1.72008140010016	0.636621979814147\\
-2.51653113853418	1.55841940079937	0.636621979814147\\
-2.61261595291793	1.39134391239504	0.636621979814147\\
-2.69686068171872	1.22005838524214	0.636621979814147\\
-2.76910763869058	1.0457738213139	0.636621979814147\\
-2.82935520847664	0.8696833356263	0.636621979814147\\
-2.87774835271353	0.692938971666735	0.636621979814147\\
-2.91456546279037	0.516631554504329	0.636621979814147\\
-2.94020245080384	0.34177411880814	0.636621979814147\\
-2.95515501596395	0.169289195234412	0.636621979814147\\
-2.96	8.38499756851901e-16	0.636621979814147\\
-2.95515501596395	-0.16928919523441	0.636621979814147\\
-2.94020245080384	-0.341774118808138	0.636621979814147\\
-2.91456546279037	-0.516631554504327	0.636621979814147\\
-2.87774835271353	-0.692938971666734	0.636621979814147\\
-2.82935520847664	-0.8696833356263	0.636621979814147\\
-2.76910763869058	-1.0457738213139	0.636621979814147\\
-2.69686068171872	-1.22005838524213	0.636621979814147\\
-2.61261595291793	-1.39134391239504	0.636621979814147\\
-2.51653113853418	-1.55841940079937	0.636621979814147\\
-2.40892506671118	-1.72008140010016	0.636621979814147\\
-2.29027778319412	-1.87516070666155	0.636621979814147\\
-2.16122532115223	-2.02254916163005	0.636621979814147\\
-2.02254916163005	-2.16122532115223	0.636621979814147\\
-1.87516070666155	-2.29027778319412	0.636621979814147\\
-1.72008140010016	-2.40892506671118	0.636621979814147\\
-1.55841940079937	-2.51653113853418	0.636621979814147\\
-1.39134391239504	-2.61261595291793	0.636621979814147\\
-1.22005838524214	-2.69686068171872	0.636621979814147\\
-1.0457738213139	-2.76910763869058	0.636621979814147\\
-0.869683335626302	-2.82935520847664	0.636621979814147\\
-0.692938971666736	-2.87774835271353	0.636621979814147\\
-0.516631554504329	-2.91456546279037	0.636621979814147\\
-0.34177411880814	-2.94020245080384	0.636621979814147\\
-0.169289195234412	-2.95515501596395	0.636621979814147\\
-1.01974748312571e-15	-2.96	0.636621979814147\\
0.16928919523441	-2.95515501596395	0.636621979814147\\
0.341774118808138	-2.94020245080384	0.636621979814147\\
0.516631554504327	-2.91456546279037	0.636621979814147\\
0.692938971666734	-2.87774835271353	0.636621979814147\\
0.8696833356263	-2.82935520847664	0.636621979814147\\
1.0457738213139	-2.76910763869058	0.636621979814147\\
1.22005838524213	-2.69686068171872	0.636621979814147\\
1.39134391239503	-2.61261595291793	0.636621979814147\\
1.55841940079937	-2.51653113853418	0.636621979814147\\
1.72008140010016	-2.40892506671118	0.636621979814147\\
1.87516070666155	-2.29027778319412	0.636621979814147\\
2.02254916163005	-2.16122532115223	0.636621979814147\\
2.16122532115223	-2.02254916163005	0.636621979814147\\
2.29027778319412	-1.87516070666155	0.636621979814147\\
2.40892506671118	-1.72008140010016	0.636621979814147\\
2.51653113853418	-1.55841940079937	0.636621979814147\\
2.61261595291793	-1.39134391239504	0.636621979814147\\
2.69686068171872	-1.22005838524214	0.636621979814147\\
2.76910763869058	-1.0457738213139	0.636621979814147\\
2.82935520847664	-0.869683335626302	0.636621979814147\\
2.87774835271353	-0.692938971666736	0.636621979814147\\
2.91456546279037	-0.516631554504329	0.636621979814147\\
2.94020245080384	-0.34177411880814	0.636621979814147\\
2.95515501596395	-0.169289195234412	0.636621979814147\\
2.96	-1.20099520939952e-15	0.636621979814147\\
2.98	0	0.636621979814147\\
2.97512227958533	0.170433041148157	0.636621979814147\\
2.96006868357954	0.34408340339468	0.636621979814147\\
2.93425847267409	0.520122308250978	0.636621979814147\\
2.89719259833997	0.697620991745564	0.636621979814147\\
2.84847247339878	0.875559574380532	0.636621979814147\\
2.78781782543849	1.0528398606471	0.636621979814147\\
2.71508271335195	1.22830202298026	0.636621979814147\\
2.63026876341062	1.40074488477608	0.636621979814147\\
2.53353472730806	1.56894926161559	0.636621979814147\\
2.4252015874322	1.73170357172246	0.636621979814147\\
2.30575263308057	1.88783071143629	0.636621979814147\\
2.1758281949438	2.03621503434377	0.636621979814147\\
2.03621503434377	2.1758281949438	0.636621979814147\\
1.88783071143629	2.30575263308057	0.636621979814147\\
1.73170357172246	2.4252015874322	0.636621979814147\\
1.56894926161559	2.53353472730806	0.636621979814147\\
1.40074488477608	2.63026876341062	0.636621979814147\\
1.22830202298026	2.71508271335195	0.636621979814147\\
1.0528398606471	2.78781782543849	0.636621979814147\\
0.875559574380532	2.84847247339878	0.636621979814147\\
0.697620991745564	2.89719259833997	0.636621979814147\\
0.520122308250979	2.93425847267409	0.636621979814147\\
0.34408340339468	2.96006868357954	0.636621979814147\\
0.170433041148158	2.97512227958533	0.636621979814147\\
6.61692922676593e-16	2.98	0.636621979814147\\
-0.170433041148156	2.97512227958533	0.636621979814147\\
-0.34408340339468	2.96006868357954	0.636621979814147\\
-0.520122308250978	2.93425847267409	0.636621979814147\\
-0.697620991745563	2.89719259833998	0.636621979814147\\
-0.875559574380531	2.84847247339878	0.636621979814147\\
-1.0528398606471	2.78781782543849	0.636621979814147\\
-1.22830202298026	2.71508271335195	0.636621979814147\\
-1.40074488477608	2.63026876341062	0.636621979814147\\
-1.56894926161559	2.53353472730806	0.636621979814147\\
-1.73170357172246	2.4252015874322	0.636621979814147\\
-1.88783071143629	2.30575263308057	0.636621979814147\\
-2.03621503434377	2.1758281949438	0.636621979814147\\
-2.1758281949438	2.03621503434377	0.636621979814147\\
-2.30575263308056	1.88783071143629	0.636621979814147\\
-2.4252015874322	1.73170357172246	0.636621979814147\\
-2.53353472730806	1.56894926161559	0.636621979814147\\
-2.63026876341062	1.40074488477608	0.636621979814147\\
-2.71508271335195	1.22830202298026	0.636621979814147\\
-2.78781782543849	1.0528398606471	0.636621979814147\\
-2.84847247339878	0.875559574380532	0.636621979814147\\
-2.89719259833998	0.697620991745564	0.636621979814147\\
-2.93425847267409	0.520122308250979	0.636621979814147\\
-2.96006868357954	0.344083403394681	0.636621979814147\\
-2.97512227958533	0.170433041148158	0.636621979814147\\
-2.98	8.44165295749549e-16	0.636621979814147\\
-2.97512227958533	-0.170433041148156	0.636621979814147\\
-2.96006868357954	-0.34408340339468	0.636621979814147\\
-2.93425847267409	-0.520122308250978	0.636621979814147\\
-2.89719259833997	-0.697620991745563	0.636621979814147\\
-2.84847247339878	-0.875559574380532	0.636621979814147\\
-2.78781782543849	-1.0528398606471	0.636621979814147\\
-2.71508271335195	-1.22830202298026	0.636621979814147\\
-2.63026876341062	-1.40074488477608	0.636621979814147\\
-2.53353472730806	-1.56894926161558	0.636621979814147\\
-2.4252015874322	-1.73170357172246	0.636621979814147\\
-2.30575263308057	-1.88783071143629	0.636621979814147\\
-2.1758281949438	-2.03621503434377	0.636621979814147\\
-2.03621503434377	-2.1758281949438	0.636621979814147\\
-1.88783071143629	-2.30575263308056	0.636621979814147\\
-1.73170357172246	-2.4252015874322	0.636621979814147\\
-1.56894926161559	-2.53353472730806	0.636621979814147\\
-1.40074488477608	-2.63026876341062	0.636621979814147\\
-1.22830202298026	-2.71508271335195	0.636621979814147\\
-1.0528398606471	-2.78781782543849	0.636621979814147\\
-0.875559574380533	-2.84847247339878	0.636621979814147\\
-0.697620991745565	-2.89719259833997	0.636621979814147\\
-0.52012230825098	-2.93425847267409	0.636621979814147\\
-0.344083403394681	-2.96006868357954	0.636621979814147\\
-0.170433041148158	-2.97512227958533	0.636621979814147\\
-1.0266376688225e-15	-2.98	0.636621979814147\\
0.170433041148156	-2.97512227958533	0.636621979814147\\
0.344083403394679	-2.96006868357954	0.636621979814147\\
0.520122308250978	-2.93425847267409	0.636621979814147\\
0.697620991745563	-2.89719259833997	0.636621979814147\\
0.875559574380532	-2.84847247339878	0.636621979814147\\
1.0528398606471	-2.78781782543849	0.636621979814147\\
1.22830202298026	-2.71508271335195	0.636621979814147\\
1.40074488477608	-2.63026876341062	0.636621979814147\\
1.56894926161558	-2.53353472730806	0.636621979814147\\
1.73170357172246	-2.4252015874322	0.636621979814147\\
1.88783071143629	-2.30575263308057	0.636621979814147\\
2.03621503434377	-2.1758281949438	0.636621979814147\\
2.1758281949438	-2.03621503434377	0.636621979814147\\
2.30575263308056	-1.88783071143629	0.636621979814147\\
2.4252015874322	-1.73170357172246	0.636621979814147\\
2.53353472730806	-1.56894926161559	0.636621979814147\\
2.63026876341062	-1.40074488477608	0.636621979814147\\
2.71508271335195	-1.22830202298026	0.636621979814147\\
2.78781782543849	-1.0528398606471	0.636621979814147\\
2.84847247339878	-0.875559574380534	0.636621979814147\\
2.89719259833997	-0.697620991745565	0.636621979814147\\
2.93425847267409	-0.52012230825098	0.636621979814147\\
2.96006868357954	-0.344083403394681	0.636621979814147\\
2.97512227958533	-0.170433041148158	0.636621979814147\\
2.98	-1.20911004189546e-15	0.636621979814147\\
3	0	0.636621979814147\\
2.9950895432067	0.171576887061903	0.636621979814147\\
2.97993491635525	0.346392687981221	0.636621979814147\\
2.95395148255781	0.523613061997629	0.636621979814147\\
2.91663684396642	0.702303011824393	0.636621979814147\\
2.86758973832092	0.881435813134763	0.636621979814147\\
2.8065280121864	1.0599058999803	0.636621979814147\\
2.73330474498519	1.23654566071838	0.636621979814147\\
2.64792157390331	1.41014585715713	0.636621979814147\\
2.55053831608194	1.5794791224318	0.636621979814147\\
2.44147810815323	1.74332574334476	0.636621979814147\\
2.32122748296701	1.90050071621103	0.636621979814147\\
2.19043106873537	2.04988090705749	0.636621979814147\\
2.04988090705749	2.19043106873537	0.636621979814147\\
1.90050071621103	2.32122748296701	0.636621979814147\\
1.74332574334476	2.44147810815323	0.636621979814147\\
1.5794791224318	2.55053831608194	0.636621979814147\\
1.41014585715713	2.64792157390331	0.636621979814147\\
1.23654566071838	2.73330474498519	0.636621979814147\\
1.0599058999803	2.8065280121864	0.636621979814147\\
0.881435813134764	2.86758973832092	0.636621979814147\\
0.702303011824394	2.91663684396642	0.636621979814147\\
0.52361306199763	2.95395148255781	0.636621979814147\\
0.346392687981222	2.97993491635525	0.636621979814147\\
0.171576887061904	2.9950895432067	0.636621979814147\\
6.66133814775094e-16	3	0.636621979814147\\
-0.171576887061903	2.9950895432067	0.636621979814147\\
-0.346392687981221	2.97993491635525	0.636621979814147\\
-0.523613061997629	2.95395148255781	0.636621979814147\\
-0.702303011824392	2.91663684396642	0.636621979814147\\
-0.881435813134763	2.86758973832092	0.636621979814147\\
-1.0599058999803	2.8065280121864	0.636621979814147\\
-1.23654566071838	2.73330474498519	0.636621979814147\\
-1.41014585715713	2.64792157390331	0.636621979814147\\
-1.5794791224318	2.55053831608194	0.636621979814147\\
-1.74332574334476	2.44147810815323	0.636621979814147\\
-1.90050071621103	2.32122748296701	0.636621979814147\\
-2.04988090705748	2.19043106873537	0.636621979814147\\
-2.19043106873537	2.04988090705749	0.636621979814147\\
-2.32122748296701	1.90050071621103	0.636621979814147\\
-2.44147810815323	1.74332574334476	0.636621979814147\\
-2.55053831608194	1.5794791224318	0.636621979814147\\
-2.64792157390331	1.41014585715713	0.636621979814147\\
-2.73330474498519	1.23654566071838	0.636621979814147\\
-2.8065280121864	1.0599058999803	0.636621979814147\\
-2.86758973832092	0.881435813134764	0.636621979814147\\
-2.91663684396642	0.702303011824393	0.636621979814147\\
-2.95395148255781	0.52361306199763	0.636621979814147\\
-2.97993491635525	0.346392687981223	0.636621979814147\\
-2.9950895432067	0.171576887061904	0.636621979814147\\
-3	8.49830834647197e-16	0.636621979814147\\
-2.9950895432067	-0.171576887061902	0.636621979814147\\
-2.97993491635525	-0.346392687981221	0.636621979814147\\
-2.95395148255781	-0.523613061997629	0.636621979814147\\
-2.91663684396642	-0.702303011824393	0.636621979814147\\
-2.86758973832092	-0.881435813134764	0.636621979814147\\
-2.8065280121864	-1.0599058999803	0.636621979814147\\
-2.73330474498519	-1.23654566071838	0.636621979814147\\
-2.64792157390331	-1.41014585715713	0.636621979814147\\
-2.55053831608194	-1.5794791224318	0.636621979814147\\
-2.44147810815323	-1.74332574334476	0.636621979814147\\
-2.32122748296701	-1.90050071621103	0.636621979814147\\
-2.19043106873537	-2.04988090705748	0.636621979814147\\
-2.04988090705749	-2.19043106873537	0.636621979814147\\
-1.90050071621103	-2.32122748296701	0.636621979814147\\
-1.74332574334476	-2.44147810815323	0.636621979814147\\
-1.5794791224318	-2.55053831608194	0.636621979814147\\
-1.41014585715713	-2.64792157390331	0.636621979814147\\
-1.23654566071838	-2.73330474498519	0.636621979814147\\
-1.0599058999803	-2.8065280121864	0.636621979814147\\
-0.881435813134765	-2.86758973832092	0.636621979814147\\
-0.702303011824394	-2.91663684396642	0.636621979814147\\
-0.523613061997631	-2.95395148255781	0.636621979814147\\
-0.346392687981223	-2.97993491635525	0.636621979814147\\
-0.171576887061904	-2.9950895432067	0.636621979814147\\
-1.0335278545193e-15	-3	0.636621979814147\\
0.171576887061902	-2.9950895432067	0.636621979814147\\
0.346392687981221	-2.97993491635525	0.636621979814147\\
0.523613061997629	-2.95395148255781	0.636621979814147\\
0.702303011824393	-2.91663684396642	0.636621979814147\\
0.881435813134764	-2.86758973832092	0.636621979814147\\
1.0599058999803	-2.8065280121864	0.636621979814147\\
1.23654566071838	-2.73330474498519	0.636621979814147\\
1.41014585715713	-2.64792157390331	0.636621979814147\\
1.5794791224318	-2.55053831608194	0.636621979814147\\
1.74332574334476	-2.44147810815323	0.636621979814147\\
1.90050071621103	-2.32122748296701	0.636621979814147\\
2.04988090705748	-2.19043106873537	0.636621979814147\\
2.19043106873537	-2.04988090705749	0.636621979814147\\
2.32122748296701	-1.90050071621103	0.636621979814147\\
2.44147810815323	-1.74332574334476	0.636621979814147\\
2.55053831608194	-1.5794791224318	0.636621979814147\\
2.64792157390331	-1.41014585715713	0.636621979814147\\
2.73330474498519	-1.23654566071838	0.636621979814147\\
2.8065280121864	-1.0599058999803	0.636621979814147\\
2.86758973832092	-0.881435813134765	0.636621979814147\\
2.91663684396642	-0.702303011824394	0.636621979814147\\
2.95395148255781	-0.523613061997631	0.636621979814147\\
2.97993491635525	-0.346392687981223	0.636621979814147\\
2.9950895432067	-0.171576887061904	0.636621979814147\\
3	-1.2172248743914e-15	0.636621979814147\\
3.02	0	0.636621979814147\\
3.01505680682808	0.172720732975649	0.636621979814147\\
2.99980114913095	0.348701972567763	0.636621979814147\\
2.97364449244153	0.52710381574428	0.636621979814147\\
2.93608108959286	0.706985031903222	0.636621979814147\\
2.88670700324306	0.887312051888995	0.636621979814147\\
2.82523819893431	1.0669719393135	0.636621979814147\\
2.75152677661842	1.2447892984565	0.636621979814147\\
2.665574384396	1.41954682953818	0.636621979814147\\
2.56754190485582	1.59000898324801	0.636621979814147\\
2.45775462887425	1.75494791496706	0.636621979814147\\
2.33670233285346	1.91317072098577	0.636621979814147\\
2.20503394252694	2.0635467797712	0.636621979814147\\
2.0635467797712	2.20503394252694	0.636621979814147\\
1.91317072098577	2.33670233285346	0.636621979814147\\
1.75494791496706	2.45775462887425	0.636621979814147\\
1.59000898324801	2.56754190485582	0.636621979814147\\
1.41954682953818	2.665574384396	0.636621979814147\\
1.24478929845651	2.75152677661842	0.636621979814147\\
1.0669719393135	2.82523819893431	0.636621979814147\\
0.887312051888996	2.88670700324306	0.636621979814147\\
0.706985031903223	2.93608108959286	0.636621979814147\\
0.527103815744281	2.97364449244153	0.636621979814147\\
0.348701972567763	2.99980114913095	0.636621979814147\\
0.17272073297565	3.01505680682808	0.636621979814147\\
6.70574706873595e-16	3.02	0.636621979814147\\
-0.172720732975649	3.01505680682808	0.636621979814147\\
-0.348701972567763	2.99980114913095	0.636621979814147\\
-0.52710381574428	2.97364449244153	0.636621979814147\\
-0.706985031903221	2.93608108959286	0.636621979814147\\
-0.887312051888995	2.88670700324306	0.636621979814147\\
-1.0669719393135	2.82523819893431	0.636621979814147\\
-1.2447892984565	2.75152677661842	0.636621979814147\\
-1.41954682953818	2.665574384396	0.636621979814147\\
-1.59000898324801	2.56754190485582	0.636621979814147\\
-1.75494791496705	2.45775462887425	0.636621979814147\\
-1.91317072098577	2.33670233285346	0.636621979814147\\
-2.0635467797712	2.20503394252694	0.636621979814147\\
-2.20503394252694	2.0635467797712	0.636621979814147\\
-2.33670233285346	1.91317072098577	0.636621979814147\\
-2.45775462887425	1.75494791496706	0.636621979814147\\
-2.56754190485581	1.59000898324801	0.636621979814147\\
-2.665574384396	1.41954682953818	0.636621979814147\\
-2.75152677661842	1.24478929845651	0.636621979814147\\
-2.82523819893431	1.0669719393135	0.636621979814147\\
-2.88670700324306	0.887312051888996	0.636621979814147\\
-2.93608108959286	0.706985031903222	0.636621979814147\\
-2.97364449244153	0.527103815744281	0.636621979814147\\
-2.99980114913095	0.348701972567764	0.636621979814147\\
-3.01505680682808	0.17272073297565	0.636621979814147\\
-3.02	8.55496373544845e-16	0.636621979814147\\
-3.01505680682808	-0.172720732975648	0.636621979814147\\
-2.99980114913095	-0.348701972567763	0.636621979814147\\
-2.97364449244153	-0.52710381574428	0.636621979814147\\
-2.93608108959286	-0.706985031903222	0.636621979814147\\
-2.88670700324306	-0.887312051888996	0.636621979814147\\
-2.82523819893431	-1.0669719393135	0.636621979814147\\
-2.75152677661842	-1.2447892984565	0.636621979814147\\
-2.665574384396	-1.41954682953818	0.636621979814147\\
-2.56754190485582	-1.59000898324801	0.636621979814147\\
-2.45775462887425	-1.75494791496705	0.636621979814147\\
-2.33670233285346	-1.91317072098577	0.636621979814147\\
-2.20503394252694	-2.0635467797712	0.636621979814147\\
-2.0635467797712	-2.20503394252694	0.636621979814147\\
-1.91317072098577	-2.33670233285346	0.636621979814147\\
-1.75494791496706	-2.45775462887425	0.636621979814147\\
-1.59000898324801	-2.56754190485582	0.636621979814147\\
-1.41954682953818	-2.665574384396	0.636621979814147\\
-1.2447892984565	-2.75152677661842	0.636621979814147\\
-1.0669719393135	-2.82523819893431	0.636621979814147\\
-0.887312051888997	-2.88670700324306	0.636621979814147\\
-0.706985031903224	-2.93608108959286	0.636621979814147\\
-0.527103815744281	-2.97364449244153	0.636621979814147\\
-0.348701972567764	-2.99980114913095	0.636621979814147\\
-0.17272073297565	-3.01505680682808	0.636621979814147\\
-1.0404180402161e-15	-3.02	0.636621979814147\\
0.172720732975648	-3.01505680682808	0.636621979814147\\
0.348701972567762	-2.99980114913095	0.636621979814147\\
0.52710381574428	-2.97364449244153	0.636621979814147\\
0.706985031903222	-2.93608108959286	0.636621979814147\\
0.887312051888996	-2.88670700324306	0.636621979814147\\
1.0669719393135	-2.82523819893431	0.636621979814147\\
1.2447892984565	-2.75152677661842	0.636621979814147\\
1.41954682953818	-2.665574384396	0.636621979814147\\
1.59000898324801	-2.56754190485582	0.636621979814147\\
1.75494791496705	-2.45775462887425	0.636621979814147\\
1.91317072098577	-2.33670233285346	0.636621979814147\\
2.0635467797712	-2.20503394252694	0.636621979814147\\
2.20503394252694	-2.0635467797712	0.636621979814147\\
2.33670233285346	-1.91317072098577	0.636621979814147\\
2.45775462887425	-1.75494791496706	0.636621979814147\\
2.56754190485582	-1.59000898324801	0.636621979814147\\
2.665574384396	-1.41954682953818	0.636621979814147\\
2.75152677661842	-1.2447892984565	0.636621979814147\\
2.82523819893431	-1.0669719393135	0.636621979814147\\
2.88670700324306	-0.887312051888997	0.636621979814147\\
2.93608108959286	-0.706985031903224	0.636621979814147\\
2.97364449244153	-0.527103815744282	0.636621979814147\\
2.99980114913095	-0.348701972567764	0.636621979814147\\
3.01505680682808	-0.17272073297565	0.636621979814147\\
3.02	-1.22533970688735e-15	0.636621979814147\\
3.04	0	0.636621979814147\\
3.03502407044946	0.173864578889395	0.636621979814147\\
3.01966738190665	0.351011257154304	0.636621979814147\\
2.99333750232524	0.530594569490931	0.636621979814147\\
2.9555253352193	0.711667051982051	0.636621979814147\\
2.9058242681652	0.893188290643227	0.636621979814147\\
2.84394838568222	1.0740379786467	0.636621979814147\\
2.76974880825166	1.25303293619463	0.636621979814147\\
2.68322719488869	1.42894780191923	0.636621979814147\\
2.5845454936297	1.60053884406422	0.636621979814147\\
2.47403114959527	1.76657008658935	0.636621979814147\\
2.35217718273991	1.92584072576051	0.636621979814147\\
2.21963681631851	2.07721265248492	0.636621979814147\\
2.07721265248492	2.21963681631851	0.636621979814147\\
1.92584072576051	2.35217718273991	0.636621979814147\\
1.76657008658935	2.47403114959527	0.636621979814147\\
1.60053884406422	2.5845454936297	0.636621979814147\\
1.42894780191923	2.68322719488869	0.636621979814147\\
1.25303293619463	2.76974880825166	0.636621979814147\\
1.0740379786467	2.84394838568222	0.636621979814147\\
0.893188290643228	2.9058242681652	0.636621979814147\\
0.711667051982052	2.9555253352193	0.636621979814147\\
0.530594569490932	2.99333750232524	0.636621979814147\\
0.351011257154305	3.01966738190665	0.636621979814147\\
0.173864578889396	3.03502407044946	0.636621979814147\\
6.75015598972095e-16	3.04	0.636621979814147\\
-0.173864578889395	3.03502407044946	0.636621979814147\\
-0.351011257154304	3.01966738190665	0.636621979814147\\
-0.530594569490931	2.99333750232524	0.636621979814147\\
-0.71166705198205	2.9555253352193	0.636621979814147\\
-0.893188290643226	2.9058242681652	0.636621979814147\\
-1.0740379786467	2.84394838568222	0.636621979814147\\
-1.25303293619463	2.76974880825166	0.636621979814147\\
-1.42894780191923	2.68322719488869	0.636621979814147\\
-1.60053884406422	2.5845454936297	0.636621979814147\\
-1.76657008658935	2.47403114959527	0.636621979814147\\
-1.92584072576051	2.35217718273991	0.636621979814147\\
-2.07721265248492	2.21963681631851	0.636621979814147\\
-2.21963681631851	2.07721265248492	0.636621979814147\\
-2.35217718273991	1.92584072576051	0.636621979814147\\
-2.47403114959527	1.76657008658935	0.636621979814147\\
-2.58454549362969	1.60053884406422	0.636621979814147\\
-2.68322719488869	1.42894780191923	0.636621979814147\\
-2.76974880825166	1.25303293619463	0.636621979814147\\
-2.84394838568222	1.0740379786467	0.636621979814147\\
-2.9058242681652	0.893188290643228	0.636621979814147\\
-2.9555253352193	0.711667051982052	0.636621979814147\\
-2.99333750232524	0.530594569490932	0.636621979814147\\
-3.01966738190665	0.351011257154305	0.636621979814147\\
-3.03502407044946	0.173864578889396	0.636621979814147\\
-3.04	8.61161912442493e-16	0.636621979814147\\
-3.03502407044946	-0.173864578889394	0.636621979814147\\
-3.01966738190665	-0.351011257154304	0.636621979814147\\
-2.99333750232524	-0.530594569490931	0.636621979814147\\
-2.9555253352193	-0.711667051982051	0.636621979814147\\
-2.9058242681652	-0.893188290643227	0.636621979814147\\
-2.84394838568222	-1.0740379786467	0.636621979814147\\
-2.76974880825166	-1.25303293619463	0.636621979814147\\
-2.68322719488869	-1.42894780191923	0.636621979814147\\
-2.5845454936297	-1.60053884406422	0.636621979814147\\
-2.47403114959527	-1.76657008658935	0.636621979814147\\
-2.35217718273991	-1.92584072576051	0.636621979814147\\
-2.21963681631851	-2.07721265248492	0.636621979814147\\
-2.07721265248492	-2.21963681631851	0.636621979814147\\
-1.92584072576051	-2.35217718273991	0.636621979814147\\
-1.76657008658935	-2.47403114959527	0.636621979814147\\
-1.60053884406422	-2.5845454936297	0.636621979814147\\
-1.42894780191923	-2.68322719488869	0.636621979814147\\
-1.25303293619463	-2.76974880825166	0.636621979814147\\
-1.07403797864671	-2.84394838568222	0.636621979814147\\
-0.893188290643229	-2.9058242681652	0.636621979814147\\
-0.711667051982053	-2.9555253352193	0.636621979814147\\
-0.530594569490932	-2.99333750232524	0.636621979814147\\
-0.351011257154306	-3.01966738190665	0.636621979814147\\
-0.173864578889396	-3.03502407044946	0.636621979814147\\
-1.04730822591289e-15	-3.04	0.636621979814147\\
0.173864578889394	-3.03502407044946	0.636621979814147\\
0.351011257154304	-3.01966738190665	0.636621979814147\\
0.53059456949093	-2.99333750232524	0.636621979814147\\
0.711667051982051	-2.9555253352193	0.636621979814147\\
0.893188290643227	-2.9058242681652	0.636621979814147\\
1.0740379786467	-2.84394838568222	0.636621979814147\\
1.25303293619463	-2.76974880825166	0.636621979814147\\
1.42894780191923	-2.68322719488869	0.636621979814147\\
1.60053884406422	-2.5845454936297	0.636621979814147\\
1.76657008658935	-2.47403114959527	0.636621979814147\\
1.92584072576051	-2.35217718273991	0.636621979814147\\
2.07721265248492	-2.21963681631851	0.636621979814147\\
2.21963681631851	-2.07721265248492	0.636621979814147\\
2.35217718273991	-1.92584072576051	0.636621979814147\\
2.47403114959527	-1.76657008658935	0.636621979814147\\
2.5845454936297	-1.60053884406422	0.636621979814147\\
2.68322719488869	-1.42894780191923	0.636621979814147\\
2.76974880825166	-1.25303293619463	0.636621979814147\\
2.84394838568222	-1.07403797864671	0.636621979814147\\
2.9058242681652	-0.893188290643229	0.636621979814147\\
2.9555253352193	-0.711667051982053	0.636621979814147\\
2.99333750232524	-0.530594569490932	0.636621979814147\\
3.01966738190665	-0.351011257154306	0.636621979814147\\
3.03502407044946	-0.173864578889396	0.636621979814147\\
3.04	-1.23345453938329e-15	0.636621979814147\\
3.06	0	0.636621979814147\\
3.05499133407084	0.175008424803141	0.636621979814147\\
3.03953361468235	0.353320541740846	0.636621979814147\\
3.01303051220896	0.534085323237582	0.636621979814147\\
2.97496958084575	0.716349072060881	0.636621979814147\\
2.92494153308734	0.899064529397459	0.636621979814147\\
2.86265857243013	1.08110401797991	0.636621979814147\\
2.78797083988489	1.26127657393275	0.636621979814147\\
2.70088000538138	1.43834877430027	0.636621979814147\\
2.60154908240358	1.61106870488043	0.636621979814147\\
2.49030767031629	1.77819225821165	0.636621979814147\\
2.36765203262635	1.93851073053525	0.636621979814147\\
2.23423969011008	2.09087852519863	0.636621979814147\\
2.09087852519863	2.23423969011008	0.636621979814147\\
1.93851073053525	2.36765203262635	0.636621979814147\\
1.77819225821165	2.49030767031629	0.636621979814147\\
1.61106870488043	2.60154908240357	0.636621979814147\\
1.43834877430027	2.70088000538138	0.636621979814147\\
1.26127657393275	2.78797083988489	0.636621979814147\\
1.08110401797991	2.86265857243013	0.636621979814147\\
0.899064529397459	2.92494153308734	0.636621979814147\\
0.716349072060882	2.97496958084575	0.636621979814147\\
0.534085323237582	3.01303051220896	0.636621979814147\\
0.353320541740846	3.03953361468235	0.636621979814147\\
0.175008424803142	3.05499133407084	0.636621979814147\\
6.79456491070596e-16	3.06	0.636621979814147\\
-0.175008424803141	3.05499133407084	0.636621979814147\\
-0.353320541740846	3.03953361468235	0.636621979814147\\
-0.534085323237582	3.01303051220896	0.636621979814147\\
-0.71634907206088	2.97496958084575	0.636621979814147\\
-0.899064529397458	2.92494153308734	0.636621979814147\\
-1.08110401797991	2.86265857243013	0.636621979814147\\
-1.26127657393275	2.78797083988489	0.636621979814147\\
-1.43834877430027	2.70088000538138	0.636621979814147\\
-1.61106870488043	2.60154908240357	0.636621979814147\\
-1.77819225821165	2.49030767031629	0.636621979814147\\
-1.93851073053525	2.36765203262635	0.636621979814147\\
-2.09087852519863	2.23423969011008	0.636621979814147\\
-2.23423969011008	2.09087852519863	0.636621979814147\\
-2.36765203262635	1.93851073053525	0.636621979814147\\
-2.49030767031629	1.77819225821165	0.636621979814147\\
-2.60154908240357	1.61106870488043	0.636621979814147\\
-2.70088000538138	1.43834877430027	0.636621979814147\\
-2.78797083988489	1.26127657393275	0.636621979814147\\
-2.86265857243013	1.08110401797991	0.636621979814147\\
-2.92494153308734	0.899064529397459	0.636621979814147\\
-2.97496958084575	0.716349072060881	0.636621979814147\\
-3.01303051220896	0.534085323237583	0.636621979814147\\
-3.03953361468235	0.353320541740847	0.636621979814147\\
-3.05499133407084	0.175008424803142	0.636621979814147\\
-3.06	8.66827451340141e-16	0.636621979814147\\
-3.05499133407084	-0.17500842480314	0.636621979814147\\
-3.03953361468235	-0.353320541740845	0.636621979814147\\
-3.01303051220896	-0.534085323237581	0.636621979814147\\
-2.97496958084575	-0.716349072060881	0.636621979814147\\
-2.92494153308734	-0.899064529397459	0.636621979814147\\
-2.86265857243013	-1.08110401797991	0.636621979814147\\
-2.78797083988489	-1.26127657393275	0.636621979814147\\
-2.70088000538138	-1.43834877430027	0.636621979814147\\
-2.60154908240358	-1.61106870488043	0.636621979814147\\
-2.49030767031629	-1.77819225821165	0.636621979814147\\
-2.36765203262635	-1.93851073053525	0.636621979814147\\
-2.23423969011008	-2.09087852519863	0.636621979814147\\
-2.09087852519863	-2.23423969011008	0.636621979814147\\
-1.93851073053525	-2.36765203262635	0.636621979814147\\
-1.77819225821165	-2.49030767031629	0.636621979814147\\
-1.61106870488043	-2.60154908240357	0.636621979814147\\
-1.43834877430027	-2.70088000538138	0.636621979814147\\
-1.26127657393275	-2.78797083988489	0.636621979814147\\
-1.08110401797991	-2.86265857243013	0.636621979814147\\
-0.899064529397461	-2.92494153308734	0.636621979814147\\
-0.716349072060882	-2.97496958084575	0.636621979814147\\
-0.534085323237583	-3.01303051220896	0.636621979814147\\
-0.353320541740847	-3.03953361468235	0.636621979814147\\
-0.175008424803142	-3.05499133407084	0.636621979814147\\
-1.05419841160969e-15	-3.06	0.636621979814147\\
0.17500842480314	-3.05499133407084	0.636621979814147\\
0.353320541740845	-3.03953361468235	0.636621979814147\\
0.534085323237581	-3.01303051220896	0.636621979814147\\
0.716349072060881	-2.97496958084575	0.636621979814147\\
0.899064529397459	-2.92494153308734	0.636621979814147\\
1.08110401797991	-2.86265857243013	0.636621979814147\\
1.26127657393275	-2.78797083988489	0.636621979814147\\
1.43834877430027	-2.70088000538138	0.636621979814147\\
1.61106870488043	-2.60154908240358	0.636621979814147\\
1.77819225821165	-2.49030767031629	0.636621979814147\\
1.93851073053525	-2.36765203262635	0.636621979814147\\
2.09087852519863	-2.23423969011008	0.636621979814147\\
2.23423969011008	-2.09087852519863	0.636621979814147\\
2.36765203262635	-1.93851073053525	0.636621979814147\\
2.49030767031629	-1.77819225821165	0.636621979814147\\
2.60154908240357	-1.61106870488043	0.636621979814147\\
2.70088000538138	-1.43834877430027	0.636621979814147\\
2.78797083988489	-1.26127657393275	0.636621979814147\\
2.86265857243013	-1.08110401797991	0.636621979814147\\
2.92494153308734	-0.899064529397461	0.636621979814147\\
2.97496958084575	-0.716349072060882	0.636621979814147\\
3.01303051220896	-0.534085323237583	0.636621979814147\\
3.03953361468235	-0.353320541740847	0.636621979814147\\
3.05499133407084	-0.175008424803142	0.636621979814147\\
3.06	-1.24156937187923e-15	0.636621979814147\\
3.08	0	0.636621979814147\\
3.07495859769222	0.176152270716887	0.636621979814147\\
3.05939984745805	0.355629826327387	0.636621979814147\\
3.03272352209268	0.537576076984233	0.636621979814147\\
2.99441382647219	0.72103109213971	0.636621979814147\\
2.94405879800948	0.90494076815169	0.636621979814147\\
2.88136875917804	1.08817005731311	0.636621979814147\\
2.80619287151813	1.26952021167087	0.636621979814147\\
2.71853281587406	1.44774974668132	0.636621979814147\\
2.61855267117745	1.62159856569665	0.636621979814147\\
2.50658419103731	1.78981442983395	0.636621979814147\\
2.3831268825128	1.95118073530999	0.636621979814147\\
2.24884256390165	2.10454439791235	0.636621979814147\\
2.10454439791235	2.24884256390165	0.636621979814147\\
1.95118073530999	2.3831268825128	0.636621979814147\\
1.78981442983395	2.50658419103731	0.636621979814147\\
1.62159856569665	2.61855267117745	0.636621979814147\\
1.44774974668132	2.71853281587406	0.636621979814147\\
1.26952021167087	2.80619287151813	0.636621979814147\\
1.08817005731311	2.88136875917804	0.636621979814147\\
0.904940768151691	2.94405879800948	0.636621979814147\\
0.721031092139711	2.99441382647219	0.636621979814147\\
0.537576076984233	3.03272352209268	0.636621979814147\\
0.355629826327388	3.05939984745805	0.636621979814147\\
0.176152270716888	3.07495859769222	0.636621979814147\\
6.83897383169096e-16	3.08	0.636621979814147\\
-0.176152270716887	3.07495859769222	0.636621979814147\\
-0.355629826327387	3.05939984745805	0.636621979814147\\
-0.537576076984232	3.03272352209268	0.636621979814147\\
-0.721031092139709	2.99441382647219	0.636621979814147\\
-0.90494076815169	2.94405879800948	0.636621979814147\\
-1.08817005731311	2.88136875917804	0.636621979814147\\
-1.26952021167087	2.80619287151813	0.636621979814147\\
-1.44774974668132	2.71853281587406	0.636621979814147\\
-1.62159856569665	2.61855267117745	0.636621979814147\\
-1.78981442983395	2.50658419103731	0.636621979814147\\
-1.95118073530999	2.3831268825128	0.636621979814147\\
-2.10454439791235	2.24884256390165	0.636621979814147\\
-2.24884256390165	2.10454439791235	0.636621979814147\\
-2.3831268825128	1.95118073530999	0.636621979814147\\
-2.50658419103731	1.78981442983395	0.636621979814147\\
-2.61855267117745	1.62159856569665	0.636621979814147\\
-2.71853281587406	1.44774974668132	0.636621979814147\\
-2.80619287151813	1.26952021167087	0.636621979814147\\
-2.88136875917804	1.08817005731311	0.636621979814147\\
-2.94405879800948	0.904940768151691	0.636621979814147\\
-2.99441382647219	0.72103109213971	0.636621979814147\\
-3.03272352209268	0.537576076984234	0.636621979814147\\
-3.05939984745805	0.355629826327388	0.636621979814147\\
-3.07495859769222	0.176152270716888	0.636621979814147\\
-3.08	8.72492990237789e-16	0.636621979814147\\
-3.07495859769222	-0.176152270716886	0.636621979814147\\
-3.05939984745805	-0.355629826327387	0.636621979814147\\
-3.03272352209268	-0.537576076984232	0.636621979814147\\
-2.99441382647219	-0.72103109213971	0.636621979814147\\
-2.94405879800948	-0.904940768151691	0.636621979814147\\
-2.88136875917804	-1.08817005731311	0.636621979814147\\
-2.80619287151813	-1.26952021167087	0.636621979814147\\
-2.71853281587406	-1.44774974668132	0.636621979814147\\
-2.61855267117745	-1.62159856569664	0.636621979814147\\
-2.50658419103731	-1.78981442983395	0.636621979814147\\
-2.3831268825128	-1.95118073530999	0.636621979814147\\
-2.24884256390165	-2.10454439791235	0.636621979814147\\
-2.10454439791235	-2.24884256390165	0.636621979814147\\
-1.95118073530999	-2.3831268825128	0.636621979814147\\
-1.78981442983395	-2.50658419103731	0.636621979814147\\
-1.62159856569665	-2.61855267117745	0.636621979814147\\
-1.44774974668132	-2.71853281587406	0.636621979814147\\
-1.26952021167087	-2.80619287151813	0.636621979814147\\
-1.08817005731311	-2.88136875917804	0.636621979814147\\
-0.904940768151692	-2.94405879800948	0.636621979814147\\
-0.721031092139712	-2.99441382647219	0.636621979814147\\
-0.537576076984234	-3.03272352209268	0.636621979814147\\
-0.355629826327389	-3.05939984745805	0.636621979814147\\
-0.176152270716888	-3.07495859769222	0.636621979814147\\
-1.06108859730648e-15	-3.08	0.636621979814147\\
0.176152270716886	-3.07495859769222	0.636621979814147\\
0.355629826327387	-3.05939984745805	0.636621979814147\\
0.537576076984232	-3.03272352209268	0.636621979814147\\
0.72103109213971	-2.99441382647219	0.636621979814147\\
0.904940768151691	-2.94405879800948	0.636621979814147\\
1.08817005731311	-2.88136875917804	0.636621979814147\\
1.26952021167087	-2.80619287151813	0.636621979814147\\
1.44774974668132	-2.71853281587406	0.636621979814147\\
1.62159856569664	-2.61855267117745	0.636621979814147\\
1.78981442983395	-2.50658419103731	0.636621979814147\\
1.95118073530999	-2.3831268825128	0.636621979814147\\
2.10454439791235	-2.24884256390165	0.636621979814147\\
2.24884256390165	-2.10454439791235	0.636621979814147\\
2.3831268825128	-1.95118073530999	0.636621979814147\\
2.50658419103731	-1.78981442983395	0.636621979814147\\
2.61855267117745	-1.62159856569665	0.636621979814147\\
2.71853281587406	-1.44774974668132	0.636621979814147\\
2.80619287151813	-1.26952021167087	0.636621979814147\\
2.88136875917804	-1.08817005731311	0.636621979814147\\
2.94405879800948	-0.904940768151692	0.636621979814147\\
2.99441382647219	-0.721031092139712	0.636621979814147\\
3.03272352209268	-0.537576076984234	0.636621979814147\\
3.05939984745805	-0.355629826327389	0.636621979814147\\
3.07495859769222	-0.176152270716888	0.636621979814147\\
3.08	-1.24968420437517e-15	0.636621979814147\\
3.1	0	0.636621979814147\\
3.0949258613136	0.177296116630633	0.636621979814147\\
3.07926608023375	0.357939110913929	0.636621979814147\\
3.0524165319764	0.541066830730883	0.636621979814147\\
3.01385807209863	0.725713112218539	0.636621979814147\\
2.96317606293162	0.910817006905922	0.636621979814147\\
2.90007894592595	1.09523609664631	0.636621979814147\\
2.82441490315136	1.27776384940899	0.636621979814147\\
2.73618562636675	1.45715071906237	0.636621979814147\\
2.63555625995133	1.63212842651286	0.636621979814147\\
2.52286071175833	1.80143660145625	0.636621979814147\\
2.39860173239925	1.96385074008473	0.636621979814147\\
2.26344543769322	2.11821027062607	0.636621979814147\\
2.11821027062607	2.26344543769322	0.636621979814147\\
1.96385074008473	2.39860173239925	0.636621979814147\\
1.80143660145625	2.52286071175833	0.636621979814147\\
1.63212842651286	2.63555625995133	0.636621979814147\\
1.45715071906237	2.73618562636675	0.636621979814147\\
1.277763849409	2.82441490315136	0.636621979814147\\
1.09523609664631	2.90007894592595	0.636621979814147\\
0.910817006905923	2.96317606293162	0.636621979814147\\
0.72571311221854	3.01385807209863	0.636621979814147\\
0.541066830730884	3.0524165319764	0.636621979814147\\
0.357939110913929	3.07926608023375	0.636621979814147\\
0.177296116630634	3.0949258613136	0.636621979814147\\
6.88338275267597e-16	3.1	0.636621979814147\\
-0.177296116630633	3.09492586131359	0.636621979814147\\
-0.357939110913929	3.07926608023375	0.636621979814147\\
-0.541066830730883	3.0524165319764	0.636621979814147\\
-0.725713112218538	3.01385807209863	0.636621979814147\\
-0.910817006905922	2.96317606293162	0.636621979814147\\
-1.09523609664631	2.90007894592595	0.636621979814147\\
-1.27776384940899	2.82441490315136	0.636621979814147\\
-1.45715071906237	2.73618562636675	0.636621979814147\\
-1.63212842651286	2.63555625995133	0.636621979814147\\
-1.80143660145625	2.52286071175833	0.636621979814147\\
-1.96385074008473	2.39860173239925	0.636621979814147\\
-2.11821027062607	2.26344543769322	0.636621979814147\\
-2.26344543769322	2.11821027062607	0.636621979814147\\
-2.39860173239925	1.96385074008473	0.636621979814147\\
-2.52286071175833	1.80143660145625	0.636621979814147\\
-2.63555625995133	1.63212842651286	0.636621979814147\\
-2.73618562636675	1.45715071906237	0.636621979814147\\
-2.82441490315136	1.277763849409	0.636621979814147\\
-2.90007894592595	1.09523609664631	0.636621979814147\\
-2.96317606293162	0.910817006905923	0.636621979814147\\
-3.01385807209863	0.72571311221854	0.636621979814147\\
-3.0524165319764	0.541066830730885	0.636621979814147\\
-3.07926608023375	0.35793911091393	0.636621979814147\\
-3.0949258613136	0.177296116630634	0.636621979814147\\
-3.1	8.78158529135437e-16	0.636621979814147\\
-3.09492586131359	-0.177296116630632	0.636621979814147\\
-3.07926608023375	-0.357939110913928	0.636621979814147\\
-3.0524165319764	-0.541066830730883	0.636621979814147\\
-3.01385807209863	-0.725713112218539	0.636621979814147\\
-2.96317606293162	-0.910817006905923	0.636621979814147\\
-2.90007894592595	-1.09523609664631	0.636621979814147\\
-2.82441490315136	-1.27776384940899	0.636621979814147\\
-2.73618562636675	-1.45715071906237	0.636621979814147\\
-2.63555625995133	-1.63212842651286	0.636621979814147\\
-2.52286071175833	-1.80143660145625	0.636621979814147\\
-2.39860173239925	-1.96385074008473	0.636621979814147\\
-2.26344543769322	-2.11821027062607	0.636621979814147\\
-2.11821027062607	-2.26344543769322	0.636621979814147\\
-1.96385074008473	-2.39860173239925	0.636621979814147\\
-1.80143660145625	-2.52286071175833	0.636621979814147\\
-1.63212842651286	-2.63555625995133	0.636621979814147\\
-1.45715071906237	-2.73618562636675	0.636621979814147\\
-1.27776384940899	-2.82441490315136	0.636621979814147\\
-1.09523609664631	-2.90007894592595	0.636621979814147\\
-0.910817006905924	-2.96317606293162	0.636621979814147\\
-0.725713112218541	-3.01385807209863	0.636621979814147\\
-0.541066830730885	-3.0524165319764	0.636621979814147\\
-0.35793911091393	-3.07926608023375	0.636621979814147\\
-0.177296116630634	-3.0949258613136	0.636621979814147\\
-1.06797878300328e-15	-3.1	0.636621979814147\\
0.177296116630632	-3.09492586131359	0.636621979814147\\
0.357939110913928	-3.07926608023375	0.636621979814147\\
0.541066830730883	-3.0524165319764	0.636621979814147\\
0.725713112218539	-3.01385807209863	0.636621979814147\\
0.910817006905923	-2.96317606293162	0.636621979814147\\
1.09523609664631	-2.90007894592595	0.636621979814147\\
1.27776384940899	-2.82441490315136	0.636621979814147\\
1.45715071906237	-2.73618562636675	0.636621979814147\\
1.63212842651286	-2.63555625995133	0.636621979814147\\
1.80143660145625	-2.52286071175833	0.636621979814147\\
1.96385074008473	-2.39860173239925	0.636621979814147\\
2.11821027062607	-2.26344543769322	0.636621979814147\\
2.26344543769322	-2.11821027062607	0.636621979814147\\
2.39860173239925	-1.96385074008473	0.636621979814147\\
2.52286071175833	-1.80143660145625	0.636621979814147\\
2.63555625995133	-1.63212842651286	0.636621979814147\\
2.73618562636675	-1.45715071906237	0.636621979814147\\
2.82441490315136	-1.27776384940899	0.636621979814147\\
2.90007894592595	-1.09523609664631	0.636621979814147\\
2.96317606293162	-0.910817006905924	0.636621979814147\\
3.01385807209863	-0.725713112218541	0.636621979814147\\
3.0524165319764	-0.541066830730885	0.636621979814147\\
3.07926608023375	-0.35793911091393	0.636621979814147\\
3.0949258613136	-0.177296116630634	0.636621979814147\\
3.1	-1.25779903687112e-15	0.636621979814147\\
3.12	0	0.636621979814147\\
3.11489312493497	0.178439962544379	0.636621979814147\\
3.09913231300946	0.36024839550047	0.636621979814147\\
3.07210954186012	0.544557584477534	0.636621979814147\\
3.03330231772508	0.730395132297369	0.636621979814147\\
2.98229332785376	0.916693245660154	0.636621979814147\\
2.91878913267386	1.10230213597951	0.636621979814147\\
2.84263693478459	1.28600748714712	0.636621979814147\\
2.75383843685944	1.46655169144342	0.636621979814147\\
2.65255984872521	1.64265828732907	0.636621979814147\\
2.53913723247935	1.81305877307855	0.636621979814147\\
2.41407658228569	1.97652074485947	0.636621979814147\\
2.27804831148479	2.13187614333978	0.636621979814147\\
2.13187614333978	2.27804831148479	0.636621979814147\\
1.97652074485947	2.41407658228569	0.636621979814147\\
1.81305877307855	2.53913723247935	0.636621979814147\\
1.64265828732907	2.65255984872521	0.636621979814147\\
1.46655169144342	2.75383843685944	0.636621979814147\\
1.28600748714712	2.84263693478459	0.636621979814147\\
1.10230213597951	2.91878913267386	0.636621979814147\\
0.916693245660155	2.98229332785376	0.636621979814147\\
0.730395132297369	3.03330231772508	0.636621979814147\\
0.544557584477535	3.07210954186012	0.636621979814147\\
0.360248395500471	3.09913231300946	0.636621979814147\\
0.17843996254438	3.11489312493497	0.636621979814147\\
6.92779167366098e-16	3.12	0.636621979814147\\
-0.178439962544379	3.11489312493497	0.636621979814147\\
-0.36024839550047	3.09913231300946	0.636621979814147\\
-0.544557584477534	3.07210954186012	0.636621979814147\\
-0.730395132297368	3.03330231772508	0.636621979814147\\
-0.916693245660154	2.98229332785376	0.636621979814147\\
-1.10230213597951	2.91878913267386	0.636621979814147\\
-1.28600748714712	2.84263693478459	0.636621979814147\\
-1.46655169144342	2.75383843685944	0.636621979814147\\
-1.64265828732907	2.65255984872521	0.636621979814147\\
-1.81305877307855	2.53913723247935	0.636621979814147\\
-1.97652074485947	2.41407658228569	0.636621979814147\\
-2.13187614333978	2.27804831148479	0.636621979814147\\
-2.27804831148479	2.13187614333978	0.636621979814147\\
-2.41407658228569	1.97652074485947	0.636621979814147\\
-2.53913723247935	1.81305877307855	0.636621979814147\\
-2.65255984872521	1.64265828732907	0.636621979814147\\
-2.75383843685944	1.46655169144342	0.636621979814147\\
-2.84263693478459	1.28600748714712	0.636621979814147\\
-2.91878913267386	1.10230213597951	0.636621979814147\\
-2.98229332785376	0.916693245660155	0.636621979814147\\
-3.03330231772508	0.730395132297369	0.636621979814147\\
-3.07210954186012	0.544557584477536	0.636621979814147\\
-3.09913231300946	0.360248395500471	0.636621979814147\\
-3.11489312493497	0.17843996254438	0.636621979814147\\
-3.12	8.83824068033085e-16	0.636621979814147\\
-3.11489312493497	-0.178439962544378	0.636621979814147\\
-3.09913231300946	-0.36024839550047	0.636621979814147\\
-3.07210954186012	-0.544557584477534	0.636621979814147\\
-3.03330231772508	-0.730395132297369	0.636621979814147\\
-2.98229332785376	-0.916693245660154	0.636621979814147\\
-2.91878913267386	-1.10230213597951	0.636621979814147\\
-2.8426369347846	-1.28600748714712	0.636621979814147\\
-2.75383843685944	-1.46655169144342	0.636621979814147\\
-2.65255984872521	-1.64265828732907	0.636621979814147\\
-2.53913723247935	-1.81305877307855	0.636621979814147\\
-2.41407658228569	-1.97652074485947	0.636621979814147\\
-2.27804831148479	-2.13187614333978	0.636621979814147\\
-2.13187614333978	-2.27804831148479	0.636621979814147\\
-1.97652074485947	-2.41407658228569	0.636621979814147\\
-1.81305877307855	-2.53913723247935	0.636621979814147\\
-1.64265828732907	-2.65255984872521	0.636621979814147\\
-1.46655169144342	-2.75383843685944	0.636621979814147\\
-1.28600748714712	-2.8426369347846	0.636621979814147\\
-1.10230213597951	-2.91878913267386	0.636621979814147\\
-0.916693245660156	-2.98229332785376	0.636621979814147\\
-0.73039513229737	-3.03330231772508	0.636621979814147\\
-0.544557584477536	-3.07210954186012	0.636621979814147\\
-0.360248395500472	-3.09913231300946	0.636621979814147\\
-0.17843996254438	-3.11489312493497	0.636621979814147\\
-1.07486896870007e-15	-3.12	0.636621979814147\\
0.178439962544378	-3.11489312493497	0.636621979814147\\
0.36024839550047	-3.09913231300946	0.636621979814147\\
0.544557584477534	-3.07210954186012	0.636621979814147\\
0.730395132297368	-3.03330231772508	0.636621979814147\\
0.916693245660154	-2.98229332785376	0.636621979814147\\
1.10230213597951	-2.91878913267386	0.636621979814147\\
1.28600748714712	-2.8426369347846	0.636621979814147\\
1.46655169144341	-2.75383843685944	0.636621979814147\\
1.64265828732907	-2.65255984872521	0.636621979814147\\
1.81305877307855	-2.53913723247936	0.636621979814147\\
1.97652074485947	-2.41407658228569	0.636621979814147\\
2.13187614333978	-2.27804831148479	0.636621979814147\\
2.27804831148479	-2.13187614333978	0.636621979814147\\
2.41407658228569	-1.97652074485947	0.636621979814147\\
2.53913723247935	-1.81305877307855	0.636621979814147\\
2.65255984872521	-1.64265828732907	0.636621979814147\\
2.75383843685944	-1.46655169144342	0.636621979814147\\
2.8426369347846	-1.28600748714712	0.636621979814147\\
2.91878913267386	-1.10230213597951	0.636621979814147\\
2.98229332785376	-0.916693245660156	0.636621979814147\\
3.03330231772508	-0.73039513229737	0.636621979814147\\
3.07210954186012	-0.544557584477536	0.636621979814147\\
3.09913231300946	-0.360248395500472	0.636621979814147\\
3.11489312493497	-0.17843996254438	0.636621979814147\\
3.12	-1.26591386936706e-15	0.636621979814147\\
3.14	0	0.636621979814147\\
3.13486038855635	0.179583808458125	0.636621979814147\\
3.11899854578516	0.362557680087012	0.636621979814147\\
3.09180255174384	0.548048338224185	0.636621979814147\\
3.05274656335152	0.735077152376198	0.636621979814147\\
3.0014105927759	0.922569484414386	0.636621979814147\\
2.93749931942177	1.10936817531271	0.636621979814147\\
2.86085896641783	1.29425112488524	0.636621979814147\\
2.77149124735213	1.47595266382446	0.636621979814147\\
2.66956343749909	1.65318814814528	0.636621979814147\\
2.55541375320038	1.82468094470085	0.636621979814147\\
2.42955143217214	1.98919074963421	0.636621979814147\\
2.29265118527636	2.1455420160535	0.636621979814147\\
2.1455420160535	2.29265118527636	0.636621979814147\\
1.98919074963421	2.42955143217214	0.636621979814147\\
1.82468094470085	2.55541375320038	0.636621979814147\\
1.65318814814528	2.66956343749909	0.636621979814147\\
1.47595266382446	2.77149124735213	0.636621979814147\\
1.29425112488524	2.86085896641783	0.636621979814147\\
1.10936817531271	2.93749931942177	0.636621979814147\\
0.922569484414386	3.0014105927759	0.636621979814147\\
0.735077152376199	3.05274656335152	0.636621979814147\\
0.548048338224186	3.09180255174384	0.636621979814147\\
0.362557680087012	3.11899854578516	0.636621979814147\\
0.179583808458126	3.13486038855635	0.636621979814147\\
6.97220059464598e-16	3.14	0.636621979814147\\
-0.179583808458125	3.13486038855635	0.636621979814147\\
-0.362557680087011	3.11899854578516	0.636621979814147\\
-0.548048338224185	3.09180255174384	0.636621979814147\\
-0.735077152376197	3.05274656335152	0.636621979814147\\
-0.922569484414385	3.0014105927759	0.636621979814147\\
-1.10936817531271	2.93749931942177	0.636621979814147\\
-1.29425112488524	2.86085896641783	0.636621979814147\\
-1.47595266382446	2.77149124735213	0.636621979814147\\
-1.65318814814528	2.66956343749909	0.636621979814147\\
-1.82468094470084	2.55541375320038	0.636621979814147\\
-1.98919074963421	2.42955143217214	0.636621979814147\\
-2.1455420160535	2.29265118527636	0.636621979814147\\
-2.29265118527636	2.1455420160535	0.636621979814147\\
-2.42955143217214	1.98919074963421	0.636621979814147\\
-2.55541375320038	1.82468094470085	0.636621979814147\\
-2.66956343749909	1.65318814814528	0.636621979814147\\
-2.77149124735213	1.47595266382447	0.636621979814147\\
-2.86085896641783	1.29425112488524	0.636621979814147\\
-2.93749931942177	1.10936817531271	0.636621979814147\\
-3.0014105927759	0.922569484414386	0.636621979814147\\
-3.05274656335152	0.735077152376198	0.636621979814147\\
-3.09180255174384	0.548048338224186	0.636621979814147\\
-3.11899854578516	0.362557680087013	0.636621979814147\\
-3.13486038855635	0.179583808458126	0.636621979814147\\
-3.14	8.89489606930733e-16	0.636621979814147\\
-3.13486038855635	-0.179583808458124	0.636621979814147\\
-3.11899854578516	-0.362557680087011	0.636621979814147\\
-3.09180255174384	-0.548048338224185	0.636621979814147\\
-3.05274656335152	-0.735077152376198	0.636621979814147\\
-3.0014105927759	-0.922569484414386	0.636621979814147\\
-2.93749931942177	-1.10936817531271	0.636621979814147\\
-2.86085896641783	-1.29425112488524	0.636621979814147\\
-2.77149124735213	-1.47595266382446	0.636621979814147\\
-2.66956343749909	-1.65318814814528	0.636621979814147\\
-2.55541375320038	-1.82468094470084	0.636621979814147\\
-2.42955143217214	-1.98919074963421	0.636621979814147\\
-2.29265118527636	-2.1455420160535	0.636621979814147\\
-2.1455420160535	-2.29265118527636	0.636621979814147\\
-1.98919074963421	-2.42955143217214	0.636621979814147\\
-1.82468094470085	-2.55541375320038	0.636621979814147\\
-1.65318814814528	-2.66956343749909	0.636621979814147\\
-1.47595266382446	-2.77149124735213	0.636621979814147\\
-1.29425112488524	-2.86085896641783	0.636621979814147\\
-1.10936817531271	-2.93749931942177	0.636621979814147\\
-0.922569484414387	-3.0014105927759	0.636621979814147\\
-0.735077152376199	-3.05274656335152	0.636621979814147\\
-0.548048338224187	-3.09180255174384	0.636621979814147\\
-0.362557680087013	-3.11899854578516	0.636621979814147\\
-0.179583808458126	-3.13486038855635	0.636621979814147\\
-1.08175915439687e-15	-3.14	0.636621979814147\\
0.179583808458124	-3.13486038855635	0.636621979814147\\
0.362557680087011	-3.11899854578516	0.636621979814147\\
0.548048338224185	-3.09180255174384	0.636621979814147\\
0.735077152376198	-3.05274656335152	0.636621979814147\\
0.922569484414386	-3.0014105927759	0.636621979814147\\
1.10936817531271	-2.93749931942177	0.636621979814147\\
1.29425112488524	-2.86085896641783	0.636621979814147\\
1.47595266382446	-2.77149124735213	0.636621979814147\\
1.65318814814528	-2.66956343749909	0.636621979814147\\
1.82468094470084	-2.55541375320038	0.636621979814147\\
1.98919074963421	-2.42955143217214	0.636621979814147\\
2.1455420160535	-2.29265118527636	0.636621979814147\\
2.29265118527636	-2.1455420160535	0.636621979814147\\
2.42955143217214	-1.98919074963421	0.636621979814147\\
2.55541375320038	-1.82468094470085	0.636621979814147\\
2.66956343749909	-1.65318814814528	0.636621979814147\\
2.77149124735213	-1.47595266382446	0.636621979814147\\
2.86085896641783	-1.29425112488524	0.636621979814147\\
2.93749931942177	-1.10936817531271	0.636621979814147\\
3.0014105927759	-0.922569484414388	0.636621979814147\\
3.05274656335152	-0.7350771523762	0.636621979814147\\
3.09180255174384	-0.548048338224187	0.636621979814147\\
3.11899854578516	-0.362557680087013	0.636621979814147\\
3.13486038855635	-0.179583808458127	0.636621979814147\\
3.14	-1.274028701863e-15	0.636621979814147\\
3.16	0	0.636621979814147\\
3.15482765217773	0.180727654371871	0.636621979814147\\
3.13886477856086	0.364866964673553	0.636621979814147\\
3.11149556162756	0.551539091970836	0.636621979814147\\
3.07219080897796	0.739759172455027	0.636621979814147\\
3.02052785769804	0.928445723168617	0.636621979814147\\
2.95620950616968	1.11643421464592	0.636621979814147\\
2.87908099805106	1.30249476262336	0.636621979814147\\
2.78914405784482	1.48535363620551	0.636621979814147\\
2.68656702627297	1.66371800896149	0.636621979814147\\
2.5716902739214	1.83630311632314	0.636621979814147\\
2.44502628205859	2.00186075440895	0.636621979814147\\
2.30725405906793	2.15920788876722	0.636621979814147\\
2.15920788876722	2.30725405906793	0.636621979814147\\
2.00186075440895	2.44502628205859	0.636621979814147\\
1.83630311632314	2.5716902739214	0.636621979814147\\
1.66371800896149	2.68656702627297	0.636621979814147\\
1.48535363620551	2.78914405784482	0.636621979814147\\
1.30249476262336	2.87908099805106	0.636621979814147\\
1.11643421464592	2.95620950616968	0.636621979814147\\
0.928445723168618	3.02052785769804	0.636621979814147\\
0.739759172455028	3.07219080897796	0.636621979814147\\
0.551539091970837	3.11149556162756	0.636621979814147\\
0.364866964673554	3.13886477856086	0.636621979814147\\
0.180727654371872	3.15482765217773	0.636621979814147\\
7.01660951563099e-16	3.16	0.636621979814147\\
-0.180727654371871	3.15482765217773	0.636621979814147\\
-0.364866964673553	3.13886477856086	0.636621979814147\\
-0.551539091970836	3.11149556162756	0.636621979814147\\
-0.739759172455026	3.07219080897796	0.636621979814147\\
-0.928445723168617	3.02052785769804	0.636621979814147\\
-1.11643421464591	2.95620950616968	0.636621979814147\\
-1.30249476262336	2.87908099805106	0.636621979814147\\
-1.48535363620551	2.78914405784482	0.636621979814147\\
-1.66371800896149	2.68656702627297	0.636621979814147\\
-1.83630311632314	2.5716902739214	0.636621979814147\\
-2.00186075440895	2.44502628205859	0.636621979814147\\
-2.15920788876722	2.30725405906793	0.636621979814147\\
-2.30725405906792	2.15920788876722	0.636621979814147\\
-2.44502628205859	2.00186075440895	0.636621979814147\\
-2.5716902739214	1.83630311632314	0.636621979814147\\
-2.68656702627297	1.66371800896149	0.636621979814147\\
-2.78914405784482	1.48535363620551	0.636621979814147\\
-2.87908099805106	1.30249476262336	0.636621979814147\\
-2.95620950616968	1.11643421464592	0.636621979814147\\
-3.02052785769804	0.928445723168618	0.636621979814147\\
-3.07219080897796	0.739759172455028	0.636621979814147\\
-3.11149556162756	0.551539091970837	0.636621979814147\\
-3.13886477856086	0.364866964673554	0.636621979814147\\
-3.15482765217773	0.180727654371872	0.636621979814147\\
-3.16	8.95155145828381e-16	0.636621979814147\\
-3.15482765217773	-0.18072765437187	0.636621979814147\\
-3.13886477856086	-0.364866964673553	0.636621979814147\\
-3.11149556162756	-0.551539091970836	0.636621979814147\\
-3.07219080897796	-0.739759172455027	0.636621979814147\\
-3.02052785769804	-0.928445723168618	0.636621979814147\\
-2.95620950616968	-1.11643421464592	0.636621979814147\\
-2.87908099805106	-1.30249476262336	0.636621979814147\\
-2.78914405784482	-1.48535363620551	0.636621979814147\\
-2.68656702627297	-1.66371800896149	0.636621979814147\\
-2.5716902739214	-1.83630311632314	0.636621979814147\\
-2.44502628205859	-2.00186075440895	0.636621979814147\\
-2.30725405906793	-2.15920788876722	0.636621979814147\\
-2.15920788876722	-2.30725405906792	0.636621979814147\\
-2.00186075440895	-2.44502628205859	0.636621979814147\\
-1.83630311632314	-2.5716902739214	0.636621979814147\\
-1.66371800896149	-2.68656702627297	0.636621979814147\\
-1.48535363620551	-2.78914405784482	0.636621979814147\\
-1.30249476262336	-2.87908099805106	0.636621979814147\\
-1.11643421464592	-2.95620950616968	0.636621979814147\\
-0.928445723168619	-3.02052785769804	0.636621979814147\\
-0.739759172455029	-3.07219080897796	0.636621979814147\\
-0.551539091970838	-3.11149556162756	0.636621979814147\\
-0.364866964673555	-3.13886477856086	0.636621979814147\\
-0.180727654371872	-3.15482765217773	0.636621979814147\\
-1.08864934009366e-15	-3.16	0.636621979814147\\
0.18072765437187	-3.15482765217773	0.636621979814147\\
0.364866964673553	-3.13886477856086	0.636621979814147\\
0.551539091970836	-3.11149556162756	0.636621979814147\\
0.739759172455027	-3.07219080897796	0.636621979814147\\
0.928445723168618	-3.02052785769804	0.636621979814147\\
1.11643421464592	-2.95620950616968	0.636621979814147\\
1.30249476262336	-2.87908099805106	0.636621979814147\\
1.48535363620551	-2.78914405784482	0.636621979814147\\
1.66371800896149	-2.68656702627297	0.636621979814147\\
1.83630311632314	-2.5716902739214	0.636621979814147\\
2.00186075440895	-2.44502628205859	0.636621979814147\\
2.15920788876722	-2.30725405906793	0.636621979814147\\
2.30725405906792	-2.15920788876722	0.636621979814147\\
2.44502628205859	-2.00186075440895	0.636621979814147\\
2.5716902739214	-1.83630311632314	0.636621979814147\\
2.68656702627297	-1.66371800896149	0.636621979814147\\
2.78914405784482	-1.48535363620551	0.636621979814147\\
2.87908099805106	-1.30249476262336	0.636621979814147\\
2.95620950616968	-1.11643421464592	0.636621979814147\\
3.02052785769804	-0.92844572316862	0.636621979814147\\
3.07219080897796	-0.739759172455029	0.636621979814147\\
3.11149556162756	-0.551539091970838	0.636621979814147\\
3.13886477856086	-0.364866964673555	0.636621979814147\\
3.15482765217773	-0.180727654371873	0.636621979814147\\
3.16	-1.28214353435894e-15	0.636621979814147\\
3.18	0	0.636621979814147\\
3.17479491579911	0.181871500285617	0.636621979814147\\
3.15873101133656	0.367176249260095	0.636621979814147\\
3.13118857151128	0.555029845717487	0.636621979814147\\
3.0916350546044	0.744441192533857	0.636621979814147\\
3.03964512262018	0.934321961922849	0.636621979814147\\
2.97491969291759	1.12350025397912	0.636621979814147\\
2.8973030296843	1.31073840036148	0.636621979814147\\
2.80679686833751	1.49475460858656	0.636621979814147\\
2.70357061504685	1.67424786977771	0.636621979814147\\
2.58796679464242	1.84792528794544	0.636621979814147\\
2.46050113194503	2.01453075918369	0.636621979814147\\
2.32185693285949	2.17287376148093	0.636621979814147\\
2.17287376148093	2.32185693285949	0.636621979814147\\
2.01453075918369	2.46050113194503	0.636621979814147\\
1.84792528794544	2.58796679464242	0.636621979814147\\
1.67424786977771	2.70357061504685	0.636621979814147\\
1.49475460858656	2.80679686833751	0.636621979814147\\
1.31073840036149	2.8973030296843	0.636621979814147\\
1.12350025397912	2.97491969291758	0.636621979814147\\
0.93432196192285	3.03964512262018	0.636621979814147\\
0.744441192533857	3.0916350546044	0.636621979814147\\
0.555029845717488	3.13118857151128	0.636621979814147\\
0.367176249260095	3.15873101133656	0.636621979814147\\
0.181871500285618	3.17479491579911	0.636621979814147\\
7.061018436616e-16	3.18	0.636621979814147\\
-0.181871500285617	3.17479491579911	0.636621979814147\\
-0.367176249260094	3.15873101133656	0.636621979814147\\
-0.555029845717487	3.13118857151128	0.636621979814147\\
-0.744441192533855	3.0916350546044	0.636621979814147\\
-0.934321961922849	3.03964512262018	0.636621979814147\\
-1.12350025397912	2.97491969291759	0.636621979814147\\
-1.31073840036148	2.8973030296843	0.636621979814147\\
-1.49475460858656	2.80679686833751	0.636621979814147\\
-1.67424786977771	2.70357061504685	0.636621979814147\\
-1.84792528794544	2.58796679464242	0.636621979814147\\
-2.01453075918369	2.46050113194503	0.636621979814147\\
-2.17287376148093	2.32185693285949	0.636621979814147\\
-2.32185693285949	2.17287376148093	0.636621979814147\\
-2.46050113194503	2.01453075918369	0.636621979814147\\
-2.58796679464242	1.84792528794544	0.636621979814147\\
-2.70357061504685	1.67424786977771	0.636621979814147\\
-2.80679686833751	1.49475460858656	0.636621979814147\\
-2.8973030296843	1.31073840036149	0.636621979814147\\
-2.97491969291758	1.12350025397912	0.636621979814147\\
-3.03964512262018	0.93432196192285	0.636621979814147\\
-3.0916350546044	0.744441192533857	0.636621979814147\\
-3.13118857151128	0.555029845717488	0.636621979814147\\
-3.15873101133656	0.367176249260096	0.636621979814147\\
-3.17479491579911	0.181871500285618	0.636621979814147\\
-3.18	9.00820684726029e-16	0.636621979814147\\
-3.17479491579911	-0.181871500285616	0.636621979814147\\
-3.15873101133656	-0.367176249260094	0.636621979814147\\
-3.13118857151128	-0.555029845717487	0.636621979814147\\
-3.0916350546044	-0.744441192533856	0.636621979814147\\
-3.03964512262018	-0.934321961922849	0.636621979814147\\
-2.97491969291758	-1.12350025397912	0.636621979814147\\
-2.8973030296843	-1.31073840036148	0.636621979814147\\
-2.80679686833751	-1.49475460858656	0.636621979814147\\
-2.70357061504685	-1.6742478697777	0.636621979814147\\
-2.58796679464242	-1.84792528794544	0.636621979814147\\
-2.46050113194503	-2.01453075918369	0.636621979814147\\
-2.32185693285949	-2.17287376148093	0.636621979814147\\
-2.17287376148093	-2.32185693285949	0.636621979814147\\
-2.01453075918369	-2.46050113194503	0.636621979814147\\
-1.84792528794544	-2.58796679464242	0.636621979814147\\
-1.67424786977771	-2.70357061504685	0.636621979814147\\
-1.49475460858656	-2.80679686833751	0.636621979814147\\
-1.31073840036148	-2.8973030296843	0.636621979814147\\
-1.12350025397912	-2.97491969291758	0.636621979814147\\
-0.934321961922851	-3.03964512262018	0.636621979814147\\
-0.744441192533858	-3.0916350546044	0.636621979814147\\
-0.555029845717488	-3.13118857151128	0.636621979814147\\
-0.367176249260096	-3.15873101133656	0.636621979814147\\
-0.181871500285618	-3.17479491579911	0.636621979814147\\
-1.09553952579046e-15	-3.18	0.636621979814147\\
0.181871500285616	-3.17479491579911	0.636621979814147\\
0.367176249260094	-3.15873101133656	0.636621979814147\\
0.555029845717486	-3.13118857151128	0.636621979814147\\
0.744441192533856	-3.0916350546044	0.636621979814147\\
0.934321961922849	-3.03964512262018	0.636621979814147\\
1.12350025397912	-2.97491969291758	0.636621979814147\\
1.31073840036148	-2.8973030296843	0.636621979814147\\
1.49475460858656	-2.80679686833751	0.636621979814147\\
1.6742478697777	-2.70357061504685	0.636621979814147\\
1.84792528794544	-2.58796679464242	0.636621979814147\\
2.01453075918369	-2.46050113194503	0.636621979814147\\
2.17287376148093	-2.32185693285949	0.636621979814147\\
2.32185693285949	-2.17287376148093	0.636621979814147\\
2.46050113194503	-2.01453075918369	0.636621979814147\\
2.58796679464242	-1.84792528794544	0.636621979814147\\
2.70357061504685	-1.67424786977771	0.636621979814147\\
2.80679686833751	-1.49475460858656	0.636621979814147\\
2.8973030296843	-1.31073840036148	0.636621979814147\\
2.97491969291758	-1.12350025397912	0.636621979814147\\
3.03964512262018	-0.934321961922851	0.636621979814147\\
3.0916350546044	-0.744441192533858	0.636621979814147\\
3.13118857151128	-0.555029845717489	0.636621979814147\\
3.15873101133656	-0.367176249260096	0.636621979814147\\
3.17479491579911	-0.181871500285619	0.636621979814147\\
3.18	-1.29025836685489e-15	0.636621979814147\\
3.2	0	0.636621979814147\\
3.19476217942048	0.183015346199363	0.636621979814147\\
3.17859724411226	0.369485533846636	0.636621979814147\\
3.15088158139499	0.558520599464138	0.636621979814147\\
3.11107930023085	0.749123212612686	0.636621979814147\\
3.05876238754232	0.940198200677081	0.636621979814147\\
2.99362987966549	1.13056629331232	0.636621979814147\\
2.91552506131753	1.31898203809961	0.636621979814147\\
2.8244496788302	1.50415558096761	0.636621979814147\\
2.72057420382073	1.68477773059392	0.636621979814147\\
2.60424331536344	1.85954745956774	0.636621979814147\\
2.47597598183148	2.02720076395843	0.636621979814147\\
2.33645980665106	2.18653963419465	0.636621979814147\\
2.18653963419465	2.33645980665106	0.636621979814147\\
2.02720076395843	2.47597598183148	0.636621979814147\\
1.85954745956774	2.60424331536344	0.636621979814147\\
1.68477773059392	2.72057420382073	0.636621979814147\\
1.50415558096761	2.8244496788302	0.636621979814147\\
1.31898203809961	2.91552506131753	0.636621979814147\\
1.13056629331232	2.99362987966549	0.636621979814147\\
0.940198200677082	3.05876238754232	0.636621979814147\\
0.749123212612687	3.11107930023085	0.636621979814147\\
0.558520599464138	3.15088158139499	0.636621979814147\\
0.369485533846637	3.17859724411226	0.636621979814147\\
0.183015346199364	3.19476217942049	0.636621979814147\\
7.105427357601e-16	3.2	0.636621979814147\\
-0.183015346199363	3.19476217942049	0.636621979814147\\
-0.369485533846636	3.17859724411226	0.636621979814147\\
-0.558520599464138	3.15088158139499	0.636621979814147\\
-0.749123212612685	3.11107930023085	0.636621979814147\\
-0.94019820067708	3.05876238754232	0.636621979814147\\
-1.13056629331232	2.99362987966549	0.636621979814147\\
-1.31898203809961	2.91552506131753	0.636621979814147\\
-1.50415558096761	2.8244496788302	0.636621979814147\\
-1.68477773059392	2.72057420382073	0.636621979814147\\
-1.85954745956774	2.60424331536344	0.636621979814147\\
-2.02720076395843	2.47597598183148	0.636621979814147\\
-2.18653963419465	2.33645980665106	0.636621979814147\\
-2.33645980665106	2.18653963419465	0.636621979814147\\
-2.47597598183148	2.02720076395843	0.636621979814147\\
-2.60424331536344	1.85954745956774	0.636621979814147\\
-2.72057420382073	1.68477773059392	0.636621979814147\\
-2.8244496788302	1.50415558096761	0.636621979814147\\
-2.91552506131753	1.31898203809961	0.636621979814147\\
-2.99362987966549	1.13056629331232	0.636621979814147\\
-3.05876238754232	0.940198200677082	0.636621979814147\\
-3.11107930023085	0.749123212612686	0.636621979814147\\
-3.15088158139499	0.558520599464139	0.636621979814147\\
-3.17859724411226	0.369485533846637	0.636621979814147\\
-3.19476217942049	0.183015346199364	0.636621979814147\\
-3.2	9.06486223623677e-16	0.636621979814147\\
-3.19476217942049	-0.183015346199362	0.636621979814147\\
-3.17859724411226	-0.369485533846636	0.636621979814147\\
-3.15088158139499	-0.558520599464137	0.636621979814147\\
-3.11107930023085	-0.749123212612686	0.636621979814147\\
-3.05876238754232	-0.940198200677081	0.636621979814147\\
-2.99362987966549	-1.13056629331232	0.636621979814147\\
-2.91552506131753	-1.31898203809961	0.636621979814147\\
-2.8244496788302	-1.50415558096761	0.636621979814147\\
-2.72057420382073	-1.68477773059392	0.636621979814147\\
-2.60424331536344	-1.85954745956774	0.636621979814147\\
-2.47597598183148	-2.02720076395843	0.636621979814147\\
-2.33645980665106	-2.18653963419465	0.636621979814147\\
-2.18653963419465	-2.33645980665106	0.636621979814147\\
-2.02720076395843	-2.47597598183148	0.636621979814147\\
-1.85954745956774	-2.60424331536344	0.636621979814147\\
-1.68477773059392	-2.72057420382073	0.636621979814147\\
-1.50415558096761	-2.8244496788302	0.636621979814147\\
-1.31898203809961	-2.91552506131753	0.636621979814147\\
-1.13056629331232	-2.99362987966549	0.636621979814147\\
-0.940198200677083	-3.05876238754232	0.636621979814147\\
-0.749123212612687	-3.11107930023085	0.636621979814147\\
-0.558520599464139	-3.15088158139499	0.636621979814147\\
-0.369485533846638	-3.17859724411226	0.636621979814147\\
-0.183015346199364	-3.19476217942049	0.636621979814147\\
-1.10242971148725e-15	-3.2	0.636621979814147\\
0.183015346199362	-3.19476217942049	0.636621979814147\\
0.369485533846636	-3.17859724411226	0.636621979814147\\
0.558520599464137	-3.15088158139499	0.636621979814147\\
0.749123212612686	-3.11107930023085	0.636621979814147\\
0.940198200677081	-3.05876238754232	0.636621979814147\\
1.13056629331232	-2.99362987966549	0.636621979814147\\
1.31898203809961	-2.91552506131753	0.636621979814147\\
1.50415558096761	-2.8244496788302	0.636621979814147\\
1.68477773059392	-2.72057420382073	0.636621979814147\\
1.85954745956774	-2.60424331536344	0.636621979814147\\
2.02720076395843	-2.47597598183148	0.636621979814147\\
2.18653963419465	-2.33645980665106	0.636621979814147\\
2.33645980665106	-2.18653963419465	0.636621979814147\\
2.47597598183148	-2.02720076395843	0.636621979814147\\
2.60424331536344	-1.85954745956774	0.636621979814147\\
2.72057420382073	-1.68477773059392	0.636621979814147\\
2.8244496788302	-1.50415558096761	0.636621979814147\\
2.91552506131753	-1.31898203809961	0.636621979814147\\
2.99362987966549	-1.13056629331232	0.636621979814147\\
3.05876238754232	-0.940198200677083	0.636621979814147\\
3.11107930023085	-0.749123212612687	0.636621979814147\\
3.15088158139499	-0.558520599464139	0.636621979814147\\
3.17859724411226	-0.369485533846638	0.636621979814147\\
3.19476217942049	-0.183015346199365	0.636621979814147\\
3.2	-1.29837319935083e-15	0.636621979814147\\
3.22	0	0.636621979814147\\
3.21472944304186	0.184159192113109	0.636621979814147\\
3.19846347688796	0.371794818433178	0.636621979814147\\
3.17057459127871	0.562011353210789	0.636621979814147\\
3.13052354585729	0.753805232691515	0.636621979814147\\
3.07787965246446	0.946074439431313	0.636621979814147\\
3.0123400664134	1.13763233264552	0.636621979814147\\
2.93374709295077	1.32722567583773	0.636621979814147\\
2.84210248932289	1.51355655334865	0.636621979814147\\
2.73757779259461	1.69530759141013	0.636621979814147\\
2.62051983608446	1.87116963119004	0.636621979814147\\
2.49145083171793	2.03987076873317	0.636621979814147\\
2.35106268044263	2.20020550690837	0.636621979814147\\
2.20020550690837	2.35106268044263	0.636621979814147\\
2.03987076873317	2.49145083171793	0.636621979814147\\
1.87116963119004	2.62051983608446	0.636621979814147\\
1.69530759141013	2.73757779259461	0.636621979814147\\
1.51355655334865	2.84210248932289	0.636621979814147\\
1.32722567583773	2.93374709295077	0.636621979814147\\
1.13763233264552	3.0123400664134	0.636621979814147\\
0.946074439431313	3.07787965246446	0.636621979814147\\
0.753805232691516	3.13052354585729	0.636621979814147\\
0.562011353210789	3.17057459127871	0.636621979814147\\
0.371794818433178	3.19846347688796	0.636621979814147\\
0.18415919211311	3.21472944304186	0.636621979814147\\
7.14983627858601e-16	3.22	0.636621979814147\\
-0.184159192113109	3.21472944304186	0.636621979814147\\
-0.371794818433177	3.19846347688796	0.636621979814147\\
-0.562011353210789	3.17057459127871	0.636621979814147\\
-0.753805232691514	3.13052354585729	0.636621979814147\\
-0.946074439431312	3.07787965246446	0.636621979814147\\
-1.13763233264552	3.0123400664134	0.636621979814147\\
-1.32722567583773	2.93374709295077	0.636621979814147\\
-1.51355655334865	2.84210248932289	0.636621979814147\\
-1.69530759141013	2.73757779259461	0.636621979814147\\
-1.87116963119004	2.62051983608446	0.636621979814147\\
-2.03987076873317	2.49145083171793	0.636621979814147\\
-2.20020550690837	2.35106268044263	0.636621979814147\\
-2.35106268044263	2.20020550690837	0.636621979814147\\
-2.49145083171793	2.03987076873317	0.636621979814147\\
-2.62051983608446	1.87116963119004	0.636621979814147\\
-2.73757779259461	1.69530759141013	0.636621979814147\\
-2.84210248932289	1.51355655334866	0.636621979814147\\
-2.93374709295077	1.32722567583773	0.636621979814147\\
-3.0123400664134	1.13763233264552	0.636621979814147\\
-3.07787965246446	0.946074439431313	0.636621979814147\\
-3.13052354585729	0.753805232691515	0.636621979814147\\
-3.17057459127871	0.56201135321079	0.636621979814147\\
-3.19846347688796	0.371794818433179	0.636621979814147\\
-3.21472944304186	0.18415919211311	0.636621979814147\\
-3.22	9.12151762521325e-16	0.636621979814147\\
-3.21472944304186	-0.184159192113108	0.636621979814147\\
-3.19846347688796	-0.371794818433177	0.636621979814147\\
-3.17057459127871	-0.562011353210788	0.636621979814147\\
-3.13052354585729	-0.753805232691515	0.636621979814147\\
-3.07787965246446	-0.946074439431313	0.636621979814147\\
-3.0123400664134	-1.13763233264552	0.636621979814147\\
-2.93374709295077	-1.32722567583773	0.636621979814147\\
-2.84210248932289	-1.51355655334865	0.636621979814147\\
-2.73757779259461	-1.69530759141013	0.636621979814147\\
-2.62051983608446	-1.87116963119004	0.636621979814147\\
-2.49145083171793	-2.03987076873317	0.636621979814147\\
-2.35106268044263	-2.20020550690837	0.636621979814147\\
-2.20020550690837	-2.35106268044263	0.636621979814147\\
-2.03987076873317	-2.49145083171793	0.636621979814147\\
-1.87116963119004	-2.62051983608446	0.636621979814147\\
-1.69530759141013	-2.73757779259461	0.636621979814147\\
-1.51355655334865	-2.84210248932289	0.636621979814147\\
-1.32722567583773	-2.93374709295077	0.636621979814147\\
-1.13763233264552	-3.0123400664134	0.636621979814147\\
-0.946074439431315	-3.07787965246446	0.636621979814147\\
-0.753805232691517	-3.13052354585729	0.636621979814147\\
-0.56201135321079	-3.17057459127871	0.636621979814147\\
-0.371794818433179	-3.19846347688796	0.636621979814147\\
-0.18415919211311	-3.21472944304186	0.636621979814147\\
-1.10931989718405e-15	-3.22	0.636621979814147\\
0.184159192113108	-3.21472944304186	0.636621979814147\\
0.371794818433177	-3.19846347688796	0.636621979814147\\
0.562011353210788	-3.17057459127871	0.636621979814147\\
0.753805232691515	-3.13052354585729	0.636621979814147\\
0.946074439431313	-3.07787965246446	0.636621979814147\\
1.13763233264552	-3.0123400664134	0.636621979814147\\
1.32722567583773	-2.93374709295077	0.636621979814147\\
1.51355655334865	-2.84210248932289	0.636621979814147\\
1.69530759141013	-2.73757779259461	0.636621979814147\\
1.87116963119004	-2.62051983608446	0.636621979814147\\
2.03987076873317	-2.49145083171793	0.636621979814147\\
2.20020550690837	-2.35106268044263	0.636621979814147\\
2.35106268044263	-2.20020550690837	0.636621979814147\\
2.49145083171793	-2.03987076873317	0.636621979814147\\
2.62051983608446	-1.87116963119004	0.636621979814147\\
2.73757779259461	-1.69530759141013	0.636621979814147\\
2.84210248932289	-1.51355655334865	0.636621979814147\\
2.93374709295077	-1.32722567583773	0.636621979814147\\
3.0123400664134	-1.13763233264552	0.636621979814147\\
3.07787965246446	-0.946074439431315	0.636621979814147\\
3.13052354585729	-0.753805232691517	0.636621979814147\\
3.17057459127871	-0.56201135321079	0.636621979814147\\
3.19846347688796	-0.371794818433179	0.636621979814147\\
3.21472944304186	-0.184159192113111	0.636621979814147\\
3.22	-1.30648803184677e-15	0.636621979814147\\
3.24	0	0.636621979814147\\
3.23469670666324	0.185303038026855	0.636621979814147\\
3.21832970966366	0.374104103019719	0.636621979814147\\
3.19026760116243	0.56550210695744	0.636621979814147\\
3.14996779148373	0.758487252770344	0.636621979814147\\
3.0969969173866	0.951950678185544	0.636621979814147\\
3.03105025316131	1.14469837197872	0.636621979814147\\
2.951969124584	1.33546931357585	0.636621979814147\\
2.85975529981557	1.5229575257297	0.636621979814147\\
2.75458138136849	1.70583745222634	0.636621979814147\\
2.63679635680548	1.88279180281234	0.636621979814147\\
2.50692568160437	2.05254077350791	0.636621979814147\\
2.3656655542342	2.21387137962208	0.636621979814147\\
2.21387137962208	2.3656655542342	0.636621979814147\\
2.05254077350791	2.50692568160437	0.636621979814147\\
1.88279180281234	2.63679635680548	0.636621979814147\\
1.70583745222634	2.75458138136849	0.636621979814147\\
1.5229575257297	2.85975529981557	0.636621979814147\\
1.33546931357585	2.951969124584	0.636621979814147\\
1.14469837197872	3.03105025316131	0.636621979814147\\
0.951950678185545	3.0969969173866	0.636621979814147\\
0.758487252770345	3.14996779148373	0.636621979814147\\
0.56550210695744	3.19026760116243	0.636621979814147\\
0.37410410301972	3.21832970966366	0.636621979814147\\
0.185303038026856	3.23469670666324	0.636621979814147\\
7.19424519957101e-16	3.24	0.636621979814147\\
-0.185303038026855	3.23469670666324	0.636621979814147\\
-0.374104103019719	3.21832970966367	0.636621979814147\\
-0.565502106957439	3.19026760116243	0.636621979814147\\
-0.758487252770343	3.14996779148373	0.636621979814147\\
-0.951950678185544	3.0969969173866	0.636621979814147\\
-1.14469837197872	3.03105025316131	0.636621979814147\\
-1.33546931357585	2.951969124584	0.636621979814147\\
-1.5229575257297	2.85975529981557	0.636621979814147\\
-1.70583745222634	2.75458138136849	0.636621979814147\\
-1.88279180281234	2.63679635680548	0.636621979814147\\
-2.05254077350791	2.50692568160437	0.636621979814147\\
-2.21387137962208	2.3656655542342	0.636621979814147\\
-2.3656655542342	2.21387137962208	0.636621979814147\\
-2.50692568160437	2.05254077350791	0.636621979814147\\
-2.63679635680548	1.88279180281234	0.636621979814147\\
-2.75458138136849	1.70583745222634	0.636621979814147\\
-2.85975529981557	1.5229575257297	0.636621979814147\\
-2.951969124584	1.33546931357585	0.636621979814147\\
-3.03105025316131	1.14469837197872	0.636621979814147\\
-3.0969969173866	0.951950678185545	0.636621979814147\\
-3.14996779148373	0.758487252770345	0.636621979814147\\
-3.19026760116243	0.565502106957441	0.636621979814147\\
-3.21832970966366	0.37410410301972	0.636621979814147\\
-3.23469670666324	0.185303038026856	0.636621979814147\\
-3.24	9.17817301418973e-16	0.636621979814147\\
-3.23469670666324	-0.185303038026854	0.636621979814147\\
-3.21832970966367	-0.374104103019719	0.636621979814147\\
-3.19026760116243	-0.565502106957439	0.636621979814147\\
-3.14996779148373	-0.758487252770344	0.636621979814147\\
-3.0969969173866	-0.951950678185545	0.636621979814147\\
-3.03105025316131	-1.14469837197872	0.636621979814147\\
-2.951969124584	-1.33546931357585	0.636621979814147\\
-2.85975529981557	-1.5229575257297	0.636621979814147\\
-2.75458138136849	-1.70583745222634	0.636621979814147\\
-2.63679635680548	-1.88279180281234	0.636621979814147\\
-2.50692568160437	-2.05254077350791	0.636621979814147\\
-2.3656655542342	-2.21387137962208	0.636621979814147\\
-2.21387137962208	-2.3656655542342	0.636621979814147\\
-2.05254077350791	-2.50692568160437	0.636621979814147\\
-1.88279180281234	-2.63679635680548	0.636621979814147\\
-1.70583745222634	-2.75458138136849	0.636621979814147\\
-1.5229575257297	-2.85975529981557	0.636621979814147\\
-1.33546931357585	-2.951969124584	0.636621979814147\\
-1.14469837197873	-3.03105025316131	0.636621979814147\\
-0.951950678185546	-3.0969969173866	0.636621979814147\\
-0.758487252770346	-3.14996779148373	0.636621979814147\\
-0.565502106957441	-3.19026760116243	0.636621979814147\\
-0.374104103019721	-3.21832970966366	0.636621979814147\\
-0.185303038026856	-3.23469670666324	0.636621979814147\\
-1.11621008288084e-15	-3.24	0.636621979814147\\
0.185303038026854	-3.23469670666324	0.636621979814147\\
0.374104103019719	-3.21832970966367	0.636621979814147\\
0.565502106957439	-3.19026760116243	0.636621979814147\\
0.758487252770344	-3.14996779148373	0.636621979814147\\
0.951950678185545	-3.0969969173866	0.636621979814147\\
1.14469837197872	-3.03105025316131	0.636621979814147\\
1.33546931357585	-2.951969124584	0.636621979814147\\
1.5229575257297	-2.85975529981557	0.636621979814147\\
1.70583745222634	-2.75458138136849	0.636621979814147\\
1.88279180281234	-2.63679635680548	0.636621979814147\\
2.05254077350791	-2.50692568160437	0.636621979814147\\
2.21387137962208	-2.3656655542342	0.636621979814147\\
2.3656655542342	-2.21387137962208	0.636621979814147\\
2.50692568160437	-2.05254077350791	0.636621979814147\\
2.63679635680548	-1.88279180281234	0.636621979814147\\
2.75458138136849	-1.70583745222634	0.636621979814147\\
2.85975529981557	-1.5229575257297	0.636621979814147\\
2.951969124584	-1.33546931357585	0.636621979814147\\
3.03105025316131	-1.14469837197873	0.636621979814147\\
3.0969969173866	-0.951950678185547	0.636621979814147\\
3.14996779148373	-0.758487252770346	0.636621979814147\\
3.19026760116243	-0.565502106957441	0.636621979814147\\
3.21832970966366	-0.374104103019721	0.636621979814147\\
3.23469670666324	-0.185303038026857	0.636621979814147\\
3.24	-1.31460286434272e-15	0.636621979814147\\
3.26	0	0.636621979814147\\
3.25466397028462	0.186446883940601	0.636621979814147\\
3.23819594243937	0.376413387606261	0.636621979814147\\
3.20996061104615	0.568992860704091	0.636621979814147\\
3.16941203711017	0.763169272849174	0.636621979814147\\
3.11611418230874	0.957826916939776	0.636621979814147\\
3.04976043990922	1.15176441131193	0.636621979814147\\
2.97019115621724	1.34371295131398	0.636621979814147\\
2.87740811030826	1.53235849811075	0.636621979814147\\
2.77158497014237	1.71636731304255	0.636621979814147\\
2.6530728775265	1.89441397443464	0.636621979814147\\
2.52240053149082	2.06521077828265	0.636621979814147\\
2.38026842802577	2.2275372523358	0.636621979814147\\
2.2275372523358	2.38026842802577	0.636621979814147\\
2.06521077828265	2.52240053149082	0.636621979814147\\
1.89441397443464	2.6530728775265	0.636621979814147\\
1.71636731304255	2.77158497014237	0.636621979814147\\
1.53235849811075	2.87740811030826	0.636621979814147\\
1.34371295131398	2.97019115621724	0.636621979814147\\
1.15176441131193	3.04976043990922	0.636621979814147\\
0.957826916939777	3.11611418230874	0.636621979814147\\
0.763169272849174	3.16941203711017	0.636621979814147\\
0.568992860704091	3.20996061104615	0.636621979814147\\
0.376413387606261	3.23819594243937	0.636621979814147\\
0.186446883940602	3.25466397028462	0.636621979814147\\
7.23865412055602e-16	3.26	0.636621979814147\\
-0.186446883940601	3.25466397028462	0.636621979814147\\
-0.37641338760626	3.23819594243937	0.636621979814147\\
-0.56899286070409	3.20996061104615	0.636621979814147\\
-0.763169272849173	3.16941203711017	0.636621979814147\\
-0.957826916939776	3.11611418230874	0.636621979814147\\
-1.15176441131192	3.04976043990922	0.636621979814147\\
-1.34371295131397	2.97019115621724	0.636621979814147\\
-1.53235849811075	2.87740811030826	0.636621979814147\\
-1.71636731304255	2.77158497014237	0.636621979814147\\
-1.89441397443463	2.65307287752651	0.636621979814147\\
-2.06521077828265	2.52240053149082	0.636621979814147\\
-2.2275372523358	2.38026842802577	0.636621979814147\\
-2.38026842802577	2.2275372523358	0.636621979814147\\
-2.52240053149082	2.06521077828265	0.636621979814147\\
-2.6530728775265	1.89441397443464	0.636621979814147\\
-2.77158497014237	1.71636731304255	0.636621979814147\\
-2.87740811030826	1.53235849811075	0.636621979814147\\
-2.97019115621724	1.34371295131398	0.636621979814147\\
-3.04976043990922	1.15176441131193	0.636621979814147\\
-3.11611418230874	0.957826916939777	0.636621979814147\\
-3.16941203711017	0.763169272849174	0.636621979814147\\
-3.20996061104615	0.568992860704092	0.636621979814147\\
-3.23819594243937	0.376413387606262	0.636621979814147\\
-3.25466397028462	0.186446883940602	0.636621979814147\\
-3.26	9.23482840316621e-16	0.636621979814147\\
-3.25466397028462	-0.186446883940601	0.636621979814147\\
-3.23819594243937	-0.37641338760626	0.636621979814147\\
-3.20996061104615	-0.56899286070409	0.636621979814147\\
-3.16941203711017	-0.763169272849174	0.636621979814147\\
-3.11611418230874	-0.957826916939777	0.636621979814147\\
-3.04976043990922	-1.15176441131193	0.636621979814147\\
-2.97019115621724	-1.34371295131397	0.636621979814147\\
-2.87740811030826	-1.53235849811075	0.636621979814147\\
-2.77158497014237	-1.71636731304255	0.636621979814147\\
-2.65307287752651	-1.89441397443463	0.636621979814147\\
-2.52240053149082	-2.06521077828265	0.636621979814147\\
-2.38026842802577	-2.2275372523358	0.636621979814147\\
-2.2275372523358	-2.38026842802577	0.636621979814147\\
-2.06521077828265	-2.52240053149082	0.636621979814147\\
-1.89441397443464	-2.6530728775265	0.636621979814147\\
-1.71636731304255	-2.77158497014237	0.636621979814147\\
-1.53235849811075	-2.87740811030826	0.636621979814147\\
-1.34371295131397	-2.97019115621724	0.636621979814147\\
-1.15176441131193	-3.04976043990922	0.636621979814147\\
-0.957826916939778	-3.11611418230874	0.636621979814147\\
-0.763169272849175	-3.16941203711017	0.636621979814147\\
-0.568992860704092	-3.20996061104615	0.636621979814147\\
-0.376413387606262	-3.23819594243937	0.636621979814147\\
-0.186446883940602	-3.25466397028462	0.636621979814147\\
-1.12310026857764e-15	-3.26	0.636621979814147\\
0.1864468839406	-3.25466397028462	0.636621979814147\\
0.37641338760626	-3.23819594243937	0.636621979814147\\
0.56899286070409	-3.20996061104615	0.636621979814147\\
0.763169272849173	-3.16941203711017	0.636621979814147\\
0.957826916939777	-3.11611418230874	0.636621979814147\\
1.15176441131193	-3.04976043990922	0.636621979814147\\
1.34371295131397	-2.97019115621724	0.636621979814147\\
1.53235849811075	-2.87740811030826	0.636621979814147\\
1.71636731304255	-2.77158497014237	0.636621979814147\\
1.89441397443463	-2.65307287752651	0.636621979814147\\
2.06521077828265	-2.52240053149082	0.636621979814147\\
2.2275372523358	-2.38026842802577	0.636621979814147\\
2.38026842802577	-2.2275372523358	0.636621979814147\\
2.52240053149082	-2.06521077828265	0.636621979814147\\
2.6530728775265	-1.89441397443464	0.636621979814147\\
2.77158497014237	-1.71636731304255	0.636621979814147\\
2.87740811030826	-1.53235849811075	0.636621979814147\\
2.97019115621724	-1.34371295131397	0.636621979814147\\
3.04976043990922	-1.15176441131193	0.636621979814147\\
3.11611418230874	-0.957826916939779	0.636621979814147\\
3.16941203711017	-0.763169272849175	0.636621979814147\\
3.20996061104615	-0.568992860704092	0.636621979814147\\
3.23819594243937	-0.376413387606262	0.636621979814147\\
3.25466397028462	-0.186446883940603	0.636621979814147\\
3.26	-1.32271769683866e-15	0.636621979814147\\
3.28	0	0.636621979814147\\
3.274631233906	0.187590729854347	0.636621979814147\\
3.25806217521507	0.378722672192802	0.636621979814147\\
3.22965362092987	0.572483614450741	0.636621979814147\\
3.18885628273662	0.767851292928003	0.636621979814147\\
3.13523144723088	0.963703155694008	0.636621979814147\\
3.06847062665713	1.15883045064513	0.636621979814147\\
2.98841318785047	1.3519565890521	0.636621979814147\\
2.89506092080095	1.5417594704918	0.636621979814147\\
2.78858855891625	1.72689717385877	0.636621979814147\\
2.66934939824753	1.90603614605693	0.636621979814147\\
2.53787538137727	2.07788078305739	0.636621979814147\\
2.39487130181734	2.24120312504952	0.636621979814147\\
2.24120312504952	2.39487130181734	0.636621979814147\\
2.07788078305739	2.53787538137727	0.636621979814147\\
1.90603614605693	2.66934939824753	0.636621979814147\\
1.72689717385877	2.78858855891625	0.636621979814147\\
1.5417594704918	2.89506092080095	0.636621979814147\\
1.3519565890521	2.98841318785047	0.636621979814147\\
1.15883045064513	3.06847062665713	0.636621979814147\\
0.963703155694009	3.13523144723088	0.636621979814147\\
0.767851292928004	3.18885628273662	0.636621979814147\\
0.572483614450742	3.22965362092987	0.636621979814147\\
0.378722672192803	3.25806217521507	0.636621979814147\\
0.187590729854348	3.274631233906	0.636621979814147\\
7.28306304154103e-16	3.28	0.636621979814147\\
-0.187590729854347	3.274631233906	0.636621979814147\\
-0.378722672192802	3.25806217521507	0.636621979814147\\
-0.572483614450741	3.22965362092987	0.636621979814147\\
-0.767851292928002	3.18885628273662	0.636621979814147\\
-0.963703155694007	3.13523144723088	0.636621979814147\\
-1.15883045064513	3.06847062665713	0.636621979814147\\
-1.3519565890521	2.98841318785047	0.636621979814147\\
-1.5417594704918	2.89506092080095	0.636621979814147\\
-1.72689717385877	2.78858855891625	0.636621979814147\\
-1.90603614605693	2.66934939824753	0.636621979814147\\
-2.07788078305739	2.53787538137727	0.636621979814147\\
-2.24120312504952	2.39487130181734	0.636621979814147\\
-2.39487130181734	2.24120312504952	0.636621979814147\\
-2.53787538137727	2.07788078305739	0.636621979814147\\
-2.66934939824753	1.90603614605693	0.636621979814147\\
-2.78858855891625	1.72689717385877	0.636621979814147\\
-2.89506092080095	1.5417594704918	0.636621979814147\\
-2.98841318785047	1.3519565890521	0.636621979814147\\
-3.06847062665713	1.15883045064513	0.636621979814147\\
-3.13523144723088	0.963703155694009	0.636621979814147\\
-3.18885628273662	0.767851292928003	0.636621979814147\\
-3.22965362092987	0.572483614450743	0.636621979814147\\
-3.25806217521507	0.378722672192803	0.636621979814147\\
-3.274631233906	0.187590729854348	0.636621979814147\\
-3.28	9.29148379214269e-16	0.636621979814147\\
-3.274631233906	-0.187590729854347	0.636621979814147\\
-3.25806217521507	-0.378722672192802	0.636621979814147\\
-3.22965362092987	-0.572483614450741	0.636621979814147\\
-3.18885628273662	-0.767851292928003	0.636621979814147\\
-3.13523144723087	-0.963703155694008	0.636621979814147\\
-3.06847062665713	-1.15883045064513	0.636621979814147\\
-2.98841318785047	-1.3519565890521	0.636621979814147\\
-2.89506092080095	-1.5417594704918	0.636621979814147\\
-2.78858855891625	-1.72689717385876	0.636621979814147\\
-2.66934939824753	-1.90603614605693	0.636621979814147\\
-2.53787538137727	-2.07788078305739	0.636621979814147\\
-2.39487130181734	-2.24120312504952	0.636621979814147\\
-2.24120312504952	-2.39487130181734	0.636621979814147\\
-2.07788078305739	-2.53787538137727	0.636621979814147\\
-1.90603614605693	-2.66934939824753	0.636621979814147\\
-1.72689717385877	-2.78858855891625	0.636621979814147\\
-1.5417594704918	-2.89506092080095	0.636621979814147\\
-1.3519565890521	-2.98841318785047	0.636621979814147\\
-1.15883045064513	-3.06847062665713	0.636621979814147\\
-0.96370315569401	-3.13523144723087	0.636621979814147\\
-0.767851292928005	-3.18885628273662	0.636621979814147\\
-0.572483614450743	-3.22965362092987	0.636621979814147\\
-0.378722672192803	-3.25806217521507	0.636621979814147\\
-0.187590729854349	-3.274631233906	0.636621979814147\\
-1.12999045427443e-15	-3.28	0.636621979814147\\
0.187590729854346	-3.274631233906	0.636621979814147\\
0.378722672192801	-3.25806217521507	0.636621979814147\\
0.572483614450741	-3.22965362092987	0.636621979814147\\
0.767851292928003	-3.18885628273662	0.636621979814147\\
0.963703155694008	-3.13523144723087	0.636621979814147\\
1.15883045064513	-3.06847062665713	0.636621979814147\\
1.3519565890521	-2.98841318785047	0.636621979814147\\
1.5417594704918	-2.89506092080095	0.636621979814147\\
1.72689717385876	-2.78858855891625	0.636621979814147\\
1.90603614605693	-2.66934939824753	0.636621979814147\\
2.07788078305739	-2.53787538137727	0.636621979814147\\
2.24120312504952	-2.39487130181734	0.636621979814147\\
2.39487130181734	-2.24120312504952	0.636621979814147\\
2.53787538137727	-2.07788078305739	0.636621979814147\\
2.66934939824753	-1.90603614605693	0.636621979814147\\
2.78858855891625	-1.72689717385877	0.636621979814147\\
2.89506092080095	-1.5417594704918	0.636621979814147\\
2.98841318785047	-1.3519565890521	0.636621979814147\\
3.06847062665713	-1.15883045064513	0.636621979814147\\
3.13523144723087	-0.96370315569401	0.636621979814147\\
3.18885628273662	-0.767851292928005	0.636621979814147\\
3.22965362092987	-0.572483614450743	0.636621979814147\\
3.25806217521507	-0.378722672192804	0.636621979814147\\
3.274631233906	-0.187590729854349	0.636621979814147\\
3.28	-1.3308325293346e-15	0.636621979814147\\
3.3	0	0.636621979814147\\
3.29459849752738	0.188734575768093	0.636621979814147\\
3.27792840799077	0.381031956779343	0.636621979814147\\
3.24934663081359	0.575974368197392	0.636621979814147\\
3.20830052836306	0.772533313006832	0.636621979814147\\
3.15434871215302	0.96957939444824	0.636621979814147\\
3.08718081340504	1.16589648997833	0.636621979814147\\
3.00663521948371	1.36020022679022	0.636621979814147\\
2.91271373129364	1.55116044287284	0.636621979814147\\
2.80559214769013	1.73742703467498	0.636621979814147\\
2.68562591896855	1.91765831767923	0.636621979814147\\
2.55335023126371	2.09055078783213	0.636621979814147\\
2.40947417560891	2.25486899776323	0.636621979814147\\
2.25486899776323	2.40947417560891	0.636621979814147\\
2.09055078783213	2.55335023126371	0.636621979814147\\
1.91765831767923	2.68562591896855	0.636621979814147\\
1.73742703467498	2.80559214769013	0.636621979814147\\
1.55116044287284	2.91271373129364	0.636621979814147\\
1.36020022679022	3.00663521948371	0.636621979814147\\
1.16589648997833	3.08718081340504	0.636621979814147\\
0.96957939444824	3.15434871215301	0.636621979814147\\
0.772533313006833	3.20830052836306	0.636621979814147\\
0.575974368197393	3.24934663081359	0.636621979814147\\
0.381031956779344	3.27792840799077	0.636621979814147\\
0.188734575768094	3.29459849752738	0.636621979814147\\
7.32747196252603e-16	3.3	0.636621979814147\\
-0.188734575768093	3.29459849752738	0.636621979814147\\
-0.381031956779343	3.27792840799077	0.636621979814147\\
-0.575974368197392	3.24934663081359	0.636621979814147\\
-0.772533313006831	3.20830052836306	0.636621979814147\\
-0.969579394448239	3.15434871215301	0.636621979814147\\
-1.16589648997833	3.08718081340504	0.636621979814147\\
-1.36020022679022	3.00663521948371	0.636621979814147\\
-1.55116044287284	2.91271373129364	0.636621979814147\\
-1.73742703467498	2.80559214769013	0.636621979814147\\
-1.91765831767923	2.68562591896855	0.636621979814147\\
-2.09055078783213	2.55335023126371	0.636621979814147\\
-2.25486899776323	2.40947417560891	0.636621979814147\\
-2.40947417560891	2.25486899776323	0.636621979814147\\
-2.55335023126371	2.09055078783213	0.636621979814147\\
-2.68562591896855	1.91765831767923	0.636621979814147\\
-2.80559214769013	1.73742703467498	0.636621979814147\\
-2.91271373129364	1.55116044287285	0.636621979814147\\
-3.00663521948371	1.36020022679022	0.636621979814147\\
-3.08718081340504	1.16589648997833	0.636621979814147\\
-3.15434871215301	0.96957939444824	0.636621979814147\\
-3.20830052836306	0.772533313006833	0.636621979814147\\
-3.24934663081359	0.575974368197393	0.636621979814147\\
-3.27792840799077	0.381031956779345	0.636621979814147\\
-3.29459849752738	0.188734575768094	0.636621979814147\\
-3.3	9.34813918111917e-16	0.636621979814147\\
-3.29459849752738	-0.188734575768093	0.636621979814147\\
-3.27792840799077	-0.381031956779343	0.636621979814147\\
-3.24934663081359	-0.575974368197392	0.636621979814147\\
-3.20830052836306	-0.772533313006832	0.636621979814147\\
-3.15434871215301	-0.96957939444824	0.636621979814147\\
-3.08718081340504	-1.16589648997833	0.636621979814147\\
-3.00663521948371	-1.36020022679022	0.636621979814147\\
-2.91271373129364	-1.55116044287284	0.636621979814147\\
-2.80559214769013	-1.73742703467498	0.636621979814147\\
-2.68562591896855	-1.91765831767923	0.636621979814147\\
-2.55335023126371	-2.09055078783213	0.636621979814147\\
-2.40947417560891	-2.25486899776323	0.636621979814147\\
-2.25486899776323	-2.40947417560891	0.636621979814147\\
-2.09055078783213	-2.55335023126371	0.636621979814147\\
-1.91765831767923	-2.68562591896855	0.636621979814147\\
-1.73742703467498	-2.80559214769013	0.636621979814147\\
-1.55116044287284	-2.91271373129364	0.636621979814147\\
-1.36020022679022	-3.00663521948371	0.636621979814147\\
-1.16589648997833	-3.08718081340504	0.636621979814147\\
-0.969579394448242	-3.15434871215301	0.636621979814147\\
-0.772533313006834	-3.20830052836306	0.636621979814147\\
-0.575974368197394	-3.24934663081359	0.636621979814147\\
-0.381031956779345	-3.27792840799077	0.636621979814147\\
-0.188734575768095	-3.29459849752738	0.636621979814147\\
-1.13688063997123e-15	-3.3	0.636621979814147\\
0.188734575768092	-3.29459849752738	0.636621979814147\\
0.381031956779343	-3.27792840799077	0.636621979814147\\
0.575974368197392	-3.24934663081359	0.636621979814147\\
0.772533313006832	-3.20830052836306	0.636621979814147\\
0.96957939444824	-3.15434871215301	0.636621979814147\\
1.16589648997833	-3.08718081340504	0.636621979814147\\
1.36020022679022	-3.00663521948371	0.636621979814147\\
1.55116044287284	-2.91271373129364	0.636621979814147\\
1.73742703467498	-2.80559214769013	0.636621979814147\\
1.91765831767923	-2.68562591896855	0.636621979814147\\
2.09055078783213	-2.55335023126371	0.636621979814147\\
2.25486899776323	-2.40947417560891	0.636621979814147\\
2.40947417560891	-2.25486899776323	0.636621979814147\\
2.55335023126371	-2.09055078783213	0.636621979814147\\
2.68562591896855	-1.91765831767923	0.636621979814147\\
2.80559214769013	-1.73742703467498	0.636621979814147\\
2.91271373129364	-1.55116044287284	0.636621979814147\\
3.00663521948371	-1.36020022679022	0.636621979814147\\
3.08718081340504	-1.16589648997833	0.636621979814147\\
3.15434871215301	-0.969579394448242	0.636621979814147\\
3.20830052836306	-0.772533313006834	0.636621979814147\\
3.24934663081359	-0.575974368197394	0.636621979814147\\
3.27792840799077	-0.381031956779345	0.636621979814147\\
3.29459849752738	-0.188734575768095	0.636621979814147\\
3.3	-1.33894736183054e-15	0.636621979814147\\
3.32	0	0.636621979814147\\
3.31456576114875	0.189878421681839	0.636621979814147\\
3.29779464076647	0.383341241365885	0.636621979814147\\
3.26903964069731	0.579465121944043	0.636621979814147\\
3.2277447739895	0.777215333085661	0.636621979814147\\
3.17346597707515	0.975455633202472	0.636621979814147\\
3.10589100015295	1.17296252931153	0.636621979814147\\
3.02485725111694	1.36844386452834	0.636621979814147\\
2.93036654178633	1.56056141525389	0.636621979814147\\
2.82259573646401	1.74795689549119	0.636621979814147\\
2.70190243968957	1.92928048930153	0.636621979814147\\
2.56882508115016	2.10322079260687	0.636621979814147\\
2.42407704940048	2.26853487047695	0.636621979814147\\
2.26853487047695	2.42407704940048	0.636621979814147\\
2.10322079260687	2.56882508115016	0.636621979814147\\
1.92928048930153	2.70190243968957	0.636621979814147\\
1.74795689549119	2.82259573646401	0.636621979814147\\
1.56056141525389	2.93036654178633	0.636621979814147\\
1.36844386452834	3.02485725111694	0.636621979814147\\
1.17296252931153	3.10589100015295	0.636621979814147\\
0.975455633202472	3.17346597707515	0.636621979814147\\
0.777215333085662	3.2277447739895	0.636621979814147\\
0.579465121944044	3.26903964069731	0.636621979814147\\
0.383341241365886	3.29779464076647	0.636621979814147\\
0.18987842168184	3.31456576114875	0.636621979814147\\
7.37188088351104e-16	3.32	0.636621979814147\\
-0.189878421681839	3.31456576114875	0.636621979814147\\
-0.383341241365885	3.29779464076647	0.636621979814147\\
-0.579465121944043	3.26903964069731	0.636621979814147\\
-0.777215333085661	3.2277447739895	0.636621979814147\\
-0.975455633202471	3.17346597707515	0.636621979814147\\
-1.17296252931153	3.10589100015295	0.636621979814147\\
-1.36844386452834	3.02485725111694	0.636621979814147\\
-1.56056141525389	2.93036654178633	0.636621979814147\\
-1.74795689549119	2.82259573646401	0.636621979814147\\
-1.92928048930153	2.70190243968957	0.636621979814147\\
-2.10322079260687	2.56882508115016	0.636621979814147\\
-2.26853487047695	2.42407704940048	0.636621979814147\\
-2.42407704940048	2.26853487047695	0.636621979814147\\
-2.56882508115016	2.10322079260687	0.636621979814147\\
-2.70190243968957	1.92928048930153	0.636621979814147\\
-2.82259573646401	1.74795689549119	0.636621979814147\\
-2.93036654178633	1.56056141525389	0.636621979814147\\
-3.02485725111694	1.36844386452834	0.636621979814147\\
-3.10589100015295	1.17296252931153	0.636621979814147\\
-3.17346597707515	0.975455633202472	0.636621979814147\\
-3.2277447739895	0.777215333085662	0.636621979814147\\
-3.26903964069731	0.579465121944044	0.636621979814147\\
-3.29779464076647	0.383341241365886	0.636621979814147\\
-3.31456576114875	0.18987842168184	0.636621979814147\\
-3.32	9.40479457009565e-16	0.636621979814147\\
-3.31456576114875	-0.189878421681839	0.636621979814147\\
-3.29779464076647	-0.383341241365885	0.636621979814147\\
-3.26903964069731	-0.579465121944043	0.636621979814147\\
-3.2277447739895	-0.777215333085661	0.636621979814147\\
-3.17346597707515	-0.975455633202472	0.636621979814147\\
-3.10589100015295	-1.17296252931153	0.636621979814147\\
-3.02485725111694	-1.36844386452834	0.636621979814147\\
-2.93036654178633	-1.56056141525389	0.636621979814147\\
-2.82259573646401	-1.74795689549119	0.636621979814147\\
-2.70190243968957	-1.92928048930153	0.636621979814147\\
-2.56882508115016	-2.10322079260687	0.636621979814147\\
-2.42407704940048	-2.26853487047695	0.636621979814147\\
-2.26853487047695	-2.42407704940048	0.636621979814147\\
-2.10322079260687	-2.56882508115016	0.636621979814147\\
-1.92928048930153	-2.70190243968957	0.636621979814147\\
-1.74795689549119	-2.82259573646401	0.636621979814147\\
-1.56056141525389	-2.93036654178633	0.636621979814147\\
-1.36844386452834	-3.02485725111694	0.636621979814147\\
-1.17296252931153	-3.10589100015295	0.636621979814147\\
-0.975455633202474	-3.17346597707515	0.636621979814147\\
-0.777215333085663	-3.2277447739895	0.636621979814147\\
-0.579465121944044	-3.26903964069731	0.636621979814147\\
-0.383341241365886	-3.29779464076647	0.636621979814147\\
-0.189878421681841	-3.31456576114875	0.636621979814147\\
-1.14377082566803e-15	-3.32	0.636621979814147\\
0.189878421681838	-3.31456576114875	0.636621979814147\\
0.383341241365884	-3.29779464076647	0.636621979814147\\
0.579465121944042	-3.26903964069731	0.636621979814147\\
0.777215333085661	-3.2277447739895	0.636621979814147\\
0.975455633202472	-3.17346597707515	0.636621979814147\\
1.17296252931153	-3.10589100015295	0.636621979814147\\
1.36844386452834	-3.02485725111694	0.636621979814147\\
1.56056141525389	-2.93036654178633	0.636621979814147\\
1.74795689549119	-2.82259573646401	0.636621979814147\\
1.92928048930153	-2.70190243968957	0.636621979814147\\
2.10322079260687	-2.56882508115016	0.636621979814147\\
2.26853487047695	-2.42407704940048	0.636621979814147\\
2.42407704940048	-2.26853487047695	0.636621979814147\\
2.56882508115016	-2.10322079260687	0.636621979814147\\
2.70190243968957	-1.92928048930153	0.636621979814147\\
2.82259573646401	-1.74795689549119	0.636621979814147\\
2.93036654178633	-1.56056141525389	0.636621979814147\\
3.02485725111694	-1.36844386452834	0.636621979814147\\
3.10589100015295	-1.17296252931153	0.636621979814147\\
3.17346597707515	-0.975455633202474	0.636621979814147\\
3.2277447739895	-0.777215333085663	0.636621979814147\\
3.26903964069731	-0.579465121944045	0.636621979814147\\
3.29779464076647	-0.383341241365887	0.636621979814147\\
3.31456576114875	-0.189878421681841	0.636621979814147\\
3.32	-1.34706219432649e-15	0.636621979814147\\
3.34	0	0.636621979814147\\
3.33453302477013	0.191022267595585	0.636621979814147\\
3.31766087354217	0.385650525952426	0.636621979814147\\
3.28873265058103	0.582955875690694	0.636621979814147\\
3.24718901961595	0.781897353164491	0.636621979814147\\
3.19258324199729	0.981331871956703	0.636621979814147\\
3.12460118690086	1.18002856864473	0.636621979814147\\
3.04307928275018	1.37668750226647	0.636621979814147\\
2.94801935227902	1.56996238763494	0.636621979814147\\
2.83959932523789	1.7584867563074	0.636621979814147\\
2.71817896041059	1.94090266092383	0.636621979814147\\
2.58429993103661	2.11589079738161	0.636621979814147\\
2.43867992319205	2.28220074319067	0.636621979814147\\
2.28220074319067	2.43867992319205	0.636621979814147\\
2.11589079738161	2.58429993103661	0.636621979814147\\
1.94090266092383	2.71817896041059	0.636621979814147\\
1.7584867563074	2.83959932523789	0.636621979814147\\
1.56996238763494	2.94801935227902	0.636621979814147\\
1.37668750226647	3.04307928275018	0.636621979814147\\
1.18002856864473	3.12460118690086	0.636621979814147\\
0.981331871956704	3.19258324199729	0.636621979814147\\
0.781897353164492	3.24718901961595	0.636621979814147\\
0.582955875690695	3.28873265058103	0.636621979814147\\
0.385650525952427	3.31766087354217	0.636621979814147\\
0.191022267595586	3.33453302477013	0.636621979814147\\
7.41628980449605e-16	3.34	0.636621979814147\\
-0.191022267595585	3.33453302477013	0.636621979814147\\
-0.385650525952426	3.31766087354217	0.636621979814147\\
-0.582955875690694	3.28873265058103	0.636621979814147\\
-0.78189735316449	3.24718901961595	0.636621979814147\\
-0.981331871956703	3.19258324199729	0.636621979814147\\
-1.18002856864473	3.12460118690086	0.636621979814147\\
-1.37668750226646	3.04307928275018	0.636621979814147\\
-1.56996238763494	2.94801935227902	0.636621979814147\\
-1.7584867563074	2.83959932523789	0.636621979814147\\
-1.94090266092383	2.71817896041059	0.636621979814147\\
-2.11589079738161	2.58429993103661	0.636621979814147\\
-2.28220074319067	2.43867992319205	0.636621979814147\\
-2.43867992319205	2.28220074319067	0.636621979814147\\
-2.58429993103661	2.11589079738161	0.636621979814147\\
-2.71817896041059	1.94090266092383	0.636621979814147\\
-2.83959932523789	1.7584867563074	0.636621979814147\\
-2.94801935227902	1.56996238763494	0.636621979814147\\
-3.04307928275018	1.37668750226647	0.636621979814147\\
-3.12460118690086	1.18002856864473	0.636621979814147\\
-3.19258324199729	0.981331871956704	0.636621979814147\\
-3.24718901961595	0.781897353164491	0.636621979814147\\
-3.28873265058103	0.582955875690695	0.636621979814147\\
-3.31766087354217	0.385650525952428	0.636621979814147\\
-3.33453302477013	0.191022267595586	0.636621979814147\\
-3.34	9.46144995907213e-16	0.636621979814147\\
-3.33453302477013	-0.191022267595585	0.636621979814147\\
-3.31766087354217	-0.385650525952426	0.636621979814147\\
-3.28873265058103	-0.582955875690694	0.636621979814147\\
-3.24718901961595	-0.781897353164491	0.636621979814147\\
-3.19258324199729	-0.981331871956704	0.636621979814147\\
-3.12460118690086	-1.18002856864473	0.636621979814147\\
-3.04307928275017	-1.37668750226646	0.636621979814147\\
-2.94801935227902	-1.56996238763494	0.636621979814147\\
-2.83959932523789	-1.7584867563074	0.636621979814147\\
-2.71817896041059	-1.94090266092383	0.636621979814147\\
-2.58429993103661	-2.11589079738161	0.636621979814147\\
-2.43867992319205	-2.28220074319067	0.636621979814147\\
-2.28220074319067	-2.43867992319205	0.636621979814147\\
-2.11589079738161	-2.58429993103661	0.636621979814147\\
-1.94090266092383	-2.71817896041059	0.636621979814147\\
-1.7584867563074	-2.83959932523789	0.636621979814147\\
-1.56996238763494	-2.94801935227902	0.636621979814147\\
-1.37668750226646	-3.04307928275018	0.636621979814147\\
-1.18002856864474	-3.12460118690086	0.636621979814147\\
-0.981331871956705	-3.19258324199729	0.636621979814147\\
-0.781897353164492	-3.24718901961595	0.636621979814147\\
-0.582955875690695	-3.28873265058103	0.636621979814147\\
-0.385650525952428	-3.31766087354217	0.636621979814147\\
-0.191022267595587	-3.33453302477013	0.636621979814147\\
-1.15066101136482e-15	-3.34	0.636621979814147\\
0.191022267595584	-3.33453302477013	0.636621979814147\\
0.385650525952426	-3.31766087354217	0.636621979814147\\
0.582955875690693	-3.28873265058103	0.636621979814147\\
0.78189735316449	-3.24718901961595	0.636621979814147\\
0.981331871956704	-3.19258324199729	0.636621979814147\\
1.18002856864473	-3.12460118690086	0.636621979814147\\
1.37668750226646	-3.04307928275017	0.636621979814147\\
1.56996238763494	-2.94801935227902	0.636621979814147\\
1.7584867563074	-2.83959932523789	0.636621979814147\\
1.94090266092383	-2.71817896041059	0.636621979814147\\
2.11589079738161	-2.58429993103661	0.636621979814147\\
2.28220074319067	-2.43867992319205	0.636621979814147\\
2.43867992319205	-2.28220074319067	0.636621979814147\\
2.58429993103661	-2.11589079738161	0.636621979814147\\
2.71817896041059	-1.94090266092383	0.636621979814147\\
2.83959932523789	-1.7584867563074	0.636621979814147\\
2.94801935227902	-1.56996238763494	0.636621979814147\\
3.04307928275018	-1.37668750226647	0.636621979814147\\
3.12460118690086	-1.18002856864474	0.636621979814147\\
3.19258324199729	-0.981331871956706	0.636621979814147\\
3.24718901961595	-0.781897353164492	0.636621979814147\\
3.28873265058103	-0.582955875690695	0.636621979814147\\
3.31766087354217	-0.385650525952428	0.636621979814147\\
3.33453302477013	-0.191022267595587	0.636621979814147\\
3.34	-1.35517702682243e-15	0.636621979814147\\
3.36	0	0.636621979814147\\
3.35450028839151	0.192166113509331	0.636621979814147\\
3.33752710631788	0.387959810538968	0.636621979814147\\
3.30842566046474	0.586446629437345	0.636621979814147\\
3.26663326524239	0.78657937324332	0.636621979814147\\
3.21170050691943	0.987208110710935	0.636621979814147\\
3.14331137364877	1.18709460797794	0.636621979814147\\
3.06130131438341	1.38493114000459	0.636621979814147\\
2.96567216277171	1.57936336001599	0.636621979814147\\
2.85660291401177	1.76901661712361	0.636621979814147\\
2.73445548113161	1.95252483254613	0.636621979814147\\
2.59977478092305	2.12856080215635	0.636621979814147\\
2.45328279698362	2.29586661590438	0.636621979814147\\
2.29586661590438	2.45328279698362	0.636621979814147\\
2.12856080215635	2.59977478092305	0.636621979814147\\
1.95252483254613	2.73445548113161	0.636621979814147\\
1.76901661712361	2.85660291401177	0.636621979814147\\
1.57936336001599	2.96567216277171	0.636621979814147\\
1.38493114000459	3.06130131438341	0.636621979814147\\
1.18709460797794	3.14331137364877	0.636621979814147\\
0.987208110710936	3.21170050691943	0.636621979814147\\
0.786579373243321	3.26663326524239	0.636621979814147\\
0.586446629437345	3.30842566046474	0.636621979814147\\
0.387959810538969	3.33752710631787	0.636621979814147\\
0.192166113509332	3.35450028839151	0.636621979814147\\
7.46069872548105e-16	3.36	0.636621979814147\\
-0.192166113509331	3.35450028839151	0.636621979814147\\
-0.387959810538968	3.33752710631788	0.636621979814147\\
-0.586446629437345	3.30842566046474	0.636621979814147\\
-0.786579373243319	3.26663326524239	0.636621979814147\\
-0.987208110710935	3.21170050691943	0.636621979814147\\
-1.18709460797793	3.14331137364877	0.636621979814147\\
-1.38493114000459	3.06130131438341	0.636621979814147\\
-1.57936336001599	2.96567216277171	0.636621979814147\\
-1.76901661712361	2.85660291401177	0.636621979814147\\
-1.95252483254613	2.73445548113161	0.636621979814147\\
-2.12856080215635	2.59977478092305	0.636621979814147\\
-2.29586661590438	2.45328279698362	0.636621979814147\\
-2.45328279698362	2.29586661590438	0.636621979814147\\
-2.59977478092305	2.12856080215635	0.636621979814147\\
-2.73445548113161	1.95252483254613	0.636621979814147\\
-2.85660291401177	1.76901661712361	0.636621979814147\\
-2.96567216277171	1.57936336001599	0.636621979814147\\
-3.06130131438341	1.38493114000459	0.636621979814147\\
-3.14331137364877	1.18709460797794	0.636621979814147\\
-3.21170050691943	0.987208110710936	0.636621979814147\\
-3.26663326524239	0.786579373243321	0.636621979814147\\
-3.30842566046474	0.586446629437346	0.636621979814147\\
-3.33752710631788	0.387959810538969	0.636621979814147\\
-3.35450028839151	0.192166113509332	0.636621979814147\\
-3.36	9.51810534804861e-16	0.636621979814147\\
-3.35450028839151	-0.192166113509331	0.636621979814147\\
-3.33752710631788	-0.387959810538968	0.636621979814147\\
-3.30842566046474	-0.586446629437344	0.636621979814147\\
-3.26663326524239	-0.78657937324332	0.636621979814147\\
-3.21170050691943	-0.987208110710935	0.636621979814147\\
-3.14331137364877	-1.18709460797794	0.636621979814147\\
-3.06130131438341	-1.38493114000459	0.636621979814147\\
-2.96567216277171	-1.57936336001599	0.636621979814147\\
-2.85660291401177	-1.76901661712361	0.636621979814147\\
-2.73445548113161	-1.95252483254613	0.636621979814147\\
-2.59977478092305	-2.12856080215635	0.636621979814147\\
-2.45328279698362	-2.29586661590438	0.636621979814147\\
-2.29586661590438	-2.45328279698362	0.636621979814147\\
-2.12856080215635	-2.59977478092305	0.636621979814147\\
-1.95252483254613	-2.73445548113161	0.636621979814147\\
-1.76901661712361	-2.85660291401177	0.636621979814147\\
-1.57936336001599	-2.96567216277171	0.636621979814147\\
-1.38493114000459	-3.06130131438341	0.636621979814147\\
-1.18709460797794	-3.14331137364877	0.636621979814147\\
-0.987208110710937	-3.21170050691943	0.636621979814147\\
-0.786579373243322	-3.26663326524239	0.636621979814147\\
-0.586446629437346	-3.30842566046474	0.636621979814147\\
-0.387959810538969	-3.33752710631788	0.636621979814147\\
-0.192166113509333	-3.35450028839151	0.636621979814147\\
-1.15755119706162e-15	-3.36	0.636621979814147\\
0.19216611350933	-3.35450028839151	0.636621979814147\\
0.387959810538967	-3.33752710631788	0.636621979814147\\
0.586446629437344	-3.30842566046474	0.636621979814147\\
0.78657937324332	-3.26663326524239	0.636621979814147\\
0.987208110710935	-3.21170050691943	0.636621979814147\\
1.18709460797794	-3.14331137364877	0.636621979814147\\
1.38493114000459	-3.06130131438341	0.636621979814147\\
1.57936336001599	-2.96567216277171	0.636621979814147\\
1.76901661712361	-2.85660291401177	0.636621979814147\\
1.95252483254613	-2.73445548113161	0.636621979814147\\
2.12856080215635	-2.59977478092305	0.636621979814147\\
2.29586661590438	-2.45328279698362	0.636621979814147\\
2.45328279698362	-2.29586661590438	0.636621979814147\\
2.59977478092305	-2.12856080215635	0.636621979814147\\
2.73445548113161	-1.95252483254613	0.636621979814147\\
2.85660291401177	-1.76901661712361	0.636621979814147\\
2.96567216277171	-1.57936336001599	0.636621979814147\\
3.06130131438341	-1.38493114000459	0.636621979814147\\
3.14331137364877	-1.18709460797794	0.636621979814147\\
3.21170050691943	-0.987208110710937	0.636621979814147\\
3.26663326524239	-0.786579373243322	0.636621979814147\\
3.30842566046474	-0.586446629437346	0.636621979814147\\
3.33752710631788	-0.38795981053897	0.636621979814147\\
3.35450028839151	-0.192166113509333	0.636621979814147\\
3.36	-1.36329185931837e-15	0.636621979814147\\
3.38	0	0.636621979814147\\
3.37446755201289	0.193309959423077	0.636621979814147\\
3.35739333909358	0.390269095125509	0.636621979814147\\
3.32811867034846	0.589937383183996	0.636621979814147\\
3.28607751086883	0.791261393322149	0.636621979814147\\
3.23081777184157	0.993084349465167	0.636621979814147\\
3.16202156039668	1.19416064731114	0.636621979814147\\
3.07952334601664	1.39317477774271	0.636621979814147\\
2.98332497326439	1.58876433239703	0.636621979814147\\
2.87360650278565	1.77954647793983	0.636621979814147\\
2.75073200185263	1.96414700416843	0.636621979814147\\
2.6152496308095	2.14123080693109	0.636621979814147\\
2.46788567077519	2.3095324886181	0.636621979814147\\
2.3095324886181	2.46788567077519	0.636621979814147\\
2.14123080693109	2.6152496308095	0.636621979814147\\
1.96414700416843	2.75073200185263	0.636621979814147\\
1.77954647793983	2.87360650278565	0.636621979814147\\
1.58876433239704	2.98332497326439	0.636621979814147\\
1.39317477774271	3.07952334601664	0.636621979814147\\
1.19416064731114	3.16202156039668	0.636621979814147\\
0.993084349465167	3.23081777184157	0.636621979814147\\
0.79126139332215	3.28607751086883	0.636621979814147\\
0.589937383183996	3.32811867034846	0.636621979814147\\
0.39026909512551	3.35739333909358	0.636621979814147\\
0.193309959423078	3.37446755201289	0.636621979814147\\
7.50510764646606e-16	3.38	0.636621979814147\\
-0.193309959423077	3.37446755201289	0.636621979814147\\
-0.390269095125509	3.35739333909358	0.636621979814147\\
-0.589937383183995	3.32811867034846	0.636621979814147\\
-0.791261393322148	3.28607751086883	0.636621979814147\\
-0.993084349465166	3.23081777184157	0.636621979814147\\
-1.19416064731114	3.16202156039668	0.636621979814147\\
-1.39317477774271	3.07952334601664	0.636621979814147\\
-1.58876433239703	2.98332497326439	0.636621979814147\\
-1.77954647793983	2.87360650278565	0.636621979814147\\
-1.96414700416842	2.75073200185263	0.636621979814147\\
-2.14123080693109	2.6152496308095	0.636621979814147\\
-2.3095324886181	2.46788567077519	0.636621979814147\\
-2.46788567077519	2.3095324886181	0.636621979814147\\
-2.6152496308095	2.14123080693109	0.636621979814147\\
-2.75073200185263	1.96414700416843	0.636621979814147\\
-2.87360650278565	1.77954647793983	0.636621979814147\\
-2.98332497326439	1.58876433239704	0.636621979814147\\
-3.07952334601664	1.39317477774271	0.636621979814147\\
-3.16202156039668	1.19416064731114	0.636621979814147\\
-3.23081777184157	0.993084349465167	0.636621979814147\\
-3.28607751086883	0.79126139332215	0.636621979814147\\
-3.32811867034846	0.589937383183997	0.636621979814147\\
-3.35739333909358	0.390269095125511	0.636621979814147\\
-3.37446755201289	0.193309959423078	0.636621979814147\\
-3.38	9.57476073702509e-16	0.636621979814147\\
-3.37446755201289	-0.193309959423077	0.636621979814147\\
-3.35739333909358	-0.390269095125509	0.636621979814147\\
-3.32811867034846	-0.589937383183995	0.636621979814147\\
-3.28607751086883	-0.791261393322149	0.636621979814147\\
-3.23081777184157	-0.993084349465167	0.636621979814147\\
-3.16202156039668	-1.19416064731114	0.636621979814147\\
-3.07952334601664	-1.39317477774271	0.636621979814147\\
-2.98332497326439	-1.58876433239703	0.636621979814147\\
-2.87360650278565	-1.77954647793982	0.636621979814147\\
-2.75073200185263	-1.96414700416842	0.636621979814147\\
-2.6152496308095	-2.14123080693109	0.636621979814147\\
-2.46788567077519	-2.3095324886181	0.636621979814147\\
-2.3095324886181	-2.46788567077519	0.636621979814147\\
-2.14123080693109	-2.6152496308095	0.636621979814147\\
-1.96414700416843	-2.75073200185263	0.636621979814147\\
-1.77954647793983	-2.87360650278565	0.636621979814147\\
-1.58876433239703	-2.9833249732644	0.636621979814147\\
-1.39317477774271	-3.07952334601664	0.636621979814147\\
-1.19416064731114	-3.16202156039668	0.636621979814147\\
-0.993084349465169	-3.23081777184157	0.636621979814147\\
-0.791261393322151	-3.28607751086883	0.636621979814147\\
-0.589937383183997	-3.32811867034846	0.636621979814147\\
-0.390269095125511	-3.35739333909358	0.636621979814147\\
-0.193309959423079	-3.37446755201289	0.636621979814147\\
-1.16444138275841e-15	-3.38	0.636621979814147\\
0.193309959423076	-3.37446755201289	0.636621979814147\\
0.390269095125509	-3.35739333909358	0.636621979814147\\
0.589937383183995	-3.32811867034846	0.636621979814147\\
0.791261393322149	-3.28607751086883	0.636621979814147\\
0.993084349465167	-3.23081777184157	0.636621979814147\\
1.19416064731114	-3.16202156039668	0.636621979814147\\
1.39317477774271	-3.07952334601664	0.636621979814147\\
1.58876433239703	-2.98332497326439	0.636621979814147\\
1.77954647793982	-2.87360650278565	0.636621979814147\\
1.96414700416842	-2.75073200185263	0.636621979814147\\
2.14123080693109	-2.6152496308095	0.636621979814147\\
2.3095324886181	-2.46788567077519	0.636621979814147\\
2.46788567077519	-2.3095324886181	0.636621979814147\\
2.6152496308095	-2.14123080693109	0.636621979814147\\
2.75073200185263	-1.96414700416843	0.636621979814147\\
2.87360650278565	-1.77954647793983	0.636621979814147\\
2.9833249732644	-1.58876433239703	0.636621979814147\\
3.07952334601664	-1.39317477774271	0.636621979814147\\
3.16202156039668	-1.19416064731114	0.636621979814147\\
3.23081777184157	-0.993084349465169	0.636621979814147\\
3.28607751086883	-0.791261393322151	0.636621979814147\\
3.32811867034846	-0.589937383183997	0.636621979814147\\
3.35739333909358	-0.390269095125511	0.636621979814147\\
3.37446755201289	-0.193309959423079	0.636621979814147\\
3.38	-1.37140669181431e-15	0.636621979814147\\
3.4	0	0.636621979814147\\
3.39443481563427	0.194453805336823	0.636621979814147\\
3.37725957186928	0.392578379712051	0.636621979814147\\
3.34781168023218	0.593428136930646	0.636621979814147\\
3.30552175649527	0.795943413400979	0.636621979814147\\
3.24993503676371	0.998960588219399	0.636621979814147\\
3.18073174714459	1.20122668664434	0.636621979814147\\
3.09774537764988	1.40141841548083	0.636621979814147\\
3.00097778375708	1.59816530477808	0.636621979814147\\
2.89061009155953	1.79007633875604	0.636621979814147\\
2.76700852257365	1.97576917579072	0.636621979814147\\
2.63072448069595	2.15390081170583	0.636621979814147\\
2.48248854456676	2.32319836133182	0.636621979814147\\
2.32319836133182	2.48248854456676	0.636621979814147\\
2.15390081170583	2.63072448069595	0.636621979814147\\
1.97576917579072	2.76700852257365	0.636621979814147\\
1.79007633875604	2.89061009155953	0.636621979814147\\
1.59816530477808	3.00097778375708	0.636621979814147\\
1.40141841548083	3.09774537764988	0.636621979814147\\
1.20122668664434	3.18073174714459	0.636621979814147\\
0.998960588219399	3.24993503676371	0.636621979814147\\
0.795943413400979	3.30552175649527	0.636621979814147\\
0.593428136930647	3.34781168023218	0.636621979814147\\
0.392578379712051	3.37725957186928	0.636621979814147\\
0.194453805336824	3.39443481563427	0.636621979814147\\
7.54951656745106e-16	3.4	0.636621979814147\\
-0.194453805336823	3.39443481563427	0.636621979814147\\
-0.392578379712051	3.37725957186928	0.636621979814147\\
-0.593428136930646	3.34781168023218	0.636621979814147\\
-0.795943413400978	3.30552175649527	0.636621979814147\\
-0.998960588219398	3.24993503676371	0.636621979814147\\
-1.20122668664434	3.18073174714459	0.636621979814147\\
-1.40141841548083	3.09774537764988	0.636621979814147\\
-1.59816530477808	3.00097778375708	0.636621979814147\\
-1.79007633875604	2.89061009155953	0.636621979814147\\
-1.97576917579072	2.76700852257366	0.636621979814147\\
-2.15390081170583	2.63072448069595	0.636621979814147\\
-2.32319836133182	2.48248854456676	0.636621979814147\\
-2.48248854456675	2.32319836133182	0.636621979814147\\
-2.63072448069595	2.15390081170583	0.636621979814147\\
-2.76700852257365	1.97576917579072	0.636621979814147\\
-2.89061009155953	1.79007633875604	0.636621979814147\\
-3.00097778375708	1.59816530477808	0.636621979814147\\
-3.09774537764988	1.40141841548083	0.636621979814147\\
-3.18073174714459	1.20122668664434	0.636621979814147\\
-3.24993503676371	0.998960588219399	0.636621979814147\\
-3.30552175649527	0.795943413400979	0.636621979814147\\
-3.34781168023218	0.593428136930648	0.636621979814147\\
-3.37725957186928	0.392578379712052	0.636621979814147\\
-3.39443481563427	0.194453805336824	0.636621979814147\\
-3.4	9.63141612600157e-16	0.636621979814147\\
-3.39443481563427	-0.194453805336823	0.636621979814147\\
-3.37725957186928	-0.392578379712051	0.636621979814147\\
-3.34781168023218	-0.593428136930646	0.636621979814147\\
-3.30552175649527	-0.795943413400979	0.636621979814147\\
-3.24993503676371	-0.998960588219399	0.636621979814147\\
-3.18073174714459	-1.20122668664434	0.636621979814147\\
-3.09774537764988	-1.40141841548083	0.636621979814147\\
-3.00097778375708	-1.59816530477808	0.636621979814147\\
-2.89061009155953	-1.79007633875604	0.636621979814147\\
-2.76700852257366	-1.97576917579072	0.636621979814147\\
-2.63072448069595	-2.15390081170583	0.636621979814147\\
-2.48248854456676	-2.32319836133182	0.636621979814147\\
-2.32319836133182	-2.48248854456675	0.636621979814147\\
-2.15390081170583	-2.63072448069595	0.636621979814147\\
-1.97576917579072	-2.76700852257365	0.636621979814147\\
-1.79007633875604	-2.89061009155953	0.636621979814147\\
-1.59816530477808	-3.00097778375708	0.636621979814147\\
-1.40141841548083	-3.09774537764988	0.636621979814147\\
-1.20122668664434	-3.18073174714459	0.636621979814147\\
-0.998960588219401	-3.24993503676371	0.636621979814147\\
-0.79594341340098	-3.30552175649527	0.636621979814147\\
-0.593428136930648	-3.34781168023218	0.636621979814147\\
-0.392578379712052	-3.37725957186928	0.636621979814147\\
-0.194453805336825	-3.39443481563427	0.636621979814147\\
-1.17133156845521e-15	-3.4	0.636621979814147\\
0.194453805336822	-3.39443481563427	0.636621979814147\\
0.39257837971205	-3.37725957186928	0.636621979814147\\
0.593428136930646	-3.34781168023218	0.636621979814147\\
0.795943413400978	-3.30552175649527	0.636621979814147\\
0.998960588219399	-3.24993503676371	0.636621979814147\\
1.20122668664434	-3.18073174714459	0.636621979814147\\
1.40141841548083	-3.09774537764988	0.636621979814147\\
1.59816530477808	-3.00097778375708	0.636621979814147\\
1.79007633875604	-2.89061009155953	0.636621979814147\\
1.97576917579072	-2.76700852257366	0.636621979814147\\
2.15390081170583	-2.63072448069595	0.636621979814147\\
2.32319836133182	-2.48248854456676	0.636621979814147\\
2.48248854456675	-2.32319836133182	0.636621979814147\\
2.63072448069595	-2.15390081170583	0.636621979814147\\
2.76700852257365	-1.97576917579072	0.636621979814147\\
2.89061009155953	-1.79007633875604	0.636621979814147\\
3.00097778375708	-1.59816530477808	0.636621979814147\\
3.09774537764988	-1.40141841548083	0.636621979814147\\
3.18073174714459	-1.20122668664434	0.636621979814147\\
3.24993503676371	-0.998960588219401	0.636621979814147\\
3.30552175649527	-0.79594341340098	0.636621979814147\\
3.34781168023218	-0.593428136930648	0.636621979814147\\
3.37725957186928	-0.392578379712053	0.636621979814147\\
3.39443481563427	-0.194453805336825	0.636621979814147\\
3.4	-1.37952152431026e-15	0.636621979814147\\
3.42	0	0.636621979814147\\
3.41440207925564	0.195597651250569	0.636621979814147\\
3.39712580464498	0.394887664298592	0.636621979814147\\
3.3675046901159	0.596918890677297	0.636621979814147\\
3.32496600212172	0.800625433479808	0.636621979814147\\
3.26905230168585	1.00483682697363	0.636621979814147\\
3.1994419338925	1.20829272597754	0.636621979814147\\
3.11596740928311	1.40966205321896	0.636621979814147\\
3.01863059424977	1.60756627715913	0.636621979814147\\
2.90761368033341	1.80060619957225	0.636621979814147\\
2.78328504329468	1.98739134741302	0.636621979814147\\
2.64619933058239	2.16657081648057	0.636621979814147\\
2.49709141835832	2.33686423404553	0.636621979814147\\
2.33686423404553	2.49709141835832	0.636621979814147\\
2.16657081648057	2.64619933058239	0.636621979814147\\
1.98739134741302	2.78328504329468	0.636621979814147\\
1.80060619957225	2.90761368033341	0.636621979814147\\
1.60756627715913	3.01863059424977	0.636621979814147\\
1.40966205321896	3.11596740928311	0.636621979814147\\
1.20829272597754	3.1994419338925	0.636621979814147\\
1.00483682697363	3.26905230168585	0.636621979814147\\
0.800625433479809	3.32496600212172	0.636621979814147\\
0.596918890677298	3.3675046901159	0.636621979814147\\
0.394887664298593	3.39712580464498	0.636621979814147\\
0.19559765125057	3.41440207925564	0.636621979814147\\
7.59392548843607e-16	3.42	0.636621979814147\\
-0.195597651250569	3.41440207925564	0.636621979814147\\
-0.394887664298592	3.39712580464498	0.636621979814147\\
-0.596918890677297	3.3675046901159	0.636621979814147\\
-0.800625433479807	3.32496600212172	0.636621979814147\\
-1.00483682697363	3.26905230168585	0.636621979814147\\
-1.20829272597754	3.1994419338925	0.636621979814147\\
-1.40966205321896	3.11596740928311	0.636621979814147\\
-1.60756627715913	3.01863059424977	0.636621979814147\\
-1.80060619957225	2.90761368033341	0.636621979814147\\
-1.98739134741302	2.78328504329468	0.636621979814147\\
-2.16657081648057	2.64619933058239	0.636621979814147\\
-2.33686423404553	2.49709141835832	0.636621979814147\\
-2.49709141835832	2.33686423404553	0.636621979814147\\
-2.64619933058239	2.16657081648057	0.636621979814147\\
-2.78328504329468	1.98739134741302	0.636621979814147\\
-2.90761368033341	1.80060619957225	0.636621979814147\\
-3.01863059424977	1.60756627715913	0.636621979814147\\
-3.11596740928311	1.40966205321896	0.636621979814147\\
-3.1994419338925	1.20829272597754	0.636621979814147\\
-3.26905230168585	1.00483682697363	0.636621979814147\\
-3.32496600212172	0.800625433479808	0.636621979814147\\
-3.3675046901159	0.596918890677299	0.636621979814147\\
-3.39712580464498	0.394887664298594	0.636621979814147\\
-3.41440207925564	0.195597651250571	0.636621979814147\\
-3.42	9.68807151497804e-16	0.636621979814147\\
-3.41440207925564	-0.195597651250569	0.636621979814147\\
-3.39712580464498	-0.394887664298592	0.636621979814147\\
-3.3675046901159	-0.596918890677297	0.636621979814147\\
-3.32496600212172	-0.800625433479808	0.636621979814147\\
-3.26905230168585	-1.00483682697363	0.636621979814147\\
-3.1994419338925	-1.20829272597754	0.636621979814147\\
-3.11596740928311	-1.40966205321895	0.636621979814147\\
-3.01863059424977	-1.60756627715913	0.636621979814147\\
-2.90761368033341	-1.80060619957225	0.636621979814147\\
-2.78328504329468	-1.98739134741302	0.636621979814147\\
-2.64619933058239	-2.16657081648057	0.636621979814147\\
-2.49709141835832	-2.33686423404553	0.636621979814147\\
-2.33686423404553	-2.49709141835832	0.636621979814147\\
-2.16657081648057	-2.64619933058239	0.636621979814147\\
-1.98739134741302	-2.78328504329468	0.636621979814147\\
-1.80060619957225	-2.90761368033341	0.636621979814147\\
-1.60756627715913	-3.01863059424977	0.636621979814147\\
-1.40966205321896	-3.11596740928311	0.636621979814147\\
-1.20829272597754	-3.1994419338925	0.636621979814147\\
-1.00483682697363	-3.26905230168585	0.636621979814147\\
-0.80062543347981	-3.32496600212172	0.636621979814147\\
-0.596918890677299	-3.3675046901159	0.636621979814147\\
-0.394887664298594	-3.39712580464498	0.636621979814147\\
-0.195597651250571	-3.41440207925564	0.636621979814147\\
-1.178221754152e-15	-3.42	0.636621979814147\\
0.195597651250568	-3.41440207925564	0.636621979814147\\
0.394887664298592	-3.39712580464498	0.636621979814147\\
0.596918890677297	-3.3675046901159	0.636621979814147\\
0.800625433479808	-3.32496600212172	0.636621979814147\\
1.00483682697363	-3.26905230168585	0.636621979814147\\
1.20829272597754	-3.1994419338925	0.636621979814147\\
1.40966205321895	-3.11596740928311	0.636621979814147\\
1.60756627715913	-3.01863059424977	0.636621979814147\\
1.80060619957225	-2.90761368033341	0.636621979814147\\
1.98739134741302	-2.78328504329468	0.636621979814147\\
2.16657081648057	-2.64619933058239	0.636621979814147\\
2.33686423404553	-2.49709141835832	0.636621979814147\\
2.49709141835832	-2.33686423404553	0.636621979814147\\
2.64619933058239	-2.16657081648057	0.636621979814147\\
2.78328504329468	-1.98739134741302	0.636621979814147\\
2.90761368033341	-1.80060619957225	0.636621979814147\\
3.01863059424977	-1.60756627715913	0.636621979814147\\
3.11596740928311	-1.40966205321896	0.636621979814147\\
3.1994419338925	-1.20829272597754	0.636621979814147\\
3.26905230168585	-1.00483682697363	0.636621979814147\\
3.32496600212172	-0.80062543347981	0.636621979814147\\
3.3675046901159	-0.596918890677299	0.636621979814147\\
3.39712580464498	-0.394887664298594	0.636621979814147\\
3.41440207925564	-0.195597651250571	0.636621979814147\\
3.42	-1.3876363568062e-15	0.636621979814147\\
3.44	0	0.636621979814147\\
3.43436934287702	0.196741497164315	0.636621979814147\\
3.41699203742068	0.397196948885134	0.636621979814147\\
3.38719769999962	0.600409644423948	0.636621979814147\\
3.34441024774816	0.805307453558637	0.636621979814147\\
3.28816956660799	1.01071306572786	0.636621979814147\\
3.21815212064041	1.21535876531074	0.636621979814147\\
3.13418944091635	1.41790569095708	0.636621979814147\\
3.03628340474246	1.61696724954018	0.636621979814147\\
2.92461726910729	1.81113606038846	0.636621979814147\\
2.7995615640157	1.99901351903532	0.636621979814147\\
2.66167418046884	2.17924082125531	0.636621979814147\\
2.51169429214989	2.35053010675925	0.636621979814147\\
2.35053010675925	2.51169429214989	0.636621979814147\\
2.17924082125531	2.66167418046884	0.636621979814147\\
1.99901351903532	2.7995615640157	0.636621979814147\\
1.81113606038846	2.92461726910729	0.636621979814147\\
1.61696724954018	3.03628340474246	0.636621979814147\\
1.41790569095708	3.13418944091635	0.636621979814147\\
1.21535876531074	3.21815212064041	0.636621979814147\\
1.01071306572786	3.28816956660799	0.636621979814147\\
0.805307453558638	3.34441024774816	0.636621979814147\\
0.600409644423949	3.38719769999962	0.636621979814147\\
0.397196948885134	3.41699203742068	0.636621979814147\\
0.196741497164316	3.43436934287702	0.636621979814147\\
7.63833440942108e-16	3.44	0.636621979814147\\
-0.196741497164315	3.43436934287702	0.636621979814147\\
-0.397196948885134	3.41699203742068	0.636621979814147\\
-0.600409644423948	3.38719769999962	0.636621979814147\\
-0.805307453558636	3.34441024774816	0.636621979814147\\
-1.01071306572786	3.28816956660799	0.636621979814147\\
-1.21535876531074	3.21815212064041	0.636621979814147\\
-1.41790569095708	3.13418944091635	0.636621979814147\\
-1.61696724954018	3.03628340474246	0.636621979814147\\
-1.81113606038846	2.92461726910729	0.636621979814147\\
-1.99901351903532	2.7995615640157	0.636621979814147\\
-2.17924082125531	2.66167418046884	0.636621979814147\\
-2.35053010675925	2.51169429214989	0.636621979814147\\
-2.51169429214989	2.35053010675925	0.636621979814147\\
-2.66167418046884	2.17924082125531	0.636621979814147\\
-2.7995615640157	1.99901351903532	0.636621979814147\\
-2.92461726910729	1.81113606038846	0.636621979814147\\
-3.03628340474246	1.61696724954018	0.636621979814147\\
-3.13418944091635	1.41790569095708	0.636621979814147\\
-3.21815212064041	1.21535876531074	0.636621979814147\\
-3.28816956660799	1.01071306572786	0.636621979814147\\
-3.34441024774816	0.805307453558638	0.636621979814147\\
-3.38719769999962	0.60040964442395	0.636621979814147\\
-3.41699203742068	0.397196948885135	0.636621979814147\\
-3.43436934287702	0.196741497164317	0.636621979814147\\
-3.44	9.74472690395452e-16	0.636621979814147\\
-3.43436934287702	-0.196741497164315	0.636621979814147\\
-3.41699203742068	-0.397196948885133	0.636621979814147\\
-3.38719769999962	-0.600409644423948	0.636621979814147\\
-3.34441024774816	-0.805307453558637	0.636621979814147\\
-3.28816956660799	-1.01071306572786	0.636621979814147\\
-3.21815212064041	-1.21535876531074	0.636621979814147\\
-3.13418944091635	-1.41790569095708	0.636621979814147\\
-3.03628340474246	-1.61696724954018	0.636621979814147\\
-2.92461726910729	-1.81113606038846	0.636621979814147\\
-2.7995615640157	-1.99901351903532	0.636621979814147\\
-2.66167418046884	-2.17924082125531	0.636621979814147\\
-2.51169429214989	-2.35053010675925	0.636621979814147\\
-2.35053010675925	-2.51169429214989	0.636621979814147\\
-2.17924082125531	-2.66167418046884	0.636621979814147\\
-1.99901351903532	-2.7995615640157	0.636621979814147\\
-1.81113606038846	-2.92461726910729	0.636621979814147\\
-1.61696724954018	-3.03628340474246	0.636621979814147\\
-1.41790569095708	-3.13418944091635	0.636621979814147\\
-1.21535876531075	-3.21815212064041	0.636621979814147\\
-1.01071306572786	-3.28816956660799	0.636621979814147\\
-0.805307453558639	-3.34441024774816	0.636621979814147\\
-0.60040964442395	-3.38719769999962	0.636621979814147\\
-0.397196948885135	-3.41699203742068	0.636621979814147\\
-0.196741497164317	-3.43436934287702	0.636621979814147\\
-1.1851119398488e-15	-3.44	0.636621979814147\\
0.196741497164314	-3.43436934287702	0.636621979814147\\
0.397196948885133	-3.41699203742068	0.636621979814147\\
0.600409644423948	-3.38719769999962	0.636621979814147\\
0.805307453558637	-3.34441024774816	0.636621979814147\\
1.01071306572786	-3.28816956660799	0.636621979814147\\
1.21535876531074	-3.21815212064041	0.636621979814147\\
1.41790569095708	-3.13418944091635	0.636621979814147\\
1.61696724954018	-3.03628340474246	0.636621979814147\\
1.81113606038846	-2.92461726910729	0.636621979814147\\
1.99901351903532	-2.7995615640157	0.636621979814147\\
2.17924082125531	-2.66167418046884	0.636621979814147\\
2.35053010675925	-2.51169429214989	0.636621979814147\\
2.51169429214989	-2.35053010675925	0.636621979814147\\
2.66167418046884	-2.17924082125531	0.636621979814147\\
2.7995615640157	-1.99901351903532	0.636621979814147\\
2.92461726910729	-1.81113606038846	0.636621979814147\\
3.03628340474246	-1.61696724954018	0.636621979814147\\
3.13418944091635	-1.41790569095708	0.636621979814147\\
3.21815212064041	-1.21535876531075	0.636621979814147\\
3.28816956660799	-1.01071306572786	0.636621979814147\\
3.34441024774816	-0.805307453558639	0.636621979814147\\
3.38719769999962	-0.60040964442395	0.636621979814147\\
3.41699203742068	-0.397196948885136	0.636621979814147\\
3.43436934287702	-0.196741497164317	0.636621979814147\\
3.44	-1.39575118930214e-15	0.636621979814147\\
3.46	0	0.636621979814147\\
3.4543366064984	0.197885343078061	0.636621979814147\\
3.43685827019638	0.399506233471675	0.636621979814147\\
3.40689070988334	0.603900398170599	0.636621979814147\\
3.3638544933746	0.809989473637467	0.636621979814147\\
3.30728683153013	1.01658930448209	0.636621979814147\\
3.23686230738832	1.22242480464395	0.636621979814147\\
3.15241147254958	1.4261493286952	0.636621979814147\\
3.05393621523515	1.62636822192122	0.636621979814147\\
2.94162085788117	1.82166592120467	0.636621979814147\\
2.81583808473672	2.01063569065762	0.636621979814147\\
2.67714903035529	2.19191082603005	0.636621979814147\\
2.52629716594146	2.36419597947297	0.636621979814147\\
2.36419597947297	2.52629716594146	0.636621979814147\\
2.19191082603005	2.67714903035529	0.636621979814147\\
2.01063569065762	2.81583808473672	0.636621979814147\\
1.82166592120467	2.94162085788117	0.636621979814147\\
1.62636822192123	3.05393621523515	0.636621979814147\\
1.4261493286952	3.15241147254958	0.636621979814147\\
1.22242480464395	3.23686230738832	0.636621979814147\\
1.01658930448209	3.30728683153013	0.636621979814147\\
0.809989473637467	3.3638544933746	0.636621979814147\\
0.6039003981706	3.40689070988334	0.636621979814147\\
0.399506233471676	3.43685827019638	0.636621979814147\\
0.197885343078062	3.4543366064984	0.636621979814147\\
7.68274333040608e-16	3.46	0.636621979814147\\
-0.197885343078061	3.4543366064984	0.636621979814147\\
-0.399506233471675	3.43685827019638	0.636621979814147\\
-0.603900398170599	3.40689070988334	0.636621979814147\\
-0.809989473637465	3.3638544933746	0.636621979814147\\
-1.01658930448209	3.30728683153013	0.636621979814147\\
-1.22242480464394	3.23686230738832	0.636621979814147\\
-1.4261493286952	3.15241147254958	0.636621979814147\\
-1.62636822192122	3.05393621523515	0.636621979814147\\
-1.82166592120467	2.94162085788117	0.636621979814147\\
-2.01063569065762	2.81583808473672	0.636621979814147\\
-2.19191082603005	2.67714903035529	0.636621979814147\\
-2.36419597947297	2.52629716594146	0.636621979814147\\
-2.52629716594146	2.36419597947297	0.636621979814147\\
-2.67714903035529	2.19191082603005	0.636621979814147\\
-2.81583808473672	2.01063569065762	0.636621979814147\\
-2.94162085788117	1.82166592120467	0.636621979814147\\
-3.05393621523515	1.62636822192123	0.636621979814147\\
-3.15241147254958	1.4261493286952	0.636621979814147\\
-3.23686230738832	1.22242480464395	0.636621979814147\\
-3.30728683153013	1.01658930448209	0.636621979814147\\
-3.3638544933746	0.809989473637467	0.636621979814147\\
-3.40689070988334	0.6039003981706	0.636621979814147\\
-3.43685827019638	0.399506233471677	0.636621979814147\\
-3.4543366064984	0.197885343078063	0.636621979814147\\
-3.46	9.801382292931e-16	0.636621979814147\\
-3.4543366064984	-0.197885343078061	0.636621979814147\\
-3.43685827019638	-0.399506233471675	0.636621979814147\\
-3.40689070988334	-0.603900398170599	0.636621979814147\\
-3.3638544933746	-0.809989473637466	0.636621979814147\\
-3.30728683153013	-1.01658930448209	0.636621979814147\\
-3.23686230738832	-1.22242480464395	0.636621979814147\\
-3.15241147254958	-1.4261493286952	0.636621979814147\\
-3.05393621523515	-1.62636822192122	0.636621979814147\\
-2.94162085788117	-1.82166592120467	0.636621979814147\\
-2.81583808473672	-2.01063569065762	0.636621979814147\\
-2.67714903035529	-2.19191082603005	0.636621979814147\\
-2.52629716594146	-2.36419597947297	0.636621979814147\\
-2.36419597947297	-2.52629716594146	0.636621979814147\\
-2.19191082603005	-2.67714903035529	0.636621979814147\\
-2.01063569065762	-2.81583808473672	0.636621979814147\\
-1.82166592120467	-2.94162085788117	0.636621979814147\\
-1.62636822192122	-3.05393621523515	0.636621979814147\\
-1.4261493286952	-3.15241147254958	0.636621979814147\\
-1.22242480464395	-3.23686230738832	0.636621979814147\\
-1.0165893044821	-3.30728683153013	0.636621979814147\\
-0.809989473637468	-3.3638544933746	0.636621979814147\\
-0.603900398170601	-3.40689070988334	0.636621979814147\\
-0.399506233471677	-3.43685827019638	0.636621979814147\\
-0.197885343078063	-3.4543366064984	0.636621979814147\\
-1.19200212554559e-15	-3.46	0.636621979814147\\
0.19788534307806	-3.4543366064984	0.636621979814147\\
0.399506233471675	-3.43685827019638	0.636621979814147\\
0.603900398170598	-3.40689070988334	0.636621979814147\\
0.809989473637466	-3.3638544933746	0.636621979814147\\
1.01658930448209	-3.30728683153013	0.636621979814147\\
1.22242480464395	-3.23686230738832	0.636621979814147\\
1.4261493286952	-3.15241147254958	0.636621979814147\\
1.62636822192122	-3.05393621523515	0.636621979814147\\
1.82166592120467	-2.94162085788117	0.636621979814147\\
2.01063569065762	-2.81583808473672	0.636621979814147\\
2.19191082603005	-2.67714903035529	0.636621979814147\\
2.36419597947296	-2.52629716594146	0.636621979814147\\
2.52629716594146	-2.36419597947297	0.636621979814147\\
2.67714903035529	-2.19191082603005	0.636621979814147\\
2.81583808473672	-2.01063569065762	0.636621979814147\\
2.94162085788117	-1.82166592120467	0.636621979814147\\
3.05393621523515	-1.62636822192122	0.636621979814147\\
3.15241147254958	-1.4261493286952	0.636621979814147\\
3.23686230738832	-1.22242480464395	0.636621979814147\\
3.30728683153013	-1.0165893044821	0.636621979814147\\
3.3638544933746	-0.809989473637468	0.636621979814147\\
3.40689070988334	-0.603900398170601	0.636621979814147\\
3.43685827019638	-0.399506233471677	0.636621979814147\\
3.4543366064984	-0.197885343078063	0.636621979814147\\
3.46	-1.40386602179808e-15	0.636621979814147\\
3.48	0	0.636621979814147\\
3.47430387011978	0.199029188991807	0.636621979814147\\
3.45672450297209	0.401815518058217	0.636621979814147\\
3.42658371976706	0.60739115191725	0.636621979814147\\
3.38329873900104	0.814671493716296	0.636621979814147\\
3.32640409645227	1.02246554323633	0.636621979814147\\
3.25557249413623	1.22949084397715	0.636621979814147\\
3.17063350418282	1.43439296643332	0.636621979814147\\
3.07158902572784	1.63576919430227	0.636621979814147\\
2.95862444665505	1.83219578202089	0.636621979814147\\
2.83211460545774	2.02225786227992	0.636621979814147\\
2.69262388024173	2.20458083080479	0.636621979814147\\
2.54090003973303	2.37786185218668	0.636621979814147\\
2.37786185218668	2.54090003973303	0.636621979814147\\
2.20458083080479	2.69262388024173	0.636621979814147\\
2.02225786227992	2.83211460545774	0.636621979814147\\
1.83219578202089	2.95862444665505	0.636621979814147\\
1.63576919430227	3.07158902572784	0.636621979814147\\
1.43439296643332	3.17063350418282	0.636621979814147\\
1.22949084397715	3.25557249413622	0.636621979814147\\
1.02246554323633	3.32640409645227	0.636621979814147\\
0.814671493716297	3.38329873900104	0.636621979814147\\
0.60739115191725	3.42658371976706	0.636621979814147\\
0.401815518058217	3.45672450297208	0.636621979814147\\
0.199029188991808	3.47430387011978	0.636621979814147\\
7.72715225139109e-16	3.48	0.636621979814147\\
-0.199029188991807	3.47430387011978	0.636621979814147\\
-0.401815518058217	3.45672450297208	0.636621979814147\\
-0.60739115191725	3.42658371976706	0.636621979814147\\
-0.814671493716295	3.38329873900104	0.636621979814147\\
-1.02246554323633	3.32640409645227	0.636621979814147\\
-1.22949084397715	3.25557249413623	0.636621979814147\\
-1.43439296643332	3.17063350418282	0.636621979814147\\
-1.63576919430227	3.07158902572784	0.636621979814147\\
-1.83219578202089	2.95862444665505	0.636621979814147\\
-2.02225786227992	2.83211460545774	0.636621979814147\\
-2.20458083080479	2.69262388024173	0.636621979814147\\
-2.37786185218668	2.54090003973303	0.636621979814147\\
-2.54090003973303	2.37786185218668	0.636621979814147\\
-2.69262388024173	2.20458083080479	0.636621979814147\\
-2.83211460545774	2.02225786227992	0.636621979814147\\
-2.95862444665504	1.83219578202089	0.636621979814147\\
-3.07158902572784	1.63576919430227	0.636621979814147\\
-3.17063350418282	1.43439296643332	0.636621979814147\\
-3.25557249413622	1.22949084397715	0.636621979814147\\
-3.32640409645227	1.02246554323633	0.636621979814147\\
-3.38329873900104	0.814671493716296	0.636621979814147\\
-3.42658371976706	0.607391151917251	0.636621979814147\\
-3.45672450297208	0.401815518058218	0.636621979814147\\
-3.47430387011978	0.199029188991809	0.636621979814147\\
-3.48	9.85803768190748e-16	0.636621979814147\\
-3.47430387011978	-0.199029188991807	0.636621979814147\\
-3.45672450297208	-0.401815518058216	0.636621979814147\\
-3.42658371976706	-0.60739115191725	0.636621979814147\\
-3.38329873900104	-0.814671493716296	0.636621979814147\\
-3.32640409645227	-1.02246554323633	0.636621979814147\\
-3.25557249413622	-1.22949084397715	0.636621979814147\\
-3.17063350418282	-1.43439296643332	0.636621979814147\\
-3.07158902572784	-1.63576919430227	0.636621979814147\\
-2.95862444665505	-1.83219578202088	0.636621979814147\\
-2.83211460545774	-2.02225786227992	0.636621979814147\\
-2.69262388024173	-2.20458083080479	0.636621979814147\\
-2.54090003973303	-2.37786185218668	0.636621979814147\\
-2.37786185218668	-2.54090003973303	0.636621979814147\\
-2.20458083080479	-2.69262388024173	0.636621979814147\\
-2.02225786227992	-2.83211460545774	0.636621979814147\\
-1.83219578202089	-2.95862444665505	0.636621979814147\\
-1.63576919430227	-3.07158902572784	0.636621979814147\\
-1.43439296643332	-3.17063350418282	0.636621979814147\\
-1.22949084397715	-3.25557249413622	0.636621979814147\\
-1.02246554323633	-3.32640409645227	0.636621979814147\\
-0.814671493716297	-3.38329873900104	0.636621979814147\\
-0.607391151917251	-3.42658371976706	0.636621979814147\\
-0.401815518058218	-3.45672450297208	0.636621979814147\\
-0.199029188991809	-3.47430387011978	0.636621979814147\\
-1.19889231124239e-15	-3.48	0.636621979814147\\
0.199029188991806	-3.47430387011978	0.636621979814147\\
0.401815518058216	-3.45672450297208	0.636621979814147\\
0.607391151917249	-3.42658371976706	0.636621979814147\\
0.814671493716295	-3.38329873900104	0.636621979814147\\
1.02246554323633	-3.32640409645227	0.636621979814147\\
1.22949084397715	-3.25557249413622	0.636621979814147\\
1.43439296643332	-3.17063350418282	0.636621979814147\\
1.63576919430227	-3.07158902572784	0.636621979814147\\
1.83219578202088	-2.95862444665505	0.636621979814147\\
2.02225786227992	-2.83211460545774	0.636621979814147\\
2.20458083080479	-2.69262388024173	0.636621979814147\\
2.37786185218668	-2.54090003973303	0.636621979814147\\
2.54090003973303	-2.37786185218668	0.636621979814147\\
2.69262388024173	-2.20458083080479	0.636621979814147\\
2.83211460545774	-2.02225786227992	0.636621979814147\\
2.95862444665505	-1.83219578202089	0.636621979814147\\
3.07158902572784	-1.63576919430227	0.636621979814147\\
3.17063350418282	-1.43439296643332	0.636621979814147\\
3.25557249413622	-1.22949084397715	0.636621979814147\\
3.32640409645227	-1.02246554323633	0.636621979814147\\
3.38329873900104	-0.814671493716298	0.636621979814147\\
3.42658371976706	-0.607391151917252	0.636621979814147\\
3.45672450297208	-0.401815518058219	0.636621979814147\\
3.47430387011978	-0.199029188991809	0.636621979814147\\
3.48	-1.41198085429403e-15	0.636621979814147\\
3.5	0	0.636621979814147\\
3.49427113374116	0.200173034905553	0.636621979814147\\
3.47659073574779	0.404124802644758	0.636621979814147\\
3.44627672965078	0.610881905663901	0.636621979814147\\
3.40274298462749	0.819353513795125	0.636621979814147\\
3.34552136137441	1.02834178199056	0.636621979814147\\
3.27428268088413	1.23655688331035	0.636621979814147\\
3.18885553581605	1.44263660417145	0.636621979814147\\
3.08924183622053	1.64517016668332	0.636621979814147\\
2.97562803542893	1.8427256428371	0.636621979814147\\
2.84839112617876	2.03388003390222	0.636621979814147\\
2.70809873012818	2.21725083557953	0.636621979814147\\
2.5555029135246	2.3915277249004	0.636621979814147\\
2.3915277249004	2.5555029135246	0.636621979814147\\
2.21725083557953	2.70809873012818	0.636621979814147\\
2.03388003390222	2.84839112617876	0.636621979814147\\
1.8427256428371	2.97562803542893	0.636621979814147\\
1.64517016668332	3.08924183622053	0.636621979814147\\
1.44263660417145	3.18885553581605	0.636621979814147\\
1.23655688331035	3.27428268088413	0.636621979814147\\
1.02834178199056	3.34552136137441	0.636621979814147\\
0.819353513795126	3.40274298462749	0.636621979814147\\
0.610881905663901	3.44627672965077	0.636621979814147\\
0.404124802644759	3.47659073574779	0.636621979814147\\
0.200173034905554	3.49427113374116	0.636621979814147\\
7.7715611723761e-16	3.5	0.636621979814147\\
-0.200173034905553	3.49427113374116	0.636621979814147\\
-0.404124802644758	3.47659073574779	0.636621979814147\\
-0.610881905663901	3.44627672965077	0.636621979814147\\
-0.819353513795124	3.40274298462749	0.636621979814147\\
-1.02834178199056	3.34552136137441	0.636621979814147\\
-1.23655688331035	3.27428268088413	0.636621979814147\\
-1.44263660417145	3.18885553581605	0.636621979814147\\
-1.64517016668332	3.08924183622053	0.636621979814147\\
-1.8427256428371	2.97562803542893	0.636621979814147\\
-2.03388003390222	2.84839112617876	0.636621979814147\\
-2.21725083557953	2.70809873012818	0.636621979814147\\
-2.3915277249004	2.5555029135246	0.636621979814147\\
-2.5555029135246	2.3915277249004	0.636621979814147\\
-2.70809873012818	2.21725083557953	0.636621979814147\\
-2.84839112617876	2.03388003390222	0.636621979814147\\
-2.97562803542892	1.8427256428371	0.636621979814147\\
-3.08924183622053	1.64517016668332	0.636621979814147\\
-3.18885553581605	1.44263660417145	0.636621979814147\\
-3.27428268088413	1.23655688331035	0.636621979814147\\
-3.34552136137441	1.02834178199056	0.636621979814147\\
-3.40274298462749	0.819353513795125	0.636621979814147\\
-3.44627672965077	0.610881905663902	0.636621979814147\\
-3.47659073574779	0.40412480264476	0.636621979814147\\
-3.49427113374116	0.200173034905555	0.636621979814147\\
-3.5	9.91469307088396e-16	0.636621979814147\\
-3.49427113374116	-0.200173034905553	0.636621979814147\\
-3.47659073574779	-0.404124802644758	0.636621979814147\\
-3.44627672965077	-0.6108819056639	0.636621979814147\\
-3.40274298462749	-0.819353513795125	0.636621979814147\\
-3.34552136137441	-1.02834178199056	0.636621979814147\\
-3.27428268088413	-1.23655688331035	0.636621979814147\\
-3.18885553581605	-1.44263660417144	0.636621979814147\\
-3.08924183622053	-1.64517016668332	0.636621979814147\\
-2.97562803542893	-1.8427256428371	0.636621979814147\\
-2.84839112617876	-2.03388003390222	0.636621979814147\\
-2.70809873012818	-2.21725083557953	0.636621979814147\\
-2.5555029135246	-2.3915277249004	0.636621979814147\\
-2.3915277249004	-2.5555029135246	0.636621979814147\\
-2.21725083557953	-2.70809873012818	0.636621979814147\\
-2.03388003390222	-2.84839112617876	0.636621979814147\\
-1.8427256428371	-2.97562803542893	0.636621979814147\\
-1.64517016668332	-3.08924183622053	0.636621979814147\\
-1.44263660417145	-3.18885553581605	0.636621979814147\\
-1.23655688331035	-3.27428268088413	0.636621979814147\\
-1.02834178199056	-3.34552136137441	0.636621979814147\\
-0.819353513795127	-3.40274298462749	0.636621979814147\\
-0.610881905663902	-3.44627672965077	0.636621979814147\\
-0.40412480264476	-3.47659073574779	0.636621979814147\\
-0.200173034905555	-3.49427113374116	0.636621979814147\\
-1.20578249693918e-15	-3.5	0.636621979814147\\
0.200173034905553	-3.49427113374116	0.636621979814147\\
0.404124802644758	-3.47659073574779	0.636621979814147\\
0.6108819056639	-3.44627672965077	0.636621979814147\\
0.819353513795125	-3.40274298462749	0.636621979814147\\
1.02834178199056	-3.34552136137441	0.636621979814147\\
1.23655688331035	-3.27428268088413	0.636621979814147\\
1.44263660417144	-3.18885553581605	0.636621979814147\\
1.64517016668332	-3.08924183622053	0.636621979814147\\
1.8427256428371	-2.97562803542893	0.636621979814147\\
2.03388003390222	-2.84839112617876	0.636621979814147\\
2.21725083557953	-2.70809873012818	0.636621979814147\\
2.3915277249004	-2.5555029135246	0.636621979814147\\
2.5555029135246	-2.3915277249004	0.636621979814147\\
2.70809873012818	-2.21725083557953	0.636621979814147\\
2.84839112617876	-2.03388003390222	0.636621979814147\\
2.97562803542893	-1.8427256428371	0.636621979814147\\
3.08924183622053	-1.64517016668332	0.636621979814147\\
3.18885553581605	-1.44263660417145	0.636621979814147\\
3.27428268088413	-1.23655688331035	0.636621979814147\\
3.34552136137441	-1.02834178199056	0.636621979814147\\
3.40274298462749	-0.819353513795127	0.636621979814147\\
3.44627672965077	-0.610881905663902	0.636621979814147\\
3.47659073574779	-0.40412480264476	0.636621979814147\\
3.49427113374116	-0.200173034905555	0.636621979814147\\
3.5	-1.42009568678997e-15	0.636621979814147\\
3.52	0	0.636621979814147\\
3.51423839736253	0.201316880819299	0.636621979814147\\
3.49645696852349	0.4064340872313	0.636621979814147\\
3.46596973953449	0.614372659410552	0.636621979814147\\
3.42218723025393	0.824035533873954	0.636621979814147\\
3.36463862629655	1.03421802074479	0.636621979814147\\
3.29299286763204	1.24362292264355	0.636621979814147\\
3.20707756744929	1.45088024190957	0.636621979814147\\
3.10689464671322	1.65457113906437	0.636621979814147\\
2.9926316242028	1.85325550365331	0.636621979814147\\
2.86466764689978	2.04550220552451	0.636621979814147\\
2.72357358001463	2.22992084035427	0.636621979814147\\
2.57010578731617	2.40519359761412	0.636621979814147\\
2.40519359761412	2.57010578731617	0.636621979814147\\
2.22992084035427	2.72357358001463	0.636621979814147\\
2.04550220552451	2.86466764689978	0.636621979814147\\
1.85325550365331	2.9926316242028	0.636621979814147\\
1.65457113906437	3.10689464671322	0.636621979814147\\
1.45088024190957	3.20707756744929	0.636621979814147\\
1.24362292264355	3.29299286763204	0.636621979814147\\
1.03421802074479	3.36463862629655	0.636621979814147\\
0.824035533873955	3.42218723025393	0.636621979814147\\
0.614372659410552	3.46596973953449	0.636621979814147\\
0.4064340872313	3.49645696852349	0.636621979814147\\
0.2013168808193	3.51423839736253	0.636621979814147\\
7.8159700933611e-16	3.52	0.636621979814147\\
-0.201316880819299	3.51423839736253	0.636621979814147\\
-0.4064340872313	3.49645696852349	0.636621979814147\\
-0.614372659410551	3.46596973953449	0.636621979814147\\
-0.824035533873953	3.42218723025393	0.636621979814147\\
-1.03421802074479	3.36463862629655	0.636621979814147\\
-1.24362292264355	3.29299286763204	0.636621979814147\\
-1.45088024190957	3.20707756744929	0.636621979814147\\
-1.65457113906437	3.10689464671322	0.636621979814147\\
-1.85325550365331	2.9926316242028	0.636621979814147\\
-2.04550220552451	2.86466764689978	0.636621979814147\\
-2.22992084035427	2.72357358001463	0.636621979814147\\
-2.40519359761412	2.57010578731617	0.636621979814147\\
-2.57010578731617	2.40519359761412	0.636621979814147\\
-2.72357358001463	2.22992084035427	0.636621979814147\\
-2.86466764689978	2.04550220552451	0.636621979814147\\
-2.9926316242028	1.85325550365331	0.636621979814147\\
-3.10689464671322	1.65457113906437	0.636621979814147\\
-3.20707756744929	1.45088024190957	0.636621979814147\\
-3.29299286763204	1.24362292264355	0.636621979814147\\
-3.36463862629655	1.03421802074479	0.636621979814147\\
-3.42218723025393	0.824035533873955	0.636621979814147\\
-3.46596973953449	0.614372659410553	0.636621979814147\\
-3.49645696852349	0.406434087231301	0.636621979814147\\
-3.51423839736253	0.201316880819301	0.636621979814147\\
-3.52	9.97134845986044e-16	0.636621979814147\\
-3.51423839736253	-0.201316880819299	0.636621979814147\\
-3.49645696852349	-0.406434087231299	0.636621979814147\\
-3.46596973953449	-0.614372659410551	0.636621979814147\\
-3.42218723025393	-0.824035533873954	0.636621979814147\\
-3.36463862629655	-1.03421802074479	0.636621979814147\\
-3.29299286763204	-1.24362292264355	0.636621979814147\\
-3.20707756744929	-1.45088024190957	0.636621979814147\\
-3.10689464671322	-1.65457113906437	0.636621979814147\\
-2.99263162420281	-1.85325550365331	0.636621979814147\\
-2.86466764689978	-2.04550220552451	0.636621979814147\\
-2.72357358001463	-2.22992084035427	0.636621979814147\\
-2.57010578731617	-2.40519359761412	0.636621979814147\\
-2.40519359761412	-2.57010578731617	0.636621979814147\\
-2.22992084035427	-2.72357358001463	0.636621979814147\\
-2.04550220552451	-2.86466764689978	0.636621979814147\\
-1.85325550365331	-2.9926316242028	0.636621979814147\\
-1.65457113906437	-3.10689464671322	0.636621979814147\\
-1.45088024190957	-3.20707756744929	0.636621979814147\\
-1.24362292264355	-3.29299286763204	0.636621979814147\\
-1.03421802074479	-3.36463862629655	0.636621979814147\\
-0.824035533873956	-3.42218723025393	0.636621979814147\\
-0.614372659410553	-3.46596973953449	0.636621979814147\\
-0.406434087231301	-3.49645696852349	0.636621979814147\\
-0.201316880819301	-3.51423839736253	0.636621979814147\\
-1.21267268263598e-15	-3.52	0.636621979814147\\
0.201316880819299	-3.51423839736253	0.636621979814147\\
0.406434087231299	-3.49645696852349	0.636621979814147\\
0.614372659410551	-3.46596973953449	0.636621979814147\\
0.824035533873954	-3.42218723025393	0.636621979814147\\
1.03421802074479	-3.36463862629655	0.636621979814147\\
1.24362292264355	-3.29299286763204	0.636621979814147\\
1.45088024190957	-3.20707756744929	0.636621979814147\\
1.65457113906437	-3.10689464671322	0.636621979814147\\
1.85325550365331	-2.99263162420281	0.636621979814147\\
2.04550220552451	-2.86466764689979	0.636621979814147\\
2.22992084035427	-2.72357358001463	0.636621979814147\\
2.40519359761412	-2.57010578731617	0.636621979814147\\
2.57010578731617	-2.40519359761412	0.636621979814147\\
2.72357358001463	-2.22992084035427	0.636621979814147\\
2.86466764689978	-2.04550220552451	0.636621979814147\\
2.9926316242028	-1.85325550365331	0.636621979814147\\
3.10689464671322	-1.65457113906437	0.636621979814147\\
3.20707756744929	-1.45088024190957	0.636621979814147\\
3.29299286763204	-1.24362292264355	0.636621979814147\\
3.36463862629655	-1.03421802074479	0.636621979814147\\
3.42218723025393	-0.824035533873956	0.636621979814147\\
3.46596973953449	-0.614372659410553	0.636621979814147\\
3.49645696852349	-0.406434087231302	0.636621979814147\\
3.51423839736253	-0.201316880819301	0.636621979814147\\
3.52	-1.42821051928591e-15	0.636621979814147\\
3.54	0	0.636621979814147\\
3.53420566098391	0.202460726733046	0.636621979814147\\
3.51632320129919	0.408743371817841	0.636621979814147\\
3.48566274941821	0.617863413157203	0.636621979814147\\
3.44163147588037	0.828717553952784	0.636621979814147\\
3.38375589121869	1.04009425949902	0.636621979814147\\
3.31170305437995	1.25068896197675	0.636621979814147\\
3.22529959908252	1.45912387964769	0.636621979814147\\
3.12454745720591	1.66397211144542	0.636621979814147\\
3.00963521297668	1.86378536446952	0.636621979814147\\
2.88094416762081	2.05712437714681	0.636621979814147\\
2.73904842990107	2.24259084512901	0.636621979814147\\
2.58470866110774	2.41885947032783	0.636621979814147\\
2.41885947032783	2.58470866110774	0.636621979814147\\
2.24259084512901	2.73904842990107	0.636621979814147\\
2.05712437714681	2.88094416762081	0.636621979814147\\
1.86378536446952	3.00963521297668	0.636621979814147\\
1.66397211144542	3.1245474572059	0.636621979814147\\
1.45912387964769	3.22529959908252	0.636621979814147\\
1.25068896197675	3.31170305437995	0.636621979814147\\
1.04009425949902	3.38375589121869	0.636621979814147\\
0.828717553952784	3.44163147588037	0.636621979814147\\
0.617863413157203	3.48566274941821	0.636621979814147\\
0.408743371817842	3.51632320129919	0.636621979814147\\
0.202460726733046	3.53420566098391	0.636621979814147\\
7.86037901434611e-16	3.54	0.636621979814147\\
-0.202460726733045	3.53420566098391	0.636621979814147\\
-0.408743371817841	3.51632320129919	0.636621979814147\\
-0.617863413157202	3.48566274941821	0.636621979814147\\
-0.828717553952782	3.44163147588037	0.636621979814147\\
-1.04009425949902	3.38375589121869	0.636621979814147\\
-1.25068896197675	3.31170305437995	0.636621979814147\\
-1.45912387964769	3.22529959908252	0.636621979814147\\
-1.66397211144541	3.12454745720591	0.636621979814147\\
-1.86378536446952	3.00963521297668	0.636621979814147\\
-2.05712437714681	2.88094416762081	0.636621979814147\\
-2.24259084512901	2.73904842990107	0.636621979814147\\
-2.41885947032783	2.58470866110774	0.636621979814147\\
-2.58470866110774	2.41885947032783	0.636621979814147\\
-2.73904842990107	2.24259084512901	0.636621979814147\\
-2.88094416762081	2.05712437714681	0.636621979814147\\
-3.00963521297668	1.86378536446952	0.636621979814147\\
-3.12454745720591	1.66397211144542	0.636621979814147\\
-3.22529959908252	1.45912387964769	0.636621979814147\\
-3.31170305437995	1.25068896197675	0.636621979814147\\
-3.38375589121869	1.04009425949902	0.636621979814147\\
-3.44163147588037	0.828717553952784	0.636621979814147\\
-3.48566274941821	0.617863413157204	0.636621979814147\\
-3.51632320129919	0.408743371817843	0.636621979814147\\
-3.53420566098391	0.202460726733047	0.636621979814147\\
-3.54	1.00280038488369e-15	0.636621979814147\\
-3.53420566098391	-0.202460726733045	0.636621979814147\\
-3.51632320129919	-0.408743371817841	0.636621979814147\\
-3.48566274941821	-0.617863413157202	0.636621979814147\\
-3.44163147588037	-0.828717553952783	0.636621979814147\\
-3.38375589121869	-1.04009425949902	0.636621979814147\\
-3.31170305437995	-1.25068896197675	0.636621979814147\\
-3.22529959908252	-1.45912387964769	0.636621979814147\\
-3.12454745720591	-1.66397211144541	0.636621979814147\\
-3.00963521297668	-1.86378536446952	0.636621979814147\\
-2.88094416762081	-2.05712437714681	0.636621979814147\\
-2.73904842990107	-2.24259084512901	0.636621979814147\\
-2.58470866110774	-2.41885947032783	0.636621979814147\\
-2.41885947032783	-2.58470866110774	0.636621979814147\\
-2.24259084512901	-2.73904842990107	0.636621979814147\\
-2.05712437714681	-2.88094416762081	0.636621979814147\\
-1.86378536446952	-3.00963521297668	0.636621979814147\\
-1.66397211144542	-3.12454745720591	0.636621979814147\\
-1.45912387964769	-3.22529959908252	0.636621979814147\\
-1.25068896197676	-3.31170305437995	0.636621979814147\\
-1.04009425949902	-3.38375589121869	0.636621979814147\\
-0.828717553952785	-3.44163147588037	0.636621979814147\\
-0.617863413157204	-3.48566274941821	0.636621979814147\\
-0.408743371817843	-3.51632320129919	0.636621979814147\\
-0.202460726733047	-3.53420566098391	0.636621979814147\\
-1.21956286833277e-15	-3.54	0.636621979814147\\
0.202460726733045	-3.53420566098391	0.636621979814147\\
0.408743371817841	-3.51632320129919	0.636621979814147\\
0.617863413157202	-3.48566274941821	0.636621979814147\\
0.828717553952783	-3.44163147588037	0.636621979814147\\
1.04009425949902	-3.38375589121869	0.636621979814147\\
1.25068896197675	-3.31170305437995	0.636621979814147\\
1.45912387964769	-3.22529959908252	0.636621979814147\\
1.66397211144541	-3.12454745720591	0.636621979814147\\
1.86378536446952	-3.00963521297668	0.636621979814147\\
2.05712437714681	-2.88094416762081	0.636621979814147\\
2.24259084512901	-2.73904842990107	0.636621979814147\\
2.41885947032783	-2.58470866110774	0.636621979814147\\
2.58470866110774	-2.41885947032783	0.636621979814147\\
2.73904842990107	-2.24259084512901	0.636621979814147\\
2.88094416762081	-2.05712437714681	0.636621979814147\\
3.00963521297668	-1.86378536446952	0.636621979814147\\
3.12454745720591	-1.66397211144542	0.636621979814147\\
3.22529959908252	-1.45912387964769	0.636621979814147\\
3.31170305437995	-1.25068896197676	0.636621979814147\\
3.38375589121869	-1.04009425949902	0.636621979814147\\
3.44163147588037	-0.828717553952785	0.636621979814147\\
3.48566274941821	-0.617863413157204	0.636621979814147\\
3.51632320129919	-0.408743371817843	0.636621979814147\\
3.53420566098391	-0.202460726733047	0.636621979814147\\
3.54	-1.43632535178186e-15	0.636621979814147\\
3.56	0	0.636621979814147\\
3.55417292460529	0.203604572646792	0.636621979814147\\
3.53618943407489	0.411052656404383	0.636621979814147\\
3.50535575930193	0.621354166903853	0.636621979814147\\
3.46107572150682	0.833399574031613	0.636621979814147\\
3.40287315614083	1.04597049825325	0.636621979814147\\
3.33041324112786	1.25775500130996	0.636621979814147\\
3.24352163071576	1.46736751738581	0.636621979814147\\
3.14220026769859	1.67337308382646	0.636621979814147\\
3.02663880175056	1.87431522528573	0.636621979814147\\
2.89722068834183	2.06874654876911	0.636621979814147\\
2.75452327978752	2.25526084990375	0.636621979814147\\
2.59931153489931	2.43252534304155	0.636621979814147\\
2.43252534304155	2.59931153489931	0.636621979814147\\
2.25526084990375	2.75452327978752	0.636621979814147\\
2.06874654876911	2.89722068834183	0.636621979814147\\
1.87431522528573	3.02663880175056	0.636621979814147\\
1.67337308382646	3.14220026769859	0.636621979814147\\
1.46736751738581	3.24352163071576	0.636621979814147\\
1.25775500130996	3.33041324112786	0.636621979814147\\
1.04597049825325	3.40287315614083	0.636621979814147\\
0.833399574031614	3.46107572150682	0.636621979814147\\
0.621354166903854	3.50535575930193	0.636621979814147\\
0.411052656404383	3.53618943407489	0.636621979814147\\
0.203604572646793	3.55417292460529	0.636621979814147\\
7.90478793533111e-16	3.56	0.636621979814147\\
-0.203604572646791	3.55417292460529	0.636621979814147\\
-0.411052656404382	3.53618943407489	0.636621979814147\\
-0.621354166903853	3.50535575930193	0.636621979814147\\
-0.833399574031612	3.46107572150682	0.636621979814147\\
-1.04597049825325	3.40287315614083	0.636621979814147\\
-1.25775500130995	3.33041324112786	0.636621979814147\\
-1.46736751738581	3.24352163071576	0.636621979814147\\
-1.67337308382646	3.14220026769859	0.636621979814147\\
-1.87431522528573	3.02663880175056	0.636621979814147\\
-2.06874654876911	2.89722068834183	0.636621979814147\\
-2.25526084990375	2.75452327978752	0.636621979814147\\
-2.43252534304155	2.59931153489931	0.636621979814147\\
-2.59931153489931	2.43252534304155	0.636621979814147\\
-2.75452327978752	2.25526084990375	0.636621979814147\\
-2.89722068834183	2.06874654876911	0.636621979814147\\
-3.02663880175056	1.87431522528573	0.636621979814147\\
-3.14220026769859	1.67337308382646	0.636621979814147\\
-3.24352163071576	1.46736751738581	0.636621979814147\\
-3.33041324112786	1.25775500130996	0.636621979814147\\
-3.40287315614083	1.04597049825325	0.636621979814147\\
-3.46107572150682	0.833399574031613	0.636621979814147\\
-3.50535575930193	0.621354166903855	0.636621979814147\\
-3.53618943407489	0.411052656404384	0.636621979814147\\
-3.55417292460529	0.203604572646793	0.636621979814147\\
-3.56	1.00846592378134e-15	0.636621979814147\\
-3.55417292460529	-0.203604572646791	0.636621979814147\\
-3.53618943407489	-0.411052656404382	0.636621979814147\\
-3.50535575930193	-0.621354166903853	0.636621979814147\\
-3.46107572150682	-0.833399574031613	0.636621979814147\\
-3.40287315614083	-1.04597049825325	0.636621979814147\\
-3.33041324112786	-1.25775500130996	0.636621979814147\\
-3.24352163071576	-1.46736751738581	0.636621979814147\\
-3.14220026769859	-1.67337308382646	0.636621979814147\\
-3.02663880175056	-1.87431522528573	0.636621979814147\\
-2.89722068834183	-2.06874654876911	0.636621979814147\\
-2.75452327978752	-2.25526084990375	0.636621979814147\\
-2.59931153489931	-2.43252534304155	0.636621979814147\\
-2.43252534304155	-2.59931153489931	0.636621979814147\\
-2.25526084990375	-2.75452327978752	0.636621979814147\\
-2.06874654876911	-2.89722068834183	0.636621979814147\\
-1.87431522528573	-3.02663880175056	0.636621979814147\\
-1.67337308382646	-3.14220026769859	0.636621979814147\\
-1.46736751738581	-3.24352163071576	0.636621979814147\\
-1.25775500130996	-3.33041324112786	0.636621979814147\\
-1.04597049825325	-3.40287315614083	0.636621979814147\\
-0.833399574031615	-3.46107572150682	0.636621979814147\\
-0.621354166903855	-3.50535575930193	0.636621979814147\\
-0.411052656404384	-3.53618943407489	0.636621979814147\\
-0.203604572646793	-3.55417292460529	0.636621979814147\\
-1.22645305402957e-15	-3.56	0.636621979814147\\
0.203604572646791	-3.55417292460529	0.636621979814147\\
0.411052656404382	-3.53618943407489	0.636621979814147\\
0.621354166903853	-3.50535575930193	0.636621979814147\\
0.833399574031613	-3.46107572150682	0.636621979814147\\
1.04597049825325	-3.40287315614083	0.636621979814147\\
1.25775500130996	-3.33041324112786	0.636621979814147\\
1.46736751738581	-3.24352163071576	0.636621979814147\\
1.67337308382646	-3.14220026769859	0.636621979814147\\
1.87431522528573	-3.02663880175056	0.636621979814147\\
2.06874654876911	-2.89722068834183	0.636621979814147\\
2.25526084990375	-2.75452327978752	0.636621979814147\\
2.43252534304155	-2.59931153489931	0.636621979814147\\
2.59931153489931	-2.43252534304155	0.636621979814147\\
2.75452327978752	-2.25526084990375	0.636621979814147\\
2.89722068834183	-2.06874654876911	0.636621979814147\\
3.02663880175056	-1.87431522528573	0.636621979814147\\
3.14220026769859	-1.67337308382646	0.636621979814147\\
3.24352163071576	-1.46736751738581	0.636621979814147\\
3.33041324112786	-1.25775500130996	0.636621979814147\\
3.40287315614083	-1.04597049825325	0.636621979814147\\
3.46107572150682	-0.833399574031615	0.636621979814147\\
3.50535575930193	-0.621354166903855	0.636621979814147\\
3.53618943407489	-0.411052656404384	0.636621979814147\\
3.55417292460529	-0.203604572646793	0.636621979814147\\
3.56	-1.4444401842778e-15	0.636621979814147\\
3.58	0	0.636621979814147\\
3.57414018822667	0.204748418560538	0.636621979814147\\
3.55605566685059	0.413361940990924	0.636621979814147\\
3.52504876918565	0.624844920650504	0.636621979814147\\
3.48051996713326	0.838081594110442	0.636621979814147\\
3.42199042106297	1.05184673700748	0.636621979814147\\
3.34912342787577	1.26482104064316	0.636621979814147\\
3.26174366234899	1.47561115512394	0.636621979814147\\
3.15985307819128	1.68277405620751	0.636621979814147\\
3.04364239052444	1.88484508610195	0.636621979814147\\
2.91349720906285	2.08036872039141	0.636621979814147\\
2.76999812967397	2.26793085467849	0.636621979814147\\
2.61391440869088	2.44619121575527	0.636621979814147\\
2.44619121575527	2.61391440869088	0.636621979814147\\
2.26793085467849	2.76999812967397	0.636621979814147\\
2.08036872039141	2.91349720906285	0.636621979814147\\
1.88484508610195	3.04364239052444	0.636621979814147\\
1.68277405620751	3.15985307819128	0.636621979814147\\
1.47561115512394	3.26174366234899	0.636621979814147\\
1.26482104064316	3.34912342787577	0.636621979814147\\
1.05184673700749	3.42199042106297	0.636621979814147\\
0.838081594110443	3.48051996713326	0.636621979814147\\
0.624844920650505	3.52504876918565	0.636621979814147\\
0.413361940990925	3.55605566685059	0.636621979814147\\
0.204748418560539	3.57414018822667	0.636621979814147\\
7.94919685631612e-16	3.58	0.636621979814147\\
-0.204748418560537	3.57414018822667	0.636621979814147\\
-0.413361940990924	3.55605566685059	0.636621979814147\\
-0.624844920650504	3.52504876918565	0.636621979814147\\
-0.838081594110441	3.48051996713326	0.636621979814147\\
-1.05184673700748	3.42199042106297	0.636621979814147\\
-1.26482104064316	3.34912342787577	0.636621979814147\\
-1.47561115512394	3.26174366234899	0.636621979814147\\
-1.68277405620751	3.15985307819128	0.636621979814147\\
-1.88484508610195	3.04364239052444	0.636621979814147\\
-2.08036872039141	2.91349720906285	0.636621979814147\\
-2.26793085467849	2.76999812967397	0.636621979814147\\
-2.44619121575527	2.61391440869088	0.636621979814147\\
-2.61391440869088	2.44619121575527	0.636621979814147\\
-2.76999812967397	2.26793085467849	0.636621979814147\\
-2.91349720906285	2.08036872039141	0.636621979814147\\
-3.04364239052444	1.88484508610195	0.636621979814147\\
-3.15985307819128	1.68277405620751	0.636621979814147\\
-3.26174366234899	1.47561115512394	0.636621979814147\\
-3.34912342787577	1.26482104064316	0.636621979814147\\
-3.42199042106297	1.05184673700749	0.636621979814147\\
-3.48051996713326	0.838081594110443	0.636621979814147\\
-3.52504876918565	0.624844920650506	0.636621979814147\\
-3.55605566685059	0.413361940990926	0.636621979814147\\
-3.57414018822667	0.204748418560539	0.636621979814147\\
-3.58	1.01413146267899e-15	0.636621979814147\\
-3.57414018822667	-0.204748418560537	0.636621979814147\\
-3.55605566685059	-0.413361940990924	0.636621979814147\\
-3.52504876918565	-0.624844920650504	0.636621979814147\\
-3.48051996713326	-0.838081594110442	0.636621979814147\\
-3.42199042106297	-1.05184673700748	0.636621979814147\\
-3.34912342787577	-1.26482104064316	0.636621979814147\\
-3.26174366234899	-1.47561115512393	0.636621979814147\\
-3.15985307819128	-1.68277405620751	0.636621979814147\\
-3.04364239052444	-1.88484508610194	0.636621979814147\\
-2.91349720906285	-2.08036872039141	0.636621979814147\\
-2.76999812967397	-2.26793085467849	0.636621979814147\\
-2.61391440869088	-2.44619121575527	0.636621979814147\\
-2.44619121575527	-2.61391440869088	0.636621979814147\\
-2.26793085467849	-2.76999812967397	0.636621979814147\\
-2.08036872039141	-2.91349720906285	0.636621979814147\\
-1.88484508610195	-3.04364239052444	0.636621979814147\\
-1.68277405620751	-3.15985307819128	0.636621979814147\\
-1.47561115512394	-3.26174366234899	0.636621979814147\\
-1.26482104064316	-3.34912342787577	0.636621979814147\\
-1.05184673700749	-3.42199042106297	0.636621979814147\\
-0.838081594110444	-3.48051996713326	0.636621979814147\\
-0.624844920650506	-3.52504876918565	0.636621979814147\\
-0.413361940990926	-3.55605566685059	0.636621979814147\\
-0.204748418560539	-3.57414018822667	0.636621979814147\\
-1.23334323972636e-15	-3.58	0.636621979814147\\
0.204748418560537	-3.57414018822667	0.636621979814147\\
0.413361940990924	-3.55605566685059	0.636621979814147\\
0.624844920650504	-3.52504876918565	0.636621979814147\\
0.838081594110442	-3.48051996713326	0.636621979814147\\
1.05184673700748	-3.42199042106297	0.636621979814147\\
1.26482104064316	-3.34912342787577	0.636621979814147\\
1.47561115512393	-3.26174366234899	0.636621979814147\\
1.68277405620751	-3.15985307819128	0.636621979814147\\
1.88484508610194	-3.04364239052444	0.636621979814147\\
2.08036872039141	-2.91349720906285	0.636621979814147\\
2.26793085467849	-2.76999812967397	0.636621979814147\\
2.44619121575526	-2.61391440869088	0.636621979814147\\
2.61391440869088	-2.44619121575527	0.636621979814147\\
2.76999812967397	-2.26793085467849	0.636621979814147\\
2.91349720906285	-2.08036872039141	0.636621979814147\\
3.04364239052444	-1.88484508610195	0.636621979814147\\
3.15985307819128	-1.68277405620751	0.636621979814147\\
3.26174366234899	-1.47561115512394	0.636621979814147\\
3.34912342787577	-1.26482104064316	0.636621979814147\\
3.42199042106297	-1.05184673700749	0.636621979814147\\
3.48051996713326	-0.838081594110444	0.636621979814147\\
3.52504876918565	-0.624844920650506	0.636621979814147\\
3.55605566685059	-0.413361940990926	0.636621979814147\\
3.57414018822667	-0.204748418560539	0.636621979814147\\
3.58	-1.45255501677374e-15	0.636621979814147\\
3.6	0	0.636621979814147\\
3.59410745184805	0.205892264474284	0.636621979814147\\
3.57592189962629	0.415671225577466	0.636621979814147\\
3.54474177906937	0.628335674397155	0.636621979814147\\
3.4999642127597	0.842763614189272	0.636621979814147\\
3.44110768598511	1.05772297576172	0.636621979814147\\
3.36783361462368	1.27188707997636	0.636621979814147\\
3.27996569398222	1.48385479286206	0.636621979814147\\
3.17750588868397	1.69217502858856	0.636621979814147\\
3.06064597929832	1.89537494691816	0.636621979814147\\
2.92977372978387	2.09199089201371	0.636621979814147\\
2.78547297956042	2.28060085945323	0.636621979814147\\
2.62851728248245	2.45985708846898	0.636621979814147\\
2.45985708846898	2.62851728248245	0.636621979814147\\
2.28060085945323	2.78547297956041	0.636621979814147\\
2.09199089201371	2.92977372978387	0.636621979814147\\
1.89537494691816	3.06064597929832	0.636621979814147\\
1.69217502858856	3.17750588868397	0.636621979814147\\
1.48385479286206	3.27996569398222	0.636621979814147\\
1.27188707997636	3.36783361462368	0.636621979814147\\
1.05772297576172	3.44110768598511	0.636621979814147\\
0.842763614189272	3.4999642127597	0.636621979814147\\
0.628335674397156	3.54474177906937	0.636621979814147\\
0.415671225577466	3.57592189962629	0.636621979814147\\
0.205892264474285	3.59410745184805	0.636621979814147\\
7.99360577730113e-16	3.6	0.636621979814147\\
-0.205892264474283	3.59410745184805	0.636621979814147\\
-0.415671225577465	3.5759218996263	0.636621979814147\\
-0.628335674397155	3.54474177906937	0.636621979814147\\
-0.84276361418927	3.4999642127597	0.636621979814147\\
-1.05772297576172	3.44110768598511	0.636621979814147\\
-1.27188707997636	3.36783361462368	0.636621979814147\\
-1.48385479286206	3.27996569398222	0.636621979814147\\
-1.69217502858856	3.17750588868397	0.636621979814147\\
-1.89537494691816	3.06064597929832	0.636621979814147\\
-2.09199089201371	2.92977372978387	0.636621979814147\\
-2.28060085945323	2.78547297956041	0.636621979814147\\
-2.45985708846898	2.62851728248245	0.636621979814147\\
-2.62851728248245	2.45985708846898	0.636621979814147\\
-2.78547297956041	2.28060085945323	0.636621979814147\\
-2.92977372978387	2.09199089201371	0.636621979814147\\
-3.06064597929832	1.89537494691816	0.636621979814147\\
-3.17750588868397	1.69217502858856	0.636621979814147\\
-3.27996569398222	1.48385479286206	0.636621979814147\\
-3.36783361462368	1.27188707997636	0.636621979814147\\
-3.44110768598511	1.05772297576172	0.636621979814147\\
-3.4999642127597	0.842763614189272	0.636621979814147\\
-3.54474177906937	0.628335674397157	0.636621979814147\\
-3.57592189962629	0.415671225577467	0.636621979814147\\
-3.59410745184805	0.205892264474285	0.636621979814147\\
-3.6	1.01979700157664e-15	0.636621979814147\\
-3.59410745184805	-0.205892264474283	0.636621979814147\\
-3.5759218996263	-0.415671225577465	0.636621979814147\\
-3.54474177906937	-0.628335674397155	0.636621979814147\\
-3.4999642127597	-0.842763614189271	0.636621979814147\\
-3.44110768598511	-1.05772297576172	0.636621979814147\\
-3.36783361462368	-1.27188707997636	0.636621979814147\\
-3.27996569398223	-1.48385479286206	0.636621979814147\\
-3.17750588868397	-1.69217502858856	0.636621979814147\\
-3.06064597929832	-1.89537494691816	0.636621979814147\\
-2.92977372978387	-2.09199089201371	0.636621979814147\\
-2.78547297956042	-2.28060085945323	0.636621979814147\\
-2.62851728248245	-2.45985708846898	0.636621979814147\\
-2.45985708846898	-2.62851728248245	0.636621979814147\\
-2.28060085945323	-2.78547297956041	0.636621979814147\\
-2.09199089201371	-2.92977372978387	0.636621979814147\\
-1.89537494691816	-3.06064597929832	0.636621979814147\\
-1.69217502858856	-3.17750588868397	0.636621979814147\\
-1.48385479286206	-3.27996569398223	0.636621979814147\\
-1.27188707997636	-3.36783361462368	0.636621979814147\\
-1.05772297576172	-3.44110768598511	0.636621979814147\\
-0.842763614189273	-3.4999642127597	0.636621979814147\\
-0.628335674397157	-3.54474177906937	0.636621979814147\\
-0.415671225577467	-3.57592189962629	0.636621979814147\\
-0.205892264474285	-3.59410745184805	0.636621979814147\\
-1.24023342542316e-15	-3.6	0.636621979814147\\
0.205892264474283	-3.59410745184805	0.636621979814147\\
0.415671225577465	-3.5759218996263	0.636621979814147\\
0.628335674397155	-3.54474177906937	0.636621979814147\\
0.842763614189271	-3.4999642127597	0.636621979814147\\
1.05772297576172	-3.44110768598511	0.636621979814147\\
1.27188707997636	-3.36783361462368	0.636621979814147\\
1.48385479286206	-3.27996569398223	0.636621979814147\\
1.69217502858856	-3.17750588868397	0.636621979814147\\
1.89537494691816	-3.06064597929832	0.636621979814147\\
2.09199089201371	-2.92977372978387	0.636621979814147\\
2.28060085945323	-2.78547297956042	0.636621979814147\\
2.45985708846898	-2.62851728248245	0.636621979814147\\
2.62851728248245	-2.45985708846898	0.636621979814147\\
2.78547297956041	-2.28060085945323	0.636621979814147\\
2.92977372978387	-2.09199089201371	0.636621979814147\\
3.06064597929832	-1.89537494691816	0.636621979814147\\
3.17750588868397	-1.69217502858856	0.636621979814147\\
3.27996569398223	-1.48385479286206	0.636621979814147\\
3.36783361462368	-1.27188707997636	0.636621979814147\\
3.44110768598511	-1.05772297576172	0.636621979814147\\
3.4999642127597	-0.842763614189273	0.636621979814147\\
3.54474177906937	-0.628335674397157	0.636621979814147\\
3.57592189962629	-0.415671225577467	0.636621979814147\\
3.59410745184805	-0.205892264474285	0.636621979814147\\
3.6	-1.46066984926968e-15	0.636621979814147\\
3.62	0	0.636621979814147\\
3.61407471546942	0.20703611038803	0.636621979814147\\
3.595788132402	0.417980510164007	0.636621979814147\\
3.56443478895309	0.631826428143806	0.636621979814147\\
3.51940845838614	0.847445634268101	0.636621979814147\\
3.46022495090725	1.06359921451595	0.636621979814147\\
3.38654380137159	1.27895311930956	0.636621979814147\\
3.29818772561546	1.49209843060018	0.636621979814147\\
3.19515869917666	1.70157600096961	0.636621979814147\\
3.0776495680722	1.90590480773437	0.636621979814147\\
2.94605025050489	2.10361306363601	0.636621979814147\\
2.80094782944686	2.29327086422797	0.636621979814147\\
2.64312015627402	2.4735229611827	0.636621979814147\\
2.4735229611827	2.64312015627402	0.636621979814147\\
2.29327086422797	2.80094782944686	0.636621979814147\\
2.10361306363601	2.94605025050489	0.636621979814147\\
1.90590480773437	3.0776495680722	0.636621979814147\\
1.70157600096961	3.19515869917666	0.636621979814147\\
1.49209843060018	3.29818772561546	0.636621979814147\\
1.27895311930956	3.38654380137159	0.636621979814147\\
1.06359921451595	3.46022495090725	0.636621979814147\\
0.847445634268102	3.51940845838614	0.636621979814147\\
0.631826428143807	3.56443478895309	0.636621979814147\\
0.417980510164008	3.595788132402	0.636621979814147\\
0.207036110388031	3.61407471546942	0.636621979814147\\
8.03801469828613e-16	3.62	0.636621979814147\\
-0.207036110388029	3.61407471546942	0.636621979814147\\
-0.417980510164007	3.595788132402	0.636621979814147\\
-0.631826428143806	3.56443478895309	0.636621979814147\\
-0.8474456342681	3.51940845838614	0.636621979814147\\
-1.06359921451595	3.46022495090725	0.636621979814147\\
-1.27895311930956	3.38654380137159	0.636621979814147\\
-1.49209843060018	3.29818772561546	0.636621979814147\\
-1.7015760009696	3.19515869917666	0.636621979814147\\
-1.90590480773437	3.0776495680722	0.636621979814147\\
-2.10361306363601	2.94605025050489	0.636621979814147\\
-2.29327086422797	2.80094782944686	0.636621979814147\\
-2.4735229611827	2.64312015627402	0.636621979814147\\
-2.64312015627402	2.4735229611827	0.636621979814147\\
-2.80094782944686	2.29327086422797	0.636621979814147\\
-2.94605025050489	2.10361306363601	0.636621979814147\\
-3.0776495680722	1.90590480773437	0.636621979814147\\
-3.19515869917666	1.70157600096961	0.636621979814147\\
-3.29818772561546	1.49209843060018	0.636621979814147\\
-3.38654380137159	1.27895311930956	0.636621979814147\\
-3.46022495090725	1.06359921451595	0.636621979814147\\
-3.51940845838614	0.847445634268101	0.636621979814147\\
-3.56443478895309	0.631826428143807	0.636621979814147\\
-3.595788132402	0.417980510164009	0.636621979814147\\
-3.61407471546942	0.207036110388031	0.636621979814147\\
-3.62	1.02546254047428e-15	0.636621979814147\\
-3.61407471546942	-0.207036110388029	0.636621979814147\\
-3.595788132402	-0.417980510164007	0.636621979814147\\
-3.56443478895309	-0.631826428143806	0.636621979814147\\
-3.51940845838614	-0.847445634268101	0.636621979814147\\
-3.46022495090725	-1.06359921451595	0.636621979814147\\
-3.38654380137159	-1.27895311930956	0.636621979814147\\
-3.29818772561546	-1.49209843060018	0.636621979814147\\
-3.19515869917666	-1.7015760009696	0.636621979814147\\
-3.0776495680722	-1.90590480773437	0.636621979814147\\
-2.94605025050489	-2.10361306363601	0.636621979814147\\
-2.80094782944686	-2.29327086422797	0.636621979814147\\
-2.64312015627402	-2.4735229611827	0.636621979814147\\
-2.4735229611827	-2.64312015627402	0.636621979814147\\
-2.29327086422797	-2.80094782944686	0.636621979814147\\
-2.10361306363601	-2.94605025050489	0.636621979814147\\
-1.90590480773437	-3.0776495680722	0.636621979814147\\
-1.70157600096961	-3.19515869917666	0.636621979814147\\
-1.49209843060018	-3.29818772561546	0.636621979814147\\
-1.27895311930956	-3.38654380137159	0.636621979814147\\
-1.06359921451595	-3.46022495090725	0.636621979814147\\
-0.847445634268102	-3.51940845838614	0.636621979814147\\
-0.631826428143808	-3.56443478895309	0.636621979814147\\
-0.417980510164009	-3.595788132402	0.636621979814147\\
-0.207036110388031	-3.61407471546942	0.636621979814147\\
-1.24712361111996e-15	-3.62	0.636621979814147\\
0.207036110388029	-3.61407471546942	0.636621979814147\\
0.417980510164007	-3.595788132402	0.636621979814147\\
0.631826428143805	-3.56443478895309	0.636621979814147\\
0.8474456342681	-3.51940845838614	0.636621979814147\\
1.06359921451595	-3.46022495090725	0.636621979814147\\
1.27895311930956	-3.38654380137159	0.636621979814147\\
1.49209843060018	-3.29818772561546	0.636621979814147\\
1.7015760009696	-3.19515869917666	0.636621979814147\\
1.90590480773437	-3.0776495680722	0.636621979814147\\
2.10361306363601	-2.94605025050489	0.636621979814147\\
2.29327086422797	-2.80094782944686	0.636621979814147\\
2.4735229611827	-2.64312015627402	0.636621979814147\\
2.64312015627402	-2.4735229611827	0.636621979814147\\
2.80094782944686	-2.29327086422797	0.636621979814147\\
2.94605025050489	-2.10361306363601	0.636621979814147\\
3.0776495680722	-1.90590480773437	0.636621979814147\\
3.19515869917666	-1.70157600096961	0.636621979814147\\
3.29818772561546	-1.49209843060018	0.636621979814147\\
3.38654380137159	-1.27895311930956	0.636621979814147\\
3.46022495090725	-1.06359921451595	0.636621979814147\\
3.51940845838614	-0.847445634268102	0.636621979814147\\
3.56443478895309	-0.631826428143808	0.636621979814147\\
3.595788132402	-0.417980510164009	0.636621979814147\\
3.61407471546942	-0.207036110388031	0.636621979814147\\
3.62	-1.46878468176563e-15	0.636621979814147\\
3.64	0	0.636621979814147\\
3.6340419790908	0.208179956301776	0.636621979814147\\
3.6156543651777	0.420289794750549	0.636621979814147\\
3.58412779883681	0.635317181890457	0.636621979814147\\
3.53885270401259	0.85212765434693	0.636621979814147\\
3.47934221582939	1.06947545327018	0.636621979814147\\
3.4052539881195	1.28601915864276	0.636621979814147\\
3.31640975724869	1.5003420683383	0.636621979814147\\
3.21281150966935	1.71097697335065	0.636621979814147\\
3.09465315684608	1.91643466855058	0.636621979814147\\
2.96232677122591	2.1152352352583	0.636621979814147\\
2.81642267933331	2.30594086900271	0.636621979814147\\
2.65772303006558	2.48718883389642	0.636621979814147\\
2.48718883389642	2.65772303006558	0.636621979814147\\
2.30594086900271	2.81642267933331	0.636621979814147\\
2.11523523525831	2.96232677122591	0.636621979814147\\
1.91643466855058	3.09465315684608	0.636621979814147\\
1.71097697335065	3.21281150966935	0.636621979814147\\
1.5003420683383	3.31640975724869	0.636621979814147\\
1.28601915864276	3.4052539881195	0.636621979814147\\
1.06947545327018	3.47934221582939	0.636621979814147\\
0.852127654346931	3.53885270401259	0.636621979814147\\
0.635317181890457	3.58412779883681	0.636621979814147\\
0.420289794750549	3.6156543651777	0.636621979814147\\
0.208179956301777	3.6340419790908	0.636621979814147\\
8.08242361927114e-16	3.64	0.636621979814147\\
-0.208179956301775	3.6340419790908	0.636621979814147\\
-0.420289794750548	3.6156543651777	0.636621979814147\\
-0.635317181890457	3.58412779883681	0.636621979814147\\
-0.852127654346929	3.53885270401259	0.636621979814147\\
-1.06947545327018	3.47934221582939	0.636621979814147\\
-1.28601915864276	3.4052539881195	0.636621979814147\\
-1.5003420683383	3.31640975724869	0.636621979814147\\
-1.71097697335065	3.21281150966935	0.636621979814147\\
-1.91643466855058	3.09465315684608	0.636621979814147\\
-2.1152352352583	2.96232677122591	0.636621979814147\\
-2.30594086900271	2.81642267933331	0.636621979814147\\
-2.48718883389641	2.65772303006559	0.636621979814147\\
-2.65772303006558	2.48718883389642	0.636621979814147\\
-2.81642267933331	2.30594086900271	0.636621979814147\\
-2.96232677122591	2.11523523525831	0.636621979814147\\
-3.09465315684608	1.91643466855058	0.636621979814147\\
-3.21281150966935	1.71097697335065	0.636621979814147\\
-3.31640975724869	1.5003420683383	0.636621979814147\\
-3.4052539881195	1.28601915864276	0.636621979814147\\
-3.47934221582939	1.06947545327018	0.636621979814147\\
-3.53885270401259	0.85212765434693	0.636621979814147\\
-3.58412779883681	0.635317181890458	0.636621979814147\\
-3.6156543651777	0.42028979475055	0.636621979814147\\
-3.6340419790908	0.208179956301777	0.636621979814147\\
-3.64	1.03112807937193e-15	0.636621979814147\\
-3.6340419790908	-0.208179956301775	0.636621979814147\\
-3.6156543651777	-0.420289794750548	0.636621979814147\\
-3.58412779883681	-0.635317181890456	0.636621979814147\\
-3.53885270401259	-0.85212765434693	0.636621979814147\\
-3.47934221582939	-1.06947545327018	0.636621979814147\\
-3.4052539881195	-1.28601915864276	0.636621979814147\\
-3.31640975724869	-1.5003420683383	0.636621979814147\\
-3.21281150966935	-1.71097697335065	0.636621979814147\\
-3.09465315684608	-1.91643466855058	0.636621979814147\\
-2.96232677122591	-2.1152352352583	0.636621979814147\\
-2.81642267933331	-2.30594086900271	0.636621979814147\\
-2.65772303006559	-2.48718883389641	0.636621979814147\\
-2.48718883389642	-2.65772303006558	0.636621979814147\\
-2.30594086900271	-2.81642267933331	0.636621979814147\\
-2.11523523525831	-2.96232677122591	0.636621979814147\\
-1.91643466855058	-3.09465315684608	0.636621979814147\\
-1.71097697335065	-3.21281150966935	0.636621979814147\\
-1.5003420683383	-3.31640975724869	0.636621979814147\\
-1.28601915864277	-3.4052539881195	0.636621979814147\\
-1.06947545327018	-3.47934221582939	0.636621979814147\\
-0.852127654346932	-3.53885270401259	0.636621979814147\\
-0.635317181890458	-3.58412779883681	0.636621979814147\\
-0.42028979475055	-3.6156543651777	0.636621979814147\\
-0.208179956301777	-3.6340419790908	0.636621979814147\\
-1.25401379681675e-15	-3.64	0.636621979814147\\
0.208179956301775	-3.6340419790908	0.636621979814147\\
0.420289794750548	-3.6156543651777	0.636621979814147\\
0.635317181890456	-3.58412779883681	0.636621979814147\\
0.85212765434693	-3.53885270401259	0.636621979814147\\
1.06947545327018	-3.47934221582939	0.636621979814147\\
1.28601915864276	-3.4052539881195	0.636621979814147\\
1.5003420683383	-3.31640975724869	0.636621979814147\\
1.71097697335065	-3.21281150966935	0.636621979814147\\
1.91643466855058	-3.09465315684608	0.636621979814147\\
2.1152352352583	-2.96232677122591	0.636621979814147\\
2.30594086900271	-2.81642267933331	0.636621979814147\\
2.48718883389641	-2.65772303006559	0.636621979814147\\
2.65772303006558	-2.48718883389642	0.636621979814147\\
2.81642267933331	-2.30594086900271	0.636621979814147\\
2.96232677122591	-2.11523523525831	0.636621979814147\\
3.09465315684608	-1.91643466855058	0.636621979814147\\
3.21281150966935	-1.71097697335065	0.636621979814147\\
3.31640975724869	-1.5003420683383	0.636621979814147\\
3.4052539881195	-1.28601915864277	0.636621979814147\\
3.47934221582939	-1.06947545327018	0.636621979814147\\
3.53885270401259	-0.852127654346932	0.636621979814147\\
3.58412779883681	-0.635317181890458	0.636621979814147\\
3.6156543651777	-0.42028979475055	0.636621979814147\\
3.6340419790908	-0.208179956301777	0.636621979814147\\
3.64	-1.47689951426157e-15	0.636621979814147\\
3.66	0	0.636621979814147\\
3.65400924271218	0.209323802215522	0.636621979814147\\
3.6355205979534	0.42259907933709	0.636621979814147\\
3.60382080872052	0.638807935637108	0.636621979814147\\
3.55829694963903	0.856809674425759	0.636621979814147\\
3.49845948075153	1.07535169202441	0.636621979814147\\
3.42396417486741	1.29308519797597	0.636621979814147\\
3.33463178888193	1.50858570607643	0.636621979814147\\
3.23046432016204	1.7203779457317	0.636621979814147\\
3.11165674561996	1.92696452936679	0.636621979814147\\
2.97860329194693	2.1268574068806	0.636621979814147\\
2.83189752921976	2.31861087377745	0.636621979814147\\
2.67232590385715	2.50085470661013	0.636621979814147\\
2.50085470661013	2.67232590385715	0.636621979814147\\
2.31861087377745	2.83189752921975	0.636621979814147\\
2.1268574068806	2.97860329194693	0.636621979814147\\
1.92696452936679	3.11165674561996	0.636621979814147\\
1.7203779457317	3.23046432016204	0.636621979814147\\
1.50858570607643	3.33463178888193	0.636621979814147\\
1.29308519797597	3.42396417486741	0.636621979814147\\
1.07535169202441	3.49845948075153	0.636621979814147\\
0.85680967442576	3.55829694963903	0.636621979814147\\
0.638807935637108	3.60382080872052	0.636621979814147\\
0.422599079337091	3.6355205979534	0.636621979814147\\
0.209323802215523	3.65400924271218	0.636621979814147\\
8.12683254025615e-16	3.66	0.636621979814147\\
-0.209323802215521	3.65400924271218	0.636621979814147\\
-0.42259907933709	3.6355205979534	0.636621979814147\\
-0.638807935637107	3.60382080872052	0.636621979814147\\
-0.856809674425758	3.55829694963903	0.636621979814147\\
-1.07535169202441	3.49845948075153	0.636621979814147\\
-1.29308519797596	3.42396417486741	0.636621979814147\\
-1.50858570607643	3.33463178888193	0.636621979814147\\
-1.7203779457317	3.23046432016204	0.636621979814147\\
-1.92696452936679	3.11165674561996	0.636621979814147\\
-2.1268574068806	2.97860329194694	0.636621979814147\\
-2.31861087377745	2.83189752921975	0.636621979814147\\
-2.50085470661013	2.67232590385715	0.636621979814147\\
-2.67232590385715	2.50085470661013	0.636621979814147\\
-2.83189752921975	2.31861087377745	0.636621979814147\\
-2.97860329194693	2.1268574068806	0.636621979814147\\
-3.11165674561996	1.92696452936679	0.636621979814147\\
-3.23046432016204	1.7203779457317	0.636621979814147\\
-3.33463178888193	1.50858570607643	0.636621979814147\\
-3.42396417486741	1.29308519797597	0.636621979814147\\
-3.49845948075153	1.07535169202441	0.636621979814147\\
-3.55829694963903	0.85680967442576	0.636621979814147\\
-3.60382080872052	0.638807935637109	0.636621979814147\\
-3.6355205979534	0.422599079337091	0.636621979814147\\
-3.65400924271218	0.209323802215523	0.636621979814147\\
-3.66	1.03679361826958e-15	0.636621979814147\\
-3.65400924271218	-0.209323802215521	0.636621979814147\\
-3.6355205979534	-0.42259907933709	0.636621979814147\\
-3.60382080872052	-0.638807935637107	0.636621979814147\\
-3.55829694963903	-0.856809674425759	0.636621979814147\\
-3.49845948075153	-1.07535169202441	0.636621979814147\\
-3.42396417486741	-1.29308519797597	0.636621979814147\\
-3.33463178888193	-1.50858570607642	0.636621979814147\\
-3.23046432016204	-1.7203779457317	0.636621979814147\\
-3.11165674561996	-1.92696452936679	0.636621979814147\\
-2.97860329194694	-2.1268574068806	0.636621979814147\\
-2.83189752921976	-2.31861087377745	0.636621979814147\\
-2.67232590385715	-2.50085470661013	0.636621979814147\\
-2.50085470661013	-2.67232590385715	0.636621979814147\\
-2.31861087377745	-2.83189752921975	0.636621979814147\\
-2.1268574068806	-2.97860329194693	0.636621979814147\\
-1.92696452936679	-3.11165674561996	0.636621979814147\\
-1.7203779457317	-3.23046432016204	0.636621979814147\\
-1.50858570607643	-3.33463178888193	0.636621979814147\\
-1.29308519797597	-3.42396417486741	0.636621979814147\\
-1.07535169202441	-3.49845948075153	0.636621979814147\\
-0.856809674425761	-3.55829694963903	0.636621979814147\\
-0.638807935637109	-3.60382080872052	0.636621979814147\\
-0.422599079337092	-3.6355205979534	0.636621979814147\\
-0.209323802215523	-3.65400924271218	0.636621979814147\\
-1.26090398251355e-15	-3.66	0.636621979814147\\
0.209323802215521	-3.65400924271218	0.636621979814147\\
0.422599079337089	-3.6355205979534	0.636621979814147\\
0.638807935637107	-3.60382080872052	0.636621979814147\\
0.856809674425759	-3.55829694963903	0.636621979814147\\
1.07535169202441	-3.49845948075153	0.636621979814147\\
1.29308519797597	-3.42396417486741	0.636621979814147\\
1.50858570607642	-3.33463178888193	0.636621979814147\\
1.7203779457317	-3.23046432016204	0.636621979814147\\
1.92696452936679	-3.11165674561996	0.636621979814147\\
2.1268574068806	-2.97860329194694	0.636621979814147\\
2.31861087377745	-2.83189752921976	0.636621979814147\\
2.50085470661013	-2.67232590385715	0.636621979814147\\
2.67232590385715	-2.50085470661013	0.636621979814147\\
2.83189752921975	-2.31861087377745	0.636621979814147\\
2.97860329194693	-2.1268574068806	0.636621979814147\\
3.11165674561996	-1.92696452936679	0.636621979814147\\
3.23046432016204	-1.7203779457317	0.636621979814147\\
3.33463178888193	-1.50858570607643	0.636621979814147\\
3.42396417486741	-1.29308519797597	0.636621979814147\\
3.49845948075153	-1.07535169202441	0.636621979814147\\
3.55829694963903	-0.856809674425761	0.636621979814147\\
3.60382080872052	-0.638807935637109	0.636621979814147\\
3.6355205979534	-0.422599079337092	0.636621979814147\\
3.65400924271218	-0.209323802215523	0.636621979814147\\
3.66	-1.48501434675751e-15	0.636621979814147\\
3.68	0	0.636621979814147\\
3.67397650633356	0.210467648129268	0.636621979814147\\
3.6553868307291	0.424908363923632	0.636621979814147\\
3.62351381860424	0.642298689383758	0.636621979814147\\
3.57774119526547	0.861491694504589	0.636621979814147\\
3.51757674567367	1.08122793077864	0.636621979814147\\
3.44267436161532	1.30015123730917	0.636621979814147\\
3.35285382051516	1.51682934381455	0.636621979814147\\
3.24811713065473	1.72977891811275	0.636621979814147\\
3.12866033439384	1.93749439018301	0.636621979814147\\
2.99487981266796	2.1384795785029	0.636621979814147\\
2.8473723791062	2.33128087855219	0.636621979814147\\
2.68692877764872	2.51452057932385	0.636621979814147\\
2.51452057932385	2.68692877764872	0.636621979814147\\
2.33128087855219	2.8473723791062	0.636621979814147\\
2.1384795785029	2.99487981266796	0.636621979814147\\
1.93749439018301	3.12866033439384	0.636621979814147\\
1.72977891811275	3.24811713065473	0.636621979814147\\
1.51682934381455	3.35285382051516	0.636621979814147\\
1.30015123730917	3.44267436161532	0.636621979814147\\
1.08122793077864	3.51757674567367	0.636621979814147\\
0.86149169450459	3.57774119526547	0.636621979814147\\
0.642298689383759	3.62351381860424	0.636621979814147\\
0.424908363923632	3.6553868307291	0.636621979814147\\
0.210467648129269	3.67397650633356	0.636621979814147\\
8.17124146124115e-16	3.68	0.636621979814147\\
-0.210467648129267	3.67397650633356	0.636621979814147\\
-0.424908363923631	3.6553868307291	0.636621979814147\\
-0.642298689383758	3.62351381860424	0.636621979814147\\
-0.861491694504587	3.57774119526547	0.636621979814147\\
-1.08122793077864	3.51757674567367	0.636621979814147\\
-1.30015123730917	3.44267436161532	0.636621979814147\\
-1.51682934381455	3.35285382051516	0.636621979814147\\
-1.72977891811275	3.24811713065473	0.636621979814147\\
-1.937494390183	3.12866033439384	0.636621979814147\\
-2.1384795785029	2.99487981266796	0.636621979814147\\
-2.33128087855219	2.8473723791062	0.636621979814147\\
-2.51452057932385	2.68692877764872	0.636621979814147\\
-2.68692877764872	2.51452057932385	0.636621979814147\\
-2.8473723791062	2.33128087855219	0.636621979814147\\
-2.99487981266796	2.1384795785029	0.636621979814147\\
-3.12866033439384	1.93749439018301	0.636621979814147\\
-3.24811713065473	1.72977891811275	0.636621979814147\\
-3.35285382051516	1.51682934381455	0.636621979814147\\
-3.44267436161532	1.30015123730917	0.636621979814147\\
-3.51757674567367	1.08122793077864	0.636621979814147\\
-3.57774119526547	0.861491694504589	0.636621979814147\\
-3.62351381860424	0.64229868938376	0.636621979814147\\
-3.6553868307291	0.424908363923633	0.636621979814147\\
-3.67397650633356	0.210467648129269	0.636621979814147\\
-3.68	1.04245915716723e-15	0.636621979814147\\
-3.67397650633356	-0.210467648129267	0.636621979814147\\
-3.6553868307291	-0.424908363923631	0.636621979814147\\
-3.62351381860424	-0.642298689383758	0.636621979814147\\
-3.57774119526547	-0.861491694504589	0.636621979814147\\
-3.51757674567367	-1.08122793077864	0.636621979814147\\
-3.44267436161532	-1.30015123730917	0.636621979814147\\
-3.35285382051516	-1.51682934381455	0.636621979814147\\
-3.24811713065473	-1.72977891811275	0.636621979814147\\
-3.12866033439384	-1.937494390183	0.636621979814147\\
-2.99487981266796	-2.1384795785029	0.636621979814147\\
-2.8473723791062	-2.33128087855219	0.636621979814147\\
-2.68692877764872	-2.51452057932385	0.636621979814147\\
-2.51452057932385	-2.68692877764872	0.636621979814147\\
-2.33128087855219	-2.8473723791062	0.636621979814147\\
-2.1384795785029	-2.99487981266796	0.636621979814147\\
-1.93749439018301	-3.12866033439384	0.636621979814147\\
-1.72977891811275	-3.24811713065473	0.636621979814147\\
-1.51682934381455	-3.35285382051516	0.636621979814147\\
-1.30015123730917	-3.44267436161532	0.636621979814147\\
-1.08122793077865	-3.51757674567367	0.636621979814147\\
-0.86149169450459	-3.57774119526547	0.636621979814147\\
-0.64229868938376	-3.62351381860424	0.636621979814147\\
-0.424908363923633	-3.6553868307291	0.636621979814147\\
-0.210467648129269	-3.67397650633356	0.636621979814147\\
-1.26779416821034e-15	-3.68	0.636621979814147\\
0.210467648129267	-3.67397650633356	0.636621979814147\\
0.424908363923631	-3.6553868307291	0.636621979814147\\
0.642298689383758	-3.62351381860424	0.636621979814147\\
0.861491694504588	-3.57774119526547	0.636621979814147\\
1.08122793077864	-3.51757674567367	0.636621979814147\\
1.30015123730917	-3.44267436161532	0.636621979814147\\
1.51682934381455	-3.35285382051516	0.636621979814147\\
1.72977891811275	-3.24811713065473	0.636621979814147\\
1.937494390183	-3.12866033439384	0.636621979814147\\
2.1384795785029	-2.99487981266796	0.636621979814147\\
2.33128087855219	-2.8473723791062	0.636621979814147\\
2.51452057932385	-2.68692877764872	0.636621979814147\\
2.68692877764872	-2.51452057932385	0.636621979814147\\
2.8473723791062	-2.33128087855219	0.636621979814147\\
2.99487981266796	-2.1384795785029	0.636621979814147\\
3.12866033439384	-1.93749439018301	0.636621979814147\\
3.24811713065473	-1.72977891811275	0.636621979814147\\
3.35285382051516	-1.51682934381455	0.636621979814147\\
3.44267436161532	-1.30015123730917	0.636621979814147\\
3.51757674567367	-1.08122793077865	0.636621979814147\\
3.57774119526547	-0.86149169450459	0.636621979814147\\
3.62351381860424	-0.64229868938376	0.636621979814147\\
3.6553868307291	-0.424908363923633	0.636621979814147\\
3.67397650633356	-0.210467648129269	0.636621979814147\\
3.68	-1.49312917925345e-15	0.636621979814147\\
3.7	0	0.636621979814147\\
3.69394376995494	0.211611494043014	0.636621979814147\\
3.6752530635048	0.427217648510173	0.636621979814147\\
3.64320682848796	0.645789443130409	0.636621979814147\\
3.59718544089192	0.866173714583418	0.636621979814147\\
3.5366940105958	1.08710416953287	0.636621979814147\\
3.46138454836323	1.30721727664237	0.636621979814147\\
3.3710758521484	1.52507298155267	0.636621979814147\\
3.26576994114741	1.7391798904938	0.636621979814147\\
3.14566392316772	1.94802425099922	0.636621979814147\\
3.01115633338898	2.1501017501252	0.636621979814147\\
2.86284722899265	2.34395088332693	0.636621979814147\\
2.70153165144029	2.52818645203757	0.636621979814147\\
2.52818645203757	2.70153165144029	0.636621979814147\\
2.34395088332693	2.86284722899265	0.636621979814147\\
2.1501017501252	3.01115633338898	0.636621979814147\\
1.94802425099922	3.14566392316772	0.636621979814147\\
1.7391798904938	3.26576994114741	0.636621979814147\\
1.52507298155267	3.3710758521484	0.636621979814147\\
1.30721727664237	3.46138454836323	0.636621979814147\\
1.08710416953288	3.5366940105958	0.636621979814147\\
0.866173714583419	3.59718544089192	0.636621979814147\\
0.64578944313041	3.64320682848796	0.636621979814147\\
0.427217648510174	3.6752530635048	0.636621979814147\\
0.211611494043015	3.69394376995494	0.636621979814147\\
8.21565038222616e-16	3.7	0.636621979814147\\
-0.211611494043013	3.69394376995494	0.636621979814147\\
-0.427217648510173	3.6752530635048	0.636621979814147\\
-0.645789443130409	3.64320682848796	0.636621979814147\\
-0.866173714583417	3.59718544089192	0.636621979814147\\
-1.08710416953287	3.5366940105958	0.636621979814147\\
-1.30721727664237	3.46138454836323	0.636621979814147\\
-1.52507298155267	3.3710758521484	0.636621979814147\\
-1.7391798904938	3.26576994114741	0.636621979814147\\
-1.94802425099922	3.14566392316772	0.636621979814147\\
-2.1501017501252	3.01115633338898	0.636621979814147\\
-2.34395088332693	2.86284722899265	0.636621979814147\\
-2.52818645203756	2.70153165144029	0.636621979814147\\
-2.70153165144029	2.52818645203757	0.636621979814147\\
-2.86284722899265	2.34395088332693	0.636621979814147\\
-3.01115633338898	2.1501017501252	0.636621979814147\\
-3.14566392316772	1.94802425099922	0.636621979814147\\
-3.26576994114741	1.7391798904938	0.636621979814147\\
-3.3710758521484	1.52507298155267	0.636621979814147\\
-3.46138454836323	1.30721727664237	0.636621979814147\\
-3.5366940105958	1.08710416953288	0.636621979814147\\
-3.59718544089192	0.866173714583418	0.636621979814147\\
-3.64320682848796	0.645789443130411	0.636621979814147\\
-3.6752530635048	0.427217648510174	0.636621979814147\\
-3.69394376995494	0.211611494043015	0.636621979814147\\
-3.7	1.04812469606488e-15	0.636621979814147\\
-3.69394376995494	-0.211611494043013	0.636621979814147\\
-3.6752530635048	-0.427217648510173	0.636621979814147\\
-3.64320682848796	-0.645789443130409	0.636621979814147\\
-3.59718544089192	-0.866173714583418	0.636621979814147\\
-3.5366940105958	-1.08710416953288	0.636621979814147\\
-3.46138454836323	-1.30721727664237	0.636621979814147\\
-3.3710758521484	-1.52507298155267	0.636621979814147\\
-3.26576994114742	-1.73917989049379	0.636621979814147\\
-3.14566392316772	-1.94802425099922	0.636621979814147\\
-3.01115633338898	-2.1501017501252	0.636621979814147\\
-2.86284722899265	-2.34395088332693	0.636621979814147\\
-2.70153165144029	-2.52818645203756	0.636621979814147\\
-2.52818645203757	-2.70153165144029	0.636621979814147\\
-2.34395088332693	-2.86284722899265	0.636621979814147\\
-2.1501017501252	-3.01115633338898	0.636621979814147\\
-1.94802425099922	-3.14566392316772	0.636621979814147\\
-1.7391798904938	-3.26576994114742	0.636621979814147\\
-1.52507298155267	-3.3710758521484	0.636621979814147\\
-1.30721727664237	-3.46138454836323	0.636621979814147\\
-1.08710416953288	-3.5366940105958	0.636621979814147\\
-0.86617371458342	-3.59718544089192	0.636621979814147\\
-0.645789443130411	-3.64320682848796	0.636621979814147\\
-0.427217648510175	-3.6752530635048	0.636621979814147\\
-0.211611494043015	-3.69394376995494	0.636621979814147\\
-1.27468435390714e-15	-3.7	0.636621979814147\\
0.211611494043013	-3.69394376995494	0.636621979814147\\
0.427217648510172	-3.6752530635048	0.636621979814147\\
0.645789443130409	-3.64320682848796	0.636621979814147\\
0.866173714583418	-3.59718544089192	0.636621979814147\\
1.08710416953288	-3.5366940105958	0.636621979814147\\
1.30721727664237	-3.46138454836323	0.636621979814147\\
1.52507298155267	-3.3710758521484	0.636621979814147\\
1.73917989049379	-3.26576994114742	0.636621979814147\\
1.94802425099922	-3.14566392316772	0.636621979814147\\
2.1501017501252	-3.01115633338898	0.636621979814147\\
2.34395088332693	-2.86284722899265	0.636621979814147\\
2.52818645203756	-2.70153165144029	0.636621979814147\\
2.70153165144029	-2.52818645203757	0.636621979814147\\
2.86284722899265	-2.34395088332693	0.636621979814147\\
3.01115633338898	-2.1501017501252	0.636621979814147\\
3.14566392316772	-1.94802425099922	0.636621979814147\\
3.26576994114742	-1.7391798904938	0.636621979814147\\
3.3710758521484	-1.52507298155267	0.636621979814147\\
3.46138454836323	-1.30721727664237	0.636621979814147\\
3.5366940105958	-1.08710416953288	0.636621979814147\\
3.59718544089192	-0.86617371458342	0.636621979814147\\
3.64320682848796	-0.645789443130411	0.636621979814147\\
3.6752530635048	-0.427217648510175	0.636621979814147\\
3.69394376995494	-0.211611494043015	0.636621979814147\\
3.7	-1.5012440117494e-15	0.636621979814147\\
3.72	0	0.636621979814147\\
3.71391103357631	0.21275533995676	0.636621979814147\\
3.6951192962805	0.429526933096715	0.636621979814147\\
3.66289983837168	0.64928019687706	0.636621979814147\\
3.61662968651836	0.870855734662247	0.636621979814147\\
3.55581127551794	1.09298040828711	0.636621979814147\\
3.48009473511114	1.31428331597557	0.636621979814147\\
3.38929788378163	1.53331661929079	0.636621979814147\\
3.2834227516401	1.74858086287484	0.636621979814147\\
3.1626675119416	1.95855411181543	0.636621979814147\\
3.02743285411	2.1617239217475	0.636621979814147\\
2.87832207887909	2.35662088810167	0.636621979814147\\
2.71613452523186	2.54185232475128	0.636621979814147\\
2.54185232475128	2.71613452523186	0.636621979814147\\
2.35662088810167	2.87832207887909	0.636621979814147\\
2.1617239217475	3.02743285411	0.636621979814147\\
1.95855411181543	3.1626675119416	0.636621979814147\\
1.74858086287484	3.2834227516401	0.636621979814147\\
1.53331661929079	3.38929788378163	0.636621979814147\\
1.31428331597557	3.48009473511114	0.636621979814147\\
1.09298040828711	3.55581127551794	0.636621979814147\\
0.870855734662248	3.61662968651836	0.636621979814147\\
0.649280196877061	3.66289983837168	0.636621979814147\\
0.429526933096715	3.6951192962805	0.636621979814147\\
0.212755339956761	3.71391103357631	0.636621979814147\\
8.26005930321116e-16	3.72	0.636621979814147\\
-0.212755339956759	3.71391103357631	0.636621979814147\\
-0.429526933096714	3.6951192962805	0.636621979814147\\
-0.64928019687706	3.66289983837168	0.636621979814147\\
-0.870855734662246	3.61662968651836	0.636621979814147\\
-1.09298040828711	3.55581127551794	0.636621979814147\\
-1.31428331597557	3.48009473511114	0.636621979814147\\
-1.53331661929079	3.38929788378163	0.636621979814147\\
-1.74858086287484	3.2834227516401	0.636621979814147\\
-1.95855411181543	3.1626675119416	0.636621979814147\\
-2.1617239217475	3.02743285411	0.636621979814147\\
-2.35662088810167	2.87832207887909	0.636621979814147\\
-2.54185232475128	2.71613452523186	0.636621979814147\\
-2.71613452523186	2.54185232475128	0.636621979814147\\
-2.87832207887909	2.35662088810167	0.636621979814147\\
-3.02743285411	2.1617239217475	0.636621979814147\\
-3.1626675119416	1.95855411181543	0.636621979814147\\
-3.2834227516401	1.74858086287484	0.636621979814147\\
-3.38929788378163	1.53331661929079	0.636621979814147\\
-3.48009473511114	1.31428331597557	0.636621979814147\\
-3.55581127551794	1.09298040828711	0.636621979814147\\
-3.61662968651836	0.870855734662248	0.636621979814147\\
-3.66289983837168	0.649280196877062	0.636621979814147\\
-3.6951192962805	0.429526933096716	0.636621979814147\\
-3.71391103357631	0.212755339956761	0.636621979814147\\
-3.72	1.05379023496252e-15	0.636621979814147\\
-3.71391103357631	-0.212755339956759	0.636621979814147\\
-3.6951192962805	-0.429526933096714	0.636621979814147\\
-3.66289983837168	-0.64928019687706	0.636621979814147\\
-3.61662968651836	-0.870855734662247	0.636621979814147\\
-3.55581127551794	-1.09298040828711	0.636621979814147\\
-3.48009473511114	-1.31428331597557	0.636621979814147\\
-3.38929788378163	-1.53331661929079	0.636621979814147\\
-3.2834227516401	-1.74858086287484	0.636621979814147\\
-3.1626675119416	-1.95855411181543	0.636621979814147\\
-3.02743285411	-2.1617239217475	0.636621979814147\\
-2.8783220788791	-2.35662088810167	0.636621979814147\\
-2.71613452523186	-2.54185232475128	0.636621979814147\\
-2.54185232475128	-2.71613452523186	0.636621979814147\\
-2.35662088810167	-2.87832207887909	0.636621979814147\\
-2.1617239217475	-3.02743285411	0.636621979814147\\
-1.95855411181543	-3.1626675119416	0.636621979814147\\
-1.74858086287484	-3.2834227516401	0.636621979814147\\
-1.53331661929079	-3.38929788378163	0.636621979814147\\
-1.31428331597557	-3.48009473511114	0.636621979814147\\
-1.09298040828711	-3.55581127551794	0.636621979814147\\
-0.870855734662249	-3.61662968651836	0.636621979814147\\
-0.649280196877062	-3.66289983837168	0.636621979814147\\
-0.429526933096716	-3.6951192962805	0.636621979814147\\
-0.212755339956761	-3.71391103357631	0.636621979814147\\
-1.28157453960393e-15	-3.72	0.636621979814147\\
0.212755339956759	-3.71391103357631	0.636621979814147\\
0.429526933096714	-3.6951192962805	0.636621979814147\\
0.64928019687706	-3.66289983837168	0.636621979814147\\
0.870855734662247	-3.61662968651836	0.636621979814147\\
1.09298040828711	-3.55581127551794	0.636621979814147\\
1.31428331597557	-3.48009473511114	0.636621979814147\\
1.53331661929079	-3.38929788378163	0.636621979814147\\
1.74858086287484	-3.2834227516401	0.636621979814147\\
1.95855411181543	-3.1626675119416	0.636621979814147\\
2.1617239217475	-3.02743285411	0.636621979814147\\
2.35662088810167	-2.8783220788791	0.636621979814147\\
2.54185232475128	-2.71613452523186	0.636621979814147\\
2.71613452523186	-2.54185232475128	0.636621979814147\\
2.87832207887909	-2.35662088810167	0.636621979814147\\
3.02743285411	-2.1617239217475	0.636621979814147\\
3.1626675119416	-1.95855411181543	0.636621979814147\\
3.2834227516401	-1.74858086287484	0.636621979814147\\
3.38929788378163	-1.53331661929079	0.636621979814147\\
3.48009473511114	-1.31428331597557	0.636621979814147\\
3.55581127551794	-1.09298040828711	0.636621979814147\\
3.61662968651836	-0.870855734662249	0.636621979814147\\
3.66289983837168	-0.649280196877062	0.636621979814147\\
3.6951192962805	-0.429526933096716	0.636621979814147\\
3.71391103357631	-0.212755339956761	0.636621979814147\\
3.72	-1.50935884424534e-15	0.636621979814147\\
3.74	0	0.636621979814147\\
3.73387829719769	0.213899185870506	0.636621979814147\\
3.71498552905621	0.431836217683256	0.636621979814147\\
3.6825928482554	0.652770950623711	0.636621979814147\\
3.6360739321448	0.875537754741077	0.636621979814147\\
3.57492854044008	1.09885664704134	0.636621979814147\\
3.49880492185905	1.32134935530877	0.636621979814147\\
3.40751991541487	1.54156025702892	0.636621979814147\\
3.30107556213279	1.75798183525589	0.636621979814147\\
3.17967110071548	1.96908397263164	0.636621979814147\\
3.04370937483102	2.1733460933698	0.636621979814147\\
2.89379692876554	2.36929089287641	0.636621979814147\\
2.73073739902343	2.555518197465	0.636621979814147\\
2.555518197465	2.73073739902343	0.636621979814147\\
2.36929089287641	2.89379692876554	0.636621979814147\\
2.1733460933698	3.04370937483102	0.636621979814147\\
1.96908397263164	3.17967110071548	0.636621979814147\\
1.75798183525589	3.30107556213279	0.636621979814147\\
1.54156025702892	3.40751991541487	0.636621979814147\\
1.32134935530877	3.49880492185905	0.636621979814147\\
1.09885664704134	3.57492854044008	0.636621979814147\\
0.875537754741077	3.6360739321448	0.636621979814147\\
0.652770950623712	3.6825928482554	0.636621979814147\\
0.431836217683257	3.71498552905621	0.636621979814147\\
0.213899185870507	3.73387829719769	0.636621979814147\\
8.30446822419617e-16	3.74	0.636621979814147\\
-0.213899185870505	3.73387829719769	0.636621979814147\\
-0.431836217683256	3.71498552905621	0.636621979814147\\
-0.652770950623711	3.6825928482554	0.636621979814147\\
-0.875537754741075	3.6360739321448	0.636621979814147\\
-1.09885664704134	3.57492854044008	0.636621979814147\\
-1.32134935530877	3.49880492185905	0.636621979814147\\
-1.54156025702892	3.40751991541487	0.636621979814147\\
-1.75798183525589	3.30107556213279	0.636621979814147\\
-1.96908397263164	3.17967110071548	0.636621979814147\\
-2.1733460933698	3.04370937483102	0.636621979814147\\
-2.36929089287641	2.89379692876554	0.636621979814147\\
-2.555518197465	2.73073739902343	0.636621979814147\\
-2.73073739902343	2.555518197465	0.636621979814147\\
-2.89379692876554	2.36929089287641	0.636621979814147\\
-3.04370937483102	2.1733460933698	0.636621979814147\\
-3.17967110071548	1.96908397263164	0.636621979814147\\
-3.30107556213279	1.75798183525589	0.636621979814147\\
-3.40751991541487	1.54156025702892	0.636621979814147\\
-3.49880492185905	1.32134935530877	0.636621979814147\\
-3.57492854044008	1.09885664704134	0.636621979814147\\
-3.6360739321448	0.875537754741077	0.636621979814147\\
-3.6825928482554	0.652770950623712	0.636621979814147\\
-3.71498552905621	0.431836217683257	0.636621979814147\\
-3.73387829719769	0.213899185870507	0.636621979814147\\
-3.74	1.05945577386017e-15	0.636621979814147\\
-3.73387829719769	-0.213899185870505	0.636621979814147\\
-3.71498552905621	-0.431836217683256	0.636621979814147\\
-3.6825928482554	-0.652770950623711	0.636621979814147\\
-3.6360739321448	-0.875537754741076	0.636621979814147\\
-3.57492854044008	-1.09885664704134	0.636621979814147\\
-3.49880492185905	-1.32134935530877	0.636621979814147\\
-3.40751991541487	-1.54156025702891	0.636621979814147\\
-3.30107556213279	-1.75798183525589	0.636621979814147\\
-3.17967110071548	-1.96908397263164	0.636621979814147\\
-3.04370937483102	-2.1733460933698	0.636621979814147\\
-2.89379692876554	-2.36929089287641	0.636621979814147\\
-2.73073739902343	-2.555518197465	0.636621979814147\\
-2.555518197465	-2.73073739902343	0.636621979814147\\
-2.36929089287641	-2.89379692876554	0.636621979814147\\
-2.1733460933698	-3.04370937483102	0.636621979814147\\
-1.96908397263164	-3.17967110071548	0.636621979814147\\
-1.75798183525589	-3.30107556213279	0.636621979814147\\
-1.54156025702892	-3.40751991541487	0.636621979814147\\
-1.32134935530878	-3.49880492185905	0.636621979814147\\
-1.09885664704134	-3.57492854044008	0.636621979814147\\
-0.875537754741078	-3.6360739321448	0.636621979814147\\
-0.652770950623713	-3.6825928482554	0.636621979814147\\
-0.431836217683258	-3.71498552905621	0.636621979814147\\
-0.213899185870507	-3.73387829719769	0.636621979814147\\
-1.28846472530073e-15	-3.74	0.636621979814147\\
0.213899185870505	-3.73387829719769	0.636621979814147\\
0.431836217683255	-3.71498552905621	0.636621979814147\\
0.65277095062371	-3.6825928482554	0.636621979814147\\
0.875537754741076	-3.6360739321448	0.636621979814147\\
1.09885664704134	-3.57492854044008	0.636621979814147\\
1.32134935530877	-3.49880492185905	0.636621979814147\\
1.54156025702891	-3.40751991541487	0.636621979814147\\
1.75798183525589	-3.30107556213279	0.636621979814147\\
1.96908397263164	-3.17967110071548	0.636621979814147\\
2.1733460933698	-3.04370937483102	0.636621979814147\\
2.36929089287641	-2.89379692876554	0.636621979814147\\
2.555518197465	-2.73073739902343	0.636621979814147\\
2.73073739902343	-2.555518197465	0.636621979814147\\
2.89379692876554	-2.36929089287641	0.636621979814147\\
3.04370937483102	-2.1733460933698	0.636621979814147\\
3.17967110071548	-1.96908397263164	0.636621979814147\\
3.30107556213279	-1.75798183525589	0.636621979814147\\
3.40751991541487	-1.54156025702892	0.636621979814147\\
3.49880492185905	-1.32134935530878	0.636621979814147\\
3.57492854044008	-1.09885664704134	0.636621979814147\\
3.6360739321448	-0.875537754741078	0.636621979814147\\
3.6825928482554	-0.652770950623713	0.636621979814147\\
3.71498552905621	-0.431836217683258	0.636621979814147\\
3.73387829719769	-0.213899185870507	0.636621979814147\\
3.74	-1.51747367674128e-15	0.636621979814147\\
3.76	0	0.636621979814147\\
3.75384556081907	0.215043031784252	0.636621979814147\\
3.73485176183191	0.434145502269797	0.636621979814147\\
3.70228585813912	0.656261704370362	0.636621979814147\\
3.65551817777124	0.880219774819906	0.636621979814147\\
3.59404580536222	1.10473288579557	0.636621979814147\\
3.51751510860696	1.32841539464198	0.636621979814147\\
3.4257419470481	1.54980389476704	0.636621979814147\\
3.31872837262548	1.76738280763694	0.636621979814147\\
3.19667468948936	1.97961383344785	0.636621979814147\\
3.05998589555204	2.1849682649921	0.636621979814147\\
2.90927177865199	2.38196089765115	0.636621979814147\\
2.745340272815	2.56918407017871	0.636621979814147\\
2.56918407017871	2.745340272815	0.636621979814147\\
2.38196089765115	2.90927177865199	0.636621979814147\\
2.1849682649921	3.05998589555204	0.636621979814147\\
1.97961383344785	3.19667468948936	0.636621979814147\\
1.76738280763694	3.31872837262548	0.636621979814147\\
1.54980389476704	3.4257419470481	0.636621979814147\\
1.32841539464198	3.51751510860696	0.636621979814147\\
1.10473288579557	3.59404580536222	0.636621979814147\\
0.880219774819907	3.65551817777124	0.636621979814147\\
0.656261704370363	3.70228585813912	0.636621979814147\\
0.434145502269798	3.73485176183191	0.636621979814147\\
0.215043031784253	3.75384556081907	0.636621979814147\\
8.34887714518118e-16	3.76	0.636621979814147\\
-0.215043031784251	3.75384556081907	0.636621979814147\\
-0.434145502269797	3.73485176183191	0.636621979814147\\
-0.656261704370362	3.70228585813912	0.636621979814147\\
-0.880219774819905	3.65551817777124	0.636621979814147\\
-1.10473288579557	3.59404580536222	0.636621979814147\\
-1.32841539464197	3.51751510860696	0.636621979814147\\
-1.54980389476704	3.4257419470481	0.636621979814147\\
-1.76738280763694	3.31872837262548	0.636621979814147\\
-1.97961383344785	3.19667468948936	0.636621979814147\\
-2.18496826499209	3.05998589555204	0.636621979814147\\
-2.38196089765115	2.90927177865199	0.636621979814147\\
-2.56918407017871	2.745340272815	0.636621979814147\\
-2.745340272815	2.56918407017871	0.636621979814147\\
-2.90927177865199	2.38196089765115	0.636621979814147\\
-3.05998589555204	2.1849682649921	0.636621979814147\\
-3.19667468948936	1.97961383344785	0.636621979814147\\
-3.31872837262548	1.76738280763694	0.636621979814147\\
-3.4257419470481	1.54980389476704	0.636621979814147\\
-3.51751510860696	1.32841539464198	0.636621979814147\\
-3.59404580536222	1.10473288579557	0.636621979814147\\
-3.65551817777124	0.880219774819906	0.636621979814147\\
-3.70228585813912	0.656261704370363	0.636621979814147\\
-3.73485176183191	0.434145502269799	0.636621979814147\\
-3.75384556081907	0.215043031784253	0.636621979814147\\
-3.76	1.06512131275782e-15	0.636621979814147\\
-3.75384556081907	-0.215043031784251	0.636621979814147\\
-3.73485176183191	-0.434145502269797	0.636621979814147\\
-3.70228585813912	-0.656261704370362	0.636621979814147\\
-3.65551817777124	-0.880219774819906	0.636621979814147\\
-3.59404580536222	-1.10473288579557	0.636621979814147\\
-3.51751510860696	-1.32841539464198	0.636621979814147\\
-3.4257419470481	-1.54980389476704	0.636621979814147\\
-3.31872837262548	-1.76738280763694	0.636621979814147\\
-3.19667468948936	-1.97961383344785	0.636621979814147\\
-3.05998589555204	-2.18496826499209	0.636621979814147\\
-2.90927177865199	-2.38196089765115	0.636621979814147\\
-2.745340272815	-2.56918407017871	0.636621979814147\\
-2.56918407017871	-2.745340272815	0.636621979814147\\
-2.38196089765115	-2.90927177865199	0.636621979814147\\
-2.1849682649921	-3.05998589555204	0.636621979814147\\
-1.97961383344785	-3.19667468948936	0.636621979814147\\
-1.76738280763694	-3.31872837262548	0.636621979814147\\
-1.54980389476704	-3.4257419470481	0.636621979814147\\
-1.32841539464198	-3.51751510860695	0.636621979814147\\
-1.10473288579557	-3.59404580536222	0.636621979814147\\
-0.880219774819908	-3.65551817777124	0.636621979814147\\
-0.656261704370364	-3.70228585813912	0.636621979814147\\
-0.434145502269799	-3.73485176183191	0.636621979814147\\
-0.215043031784253	-3.75384556081907	0.636621979814147\\
-1.29535491099752e-15	-3.76	0.636621979814147\\
0.215043031784251	-3.75384556081907	0.636621979814147\\
0.434145502269797	-3.73485176183191	0.636621979814147\\
0.656261704370361	-3.70228585813912	0.636621979814147\\
0.880219774819905	-3.65551817777124	0.636621979814147\\
1.10473288579557	-3.59404580536222	0.636621979814147\\
1.32841539464198	-3.51751510860696	0.636621979814147\\
1.54980389476704	-3.4257419470481	0.636621979814147\\
1.76738280763694	-3.31872837262548	0.636621979814147\\
1.97961383344785	-3.19667468948936	0.636621979814147\\
2.18496826499209	-3.05998589555204	0.636621979814147\\
2.38196089765115	-2.90927177865199	0.636621979814147\\
2.56918407017871	-2.745340272815	0.636621979814147\\
2.745340272815	-2.56918407017871	0.636621979814147\\
2.90927177865199	-2.38196089765115	0.636621979814147\\
3.05998589555204	-2.1849682649921	0.636621979814147\\
3.19667468948936	-1.97961383344785	0.636621979814147\\
3.31872837262548	-1.76738280763694	0.636621979814147\\
3.4257419470481	-1.54980389476704	0.636621979814147\\
3.51751510860695	-1.32841539464198	0.636621979814147\\
3.59404580536222	-1.10473288579557	0.636621979814147\\
3.65551817777124	-0.880219774819908	0.636621979814147\\
3.70228585813912	-0.656261704370364	0.636621979814147\\
3.73485176183191	-0.434145502269799	0.636621979814147\\
3.75384556081907	-0.215043031784253	0.636621979814147\\
3.76	-1.52558850923723e-15	0.636621979814147\\
3.78	0	0.636621979814147\\
3.77381282444045	0.216186877697998	0.636621979814147\\
3.75471799460761	0.436454786856339	0.636621979814147\\
3.72197886802284	0.659752458117013	0.636621979814147\\
3.67496242339769	0.884901794898735	0.636621979814147\\
3.61316307028436	1.1106091245498	0.636621979814147\\
3.53622529535487	1.33548143397518	0.636621979814147\\
3.44396397868134	1.55804753250516	0.636621979814147\\
3.33638118311817	1.77678378001799	0.636621979814147\\
3.21367827826324	1.99014369426407	0.636621979814147\\
3.07626241627306	2.19659043661439	0.636621979814147\\
2.92474662853844	2.3946309024259	0.636621979814147\\
2.75994314660657	2.58284994289243	0.636621979814147\\
2.58284994289243	2.75994314660657	0.636621979814147\\
2.3946309024259	2.92474662853844	0.636621979814147\\
2.19659043661439	3.07626241627306	0.636621979814147\\
1.99014369426407	3.21367827826324	0.636621979814147\\
1.77678378001799	3.33638118311817	0.636621979814147\\
1.55804753250516	3.44396397868134	0.636621979814147\\
1.33548143397518	3.53622529535487	0.636621979814147\\
1.1106091245498	3.61316307028436	0.636621979814147\\
0.884901794898736	3.67496242339769	0.636621979814147\\
0.659752458117013	3.72197886802284	0.636621979814147\\
0.43645478685634	3.75471799460761	0.636621979814147\\
0.216186877697999	3.77381282444045	0.636621979814147\\
8.39328606616618e-16	3.78	0.636621979814147\\
-0.216186877697997	3.77381282444045	0.636621979814147\\
-0.436454786856339	3.75471799460761	0.636621979814147\\
-0.659752458117013	3.72197886802284	0.636621979814147\\
-0.884901794898734	3.67496242339769	0.636621979814147\\
-1.1106091245498	3.61316307028436	0.636621979814147\\
-1.33548143397518	3.53622529535487	0.636621979814147\\
-1.55804753250516	3.44396397868134	0.636621979814147\\
-1.77678378001799	3.33638118311817	0.636621979814147\\
-1.99014369426406	3.21367827826324	0.636621979814147\\
-2.19659043661439	3.07626241627306	0.636621979814147\\
-2.39463090242589	2.92474662853844	0.636621979814147\\
-2.58284994289243	2.75994314660657	0.636621979814147\\
-2.75994314660657	2.58284994289243	0.636621979814147\\
-2.92474662853843	2.3946309024259	0.636621979814147\\
-3.07626241627306	2.19659043661439	0.636621979814147\\
-3.21367827826324	1.99014369426407	0.636621979814147\\
-3.33638118311817	1.77678378001799	0.636621979814147\\
-3.44396397868134	1.55804753250516	0.636621979814147\\
-3.53622529535487	1.33548143397518	0.636621979814147\\
-3.61316307028436	1.1106091245498	0.636621979814147\\
-3.67496242339769	0.884901794898735	0.636621979814147\\
-3.72197886802284	0.659752458117014	0.636621979814147\\
-3.75471799460761	0.43645478685634	0.636621979814147\\
-3.77381282444045	0.216186877697999	0.636621979814147\\
-3.78	1.07078685165547e-15	0.636621979814147\\
-3.77381282444045	-0.216186877697997	0.636621979814147\\
-3.75471799460761	-0.436454786856339	0.636621979814147\\
-3.72197886802284	-0.659752458117012	0.636621979814147\\
-3.67496242339769	-0.884901794898735	0.636621979814147\\
-3.61316307028436	-1.1106091245498	0.636621979814147\\
-3.53622529535487	-1.33548143397518	0.636621979814147\\
-3.44396397868134	-1.55804753250516	0.636621979814147\\
-3.33638118311817	-1.77678378001798	0.636621979814147\\
-3.21367827826324	-1.99014369426406	0.636621979814147\\
-3.07626241627306	-2.19659043661439	0.636621979814147\\
-2.92474662853844	-2.39463090242589	0.636621979814147\\
-2.75994314660657	-2.58284994289243	0.636621979814147\\
-2.58284994289243	-2.75994314660657	0.636621979814147\\
-2.3946309024259	-2.92474662853843	0.636621979814147\\
-2.19659043661439	-3.07626241627306	0.636621979814147\\
-1.99014369426407	-3.21367827826324	0.636621979814147\\
-1.77678378001799	-3.33638118311817	0.636621979814147\\
-1.55804753250516	-3.44396397868134	0.636621979814147\\
-1.33548143397518	-3.53622529535486	0.636621979814147\\
-1.1106091245498	-3.61316307028436	0.636621979814147\\
-0.884901794898737	-3.67496242339769	0.636621979814147\\
-0.659752458117015	-3.72197886802284	0.636621979814147\\
-0.436454786856341	-3.75471799460761	0.636621979814147\\
-0.216186877697999	-3.77381282444045	0.636621979814147\\
-1.30224509669432e-15	-3.78	0.636621979814147\\
0.216186877697997	-3.77381282444045	0.636621979814147\\
0.436454786856338	-3.75471799460761	0.636621979814147\\
0.659752458117012	-3.72197886802284	0.636621979814147\\
0.884901794898735	-3.67496242339769	0.636621979814147\\
1.1106091245498	-3.61316307028436	0.636621979814147\\
1.33548143397518	-3.53622529535487	0.636621979814147\\
1.55804753250516	-3.44396397868134	0.636621979814147\\
1.77678378001798	-3.33638118311817	0.636621979814147\\
1.99014369426406	-3.21367827826324	0.636621979814147\\
2.19659043661439	-3.07626241627306	0.636621979814147\\
2.39463090242589	-2.92474662853844	0.636621979814147\\
2.58284994289243	-2.75994314660657	0.636621979814147\\
2.75994314660657	-2.58284994289243	0.636621979814147\\
2.92474662853843	-2.3946309024259	0.636621979814147\\
3.07626241627306	-2.19659043661439	0.636621979814147\\
3.21367827826324	-1.99014369426407	0.636621979814147\\
3.33638118311817	-1.77678378001799	0.636621979814147\\
3.44396397868134	-1.55804753250516	0.636621979814147\\
3.53622529535486	-1.33548143397518	0.636621979814147\\
3.61316307028436	-1.1106091245498	0.636621979814147\\
3.67496242339769	-0.884901794898737	0.636621979814147\\
3.72197886802284	-0.659752458117015	0.636621979814147\\
3.75471799460761	-0.436454786856341	0.636621979814147\\
3.77381282444045	-0.216186877697999	0.636621979814147\\
3.78	-1.53370334173317e-15	0.636621979814147\\
3.8	0	0.636621979814147\\
3.79378008806183	0.217330723611744	0.636621979814147\\
3.77458422738331	0.43876407144288	0.636621979814147\\
3.74167187790656	0.663243211863664	0.636621979814147\\
3.69440666902413	0.889583814977564	0.636621979814147\\
3.6322803352065	1.11648536330403	0.636621979814147\\
3.55493548210277	1.34254747330838	0.636621979814147\\
3.46218601031457	1.56629117024328	0.636621979814147\\
3.35403399361086	1.78618475239903	0.636621979814147\\
3.23068186703712	2.00067355508028	0.636621979814147\\
3.09253893699409	2.20821260823669	0.636621979814147\\
2.94022147842488	2.40730090720063	0.636621979814147\\
2.77454602039814	2.59651581560615	0.636621979814147\\
2.59651581560615	2.77454602039814	0.636621979814147\\
2.40730090720064	2.94022147842488	0.636621979814147\\
2.20821260823669	3.09253893699409	0.636621979814147\\
2.00067355508028	3.23068186703712	0.636621979814147\\
1.78618475239903	3.35403399361086	0.636621979814147\\
1.56629117024328	3.46218601031457	0.636621979814147\\
1.34254747330838	3.55493548210277	0.636621979814147\\
1.11648536330403	3.6322803352065	0.636621979814147\\
0.889583814977565	3.69440666902413	0.636621979814147\\
0.663243211863664	3.74167187790656	0.636621979814147\\
0.438764071442881	3.77458422738331	0.636621979814147\\
0.217330723611745	3.79378008806183	0.636621979814147\\
8.43769498715119e-16	3.8	0.636621979814147\\
-0.217330723611743	3.79378008806183	0.636621979814147\\
-0.43876407144288	3.77458422738331	0.636621979814147\\
-0.663243211863664	3.74167187790656	0.636621979814147\\
-0.889583814977563	3.69440666902413	0.636621979814147\\
-1.11648536330403	3.6322803352065	0.636621979814147\\
-1.34254747330838	3.55493548210277	0.636621979814147\\
-1.56629117024328	3.46218601031457	0.636621979814147\\
-1.78618475239903	3.35403399361086	0.636621979814147\\
-2.00067355508028	3.23068186703712	0.636621979814147\\
-2.20821260823669	3.09253893699409	0.636621979814147\\
-2.40730090720063	2.94022147842488	0.636621979814147\\
-2.59651581560615	2.77454602039814	0.636621979814147\\
-2.77454602039814	2.59651581560615	0.636621979814147\\
-2.94022147842488	2.40730090720064	0.636621979814147\\
-3.09253893699409	2.20821260823669	0.636621979814147\\
-3.23068186703712	2.00067355508028	0.636621979814147\\
-3.35403399361086	1.78618475239903	0.636621979814147\\
-3.46218601031457	1.56629117024328	0.636621979814147\\
-3.55493548210277	1.34254747330838	0.636621979814147\\
-3.6322803352065	1.11648536330403	0.636621979814147\\
-3.69440666902413	0.889583814977565	0.636621979814147\\
-3.74167187790656	0.663243211863665	0.636621979814147\\
-3.77458422738331	0.438764071442882	0.636621979814147\\
-3.79378008806183	0.217330723611745	0.636621979814147\\
-3.8	1.07645239055312e-15	0.636621979814147\\
-3.79378008806183	-0.217330723611743	0.636621979814147\\
-3.77458422738331	-0.43876407144288	0.636621979814147\\
-3.74167187790656	-0.663243211863663	0.636621979814147\\
-3.69440666902413	-0.889583814977564	0.636621979814147\\
-3.6322803352065	-1.11648536330403	0.636621979814147\\
-3.55493548210277	-1.34254747330838	0.636621979814147\\
-3.46218601031457	-1.56629117024328	0.636621979814147\\
-3.35403399361086	-1.78618475239903	0.636621979814147\\
-3.23068186703712	-2.00067355508028	0.636621979814147\\
-3.09253893699409	-2.20821260823669	0.636621979814147\\
-2.94022147842488	-2.40730090720063	0.636621979814147\\
-2.77454602039814	-2.59651581560615	0.636621979814147\\
-2.59651581560615	-2.77454602039814	0.636621979814147\\
-2.40730090720064	-2.94022147842488	0.636621979814147\\
-2.20821260823669	-3.09253893699409	0.636621979814147\\
-2.00067355508028	-3.23068186703712	0.636621979814147\\
-1.78618475239903	-3.35403399361086	0.636621979814147\\
-1.56629117024328	-3.46218601031457	0.636621979814147\\
-1.34254747330838	-3.55493548210277	0.636621979814147\\
-1.11648536330404	-3.6322803352065	0.636621979814147\\
-0.889583814977566	-3.69440666902413	0.636621979814147\\
-0.663243211863665	-3.74167187790656	0.636621979814147\\
-0.438764071442882	-3.77458422738331	0.636621979814147\\
-0.217330723611745	-3.79378008806183	0.636621979814147\\
-1.30913528239111e-15	-3.8	0.636621979814147\\
0.217330723611743	-3.79378008806183	0.636621979814147\\
0.43876407144288	-3.77458422738331	0.636621979814147\\
0.663243211863663	-3.74167187790656	0.636621979814147\\
0.889583814977564	-3.69440666902413	0.636621979814147\\
1.11648536330403	-3.6322803352065	0.636621979814147\\
1.34254747330838	-3.55493548210277	0.636621979814147\\
1.56629117024328	-3.46218601031457	0.636621979814147\\
1.78618475239903	-3.35403399361086	0.636621979814147\\
2.00067355508028	-3.23068186703712	0.636621979814147\\
2.20821260823669	-3.09253893699409	0.636621979814147\\
2.40730090720063	-2.94022147842488	0.636621979814147\\
2.59651581560615	-2.77454602039814	0.636621979814147\\
2.77454602039814	-2.59651581560615	0.636621979814147\\
2.94022147842488	-2.40730090720064	0.636621979814147\\
3.09253893699409	-2.20821260823669	0.636621979814147\\
3.23068186703712	-2.00067355508028	0.636621979814147\\
3.35403399361086	-1.78618475239903	0.636621979814147\\
3.46218601031457	-1.56629117024328	0.636621979814147\\
3.55493548210277	-1.34254747330838	0.636621979814147\\
3.6322803352065	-1.11648536330404	0.636621979814147\\
3.69440666902413	-0.889583814977566	0.636621979814147\\
3.74167187790656	-0.663243211863665	0.636621979814147\\
3.77458422738331	-0.438764071442882	0.636621979814147\\
3.79378008806183	-0.217330723611745	0.636621979814147\\
3.8	-1.54181817422911e-15	0.636621979814147\\
3.82	0	0.636621979814147\\
3.8137473516832	0.21847456952549	0.636621979814147\\
3.79445046015901	0.441073356029422	0.636621979814147\\
3.76136488779027	0.666733965610315	0.636621979814147\\
3.71385091465057	0.894265835056394	0.636621979814147\\
3.65139760012864	1.12236160205827	0.636621979814147\\
3.57364566885068	1.34961351264158	0.636621979814147\\
3.48040804194781	1.57453480798141	0.636621979814147\\
3.37168680410355	1.79558572478008	0.636621979814147\\
3.247685455811	2.01120341589649	0.636621979814147\\
3.10881545771511	2.21983477985899	0.636621979814147\\
2.95569632831133	2.41997091197538	0.636621979814147\\
2.78914889418971	2.61018168831986	0.636621979814147\\
2.61018168831987	2.78914889418971	0.636621979814147\\
2.41997091197538	2.95569632831133	0.636621979814147\\
2.21983477985899	3.10881545771511	0.636621979814147\\
2.01120341589649	3.247685455811	0.636621979814147\\
1.79558572478008	3.37168680410355	0.636621979814147\\
1.57453480798141	3.48040804194781	0.636621979814147\\
1.34961351264158	3.57364566885068	0.636621979814147\\
1.12236160205827	3.65139760012864	0.636621979814147\\
0.894265835056395	3.71385091465057	0.636621979814147\\
0.666733965610315	3.76136488779027	0.636621979814147\\
0.441073356029422	3.79445046015901	0.636621979814147\\
0.218474569525491	3.8137473516832	0.636621979814147\\
8.4821039081362e-16	3.82	0.636621979814147\\
-0.218474569525489	3.8137473516832	0.636621979814147\\
-0.441073356029422	3.79445046015901	0.636621979814147\\
-0.666733965610314	3.76136488779027	0.636621979814147\\
-0.894265835056392	3.71385091465057	0.636621979814147\\
-1.12236160205827	3.65139760012864	0.636621979814147\\
-1.34961351264158	3.57364566885068	0.636621979814147\\
-1.57453480798141	3.48040804194781	0.636621979814147\\
-1.79558572478008	3.37168680410355	0.636621979814147\\
-2.01120341589649	3.247685455811	0.636621979814147\\
-2.21983477985899	3.10881545771511	0.636621979814147\\
-2.41997091197537	2.95569632831133	0.636621979814147\\
-2.61018168831986	2.78914889418971	0.636621979814147\\
-2.78914889418971	2.61018168831987	0.636621979814147\\
-2.95569632831133	2.41997091197538	0.636621979814147\\
-3.10881545771511	2.21983477985899	0.636621979814147\\
-3.247685455811	2.01120341589649	0.636621979814147\\
-3.37168680410355	1.79558572478008	0.636621979814147\\
-3.48040804194781	1.57453480798141	0.636621979814147\\
-3.57364566885068	1.34961351264158	0.636621979814147\\
-3.65139760012864	1.12236160205827	0.636621979814147\\
-3.71385091465057	0.894265835056394	0.636621979814147\\
-3.76136488779027	0.666733965610316	0.636621979814147\\
-3.79445046015901	0.441073356029423	0.636621979814147\\
-3.8137473516832	0.218474569525491	0.636621979814147\\
-3.82	1.08211792945076e-15	0.636621979814147\\
-3.8137473516832	-0.218474569525489	0.636621979814147\\
-3.79445046015901	-0.441073356029421	0.636621979814147\\
-3.76136488779027	-0.666733965610314	0.636621979814147\\
-3.71385091465057	-0.894265835056393	0.636621979814147\\
-3.65139760012864	-1.12236160205827	0.636621979814147\\
-3.57364566885068	-1.34961351264158	0.636621979814147\\
-3.48040804194781	-1.5745348079814	0.636621979814147\\
-3.37168680410355	-1.79558572478008	0.636621979814147\\
-3.247685455811	-2.01120341589649	0.636621979814147\\
-3.10881545771511	-2.21983477985899	0.636621979814147\\
-2.95569632831133	-2.41997091197537	0.636621979814147\\
-2.78914889418971	-2.61018168831986	0.636621979814147\\
-2.61018168831987	-2.78914889418971	0.636621979814147\\
-2.41997091197538	-2.95569632831133	0.636621979814147\\
-2.21983477985899	-3.10881545771511	0.636621979814147\\
-2.01120341589649	-3.247685455811	0.636621979814147\\
-1.79558572478008	-3.37168680410355	0.636621979814147\\
-1.57453480798141	-3.48040804194781	0.636621979814147\\
-1.34961351264158	-3.57364566885068	0.636621979814147\\
-1.12236160205827	-3.65139760012864	0.636621979814147\\
-0.894265835056395	-3.71385091465057	0.636621979814147\\
-0.666733965610316	-3.76136488779027	0.636621979814147\\
-0.441073356029424	-3.79445046015901	0.636621979814147\\
-0.218474569525491	-3.8137473516832	0.636621979814147\\
-1.31602546808791e-15	-3.82	0.636621979814147\\
0.218474569525489	-3.8137473516832	0.636621979814147\\
0.441073356029421	-3.79445046015901	0.636621979814147\\
0.666733965610314	-3.76136488779027	0.636621979814147\\
0.894265835056393	-3.71385091465057	0.636621979814147\\
1.12236160205827	-3.65139760012864	0.636621979814147\\
1.34961351264158	-3.57364566885068	0.636621979814147\\
1.5745348079814	-3.48040804194781	0.636621979814147\\
1.79558572478008	-3.37168680410355	0.636621979814147\\
2.01120341589649	-3.247685455811	0.636621979814147\\
2.21983477985899	-3.10881545771511	0.636621979814147\\
2.41997091197537	-2.95569632831133	0.636621979814147\\
2.61018168831986	-2.78914889418971	0.636621979814147\\
2.78914889418971	-2.61018168831987	0.636621979814147\\
2.95569632831133	-2.41997091197538	0.636621979814147\\
3.10881545771511	-2.21983477985899	0.636621979814147\\
3.247685455811	-2.01120341589649	0.636621979814147\\
3.37168680410355	-1.79558572478008	0.636621979814147\\
3.48040804194781	-1.57453480798141	0.636621979814147\\
3.57364566885068	-1.34961351264158	0.636621979814147\\
3.65139760012864	-1.12236160205827	0.636621979814147\\
3.71385091465057	-0.894265835056396	0.636621979814147\\
3.76136488779027	-0.666733965610316	0.636621979814147\\
3.79445046015901	-0.441073356029424	0.636621979814147\\
3.8137473516832	-0.218474569525491	0.636621979814147\\
3.82	-1.54993300672505e-15	0.636621979814147\\
3.84	0	0.636621979814147\\
3.83371461530458	0.219618415439236	0.636621979814147\\
3.81431669293471	0.443382640615963	0.636621979814147\\
3.78105789767399	0.670224719356965	0.636621979814147\\
3.73329516027702	0.898947855135223	0.636621979814147\\
3.67051486505078	1.1282378408125	0.636621979814147\\
3.59235585559859	1.35667955197478	0.636621979814147\\
3.49863007358104	1.58277844571953	0.636621979814147\\
3.38933961459624	1.80498669716113	0.636621979814147\\
3.26468904458488	2.0217332767127	0.636621979814147\\
3.12509197843613	2.23145695148129	0.636621979814147\\
2.97117117819778	2.43264091675012	0.636621979814147\\
2.80375176798128	2.62384756103358	0.636621979814147\\
2.62384756103358	2.80375176798128	0.636621979814147\\
2.43264091675012	2.97117117819778	0.636621979814147\\
2.23145695148129	3.12509197843613	0.636621979814147\\
2.0217332767127	3.26468904458488	0.636621979814147\\
1.80498669716113	3.38933961459624	0.636621979814147\\
1.58277844571953	3.49863007358104	0.636621979814147\\
1.35667955197478	3.59235585559859	0.636621979814147\\
1.1282378408125	3.67051486505078	0.636621979814147\\
0.898947855135224	3.73329516027702	0.636621979814147\\
0.670224719356966	3.78105789767399	0.636621979814147\\
0.443382640615964	3.81431669293471	0.636621979814147\\
0.219618415439237	3.83371461530458	0.636621979814147\\
8.5265128291212e-16	3.84	0.636621979814147\\
-0.219618415439235	3.83371461530458	0.636621979814147\\
-0.443382640615963	3.81431669293471	0.636621979814147\\
-0.670224719356965	3.78105789767399	0.636621979814147\\
-0.898947855135222	3.73329516027702	0.636621979814147\\
-1.1282378408125	3.67051486505078	0.636621979814147\\
-1.35667955197478	3.59235585559859	0.636621979814147\\
-1.58277844571953	3.49863007358104	0.636621979814147\\
-1.80498669716113	3.38933961459624	0.636621979814147\\
-2.0217332767127	3.26468904458488	0.636621979814147\\
-2.23145695148129	3.12509197843613	0.636621979814147\\
-2.43264091675011	2.97117117819778	0.636621979814147\\
-2.62384756103358	2.80375176798128	0.636621979814147\\
-2.80375176798128	2.62384756103358	0.636621979814147\\
-2.97117117819778	2.43264091675012	0.636621979814147\\
-3.12509197843613	2.23145695148129	0.636621979814147\\
-3.26468904458488	2.0217332767127	0.636621979814147\\
-3.38933961459624	1.80498669716113	0.636621979814147\\
-3.49863007358104	1.58277844571953	0.636621979814147\\
-3.59235585559859	1.35667955197478	0.636621979814147\\
-3.67051486505078	1.1282378408125	0.636621979814147\\
-3.73329516027702	0.898947855135223	0.636621979814147\\
-3.78105789767399	0.670224719356967	0.636621979814147\\
-3.81431669293471	0.443382640615965	0.636621979814147\\
-3.83371461530458	0.219618415439237	0.636621979814147\\
-3.84	1.08778346834841e-15	0.636621979814147\\
-3.83371461530458	-0.219618415439235	0.636621979814147\\
-3.81431669293471	-0.443382640615963	0.636621979814147\\
-3.78105789767399	-0.670224719356965	0.636621979814147\\
-3.73329516027701	-0.898947855135223	0.636621979814147\\
-3.67051486505078	-1.1282378408125	0.636621979814147\\
-3.59235585559859	-1.35667955197478	0.636621979814147\\
-3.49863007358104	-1.58277844571953	0.636621979814147\\
-3.38933961459624	-1.80498669716113	0.636621979814147\\
-3.26468904458488	-2.0217332767127	0.636621979814147\\
-3.12509197843613	-2.23145695148129	0.636621979814147\\
-2.97117117819778	-2.43264091675011	0.636621979814147\\
-2.80375176798128	-2.62384756103358	0.636621979814147\\
-2.62384756103358	-2.80375176798128	0.636621979814147\\
-2.43264091675012	-2.97117117819778	0.636621979814147\\
-2.23145695148129	-3.12509197843613	0.636621979814147\\
-2.0217332767127	-3.26468904458488	0.636621979814147\\
-1.80498669716113	-3.38933961459624	0.636621979814147\\
-1.58277844571953	-3.49863007358104	0.636621979814147\\
-1.35667955197479	-3.59235585559859	0.636621979814147\\
-1.1282378408125	-3.67051486505078	0.636621979814147\\
-0.898947855135225	-3.73329516027701	0.636621979814147\\
-0.670224719356967	-3.78105789767399	0.636621979814147\\
-0.443382640615965	-3.81431669293471	0.636621979814147\\
-0.219618415439237	-3.83371461530458	0.636621979814147\\
-1.3229156537847e-15	-3.84	0.636621979814147\\
0.219618415439235	-3.83371461530458	0.636621979814147\\
0.443382640615963	-3.81431669293471	0.636621979814147\\
0.670224719356965	-3.78105789767399	0.636621979814147\\
0.898947855135223	-3.73329516027701	0.636621979814147\\
1.1282378408125	-3.67051486505078	0.636621979814147\\
1.35667955197478	-3.59235585559859	0.636621979814147\\
1.58277844571953	-3.49863007358104	0.636621979814147\\
1.80498669716113	-3.38933961459624	0.636621979814147\\
2.0217332767127	-3.26468904458488	0.636621979814147\\
2.23145695148129	-3.12509197843613	0.636621979814147\\
2.43264091675011	-2.97117117819778	0.636621979814147\\
2.62384756103358	-2.80375176798128	0.636621979814147\\
2.80375176798128	-2.62384756103358	0.636621979814147\\
2.97117117819778	-2.43264091675012	0.636621979814147\\
3.12509197843613	-2.23145695148129	0.636621979814147\\
3.26468904458488	-2.0217332767127	0.636621979814147\\
3.38933961459624	-1.80498669716113	0.636621979814147\\
3.49863007358104	-1.58277844571953	0.636621979814147\\
3.59235585559859	-1.35667955197479	0.636621979814147\\
3.67051486505078	-1.1282378408125	0.636621979814147\\
3.73329516027701	-0.898947855135225	0.636621979814147\\
3.78105789767399	-0.670224719356967	0.636621979814147\\
3.81431669293471	-0.443382640615965	0.636621979814147\\
3.83371461530458	-0.219618415439238	0.636621979814147\\
3.84	-1.558047839221e-15	0.636621979814147\\
3.86	0	0.636621979814147\\
3.85368187892596	0.220762261352982	0.636621979814147\\
3.83418292571042	0.445691925202505	0.636621979814147\\
3.80075090755771	0.673715473103616	0.636621979814147\\
3.75273940590346	0.903629875214052	0.636621979814147\\
3.68963212997292	1.13411407956673	0.636621979814147\\
3.6110660423465	1.36374559130799	0.636621979814147\\
3.51685210521427	1.59102208345765	0.636621979814147\\
3.40699242508892	1.81438766954218	0.636621979814147\\
3.28169263335876	2.03226313752891	0.636621979814147\\
3.14136849915715	2.24307912310359	0.636621979814147\\
2.98664602808422	2.44531092152486	0.636621979814147\\
2.81835464177285	2.6375134337473	0.636621979814147\\
2.6375134337473	2.81835464177285	0.636621979814147\\
2.44531092152486	2.98664602808422	0.636621979814147\\
2.24307912310359	3.14136849915715	0.636621979814147\\
2.03226313752891	3.28169263335876	0.636621979814147\\
1.81438766954218	3.40699242508893	0.636621979814147\\
1.59102208345765	3.51685210521427	0.636621979814147\\
1.36374559130799	3.6110660423465	0.636621979814147\\
1.13411407956673	3.68963212997292	0.636621979814147\\
0.903629875214053	3.75273940590346	0.636621979814147\\
0.673715473103617	3.80075090755771	0.636621979814147\\
0.445691925202505	3.83418292571042	0.636621979814147\\
0.220762261352983	3.85368187892596	0.636621979814147\\
8.57092175010621e-16	3.86	0.636621979814147\\
-0.220762261352981	3.85368187892596	0.636621979814147\\
-0.445691925202505	3.83418292571042	0.636621979814147\\
-0.673715473103616	3.80075090755771	0.636621979814147\\
-0.903629875214051	3.75273940590346	0.636621979814147\\
-1.13411407956673	3.68963212997292	0.636621979814147\\
-1.36374559130798	3.6110660423465	0.636621979814147\\
-1.59102208345765	3.51685210521427	0.636621979814147\\
-1.81438766954218	3.40699242508892	0.636621979814147\\
-2.03226313752891	3.28169263335876	0.636621979814147\\
-2.24307912310359	3.14136849915715	0.636621979814147\\
-2.44531092152485	2.98664602808422	0.636621979814147\\
-2.6375134337473	2.81835464177285	0.636621979814147\\
-2.81835464177284	2.6375134337473	0.636621979814147\\
-2.98664602808422	2.44531092152486	0.636621979814147\\
-3.14136849915715	2.24307912310359	0.636621979814147\\
-3.28169263335876	2.03226313752891	0.636621979814147\\
-3.40699242508892	1.81438766954218	0.636621979814147\\
-3.51685210521427	1.59102208345765	0.636621979814147\\
-3.6110660423465	1.36374559130799	0.636621979814147\\
-3.68963212997292	1.13411407956673	0.636621979814147\\
-3.75273940590346	0.903629875214053	0.636621979814147\\
-3.80075090755771	0.673715473103618	0.636621979814147\\
-3.83418292571042	0.445691925202506	0.636621979814147\\
-3.85368187892596	0.220762261352983	0.636621979814147\\
-3.86	1.09344900724606e-15	0.636621979814147\\
-3.85368187892596	-0.220762261352981	0.636621979814147\\
-3.83418292571042	-0.445691925202504	0.636621979814147\\
-3.80075090755771	-0.673715473103616	0.636621979814147\\
-3.75273940590346	-0.903629875214052	0.636621979814147\\
-3.68963212997292	-1.13411407956673	0.636621979814147\\
-3.6110660423465	-1.36374559130799	0.636621979814147\\
-3.51685210521428	-1.59102208345765	0.636621979814147\\
-3.40699242508893	-1.81438766954217	0.636621979814147\\
-3.28169263335876	-2.03226313752891	0.636621979814147\\
-3.14136849915715	-2.24307912310359	0.636621979814147\\
-2.98664602808422	-2.44531092152485	0.636621979814147\\
-2.81835464177285	-2.6375134337473	0.636621979814147\\
-2.6375134337473	-2.81835464177284	0.636621979814147\\
-2.44531092152486	-2.98664602808422	0.636621979814147\\
-2.24307912310359	-3.14136849915715	0.636621979814147\\
-2.03226313752891	-3.28169263335876	0.636621979814147\\
-1.81438766954218	-3.40699242508893	0.636621979814147\\
-1.59102208345765	-3.51685210521428	0.636621979814147\\
-1.36374559130799	-3.6110660423465	0.636621979814147\\
-1.13411407956673	-3.68963212997292	0.636621979814147\\
-0.903629875214054	-3.75273940590346	0.636621979814147\\
-0.673715473103618	-3.80075090755771	0.636621979814147\\
-0.445691925202507	-3.83418292571042	0.636621979814147\\
-0.220762261352983	-3.85368187892596	0.636621979814147\\
-1.3298058394815e-15	-3.86	0.636621979814147\\
0.220762261352981	-3.85368187892596	0.636621979814147\\
0.445691925202504	-3.83418292571042	0.636621979814147\\
0.673715473103616	-3.80075090755771	0.636621979814147\\
0.903629875214052	-3.75273940590346	0.636621979814147\\
1.13411407956673	-3.68963212997292	0.636621979814147\\
1.36374559130799	-3.6110660423465	0.636621979814147\\
1.59102208345765	-3.51685210521428	0.636621979814147\\
1.81438766954217	-3.40699242508893	0.636621979814147\\
2.03226313752891	-3.28169263335876	0.636621979814147\\
2.24307912310359	-3.14136849915715	0.636621979814147\\
2.44531092152485	-2.98664602808422	0.636621979814147\\
2.6375134337473	-2.81835464177285	0.636621979814147\\
2.81835464177284	-2.6375134337473	0.636621979814147\\
2.98664602808422	-2.44531092152486	0.636621979814147\\
3.14136849915715	-2.24307912310359	0.636621979814147\\
3.28169263335876	-2.03226313752891	0.636621979814147\\
3.40699242508893	-1.81438766954218	0.636621979814147\\
3.51685210521428	-1.59102208345765	0.636621979814147\\
3.6110660423465	-1.36374559130799	0.636621979814147\\
3.68963212997292	-1.13411407956673	0.636621979814147\\
3.75273940590346	-0.903629875214054	0.636621979814147\\
3.80075090755771	-0.673715473103618	0.636621979814147\\
3.83418292571042	-0.445691925202507	0.636621979814147\\
3.85368187892596	-0.220762261352984	0.636621979814147\\
3.86	-1.56616267171694e-15	0.636621979814147\\
3.88	0	0.636621979814147\\
3.87364914254734	0.221906107266728	0.636621979814147\\
3.85404915848612	0.448001209789046	0.636621979814147\\
3.82044391744143	0.677206226850267	0.636621979814147\\
3.7721836515299	0.908311895292881	0.636621979814147\\
3.70874939489506	1.13999031832096	0.636621979814147\\
3.62977622909441	1.37081163064119	0.636621979814147\\
3.53507413684751	1.59926572119577	0.636621979814147\\
3.42464523558161	1.82378864192322	0.636621979814147\\
3.29869622213264	2.04279299834513	0.636621979814147\\
3.15764501987817	2.25470129472589	0.636621979814147\\
3.00212087797067	2.4579809262996	0.636621979814147\\
2.83295751556441	2.65117930646101	0.636621979814147\\
2.65117930646101	2.83295751556441	0.636621979814147\\
2.4579809262996	3.00212087797067	0.636621979814147\\
2.25470129472589	3.15764501987817	0.636621979814147\\
2.04279299834513	3.29869622213264	0.636621979814147\\
1.82378864192322	3.42464523558161	0.636621979814147\\
1.59926572119577	3.53507413684751	0.636621979814147\\
1.37081163064119	3.62977622909441	0.636621979814147\\
1.13999031832096	3.70874939489506	0.636621979814147\\
0.908311895292882	3.7721836515299	0.636621979814147\\
0.677206226850268	3.82044391744143	0.636621979814147\\
0.448001209789047	3.85404915848612	0.636621979814147\\
0.221906107266729	3.87364914254734	0.636621979814147\\
8.61533067109121e-16	3.88	0.636621979814147\\
-0.221906107266727	3.87364914254734	0.636621979814147\\
-0.448001209789046	3.85404915848612	0.636621979814147\\
-0.677206226850267	3.82044391744143	0.636621979814147\\
-0.90831189529288	3.7721836515299	0.636621979814147\\
-1.13999031832096	3.70874939489506	0.636621979814147\\
-1.37081163064119	3.62977622909441	0.636621979814147\\
-1.59926572119577	3.53507413684751	0.636621979814147\\
-1.82378864192322	3.42464523558161	0.636621979814147\\
-2.04279299834513	3.29869622213264	0.636621979814147\\
-2.25470129472588	3.15764501987817	0.636621979814147\\
-2.45798092629959	3.00212087797067	0.636621979814147\\
-2.65117930646101	2.83295751556441	0.636621979814147\\
-2.83295751556441	2.65117930646101	0.636621979814147\\
-3.00212087797067	2.4579809262996	0.636621979814147\\
-3.15764501987817	2.25470129472589	0.636621979814147\\
-3.29869622213264	2.04279299834513	0.636621979814147\\
-3.42464523558161	1.82378864192322	0.636621979814147\\
-3.53507413684751	1.59926572119577	0.636621979814147\\
-3.62977622909441	1.37081163064119	0.636621979814147\\
-3.70874939489506	1.13999031832096	0.636621979814147\\
-3.7721836515299	0.908311895292882	0.636621979814147\\
-3.82044391744143	0.677206226850269	0.636621979814147\\
-3.85404915848612	0.448001209789048	0.636621979814147\\
-3.87364914254734	0.221906107266729	0.636621979814147\\
-3.88	1.09911454614371e-15	0.636621979814147\\
-3.87364914254734	-0.221906107266727	0.636621979814147\\
-3.85404915848612	-0.448001209789046	0.636621979814147\\
-3.82044391744143	-0.677206226850267	0.636621979814147\\
-3.7721836515299	-0.908311895292881	0.636621979814147\\
-3.70874939489506	-1.13999031832096	0.636621979814147\\
-3.62977622909441	-1.37081163064119	0.636621979814147\\
-3.53507413684751	-1.59926572119577	0.636621979814147\\
-3.42464523558161	-1.82378864192322	0.636621979814147\\
-3.29869622213264	-2.04279299834512	0.636621979814147\\
-3.15764501987817	-2.25470129472588	0.636621979814147\\
-3.00212087797067	-2.45798092629959	0.636621979814147\\
-2.83295751556441	-2.65117930646101	0.636621979814147\\
-2.65117930646101	-2.83295751556441	0.636621979814147\\
-2.4579809262996	-3.00212087797067	0.636621979814147\\
-2.25470129472589	-3.15764501987817	0.636621979814147\\
-2.04279299834513	-3.29869622213264	0.636621979814147\\
-1.82378864192322	-3.42464523558161	0.636621979814147\\
-1.59926572119577	-3.53507413684751	0.636621979814147\\
-1.37081163064119	-3.62977622909441	0.636621979814147\\
-1.13999031832096	-3.70874939489506	0.636621979814147\\
-0.908311895292883	-3.7721836515299	0.636621979814147\\
-0.677206226850269	-3.82044391744143	0.636621979814147\\
-0.448001209789048	-3.85404915848612	0.636621979814147\\
-0.221906107266729	-3.87364914254734	0.636621979814147\\
-1.33669602517829e-15	-3.88	0.636621979814147\\
0.221906107266727	-3.87364914254734	0.636621979814147\\
0.448001209789046	-3.85404915848612	0.636621979814147\\
0.677206226850267	-3.82044391744143	0.636621979814147\\
0.908311895292881	-3.7721836515299	0.636621979814147\\
1.13999031832096	-3.70874939489506	0.636621979814147\\
1.37081163064119	-3.62977622909441	0.636621979814147\\
1.59926572119577	-3.53507413684751	0.636621979814147\\
1.82378864192322	-3.42464523558161	0.636621979814147\\
2.04279299834512	-3.29869622213264	0.636621979814147\\
2.25470129472588	-3.15764501987817	0.636621979814147\\
2.45798092629959	-3.00212087797067	0.636621979814147\\
2.65117930646101	-2.83295751556441	0.636621979814147\\
2.83295751556441	-2.65117930646101	0.636621979814147\\
3.00212087797067	-2.4579809262996	0.636621979814147\\
3.15764501987817	-2.25470129472589	0.636621979814147\\
3.29869622213264	-2.04279299834513	0.636621979814147\\
3.42464523558161	-1.82378864192322	0.636621979814147\\
3.53507413684751	-1.59926572119577	0.636621979814147\\
3.62977622909441	-1.37081163064119	0.636621979814147\\
3.70874939489506	-1.13999031832096	0.636621979814147\\
3.7721836515299	-0.908311895292883	0.636621979814147\\
3.82044391744143	-0.677206226850269	0.636621979814147\\
3.85404915848612	-0.448001209789048	0.636621979814147\\
3.87364914254734	-0.22190610726673	0.636621979814147\\
3.88	-1.57427750421288e-15	0.636621979814147\\
3.9	0	0.636621979814147\\
3.89361640616872	0.223049953180474	0.636621979814147\\
3.87391539126182	0.450310494375588	0.636621979814147\\
3.84013692732515	0.680696980596918	0.636621979814147\\
3.79162789715634	0.912993915371711	0.636621979814147\\
3.7278666598172	1.14586655707519	0.636621979814147\\
3.64848641584232	1.37787766997439	0.636621979814147\\
3.55329616848074	1.6075093589339	0.636621979814147\\
3.4422980460743	1.83318961430427	0.636621979814147\\
3.31569981090652	2.05332285916134	0.636621979814147\\
3.17392154059919	2.26632346634818	0.636621979814147\\
3.01759572785712	2.47065093107434	0.636621979814147\\
2.84756038935598	2.66484517917473	0.636621979814147\\
2.66484517917473	2.84756038935598	0.636621979814147\\
2.47065093107434	3.01759572785712	0.636621979814147\\
2.26632346634818	3.17392154059919	0.636621979814147\\
2.05332285916134	3.31569981090652	0.636621979814147\\
1.83318961430427	3.4422980460743	0.636621979814147\\
1.6075093589339	3.55329616848074	0.636621979814147\\
1.37787766997439	3.64848641584232	0.636621979814147\\
1.14586655707519	3.7278666598172	0.636621979814147\\
0.912993915371712	3.79162789715634	0.636621979814147\\
0.680696980596919	3.84013692732515	0.636621979814147\\
0.450310494375588	3.87391539126182	0.636621979814147\\
0.223049953180475	3.89361640616872	0.636621979814147\\
8.65973959207622e-16	3.9	0.636621979814147\\
-0.223049953180473	3.89361640616872	0.636621979814147\\
-0.450310494375588	3.87391539126182	0.636621979814147\\
-0.680696980596918	3.84013692732515	0.636621979814147\\
-0.91299391537171	3.79162789715634	0.636621979814147\\
-1.14586655707519	3.7278666598172	0.636621979814147\\
-1.37787766997439	3.64848641584232	0.636621979814147\\
-1.6075093589339	3.55329616848074	0.636621979814147\\
-1.83318961430427	3.4422980460743	0.636621979814147\\
-2.05332285916134	3.31569981090652	0.636621979814147\\
-2.26632346634818	3.17392154059919	0.636621979814147\\
-2.47065093107434	3.01759572785712	0.636621979814147\\
-2.66484517917473	2.84756038935598	0.636621979814147\\
-2.84756038935598	2.66484517917473	0.636621979814147\\
-3.01759572785712	2.47065093107434	0.636621979814147\\
-3.17392154059919	2.26632346634818	0.636621979814147\\
-3.31569981090652	2.05332285916134	0.636621979814147\\
-3.4422980460743	1.83318961430427	0.636621979814147\\
-3.55329616848074	1.6075093589339	0.636621979814147\\
-3.64848641584232	1.37787766997439	0.636621979814147\\
-3.7278666598172	1.14586655707519	0.636621979814147\\
-3.79162789715634	0.912993915371711	0.636621979814147\\
-3.84013692732515	0.68069698059692	0.636621979814147\\
-3.87391539126182	0.450310494375589	0.636621979814147\\
-3.89361640616872	0.223049953180475	0.636621979814147\\
-3.9	1.10478008504136e-15	0.636621979814147\\
-3.89361640616872	-0.223049953180473	0.636621979814147\\
-3.87391539126182	-0.450310494375587	0.636621979814147\\
-3.84013692732515	-0.680696980596917	0.636621979814147\\
-3.79162789715634	-0.912993915371711	0.636621979814147\\
-3.7278666598172	-1.14586655707519	0.636621979814147\\
-3.64848641584232	-1.37787766997439	0.636621979814147\\
-3.55329616848074	-1.60750935893389	0.636621979814147\\
-3.4422980460743	-1.83318961430427	0.636621979814147\\
-3.31569981090652	-2.05332285916134	0.636621979814147\\
-3.17392154059919	-2.26632346634818	0.636621979814147\\
-3.01759572785712	-2.47065093107434	0.636621979814147\\
-2.84756038935598	-2.66484517917473	0.636621979814147\\
-2.66484517917473	-2.84756038935598	0.636621979814147\\
-2.47065093107434	-3.01759572785712	0.636621979814147\\
-2.26632346634818	-3.17392154059919	0.636621979814147\\
-2.05332285916134	-3.31569981090652	0.636621979814147\\
-1.83318961430427	-3.4422980460743	0.636621979814147\\
-1.6075093589339	-3.55329616848074	0.636621979814147\\
-1.37787766997439	-3.64848641584232	0.636621979814147\\
-1.14586655707519	-3.7278666598172	0.636621979814147\\
-0.912993915371712	-3.79162789715634	0.636621979814147\\
-0.68069698059692	-3.84013692732515	0.636621979814147\\
-0.45031049437559	-3.87391539126182	0.636621979814147\\
-0.223049953180475	-3.89361640616872	0.636621979814147\\
-1.34358621087509e-15	-3.9	0.636621979814147\\
0.223049953180473	-3.89361640616872	0.636621979814147\\
0.450310494375587	-3.87391539126182	0.636621979814147\\
0.680696980596917	-3.84013692732515	0.636621979814147\\
0.91299391537171	-3.79162789715634	0.636621979814147\\
1.14586655707519	-3.7278666598172	0.636621979814147\\
1.37787766997439	-3.64848641584232	0.636621979814147\\
1.60750935893389	-3.55329616848074	0.636621979814147\\
1.83318961430427	-3.4422980460743	0.636621979814147\\
2.05332285916134	-3.31569981090652	0.636621979814147\\
2.26632346634818	-3.17392154059919	0.636621979814147\\
2.47065093107434	-3.01759572785712	0.636621979814147\\
2.66484517917473	-2.84756038935598	0.636621979814147\\
2.84756038935598	-2.66484517917473	0.636621979814147\\
3.01759572785712	-2.47065093107434	0.636621979814147\\
3.17392154059919	-2.26632346634818	0.636621979814147\\
3.31569981090652	-2.05332285916134	0.636621979814147\\
3.4422980460743	-1.83318961430427	0.636621979814147\\
3.55329616848074	-1.6075093589339	0.636621979814147\\
3.64848641584232	-1.37787766997439	0.636621979814147\\
3.7278666598172	-1.1458665570752	0.636621979814147\\
3.79162789715634	-0.912993915371713	0.636621979814147\\
3.84013692732515	-0.68069698059692	0.636621979814147\\
3.87391539126182	-0.45031049437559	0.636621979814147\\
3.89361640616872	-0.223049953180476	0.636621979814147\\
3.9	-1.58239233670882e-15	0.636621979814147\\
3.92	0	0.636621979814147\\
3.91358366979009	0.22419379909422	0.636621979814147\\
3.89378162403752	0.452619778962129	0.636621979814147\\
3.85982993720887	0.684187734343569	0.636621979814147\\
3.81107214278279	0.91767593545054	0.636621979814147\\
3.74698392473934	1.15174279582942	0.636621979814147\\
3.66719660259023	1.38494370930759	0.636621979814147\\
3.57151820011398	1.61575299667202	0.636621979814147\\
3.45995085656699	1.84259058668532	0.636621979814147\\
3.3327033996804	2.06385271997755	0.636621979814147\\
3.19019806132021	2.27794563797048	0.636621979814147\\
3.03307057774356	2.48332093584908	0.636621979814147\\
2.86216326314755	2.67851105188845	0.636621979814147\\
2.67851105188845	2.86216326314755	0.636621979814147\\
2.48332093584908	3.03307057774356	0.636621979814147\\
2.27794563797048	3.19019806132021	0.636621979814147\\
2.06385271997755	3.3327033996804	0.636621979814147\\
1.84259058668532	3.45995085656699	0.636621979814147\\
1.61575299667202	3.57151820011398	0.636621979814147\\
1.38494370930759	3.66719660259023	0.636621979814147\\
1.15174279582943	3.74698392473934	0.636621979814147\\
0.917675935450541	3.81107214278279	0.636621979814147\\
0.68418773434357	3.85982993720887	0.636621979814147\\
0.45261977896213	3.89378162403752	0.636621979814147\\
0.224193799094221	3.91358366979009	0.636621979814147\\
8.70414851306123e-16	3.92	0.636621979814147\\
-0.224193799094219	3.91358366979009	0.636621979814147\\
-0.452619778962129	3.89378162403752	0.636621979814147\\
-0.684187734343569	3.85982993720887	0.636621979814147\\
-0.917675935450539	3.81107214278279	0.636621979814147\\
-1.15174279582942	3.74698392473934	0.636621979814147\\
-1.38494370930759	3.66719660259023	0.636621979814147\\
-1.61575299667202	3.57151820011398	0.636621979814147\\
-1.84259058668532	3.45995085656699	0.636621979814147\\
-2.06385271997755	3.3327033996804	0.636621979814147\\
-2.27794563797048	3.19019806132021	0.636621979814147\\
-2.48332093584908	3.03307057774356	0.636621979814147\\
-2.67851105188845	2.86216326314755	0.636621979814147\\
-2.86216326314755	2.67851105188845	0.636621979814147\\
-3.03307057774356	2.48332093584908	0.636621979814147\\
-3.19019806132021	2.27794563797048	0.636621979814147\\
-3.3327033996804	2.06385271997755	0.636621979814147\\
-3.45995085656699	1.84259058668532	0.636621979814147\\
-3.57151820011398	1.61575299667202	0.636621979814147\\
-3.66719660259023	1.38494370930759	0.636621979814147\\
-3.74698392473934	1.15174279582943	0.636621979814147\\
-3.81107214278279	0.91767593545054	0.636621979814147\\
-3.85982993720887	0.68418773434357	0.636621979814147\\
-3.89378162403752	0.452619778962131	0.636621979814147\\
-3.91358366979009	0.224193799094221	0.636621979814147\\
-3.92	1.110445623939e-15	0.636621979814147\\
-3.91358366979009	-0.224193799094219	0.636621979814147\\
-3.89378162403752	-0.452619778962129	0.636621979814147\\
-3.85982993720887	-0.684187734343568	0.636621979814147\\
-3.81107214278279	-0.91767593545054	0.636621979814147\\
-3.74698392473934	-1.15174279582942	0.636621979814147\\
-3.66719660259023	-1.38494370930759	0.636621979814147\\
-3.57151820011398	-1.61575299667202	0.636621979814147\\
-3.45995085656699	-1.84259058668532	0.636621979814147\\
-3.3327033996804	-2.06385271997755	0.636621979814147\\
-3.19019806132021	-2.27794563797048	0.636621979814147\\
-3.03307057774356	-2.48332093584908	0.636621979814147\\
-2.86216326314755	-2.67851105188845	0.636621979814147\\
-2.67851105188845	-2.86216326314755	0.636621979814147\\
-2.48332093584908	-3.03307057774356	0.636621979814147\\
-2.27794563797048	-3.19019806132021	0.636621979814147\\
-2.06385271997755	-3.3327033996804	0.636621979814147\\
-1.84259058668532	-3.45995085656699	0.636621979814147\\
-1.61575299667202	-3.57151820011398	0.636621979814147\\
-1.38494370930759	-3.66719660259023	0.636621979814147\\
-1.15174279582943	-3.74698392473934	0.636621979814147\\
-0.917675935450542	-3.81107214278279	0.636621979814147\\
-0.684187734343571	-3.85982993720887	0.636621979814147\\
-0.452619778962131	-3.89378162403752	0.636621979814147\\
-0.224193799094221	-3.91358366979009	0.636621979814147\\
-1.35047639657189e-15	-3.92	0.636621979814147\\
0.224193799094219	-3.91358366979009	0.636621979814147\\
0.452619778962129	-3.89378162403752	0.636621979814147\\
0.684187734343568	-3.85982993720887	0.636621979814147\\
0.91767593545054	-3.81107214278279	0.636621979814147\\
1.15174279582942	-3.74698392473934	0.636621979814147\\
1.38494370930759	-3.66719660259023	0.636621979814147\\
1.61575299667202	-3.57151820011398	0.636621979814147\\
1.84259058668532	-3.45995085656699	0.636621979814147\\
2.06385271997755	-3.3327033996804	0.636621979814147\\
2.27794563797048	-3.19019806132022	0.636621979814147\\
2.48332093584908	-3.03307057774356	0.636621979814147\\
2.67851105188845	-2.86216326314755	0.636621979814147\\
2.86216326314755	-2.67851105188845	0.636621979814147\\
3.03307057774356	-2.48332093584908	0.636621979814147\\
3.19019806132021	-2.27794563797048	0.636621979814147\\
3.3327033996804	-2.06385271997755	0.636621979814147\\
3.45995085656699	-1.84259058668532	0.636621979814147\\
3.57151820011398	-1.61575299667202	0.636621979814147\\
3.66719660259023	-1.38494370930759	0.636621979814147\\
3.74698392473934	-1.15174279582943	0.636621979814147\\
3.81107214278279	-0.917675935450542	0.636621979814147\\
3.85982993720887	-0.684187734343571	0.636621979814147\\
3.89378162403752	-0.452619778962131	0.636621979814147\\
3.91358366979009	-0.224193799094222	0.636621979814147\\
3.92	-1.59050716920477e-15	0.636621979814147\\
3.94	0	0.636621979814147\\
3.93355093341147	0.225337645007966	0.636621979814147\\
3.91364785681322	0.454929063548671	0.636621979814147\\
3.87952294709259	0.68767848809022	0.636621979814147\\
3.83051638840923	0.922357955529369	0.636621979814147\\
3.76610118966148	1.15761903458366	0.636621979814147\\
3.68590678933814	1.39200974864079	0.636621979814147\\
3.58974023174721	1.62399663441014	0.636621979814147\\
3.47760366705968	1.85199155906637	0.636621979814147\\
3.34970698845428	2.07438258079376	0.636621979814147\\
3.20647458204124	2.28956780959278	0.636621979814147\\
3.04854542763001	2.49599094062382	0.636621979814147\\
2.87676613693912	2.69217692460216	0.636621979814147\\
2.69217692460216	2.87676613693912	0.636621979814147\\
2.49599094062382	3.04854542763001	0.636621979814147\\
2.28956780959278	3.20647458204124	0.636621979814147\\
2.07438258079376	3.34970698845428	0.636621979814147\\
1.85199155906637	3.47760366705968	0.636621979814147\\
1.62399663441014	3.58974023174721	0.636621979814147\\
1.39200974864079	3.68590678933814	0.636621979814147\\
1.15761903458366	3.76610118966148	0.636621979814147\\
0.92235795552937	3.83051638840923	0.636621979814147\\
0.68767848809022	3.87952294709259	0.636621979814147\\
0.454929063548671	3.91364785681322	0.636621979814147\\
0.225337645007967	3.93355093341147	0.636621979814147\\
8.74855743404623e-16	3.94	0.636621979814147\\
-0.225337645007965	3.93355093341147	0.636621979814147\\
-0.45492906354867	3.91364785681322	0.636621979814147\\
-0.687678488090219	3.87952294709259	0.636621979814147\\
-0.922357955529368	3.83051638840923	0.636621979814147\\
-1.15761903458366	3.76610118966148	0.636621979814147\\
-1.39200974864079	3.68590678933814	0.636621979814147\\
-1.62399663441014	3.58974023174721	0.636621979814147\\
-1.85199155906637	3.47760366705968	0.636621979814147\\
-2.07438258079376	3.34970698845428	0.636621979814147\\
-2.28956780959278	3.20647458204124	0.636621979814147\\
-2.49599094062382	3.04854542763001	0.636621979814147\\
-2.69217692460216	2.87676613693912	0.636621979814147\\
-2.87676613693912	2.69217692460216	0.636621979814147\\
-3.04854542763001	2.49599094062382	0.636621979814147\\
-3.20647458204124	2.28956780959278	0.636621979814147\\
-3.34970698845427	2.07438258079376	0.636621979814147\\
-3.47760366705968	1.85199155906637	0.636621979814147\\
-3.58974023174721	1.62399663441014	0.636621979814147\\
-3.68590678933814	1.39200974864079	0.636621979814147\\
-3.76610118966148	1.15761903458366	0.636621979814147\\
-3.83051638840923	0.92235795552937	0.636621979814147\\
-3.87952294709259	0.687678488090221	0.636621979814147\\
-3.91364785681322	0.454929063548672	0.636621979814147\\
-3.93355093341147	0.225337645007967	0.636621979814147\\
-3.94	1.11611116283665e-15	0.636621979814147\\
-3.93355093341147	-0.225337645007965	0.636621979814147\\
-3.91364785681322	-0.45492906354867	0.636621979814147\\
-3.87952294709259	-0.687678488090219	0.636621979814147\\
-3.83051638840923	-0.922357955529369	0.636621979814147\\
-3.76610118966148	-1.15761903458366	0.636621979814147\\
-3.68590678933814	-1.39200974864079	0.636621979814147\\
-3.58974023174721	-1.62399663441014	0.636621979814147\\
-3.47760366705968	-1.85199155906636	0.636621979814147\\
-3.34970698845428	-2.07438258079376	0.636621979814147\\
-3.20647458204124	-2.28956780959278	0.636621979814147\\
-3.04854542763001	-2.49599094062382	0.636621979814147\\
-2.87676613693912	-2.69217692460216	0.636621979814147\\
-2.69217692460216	-2.87676613693912	0.636621979814147\\
-2.49599094062382	-3.04854542763001	0.636621979814147\\
-2.28956780959278	-3.20647458204124	0.636621979814147\\
-2.07438258079376	-3.34970698845428	0.636621979814147\\
-1.85199155906637	-3.47760366705968	0.636621979814147\\
-1.62399663441014	-3.58974023174721	0.636621979814147\\
-1.3920097486408	-3.68590678933814	0.636621979814147\\
-1.15761903458366	-3.76610118966148	0.636621979814147\\
-0.922357955529371	-3.83051638840923	0.636621979814147\\
-0.687678488090221	-3.87952294709259	0.636621979814147\\
-0.454929063548672	-3.91364785681322	0.636621979814147\\
-0.225337645007967	-3.93355093341147	0.636621979814147\\
-1.35736658226868e-15	-3.94	0.636621979814147\\
0.225337645007965	-3.93355093341147	0.636621979814147\\
0.45492906354867	-3.91364785681322	0.636621979814147\\
0.687678488090219	-3.87952294709259	0.636621979814147\\
0.922357955529369	-3.83051638840923	0.636621979814147\\
1.15761903458366	-3.76610118966148	0.636621979814147\\
1.39200974864079	-3.68590678933814	0.636621979814147\\
1.62399663441014	-3.58974023174721	0.636621979814147\\
1.85199155906636	-3.47760366705968	0.636621979814147\\
2.07438258079376	-3.34970698845428	0.636621979814147\\
2.28956780959278	-3.20647458204124	0.636621979814147\\
2.49599094062382	-3.04854542763001	0.636621979814147\\
2.69217692460216	-2.87676613693912	0.636621979814147\\
2.87676613693912	-2.69217692460216	0.636621979814147\\
3.04854542763001	-2.49599094062382	0.636621979814147\\
3.20647458204124	-2.28956780959278	0.636621979814147\\
3.34970698845428	-2.07438258079376	0.636621979814147\\
3.47760366705968	-1.85199155906637	0.636621979814147\\
3.58974023174721	-1.62399663441014	0.636621979814147\\
3.68590678933814	-1.3920097486408	0.636621979814147\\
3.76610118966148	-1.15761903458366	0.636621979814147\\
3.83051638840923	-0.922357955529371	0.636621979814147\\
3.87952294709259	-0.687678488090222	0.636621979814147\\
3.91364785681322	-0.454929063548673	0.636621979814147\\
3.93355093341147	-0.225337645007968	0.636621979814147\\
3.94	-1.59862200170071e-15	0.636621979814147\\
3.96	0	0.636621979814147\\
3.95351819703285	0.226481490921712	0.636621979814147\\
3.93351408958892	0.457238348135212	0.636621979814147\\
3.89921595697631	0.691169241836871	0.636621979814147\\
3.84996063403567	0.927039975608199	0.636621979814147\\
3.78521845458362	1.16349527333789	0.636621979814147\\
3.70461697608605	1.399075787974	0.636621979814147\\
3.60796226338045	1.63224027214826	0.636621979814147\\
3.49525647755237	1.86139253144741	0.636621979814147\\
3.36671057722816	2.08491244160997	0.636621979814147\\
3.22275110276226	2.30118998121508	0.636621979814147\\
3.06402027751646	2.50866094539856	0.636621979814147\\
2.89136901073069	2.70584279731588	0.636621979814147\\
2.70584279731588	2.89136901073069	0.636621979814147\\
2.50866094539856	3.06402027751646	0.636621979814147\\
2.30118998121508	3.22275110276226	0.636621979814147\\
2.08491244160997	3.36671057722816	0.636621979814147\\
1.86139253144741	3.49525647755237	0.636621979814147\\
1.63224027214826	3.60796226338045	0.636621979814147\\
1.399075787974	3.70461697608605	0.636621979814147\\
1.16349527333789	3.78521845458362	0.636621979814147\\
0.9270399756082	3.84996063403567	0.636621979814147\\
0.691169241836871	3.89921595697631	0.636621979814147\\
0.457238348135213	3.93351408958892	0.636621979814147\\
0.226481490921713	3.95351819703285	0.636621979814147\\
8.79296635503124e-16	3.96	0.636621979814147\\
-0.226481490921711	3.95351819703285	0.636621979814147\\
-0.457238348135212	3.93351408958892	0.636621979814147\\
-0.69116924183687	3.89921595697631	0.636621979814147\\
-0.927039975608197	3.84996063403567	0.636621979814147\\
-1.16349527333789	3.78521845458362	0.636621979814147\\
-1.39907578797399	3.70461697608605	0.636621979814147\\
-1.63224027214826	3.60796226338045	0.636621979814147\\
-1.86139253144741	3.49525647755237	0.636621979814147\\
-2.08491244160997	3.36671057722816	0.636621979814147\\
-2.30118998121508	3.22275110276226	0.636621979814147\\
-2.50866094539856	3.06402027751646	0.636621979814147\\
-2.70584279731588	2.89136901073069	0.636621979814147\\
-2.89136901073069	2.70584279731588	0.636621979814147\\
-3.06402027751646	2.50866094539856	0.636621979814147\\
-3.22275110276226	2.30118998121508	0.636621979814147\\
-3.36671057722815	2.08491244160997	0.636621979814147\\
-3.49525647755237	1.86139253144741	0.636621979814147\\
-3.60796226338045	1.63224027214826	0.636621979814147\\
-3.70461697608605	1.399075787974	0.636621979814147\\
-3.78521845458362	1.16349527333789	0.636621979814147\\
-3.84996063403567	0.927039975608199	0.636621979814147\\
-3.89921595697631	0.691169241836872	0.636621979814147\\
-3.93351408958892	0.457238348135214	0.636621979814147\\
-3.95351819703285	0.226481490921713	0.636621979814147\\
-3.96	1.1217767017343e-15	0.636621979814147\\
-3.95351819703285	-0.226481490921711	0.636621979814147\\
-3.93351408958892	-0.457238348135212	0.636621979814147\\
-3.89921595697631	-0.69116924183687	0.636621979814147\\
-3.84996063403567	-0.927039975608199	0.636621979814147\\
-3.78521845458362	-1.16349527333789	0.636621979814147\\
-3.70461697608605	-1.399075787974	0.636621979814147\\
-3.60796226338045	-1.63224027214826	0.636621979814147\\
-3.49525647755237	-1.86139253144741	0.636621979814147\\
-3.36671057722816	-2.08491244160997	0.636621979814147\\
-3.22275110276226	-2.30118998121508	0.636621979814147\\
-3.06402027751646	-2.50866094539856	0.636621979814147\\
-2.89136901073069	-2.70584279731588	0.636621979814147\\
-2.70584279731588	-2.89136901073069	0.636621979814147\\
-2.50866094539856	-3.06402027751646	0.636621979814147\\
-2.30118998121508	-3.22275110276226	0.636621979814147\\
-2.08491244160997	-3.36671057722816	0.636621979814147\\
-1.86139253144741	-3.49525647755237	0.636621979814147\\
-1.63224027214826	-3.60796226338045	0.636621979814147\\
-1.399075787974	-3.70461697608605	0.636621979814147\\
-1.16349527333789	-3.78521845458362	0.636621979814147\\
-0.927039975608201	-3.84996063403567	0.636621979814147\\
-0.691169241836872	-3.89921595697631	0.636621979814147\\
-0.457238348135214	-3.93351408958892	0.636621979814147\\
-0.226481490921713	-3.95351819703285	0.636621979814147\\
-1.36425676796548e-15	-3.96	0.636621979814147\\
0.226481490921711	-3.95351819703285	0.636621979814147\\
0.457238348135212	-3.93351408958892	0.636621979814147\\
0.69116924183687	-3.89921595697631	0.636621979814147\\
0.927039975608198	-3.84996063403567	0.636621979814147\\
1.16349527333789	-3.78521845458362	0.636621979814147\\
1.399075787974	-3.70461697608605	0.636621979814147\\
1.63224027214826	-3.60796226338045	0.636621979814147\\
1.86139253144741	-3.49525647755237	0.636621979814147\\
2.08491244160997	-3.36671057722816	0.636621979814147\\
2.30118998121508	-3.22275110276226	0.636621979814147\\
2.50866094539856	-3.06402027751646	0.636621979814147\\
2.70584279731588	-2.89136901073069	0.636621979814147\\
2.89136901073069	-2.70584279731588	0.636621979814147\\
3.06402027751646	-2.50866094539856	0.636621979814147\\
3.22275110276226	-2.30118998121508	0.636621979814147\\
3.36671057722816	-2.08491244160997	0.636621979814147\\
3.49525647755237	-1.86139253144741	0.636621979814147\\
3.60796226338045	-1.63224027214826	0.636621979814147\\
3.70461697608605	-1.399075787974	0.636621979814147\\
3.78521845458362	-1.16349527333789	0.636621979814147\\
3.84996063403567	-0.927039975608201	0.636621979814147\\
3.89921595697631	-0.691169241836872	0.636621979814147\\
3.93351408958892	-0.457238348135214	0.636621979814147\\
3.95351819703285	-0.226481490921714	0.636621979814147\\
3.96	-1.60673683419665e-15	0.636621979814147\\
3.98	0	0.636621979814147\\
3.97348546065423	0.227625336835458	0.636621979814147\\
3.95338032236463	0.459547632721754	0.636621979814147\\
3.91890896686002	0.694659995583521	0.636621979814147\\
3.86940487966211	0.931721995687028	0.636621979814147\\
3.80433571950576	1.16937151209212	0.636621979814147\\
3.72332716283396	1.4061418273072	0.636621979814147\\
3.62618429501368	1.64048390988639	0.636621979814147\\
3.51290928804506	1.87079350382846	0.636621979814147\\
3.38371416600204	2.09544230242619	0.636621979814147\\
3.23902762348328	2.31281215283738	0.636621979814147\\
3.0794951274029	2.5213309501733	0.636621979814147\\
2.90597188452226	2.7195086700296	0.636621979814147\\
2.7195086700296	2.90597188452226	0.636621979814147\\
2.5213309501733	3.0794951274029	0.636621979814147\\
2.31281215283738	3.23902762348328	0.636621979814147\\
2.09544230242619	3.38371416600203	0.636621979814147\\
1.87079350382846	3.51290928804506	0.636621979814147\\
1.64048390988639	3.62618429501368	0.636621979814147\\
1.4061418273072	3.72332716283396	0.636621979814147\\
1.16937151209212	3.80433571950576	0.636621979814147\\
0.931721995687029	3.86940487966211	0.636621979814147\\
0.694659995583522	3.91890896686002	0.636621979814147\\
0.459547632721754	3.95338032236463	0.636621979814147\\
0.227625336835459	3.97348546065423	0.636621979814147\\
8.83737527601625e-16	3.98	0.636621979814147\\
-0.227625336835457	3.97348546065423	0.636621979814147\\
-0.459547632721753	3.95338032236463	0.636621979814147\\
-0.694659995583521	3.91890896686002	0.636621979814147\\
-0.931721995687027	3.86940487966211	0.636621979814147\\
-1.16937151209212	3.80433571950576	0.636621979814147\\
-1.4061418273072	3.72332716283396	0.636621979814147\\
-1.64048390988639	3.62618429501368	0.636621979814147\\
-1.87079350382846	3.51290928804506	0.636621979814147\\
-2.09544230242618	3.38371416600203	0.636621979814147\\
-2.31281215283738	3.23902762348328	0.636621979814147\\
-2.5213309501733	3.0794951274029	0.636621979814147\\
-2.7195086700296	2.90597188452226	0.636621979814147\\
-2.90597188452226	2.7195086700296	0.636621979814147\\
-3.0794951274029	2.5213309501733	0.636621979814147\\
-3.23902762348328	2.31281215283738	0.636621979814147\\
-3.38371416600203	2.09544230242619	0.636621979814147\\
-3.51290928804506	1.87079350382846	0.636621979814147\\
-3.62618429501368	1.64048390988639	0.636621979814147\\
-3.72332716283396	1.4061418273072	0.636621979814147\\
-3.80433571950576	1.16937151209212	0.636621979814147\\
-3.86940487966211	0.931721995687028	0.636621979814147\\
-3.91890896686002	0.694659995583523	0.636621979814147\\
-3.95338032236463	0.459547632721755	0.636621979814147\\
-3.97348546065423	0.227625336835459	0.636621979814147\\
-3.98	1.12744224063195e-15	0.636621979814147\\
-3.97348546065423	-0.227625336835457	0.636621979814147\\
-3.95338032236463	-0.459547632721753	0.636621979814147\\
-3.91890896686002	-0.694659995583521	0.636621979814147\\
-3.86940487966211	-0.931721995687028	0.636621979814147\\
-3.80433571950576	-1.16937151209212	0.636621979814147\\
-3.72332716283396	-1.4061418273072	0.636621979814147\\
-3.62618429501368	-1.64048390988638	0.636621979814147\\
-3.51290928804506	-1.87079350382846	0.636621979814147\\
-3.38371416600204	-2.09544230242618	0.636621979814147\\
-3.23902762348328	-2.31281215283738	0.636621979814147\\
-3.0794951274029	-2.5213309501733	0.636621979814147\\
-2.90597188452226	-2.7195086700296	0.636621979814147\\
-2.7195086700296	-2.90597188452226	0.636621979814147\\
-2.5213309501733	-3.0794951274029	0.636621979814147\\
-2.31281215283738	-3.23902762348328	0.636621979814147\\
-2.09544230242619	-3.38371416600204	0.636621979814147\\
-1.87079350382846	-3.51290928804506	0.636621979814147\\
-1.64048390988639	-3.62618429501368	0.636621979814147\\
-1.4061418273072	-3.72332716283396	0.636621979814147\\
-1.16937151209212	-3.80433571950576	0.636621979814147\\
-0.93172199568703	-3.86940487966211	0.636621979814147\\
-0.694659995583523	-3.91890896686002	0.636621979814147\\
-0.459547632721755	-3.95338032236463	0.636621979814147\\
-0.227625336835459	-3.97348546065423	0.636621979814147\\
-1.37114695366227e-15	-3.98	0.636621979814147\\
0.227625336835457	-3.97348546065423	0.636621979814147\\
0.459547632721753	-3.95338032236463	0.636621979814147\\
0.694659995583521	-3.91890896686002	0.636621979814147\\
0.931721995687028	-3.86940487966211	0.636621979814147\\
1.16937151209212	-3.80433571950576	0.636621979814147\\
1.4061418273072	-3.72332716283396	0.636621979814147\\
1.64048390988638	-3.62618429501368	0.636621979814147\\
1.87079350382846	-3.51290928804506	0.636621979814147\\
2.09544230242618	-3.38371416600204	0.636621979814147\\
2.31281215283738	-3.23902762348328	0.636621979814147\\
2.5213309501733	-3.0794951274029	0.636621979814147\\
2.7195086700296	-2.90597188452226	0.636621979814147\\
2.90597188452226	-2.7195086700296	0.636621979814147\\
3.0794951274029	-2.5213309501733	0.636621979814147\\
3.23902762348328	-2.31281215283738	0.636621979814147\\
3.38371416600204	-2.09544230242619	0.636621979814147\\
3.51290928804506	-1.87079350382846	0.636621979814147\\
3.62618429501368	-1.64048390988639	0.636621979814147\\
3.72332716283396	-1.4061418273072	0.636621979814147\\
3.80433571950576	-1.16937151209212	0.636621979814147\\
3.86940487966211	-0.93172199568703	0.636621979814147\\
3.91890896686002	-0.694659995583523	0.636621979814147\\
3.95338032236463	-0.459547632721756	0.636621979814147\\
3.97348546065423	-0.22762533683546	0.636621979814147\\
3.98	-1.61485166669259e-15	0.636621979814147\\
4	0	0.636621979814147\\
3.99345272427561	0.228769182749204	0.636621979814147\\
3.97324655514033	0.461856917308295	0.636621979814147\\
3.93860197674374	0.698150749330172	0.636621979814147\\
3.88884912528856	0.936404015765857	0.636621979814147\\
3.8234529844279	1.17524775084635	0.636621979814147\\
3.74203734958187	1.4132078666404	0.636621979814147\\
3.64440632664692	1.64872754762451	0.636621979814147\\
3.53056209853775	1.88019447620951	0.636621979814147\\
3.40071775477591	2.1059721632424	0.636621979814147\\
3.2553041442043	2.32443432445968	0.636621979814147\\
3.09496997728935	2.53400095494804	0.636621979814147\\
2.92057475831383	2.73317454274331	0.636621979814147\\
2.73317454274331	2.92057475831383	0.636621979814147\\
2.53400095494804	3.09496997728935	0.636621979814147\\
2.32443432445968	3.2553041442043	0.636621979814147\\
2.1059721632424	3.40071775477591	0.636621979814147\\
1.88019447620951	3.53056209853775	0.636621979814147\\
1.64872754762451	3.64440632664692	0.636621979814147\\
1.4132078666404	3.74203734958187	0.636621979814147\\
1.17524775084635	3.8234529844279	0.636621979814147\\
0.936404015765858	3.88884912528856	0.636621979814147\\
0.698150749330173	3.93860197674374	0.636621979814147\\
0.461856917308296	3.97324655514033	0.636621979814147\\
0.228769182749205	3.99345272427561	0.636621979814147\\
8.88178419700125e-16	4	0.636621979814147\\
-0.228769182749203	3.99345272427561	0.636621979814147\\
-0.461856917308295	3.97324655514033	0.636621979814147\\
-0.698150749330172	3.93860197674374	0.636621979814147\\
-0.936404015765856	3.88884912528856	0.636621979814147\\
-1.17524775084635	3.8234529844279	0.636621979814147\\
-1.4132078666404	3.74203734958187	0.636621979814147\\
-1.64872754762451	3.64440632664692	0.636621979814147\\
-1.88019447620951	3.53056209853775	0.636621979814147\\
-2.1059721632424	3.40071775477591	0.636621979814147\\
-2.32443432445967	3.2553041442043	0.636621979814147\\
-2.53400095494804	3.09496997728935	0.636621979814147\\
-2.73317454274331	2.92057475831383	0.636621979814147\\
-2.92057475831383	2.73317454274331	0.636621979814147\\
-3.09496997728935	2.53400095494804	0.636621979814147\\
-3.2553041442043	2.32443432445968	0.636621979814147\\
-3.40071775477591	2.1059721632424	0.636621979814147\\
-3.53056209853775	1.88019447620951	0.636621979814147\\
-3.64440632664692	1.64872754762451	0.636621979814147\\
-3.74203734958187	1.4132078666404	0.636621979814147\\
-3.8234529844279	1.17524775084635	0.636621979814147\\
-3.88884912528856	0.936404015765858	0.636621979814147\\
-3.93860197674374	0.698150749330174	0.636621979814147\\
-3.97324655514033	0.461856917308297	0.636621979814147\\
-3.99345272427561	0.228769182749205	0.636621979814147\\
-4	1.1331077795296e-15	0.636621979814147\\
-3.99345272427561	-0.228769182749203	0.636621979814147\\
-3.97324655514033	-0.461856917308295	0.636621979814147\\
-3.93860197674374	-0.698150749330172	0.636621979814147\\
-3.88884912528856	-0.936404015765857	0.636621979814147\\
-3.8234529844279	-1.17524775084635	0.636621979814147\\
-3.74203734958187	-1.4132078666404	0.636621979814147\\
-3.64440632664692	-1.64872754762451	0.636621979814147\\
-3.53056209853775	-1.88019447620951	0.636621979814147\\
-3.40071775477591	-2.1059721632424	0.636621979814147\\
-3.2553041442043	-2.32443432445967	0.636621979814147\\
-3.09496997728935	-2.53400095494804	0.636621979814147\\
-2.92057475831383	-2.73317454274331	0.636621979814147\\
-2.73317454274331	-2.92057475831383	0.636621979814147\\
-2.53400095494804	-3.09496997728935	0.636621979814147\\
-2.32443432445968	-3.2553041442043	0.636621979814147\\
-2.1059721632424	-3.40071775477591	0.636621979814147\\
-1.88019447620951	-3.53056209853775	0.636621979814147\\
-1.64872754762451	-3.64440632664692	0.636621979814147\\
-1.4132078666404	-3.74203734958187	0.636621979814147\\
-1.17524775084635	-3.8234529844279	0.636621979814147\\
-0.936404015765859	-3.88884912528856	0.636621979814147\\
-0.698150749330174	-3.93860197674374	0.636621979814147\\
-0.461856917308297	-3.97324655514033	0.636621979814147\\
-0.228769182749206	-3.99345272427561	0.636621979814147\\
-1.37803713935907e-15	-4	0.636621979814147\\
0.228769182749203	-3.99345272427561	0.636621979814147\\
0.461856917308294	-3.97324655514033	0.636621979814147\\
0.698150749330172	-3.93860197674374	0.636621979814147\\
0.936404015765857	-3.88884912528856	0.636621979814147\\
1.17524775084635	-3.8234529844279	0.636621979814147\\
1.4132078666404	-3.74203734958187	0.636621979814147\\
1.64872754762451	-3.64440632664692	0.636621979814147\\
1.88019447620951	-3.53056209853775	0.636621979814147\\
2.1059721632424	-3.40071775477591	0.636621979814147\\
2.32443432445967	-3.2553041442043	0.636621979814147\\
2.53400095494804	-3.09496997728935	0.636621979814147\\
2.73317454274331	-2.92057475831383	0.636621979814147\\
2.92057475831383	-2.73317454274331	0.636621979814147\\
3.09496997728935	-2.53400095494804	0.636621979814147\\
3.2553041442043	-2.32443432445968	0.636621979814147\\
3.40071775477591	-2.1059721632424	0.636621979814147\\
3.53056209853775	-1.88019447620951	0.636621979814147\\
3.64440632664692	-1.64872754762451	0.636621979814147\\
3.74203734958187	-1.4132078666404	0.636621979814147\\
3.8234529844279	-1.17524775084635	0.636621979814147\\
3.88884912528856	-0.936404015765859	0.636621979814147\\
3.93860197674374	-0.698150749330174	0.636621979814147\\
3.97324655514033	-0.461856917308297	0.636621979814147\\
3.99345272427561	-0.228769182749206	0.636621979814147\\
4	-1.62296649918854e-15	0.636621979814147\\
};
\addplot [color=mycolor1,solid,line width=1.5pt,forget plot]
  table[row sep=crcr]{%
2	0\\
2.02	0\\
2.04	0\\
2.06	0\\
2.08	0\\
2.1	0\\
2.12	0\\
2.14	0\\
2.16	0\\
2.18	0\\
2.2	0\\
2.22	0\\
2.24	0\\
2.26	0\\
2.28	0\\
2.3	0\\
2.32	0\\
2.34	0\\
2.36	0\\
2.38	0\\
2.4	0\\
2.42	0\\
2.44	0\\
2.46	0\\
2.48	0\\
2.5	0\\
2.52	0\\
2.54	0\\
2.56	0\\
2.58	0\\
2.6	0\\
2.62	0\\
2.64	0\\
2.66	0\\
2.68	0\\
2.7	0\\
2.72	0\\
2.74	0\\
2.76	0\\
2.78	0\\
2.8	0\\
2.82	0\\
2.84	0\\
2.86	0\\
2.88	0\\
2.9	0\\
2.92	0\\
2.94	0\\
2.96	0\\
2.98	0\\
3	0\\
3.02	0\\
3.04	0\\
3.06	0\\
3.08	0\\
3.1	0\\
3.12	0\\
3.14	0\\
3.16	0\\
3.18	0\\
3.2	0\\
3.22	0\\
3.24	0\\
3.26	0\\
3.28	0\\
3.3	0\\
3.32	0\\
3.34	0\\
3.36	0\\
3.38	0\\
3.4	0\\
3.42	0\\
3.44	0\\
3.46	0\\
3.48	0\\
3.5	0\\
3.52	0\\
3.54	0\\
3.56	0\\
3.58	0\\
3.6	0\\
3.62	0\\
3.64	0\\
3.66	0\\
3.68	0\\
3.7	0\\
3.72	0\\
3.74	0\\
3.76	0\\
3.78	0\\
3.8	0\\
3.82	0\\
3.84	0\\
3.86	0\\
3.88	0\\
3.9	0\\
3.92	0\\
3.94	0\\
3.96	0\\
3.98	0\\
4	0\\
};
\addplot [color=mycolor1,solid,line width=1.5pt,forget plot]
  table[row sep=crcr]{%
4.44089209850063e-16	2\\
4.48530101948563e-16	2.02\\
4.52970994047064e-16	2.04\\
4.57411886145565e-16	2.06\\
4.61852778244065e-16	2.08\\
4.66293670342566e-16	2.1\\
4.70734562441066e-16	2.12\\
4.75175454539567e-16	2.14\\
4.79616346638068e-16	2.16\\
4.84057238736568e-16	2.18\\
4.88498130835069e-16	2.2\\
4.9293902293357e-16	2.22\\
4.9737991503207e-16	2.24\\
5.01820807130571e-16	2.26\\
5.06261699229071e-16	2.28\\
5.10702591327572e-16	2.3\\
5.15143483426073e-16	2.32\\
5.19584375524573e-16	2.34\\
5.24025267623074e-16	2.36\\
5.28466159721574e-16	2.38\\
5.32907051820075e-16	2.4\\
5.37347943918576e-16	2.42\\
5.41788836017076e-16	2.44\\
5.46229728115577e-16	2.46\\
5.50670620214078e-16	2.48\\
5.55111512312578e-16	2.5\\
5.59552404411079e-16	2.52\\
5.6399329650958e-16	2.54\\
5.6843418860808e-16	2.56\\
5.72875080706581e-16	2.58\\
5.77315972805081e-16	2.6\\
5.81756864903582e-16	2.62\\
5.86197757002083e-16	2.64\\
5.90638649100583e-16	2.66\\
5.95079541199084e-16	2.68\\
5.99520433297585e-16	2.7\\
6.03961325396085e-16	2.72\\
6.08402217494586e-16	2.74\\
6.12843109593086e-16	2.76\\
6.17284001691587e-16	2.78\\
6.21724893790088e-16	2.8\\
6.26165785888588e-16	2.82\\
6.30606677987089e-16	2.84\\
6.3504757008559e-16	2.86\\
6.3948846218409e-16	2.88\\
6.43929354282591e-16	2.9\\
6.48370246381091e-16	2.92\\
6.52811138479592e-16	2.94\\
6.57252030578093e-16	2.96\\
6.61692922676593e-16	2.98\\
6.66133814775094e-16	3\\
6.70574706873595e-16	3.02\\
6.75015598972095e-16	3.04\\
6.79456491070596e-16	3.06\\
6.83897383169096e-16	3.08\\
6.88338275267597e-16	3.1\\
6.92779167366098e-16	3.12\\
6.97220059464598e-16	3.14\\
7.01660951563099e-16	3.16\\
7.061018436616e-16	3.18\\
7.105427357601e-16	3.2\\
7.14983627858601e-16	3.22\\
7.19424519957101e-16	3.24\\
7.23865412055602e-16	3.26\\
7.28306304154103e-16	3.28\\
7.32747196252603e-16	3.3\\
7.37188088351104e-16	3.32\\
7.41628980449605e-16	3.34\\
7.46069872548105e-16	3.36\\
7.50510764646606e-16	3.38\\
7.54951656745106e-16	3.4\\
7.59392548843607e-16	3.42\\
7.63833440942108e-16	3.44\\
7.68274333040608e-16	3.46\\
7.72715225139109e-16	3.48\\
7.7715611723761e-16	3.5\\
7.8159700933611e-16	3.52\\
7.86037901434611e-16	3.54\\
7.90478793533111e-16	3.56\\
7.94919685631612e-16	3.58\\
7.99360577730113e-16	3.6\\
8.03801469828613e-16	3.62\\
8.08242361927114e-16	3.64\\
8.12683254025615e-16	3.66\\
8.17124146124115e-16	3.68\\
8.21565038222616e-16	3.7\\
8.26005930321116e-16	3.72\\
8.30446822419617e-16	3.74\\
8.34887714518118e-16	3.76\\
8.39328606616618e-16	3.78\\
8.43769498715119e-16	3.8\\
8.4821039081362e-16	3.82\\
8.5265128291212e-16	3.84\\
8.57092175010621e-16	3.86\\
8.61533067109121e-16	3.88\\
8.65973959207622e-16	3.9\\
8.70414851306123e-16	3.92\\
8.74855743404623e-16	3.94\\
8.79296635503124e-16	3.96\\
8.83737527601625e-16	3.98\\
8.88178419700125e-16	4\\
};
\addplot [color=mycolor1,solid,line width=1.5pt,forget plot]
  table[row sep=crcr]{%
-2	5.66553889764798e-16\\
-2.02	5.72219428662446e-16\\
-2.04	5.77884967560094e-16\\
-2.06	5.83550506457742e-16\\
-2.08	5.8921604535539e-16\\
-2.1	5.94881584253038e-16\\
-2.12	6.00547123150686e-16\\
-2.14	6.06212662048334e-16\\
-2.16	6.11878200945982e-16\\
-2.18	6.1754373984363e-16\\
-2.2	6.23209278741278e-16\\
-2.22	6.28874817638926e-16\\
-2.24	6.34540356536574e-16\\
-2.26	6.40205895434222e-16\\
-2.28	6.4587143433187e-16\\
-2.3	6.51536973229518e-16\\
-2.32	6.57202512127166e-16\\
-2.34	6.62868051024814e-16\\
-2.36	6.68533589922462e-16\\
-2.38	6.7419912882011e-16\\
-2.4	6.79864667717757e-16\\
-2.42	6.85530206615406e-16\\
-2.44	6.91195745513053e-16\\
-2.46	6.96861284410701e-16\\
-2.48	7.0252682330835e-16\\
-2.5	7.08192362205997e-16\\
-2.52	7.13857901103645e-16\\
-2.54	7.19523440001293e-16\\
-2.56	7.25188978898941e-16\\
-2.58	7.30854517796589e-16\\
-2.6	7.36520056694237e-16\\
-2.62	7.42185595591885e-16\\
-2.64	7.47851134489533e-16\\
-2.66	7.53516673387181e-16\\
-2.68	7.59182212284829e-16\\
-2.7	7.64847751182477e-16\\
-2.72	7.70513290080125e-16\\
-2.74	7.76178828977773e-16\\
-2.76	7.81844367875421e-16\\
-2.78	7.87509906773069e-16\\
-2.8	7.93175445670717e-16\\
-2.82	7.98840984568365e-16\\
-2.84	8.04506523466013e-16\\
-2.86	8.10172062363661e-16\\
-2.88	8.15837601261309e-16\\
-2.9	8.21503140158957e-16\\
-2.92	8.27168679056605e-16\\
-2.94	8.32834217954253e-16\\
-2.96	8.38499756851901e-16\\
-2.98	8.44165295749549e-16\\
-3	8.49830834647197e-16\\
-3.02	8.55496373544845e-16\\
-3.04	8.61161912442493e-16\\
-3.06	8.66827451340141e-16\\
-3.08	8.72492990237789e-16\\
-3.1	8.78158529135437e-16\\
-3.12	8.83824068033085e-16\\
-3.14	8.89489606930733e-16\\
-3.16	8.95155145828381e-16\\
-3.18	9.00820684726029e-16\\
-3.2	9.06486223623677e-16\\
-3.22	9.12151762521325e-16\\
-3.24	9.17817301418973e-16\\
-3.26	9.23482840316621e-16\\
-3.28	9.29148379214269e-16\\
-3.3	9.34813918111917e-16\\
-3.32	9.40479457009565e-16\\
-3.34	9.46144995907213e-16\\
-3.36	9.51810534804861e-16\\
-3.38	9.57476073702509e-16\\
-3.4	9.63141612600157e-16\\
-3.42	9.68807151497804e-16\\
-3.44	9.74472690395452e-16\\
-3.46	9.801382292931e-16\\
-3.48	9.85803768190748e-16\\
-3.5	9.91469307088396e-16\\
-3.52	9.97134845986044e-16\\
-3.54	1.00280038488369e-15\\
-3.56	1.00846592378134e-15\\
-3.58	1.01413146267899e-15\\
-3.6	1.01979700157664e-15\\
-3.62	1.02546254047428e-15\\
-3.64	1.03112807937193e-15\\
-3.66	1.03679361826958e-15\\
-3.68	1.04245915716723e-15\\
-3.7	1.04812469606488e-15\\
-3.72	1.05379023496252e-15\\
-3.74	1.05945577386017e-15\\
-3.76	1.06512131275782e-15\\
-3.78	1.07078685165547e-15\\
-3.8	1.07645239055312e-15\\
-3.82	1.08211792945076e-15\\
-3.84	1.08778346834841e-15\\
-3.86	1.09344900724606e-15\\
-3.88	1.09911454614371e-15\\
-3.9	1.10478008504136e-15\\
-3.92	1.110445623939e-15\\
-3.94	1.11611116283665e-15\\
-3.96	1.1217767017343e-15\\
-3.98	1.12744224063195e-15\\
-4	1.1331077795296e-15\\
};
\addplot [color=mycolor1,solid,line width=1.5pt,forget plot]
  table[row sep=crcr]{%
-6.89018569679533e-16	-2\\
-6.95908755376329e-16	-2.02\\
-7.02798941073124e-16	-2.04\\
-7.09689126769919e-16	-2.06\\
-7.16579312466715e-16	-2.08\\
-7.2346949816351e-16	-2.1\\
-7.30359683860305e-16	-2.12\\
-7.37249869557101e-16	-2.14\\
-7.44140055253896e-16	-2.16\\
-7.51030240950691e-16	-2.18\\
-7.57920426647487e-16	-2.2\\
-7.64810612344282e-16	-2.22\\
-7.71700798041077e-16	-2.24\\
-7.78590983737873e-16	-2.26\\
-7.85481169434668e-16	-2.28\\
-7.92371355131463e-16	-2.3\\
-7.99261540828259e-16	-2.32\\
-8.06151726525054e-16	-2.34\\
-8.13041912221849e-16	-2.36\\
-8.19932097918645e-16	-2.38\\
-8.2682228361544e-16	-2.4\\
-8.33712469312235e-16	-2.42\\
-8.40602655009031e-16	-2.44\\
-8.47492840705826e-16	-2.46\\
-8.54383026402621e-16	-2.48\\
-8.61273212099417e-16	-2.5\\
-8.68163397796212e-16	-2.52\\
-8.75053583493007e-16	-2.54\\
-8.81943769189803e-16	-2.56\\
-8.88833954886598e-16	-2.58\\
-8.95724140583393e-16	-2.6\\
-9.02614326280189e-16	-2.62\\
-9.09504511976984e-16	-2.64\\
-9.16394697673779e-16	-2.66\\
-9.23284883370575e-16	-2.68\\
-9.3017506906737e-16	-2.7\\
-9.37065254764165e-16	-2.72\\
-9.43955440460961e-16	-2.74\\
-9.50845626157756e-16	-2.76\\
-9.57735811854551e-16	-2.78\\
-9.64625997551347e-16	-2.8\\
-9.71516183248142e-16	-2.82\\
-9.78406368944937e-16	-2.84\\
-9.85296554641733e-16	-2.86\\
-9.92186740338528e-16	-2.88\\
-9.99076926035323e-16	-2.9\\
-1.00596711173212e-15	-2.92\\
-1.01285729742891e-15	-2.94\\
-1.01974748312571e-15	-2.96\\
-1.0266376688225e-15	-2.98\\
-1.0335278545193e-15	-3\\
-1.0404180402161e-15	-3.02\\
-1.04730822591289e-15	-3.04\\
-1.05419841160969e-15	-3.06\\
-1.06108859730648e-15	-3.08\\
-1.06797878300328e-15	-3.1\\
-1.07486896870007e-15	-3.12\\
-1.08175915439687e-15	-3.14\\
-1.08864934009366e-15	-3.16\\
-1.09553952579046e-15	-3.18\\
-1.10242971148725e-15	-3.2\\
-1.10931989718405e-15	-3.22\\
-1.11621008288084e-15	-3.24\\
-1.12310026857764e-15	-3.26\\
-1.12999045427443e-15	-3.28\\
-1.13688063997123e-15	-3.3\\
-1.14377082566803e-15	-3.32\\
-1.15066101136482e-15	-3.34\\
-1.15755119706162e-15	-3.36\\
-1.16444138275841e-15	-3.38\\
-1.17133156845521e-15	-3.4\\
-1.178221754152e-15	-3.42\\
-1.1851119398488e-15	-3.44\\
-1.19200212554559e-15	-3.46\\
-1.19889231124239e-15	-3.48\\
-1.20578249693918e-15	-3.5\\
-1.21267268263598e-15	-3.52\\
-1.21956286833277e-15	-3.54\\
-1.22645305402957e-15	-3.56\\
-1.23334323972636e-15	-3.58\\
-1.24023342542316e-15	-3.6\\
-1.24712361111996e-15	-3.62\\
-1.25401379681675e-15	-3.64\\
-1.26090398251355e-15	-3.66\\
-1.26779416821034e-15	-3.68\\
-1.27468435390714e-15	-3.7\\
-1.28157453960393e-15	-3.72\\
-1.28846472530073e-15	-3.74\\
-1.29535491099752e-15	-3.76\\
-1.30224509669432e-15	-3.78\\
-1.30913528239111e-15	-3.8\\
-1.31602546808791e-15	-3.82\\
-1.3229156537847e-15	-3.84\\
-1.3298058394815e-15	-3.86\\
-1.33669602517829e-15	-3.88\\
-1.34358621087509e-15	-3.9\\
-1.35047639657189e-15	-3.92\\
-1.35736658226868e-15	-3.94\\
-1.36425676796548e-15	-3.96\\
-1.37114695366227e-15	-3.98\\
-1.37803713935907e-15	-4\\
};
\addplot [color=mycolor1,solid,line width=1.5pt,forget plot]
  table[row sep=crcr]{%
2	-8.11483249594269e-16\\
2.02	-8.19598082090211e-16\\
2.04	-8.27712914586154e-16\\
2.06	-8.35827747082097e-16\\
2.08	-8.43942579578039e-16\\
2.1	-8.52057412073982e-16\\
2.12	-8.60172244569925e-16\\
2.14	-8.68287077065867e-16\\
2.16	-8.7640190956181e-16\\
2.18	-8.84516742057753e-16\\
2.2	-8.92631574553695e-16\\
2.22	-9.00746407049638e-16\\
2.24	-9.08861239545581e-16\\
2.26	-9.16976072041524e-16\\
2.28	-9.25090904537466e-16\\
2.3	-9.33205737033409e-16\\
2.32	-9.41320569529352e-16\\
2.34	-9.49435402025294e-16\\
2.36	-9.57550234521237e-16\\
2.38	-9.6566506701718e-16\\
2.4	-9.73779899513122e-16\\
2.42	-9.81894732009065e-16\\
2.44	-9.90009564505008e-16\\
2.46	-9.98124397000951e-16\\
2.48	-1.00623922949689e-15\\
2.5	-1.01435406199284e-15\\
2.52	-1.02246889448878e-15\\
2.54	-1.03058372698472e-15\\
2.56	-1.03869855948066e-15\\
2.58	-1.04681339197661e-15\\
2.6	-1.05492822447255e-15\\
2.62	-1.06304305696849e-15\\
2.64	-1.07115788946443e-15\\
2.66	-1.07927272196038e-15\\
2.68	-1.08738755445632e-15\\
2.7	-1.09550238695226e-15\\
2.72	-1.10361721944821e-15\\
2.74	-1.11173205194415e-15\\
2.76	-1.11984688444009e-15\\
2.78	-1.12796171693603e-15\\
2.8	-1.13607654943198e-15\\
2.82	-1.14419138192792e-15\\
2.84	-1.15230621442386e-15\\
2.86	-1.1604210469198e-15\\
2.88	-1.16853587941575e-15\\
2.9	-1.17665071191169e-15\\
2.92	-1.18476554440763e-15\\
2.94	-1.19288037690357e-15\\
2.96	-1.20099520939952e-15\\
2.98	-1.20911004189546e-15\\
3	-1.2172248743914e-15\\
3.02	-1.22533970688735e-15\\
3.04	-1.23345453938329e-15\\
3.06	-1.24156937187923e-15\\
3.08	-1.24968420437517e-15\\
3.1	-1.25779903687112e-15\\
3.12	-1.26591386936706e-15\\
3.14	-1.274028701863e-15\\
3.16	-1.28214353435894e-15\\
3.18	-1.29025836685489e-15\\
3.2	-1.29837319935083e-15\\
3.22	-1.30648803184677e-15\\
3.24	-1.31460286434272e-15\\
3.26	-1.32271769683866e-15\\
3.28	-1.3308325293346e-15\\
3.3	-1.33894736183054e-15\\
3.32	-1.34706219432649e-15\\
3.34	-1.35517702682243e-15\\
3.36	-1.36329185931837e-15\\
3.38	-1.37140669181431e-15\\
3.4	-1.37952152431026e-15\\
3.42	-1.3876363568062e-15\\
3.44	-1.39575118930214e-15\\
3.46	-1.40386602179808e-15\\
3.48	-1.41198085429403e-15\\
3.5	-1.42009568678997e-15\\
3.52	-1.42821051928591e-15\\
3.54	-1.43632535178186e-15\\
3.56	-1.4444401842778e-15\\
3.58	-1.45255501677374e-15\\
3.6	-1.46066984926968e-15\\
3.62	-1.46878468176563e-15\\
3.64	-1.47689951426157e-15\\
3.66	-1.48501434675751e-15\\
3.68	-1.49312917925345e-15\\
3.7	-1.5012440117494e-15\\
3.72	-1.50935884424534e-15\\
3.74	-1.51747367674128e-15\\
3.76	-1.52558850923723e-15\\
3.78	-1.53370334173317e-15\\
3.8	-1.54181817422911e-15\\
3.82	-1.54993300672505e-15\\
3.84	-1.558047839221e-15\\
3.86	-1.56616267171694e-15\\
3.88	-1.57427750421288e-15\\
3.9	-1.58239233670882e-15\\
3.92	-1.59050716920477e-15\\
3.94	-1.59862200170071e-15\\
3.96	-1.60673683419665e-15\\
3.98	-1.61485166669259e-15\\
4	-1.62296649918854e-15\\
};
\addplot [color=mycolor1,solid,line width=1.5pt,forget plot]
  table[row sep=crcr]{%
2	0\\
1.9967263621378	0.114384591374602\\
1.98662327757016	0.230928458654148\\
1.96930098837187	0.349075374665086\\
1.94442456264428	0.468202007882929\\
1.91172649221395	0.587623875423176\\
1.87101867479093	0.706603933320199\\
1.82220316332346	0.824363773812255\\
1.76528104926887	0.940097238104754\\
1.70035887738796	1.0529860816212\\
1.62765207210215	1.16221716222984\\
1.54748498864467	1.26700047747402\\
1.46028737915691	1.36658727137166\\
1.36658727137166	1.46028737915691\\
1.26700047747402	1.54748498864467\\
1.16221716222984	1.62765207210215\\
1.0529860816212	1.70035887738796\\
0.940097238104754	1.76528104926887\\
0.824363773812255	1.82220316332346\\
0.7066039333202	1.87101867479093\\
0.587623875423176	1.91172649221395\\
0.468202007882929	1.94442456264428\\
0.349075374665087	1.96930098837187\\
0.230928458654148	1.98662327757016\\
0.114384591374603	1.9967263621378\\
4.44089209850063e-16	2\\
-0.114384591374602	1.9967263621378\\
-0.230928458654147	1.98662327757016\\
-0.349075374665086	1.96930098837187\\
-0.468202007882928	1.94442456264428\\
-0.587623875423175	1.91172649221395\\
-0.706603933320199	1.87101867479093\\
-0.824363773812254	1.82220316332346\\
-0.940097238104754	1.76528104926887\\
-1.0529860816212	1.70035887738796\\
-1.16221716222984	1.62765207210215\\
-1.26700047747402	1.54748498864467\\
-1.36658727137166	1.46028737915691\\
-1.46028737915691	1.36658727137166\\
-1.54748498864467	1.26700047747402\\
-1.62765207210215	1.16221716222984\\
-1.70035887738796	1.0529860816212\\
-1.76528104926887	0.940097238104755\\
-1.82220316332346	0.824363773812255\\
-1.87101867479093	0.7066039333202\\
-1.91172649221395	0.587623875423176\\
-1.94442456264428	0.468202007882929\\
-1.96930098837187	0.349075374665087\\
-1.98662327757016	0.230928458654148\\
-1.9967263621378	0.114384591374603\\
-2	5.66553889764798e-16\\
-1.9967263621378	-0.114384591374602\\
-1.98662327757016	-0.230928458654147\\
-1.96930098837187	-0.349075374665086\\
-1.94442456264428	-0.468202007882929\\
-1.91172649221395	-0.587623875423176\\
-1.87101867479093	-0.7066039333202\\
-1.82220316332346	-0.824363773812253\\
-1.76528104926887	-0.940097238104753\\
-1.70035887738796	-1.0529860816212\\
-1.62765207210215	-1.16221716222984\\
-1.54748498864468	-1.26700047747402\\
-1.46028737915691	-1.36658727137166\\
-1.36658727137166	-1.46028737915691\\
-1.26700047747402	-1.54748498864467\\
-1.16221716222984	-1.62765207210215\\
-1.0529860816212	-1.70035887738796\\
-0.940097238104754	-1.76528104926887\\
-0.824363773812254	-1.82220316332346\\
-0.706603933320201	-1.87101867479093\\
-0.587623875423177	-1.91172649221395\\
-0.46820200788293	-1.94442456264428\\
-0.349075374665087	-1.96930098837187\\
-0.230928458654148	-1.98662327757016\\
-0.114384591374603	-1.9967263621378\\
-6.89018569679533e-16	-2\\
0.114384591374601	-1.9967263621378\\
0.230928458654147	-1.98662327757016\\
0.349075374665086	-1.96930098837187\\
0.468202007882928	-1.94442456264428\\
0.587623875423176	-1.91172649221395\\
0.7066039333202	-1.87101867479093\\
0.824363773812253	-1.82220316332346\\
0.940097238104753	-1.76528104926887\\
1.0529860816212	-1.70035887738796\\
1.16221716222984	-1.62765207210215\\
1.26700047747402	-1.54748498864468\\
1.36658727137166	-1.46028737915691\\
1.46028737915691	-1.36658727137166\\
1.54748498864467	-1.26700047747402\\
1.62765207210215	-1.16221716222984\\
1.70035887738796	-1.0529860816212\\
1.76528104926887	-0.940097238104754\\
1.82220316332346	-0.824363773812254\\
1.87101867479093	-0.706603933320201\\
1.91172649221395	-0.587623875423177\\
1.94442456264428	-0.46820200788293\\
1.96930098837187	-0.349075374665087\\
1.98662327757016	-0.230928458654149\\
1.9967263621378	-0.114384591374603\\
2	-8.11483249594269e-16\\
};
\addplot [color=mycolor1,solid,line width=1.5pt,forget plot]
  table[row sep=crcr]{%
4	0\\
3.99345272427561	0.228769182749204\\
3.97324655514033	0.461856917308295\\
3.93860197674374	0.698150749330172\\
3.88884912528856	0.936404015765857\\
3.8234529844279	1.17524775084635\\
3.74203734958187	1.4132078666404\\
3.64440632664692	1.64872754762451\\
3.53056209853775	1.88019447620951\\
3.40071775477591	2.1059721632424\\
3.2553041442043	2.32443432445968\\
3.09496997728935	2.53400095494804\\
2.92057475831383	2.73317454274331\\
2.73317454274331	2.92057475831383\\
2.53400095494804	3.09496997728935\\
2.32443432445968	3.2553041442043\\
2.1059721632424	3.40071775477591\\
1.88019447620951	3.53056209853775\\
1.64872754762451	3.64440632664692\\
1.4132078666404	3.74203734958187\\
1.17524775084635	3.8234529844279\\
0.936404015765858	3.88884912528856\\
0.698150749330173	3.93860197674374\\
0.461856917308296	3.97324655514033\\
0.228769182749205	3.99345272427561\\
8.88178419700125e-16	4\\
-0.228769182749203	3.99345272427561\\
-0.461856917308295	3.97324655514033\\
-0.698150749330172	3.93860197674374\\
-0.936404015765856	3.88884912528856\\
-1.17524775084635	3.8234529844279\\
-1.4132078666404	3.74203734958187\\
-1.64872754762451	3.64440632664692\\
-1.88019447620951	3.53056209853775\\
-2.1059721632424	3.40071775477591\\
-2.32443432445967	3.2553041442043\\
-2.53400095494804	3.09496997728935\\
-2.73317454274331	2.92057475831383\\
-2.92057475831383	2.73317454274331\\
-3.09496997728935	2.53400095494804\\
-3.2553041442043	2.32443432445968\\
-3.40071775477591	2.1059721632424\\
-3.53056209853775	1.88019447620951\\
-3.64440632664692	1.64872754762451\\
-3.74203734958187	1.4132078666404\\
-3.8234529844279	1.17524775084635\\
-3.88884912528856	0.936404015765858\\
-3.93860197674374	0.698150749330174\\
-3.97324655514033	0.461856917308297\\
-3.99345272427561	0.228769182749205\\
-4	1.1331077795296e-15\\
-3.99345272427561	-0.228769182749203\\
-3.97324655514033	-0.461856917308295\\
-3.93860197674374	-0.698150749330172\\
-3.88884912528856	-0.936404015765857\\
-3.8234529844279	-1.17524775084635\\
-3.74203734958187	-1.4132078666404\\
-3.64440632664692	-1.64872754762451\\
-3.53056209853775	-1.88019447620951\\
-3.40071775477591	-2.1059721632424\\
-3.2553041442043	-2.32443432445967\\
-3.09496997728935	-2.53400095494804\\
-2.92057475831383	-2.73317454274331\\
-2.73317454274331	-2.92057475831383\\
-2.53400095494804	-3.09496997728935\\
-2.32443432445968	-3.2553041442043\\
-2.1059721632424	-3.40071775477591\\
-1.88019447620951	-3.53056209853775\\
-1.64872754762451	-3.64440632664692\\
-1.4132078666404	-3.74203734958187\\
-1.17524775084635	-3.8234529844279\\
-0.936404015765859	-3.88884912528856\\
-0.698150749330174	-3.93860197674374\\
-0.461856917308297	-3.97324655514033\\
-0.228769182749206	-3.99345272427561\\
-1.37803713935907e-15	-4\\
0.228769182749203	-3.99345272427561\\
0.461856917308294	-3.97324655514033\\
0.698150749330172	-3.93860197674374\\
0.936404015765857	-3.88884912528856\\
1.17524775084635	-3.8234529844279\\
1.4132078666404	-3.74203734958187\\
1.64872754762451	-3.64440632664692\\
1.88019447620951	-3.53056209853775\\
2.1059721632424	-3.40071775477591\\
2.32443432445967	-3.2553041442043\\
2.53400095494804	-3.09496997728935\\
2.73317454274331	-2.92057475831383\\
2.92057475831383	-2.73317454274331\\
3.09496997728935	-2.53400095494804\\
3.2553041442043	-2.32443432445968\\
3.40071775477591	-2.1059721632424\\
3.53056209853775	-1.88019447620951\\
3.64440632664692	-1.64872754762451\\
3.74203734958187	-1.4132078666404\\
3.8234529844279	-1.17524775084635\\
3.88884912528856	-0.936404015765859\\
3.93860197674374	-0.698150749330174\\
3.97324655514033	-0.461856917308297\\
3.99345272427561	-0.228769182749206\\
4	-1.62296649918854e-15\\
};
\addplot [draw=none, mark size=3.3pt, scatter,mark=ball,scatter/use mapped color={ball color=red},scatter src=rand,only marks,z buffer=sort]
  table[row sep=crcr]{%
2	0\\
2	2\\
4.44089209850063e-16	2\\
-2	2\\
-2	5.66553889764798e-16\\
-2	-2\\
-6.89018569679533e-16	-2\\
2	-2\\
2	-8.11483249594269e-16\\
4	0\\
4	4\\
8.88178419700125e-16	4\\
-4	4\\
-4	1.1331077795296e-15\\
-4	-4\\
-1.37803713935907e-15	-4\\
4	-4\\
4	-1.62296649918854e-15\\
};
\addplot [color=black,dashed,forget plot]
  table[row sep=crcr]{%
2	0\\
4	0\\
};
\addplot [color=black,dashed,forget plot]
  table[row sep=crcr]{%
2	2\\
4	4\\
};
\addplot [color=black,dashed,forget plot]
  table[row sep=crcr]{%
4.44089209850063e-16	2\\
8.88178419700125e-16	4\\
};
\addplot [color=black,dashed,forget plot]
  table[row sep=crcr]{%
-2	2\\
-4	4\\
};
\addplot [color=black,dashed,forget plot]
  table[row sep=crcr]{%
-2	5.66553889764798e-16\\
-4	1.1331077795296e-15\\
};
\addplot [color=black,dashed,forget plot]
  table[row sep=crcr]{%
-2	-2\\
-4	-4\\
};
\addplot [color=black,dashed,forget plot]
  table[row sep=crcr]{%
-6.89018569679533e-16	-2\\
-1.37803713935907e-15	-4\\
};
\addplot [color=black,dashed,forget plot]
  table[row sep=crcr]{%
2	-2\\
4	-4\\
};
\addplot [color=black,dashed,forget plot]
  table[row sep=crcr]{%
2	-8.11483249594269e-16\\
4	-1.62296649918854e-15\\
};
\addplot [color=black,dashed,forget plot]
  table[row sep=crcr]{%
2	0\\
2	2\\
4.44089209850063e-16	2\\
-2	2\\
-2	5.66553889764798e-16\\
-2	-2\\
-6.89018569679533e-16	-2\\
2	-2\\
2	-8.11483249594269e-16\\
};
\addplot [color=black,dashed,forget plot]
  table[row sep=crcr]{%
4	0\\
4	4\\
8.88178419700125e-16	4\\
-4	4\\
-4	1.1331077795296e-15\\
-4	-4\\
-1.37803713935907e-15	-4\\
4	-4\\
4	-1.62296649918854e-15\\
};
\coordinate (A) at (axis cs: 3.5, 0);
\coordinate (B) at (axis cs: 2.5, 0);
\coordinate (C) at (axis cs: 3.0, 0);

\draw[red,line width=1.0pt, -stealth, decoration={
    text along path, text={$\xi$} {--} direction, raise=1ex, text align={center}, text color={red}
    }, postaction={decorate}, dashed] (3.7, 0.0) arc [start angle=0, end angle=0+70, radius=3.7];
    
\coordinate (D) at (axis cs: 2.0, 0.0);
\coordinate (E) at (axis cs: 2.0, 2.0);
\coordinate (F) at (axis cs: 0.0, 2.0);
\coordinate (G) at (axis cs: -2.0, 2.0);
\coordinate (H) at (axis cs: -2.0, 0.0);
\coordinate (I) at (axis cs: -2.0, -2.0);
\coordinate (J) at (axis cs: 0.0, -2.0);
\coordinate (K) at (axis cs: 2.0, -2.0);

\coordinate (L) at (axis cs: 4.0, 0.0);
\coordinate (M) at (axis cs: 4.0, 4.0);
\coordinate (N) at (axis cs: 0.0, 4.0);
\coordinate (O) at (axis cs: -4.0, 4.0);
\coordinate (P) at (axis cs: -4.0, 0.0);
\coordinate (Q) at (axis cs: -4.0, -4.0);
\coordinate (R) at (axis cs: 0.0, -4.0);
\coordinate (S) at (axis cs: 4.0, -4.0);

\end{axis}
%\draw[help lines,xstep=1,ystep=1] (0,0) grid (10, 6);
%\foreach \x in {0,1,...,10} { \node [anchor=north] at (\x, 0) {\x}; }
%\foreach \y in {0,1,...,6} { \node [anchor=east] at (0, \y) {\y}; }
\draw[-latex,thick](7, 3.5)node[right]{Boundary Reference $ = 1$} to[out=180,in=90] (A);
\draw[-latex,thick, red, opacity = 0.7](7, 2)node[right, red, opacity = 0.7]{Internal interface} to[out=180,in=-90] (C);
\draw[-latex,thick](7, 1)node[right]{Boundary Reference $ = 2$} to[out=180,in=-90] (B);

\node[left] at (D) {$\mathbf{P}_{1, 9}$};
\node[left] at (E) {$\mathbf{P}_{2}$};
\node[below] at (F) {$\mathbf{P}_{3}$};
\node[right] at (G) {$\mathbf{P}_{4}$};
\node[right] at (H) {$\mathbf{P}_{5}$};
\node[right] at (I) {$\mathbf{P}_{6}$};
\node[above] at (J) {$\mathbf{P}_{7}$};
\node[left] at (K) {$\mathbf{P}_{8}$};

\node[right] at (L) {$\mathbf{P}_{10, 18}$};
\node[right] at (M) {$\mathbf{P}_{11}$};
\node[above] at (N) {$\mathbf{P}_{12}$};
\node[left] at (O) {$\mathbf{P}_{13}$};
\node[left] at (P) {$\mathbf{P}_{14}$};
\node[left] at (Q) {$\mathbf{P}_{15}$};
\node[below] at (R) {$\mathbf{P}_{16}$};
\node[right] at (S) {$\mathbf{P}_{17}$};

\draw[ultra thick, red, opacity = 0.7] (D) -- (L);
\end{tikzpicture}% 
                \caption{An annular patch.}
                \label{fig:Ch3SurfAnnular}
            \end{figure}
        \item  For multi-patch problems, the additional parameters are GNum and Boundaries, where GNum is a cell array containing a collection of global numbering arrays of the corresponding patches and Boundaries is a structure data type defining the boundaries of the multi-patch geometry, i.e
            \begin{lstlisting}
                [FY, YDofs] = applyNewmannBdryVals(NURBS, Mesh, g, Refs, LAB, GNum, Boundaries)
            \end{lstlisting}
    \end{itemize}
\end{itemize}
The output parameters are
\begin{itemize}
    \item A force vector $\mathbf{F}$: \lstinline{FY}
    \item A vector containing the corresponding DOFs: \lstinline{YDofs}
\end{itemize}
these force values are then added to the global force vector $\mathbf{F}$ of the system at the appropriate positions by using the command
\begin{lstlisting}
    F(YDofs) = F(YDofs) + FY;
\end{lstlisting}
\subsection{Essential boundary condition}
Due to non-interpolatory property of NURBS basis functions, a special treatment for applying Dirichlet boundary condition is needed. This issue was first mentioned in the paper \cite{Hughes20054135}, in which, for homogeneous Dirichlet boundary condition, boundary values can be directly assigned to control variables. This approach is called direct imposition of Dirichlet BCs, which means we simply assign the prescribed boundary data (evaluated at each control point's position) to the corresponding control variables. In many cases where control points are not located on the boundary, this approach will lead to an incorrect result. For tackling this problem, several methods are proposed such as Lagrange Multiplier Method, Penalty Method, Nitsche's Method, Global Least Squares Fit Method, etc. The current implementation in SIMO Package is based on Global Least Squares Fit Method. This method is presented as follows

Defining $u ^ h$ is a projection of an arbitrary function $u \in \mathcal{L} ^ 2(\Omega)$ into a finite space of B-spline basis functions  $\mathcal{V} ^ h \in \mathcal{L} ^ 2(\Omega)$, where $\Omega \in \mathbb{R} ^ d$ is the enforced Dirichlet domain which has dimension $d$. We need to determine the coefficients $c_I$ such that
\begin{equation}
    u ^ h = \sum_{I = 1} ^ n N_I c_I,
\end{equation}
where $N_I$ is the $I^\text{th}$ B-spline basis function in $\mathcal{V} ^ h$ and $c_I$ is the associated coefficient of the $I^\text{th}$ basis function, the function $u^h$ is an approximation of function $u$. The projection is chosen in such a way that minimize the error produced when projecting $u$ to $\mathcal{V} ^ h$
\begin{equation}
    J(u ^ h) = \enVert[1]{u ^ h - u}_{\mathcal{L}^2 (\Omega)} \rightarrow \text{min},
\end{equation}
where $\mathcal{L}^2$--norm is defined by
\begin{equation}
    \enVert[1]{f}_{\mathcal{L}^2 (\Omega)} = \PARENS{\int_{\Omega} \abs[1]{f(\mathbf{x})} ^ 2 \dif \Omega} ^ {1/2}
	= \SQRT{\PARENS{f(\mathbf{x}), f(\mathbf{x})}_{\mathcal{L}^2 (\Omega)}}.
\end{equation}
Since minimizing that norm of the error is equivalent to minimizing the squared--norm error of the projection, therefore to avoid handling the square root, the problem is rewritten as
\begin{equation}
    J(u ^ h) = \enVert[1]{u ^ h - u}_{\mathcal{L}^2 (\Omega)} ^ 2 \rightarrow \text{min}.
    \label{eq:Ch3OptPro}
\end{equation}
To determine the optimal function $u ^ h$ of the functional $J(u ^ h)$ we need to compute the G\^ateaux derivative \footnote{G\^ateaux derivative of inner product $D_h(f, g) = (f, D_h g) + (D_h f, g)$.} of $J(u ^ h)$ at $u ^ h \in \mathcal{V} ^ h$ in direction $\delta u ^ h \in \mathcal{V} ^ h $
\begin{equation}
    \begin{aligned}
    	D_{\delta u ^ h} J(u ^ h) &= \dfrac{\dif }{\dif \alpha} \left(%
        u ^ h + \alpha \delta u ^ h - u, u ^ h + \alpha \delta u ^ h - u
        \right)\\%
        &= \eval[2]{2 \left(u ^ h + \alpha \delta u ^ h - u, \delta u ^ h \right)}_{\alpha = 0}\\%
        & = 2 \left(u ^ h - u, \delta u ^ h\right),
    \end{aligned}
\end{equation}
the optimum condition occurs when first derivative in all direction is vanished, leads to
\begin{equation}
    \PARENS{u^h - u, \delta u^h} = 0, \quad \forall \delta u^h \in \mathcal{V}^h,
\end{equation}
or equivalent with
\begin{equation}
    \PARENS{u^h, \delta u^h} = \PARENS{u, \delta u^h}, \quad \forall \delta u^h \in \mathcal{V}^h.
    \label{eq:Ch3OptimalityCondition}
\end{equation}
By approximating $u^h$ and $\delta u ^ h$ using basis functions, one obtain
\begin{equation}
    u ^ h = \sum_{I=1}^{n} N_I c_I = \mathbf{N} \mathbf{c}, \quad \delta u ^ h = \sum_{J=1}^{n} N_J \delta c_J = \mathbf{N} \delta \mathbf{c},
    \label{eq:Ch3DiscretizeFuns}
\end{equation}
where $\mathbf{N}$ is a row vector containing basis functions evaluated at boundary, $\mathbf{c}$ and $\delta \mathbf{c}$ are column vectors containing coefficients of the approximation and weight functions, respectively. Substituting \ref{eq:Ch3DiscretizeFuns} to \ref{eq:Ch3OptimalityCondition} lead to the system of discretized equations as follow
\begin{equation}
    \int_{\Gamma} \mathbf{N}^\text{T} \mathbf{N} \dif \Omega \mathbf{c} = \int_{\Gamma} \mathbf{N} ^ \text{T} u \dif \Omega,
\end{equation}
which including a global mass matrix in the left hand side and a force vector in the right hand side. Solving this system of equations we obtain our desired coefficient vector $\mathbf{c}$.

This method is implemented in the file named ``projDrchltBdryVals.m'' for both single patch and multi-patch problems with the supports for both 2D and 3D cases. In the current version, the implementation this routine is limited to some basic types of problems that involved elasticity, head conduction and plate problems. The input parameters are described below
\begin{itemize}
    \item NURBS: NURBS structure of the physical domain
    \item Mesh: Mesh structure of the physical domain
    \item h: the function defining the boundary which is provided as an anonymous function handle, e.g
    \begin{itemize}
        \item For homogeneous boundary condition: \lstinline{h = @(x, y) 0}
        \item For a specific boundary function: \lstinline{h = @(x, y) 2*x + 3*y}
    \end{itemize}
    \item Refs: reference index (indices) of the corresponding boundary (boundaries), in case there are more than one boundary, the reference indices are stored in an array
    \item LAB: label to identify which type of the problem is investigating, valid labels are
        \begin{itemize}
          \item \lstinline{'TEMP'} is an abbreviation of ``Temperature'' and it is used in head conduction problems
          \item \lstinline{'UX'}, \lstinline[columns=fixed]{'UY'} or \lstinline[columns=fixed]{'UZ'} are labels identifying displacement components and they are used in elasticity problems
          \item \lstinline{'PLATE'} used for applying simply support boundary condition in plate problems
        \end{itemize}
    \item varargin: an additional input parameter used to handle a NURBS patch having an internal interface and to manage multi-patch problem as mentioned before for natural boundary conditions, e.g
        \begin{itemize}
            \item For NURBS patch having an internal interface
                \begin{lstlisting}
                    [BdryVals, BdryIdcs] = projDrchltBdryVals(NURBS, Mesh, h, Refs, LAB, GNum)
                \end{lstlisting}
            \item For multi-patch problems
                \begin{lstlisting}
                    [BdryVals, BdryIdcs] = projDrchltBdryVals(NURBS, Mesh, h, Refs, LAB, GNum, Boundaries)
                \end{lstlisting}
        \end{itemize}
\end{itemize}
The output parameters are listed below
\begin{itemize}
    \item A coefficients vector $\mathbf{c}_I$: \lstinline{BdryVals}
    \item A vector containing the corresponding DOFs: \lstinline{BdryIdcs}
\end{itemize}
these parameters are then passed into a routine named ``applyDrchltBdryVals.m'' which is designed to modify the linear system of equations using partition method, the syntax is
\begin{lstlisting}
    [f, d, FreeIdcs] = applyDrchltBdryVals(BdryIdcs, BdryVals, K, f);
\end{lstlisting}

The linear system is finally solved by invoking the command
\begin{lstlisting}
    d(FreeIdcs) = K(FreeIdcs, FreeIdcs) \ f(FreeIdcs);
\end{lstlisting}
\section{Visualization}

\bibliographystyle{IEEEtran} % Use the "unsrtnat" BibTeX style for formatting the Bibliography
\bibliography{references/chapter3} % The references (bibliography) information are stored in the file named


\let\clearforchapter\par % cheating, but saves some space
\chapter{Simulation Examples}

\section{Heat Conduction}

\section{Elasticity}

\section{Limit Analysis}

\section{Gradient Elasticity}


%\include{./chapters/chapter5}
%\include{./chapters/chapter6}
%\include{./chapters/chapter7}
\backmatter%%%%%%%%%%%%%%%%%%%%%%%%%%%%%%%%%%%%%%%%%%%%%%%%%%%%%%%
\end{document}
